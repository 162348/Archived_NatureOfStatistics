\documentclass[uplatex,dvipdfmx]{jsarticle}
\title{Fourier解析・演習問題}
\author{}
\pagestyle{headings} \setcounter{secnumdepth}{4}
\input{/Users/Hirofumi Shiba/NatureOfStatistics/preamble_no_fonts.tex}
%\input{/Users/hirofumi.shiba48/NatureOfStatistics/preamble_no_fonts.tex}
\usepackage[math]{anttor}
\begin{document}

\begin{example}
    $a,b\in\R^d$に対して,
    $\delta_a*\delta_b=\delta_{a+b}$.
\end{example}
\begin{Proof}
    コンパクト台を持つ超関数との畳み込みの定義から,
    \[(\delta_a*\delta_b|\varphi)=(\delta_a|\wt{\delta_b}*\varphi).\]
    ここで,$\wt{\delta_b}*\varphi$を簡単にすることを考えると,
    \begin{align*}
        \wt{\delta_b}*\varphi(x)&=(\wt{\delta_b}|\tau_x\wt{\varphi})\\
        &=(\delta_b|\wt{\tau_x\wt{\varphi}})\\
        &=(\delta_b|\varphi(x+\bullet))=\varphi(x+b),\qquad(x\in\R^d).
    \end{align*}
    よって,
    \[(\delta_a*\delta_b|\varphi)=(\delta_a|\varphi(\bullet+b))=\varphi(a+b)=(\delta_{a+b}|\varphi)\qquad(\varphi\in\D(\R^d)).\]
\end{Proof}

\begin{proposition}\mbox{}
    \begin{enumerate}
        \item $T\in\D'(\R)$は$T'=T$を満たすとする.このとき,$T\in L^1_\loc(\R)$で,$\exists_{C\in\R}\;T(x)=Ce^x$.
        \item $g(x)=e^x$,$S\in\D'_c(\R)$とする.$g*S$は$e^x$の定数倍である.
    \end{enumerate}
\end{proposition}
\begin{Proof}
    $\varphi\in C^\infty(\R)$に対して,$(\varphi T|\psi):=(T|\varphi\psi)\;(\psi\in\D(\R))$と定める.
    \begin{enumerate}
        \item $e^{-x}T$が定数関数であることを示せば良い.
        実際,$(e^{-x}\varphi)'=-e^{-x}\varphi+e^{-x}\varphi'\;(\varphi\in\D(\R))$であるから,
        任意の$\varphi\in \D(\R)$に対して,
        \begin{align*}
            ((e^{-x}T)'|\varphi)&=(e^{-x}T|\varphi')=(T|e^{-x}\varphi')\\
            &=(T|(e^{-x}\varphi)'+e^{-x}\varphi)\\
            &=-(T'|e^{-x}\varphi)+(T|e^{-x}\varphi)=(T-T'|e^{-x}\varphi)=0.
        \end{align*}
        これより,$e^{-x}T\in L^1_\loc(\R)$かつ定数である.
        \item $e^{-x}(g*S)$が$L^1_\loc(\R)$の元でありかつ定数であることを示せば良い.
        実際,任意の$\varphi\in\D(\R)$について,
        \begin{align*}
            ((e^{-x}(g*S))'|\varphi)&=(e^{-x}(g*S)|\varphi')\\
            &=(g*S|e^{-x}\varphi')=(g*S|(e^{-x}\varphi)')+(g*S|e^{-x}\varphi)\\
            &=-((g*S)'|e^{-x}\varphi)+(g*S|e^{-x}\varphi)=0.
        \end{align*}
        ただし,$(g*S)'=g'*S=g*S$を用いた.
    \end{enumerate}
\end{Proof}

\begin{example}
    $\PV\paren{\frac{1}{x}}$の$\S$上での階数を与えよ.
\end{example}
\begin{Proof}
    任意の$\varphi\in\S(\R)$に対して,
    Taylorの定理を念頭に,
    \[\varphi(x)=\varphi(0)+x\varphi_1(x),\qquad\varphi_1(x):=\frac{\varphi(x)-\varphi(0)}{x}.\]
    と変形すると,中間値の定理より,
    \[\sup_{\abs{x}\le R}\abs{\varphi_1(x)}=\sup_{\abs{x}\le R}\Abs{\frac{\varphi(x)-\varphi(0)}{x}}\le\sup_{\abs{x}\le R}\abs{\varphi'(x)}.\]
    これを踏まえて,
    \begin{align*}
        \paren{\PV\frac{1}{x}\middle|\varphi}&=\lim_{\ep\to0}\int_{\abs{x}\ge\ep}\frac{\varphi(x)}{x}dx\\
        &=\lim_{\ep\to0}\int_{\ep\le\abs{x}\le1}\paren{\frac{\varphi(0)}{x}+\varphi_1(x)}dx+\int_{\abs{x}\ge1}\frac{\varphi(x)}{x}dx\\
        &=\lim_{\ep\to0}\int_{\ep\le\abs{x}\le1}\varphi_1(x)dx+\int_{\abs{x}\ge1}\frac{x\varphi(x)}{x^2}dx\\
        &\le2\sup_{\abs{x}\le 1}\abs{\varphi'(x)}+\sup_{x\in\R}\abs{x\varphi(x)}\int_{\abs{x}\ge1}\frac{dx}{x^2}\\
        &\le2\sup_{x\in\R}\abs{\varphi'(x)}+2\sup_{x\in\R}\abs{x\varphi(x)}\le2\norm{\varphi}_1.
    \end{align*}
    第二項の評価は\href{https://math.stackexchange.com/questions/2708497/principal-value-of-1-x}{stackexchange},第一項の評価は\cite{Grubb09-Distributions}による.

    全く同様の評価が行えて(第二項の評価は必要なくなるが),$\D$上でも階数$1$である.
\end{Proof}
\begin{remark}
    こうやって評価するのか.特異点($0$での発散と無限大での発散)は分解して一つ一つ和として処理するために,$\D,\S$のノルムの定義が一様ノルムと和で2つの流儀で与えていたのかもしれない.
\end{remark}

\end{document}