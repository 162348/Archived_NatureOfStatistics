\documentclass[uplatex,dvipdfmx]{jsreport}
\title{群上の調和解析に向けて}
\author{司馬博文}
\date{\today}
\pagestyle{headings} \setcounter{secnumdepth}{4}
\input{/Users/Hirofumi Shiba/NatureOfStatistics/preamble_no_fonts.tex}
%\input{/Users/hirofumi.shiba48/NatureOfStatistics/preamble_no_fonts.tex}
\usepackage[math]{anttor}
\begin{document}
\tableofcontents

\chapter{Hilbert空間}

\begin{quotation}
    完備な距離を内積が誘導する場合をHilbert空間という.
    これは,標準的なペアリングが存在するBanach空間論と見れる.
    些細な違いかもしれないが,理論がフルパワーを発するのはこの場合のみである.
    特に,Banach空間上の作用素にはほとんど一般論が成り立たない.
    例えばHilbert空間には,標準的なペアリングを通じてRieszの表現が存在する.

    Banach空間上に,内積$L^p(X)\times L^q(X)\ni(f,g)\mapsto\brac{f,g}:=\int f\o{g}$によって,
    $(L^p(X))^*\simeq L^q(X)$なる同型が誘導される\ref{thm-duality-of-Lp}.
    2次のノルムがHilbert空間として重要である理由は,自身と共役だからである.
\end{quotation}

\section{内積が拓く描像}

\begin{tcolorbox}[colframe=ForestGreen, colback=ForestGreen!10!white,breakable,colbacktitle=ForestGreen!40!white,coltitle=black,fonttitle=\bfseries\sffamily,
title=]
    Hilbert空間には内積という名前の標準的なペアリングが存在する.
    これを通じてRieszの表現が存在し,また直交性の概念が定義される.
    これは基底論にも影響を与え,Hilbert空間の分類は有限次元線型空間論の延長に落ちる.
\end{tcolorbox}

\begin{history}
    はじめは,Fredholmの積分方程式論,Fourier級数・積分論を綜合して,固有値問題を一般的に扱うために生まれた理論であり,
    von Neumannによる量子力学の基礎付けの研究を通じて整えられた.
\end{history}

\subsection{半双線型形式とセミノルムの関係}

\begin{tcolorbox}[colframe=ForestGreen, colback=ForestGreen!10!white,breakable,colbacktitle=ForestGreen!40!white,coltitle=black,fonttitle=\bfseries\sffamily,
title=]
    Hilbert空間とは,Banach空間であって,ノルムが中線定理
    \[\norm{x+y}^2+\norm{x-y}^2=2(\norm{x}^2+\norm{y}^2)\]
    を満たすような空間である.
    中線定理が成り立つとき,
    \[4(x|y)=\sum^3_{k=0}i^k\norm{x+i^ky}^2=\norm{x+y}+i\norm{x+iy}-\norm{x-y}-i\norm{x-iy}\]
    に従えばこれは内積を定める.このような,ノルムと内積の関係を\textbf{極化恒等式}という.
    一方で双線型形式は二次形式と対応し,これも極化恒等式と呼ばれる.
\end{tcolorbox}

\begin{definition}[sesquilinear form, adjoint form, self-adjoint, positive, semi-inner product, inner product]
    $\bF$-線型空間$X$について,
    \begin{enumerate}
        \item 写像$(-|-):X\times X\to\bF$が\textbf{半双線型形式}であるとは,第一引数について線型で,第二引数について共役線型であることをいう.\footnote{数理物理では逆.}$\bF=\R$のときは双線型性に同値.
        \item 半双線型形式$(-|-)$の\textbf{随伴形式}$(-|-)^*$とは,$(x|y)^*:=\o{(y|x)}$で定まる半双線型形式をいう.
        \item $(-|-)^*=(-|-)$を満たす半双線型形式を\textbf{自己共役}という.$\bF=\R$であるときは対称ともいう.
        \item $\forall_{x\in X}\;(x|x)\ge 0$を満たすとき,$(-|-)$を\textbf{半正定値}という.
        \item 半正定値な自己共役半双線型形式を,$X$上の\textbf{半内積}という.
        \item $(x|x)=0\Rightarrow x=0$を満たす半内積を,$X$上の\textbf{内積}という.
    \end{enumerate}
\end{definition}

\begin{lemma}[自己共役性の特徴付け]\label{lemma-characterization-of-self-adjointness}
    $\C$-線型空間上の半双線型形式について,
    \begin{enumerate}
        \item $4(x|y)=\sum^3_{k=0}i^k(x+i^ky|x+i^ky)$.
        \item $(-|-)$が自己共役であることは,$\forall_{x\in X}\;(x|x)\in\R$に同値.特に,$(-|-)$が正定値ならば自己共役である.\footnote{実数値の正定値双線型形式が対称とは限らない.}
    \end{enumerate}
\end{lemma}

\begin{proposition}[polarization identity, parallellogram law]\label{prop-polarization-identity}
    半内積$(-|-):X\times X\to\bF$について,
    \begin{enumerate}
        \item 関数$\norm{-}:X\to\R_+$を$\norm{x}:=(x|x)^{1/2}$で定めると,これは斉次関数である.
        \item 次の\textbf{極化恒等式}が成り立つ:
        \begin{enumerate}[(a)]
            \item $\bF=\C$のとき,$4(x|y)=\sum^3_{k=0}i^k\norm{x+i^ky}^2$.
            \item $\bF=\R$のとき,$4(x|y)=\norm{x+y}^2-\norm{x-y}^2$.
            \item $\bF=\C$のとき,実部と虚部に分けて$\norm{x+y}^2=\norm{x}^2+2\Re\brac{x,y}+\norm{y}^2$も極化恒等式と呼ぶ.
        \end{enumerate}
        \item (Cauchy-Bunyakowsky-Schwarz) $\abs{(x|y)}\le\norm{x}\norm{y}$.
        特に,$\norm{-}:X\to\R_+$は劣加法的であり,$X$上のセミノルムを定める.
        $(-|-)$が内積であるとき,$\norm{-}$はノルムを定める.
        \item セミノルム$\norm{-}$について中線定理が成り立つ:$\norm{x+y}^2+\norm{x-y}^2=2(\norm{x}^2+\norm{y}^2)$.
        \item (Frechet-von Neumann-Jordan) ノルム$\norm{-}$が中線定理を満たすとき,(2)の極化恒等式によって定まる半双線型形式は内積を定める.
    \end{enumerate}
\end{proposition}
\begin{remarks}
    本質的には次の4式のみである:
    \begin{align*}
        (x+y|x+y)&=(x|x)+(y|y)+2\Re(x|y),&(x-y|x-y)&=(x|x)+(y|y)-2\Re(x|y),\\
        (x-y|x+y)&=(x|x)-(y|y)+2\Im(x|y)i,&(x+y|x-y)&=(x|x)-(y|y)-2\Im(x|y)i,\\
        (x+iy|x+iy)&=(x|x)+(y|y)+2\Im(x|y),&(x-iy|x-iy)&=(x|x)+(y|y)-2\Im(x|y).
    \end{align*}
    第2行も同じくらい示唆的ではあるが,ノルムは作り出せないのであろう.
\end{remarks}

\begin{definition}[orthogonal]
    $(-|-):X\times X\to\bF$を半双線型形式とする.
    \begin{enumerate}
        \item ベクトル$x,y$が直交する$x\perp y$とは,$(x|y)=0$を満たすことをいう.
        \item 部分集合$Y,Z\subset X$が直交する$Y\perp Z$とは,$\forall_{y\in Y,z\in Z}\;y\perp z$を満たすことをいう.
        \item 部分集合$X\subset H$について,$X^\perp:=\Brace{x^\perp\in H\mid x^\perp\perp X}$と表すと,$X^\perp$は閉部分空間である.
    \end{enumerate}
\end{definition}

\begin{lemma}[Pythagoras identity]\label{lemma-Pythagoras-identity}
    $X$が実線型空間であるとき,
    $x,y\in X$について,次の2条件は同値.
    \begin{enumerate}
        \item $x\perp y=0$.
        \item $\norm{x+y}^2=\norm{x}^2+\norm{y}^2$.
    \end{enumerate}
\end{lemma}
\begin{Proof}
    (1)$\Rightarrow$(2)は極化恒等式(c)より.
    (2)$\Rightarrow$(1)は,極化恒等式(c)より$\Re(x|y)=0$と,$(x|iy)=0$より$\Im(x|y)=0$を,別々に得る.
\end{Proof}

\subsection{Hilbert空間の同型類}

\begin{tcolorbox}[colframe=ForestGreen, colback=ForestGreen!10!white,breakable,colbacktitle=ForestGreen!40!white,coltitle=black,fonttitle=\bfseries\sffamily,
title=]
    Hilbert空間の,正規直交基底$\{e_\al\}$が定まる係数空間への対応$H\to l^2(A)$をFourier変換といい,これが内積を保つことと$\{e_\al\}$が実際に正規直交基底であることは同値(Parsevalの等式).
    そこで,Hilbert空間の同型類は,係数の空間$l^2(A)$を分類すれば良い.
    これが抽象的なFourier変換論である.
\end{tcolorbox}

\begin{definition}[pre-Hilbert space]
    内積空間$H$が,付随するノルムについてBanach空間をなすとき,これを\textbf{Hilbert空間}という.
    また内積空間を前Hilbert空間ともいう.
\end{definition}

\begin{example}[二乗可積分関数の空間:特殊から一般へ]\mbox{}
    \begin{enumerate}
        \item Euclid空間$\bF^n$は,通常の内積についてHilbert空間をなし,付随するノルムは2-ノルムである.
        \item $l^2(\Z):=\Brace{(a_n)_{n\in\Z}\in\prod_{n\in\Z}\bF\;\middle|\;\sum_{n\in\Z}\abs{a_n}^2<\infty}$は,内積$((a_n)|(b_n))=\sum_{n\in\Z}a_n\o{b_n}$に関してHilbert空間をなす.
        \item コンパクト台を持つ関数の空間$C_c(\R^n)$は,内積$(f|g):=\int f(x)\o{g(x)}dx$についてpre-Hilbert空間をなし,付随するノルムは2-ノルムである.これを完備化したものは$L^2(\R^n)$であった\ref{exp-Banach-spaces}が,これがHilbert空間である.
        \item 一般に,局所コンパクトハウスドルフ空間$X$上のRadon積分$\int$について二乗可積分な関数のなす空間の完備化$L^2(X)$\ref{exp-Banach-space-of-Radon-integrable-functions}は,内積$(f|g)=\int f\o{g}$についてHilbert空間となる.
    \end{enumerate}
\end{example}

\begin{definition}[orthogonal sum / direct sum of Hilbert space]\label{def-orthogonal-sum-of-Hilbert-spaces}
    Hilbert空間の族$(H_j)_{j\in J}$について,
    \begin{enumerate}
        \item 代数的直和$\sum_{i\in J}H_j$には,$(x|y):=\sum_{j\in J}(\pr_jx|\pr_jy)$によって内積が定まる.
        \item この内積に付随するノルムは2-ノルムであり,これについての完備化を\textbf{(直交)直和}といい,$\oplus_{j\in J}H_j$と表す.
        \item 命題\ref{prop-completion-of-algebraic-direct-product}より,集合としては
        \[\bigoplus_{j\in J}=\Brace{x\in\prod_{j\in J}H_j\;\middle|\;\sum_{j\in J}\norm{\pr_j(x)}^2<\infty}\]
        と表せる.特に,Hilbert空間の直和$\oplus H_j$の元$x$は,可算個の$j\in J$を除いて$\pr_j(x)=0$である.
    \end{enumerate}
\end{definition}
\begin{remarks}
    これがHamel基底を超える,待ち望まれた無限次元の直和空間の構成である.
    有限の場合はHamel基底と変わらないが,無限の場合は,無限和が二乗収束するものからなる空間を考えると完備な内積が定まる.
\end{remarks}

\begin{corollary}[数ベクトルへの対応こそがFourier変換]
    $H$をHilbert空間,
    $\{e_n\}_{n\in\N}\subset H$を正規直交な列とする.
    \begin{enumerate}
        \item (Besselの不等式) 係数への対応$\F:H\to l^2(\N);x\mapsto(x|e_n)_{n\in\N}$はノルム減少的で全射な有界作用素である:$\sum_{n\in\N}\abs{(x|e_n)}^2\le\norm{x}$.
        \item (Riesz-Fischerの定理) $H$が可分のとき,$\F:H\to l^2(\N)$は等長でもある.すなわち,Hilbert空間の同型を定める.
        \item (Parsevalの等式) $H$が可分のとき,条件$\forall_{x,y\in H}\;(\F x|\F y)_{l^2(\N)}=\sum_{n\in\N}\F x\o{\F x}=(x|y)_H$は$\{e_n\}$が正規直交基底をなすことに同値.
    \end{enumerate}
\end{corollary}

\subsection{Hilbert空間の直交分解}

\begin{tcolorbox}[colframe=ForestGreen, colback=ForestGreen!10!white,breakable,colbacktitle=ForestGreen!40!white,coltitle=black,fonttitle=\bfseries\sffamily,
title=]
    Hilbert空間の任意の閉部分空間は,閉な直交補空間を持つ.
    直交分解を距離の言葉によって議論するところは,商ノルムの定義と同じ作戦である.
\end{tcolorbox}

\begin{lemma}
    $C$をHilbert空間$H$の非空な閉凸集合とする.このとき,任意の$y\in H$に対して,距離$d(y,C)$を最小にするときの点$x=\argmin_{x\in C}d(y,x)\in C$が唯一つ存在する.
\end{lemma}

\begin{theorem}
    任意の閉部分空間$X\subset H$について,
    \begin{enumerate}
        \item 任意の元$y\in H$は一意的な分解$y=x+x^\perp\in X\oplus X^\perp$を持つ.
        また,$H=X\oplus X^\perp$と直交直和で表せる.
        \item この$x\in X$は$y$に一番近い点$\argmin_{x\in X} d(y,x)$であり,$x^\perp\in X^\perp$も$y$に一番近い点$\argmin_{x^\perp\in X^\perp} d(y,x^\perp)$である.
        \item $(X^\perp)^\perp=X$が成り立つ.
    \end{enumerate}
\end{theorem}

\begin{corollary}[closed linear span]\label{cor-expression-of-closed-linear-span}
    任意の部分集合$X\subset H$について,$X$を含む最小の閉部分空間は$(X^\perp)^\perp$である.
    特に,$X$が$H$の部分空間ならば,$\dbloverline{X}=(X^\perp)^\perp$.
\end{corollary}

\begin{corollary}[部分空間の稠密性の直交補空間による特徴付け]\label{cor-dense-subspace}
    $X$を$H$の部分空間とする.次の2条件は同値.
    \begin{enumerate}
        \item $X$は$H$上稠密である.
        \item $X^\perp=0$.
    \end{enumerate}
\end{corollary}

\subsection{Rieszの表現定理}

\begin{tcolorbox}[colframe=ForestGreen, colback=ForestGreen!10!white,breakable,colbacktitle=ForestGreen!40!white,coltitle=black,fonttitle=\bfseries\sffamily,
title=]
    Rieszの表現定理は変種が多々あり,いずれもある種の位相線型空間とその双対空間との間に(反)同型を取る知識である.

    なお,
    Metの同型は全射な等距離写像である(全単射な等長写像は,等長な逆を持つ).
    これは擬距離空間やRiemann多様体では成り立たない.
\end{tcolorbox}

\begin{proposition}[Riesz representation theorem]\label{prop-isometry-of-Hilbert-dual}
    写像$\Phi:H\to H^*;x\mapsto(-|x)$は,共役線型な等長同型である.
\end{proposition}

\subsection{Hilbert空間の弱位相}

\begin{tcolorbox}[colframe=ForestGreen, colback=ForestGreen!10!white,breakable,colbacktitle=ForestGreen!40!white,coltitle=black,fonttitle=\bfseries\sffamily,
title=]
    $H$の閉単位球は弱コンパクトである.さらに$H$が可分ならば$B$は弱位相に関して距離化可能である.
    弱収束列$\{x_n\}$が$\norm{x_n}\to\norm{x}$も満たすことがノルム収束する必要十分条件である.
\end{tcolorbox}

\begin{definition}[weak topology on Hilbert space]\label{def-weak-topology-on-H(B)}
    Hilbert空間$H$上の弱位相とは,有界な線型汎函数の集合$H^*$が定める始位相をいう.
    これは,標準的な同型$\Phi:H\to H^*$ \ref{prop-isometry-of-Hilbert-dual}により,$H^*$上の$w^*$-位相を引き戻したものに一致する.
    よって特に,Alaogluの定理\ref{thm-Alaoglu}より,$H$内の単位球は弱コンパクトである.
\end{definition}

\begin{proposition}[Hilbert空間上の有界線型作用素の特徴付け]\label{lemma-characterization-of-bounded-operator-on-Hilbert-space}
    任意のHilbert空間の作用素$T:H\to H$について,次の2条件は同値.
    \begin{enumerate}
        \item $T\in B(H)$である($H$のノルム位相について連続,すなわち,有界).
        \item $T$は弱位相について連続である(weak-weak continuous).
        \item $T$はノルム-弱連続である(norm-weak continuous).
    \end{enumerate}
\end{proposition}

\begin{proposition}[有限ランク作用素の十分条件]
    weak-norm連続な作用素は,有限なランクを持つ.
\end{proposition}

\subsection{正規直交系}

\begin{tcolorbox}[colframe=ForestGreen, colback=ForestGreen!10!white,breakable,colbacktitle=ForestGreen!40!white,coltitle=black,fonttitle=\bfseries\sffamily,
title=]
    任意の線型空間には,極大な線型独立系が存在するという意味で,Hamel基底を持つ.
    Hilbert空間では,内積から定まる基底が存在し,これは常にHamel基底とは異なる.
    正規直交基底を用いて,Hilbert空間の同型を特徴づけることが出来る.
\end{tcolorbox}

\begin{definition}[orthonormal, orhonormal basis]
    Hilbert空間$H$の部分集合$\{e_j\}_{j\in J}$について,
    \begin{enumerate}
        \item $\{e_j\}_{j\in J}$が\textbf{正規直交}であるとは,$\forall_{j\in J}\;\norm{e_j}=1$かつ$(e_j|e_i)=\delta_{ij}$を満たすことをいう.
        \item $\{e_j\}_{j\in J}$が\textbf{正規直交基底}であるとは,生成する部分空間が$H$上稠密であることをいう:$\dbloverline{\brac{e_j}_{j\in J}}=H$.これは,$\oplus_{j\in J}\bF e_j=H$に同値\ref{def-orthogonal-sum-of-Hilbert-spaces}.すなわち,2-ノルムで収束する表示$x=\sum_{j\in J}\al_j e_j$が存在する.
        \item (Parseval identity) $\norm{x}^2=\sum_{j\in J}\abs{\al_j}^2\;\paren{x=\sum_{j\in J}\al_jx_j}$が成り立つ.
    \end{enumerate}
\end{definition}
\begin{Proof}
    (3)は,$(x|x)=\sum_{j\in J}\abs{\al_j}^2$であるが,内積$(-|-):H\times H\to\bF$が連続であることより,右辺は収束する.
\end{Proof}

\begin{proposition}
    Hilbert空間$H$の任意の正規直交系は,正規直交基底へと拡大できる.
\end{proposition}
\begin{Proof}\mbox{}
    \begin{description}
        \item[方針] $\{e_j\}_{j\in J_0}\subset H$を正規直交系とする.
        $\{e_j\}_{j\in J_0}$を含む$H$の正規直交系全体からなる集合は,包含関係について帰納的順序集合を定めるから,Zornの補題より極大元$\{e_j\}_{j\in J}\;J_0\subset J$が存在する.
        これが生成する閉部分空間を$X$としたとき,$X=H$を示せば良い.
        \item[証明] 
        $X\ne H$のとき,$X^\perp\ne 0$だから,ある$e\in X^\perp$が存在して,$\forall_{j\in J}\;(e_j|e)=0,(e|e)=1$を満たす.
        これは$\{e_j\}_{j\in J}$の極大性に矛盾する.
    \end{description}
\end{Proof}

\begin{theorem}[Gram-Schmidt Orthogonalization Process]
    $\{h_n\}_{n\in\N}$を線型独立な部分集合とする.このとき,正規直交系$\{e_n\}_{n\in\N}$が存在して,$\forall_{n\in\N}\;\brac{e_1,\cdots,e_n}=\brac{h_1,\cdots,h_n}$.
\end{theorem}

\begin{proposition}[Hilbert空間の同型類]\label{prop-characterization-of-isomorphism-of-Hilbert-spaces}
    $H,K$をHilbert空間とし,$\{e_i\}_{i\in I},\{f_j\}_{j\in J}$をそれぞれの正規直交基底とし,$I$と$J$は集合として同型であるとする.
    この時,等長同型$U:H\to K$が存在して,$\forall_{x,y\in H}\;(Ux|Uy)=(x|y)$を満たすものが存在する.
\end{proposition}

\begin{proposition}[直積空間の正規直交基底は積で与えられる]
    $L^2(X),L^2(Y)$の正規直交基底$\phi_n,\psi_n$について,$\Brace{\phi_m\psi_n}_{(m,n)\in\N^2}$は$L^2(X\times Y)$の正規直交基底である.
\end{proposition}

\section{作用素の基礎理論}

\begin{tcolorbox}[colframe=ForestGreen, colback=ForestGreen!10!white,breakable,colbacktitle=ForestGreen!40!white,coltitle=black,fonttitle=\bfseries\sffamily,
title=具体的$C^*$-環論は抽象的$C^*$-論と等価]
    任意の単位的$C^*$-代数$\A$に対して,Hilbert空間$H$が存在して,$*$-等長同型$\Phi:\A\mono B(H)$が存在する.
    \begin{enumerate}
        \item $B(H)$は単位的$C^*$-環になる.
        \item 自己共役な作用素のなす部分空間$B(H)$は,$*$-作用素について不変な閉部分空間$B(H)_\sa$である.数域半径は作用素ノルムと同値なノルムを定めるが,特に$B(H)_\sa$上では値が一致する.
        \item 正作用素のなす閉凸錐$B(H)_+$は,$\forall_{x\in H}\;(Tx|x)\ge0$を満たす$B(H)_\sa$の部分集合である.これを通じて,$B(H)_\sa$には順序が定まる.
        \item 正射影の空間$B(H)_p$は,$0\le P\le I$を満たす$B(H)_+$の部分集合であり,束の構造を持つ.これは特性関数に当たる.$B_0(H)$には直交射影からなる近似的単位元が存在する.
        \item 正規直交基底$(e_j)_{j\in J}$に関して有界な係数集合$\{\lambda_j\}\subset\bF$による和で表せる作用素を対角化可能といい,その全体を$B(H)_\diag$で表す.これは可測関数のようなものである.
        \item 正規作用素の空間$B(H)_\normal$は$B(H)_\diag$のノルム閉包である.この上ではスペクトル半径と数域半径が一致する.すると,対角化可能で固有値が無限遠で消えるクラスは,正確に正規なコンパクト作用素$B_0(H)\cap B_\normal(H)$であると分かる.
        \item ユニタリ作用素のなす群$U(H)\subset B(H)$は弧状連結である.$B(H)$の開球の元はユニタリ作用素の凸結合(特に平均)として得られる(Russo-Dye-Gardner).
        \item 跡類作用素とHilbert-Schmidt作用素$B_f(H)\subset B^1(H)=\Span\Brace{T\in B_0(H)\mid T\ge0,\Tr(T)<\infty}\subset B^2(H):=\Brace{T\in B_0(H)\mid\Tr(T)<\infty}$は$B(H)$内の自己共役なイデアルをなす.跡が定めるペアリング$\brac{S,T}:=\Tr(ST)$により,$(B_0(H))^*=B^1(H)$かつ$(B^1(H))^*=B(H)$の関係がある.
    \end{enumerate}
    さらに一般に,$B(H)$を$C^*$-代数$\A$,$B(H)_\sa$を$\Re\A$に一般化して考察する.
\end{tcolorbox}

\begin{notation}
    $H$をHilbert空間,$(-|-)$をその内積,$B(H)$をその上の有界な自己準同型,$I\in B(H)$を恒等写像$\id_H$とする.
\end{notation}

\begin{history}
    この一般化に際して,Hilbertは二次形式/双線型形式の不変式論を04-10に研究していて,\footnote{不変式論のテーマは,BooleからCayleyに引き継がれてから,大陸を渡ってHilbertに届いた.2人ともlogicに入る前は不変式論の研究をしていた.}その時にスペクトル理論を構築したが,
    基底の選択と無限次元行列としての表現,そして積は畳み込みとして成分ごとに計算するというのはあまりにも煩雑であった.
    von Neumannの成功は,補題の同型を渡って,作用素の概念の方に注目したことが大きい.
    これが作用素の研究の第一歩となる.
\end{history}

\subsection{随伴作用素の性質}

\begin{tcolorbox}[colframe=ForestGreen, colback=ForestGreen!10!white,breakable,colbacktitle=ForestGreen!40!white,coltitle=black,fonttitle=\bfseries\sffamily,
title=]
    Hilbert空間では,標準的なペアリングを通じて任意の有界線型作用素を表現できるのであった.
    これは随伴と呼ばれる対合$*:B(H)\to B(H)$を作用素の間に定める.
    これは行列の共役転置の一般化である.
    またこうして自己共役性の概念が内積から作用素へ流入する.
\end{tcolorbox}

\begin{lemma}[有界作用素の内積による特徴付け]\label{lemma-correspondence-between-sesquilinearform-and-operator}
    Hilbert空間$H$上の有界線形作用素と,これが内積$(-|-)$を通じて定める有界な半双線型形式との
    次の対応は等長同型である:
    \[\xymatrix@R-2pc{
        B(H)\ar[r]&B(H\otimes H^*;\bF)\\
        \rotatebox[origin=c]{90}{$\in$}&\rotatebox[origin=c]{90}{$\in$}\\
        T\ar@{|->}[r]&B_T(x,y):=(x|Ty)
    }\]
    ただし,終域となっている空間のノルムは作用素ノルム$\norm{B_T}:=\sup\Brace{\abs{B_T(x,y)}\in\R_+\mid\norm{x}\le 1,\norm{y}\le1}$とする.
\end{lemma}
\begin{remarks}
    右辺側のノルムと同様の発想で
    \[\nnorm{T}:=\sup_{x\in B}\abs{(Tx|x)}\]
    としたものを\textbf{数域半径}とよび,作用素ノルムと同値なノルムを定める.
\end{remarks}

\begin{theorem}[随伴作用素の定義と性質]\mbox{}\label{thm-existence-of-adjoint-operator}
    \begin{enumerate}
        \item 任意の$T\in B(H)$に対して,$\forall_{x,y\in H}\;(Tx|y)=(x|T^*y)$を満たす$T^*\in B(H)$が唯一つ存在する.
        \item これが定める全単射な対応${}^*:B(H)\to B(H)$について,
        \begin{enumerate}[(a)]
            \item 対合的(周期2)である:$T^{**}=T$.
            \item 共役線型である:$(aT)^*=\o{a}T^*$.
            \item 乗法について反変的である:$(ST)^*=T^*S^*$.\footnote{antimultiplicativeと表現されている.homomorphismが積を保つのに対し,antihomomorphismとは積を逆にする.}
            \item 等長写像である:$\norm{T}=\norm{T^*}$.
            \item $\norm{T^*T}=\norm{T}^2$を満たす.
        \end{enumerate}
    \end{enumerate}
\end{theorem}

\begin{example}[随伴]\mbox{}
    \begin{enumerate}
        \item 乗算作用素\ref{operator-multiplication}については$M^*_\varphi=M_{\o{\varphi}}$となる.よって,$\varphi\in L^\infty(^mu)$が実数値である場合に限り,自己共役である.$\abs{\phi}=1$である場合に限り,ユニタリである.
        \item $k$を核とする積分作用素$K$\ref{operator-integral-transformation}については$K^*$は$k^*(x,y)=\o{k(y,x)}$を核とする積分作用素となる.よって,$k(x,y)=\o{k(y,x)}\;\ae[\mu\times\mu]$である場合に限り,自己共役である.
        \item shift作用素\ref{operator-unilateral-shift}は方向が逆になる:$S^*(\al_1,\al_2,\cdots)=(\al_2,\al_3,\cdots)$.これを\textbf{後方シフト(backward shift)}と呼ぶ.
        \item 複素Hilbert空間$H$上の作用素$A\in B(H)$について,$B:=\frac{A+A^*}{2},C:=\frac{A-A^*}{2}$をそれぞれ実部と虚部と呼び,自己共役である.
    \end{enumerate}
\end{example}

\subsection{自己共役作用素}

\begin{tcolorbox}[colframe=ForestGreen, colback=ForestGreen!10!white,breakable,colbacktitle=ForestGreen!40!white,coltitle=black,fonttitle=\bfseries\sffamily,
title=]
    自己共役作用素はノルムに対して特殊な振る舞いをする.
    特に,作用素ノルムの特徴付け$\norm{T}=\sup_{x\in\partial B}\abs{(Ax|x)}\;(T\in B(H)_\sa)$は本質的であり,
    自己共役作用素を調べるのに,半内積$(T-|-)$の構造と,これについての一般化Cauchy-Schwarzの不等式$\abs{(Ax|y)}\le\sqrt{(Ax|x)(Ay|y)}$を多用する.
\end{tcolorbox}

\begin{definition}[self-adjoint / hermitian]\mbox{}
    \begin{enumerate}
        \item 乗法についての反準同型$*:B(H)\to B(H)$が対合な共役線型写像で,等長同型でもあるとき,$(B(H),*)$を$B^*$-代数という.$C^*$-性$\norm{T^*T}=\norm{T}^2$も満たすとき,$C^*$-代数という.\footnote{最初に定義したI. E. Segal in 1947で,$C$は"Closed"から取られた.\url{https://en.wikipedia.org/wiki/C*-algebra}}
        \item $T\in B(H)$が$T=T^*$を満たすとき,これを自己共役作用素という.これを$B_\sa(H)=B(H)^{\Brace{*}}$\footnote{群作用に対する不変式の記法を踏襲した.}と表すと,$B(H)$の閉な実部分空間となる.
    \end{enumerate}
\end{definition}


\begin{corollary}[自己共役作用素の特徴付け]\label{cor-characterization-of-self-adjointness}
    $\bF=\C$のとき,次の2条件は同値.
    \begin{enumerate}
        \item $T=T^*$.
        \item $\forall_{x\in H}\;(Tx|x)\in\R$.
    \end{enumerate}
\end{corollary}
\begin{Proof}
    補題\ref{lemma-characterization-of-self-adjointness}で与えた.
    $H$が実Hilbert空間という仮定のみでは,その上の任意の作用素$A\in B(H)$に対して$\brac{Ah,g}\in\R$が常に成り立つので,特徴付けにならないことに注意.
\end{Proof}

\subsection{作用素ノルムと数域}

\begin{tcolorbox}[colframe=ForestGreen, colback=ForestGreen!10!white,breakable,colbacktitle=ForestGreen!40!white,coltitle=black,fonttitle=\bfseries\sffamily,
title=]
    数域半径は同値なノルムを定め,さらに$B(H)_\nor$上では作用素ノルムに一致する.
\end{tcolorbox}

\begin{proposition}[自己共役作用素の作用素ノルム]\label{prop-operator-norm-of-self-adjoint-operator}
    $A$を自己共役作用素とする.このとき,作用素ノルムは
    \[\norm{A}=\sup\Brace{\abs{(Ax|x)}\in\R\mid\norm{x}=1}=\sup_{x\ne0}\frac{\abs{(Ax|x)}}{\norm{x}^2}.\]
    と,Rayleigh商の全変動にも等しい.
\end{proposition}

\begin{corollary}[半内積の非退化性]\label{cor-nondegeneratedness-of-semi-inner-product}\mbox{}
    \begin{enumerate}
        \item $A$を自己共役作用素とする.このとき,$\forall_{x\in H}\;(Ax|x)=0\Rightarrow A=0$.
        \item $H$が複素Hilbert空間ならば,一般の$A$について成り立つ.
    \end{enumerate}
\end{corollary}

\begin{proposition}[数域半径は同値なノルムを定める]
    $\nnorm{\cdot}$は$B(H)$上のノルムで,$\forall_{T\in B(H)}\;\frac{1}{2}\norm{T}\le\nnorm{T}\le\norm{T}$を満たす.
\end{proposition}
\begin{Proof}
    $\nnorm{T}$の斉次性と劣加法性は明らかで,非退化性は後半の主張から従う.また,Cauchy-Schwarzの不等式より$\nnorm{T}\le\norm{T}$もわかる.
    よって,$\norm{T}\le 2\nnorm{T}$を示せば良い.
    任意の単位ベクトル$x,y\in H$について,極化恒等式と中線定理より
    \begin{align*}
        4\abs{(Tx|y)}&=\Abs{\sum^3_{k=0}i^k(T(x+i^ky)|x+i^ky)}\\
        &\le\nnorm{T}(\norm{x+y}^2+\norm{x-y}^2+\norm{x+iy}^2+\norm{x-iy}^2)\\
        &=\nnorm{T}2(\norm{x}^2+\norm{y}^2+\norm{x}^2+\norm{iy}^2)=8\nnorm{T}
    \end{align*}
    これで,半双線型形式のノルムについて$\norm{(T-|-)}\le2\nnorm{T}$を得たが,等長同型\ref{lemma-correspondence-between-sesquilinearform-and-operator}より結論を得る.
\end{Proof}
\begin{remarks}
    また,一般の線型写像について,有界であることと$\nnorm{T}<\infty$になることは同値になる.
\end{remarks}

\begin{proposition}[数域半径の特徴付け]
    作用素$T$の自己共役部分を
    $\Re(T):=\frac{1}{2}(T+T^*)$で表す.
    \[\forall_{T\in B(H)}\;\nnorm{T}=\max\Brace{\norm{\Re(\theta T)}\in\R_+\mid\theta\in\C,\abs{\theta}=1}.\]
\end{proposition}

\begin{proposition}
    任意の$T\in B(H)$について,
    \begin{enumerate}
        \item $\nnorm{T^2}\le\nnorm{T}^2$.
        \item $T$が正規であるとき,等号成立.
    \end{enumerate}
\end{proposition}

\subsection{随伴と直交の関係}

\begin{tcolorbox}[colframe=ForestGreen, colback=ForestGreen!10!white,breakable,colbacktitle=ForestGreen!40!white,coltitle=black,fonttitle=\bfseries\sffamily,
title=]
    ${}^*:B(H)\to B(H)$というのはある種${}^\op$のように使える.
    作用素の可逆性は,随伴の消息を以て特徴付けることが出来る.
\end{tcolorbox}

\begin{proposition}\label{prop-Ker-of-adjoint-operator}
    任意の$T\in B(H)$に対して,$\Ker T^*=(\Im T)^\perp$.
\end{proposition}
\begin{Proof}
    $\forall_{x,y\in H}\;(x|T^*y)=(Tx|y)$の下で,
    $y\in\Ker T^*\Rightarrow (T(H)|y)=0$より,$y\in(T(H))^\perp$.
    逆に$y\in(T(H))^\perp\Rightarrow (H|T^*y)=0\Rightarrow T^*y\in H^\perp=\{0\}$.
\end{Proof}
\begin{remark}
    双対的な関係は$(\Ker A)^\perp=\oo{\Im A^*}$までしか成り立たない.
\end{remark}

\subsection{Banの同型の特徴付け}

\begin{proposition}\label{prop-characterization-of-invertibleness-of-operator}
    $T\in B(H)$について,次の6条件は同値.
    \begin{enumerate}
        \item $T$は可逆:$T^{-1}\in B(H)$.
        \item $T^*$は可逆.
        \item $T,T^*$はbounded away from zero:$\exists_{\ep>0}\;\forall_{x\in H}\;\norm{Tx}\ge\ep\norm{x}$.%$\inf d(0,\Im T)>0$.
        \item $T,T^*$は単射で,$\Im T$はノルム閉.
        \item $T$は全単射.
        \item $T,T^*$は全射.
    \end{enumerate}
\end{proposition}
\begin{Proof}\mbox{}
    \begin{description}
        \item[(1)$\Leftrightarrow$(2)] $(T^{-1}T)=(TT^{-1})=I$であるとき,${}^*:B(H)\to B(H)$の劣乗法性より$(T^{-1})^*$が$T^*$の逆射である.また$T^*T^{*-1}=T^{*-1}T^*=I$のときも同様に$*$を作用させれば良い.
        \item[(1)$\Rightarrow$(3)] 任意の$x\in H$について$\norm{x}=\norm{T^{-1}Tx}\le\norm{T^{-1}}\norm{Tx}$より,$\ep:=\norm{T^{-1}}^{-1}$と取れば良い.
        \item[(3)$\Rightarrow$(4)] $\exists_{\ep>0}\;\forall_{x\in H}\;\norm{Tx}\ge\ep\norm{x}$は特に$\norm{Tx-Ty}\ge\ep\norm{x-y}$より,単射性$Tx=Ty\Rightarrow x=y$を含意する.またこれより,$T(H)$の任意のCauchy列$(Tx_i)_{i\in\N}$は$H$上のCauchy列$(x_i)_{i\in H}$を定めることより,$Tx$に$T(H)$は完備で,特に閉集合.
        \item[(4)$\Rightarrow$(5)] ノルム閉包の直交補空間による特徴付け\ref{cor-expression-of-closed-linear-span}より,$T(H)=\dbloverline{T(H)}=(T(H)^\perp)^\perp$.命題より,$(T(H)^\perp)^\perp=(\Ker T^*)^\perp=0^\perp=H$.
        \item[(5)$\Rightarrow$(1)] 開写像定理の系\ref{cor-inverse-mapping-theorem}より.
        \item[(6)$\Rightarrow$(5)] $\Ker T=(T^*(H))^\perp=0$より.
        \item[(1)$\Rightarrow$(6)] 明らか.
    \end{description}
\end{Proof}

\subsection{Hilbの同型の特徴付け}

\begin{tcolorbox}[colframe=ForestGreen, colback=ForestGreen!10!white,breakable,colbacktitle=ForestGreen!40!white,coltitle=black,fonttitle=\bfseries\sffamily,
title=]
    Hilbの同型は全写な等長写像である.これを等長同型またはユニタリ同型という.
    この全体を$U(H)\subset B(H)$と表そう.
    $U(H)\subset B(H)_\nor\subset (H)$.
\end{tcolorbox}

\begin{proposition}[等長写像の特徴付け]\label{prop-characterization-of-isometry}
    $V:H\to K$を線型写像とする.
    次の3条件は同値.
    \begin{enumerate}
        \item $V$は等長写像である.
        \item 内積の保存:$\forall_{h,g\in H}\;(Vh|Vg)=(h|g)$.
        \item Gram行列が恒等:$A^*A=I$.
    \end{enumerate}
\end{proposition}

\begin{proposition}[等長同型の特徴付け]
    $A\in B(H)$について,次の3条件は同値.
    \begin{enumerate}
        \item 等長同型である.すなわち,$A$は全写な等長写像である.
        \item $A^*A=AA^*=I$.
        \item $A$は正規な等長写像である.
        \item $A$は可逆(Banの同型)であり,$A^{-1}=A^*$.
    \end{enumerate}
\end{proposition}

\begin{theorem}\mbox{}
    \begin{enumerate}
        \item $U\in B(H)$をユニタリ作用素とする.このとき,自己共役作用素$T\in B(H)$が存在して,$U=\exp(iT)$と表せる.
        \item ユニタリ作用素の群$U(H)\subset B(H)$はノルム位相について弧状連結である.
    \end{enumerate}
\end{theorem}

\subsection{正規作用素}

\begin{tcolorbox}[colframe=ForestGreen, colback=ForestGreen!10!white,breakable,colbacktitle=ForestGreen!40!white,coltitle=black,fonttitle=\bfseries\sffamily,
title=真に$C^*$-代数の性質を特徴づける作用素のクラスである]
    正規であるという可換性条件は,距離的な特徴付けがある.
    随伴とノルム的に判別不可能ならば,正規である.
    スペクトル定理は,正規作用素なるクラスが正確に対角化可能であることを主張する.
\end{tcolorbox}

\begin{definition}[normal]
    作用素$T\in B(H)$について,次の2条件は同値.$\F=\C$のとき,(3)も同値.
    \begin{enumerate}
        \item $T^*T=TT^*$.
        \item (metrically identical) $\norm{Tx}=\norm{T^*x}$.
        \item $T$の実部と虚部は可換である.
    \end{enumerate}
    この同値な条件を満たすとき,$T$を\textbf{正規作用素}という.
\end{definition}

\begin{corollary}[正規作用素の可逆性]\label{prop-characterization-of-invertibleness-of-normal-operator}
    正規作用素$T\in B(H)$について,次の2条件は同値.
    \begin{enumerate}
        \item $T$は可逆.
        \item $T$ is bounded away from zero.
    \end{enumerate}
\end{corollary}
\begin{Proof}
    作用素の可逆性の特徴付け\ref{prop-characterization-of-invertibleness-of-operator}と,正規性の特徴付け$\norm{Tx}=\norm{T^*x}$より.
\end{Proof}

\subsection{半正定値作用素}

\begin{tcolorbox}[colframe=ForestGreen, colback=ForestGreen!10!white,breakable,colbacktitle=ForestGreen!40!white,coltitle=black,fonttitle=\bfseries\sffamily,
title=]
    半正定値作用素は,density matrix formalismを通じて,量子状態を表す.
    自己共役作用素のうち,特別なクラスである.
    大雑把には,複素平面内で,自己共役作用素$B(H)_\sa$は実数,正作用素$B(H)_+$は非負実数と見れる.
\end{tcolorbox}

\begin{definition}[positive (semi-definite)]\label{def-positive-operator}
    $T\in B(H)$が\textbf{半正定値}であるとは,$T=T^*$かつ$\forall_{x\in H}\;(Tx|x)\ge 0$を満たすことをいう.\footnote{2つ条件があるように見えるが,$\bF=\C$かつ$T$が$H$全域で定義されているとき,半正定値性は自己共役性を含意する\ref{lemma-characterization-of-self-adjointness}.これは極化恒等式による.}
    これを「\textbf{作用素$T$は正である}」と省略して良い,$T\ge 0$と表す.正定値であることは$T>0$で表す.
\end{definition}

\begin{lemma}[正作用素の閉凸錐]\label{lemma-positive-closed-cone-of-positive-operator}
    $T_1,T_2\in B(H)$を半正定値とする.
    \begin{enumerate}
        \item $T_1+T_2$も半正定値:半正定値は作用素は凸錐をなす.
        \item 正作用素の凸錐は閉集合である.
        \item $T_1T_2$は一般には半正定値とも自己共役とも限らないが,この積が可換であるとき,正である.
    \end{enumerate}
\end{lemma}
\begin{Proof}\mbox{}
    \begin{enumerate}
        \item $\forall_{x\in H}\;(T_1x+T_2x|x)=(T_1x|x)+(T_2x|x)\ge0$.
        \item 閉性は,Cauchy列$(T_n)$を取ることによる.極限$T:=\lim_{n\to\infty}T_n$は正ではないと仮定する:$\exists_{x\in H}\;(Tx|x)<0$.$\ep:=-(Tx|x)>0$とおくと,$\exists_{N\in\N}\;\forall_{n\ge N}\;\norm{T_n-T}\le\ep/\norm{x}^2$だから,これについて,
        \[\Abs{(T_nx|x)-(Tx|x)}=\abs{((T_n-T)x|x)}\le\ep.\]
        よって,$T_n$が正であることに矛盾.
        \item 二乗根補題\ref{prop-square-root-lemma}より,$ST=(S^{1/2})^2T=S^{1/2}TS^{1/2}\ge S^{1/2}0S^{1/2}=0$.
        $\forall_{x\in H}\;(ABx|x)=(\sqrt{A}\sqrt{A}Bx|x)=(B\sqrt{A}x|\sqrt{A})\ge0$と言ってもよい.
    \end{enumerate}
\end{Proof}

\begin{lemma}[半正定値作用素の自己共役性]
    半正定値作用素は,$\bF=\C$のとき自己共役であるが,$\bF=\R$のときそうとは限らない.
\end{lemma}
\begin{Proof}
    $\bF=\C$の場合は特徴付け\ref{cor-characterization-of-self-adjointness}より明らか.
    $\bF=\R$の場合は$A:\R^2\to\R^2$を$\varphi\in(-\pi/2,\pi/2)$回転とすると,$A^*=A^{-1}\ne A$となる.
\end{Proof}

\subsection{自己共役作用素の順序}

\begin{tcolorbox}[colframe=ForestGreen, colback=ForestGreen!10!white,breakable,colbacktitle=ForestGreen!40!white,coltitle=black,fonttitle=\bfseries\sffamily,
title=]
    幾何学的には,正作用素の全体$B(H)_+$は,自己共役作用素のなす閉・実・順序部分空間$B(H)_\sa$の内部で閉凸錐をなす.
    しかしこの順序は束にさえならず,より深い考察には二乗根補題を必要とする.
\end{tcolorbox}

\begin{proposition}[ordering by positive operator]\mbox{}
    \begin{enumerate}
        \item 自己共役な作用素は,$B(H)$の閉な実部分空間をなす.これを$B(H)_\sa$と表す.
        \item $B(H)_\sa$は,順序$S\le T:\Leftrightarrow T-S\ge 0$を備える.
    \end{enumerate}
\end{proposition}

\begin{proposition}[自己共役作用素の順序]\label{prop-order-of-self-adjoint-operator}
    $S,T\in B(H)_\sa$について,
    \begin{enumerate}
        \item $S\le T$ならば,$\forall_{A\in B(H)}\;A^*SA\le A^*TA$.
        \item $0\le S\le T$ならば,$\norm{S}\le\norm{T}$.特に,正作用素$S\ge$について,$S\le I$と$\norm{S}\le1$とは同値.
        \item $-I\le T\le I$と$\norm{T}\le 1$は同値.
    \end{enumerate}
\end{proposition}

\subsection{二乗根補題}

\begin{tcolorbox}[colframe=ForestGreen, colback=ForestGreen!10!white,breakable,colbacktitle=ForestGreen!40!white,coltitle=black,fonttitle=\bfseries\sffamily,
title=]
    正作用素は正規作用素の代表例だが,その正規たる所以を解明する.
    基本的にはスペクトル定理の帰結でもある.

    正作用素は凸錐をなすが,その中に二乗根が必ず見つかる.
    可逆性は,順序構造を用いて特徴付けることが出来る.
    また正作用素の逆は再び正である.
\end{tcolorbox}

\begin{lemma}
    正係数を持つ多項式の列$\{p_n\}\subset\R_{>0}[x]$であって,
    和$\sum_{n\in\N}p_n$が$[0,1]$上で関数$t\mapsto 1-(1-t)^{1/2}$に一様収束するものが存在する.
\end{lemma}

\begin{proposition}[square root lemma]\label{prop-square-root-lemma}
    半正定値な作用素$T\in B(H)$について,
    \begin{enumerate}
        \item ただ一つの半正定値な作用素$T^{1/2}\ge0$が存在して,$(T^{1/2})^2=T$を満たす.
        \item $A\in B(H)$が$T$と可換ならば,$T^{1/2}$と可換である.
    \end{enumerate}
\end{proposition}

\begin{corollary}
    $A\ge 0$かつ$(Ax,x)=0$ならば,$Ax=0$.
\end{corollary}
\begin{Proof}
    $A$の二乗根$X:=\sqrt{A}$を取ると,$(Ax|x)=(Xx|Xx)=0$.内積の非退化性より,$Xx=0$.よって,$Ax=0$.
\end{Proof}

\begin{proposition}[正作用素の可逆性]
    半正定値な作用素$T\in B(H)$について,
    \begin{enumerate}
        \item $T$が可逆であることと,$\exists_{\ep>0}\;T\ge\ep I$は同値.
        \item $T$が可逆であるとき,逆作用素も正$T^{-1}\ge 0$で,$T^{1/2}$も可逆で,$(T^{-1})^{1/2}=(T^{1/2})^{-1}=:T^{-1/2}$.
        \item 可逆な$T$に対して$T\le S$ならば$S$も可逆で,$S^{-1}\le T^{-1}$.
    \end{enumerate}
\end{proposition}
\begin{remarks}
    正規作用素の可逆性はbounded away from zeroのみであった\ref{prop-characterization-of-invertibleness-of-normal-operator}.
    正作用素については,これをさらに緩めることができる.
\end{remarks}

\section{作用素の分解理論}

\subsection{作用素のユニタリ同値}

\begin{definition}[unitary equivalent]\label{def-unitary-equivalent}
    2つの作用素$S,T\in B(H)$について$\exists_{U\in U(H)}\;S=UTU^*$を満たすとき,これらは\textbf{ユニタリー同値}であるという.
\end{definition}

\begin{lemma}
    ユニタリー同値は,作用素ノルム,自己共役性,正規性,対角化可能性,ユニタリー性を保つ.
\end{lemma}

\subsection{直交射影と対角化}

\begin{tcolorbox}[colframe=ForestGreen, colback=ForestGreen!10!white,breakable,colbacktitle=ForestGreen!40!white,coltitle=black,fonttitle=\bfseries\sffamily,
title=対角化とは,直交射影への標準分解をいう.]
    直交射影は,正作用素の代表例であり,$0\le P\le I$に位置する.
    また作用素の中でも最も基本的なもので,実解析における特性関数のような役割を持つ.
    単関数は直交射影の有限和に当たる.
    これへの分解が対角化の理論である.
\end{tcolorbox}
\begin{remarks}
    この定義は,射影$P$が自己共役であることと,$\Im P=(\Ker P^*)^\perp=(\Ker P)^\perp$が成り立つことが同値であるためである.
\end{remarks}

\begin{definition}[orthogonal projection / orthoprojection]
    一般に,冪等律を満たす線型作用素$P:H\to H,P^2=P$を射影という.
    射影が自己共役であるとき,\textbf{正射影}または\textbf{直交射影}であるという.\footnote{前者を冪等作用素,後者を射影と呼び分けることもある.この場合直交射影とは,直交分解が定める射影のことを指す.}
    射影全体の空間を$B(H)_p$で表す.射影を定める部分空間の束構造から,$B(H)_p$にも束の構造が入る.
\end{definition}
\begin{example}[直交射影の直交分解による特徴付け]
    自己共役性がなぜ「直交」なのかというと,自己共役な射影は全て直交分解が定めるからである.
    \begin{enumerate}
        \item 閉部分空間$X\le H$について,直交分解$H=X+X^\perp$が導く作用素$P:H\to H;y\mapsto x$は$\norm{P}\le 1$かつ$P^2=P$を満たす半正定値作用素である.よってこれは\textbf{直交射影}である.
        \item 逆に,任意の自己共役な冪等作用素$P:H\to H$に対して,$X:=\Im P$とおくとこれは閉部分空間で,任意の$x^\perp\in X^\perp$に対して$\norm{Px^\perp}^2=(x^\perp|P^2x\perp)=0$より,$P$は直交射影である.$I-P$は$\Im(I-P)=X^\perp$を満たす直交射影である.
    \end{enumerate}
    以上の観察より,直交射影$P$について,$\Im P\perp\Ker P$すなわち$\Im P\perp\Im(I-P)$が成り立つ.
\end{example}
\begin{remarks}[diagonalizable, eigenfunction expansion]\label{remarks-diagonalizability}
    行列の対角化とJordan標準形の理論を,作用素論の言葉を用いて捉え直す.
    \begin{enumerate}
        \item 射影は実解析における特性関数にあたる.
        \item 単関数は$T=\sum_{i\in[n]}\lambda_iP_i$にあたる.
        \item 一般に,作用素$T$が\textbf{対角化可能}であるとは,ある正規直交基底$(e_j)_{j\in J}$と有界集合$\{\lambda_j\}_{j\in J}\subset\bF$が存在して,$\forall_{x\in H}\;Tx=\sum_{j\in J}\lambda_j(x|e_j)e_j=\sum_{j\in J}\lambda_jP_j\;(P_j:H\epi\bF e_j)$と表せることをいう.このとき,$(e_j)$を固有ベクトル,$(\lambda_j)$を固有値という.$(x|e_j)$は$x$の$j$-座標である.
        
        なお,無限和$T=\sum_{j\in J}\lambda_jP_j$は作用素の強位相については収束する(各点収束)が,そのほかは保証されない.
        \item $A\in M_n(\C)$を自己共役(Hermite)行列とする.すると,固有値$\lambda_1,\cdots,\lambda_m\in\C$の属する固有空間$W_1,\cdots,W_m$は互いに直交するのであった(ユニタリ行列によって対角化可能なので).
        そこで,各$W_i$への直交射影を$P_i:\C^n\epi W_i$で表すと,$A=\sum^m_{i=1}\lambda_iP_i$が成り立つ.これを作用素論的な「対角化可能性」の定義とすれば良い.
        \item 一般の行列$A\in M_n(\C)$について,固有多項式$p(\lambda)=\det(\lambda I-A)$の根を$\{\lambda_1,\cdots,\lambda_m\}$とし,最小多項式$q(\lambda)$におけるそれぞれの重複度を$k_i\in\N$とする.
        広義固有空間を$\wt{W}_i:=\Brace{u\in\C^n\mid(A-\lambda_i I)^{k_i}u=0}$と定めても,これらは$\C^n$の直和分解ではあるが互いに直交するとは限らない.
        このとき,$\wt{W}_i$への直交とは限らない射影を$\wt{P}_i$とすると,$\wt{W}_i$上の冪零作用素$N_i$を用いて,$A=\sum_{i=1}^m(\lambda_i\wt{P}_i+N_i\wt{P}_i)$と表せる.いわば,冪零・冪等分解である.
    \end{enumerate}
    (4)の式を,$A$の\textbf{固有関数展開}または\textbf{スペクトル展開}といい,(5)の式を\textbf{Jordan分解}という.
\end{remarks}

\begin{lemma}[一般の射影の性質]\label{lemma-property-of-projection}
    $P^2=P\in B(E)$を満たすとする.$E_1:=\Im P,E_2:=\Ker P$とする.
    \begin{enumerate}
        \item $P|_{E_1}=\id_{E_1}$.
        \item $Q:=I-P$は$\Im P=\Ker Q,\Ker P=\Im Q$を満たす射影である.
        \item $E=E_1\oplus E_2$.
    \end{enumerate}
\end{lemma}

\begin{lemma}[直交射影の特徴付け]
    $E$を射影とし,$E\ne 0$とする.次の6条件は同値.
    \begin{enumerate}
        \item $\Ker E=(\Im E)^\perp$.
        \item $E$はある閉部分空間$M$についての直交射影である.
        \item $\norm{E}=1$.
        \item $E$は自己共役である.
        \item $E$は正規である.
        \item $E$は正である.
    \end{enumerate}
\end{lemma}

\subsection{対角化可能作用素と正規作用素}

\begin{lemma}[対角化可能作用素の性質]
    $T$が対角化可能であるとする:$T=\sum_{j\in J}\lambda_jP_j$.
    \begin{enumerate}
        \item $T^*$も対角化可能で,固有ベクトルは$(e_j)$,固有値は$(\o{\lambda_j})$である.
        \item $T$は正規:$TT^*=T^*T$.
        \item $T$が自己共役であることと,固有値が全て実数であることは同値.
        \item $T$が半正定値であることと,固有値が全て非負実数であることは同値.
    \end{enumerate}
\end{lemma}
\begin{Proof}\mbox{}
    \begin{enumerate}
        \item $T^*x=\sum\o{\lambda_j}(x|e_j)e_j$より.
    \end{enumerate}
\end{Proof}

\begin{proposition}[有限次元線型空間論]
    $H$が有限次元であるとき,任意の正規な作用素は対角化可能である.
    また,互いに可換な正規な作用素は,同時対角化可能である.
\end{proposition}
\begin{Proof}
    有限次元線型空間論による.
\end{Proof}

\begin{lemma}
    任意の正規作用素$T\in B(H)$と$\ep>0$について,組ごとに直交な射影の有限集合$\{P_n\}$が存在して和が$I$になるものと$\{\lambda_n\}\subset\C$とが存在して,$\Norm{T-\sum\lambda_nP_n}\le\ep$を満たす.
\end{lemma}

\begin{theorem}
    $B(H)$内の対角化可能な作用素の集合$B_\diag(H)$のノルム閉包は正規作用素の集合である.
\end{theorem}

\subsection{部分等長作用素}

\begin{tcolorbox}[colframe=ForestGreen, colback=ForestGreen!10!white,breakable,colbacktitle=ForestGreen!40!white,coltitle=black,fonttitle=\bfseries\sffamily,
title=]
    自己共役作用素は実数,正作用素は正実数と見れるのであった.
    $C^*$-代数の元の標準形がほしい.
    $\forall_{\lambda\in\C}\;\lambda=\abs{\lambda}e^{i\theta}$に対応する作用素の概念は,$\abs{A}:=(A^*A)^{1/2}$と表せるが,$e^{i\theta}$に対応する作用素のクラスは,新たに用意する必要がある.
\end{tcolorbox}

\begin{definition}[partial isometry, initial subspace, final subspace]
    作用素$U\in B(H)$が\textbf{部分等長作用素}であるとは,閉部分集合$X=(\Ker U)^\perp=\oo{\Im U^*}$が存在して,制限$U|_X$は等長写像で,$U|_{X^\perp}=0$を満たすことをいう.
    $X$を\textbf{初期部分空間},$\Im U$を\textbf{最終部分空間}という.
\end{definition}
\begin{remarks}
    $\Ker U=X^\perp$は閉であるから,$X=(\Ker U)^\perp=\oo{\Im U^*}$は必然的に閉である.
\end{remarks}
\begin{lemma}[部分等長写像の特徴付け]\mbox{}
    \begin{enumerate}
        \item $P:=U^*U$とおくと,$\forall_{x\in X}\;Px=x,\;\forall_{x^\perp\in X^\perp}\;Px^\perp=0$.すなわち,$P$は直交射影である:$P^2=P\land P^*=P$.
        \item (1)は部分等長作用素を特徴付ける.すなわち,作用素$U\in B(H)$について$U^*U$が直交射影となるとき,$U,U^*$は部分等長作用素である.$U^*$を$U$の\textbf{部分逆}(partial inverse)という.
    \end{enumerate}
    結局,次の6条件は同値である.
    \begin{enumerate}
        \item $U$は部分等長作用素である.
        \item $U^*$は部分等長作用素である.
        \item $U^*U$は直交射影である.
        \item $UU^*$は直交射影である.
        \item $UU^*U=U$.
        \item $U^*UU^*=U^*$.
    \end{enumerate}
\end{lemma}
\begin{Proof}\mbox{}
    \begin{enumerate}
        \item $x^\perp\in X^\perp=\Ker U$に対して,$U^*Ux=U^*0=0$は明らか.
        等長写像の特徴付け\ref{lemma-characterization-of-unitary-operator}より,$X$上で$U^*U=I$である.
        \item $X:=\Im U^*U$とおくと,射影はこの上で恒等だから,$\forall_{x\in X}\;\norm{Ux}^2=(Ux|Ux)=(U^*Ux|x)=(x|x)=\norm{x}^2$より,$X$上で等長.$X^\perp=(\Im U^*U)=\Ker U^*U$より,$\forall_{x\in X^\perp}\;(Ux|Ux)=(U^*Ux|x)=(0|x)=0$.
        
        一般に,$U$が部分等長写像であるとき,$U^*$も部分等長写像である.
        $U(I-P)=0$である.よって,$(UU^*)^2=UU^*UU^*=UPU^*=UU^*$から,$UU^*$も直交射影である.
        よって,前述の議論を繰り返せば良い.
    \end{enumerate}
\end{Proof}
\begin{remarks}
    部分等長写像は,$0$と等長写像とに分解できる作用素をいう.
    等長写像はepi:$U^*U=I$として特徴付けられるから,射影の言葉で部分等長写像が特徴付けられることは自然である.
\end{remarks}

\begin{example}[unilateral shift operator]
    可分Hilbert空間$H$上のシフト作用素\ref{operator-unilateral-shift}
    \[S\paren{\sum\al_ne_n}=\sum\al_ne_{n+1}\]
    は等長作用素だが,ユニタリーではない(すなわち全射でない)例となっている.
    また$S,S^*$は部分等長作用素である.
    $S:H\to\{e_1\}^\perp$は等長作用素であり,$S^*S=I$(たしかに射影である).
    一方で,$SS^*=P_{\Brace{e_1}^\perp}$であり,$S^*$は$\{e_1\}^\perp$上では$H$に値を取る部分等長写像である.
\end{example}

\subsection{極分解}

\begin{theorem}[polar decomposition (von Neumann)]\mbox{}\label{thm-polar-decomposition}
    \begin{enumerate}
        \item 任意の作用素$T\in B(H)$に対して,ただ一つの半正定値作用素$\abs{T}:=(T^*T)^{1/2}\in B(H)$が存在して,$\forall_{x\in H}\;\norm{Tx}=\norm{\abs{T}x}$を満たす.特に,$\Ker T=\Ker\abs{T}$.
        \item ただ一つの部分等長写像$U$が存在して,$\Ker U=\Ker T,U\abs{T}=T$を満たす.
        \item $U^*U\abs{T}=U^*T=\abs{T},\;UU^*T=T$が成り立つ.
    \end{enumerate}
\end{theorem}
\begin{Proof}\mbox{}
    \begin{enumerate}
        \item 一意性を示せば良い.
        ある正作用素$S\ge0$が$\forall_{x\in H}\;\norm{Tx}=\norm{Sxx}$を満たすならば,$(S^2x|x)=\norm{Sx}^2=\norm{Tx}^2=(T^*Tx|x)$である.これより,$T^*T$も正だから,$S^2=T^*T$が従う\ref{cor-nondegeneratedness-of-semi-inner-product}.
        二乗根の一意性より,$S=(T^*T)^{1/2}$である.
        \item $T,\abs{T}$は任意の$x\in H$に対して像のノルムが等しいから,核も等しい.よって,$\Ker T=\Ker\abs{T}=(\Im\abs{T})^\perp$\ref{prop-Ker-of-adjoint-operator}.
        \begin{description}
            \item[存在] まず,任意の$y=\abs{T}x\in\Im\abs{T}$に対して,$U_0y=U_0\abs{T}x:=Tx$と定めると,これは等長作用素$U_0:\Im\abs{T}\to\Im T$である:$\norm{U_0y}=\norm{Tx}=\norm{\abs{T}x}=\norm{y}$.
            連続延長\ref{prop-extension-of-operator-on-dense-subset}により,等長な延長$\o{U_0}:\oo{\Im\abs{T}}\to\oo{\Im T}$が得られる.
            さらに,$\Ker T=(\Im\abs{T})^\perp$上では零とすることで,$U\abs{T}=T$を満たす部分等長写像$U$を得る.これはたしかに$\Ker U=\Ker T$を満たす.
            \item[一意性] 
            $V$を,$\Ker V=\Ker T,V\abs{T}=T$を満たす部分等長写像とする.
            すると,まず$U=V\on\;\oo{\Im\abs{T}}$が必要で,その直交補空間ではいずれも$0$であるから,$U=V$が必要.
        \end{description}
        \item $U^*U$は構成から$\oo{\Im\abs{T}}$への直交射影だから,$U^*U\abs{T}=\abs{T}$.
        $T=U\abs{T}$より,$U^*T=\abs{T}$.
        この両辺に$U$を乗じると$UU^*T=U\abs{T}=T$.
    \end{enumerate}
\end{Proof}

\begin{corollary}[随伴作用素の極分解]
    $T=U\abs{T}$を極分解とすると,$T^*=\abs{T}U^*=U^*(U\abs{T}U^*)$より,符号は$U^*$で,絶対値は$U\abs{T}U^*$である.
\end{corollary}
\begin{remark}
    和,積の極分解には殆ど法則がない.
\end{remark}

\begin{corollary}[Schmidt展開]
    $H$を可分,$T\in B_0(H)$とする.
    \begin{enumerate}
        \item 半正定値作用素を$0\le A\in B_0(H)$と,等長作用を$U:\o{R(T)}\to H$と取れて,$T=UA$.
        \item $H$の正規直交系$\{\phi_n\},\{\psi_n\}$と正数$s_n\to0$が存在して,
        \[Tu=\sum_{n\in\N}s_n(u|\psi_n)\phi_n\]
        と書ける.$\{s_n\}$を$T$の\textbf{特異数}という.
    \end{enumerate}
\end{corollary}

\subsection{ユニタリな偏角成分を持つとき}

\begin{theorem}
    $T\in B(H)$について,次の2条件は同値.
    \begin{enumerate}
        \item あるユニタリ作用素$U$について,$T=U\abs{T}$と極分解される.
        \item $T,T^*$の核が同次元である:$\Ker T\simeq_\Hilb\Ker T^*$.
    \end{enumerate}
\end{theorem}

\begin{corollary}\label{cor-polar-decomposition-of-invertible-operators}
    $T\in B(H)$が可逆であるとき,極分解に含まれる部分等長写像$U$はユニタリである.
\end{corollary}
\begin{Proof}
    ここでは定理に依らず,単独で証明する.

    $T=U\abs{T}$を極分解とすると,$\Ker U=\Ker T$で,$\Im U=\oo{\Im T}$.
    $T$が可逆のとき$\Ker U=0,\Im U=H$が従うから,部分等長写像$U$はユニタリである.
\end{Proof}

\begin{corollary}
    $T\in B(H)$を正規とする.
    このとき,$T,T^*,\abs{T}$と可換なユニタリ作用素$W$が存在して,$T=W\abs{T}$と極分解できる.
\end{corollary}

\subsection{ユニタリ作用素への分解}

\begin{tcolorbox}[colframe=ForestGreen, colback=ForestGreen!10!white,breakable,colbacktitle=ForestGreen!40!white,coltitle=black,fonttitle=\bfseries\sffamily,
title=ユニタリ作用素への分解]
    極分解をしたなら,$\C\simeq\R^2$に対応する分解も欲しくなる.これは$B\cap B(H)_\sa$上のみで存在する.
    結局,$\norm{S}\le 1$を満たす有界作用素は,ユニタリ作用素の凸結合(特に平均)として表せる.
    さらに,$B(H)_\sa$は$B(H)$を生成するから,任意の作用素はユニタリ作用素の線型結合である.
\end{tcolorbox}

\begin{lemma}[単位球内の自己共役作用素の表示]
    $T=T^*$かつ$\norm{T}\le 1$ならば,作用素$U:=T+i(I-T^2)^{1/2}$はユニタリであり,$T=\frac{1}{2}(U+U^*)$と表せる.
\end{lemma}
\begin{Proof}
    $I-T\ge0$かつ$I+T\ge0$で互いに可換であるから,$I-T^2\ge0$である\ref{lemma-positive-closed-cone-of-positive-operator}(2),\ref{prop-order-of-self-adjoint-operator}(3).
    よってたしかに二乗根を持つ.
    $UU^*=U^*U=I$がわかるから,$U$はユニタリである.
    正作用素は自己共役であることに注意すれば,$T=\frac{1}{2}(U+U^*)$は明らか.
\end{Proof}

\begin{lemma}
    $S\in B(H)$かつ$\norm{S}<1$ならば,任意のユニタリ作用素$U$に対して,ユニタリ作用素$U_1,V_1$が存在して,$S+U=U_1+V_1$を満たす.
\end{lemma}
\begin{Proof}
    $U=I$の場合について示せば,両辺に$U^*$を左から乗ずることで一般の結論を得る.

    $\norm{S}=\norm{S^*}<1$の仮定より,三角不等式$\norm{x}-\norm{y}\le\norm{x-y}$から,
    \[\norm{Sx+x}\ge\norm{x}-\norm{Sx}\ge(1-\norm{S})\norm{x}\]
    より,$I+S,I+S^*$はいずれも$0$に対して有界である.よって,作用素の可逆性の特徴付け\ref{prop-characterization-of-invertibleness-of-operator}より,
    $I+S$は可逆である.
    よって,極分解$I+S=V\abs{I+S}$において,$V$はユニタリである\ref{cor-polar-decomposition-of-invertible-operators}.
    $\norm{I+S}\le\norm{I}+\norm{S}<2$より,補題から,正作用素$\abs{I+S}$はあるユニタリ作用素$W$を用いて$\abs{I+S}=W+W^*$と表せる.
    以上より,
    \[S+I=V(W+W^*)=VW+VW^*.\]
\end{Proof}

\begin{proposition}[Russo-Dye-Gardner theorem (66)]
    $T\in B(H)$かつ$\exists_{n>2}\norm{T}<1-\frac{2}{n}$ならば,ユニタリ作用素$U_1,\cdots,U_n$が存在して,
    \[T=\frac{1}{n}(U_1+U_2+\cdots+U_n).\]
\end{proposition}
\begin{Proof}
    $S:=\frac{1}{n-1}(nT-I)$とおくと,$nT=(n-1)S+IT$で,
    \[\norm{S}\le\frac{1}{n-1}(n\norm{T}-1)<\frac{1}{n-1}(n-3)<1\]
    を満たすから,補題を$n-1$回繰り返し適用すると,
    \begin{align*}
        nT&=(n-1)S+I=(n-2)S+(S+I)=(n-2)S+(V_1+U_1)\\
        &=(n-3)S+(S+V_1)+U_1=(n-3)S+(V_2+U_2)+U_1\\
        &=(n-4)S+(S+V_2)+U_2+U_1\\
        &=(n-4)S+(V_3+U_3)+U_2+U_1=\cdots\\
        &=(S+V_{n-2})+U_{n-2}+\cdots+U_1=U_n+U_{n-1}+U_{n-2}+\cdots+U_1.
    \end{align*}
\end{Proof}
\begin{remarks}
    $B(H)$の開球の元は,ユニタリ作用素の凸結合(特に平均)で表せる.
    球面上の元はこの限りではなく,シフト作用素が反例である.
\end{remarks}

\begin{corollary}[同じ論文で紹介されている系]
    $C^*$-代数$A$のユニタリ元の全体を$U(A)$で表す.ノルム空間$B$への線型写像$f:A\to B$について,$\sup_{U\in U(A)}\norm{f(U)}$.
\end{corollary}
\begin{remarks}
    the norm of an operator can be calculated using only the unitary elements of the algebra. \footnote{\url{https://en.wikipedia.org/wiki/Russo\%E2\%80\%93Dye_theorem}}
\end{remarks}

\subsection{数域とスペクトル}

\begin{tcolorbox}[colframe=ForestGreen, colback=ForestGreen!10!white,breakable,colbacktitle=ForestGreen!40!white,coltitle=black,fonttitle=\bfseries\sffamily,
title=]
    自己共役作用素$T$に対して,$R(T,x):=\frac{(Tx|x)}{(x|x)}$をRayleigh商といい\ref{prop-operator-norm-of-self-adjoint-operator},固有値の数値計算に利用される.\footnote{\url{https://ja.wikipedia.org/wiki/レイリー商}}
    そしてこの値域を数域という.
    $T$がエルミート行列$M$のとき数域は実数に含まれ,$R(M,x)\in[\lambda_\mathrm{min},\lambda_\mathrm{max}]$となる.
\end{tcolorbox}

\begin{proposition}[Hausdorff-Toeplitz]
    $T\in B(H)$について,数域$W(T)$は凸集合である.$H$が有限次元のとき,$W(T)$はコンパクトである.
\end{proposition}

\begin{proposition}
    $T\in B(H)$を正規作用素とする.数域の閉包$\oo{W(T)}$とスペクトルの閉凸包$\oo{\Conv(\sigma(T))}$とは等しい.また,$\oo{W(T)}$の極点$x$について,次の2条件は同値.
    \begin{enumerate}
        \item $x\in W(T)$.
        \item $x$は$T$の固有値である.
    \end{enumerate}
\end{proposition}
\begin{remark}
    工学では,この定理を利用して,$T$の固有値を$W(T)$の形から推定する.\footnote{\url{https://en.wikipedia.org/wiki/Numerical_range}}
\end{remark}

\subsection{作用素の例}

\begin{tcolorbox}[colframe=ForestGreen, colback=ForestGreen!10!white,breakable,colbacktitle=ForestGreen!40!white,coltitle=black,fonttitle=\bfseries\sffamily,
title=]
    乗算作用素は対角行列の概念を一般化する.任意のHilbert空間上の自己共役作用素は,$L^2$空間上のある乗算作用素とユニタリ同値である,という主張がスペクトル定理である.
    核が超関数になることも許せば,全ての線型作用素は積分作用素として表せる,という主張がSchwartzの核定理である.
    そのほか,合成作用素と転送作用素の随伴,シフト作用素=平行移動作用素(時系列解析ではラグ作用素)などもある.

    また,作用素の随伴は,「転置」の要素は隠れて,複素共役を取ることに似る.
\end{tcolorbox}

\begin{theorem}[multiplication operator and its symbol]\label{operator-multiplication}
    $(X,\Om,\mu)$を$\sigma$-有限な測度空間とし,$H:=L^2(X,\Om,\mu)=:L^2(\mu)$とする.任意の$\varphi\in L^\infty(\mu)$に対して,これと積を取る写像
    \[\xymatrix@R-2pc{
        M_\varphi:L^2(\mu)\ar[r]&L^2(\mu)\\
        \rotatebox[origin=c]{90}{$\in$}&\rotatebox[origin=c]{90}{$\in$}\\
        f\ar@{|->}[r]&\varphi\cdot f
    }\]
    は,有界で$M_\varphi\in B(L^2(\mu))$,等長である:$\norm{M_\varphi}=\norm{\varphi}_\infty$.
    $\varphi$を\textbf{乗算作用素}$M_\varphi$の\textbf{記号}という.

    乗算作用素は対角行列の一般化であり,スペクトル定理によると,Hilbert空間上の自己共役作用素は,$L^2$-空間上の乗算作用素とユニタリ同値になる.
\end{theorem}

\begin{theorem}[Fredholm integral operator / transform, kernel]\label{operator-integral-transformation}
    $(X,\Om,\mu)$を測度空間,$k:X\times X\to\bF$を次を満たす$\Om\times\Om$-可測関数とする:
    \begin{align*}
        \int_X\abs{k(x,y)}d\mu(y)\le c_1,&\ae[\mu],&\int_X\abs{k(x,y)}d\mu(x)\le c_2,&\ae[\mu].
    \end{align*}
    このとき,写像
    \[\xymatrix@R-2pc{
        K:L^2(\mu)\ar[r]&L^2(\mu)\\
        \rotatebox[origin=c]{90}{$\in$}&\rotatebox[origin=c]{90}{$\in$}\\
        f\ar@{|->}[r]&Kf(x):=\int_Xk(x,y)f(y)d\mu(y)
    }\]
    は有界線型作用素で,ノルムは$\norm{K}\le(c_1c_2)^{1/2}$を満たす.
\end{theorem}

\begin{example}[Volterra integral operator, Volterra operator]\label{operator-Volterra}
    Voltera積分作用素は不定積分に関する積分変換全般を指すが,Volterra作用素と言った場合は,冪零作用素の一般化概念をいう.
    前者は後者の例となっている.
    \begin{enumerate}
        \item Fredholm積分作用素$K$が定積分を定める積分作用素ならば,Volterra積分作用素は不定積分を定める積分作用素である.\footnote{ヴォルテラ積分方程式は、人口学や、粘弾性物質の研究、保険数学に現れる再生方程式などへと応用されている。}
        \item $k:[0,1]\times[0,1]\to2\mono\R$を集合$\{(x,y)\in[0,1]\times[0,1]\mid y<x\}$の特性関数とする.
        これを核とする積分作用素$V:L^2(0,1)\to L^2(0,1);f\mapsto Vf(x)=\int^1_0k(x,y)f(y)dy=\int^x_0f(y)dy$を\textbf{Volterra作用素}という.
        別の抽象的な定義としては,スペクトル半径が$0$なコンパクト作用素をVoltarra作用素と定める.
    \end{enumerate}
    Volterra積分作用素はコンパクトであるが,固有値を持たず,擬冪零$\Sp(V)=\{0\}$である.
\end{example}

\begin{example}[Hilbert-Schmidt integral operator, Hilbert-Schmidt operator]\label{exp-Hilbert-Schmidt}
    積分核が2乗可積分である$k\in L^2(\Om\times\Om;\C)$とき,この積分作用素を\textbf{Hilbert-Schmidt積分作用素}といい,コンパクト作用素になる\ref{def-trace-class-Hilbert-Schmidt}.
    2乗可積分な積分核を\textbf{Hilbert-Schmidt核}という.
    $k$が自己共役ならば$K$も自己共役で,従ってスペクトル定理が適用できる.
    なお,これはHilbert-Schmidt作用素\ref{def-trace-class-Hilbert-Schmidt}の例である.
\end{example}

\begin{remark}
    \begin{align*}
        \int^b_aK(x,y)f(y)dy-\lambda f(x)&=g(x)\\
        \int^x_aK(x,y)f(y)dy-\lambda f(x)&=g(x)
    \end{align*}
    をそれぞれ,Fredholm型,Volterra型積分方程式という.
\end{remark}

\begin{example}[matrix multiplication]
    行列乗算は,離散空間上での積分変換と捉えられる.
    これが行列という形式の普遍性を説明しているのではなかろうか?
    線形代数と微分積分の概念はここに交錯する.
    積分変換はより一般に多項式関手(polynomial functor)の特別な場合で,多項式関手とは,多項式概念の関手化である.
    Volterra作用素は,不定積分概念の作用素化であろうか.
\end{example}

\begin{example}[unilateral shift]\label{operator-unilateral-shift}
    \[\xymatrix@R-2pc{
        S:l^2\ar[r]&l^2\\
        \rotatebox[origin=c]{90}{$\in$}&\rotatebox[origin=c]{90}{$\in$}\\
        (\al_1,\al_2,\cdots)\ar@{|->}[r]&(0,\al_1,\al_2,\cdots)
    }\]
    は全射でない等長作用素となる(すなわち,ノルム1の有界線型写像).
    これをシフト作用素という.
\end{example}


\section{コンパクト作用素}

\begin{tcolorbox}[colframe=ForestGreen, colback=ForestGreen!10!white,breakable,colbacktitle=ForestGreen!40!white,coltitle=black,fonttitle=\bfseries\sffamily,
title=]
    ($X$が局所コンパクトハウスドルフ空間である時,)
    有界連続関数の中でコンパクト台を持つもの$C_c(X)\subset C_b(X)$の関係と,有界作用素の中で有限ランクを持つもの$B_f(H)\subset B(H)$の関係は非常に似ている.
    These classes describe local phenomena on $H$ and on $X$.\cite{AnalysisNow}
    そこで,$C_c(X)$の完備化として得た$C_0(X)$に対応するクラス"$B_0(H)$"を,作用素論でも構成することを考える.
    このクラスはある種の有限的な性質を持ち,$B(H)$の閉イデアルをなす.

    このクラスには,行列はもちろん,積分作用素が含まれる.
    そこで,この順に,対角化の一般理論と,積分方程式の一般理論を調べていく.
\end{tcolorbox}

\subsection{定義と特徴付け}

\begin{tcolorbox}[colframe=ForestGreen, colback=ForestGreen!10!white,breakable,colbacktitle=ForestGreen!40!white,coltitle=black,fonttitle=\bfseries\sffamily,
title=]
    積分作用素も例に含むが,より行列の類似的性質が見やすい狭いクラスの作用素を定義する.
    作用素に「有界集合は相対コンパクト集合に写す」という有限性条件を課す.
    Hilbertは「完全連続作用素」と呼んだ.
\end{tcolorbox}

\begin{definition}[finite rank]
    Hilbert空間$H$上の作用素$T:H\to H$について,
    \begin{enumerate}
        \item $T$が\textbf{有限階数}または\textbf{退化}であるとは,$\Im T$が$H$の有限次元部分空間であることをいう(したがって特に閉\ref{prop-finite-subspaces}).
        \item 有限階数な有界作用素全体$B_f(H)$は,$B(H)$内の部分代数であり,かつイデアルである.
        \item $B_f(H)$はイデアルとして自己共役である:$(B_f(H))^*=B_f(H)$.
    \end{enumerate}
\end{definition}
\begin{Proof}\mbox{}
    \begin{enumerate}\setcounter{enumi}{1}
        \item 任意の$S,T\in B_f(H),U\in B(H),a\in\bF$について,$aS,S+T,ST\in B_f(H)$である:$\Im aS=\Im S,\Im(S+T)\subset\Im S\oplus\Im T,\Im(SU)\subset\Im S,\dim\Im(US)\le\dim\Im(S)$.
        \item $T\in B_f(H)\Leftrightarrow T^*\in B_f(H)$を示す.直交分解$H=\Im T\oplus\Im T^\perp$を随伴によって表現することより,$H=\Im T\oplus\Ker T^*$の関係がある.よって,$\Im T^*=T^*(\Im T)$より,$T^*$も有限ランクである.
    \end{enumerate}
\end{Proof}

\begin{lemma}[近似的単位元の構成]
    $B_f(H)$には射影からなるネット$(P_\lambda)_{\lambda\in\Lambda}$が存在し,$\forall_{x\in H}\;\norm{P_\lambda x-x}\to 0$を満たす.
\end{lemma}
\begin{Proof}
    $H$の正規直交基底${(e_j)}_{j\in J}$を取り,$\Lambda:=\Brace{\lambda\in P(J)\mid \abs{\lambda}<\infty}$からのネット$(P_\lambda:=\pr\{\brac{e_j}_{j\in J})$を考える.
    すると,$\forall_{\lambda\in\Lambda}\;\dim(\Im(P_\lambda))<\infty$より確かに$B_f(H)$上のネットなっている.
    任意の$x=\sum_{j\in J}\al_je_j\in H$について,
    系\ref{cor-well-definedness-of-Bessel's-identity}より,任意の$0\in[0,\infty)$の開近傍の基本系の元$[0,\ep)$に対して,ある有限集合$J$が存在して,$\norm{P_\lambda x-x}^2=\sum_{j\in\Lmd}\abs{\al_j}^2<\ep$(Parseval's identity)が成り立つ.
    よって,ネットとして,$\norm{P_\lmd x-x}$は$0$に収束する.
\end{Proof}

\begin{theorem}[stability of compact operator]
    $T\in B(H)$について,次の5条件は同値.
    \begin{enumerate}
        \item $T\in\dbloverline{B_f(H)}$.
        \item $T|_B:B\to H$は弱-ノルム連続な関数である.\footnote{すなわち,列$(f_n)$が$f$に弱収束するならば,像$(Kf_n)$は$Kf$にノルム収束する.}
        \item $T(B)$は$H$でコンパクトである.
        \item $\dbloverline{T(B)}$は$H$でコンパクトである.
        \item $B$上の任意のネットは,$T$での像が$H$上で強収束するような部分ネットを持つ.\footnote{$H$は距離空間だから,すなわち,$H$の任意の有界列$(f_n)$に対して,$(Kf_n)$は収束する部分列を持つ.}
        \item (Hilbert) 双線型形式$\Phi(f,g):=(Tf,g)$は弱連続である.\footnote{すなわち,弱収束列を保存する.}
    \end{enumerate}
\end{theorem}
\begin{Proof}\mbox{}
    \begin{description}
        \item[(1)$\Rightarrow$(2)] 
        $x$に弱収束する$B$のネット$(x_\lambda)_{\lambda\in\Lambda}$を任意に取る.任意の$\ep>0$に対して,仮定より,$S\in B_f(H)$が存在して$\norm{S-T}<\ep/3$を満たすから,
        \begin{align*}
            \norm{Tx_\lambda-Tx}&=\norm{(T-S)x_\lambda-(T-S)x+Sx_\lambda-Sx}\\
            &\le2\norm{T-S}+\norm{Sx_\lambda-Sx}\\
            &\le\frac{2}{3}\ep+\norm{Sx_\lambda-Sx}.
        \end{align*}
        いま,任意の$B(H)$の元は弱-弱連続\ref{lemma-characterization-of-bounded-operator-on-Hilbert-space}だから,$Sx_\lambda$は$Sx$に弱収束する.
        $\Im S$は有限次元であるから,この正規直交基底を$\{e_1,\cdots,e_n\}$とおくと,各線型汎関数$(-|e_j)$は弱連続だから,
        ノルムと内積の関係性
        \[\norm{Sx_\lambda-Sx}^2=\sum_{j\in[n]}\abs{(S(x_\lambda-x)|e_j)}^2\to0\]
        より,ノルムについても収束する.
        以上より,$\norm{Tx_\lambda-Tx}<\ep$であるから,$T$は弱-ノルム連続である.
        \item[(2)$\Rightarrow$(3)]
        任意のHilbert空間は回帰的で,$H$の単位球$B$は弱コンパクト\ref{def-weak-topology-on-H(B)}であったから,$T(B)$はノルムコンパクトである.
        \item[(3)$\Rightarrow$(4)]
        一般に,任意のHausdorff空間のコンパクト集合は閉であるから,$T(B)=\oo{T(B)}$.
        \item[(4)$\Rightarrow$(5)]
        $T(B)$は相対コンパクトであるから,任意の$T(B)$のネットは収束する部分ネットを持つ.したがって,その像も収束する.
        \item[(5)$\Rightarrow$(1)]
        補題の通りの射影からなるネット$(P_\lambda)_{\lambda\in\Lambda}$を取る.
        これについて,$(P_\lambda T)$は$B_f(H)$のネットになるが,これが$T$にノルム収束することを示す.
        仮に収束しないと仮定すると,$\ep>0$が存在して,任意の$\lambda\in\Lambda$に対して単位ベクトル$x_\lambda$が存在して,$\norm{(P_\lambda T-T)x_\lambda}\ge\ep$.
        仮定より,ネット$(Tx_\lambda)$はある極限$y$にノルム収束すると仮定してよく,このとき補題より,
        \begin{align*}
            \ep&\le\norm{(I-P_\lambda)Tx_\lambda}\le\norm{(I-P_\lambda)(Tx_\lambda-y)}+\norm{(I-P_\lambda)y}\\
            &\le\norm{Tx_\lambda-y}+\norm{(I-P_\lambda)y}\to0.
        \end{align*}
        よって矛盾.
    \end{description}
\end{Proof}
\begin{remarks}
    $B_f(H)$の元の像は有限次元であるから,そこではノルム位相と弱位相が一致する.
    これが(1)の消息となる.
\end{remarks}

\begin{definition}[compact operator]
    定理の同値な条件を満たす作用素を\textbf{コンパクト作用素}という.
    無限遠で消えるため,コンパクト作用素の空間は$B_0(H)$と表すが,$K(H),C(H)$も一般的である.
\end{definition}

\begin{example}[Hilbert-Schmidt作用素は非自明な例]
    可分な$H$の正規直交基底$(e_n),(f_n)$を用いて,
    \[\norm{K}_{H.S.}^2:=\sum_{n,m\ge1}\abs{(Ke_n|f_m)}^2=\sum_{n,m\ge1}\abs{(K^*f_m,e_n)}^2\]
    と定めると,Parsevalの等式より,$\norm{K}^2_{H.S.}=\sum^\infty_{n=1}\norm{Ke_n}^2=\sum^\infty_{n=1}\norm{K^*f_m}^2$.
    このノルムが有限な作用素をHilbert-Schmidt作用素といい,これはコンパクト作用素である.
    実際,$K_nf:=\sum^n_{m=1}(Kf,e_m)e_m$とすると明らかに
    $B_0(H)$の列であるが,$K$はこのノルム極限として得る.
\end{example}

\subsection{コンパクト作用素の空間}

\begin{lemma}[コンパクト作用素の空間の描像]
    コンパクト作用素の空間$B_0(H)=\oo{B_f(H)}\subset B(H)$は
    \begin{enumerate}
        \item ノルム閉で自己共役なイデアルである.
        \item (1)の条件を満たすもので最小のものである.特に$H$が可分である場合は唯一の非自明な閉イデアルである.
        \item $H$が無限次元であるとき,非単位的な部分代数である:$I\notin B_0(H)$.が,その場合でも,有限ランクの射影からなる近似的単位元を持つ.
        \item $H$が可分であるとき,$B_f(H)$は$B_0(H)$の中で稠密である.\footnote{これは$C_c(X)$と$C_0(X)$のアナロジーが完成する.}
    \end{enumerate}
\end{lemma}
\begin{Proof}\mbox{}
    \begin{enumerate}
        \item 自己共役性が保たれる.
        \item 
    \end{enumerate}
\end{Proof}

\begin{proposition}
    任意のBanach空間$X,Y$について,
    \begin{enumerate}
        \item $B_0(X)=B(X)$であることと,$X$が有限次元であることとは同値.
        \item $B_0(X,Y)$は$B(X,Y)$の閉部分空間である.
        \item $u\in B_0(X,Y)\Rightarrow u^*\in B_0(Y^*,X^*)$.
    \end{enumerate}
\end{proposition}

\subsection{正規なコンパクト作用素に関するスペクトル定理}

\begin{tcolorbox}[colframe=ForestGreen, colback=ForestGreen!10!white,breakable,colbacktitle=ForestGreen!40!white,coltitle=black,fonttitle=\bfseries\sffamily,
title=]
    作用素の対角化とは,直交射影による表示としたのであった\ref{remarks-diagonalizability}.
    ちょうど$B_0(H)$には\underline{直交}射影からなる近似的単位元が存在することを確認したが,
    正規コンパクト作用素は,固有値の言葉で特徴付けることが出来る.
\end{tcolorbox}

\begin{lemma}
    $T\in B(H)$が対角化可能であるとき,
    \begin{enumerate}
        \item $T$はコンパクトである.
        \item 正規直交基底$(e_j)_{j\in J}$に対応する固有値$(\lambda_j)_{j\in J}$は$c_0(J)$の元である.
    \end{enumerate}
\end{lemma}

\begin{lemma}
    $x\in H$を正規作用素$T\in B(H)$の固有ベクトルとし,対応する固有値を$\lambda\in\C$とする.
    \begin{enumerate}
        \item $x$は$T^*$の固有ベクトルであり,対応する固有値は$\o{\lambda}$である.
        \item $T$の他の固有値に対応する固有ベクトルは$x$に直交する.
    \end{enumerate}
\end{lemma}

\begin{lemma}
    複素Hilbert空間$H$上の正規なコンパクト作用素$T$は,$\abs{\lambda}=\norm{T}$を満たす固有値$\lambda\in\C$を持つ.
\end{lemma}

\begin{theorem}[正規コンパクト作用素の特徴付け]\label{thm-Spectral-Theorem-for-normal-compact-operator}
    $H$を複素Hilbert空間,$T\in B(H)$とする.次の2条件は同値.
    \begin{enumerate}
        \item $T$は正規なコンパクト作用素である.
        \item $T$は対角化可能で,その固有値は無限遠で消える.\footnote{従って,$H$には$T$の固有ベクトルからなる正規直交基底を取れることとも同値.}
    \end{enumerate}
\end{theorem}

\begin{notation}[ベクトルの定める直交射影]
    $x,y\in H$に対して,$x\odot y\in B(H)$を$(x\odot y)z=(z|y)x\;(z\in H)$で定まる$\Im x\odot y=\bF x$を満たす階数$1$の作用素とする.
    この構成は半双線型写像$H\times H\to B_f(H);(e_i,e_j)\mapsto e_i\odot e_j$を定める.
    すると,$\norm{e}=1$について,$e\odot e$は$\C e$への直交射影である.
\end{notation}

\begin{remarks}[正規なコンパクト作用素についてのスペクトル定理としての消息]
    正規なコンパクト作用素は,ある正規直交基底$(e_j)_{j\in J}$が存在して,ノルム収束する級数$T=\sum_{j\in J}\lambda_je_j\odot e_j$と表せるクラスである.
    これは,$J_0:=\Brace{j\in J\mid\lambda_j\ne0}$が有限集合であるか,係数列$(\lambda_j)_{j\in J_0}$は$0$に収束するかのいずれかであるからである.
    このとき,$T$のスペクトルは集積点$0$を併せて,$\Sp(T)=\{\lambda_j\}_{j\in J}\cup\{0\}$と表せる.
    ここで,$*$-代数の等長準同型
    \[\xymatrix@R-2pc{
        C(\Sp(T))\ar[r]&B(H)\\
        \rotatebox[origin=c]{90}{$\in$}&\rotatebox[origin=c]{90}{$\in$}\\
        f\ar@{|->}[r]&f(T):=\sum_{j\in J}f(\lambda_j)e_j\odot e_j
    }\]
    を考えると,$f(T)\in B_0(H)$であることは,$f(0)=0$であることに同値であることがわかる.
    特に$f(z)=\sum\al_{nm}z^n\o{z}^m$と表せる場合を考えると,$f(T)=\sum\al_{nm}T^nT^{*m}$となる.
\end{remarks}

\subsection{Fredholm作用素}

\begin{tcolorbox}[colframe=ForestGreen, colback=ForestGreen!10!white,breakable,colbacktitle=ForestGreen!40!white,coltitle=black,fonttitle=\bfseries\sffamily,
title=Hilbert空間の同値]
    コンパクト作用素の違い(=コンパクトな摂動)を無視して,作用素を見た世界を
    Calkin代数$B(H)\epi B(H)/B_0(H)$といい,この上に同じ像を定める作用素は,同値類を定める.
    これは応用上重要で,
    この同値類についての不変量が重要な意味を持つこととなる.

    そこで,「コンパクト代数の分だけ緩めて可逆」という準可逆な作用素をFredholm作用素とし,これを調べる.
    これは,核と余核が高々有限次元で,像が閉であることに同値.
\end{tcolorbox}

\begin{definition}[Calkin algebra]
    $B_0(H)$は閉イデアルであるから,商$B(H)/B_0(H)$は商ノルムについてBanach代数を定める.
    これを\textbf{Calkin代数}という.\footnote{Wikipediaによると$H$が可分である場合のみを指す,指数理論と作用素環論の対象である.}
\end{definition}

\begin{lemma}
    Calkin代数$B(H)/B_0(H)$は$C^*$-代数である.
\end{lemma}

\begin{definition}[compact perturbation]
    $S,T\in B(H)$が$S-T\in B_0(H)$を満たすとき,片方はもう片方の\textbf{コンパクトな摂動}であるという.
    すなわち,Calkin代数上に同じ像を定めることをいう.
\end{definition}

\begin{proposition}[Atkinson's theorem]
    $T\in B(H)$について,次の4条件は同値.これらを満たす作用素を\textbf{Fredholm作用素}といい,その全体を$F(H)$で表す.
    \begin{enumerate}
        \item ある作用素$S\in B(H)$が一意的に存在して$\Ker S=\Ker T^*$かつ$\Ker S^*=\Ker T$が成り立ち,$ST,TS$はそれぞれ$(\Ker T)^\perp,(\Ker T^*)^\perp$上への有限な余次元を持つ射影を定める.
        \item ある作用素$S\in B(H)$が存在して,$ST-I,TS-I$はいずれもコンパクトである.
        \item $T$は$B(H)/B_0(H)$で見れば可逆である.
        \item $\Ker T,\Ker T^*$は有限次元で,$\Im T$は閉である.
    \end{enumerate}
\end{proposition}
\begin{remarks}
    $F(H)$は積で閉じており,自己共役である($*$作用素について閉じている).
\end{remarks}

\begin{definition}[index, nullity, defect]
    $T\in F(H)$をFredholm作用素とする.
    \begin{enumerate}
        \item $\Index T:=\dim\Ker T-\dim\Ker T^*$を\textbf{指数}という.$\dim\Ker T$をnullity,$\dim\Ker T^*$をdefectという.
        \item このとき,射影$P,Q$を$ST=I-P,TS=I-Q$と定めると,$\Index T=\rank P-\rank Q$でもある.
        \item $F(H)$の部分集合を
        \[F_n(H):=\Brace{T\in F(H)\mid\Index T=n}\quad n\in\Z\]
        で定めると,$\forall_{n\in\Z}\;F_n(H)\ne\emptyset$.シフト作用素\ref{operator-unilateral-shift}$S$について,$\forall_{n\in\N}\;S^n\in F_{-n}(H),S^{*n}\in F_n(H)$が成り立つため.
    \end{enumerate}
\end{definition}
\begin{example}
    任意の正方行列は,指数$0$のFredholm作用素である:$\dim\Ker A=\dim\Ker A^*$.
\end{example}

\begin{lemma}
    $T\in F(H)$とする.
    \begin{enumerate}
        \item $R\in B(H)$を可逆とすると,$\Index RT=\Index TR=\Index T$.
        \item $\Index T^*=-\Index T$.
        \item $T$の部分逆を$S$とすると,$\Index S=-\Index T$.
    \end{enumerate}
\end{lemma}

\subsection{Fredholmの交代定理}

\begin{tcolorbox}[colframe=ForestGreen, colback=ForestGreen!10!white,breakable,colbacktitle=ForestGreen!40!white,coltitle=black,fonttitle=\bfseries\sffamily,
title=コンパクト作用素のスペクトルとは固有値の集合である]
    無限線型系に対して最初に正確な議論をしたのがFredholmであった.
    Dirichlet問題は積分方程式に帰着され,これは無限の連立1次方程式とみれる.
    積分方程式が解を持つための条件を,Fredholm行列式$\delta:\C\to\C$が零でないこととして特徴付けた定理がFredholmの交代定理/択一定理である.
    これは現代の言葉では,コンパクト作用素のスペクトル$\Sp(T)$の固有値の集合としての特徴付けである.
\end{tcolorbox}

\begin{lemma}
    $A\in B_f(H)$ならば,$I+A\in F_0(H)$.
\end{lemma}
\begin{remark}
    任意の$\lambda\ne0$に対して,コンパクト作用素との和$\lambda+K$は指数$0$のFredholm作用素である.
\end{remark}

\begin{lemma}
    任意の$T\in F_0(H)$について,
    \begin{enumerate}
        \item 部分等長作用素$V\in B_f(H)$が存在して,$T+V$は可逆である.
        \item $A\in B_0(H)$ならば,$T+A\in F_0(H)$.
    \end{enumerate}
\end{lemma}

\begin{corollary}[Fredholm alternative]
    $A\in B_0(H)$,$\lambda\in\C\setminus\{0\}$とする.
    このとき,次のいずれか一方のみが成り立つ.
    \begin{enumerate}
        \item $\lambda I-A\in B(H)$は可逆である.すなわち,$\lambda$はレゾルベント集合に属する.
        \item $\lambda$は$A$の有限な重複度を持った固有値である.
    \end{enumerate}
    後者の場合,複素共役$\o{\lambda}$は$A^*$の固有値で,同じ重複度を持つ.
\end{corollary}
\begin{remarks}[Riesz-Schauder]
    コンパクト作用素$T$のスペクトルは,$\{0\}$と,$T$の固有値のみからなる.
    特に,$\C$の可算部分集合で,(集積点があるとするならば)$0$のみを集積点とする.
    また,$\sigma(T)=\o{\sigma(T^*)}$(閉包ではなく実軸反転)で,
    $\dim\Ker(T-\lambda I)=\dim\Ker(T^*-\o{\lambda}I)<\infty$.
\end{remarks}

\begin{tbox}{red}{}
    $T:=I-\lambda^{-1} A$が指数$0$のFredholm作用素である,ということが古典的なFredholm交代定理の肝である.
    こうして,正方行列の基本的な事実(特に連立一次方程式系の可解性に関連して)は,この恒等作用素とコンパクト作用素の差で表される作用素クラス$T\in F_0(H)$に一般化される.
\end{tbox}

\subsection{Fredholm作用素の指数}

\begin{theorem}
    任意のFredholm作用素$T\in F(H)$とコンパクト作用素$A\in B_0(H)$について,$\Index(T+A)=\Index T$.
\end{theorem}

\begin{proposition}
    任意のFredholmクラス$F_n(H)$は$B(H)$の開集合である.
\end{proposition}

\begin{proposition}
    $T_1\in F_n(H),T_2\in F_m(H)$について,$T_1T_2\in F_{n+m}(H)$.
\end{proposition}

\begin{remarks}
    $G:=B(H)/B_0(H)$を乗法群とし,$G_0$を単位元を含む連結成分とする.すなわち,ある$A\in B(H)/B_0(H)$について$\exp A$と表せる元が生成する部分群とする.
    このとき,$G_0$は$G$内で開かつ閉で,$G/G_0$は離散群で$G$の連結成分を特徴付ける.
    今回,$G/G_0\simeq\Z$である.
    そして,Fredholm作用素$T$の指数とは,作用素$F(H)\epi G\epi G/G_0=\Z$による$T$の像である.
\end{remarks}

\subsection{Hilbert-Schmidtの定理}

\begin{theorem}
    $H$を可分Hilbert空間,$T\in B_0(H)$を自己共役とする.このとき,$T$の固有ベクトルからなる正規直交基底$\{\phi_n\}$が存在する.
    $H$が無限次元ならば,対応する固有値を$\{\lambda_n\}\in c_0$とすると,
    \[\forall_{u\in H}\;Tu=\sum_{n\in\N}\lambda_n(u|\phi_n)\phi_n.\]
\end{theorem}

\subsection{Banach空間上での再論}

\begin{definition}[ascent, descent]
    $X$を線型空間,$u:X\to X$を線型写像とすると,$(\Ker(u^n))_{n\in\N}$は部分空間の増大列で,$(u^n(X))_{n\in\N}$は部分空間の減少列である.
    \begin{enumerate}
        \item $\forall_{n\in\N}\;\Ker(u^n)\ne\Ker(u^{n+1})$のとき,$\Ascent(u)=\infty$とし,それ以外のとき,$\Ascent(u):=\min\Brace{p\in\N\mid\Ker(u^p)=\Ker(u^{p+1})}$と定める.
        \item $\Descent(u):=\Brace{p\in\N\mid u^p(X)=u^{p+1}(X)}$とする.
    \end{enumerate}
\end{definition}

\begin{theorem}
    $X$をBanach空間,$u\in B_0(X)$をコンパクト作用素,$\lambda\in\C\setminus\{0\}$とする.
    \begin{enumerate}
        \item $u-\lambda$は有限なascentとdescentを持つ.
        \item $u-\lambda$はFredholm指数$0$を持つ.
        \item $p:=\Ascent(u)$とすると,$X=\Ker(u-\lambda)^p\oplus\Im(u-\lambda)^p$.
        \item (Fredholm alternative) $u-\lambda$が単射であることと全射であることとは同値.
        \item $\Sp(u)$は可算集合で,$\Sp(u)\setminus\{0\}$は$u$の固有値の集合であり,孤立点のみからなる.
    \end{enumerate}
\end{theorem}

\section{跡}

\begin{quotation}
    関数論とヒルベルト空間上の作用素論との類比において,$C_0,B_0$と$C_c,B_f$は対応するが,
    $B(H)$は2つの役割を持つ.1つは$C_b$であるが,もう一つは$L^\infty(X)$である.そのためにはLebesgue測度にあたる概念が必要であるが,これが跡である.
    このような測度論との交差が特に美しい.いつしか積分もただの線型作用素であったし,級写像も積分作用素の特殊な例なのであった.

    跡が積分に当たることを強烈に示唆する例が,Shannonのエントロピーが事象$\rho$に対して$-E[\rho\log\rho]$であるのに対して,von Neumannエントロピーが密度行列$\rho$に対して$-\Tr(\rho\log\rho)$である.
\end{quotation}

\begin{tcolorbox}[colframe=ForestGreen, colback=ForestGreen!10!white,breakable,colbacktitle=ForestGreen!40!white,coltitle=black,fonttitle=\bfseries\sffamily,
title=跡は関数空間上の測度とみなせる]
    前節では$B_f(H)$を$C_c(X)$に,$B_0(H)$を$C_0(X)$に,$B(H)$を$C_b(X)$に見立てたが,$B(H)$は同時に$L^\infty(X)$に似た振る舞いもする.
    では$X$上のLebesgue測度にあたる$H$上の構造はというと,跡である!
    跡は関数空間上の測度と思えるのか!

    有限次元線型空間論では,跡だけがまるで異邦人のような特異な存在であった.
\end{tcolorbox}

\subsection{定義と性質}

\begin{definition}[trace]
    $H$をHilbert空間,$(e_j)_{j\in J}$をその正規直交基底とする.
    正作用素$T\in B(H)$の\textbf{跡}とは,
    \[\Tr(T):=\sum_{j\in J}(Te_j|e_j)\]
    をいう.$\Tr:B(H)_+\to[0,\infty]$である.
\end{definition}

\begin{proposition}
    $\forall_{T\in B(H)}\;\Tr(TT^*)=\Tr(T^*T)$.
\end{proposition}

\begin{corollary}[well-definedness]
    $T\in B(H)$を正作用素とする.
    任意のユニタリ作用素$U$について,$\Tr(UTU^*)=\Tr(T)$.
    
    特に,跡の定義は正規直交基底の取り方に依らず,$\norm{T}\le\Tr(T)$を満たす.
\end{corollary}

\subsection{跡の延長}

\begin{tcolorbox}[colframe=ForestGreen, colback=ForestGreen!10!white,breakable,colbacktitle=ForestGreen!40!white,coltitle=black,fonttitle=\bfseries\sffamily,
title=]
    跡が測度の類似物ならば,なんらかの意味で測度確定であることはコンパクト性を含意しそうである.
    測度確定なコンパクト作用素をHilbert-Schmidt作用素という.
    さらに中線定理や極化恒等式も成り立ち,$B(H)$は幾何学味を帯びてくる.
\end{tcolorbox}

\begin{lemma}
    正作用素$T\in B(H)$が$\exists_{p>0}\;\Tr(\abs{T}^p)<\infty$を満たすならば,コンパクト作用素である.
\end{lemma}

\begin{definition}[trace class operator, Hilbert-Schmidt operator]\mbox{}\label{def-trace-class-Hilbert-Schmidt}
    \begin{enumerate}
        \item $B^1(H):=\Span\Brace{T\in B_0(H)\mid T\ge0,\Tr(T)<\infty}$の元を\textbf{跡類作用素}という.
        \item $B^2(H):=\Brace{T\in B_0(H)\mid\Tr(T)<\infty}$.
    \end{enumerate}
\end{definition}

\begin{lemma}[跡の延長]
    $T\in B^1(H)$について,
    \begin{enumerate}
        \item ある正作用素$T_0,\cdots,T_3\ge0$が存在して,$T=\sum_{k=0}^3i^kT_k$.
        \item これを用いて$\Tr(T):=\sum_{k=0}^3i^k\Tr(T_k)$と定めると,これは線型汎函数である.
    \end{enumerate}
    以降,$\Tr$の定義域は$B(H)_++B^1(H)$とする.
\end{lemma}

\begin{lemma}[作用素の中線定理と極化恒等式]
    任意の$S,T\in B(H)$について,
    \begin{enumerate}
        \item (parallelogram law) $(S+T)^*(S+T)+(S-T)^*(S-T)=2(S^*S+T^*T)$.
        \item $(S+T)^*(S+T)\le 2(S^*S+T^*T)$.
        \item (polarization identity) $4T^*S=\sum^3_{k=0}i^k(S+i^kT)^*(S+i^kT)$.
    \end{enumerate}
\end{lemma}

\subsection{Hilbert-Schmidt作用素のHilbert空間}

\begin{tcolorbox}[colframe=ForestGreen, colback=ForestGreen!10!white,breakable,colbacktitle=ForestGreen!40!white,coltitle=black,fonttitle=\bfseries\sffamily,
title=]
    単に幾何学的なだけではなく,$B^2(H)$は実際にHilbert空間をなす.
\end{tcolorbox}

\begin{proposition}\mbox{}
    \begin{enumerate}
        \item $B^1(H),B^2(H)$は$B(H)$内の自己共役なイデアルである.
        \item $B_f(H)\subset B^1(H)\subset B^2(H)\subset B_0(H)$.
    \end{enumerate}
\end{proposition}

\begin{theorem}
    Hilbert-Schmidt作用素のイデアル$B^2(H)$は内積$(S|T)_{\tr}:=\Tr(T^*S)$についてHilbert空間をなす.
\end{theorem}

\subsection{跡類作用素のBanach代数}

\begin{lemma}
    $T\in B^1(H),S\in B(H)$について,$\abs{\Tr(ST)}\le\norm{S}\Tr(\abs{T})$.
\end{lemma}

\begin{lemma}
    $S,T\in B^2(H)$について,$\Tr(ST)=\Tr(TS)$.
\end{lemma}

\begin{theorem}
    跡類作用素のイデアル$B^1(H)$はノルム$\norm{T}:=\Tr(\abs{T})$についてBanach代数をなす.
\end{theorem}

\begin{theorem}
    双線型形式$\brac{S,T}:=\Tr(ST)$は,Banach空間$B_0(H),B^1(H)$の間のペアリングであり,また$B^1(H),B(H)$の間のペアリングでもある.
    特に,$(B_0(H))^*=B^1(H)$かつ$(B^1(H))^*=B(H)$.
\end{theorem}

\subsection{Hilbert-Schmidt作用素の表現}

\begin{tcolorbox}[colframe=ForestGreen, colback=ForestGreen!10!white,breakable,colbacktitle=ForestGreen!40!white,coltitle=black,fonttitle=\bfseries\sffamily,
title=]
    これを通じて,例\ref{exp-Hilbert-Schmidt}との同値性がわかる.
\end{tcolorbox}

\begin{proposition}
    Hilbert空間$H$の任意の正規直交基底$(e_j)_{j\in J}$について,$1$階の作用素の集合$\Brace{e_i\odot e_j\in H\mid i,j\in J}$は$B^2(H)$の正規直交基底である.
\end{proposition}
\begin{remarks}
    特に,Hilbert空間$H$が$X$上のRadon積分に関する2乗可積分関数の空間$L^2(X)$であったとき,$\int\otimes\int$とは$X^2$上の積積分(product integral)のことだから,
    これはHilbert-Schmidt作用素を$L^2(X^2)$上に実現していることとなる.
    そして,$L^2(X)$の基底$(e_j)_{j\in J}$に対して,$e_i\otimes\o{e_j}(x,y)=e_i(x)e_j(y)$が$L^2(X^2)$上の正規直交基底である.
    こうして,$U:L^2(X^2)\iso B^2(L^2(X));e_i\otimes\o{e_j}\mapsto e_i\odot e_j$は等長同型である.
    この等長同型を具体的に表現すると次の通り.
\end{remarks}

\begin{proposition}[上述の命題の$L^2$における場合での描像]\mbox{}
    \begin{enumerate}
        \item $k\in L^2(X^2)$を核とする積分作用素
        \[\xymatrix@R-2pc{
            T_k:L^2(X)\ar[r]&L^2(X)\\
            \rotatebox[origin=c]{90}{$\in$}&\rotatebox[origin=c]{90}{$\in$}\\
            f\ar@{|->}[r]&T_kf(x):=\int_{y\in X}k(x,y)f(y)
        }\]
        はHilbert-Schmidt作用素である:$T_k\in B^2(L^2(X))$.
        \item 対応$T:L^2(X^2)\to B^2(L^2(X));k\mapsto T_k$は,$L^2(X^2)$上の2-ノルムについて等長同型になる.
        \item $k^*(x,y)=\o{k(x,y)}$とすると,$T_{k^*}=T^*_k$.すなわち,$T_k$が自己共役であることと,$k$が共役対称であることとは同値.
    \end{enumerate}
\end{proposition}

\subsection{Fredholm積分方程式}

\begin{proposition}
    Fredholm型の積分方程式
    \[\int_{y\in X}k(x,y)f(y)-\lambda f(x)=g(x)\]
    について,
    \begin{enumerate}
        \item $g\in L^2(X)$かつ$k\in L^2(X^2)$が共役対称ならば,解は存在する.
        \item さらに$\forall_{j\in J}\;\lambda\ne\lambda_j$ならば,解は一意であり,次のように表示される:
        \[f=\sum_{j\in J}\frac{(g|e_j)}{\lambda_j-\lambda}e_j\]
    \end{enumerate}
\end{proposition}
\begin{Proof}\mbox{}
    \begin{enumerate}
        \item $T_k$は自己共役なコンパクト作用素(Hilbert-Schmidt作用素)になるから,
        ある直交基底$(e_i)_{i\in J}$に対して,
        スペクトル分解が存在する:
        \[T_k=\sum\lambda_je_j\odot e_j\;\in B^2(L^2(X))\qquad\lambda_j\in\R,\sum\abs{\lambda_j}^2=\norm{T_k}^2=\norm{k}^2_2\]
        なお,$\lambda_i=(T_ke_i|e_i)\in\R$\ref{cor-characterization-of-self-adjointness}より,自己共役作用素の固有値は実数である.
        また,$\norm{T_k}=\norm{k}_2$は,$T:L^2(X^2)\to B^2(L^2(X))$が等長同型であることによる.
        よって,積分方程式は,次の式に同値:
        \[\sum_{j\in J}(\lambda_j-\lambda)(f|e_j)e_j=g.\]
        \item 明らか.
    \end{enumerate}
\end{Proof}

\subsection{Strum-Liouville問題への応用}

\begin{tcolorbox}[colframe=ForestGreen, colback=ForestGreen!10!white,breakable,colbacktitle=ForestGreen!40!white,coltitle=black,fonttitle=\bfseries\sffamily,
title=]
    微分方程式の境界値問題の解はGreen関数を積分核とする積分作用素の像として与えられる.
\end{tcolorbox}

\begin{problem}[Strum-Liouville問題]
    $I:=[a,b]$上の2階線型微分方程式を,3種類考える:
    \begin{enumerate}
        \item $(pf')'+qf=0$.
        \item $(pf')'+qf=\lambda f$.
        \item $(pf')'+qf=\lambda f+h$.
    \end{enumerate}
    ただし,$p,q,h\in C(I;\R),p>0,\lambda\in\R$とした.
\end{problem}

\begin{proposition}[斉次の場合]
    次の2条件を満たす$u,v$が存在するとき,(1)は$u,v$によって生成される2次元の完備な解空間を持つ.
    \begin{enumerate}
        \item 境界条件:$\al,\beta,\gamma,\delta\in\R$について,$\al u(a)+\beta p(a)u'(a)=0$かつ$\gamma v(b)+\delta p(b)v'(b)=0$.
        \item 特殊解の線型独立性:Wronskianはある$c\ne0$について$p(uv'-u'v)=c$を満たす.
    \end{enumerate}
\end{proposition}

\begin{proposition}[非斉次の場合]
    上の命題の状況下で,Green関数を
    \[g(x,y):=\begin{cases}
        c^{-1}u(x)v(y),&a\le x\le y\le b\\
        c^{-1}u(y)v(x),&a\le y\le x\le b
    \end{cases}\]
    と定めると,$h\in L^2(I)$ならば($h\in C(I)$はこれを満たす),Hilbert-Schmidt作用素の像$f:=T_gh$が(3)の$\lambda=0$の場合の,次の境界条件を満たすただ一つの解である.
    \begin{quote}
        (B):$\al f(a)+\beta p(a)f'(a)=\gamma f(b)+\delta p(b)f'(b)=0$.
    \end{quote}
\end{proposition}

\begin{proposition}[Hilbert-Schmidt作用素の問題への還元]
    \begin{enumerate}
        \item (2)の境界条件(B)を満たす解が存在するための必要十分条件は,$\exists_{n\in\N}\;\lambda=\lambda_n^{-1}$.このとき,解は$e_n$である.
        \item (2)の境界条件(B)を満たす解が存在するための必要十分条件は,$\exists_{n\in\N}\;\lambda=\lambda_n^{-1}\;\land h\perp e_n$または$\forall_{n\in\N}\;\lambda\ne\lambda_n^{-1}$である.このとき,解は
        \[f=\sum\frac{\lambda_n(h|e_n)}{1-\lambda\lambda_n}e_n\]
    \end{enumerate}
\end{proposition}
\begin{discussion}
    いま,Green関数は共役対称だから,$L^2(I)$の基底$(e_n)_{n\in\N}$が存在して,スペクトル分解
    \[T_g=\sum\lambda_ne_n\odot e_n,\quad\{\lambda_n\}\subset\R\setminus\{0\},\sum\abs{\lambda_n}^2=\norm{g}^2_2.\]
\end{discussion}

\section{再生核}

\begin{notation}\mbox{}
    \begin{enumerate}
        \item $H(\Delta):=\O(\Delta)$とする.
        \item $T_zf=zf$を$\id_\C$による前方シフト作用素,
        $S_zf=\frac{f-f(0)}{z}$を後方シフト作用素とする.
        このとき
        \[S_zT_z-T_zS_z=\ev_0\]
        が成り立つ.
        \item 一般の$\phi\in H(\Delta)$について,$(T_\phi f)(z)=\phi(z)f(z)$を乗算作用素とする.
        \item 一般の$a\in\De$に対して,$S_{z,a}f=\frac{f-f(a)}{z-a}$とする.
    \end{enumerate}
\end{notation}

\subsection{再生核Hilbert空間の定義}

\begin{definition}[reproducing kernel]
    Hilbert空間$\H\subset H(\De)$と
    $a\in\De$について,$\ev_a\in\H^*$のRiesz表現$K_a\in H$を\textbf{$a$に関する再生核}という.
\end{definition}

\begin{lemma}
    $\H\subset H(\De)$は$T_z\in H(\De)$に関して不変であるとする.このとき,$\forall_{a\in\De}\;T^*_zK_a=\o{a}K_a$.
\end{lemma}
\begin{Proof}
    \[(f|T^*_zK_a)=(zf|K_a)=af(a)=(f|\o{a}K_a).\]
\end{Proof}

\begin{proposition}
    $1\in\H\subset H(\De),a\in\De$とする.
    \begin{enumerate}
        \item $\ev_a$は常に$\H$上有界であるとする.$\Ker\ev_a\subset\H$は閉部分空間.
        \item $\H$は$T_z$-不変とする.$\Im T_{z-a}\subset\Ker\tau_a$.特に,$\H$上稠密でない.
        \item 等号成立条件は$\H$が$S_{z,a}$-不変であることである.
    \end{enumerate}
\end{proposition}

\begin{proposition}
    $1\in\H\subset H(\De)$とする.
    \begin{enumerate}
        \item $\H$が$T_z$-不変ならば,$T_z$は$\H$上全射でない.
        \item $\H$が$S_z$-不変ならば,$S_z$が全射であることと,$\H$が$T_z$-不変であることとは同値.
    \end{enumerate}
\end{proposition}

\begin{definition}
    $\H\subset H(\De)$が次を満たすとき,\textbf{再生核Hilbert空間}という:
    \begin{enumerate}
        \item $\H$は$\De$の点を分離する.
        \item $\De\subset\H^*$.すなわち,$\forall_{a\in\De}\;\ev_a\in\H^*$.
    \end{enumerate}
\end{definition}

\subsection{性質と例}

\begin{tcolorbox}[colframe=ForestGreen, colback=ForestGreen!10!white,breakable,colbacktitle=ForestGreen!40!white,coltitle=black,fonttitle=\bfseries\sffamily,
title=]
    Hardy空間のノルムには3つの等価な表示がある.
    \begin{enumerate}
        \item $r\partial\De\;(0\le r<1)$上の積分の値として.
        \item 整関数展開の係数の二乗和として.
        \item $\partial\De$上の(境界値の)積分の値として.多項式の閉包となる再生核Hilbert空間としての特徴付けから得られる.
    \end{enumerate}
\end{tcolorbox}

\begin{proposition}[再生核Hilbert空間上の有界作用素になる条件]
    $\H\subset H(\De)$を再生核Hilbert空間とする.
    \begin{enumerate}
        \item $\H$が$T_z$-不変ならば,$T_z\in B(\H)$.
        \item $\H$が$S_z$-不変ならば,$S_z\in B(\H)$.
        \item $\forall_{\phi\in H(\De)}\;T_\phi\H\subset\H\Rightarrow T_\phi\in B(\H)$.
    \end{enumerate}
\end{proposition}

\begin{proposition}
    $\H\subset H(\De)$を可分な再生核Hilbert空間とする.
    $\{g_n\}\subset\H$について,次の2条件は同値.
    \begin{enumerate}
        \item $g_n\wto g$.
        \item $\Brace{\norm{g_n}}$が有界,かつ,$(g_n)$は$\De$上各点収束する.
    \end{enumerate}
\end{proposition}

\begin{example}\mbox{}
    \begin{enumerate}
        \item \textbf{Hardy空間}$H^2(\De)$を,$r\partial\Delta\;(0\le r<1)$上での$L^2$-ノルムの上限が有限なものとする:
        \[H^2(\De):=\Brace{f\in H(\De)\;\middle|\;\norm{f}_{H^2}^2:=\sup_{0\le r<1}\frac{1}{2\pi}\int^{2\pi}_0\abs{f(re^{i\theta})}^2d\theta<\infty}.\]
        \item \textbf{Bergman空間}$L^2_a(\De)$を,$\De$上の面積測度に関する$L^2$-空間とする:
        \[L^2_a(\De):=\Brace{f\in H(\De)\;\middle|\;\norm{f}_{L_a^2}^2:=\frac{1}{\pi}\int_\De\abs{f(re^{i\theta})}^2rdrd\theta<\infty}.\]
        \item \textbf{Dirichlet空間}$D^2(\De)$を,$\De$上の面積測度に関して$f'$が二乗可積分なもののなす空間とする:
        \[D^2(\De):=\Brace{f\in H(\De)\;\middle|\;\norm{f'}_{L_a^2}^2=\frac{1}{\pi}\int_\De\abs{f'(re^{i\theta})^2rdrd\theta}}.\]
        とし,$\norm{f}_{D^2}^2:=\norm{f}^2_{H^2}+\norm{f'}^2_{H^2}$とする.
    \end{enumerate}
\end{example}

\begin{proposition}[Hardy空間の特徴付け]\mbox{}
    \begin{enumerate}
        \item $f\in H(\De)$の整級数展開を$f(z)=\sum_{n\in\N}a_nz^n$とする.次の2条件は同値:
        \begin{enumerate}
            \item $f\in H^2$.
            \item $a_n\in l^2(\N)$.
        \end{enumerate}
        \item $\C[z]\subset D^2\subset H^2\subset L_a^2$.
        \item $f\in D^2$かつ$f(0)=0$のとき,ノルムは次のように評価できる:
        \[\norm{f}^2_{D^2}\le \frac{2}{\pi}\int_\De\abs{f'(re^{i\theta})}^2rdrd\theta.\]
    \end{enumerate}
\end{proposition}

\subsection{数列と正則関数}

\begin{tcolorbox}[colframe=ForestGreen, colback=ForestGreen!10!white,breakable,colbacktitle=ForestGreen!40!white,coltitle=black,fonttitle=\bfseries\sffamily,
title=]
    いずれの空間も,$\De$上での整関数展開係数の,重みをつけた二乗和をノルムとした空間と同一視できる.
\end{tcolorbox}

\begin{notation}\mbox{}
    \begin{enumerate}
        \item $f\in H(\De)$について,その\textbf{境界値}を
        \[f_*(e^{i\theta}):=\sum_{n\in\N}a_ne^{in\theta}\]
        で定める.
        \item $H^2(\partial\De):=\Brace{f_*\mid f\in H^2(\De)}$とおく.ノルムは引き続き$\norm{f_*}^2:=\sum_{n\in\N}\abs{a_n}^2$とする.
        \item 正数列$\{\gamma_n\}\subset\R^+$について,
        \[H^2_\gamma:=\Brace{f\in H(\De)\mid f(z)=\sum_{n\in\N}a_nz^n\land(a_n\gamma_n)\in l^2(\N)}.\]
    \end{enumerate}
\end{notation}
\begin{remarks}
    実は,$\lim_{r\to1}f(re^{i\theta})=f_*(e^{i\theta})\;\ae\theta$が成り立つ.これを\textbf{動径極限値}(radial limit)という.
\end{remarks}

\begin{theorem}\mbox{}
    \begin{enumerate}
        \item 次の2条件は同値:
        \begin{enumerate}
            \item $S_z$は$H^2_\gamma$上有界.
            \item $\sup_{n\in\N}\frac{\gamma_n}{\gamma_{n+1}}<\infty$.
        \end{enumerate}
        \item 次の2条件は同値:
        \begin{enumerate}
            \item $T_z$は$H^2_\gamma$上有界.
            \item $\sup_{n\in\N}\frac{\gamma_{n+1}}{\gamma_{n}}<\infty$.
        \end{enumerate}
    \end{enumerate}
\end{theorem}

\begin{theorem}
    次の2条件は同値:
    \begin{enumerate}
        \item $H^2_\gamma$はHilbert空間である.
        \item $\liminf_{n\to\infty}\gamma_n^{1/2}\ge1$.
    \end{enumerate}
\end{theorem}

\begin{theorem}
    $H^2_\gamma$はHilbert空間であるとする.このとき,再生核Hilbert空間でもある:
    \begin{enumerate}
        \item $\forall_{a\in\De}\;\ev_a\in(H^2_\gamma)^*$.
        \item $\forall_{a\in\De}\;K_a(z)=\sum_{n\in\N}\frac{(\o{a}z)^n}{\gamma_n^2}$.
        \item $\norm{K_a}^2_\gamma=\sum_{n\in\N}\frac{\abs{a}^{2n}}{\gamma_n^2}$,
    \end{enumerate}
\end{theorem}

\begin{corollary}[Hardy空間の性質]
    $\gamma=1$を定値数列とする.
    \begin{enumerate}
        \item $H_\gamma^2$は再生核Hilbert空間である.
        \item $H^2_\gamma$上で$T_z,S_z$は縮小作用素であり,$T^*_z=S_z$が成り立つ.
        \item $T_z$は等長写像である.
        \item $H^2_\gamma\simeq_\Ban H^2$,かつ,$\norm{-}_\gamma=\norm{-}_{H^2}$.
        \item 任意の$a\in\De$について,
        \[K_a(z)=\frac{1}{1-\o{a}z},\quad\norm{K_a}_\gamma^2=\frac{1}{1-\abs{a}^2}.\]
    \end{enumerate}
\end{corollary}

\begin{corollary}[Bregman空間の性質]
    $\gamma_n=(1+n)^{-1/2}$を正数列とする.
    \begin{enumerate}
        \item $H_\gamma^2$は再生核Hilbert空間である.
        \item $H^2_\gamma$上で$T_z,S_z$は有界である.
        \item $H^2_\gamma\simeq_\Ban L_a^2$,かつ,$\norm{-}_\gamma=\norm{-}_{L_a^2}$.
        \item 任意の$a\in\De$について,
        \[K_a(z)=\frac{1}{(1-\o{a}z)^2},\quad\norm{K_a}_\gamma^2=\frac{1}{(1-\abs{a}^2)^2}.\]
    \end{enumerate}
\end{corollary}

\begin{corollary}[Dirichlet空間の性質]
    $\gamma=(1+n)^{1/2}$を正数列とする.
    \begin{enumerate}
        \item $H_\gamma^2$は再生核Hilbert空間である.
        \item $H^2_\gamma$上で$T_z,S_z$は有界である.
        \item $H^2_\gamma\simeq_\Ban D^2$,かつ,$\norm{-}_\gamma=\norm{-}_{D^2}$.
        \item 任意の$a\in\De$について,
        \[K_a(z)=\frac{1}{\o{a}z}\log\paren{\frac{1}{1-\o{a}z}},\quad\norm{K_a}_\gamma^2=\frac{1}{\abs{a}^2}\log\paren{\frac{1}{1-\abs{a}^2}}.\]
    \end{enumerate}
\end{corollary}

\subsection{多項式と正則関数}

\begin{definition}
    $\mu:\B([\De])\to\R_+$を有限なBorel測度とする.
    \begin{enumerate}
        \item $H^2(\mu):=\oo{\C[z]}\subset L^2(\mu)$を\textbf{重み付きHardy空間}という.
        \item $a\in[\De]$について,
        \[S(\mu,a):=\inf_{f(a)=1,f\in\C[z]}\int_{[\De]}\abs{f^2}d\mu,\quad R(\mu,a):=\sup_{f\in\C[z],\norm{f}_{L^2([\De])}\le1}\abs{f(a)}^2.\]
        \item $\mu$が\textbf{動径正測度}であるとは,$\nu_0\in \o{\RM_+([0,1])}$が存在して,$d\nu=d\nu_0(r)d\theta/2\pi$と表せることをいう.
    \end{enumerate}
\end{definition}

\begin{theorem}
    次の2条件は同値:
    \begin{enumerate}
        \item $H^2(\mu)\subset H(D)$.
        \item $\forall_{K\compsub\De}\;\sup_{a\in K}R(\mu,a)<\infty$.
    \end{enumerate}
\end{theorem}

\begin{theorem}
    $H^2_\gamma=H^2(\mu)$かつ,$\C[z]$上でノルムが等しいとする:
    \[\forall_{f\in\C[z]}\;\norm{f}^2_\gamma=\int_{[\De]}\abs{f(z)}^2d\mu\]
    このとき,ある$[\De]$上の動径正測度$\nu$が存在して,$H^2_\gamma=H^2(\nu)$かつ
    \[\forall_{f\in\C[z]}\;\norm{f}^2_\gamma=\int_{[\De]}\abs{f(z)}^2d\nu\]
\end{theorem}

\begin{corollary}
    $a\in\De$について,
    \begin{enumerate}
        \item $H^2=H^2(d\theta/2\pi)$.また,
        \[\norm{f}_{H^2}=\paren{\frac{1}{2\pi}\int^{2\pi}_0\abs{f_*(e^{i\theta})}^2d\theta}^{1/2},\quad R(d\theta/2\pi,a)^2=(1-\abs{a}^2)^{-1/2}.\]
        \item $L^2_a=H^2(rdrd\theta/\pi)$.また,
        \[\norm{f}_{L^2_a}=\paren{\frac{1}{\pi}\int^1_0\int^{2\pi}_0\abs{f(re^{i\theta})}^2rdrd\theta}^{1/2},\quad R(rdrd\theta/\pi,a)^2=(1-\abs{a}^2)^{-1}.\]
        \item $D^2=H^2(\mu)$かつ$\norm{f}_{D^2}=\paren{\int_{[\De]}\abs{f}^2d\mu}^{1/2}$となる$\mu$は存在しない.
    \end{enumerate}
\end{corollary}

\subsection{3空間の関係}

\begin{tcolorbox}[colframe=ForestGreen, colback=ForestGreen!10!white,breakable,colbacktitle=ForestGreen!40!white,coltitle=black,fonttitle=\bfseries\sffamily,
title=]
    $H^2(\De)$は,その境界値の空間$H^2(\partial\De)$と等長同型であり,$H^2(\partial\De)$は$L^2(\partial\De,d\theta/2\pi)$の部分空間と同一視できる.
\end{tcolorbox}

\begin{definition}
    $\H\subset H(\De)$を再生核Hilbert空間,$\C[z]$は$\H$上稠密とする.
    \begin{enumerate}
        \item $U:\H\to\H$を次のように定める:
        \[(U\phi)(\lambda)=\Phi(\lambda)=\paren{\phi\;\middle|\;\frac{1}{1-\o{\lambda}z}}_\H\]
        \item $\cK:=U(\H)$とおく.これを$\H$の\textbf{Hardy共役空間}という.
    \end{enumerate}
\end{definition}

\begin{proposition}\mbox{}
    \begin{enumerate}
        \item $U$はHardy空間$H^2$上の恒等作用素である.
        \item $U$は$L^2_a\to D^2$に等長作用素を定める:
        \[(U\phi)(\lambda)=\frac{1}{\lambda}\int^\lambda_0\phi(z)dz.\]
        \item $U$は$D^2\to L^2_a$に等長作用素を定める:
        \[(U\phi)(\lambda)=(z\phi)'(\lambda).\]
    \end{enumerate}
\end{proposition}

\begin{proposition}[境界値の空間の特徴付け]\mbox{}
    \begin{enumerate}
        \item $H^2\simeq_\Ban H^2(\partial\De)$が成り立つ.
        \item $H^2(\partial\De)=\Brace{f\in L^2(\partial\De,d\theta/2\pi)\;\middle|\;\forall_{n<0}\;\wh{f}(n):=\frac{1}{2\pi}\int^{2\pi}_0f_*(e^{i\theta})e^{-in\theta}d\theta=0}$.
    \end{enumerate}
\end{proposition}

\begin{definition}[Dirichlet integral]
    $f\in D^2$について,
    \begin{enumerate}
        \item 次を\textbf{Dirichlet積分}という:
        \[D(f):=\int^1_02rdr\int^{2\pi}_0\abs{f'(re^{i\theta})}^2\frac{d\theta}{2\pi}.\]
        \item 境界点$\zeta\in\partial\De$に対して,$f(\zeta)$が存在するとする.次を\textbf{局所Dirichlet積分}という:
        \[D_\zeta(f):=\int^{2\pi}_0\Abs{\frac{f_*(e^{it})-f(\zeta)}{e^{it}-\zeta}}^2\frac{dt}{2\pi}.\]
        存在しないときは$D_\zeta(f)=\infty$とする.
    \end{enumerate}
\end{definition}

\begin{theorem}[Douglasの公式]
    $f\in H^2$ならば,
    \[\int^{2\pi}_0D_{e^{i\theta}}(f)\frac{d\theta}{2\pi}=\int_\De\abs{f'(re^{i\theta})}^2\frac{rdrd\theta}{\pi}.\]
\end{theorem}

\section{Hardy空間}

\begin{tcolorbox}[colframe=ForestGreen, colback=ForestGreen!10!white,breakable,colbacktitle=ForestGreen!40!white,coltitle=black,fonttitle=\bfseries\sffamily,
title=]
    一般の$0<p\le\infty$について,Banach空間的な視点から再度みる.
\end{tcolorbox}

\subsection{定義}

\begin{definition}
    $p\in(0,\infty)$とする.
    \begin{enumerate}
        \item $f\in H(\De)$について,
        \[M_p(r,f):=\paren{\int^{2\pi}_0\abs{f(re^{i\theta})}^p\frac{d\theta}{2\pi}}^{1/p},\quad M_\infty(r,f):=\max_{\theta\in[0,2\pi]}\abs{f(re^{i\theta})}.\]
        \item $0<p\le\infty$について,\textbf{Hardy空間}を次のように定める:
        \[H^p(\De):=\Brace{f\in H(D)\mid\sup_{0\le r<1}M_p(r,f)<\infty}.\]
        $H^\infty\subset H^p\subset H^q\;(p>q)$の関係がある.
        \item 次を\textbf{Nevanlinna空間}という:
        \[N(\De):=\Brace{f\in H(D)\;\middle|\;\sup_{0\le r<1}\int^{2\pi}_0\log^+\abs{f(re^{i\theta})}\frac{d\theta}{2\pi}<\infty}.\]
        $\forall_{p\in(0,\infty)}\;H^p\subset N$である.
        \item 次の$\De$上調和な関数$P:[0,1)\times[0,2\pi]\to\R$を\textbf{Poisson核}という:
        \[P(r,\theta):=\frac{1-r^2}{1-2r\cos\theta+r^2}.\]
        \item $mu$を$\partial\De$上の実Borel測度とする.次を\textbf{Poisson-Stieltjes積分},特に$d\mu=F(e^{it})dt$の形のときPoisson積分という:
        \[\int^{2\pi}_0P(r,\theta-t)\frac{d\mu(t)}{2\pi}\quad(re^{i\theta}\in\De).\]
    \end{enumerate}
\end{definition}
\begin{remarks}
    $\De$上の実調和関数の全体を$h(\De)$とすると同様の論理展開ができて,$\Re H^p\subset h^p$が成り立つ.
\end{remarks}

\begin{theorem}[Poisson-Stieltjes積分による$h^1$の特徴付け]
    $u:\De\to\R$について,
    \begin{enumerate}
        \item $u\in h^1$.
        \item ある$\partial D$上の実Borel測度$\mu$が存在して,
        \[u(re^{i\theta})=\int^{2\pi}_0P(r,\theta-t)\frac{d\mu(t)}{2\pi}.\]
    \end{enumerate}
\end{theorem}

\begin{theorem}[(F. and R. Nevanlinna)]
    $f\in H(\De)$について,次は同値:
    \begin{enumerate}
        \item $f\in N$.
        \item $\exists_{h,k\in H^\infty}\;f=h/k$.
    \end{enumerate}
\end{theorem}

\section{Toeplitz作用素とHankel作用素}

\begin{tcolorbox}[colframe=ForestGreen, colback=ForestGreen!10!white,breakable,colbacktitle=ForestGreen!40!white,coltitle=black,fonttitle=\bfseries\sffamily,
title=]
    任意の対角線について同じ複素数が並ぶとき,Toeplitz行列という.
    Hardy空間上のToeplitz作用素の行列表示がそうなる.
    逆方向の対角線について同じ複素数が並ぶ行列をHankel行列という.
    Toeplitz行列については乗算が$O(n^2)$で計算できるため,工学で応用が深い.
\end{tcolorbox}

\subsection{Toeplitz作用素の定義}

\begin{assumption}\mbox{}
    \begin{enumerate}
        \item $\C[z]\subset\H\subset H(\De)$を再生核Hilbert空間とする.
        \item $a\in\De$について,$K_a=K(-,a)\in\H$をその再生核とすると,$K(-,-):\De^2\to\C$は自己共役である.
        \item $K_a$は$[\De]$上に連続延長可能とする.
        \item $k_a(z):=\frac{K_a(z)}{\norm{K_a}}$を\textbf{正規化された再生核}という.
        \item $\sigma\in\RM([\De];\C)$の\textbf{Berezin変換}とは,次で定まる$\wt{-}:\RM([\De];\C)\to H(\De)$をいう:
        \[\wt{\sigma}(a):=\brac{T_\sigma k_a,k_a}=\int_{[\De]}\abs{k_a(z)}^2d\sigma(z)\quad(a\in\De)\]
    \end{enumerate}
\end{assumption}

\begin{definition}
    $\sigma\in\RM([\De];\C)$を複素Borel測度とする.\textbf{$\sigma$をシンボルとするToeplitz作用素}とは,次で定まる$T_\sigma:\C[z]\to H(\De)$をいう:
    \[(T_\sigma f)(a)=\int_{[\De]}f(z)K(a,z)d\sigma(z)\quad(a\in\De).\]
\end{definition}

\subsection{Toeplitz作用素の性質}

\begin{lemma}\mbox{}
    \begin{enumerate}
        \item $T_\sigma$が$\H$上有界ならば,そのシンボルのBerezin変換$\wh{\sigma}$は$D$上で有界である.
        \item $\abs{a}\to1$のとき$k_a$は$\H$上$0$に弱収束するとする.このとき,$T_\sigma$が$\H$上コンパクトならば,そのシンボルのBerezin変換$\wh{\sigma}$は$\abs{a}\to1$で$0$に収束する.
    \end{enumerate}
\end{lemma}

\begin{theorem}
    $\H$が特別な空間の場合を考える.
    \begin{enumerate}
        \item $\H=H^2(\mu)=\oo{\C[z]}\subset L^2(\mu)$とする.次を満たすならば$T_\sigma$はコンパクト,特に有界である:
        \begin{enumerate}
            \item $\supp\sigma\compsub D$である.
        \end{enumerate}
        \item $\sigma=\phi d\mu\;(\phi\in L^\infty(\mu))$とする.次が成り立つ:
        \begin{enumerate}
            \item $\forall_{f\in\C[z]}\;T_\sigma fa=P\phi f\;\on\De$.
            \item $\norm{T_\sigma}\le\norm{\phi}_\infty$.
        \end{enumerate}
    \end{enumerate}
\end{theorem}

\begin{proposition}
    $a,b\in\C,\phi,\psi\in L^\infty(\mu)$について,
    \begin{enumerate}
        \item $T_{a\phi +b\psi}=aT_\phi+bT_\psi$.
        \item $T_{\o{\phi}}=T^*_\phi$.
        \item $\phi\ge0\ae\Rightarrow T_\phi\ge0$.
        \item $\phi\in H^2(\mu)$でもあるならば,$T_\psi T_\phi=T_{\psi\phi}$.
    \end{enumerate}
\end{proposition}

\subsection{Hankel作用素の定義}

\begin{definition}
    $H^2(\mu)\subset H(D)$を仮定する.$\phi\in L^2(\mu)$と,直交射影$P:L^2(\mu)\epi H^2(\mu)$とについて,
    \[H_\phi f=(I-P)(\phi f)\;(f\in\C[z])\]
    で定まる$H_\phi:\C[z]\to H(\De)$を\textbf{Hankel作用素}という.このとき,
    \[(H_\phi f)(a)=\int_{[\De]}f(z)(\phi(a)-\phi(z))K(a,z)d\mu(z)\]
    が成り立つ.
\end{definition}

\begin{proposition}
    $\phi\in L^\infty(\mu)$ならば,$H^*_\phi H_\phi=T_{\abs{\phi}^2}-T^*_\phi T_\phi$.
\end{proposition}

\section{定常確率過程}

\begin{tcolorbox}[colframe=ForestGreen, colback=ForestGreen!10!white,breakable,colbacktitle=ForestGreen!40!white,coltitle=black,fonttitle=\bfseries\sffamily,
title=]
    $L^2(\mu)$内での$\C[z]$の閉包である重み付きHardy空間$H^2(\mu)$の研究は,多項式近似と予測問題に関連する.
    列$\{z^n\}_{n\in\Z}$は$L^2(\mu)$内の定常確率過程である:$\brac{z^n,z^l}_\mu=\brac{1,z^{l-n}}_\mu$.
    実は,任意の定常確率過程が,ある確率測度$\mu\in P(\partial\De)$についての$L^2(\mu)$における定常確率過程$\{z^n\}_{n\in\Z}$と見ることができる!(Herglotzの定理).
    すると,現在$z^0=1$と$\o{\P_0}=\o{z}\o{\C[z]}$との距離は,過去から現在の予測誤差とみれる.
    多項式近似の問題をこのように予測理論に応用したのはWiener (1949)による.
\end{tcolorbox}

\begin{notation}
    $\P_n\subset\C[z]$で$n$次以下の多項式の全体を表す.
\end{notation}

\begin{definition}
    $\o{z}\o{H^2(\mu)},H^2(\mu)$をそれぞれ,過去と未来という.$z$を乗じて定数となっていくか,$z^{-1}$を乗じると定数となるかの違いである.
\end{definition}

\subsection{Szegoの第一予測定理}

\begin{tcolorbox}[colframe=ForestGreen, colback=ForestGreen!10!white,breakable,colbacktitle=ForestGreen!40!white,coltitle=black,fonttitle=\bfseries\sffamily,
title=]
    現在$z^0=1$と$\o{\P_0}=\o{z}\o{\C[z]}$との距離は,過去から現在の予測誤差とみれる.
    また,$H^2(\mu)$上の$\ev_0$のノルムの逆数の正確な値を与えているともみれる.
\end{tcolorbox}

\begin{theorem}
    $\mu\in\RM(\partial\Delta;\R_+)$に対して,Lebesgue分解を$d\mu=\frac{Wd\theta}{2\pi}+d\mu_s$とすると,
    \[\inf_{g\in\P_0}\int^{2\pi}_0\abs{1-g}^2d\mu=\int_{f\in\P_0}\int^{2\pi}_0\abs{1-g}^2\frac{Wd\theta}{2\pi}=\exp\int^{2\pi}_0\log W\frac{d\theta}{2\pi}.\]
\end{theorem}

\begin{corollary}
    $W\in L^1(d\theta/2\pi)$は非負値,かつ,$\log W\notin L^1(d\theta/2\pi)$を満たすならば,$H^2(Wd\theta/2\pi)=L^2(Wd\theta/2\pi)$である.
\end{corollary}



\subsection{Kolmogorovの第二予測定理}

\begin{tcolorbox}[colframe=ForestGreen, colback=ForestGreen!10!white,breakable,colbacktitle=ForestGreen!40!white,coltitle=black,fonttitle=\bfseries\sffamily,
title=]
    予測問題の立場から見ると,失われたデータに対する補間の誤差を与えている.
\end{tcolorbox}

\begin{theorem}
    $\mu\in\RM(\partial\Delta;\R_+)$に対して,Lebesgue分解を$d\mu=\frac{Wd\theta}{2\pi}+d\mu_s$とすると,
    \[\inf_{g+\o{f}\in\P_0+\o{\P_0}}\int^{2\pi}_0\abs{1-(g+\o{f})}^2d\mu=\paren{\int^{2\pi}_0W^{-1}\frac{d\theta}{2\pi}}^{-1}.\]
\end{theorem}

\section{半正定値関数論}

\begin{tcolorbox}[colframe=ForestGreen, colback=ForestGreen!10!white,breakable,colbacktitle=ForestGreen!40!white,coltitle=black,fonttitle=\bfseries\sffamily,
title=]
    半正定値性は接頭辞の"semi-"を落として,単に"positive"または"positive definite"とも呼ばれる.
    Radon積分を特徴付けたのも正性であった.
    Abel $*$-半群$(S,+,*)$上の半正定値関数論を考える.
    Laplace, Fourier変換,確率母関数はいずれも半正定値性を持つ.
    これは特定の半正定値性を持つ関数の積分変換であるから当然である.
    そしてその半正定値性を確率論の言葉で特徴付けるのがHoeffdingの不等式である.
    Bochner, Bernstein-Widder, Hamburgerの定理はそれぞれ,$(\R,+,x^*=-x),(\R_+,+,x^*=x),(\N,+,n^*=N)$に対する特殊化である.
\end{tcolorbox}

\subsection{定義と特徴付け}


\begin{tcolorbox}[colframe=ForestGreen, colback=ForestGreen!10!white,breakable,colbacktitle=ForestGreen!40!white,coltitle=black,fonttitle=\bfseries\sffamily,
    title=]
    Fourier解析をさらに一般のHilbert空間で考えるとき,調和解析という.\cite{吉田}
\end{tcolorbox}

\begin{definition}[positive definite function]
    関数$\varphi:\R\to\C$が\textbf{正の定符号関数}であるとは,
    \begin{enumerate}
        \item $t=0$において連続.
        \item $\forall_{t_i\in\R}\;\forall_{\xi_k\in\C}\;\forall_{n\in\N^+}\;\sum_{i,j=0}^n\varphi(t_i-t_j)\xi_i\o{\xi_j}\ge0$.
    \end{enumerate}
    を満たすことをいう.
\end{definition}

\begin{lemma}
    任意の正の定符号関数$\varphi$は,次の3条件を満たす:
    \begin{enumerate}
        \item $\varphi(0)\ge\abs{\varphi(t)}$.
        \item $\R$上で一様連続.
        \item 連続関数$h\in C(\R)$に対して,積分
        \[\int_\R\int_\R\varphi(s-t)h(s)\o{h(t)}dsdt\]
        が存在すれば,$\ge0\;\fe$
    \end{enumerate}
\end{lemma}

\begin{theorem}[Bochner (1932)\footnote{Vorlesungen uber Fouriersche Integrale, Leipzig}]
    正の定符号関数$\varphi$に対して,有界で単調増加かつ右連続な$v(\lambda)$が存在して,
    \[\varphi(t)=\int_\R e^{it\lambda}dv(\lambda)\]
    が成り立つ.この$v$はさらに条件$v(-\infty)=0$を課せば,一意に取れる.
\end{theorem}

\begin{theorem}[Khintchine]
    $\varphi\in C_b(\R;\C)$を有界連続関数であるとする.次の2条件は同値:
    \begin{enumerate}
        \item $\varphi$は正の定符号関数である.
        \item $\norm{f_n}_{L^2(\R)}^2\le\varphi(0)$を満たす列$\{f_n\}\subset L^2(\R)$が存在して,コンパクト一様収束$\lim_{n\to\infty}\int_\R f_n(t+s)\o{f_n(s)}ds=\varphi(t)$が成り立つ.
    \end{enumerate}
\end{theorem}

\section{再生核}

\begin{tcolorbox}[colframe=ForestGreen, colback=ForestGreen!10!white,breakable,colbacktitle=ForestGreen!40!white,coltitle=black,fonttitle=\bfseries\sffamily,
title=]
    積分方程式に関する研究から,振る舞いのよい積分核のクラスを確定させる研究が派生し,Moore, E. H.がこれに自覚的になり抽象的に扱って核の再生性を定義したところから,楕円型偏微分方程式への応用がみつかり,そこで再生核の理論が確立された.
    また,Cartan, E.やWeyl, Kreinの等質空間上の調和解析でも中心的役割を演じた.
    確率論では確率過程の共分散として現れる.
\end{tcolorbox}

\begin{history}\mbox{}
    \begin{enumerate}
        \item 正定値核は,第II種Fredholm積分方程式に関するHilbert (1904)の仕事を作用素論的に解釈したJames Mercer (1909)が定義した.
        正定値カーネルが次元を上げれば標準内積で表現出来るというMercerの定理が,現代の線型アルゴリズムを非線型アルゴリズムに変換するKernelトリックの基礎となっている(Aizerman 1964).
        \item 一方で,Mathias, M.とBochner, S.は正定値核を知らずに,独立に正定値関数を定義したと考えられる.
        \item 再生核:N. Aronszajnの標記理論\footnote{Aronszajn, N. (1950) \textit{Theory of reproducing kernels}. Trans. Amer. Math. Soc. 68: 337-404.}が,複素数関数論や偏微分方程式論に関連して有効性が認められてきたという(吉田1953\cite{吉田}).
        そもそも,超関数以外にもデルタ関数を積分表示で与える手法として提案されたものである.
    \end{enumerate}
\end{history}

\subsection{定義と存在}

\begin{definition}[reproducing kernel]
    $\Om\in\Set,K:\Om^2\to\bF$とHilbert空間$X\subset\Map(\Om,\bF)$について,
    \begin{enumerate}
        \item $K$が\textbf{$X$-再生核}であるとは,
        \begin{enumerate}[(a)]
            \item $\forall_{y\in\Om}\;K(-,y)\in X$.
            \item $\forall_{y\in\Om}\;\forall_{f\in X}\;f(y)=(f(x)|K(x,y))$.
        \end{enumerate}
    \end{enumerate}
\end{definition}

\begin{theorem}[存在と一意性]
    Hilbert空間$X\subset\Map(\Om,\bF)$について,次の2条件は同値:
    \begin{enumerate}
        \item $X$-再生核$K$が存在する.
        \item $\forall_{y_0\in\Om}\;\exists_{C\in\R}\;\forall_{f\in X}\;\abs{f(y_0)}\le C\norm{f}$.
    \end{enumerate}
    また,この条件が成り立つとき,再生核$K$は一意的に存在する.
\end{theorem}

\begin{corollary}
    Hilbert空間$X\subset\Map(\Om,\bF)$に再生核$K$が存在するとする.このとき,
    \begin{enumerate}
        \item $\max_{f\in\partial B}\abs{f(y_0)}=K(y_0,y_0)^{1/2}$.
        \item 等号成立条件は$f_0(x)=\rho\frac{K(x,y_0)}{K(y_0,y_0)^{1/2}}\;(\rho\in\partial\Delta)$のときに限る.
    \end{enumerate}
\end{corollary}

\subsection{半正定値性による特徴付け}

\begin{theorem}[半正定値性による特徴付け]
    $\Om\in\Set,K:\Om\times\Om\to\bF$について,次の2条件は同値.
    \begin{enumerate}
        \item $K$は半正定値である.
        \item あるHilbert空間$\F\subset\Map(\Om,\bF)$が存在して,$K$はその再生核である.
    \end{enumerate}
\end{theorem}

\subsection{再生核の表示}

\begin{theorem}
    $\F$を可分,$K$をその再生核とする.$\F$の任意の正規直交系$(\varphi_n)$に対して,
    \begin{enumerate}
        \item $\sum_{n\in\N}\varphi_n(x)\o{\varphi_n}(y)=K(x,y)$.
        \item $x$の関数として$K(x,y)=\lim_{n\to\infty}\sum_{i=1}^n\varphi_i(x)\o{\varphi_i(y)}$は各点収束する.
    \end{enumerate}
\end{theorem}

\subsection{Bergmanの核関数}

\begin{definition}
    $\Om\osub\C$を領域とすると,$H(\Om)\cap L^2(\Om)$は可分Hilbert空間をなし,再生核を持つ.
\end{definition}

\begin{theorem}
    $\Om\osub\C$を単連結領域,$f:E\to\Delta(0,r)$を双正則写像とする.
    \[f(z_0)=0,\quad\dd{f}{z}(z_0)=1\quad(z_0\in\Om)\]
    を満たすならば,次が成り立つ:
    \[f(z)=\frac{1}{K(z_0,z)}\int^z_{z_0}K(t,z_0)dt.\]
\end{theorem}

\subsection{Stoneの定理}

\begin{theorem}[Stone]
    ユニタリ作用素の系$\{U_t\}_{t\in\R}\subset\Aut(H)$が群の性質$U_tU_s=U_{t+s},U_0=I$と$H$上の各点収束に関する連続性$\lim_{t\to t_0}U_t=U_{t_0}$を満たすならば,単位の分解$(E(\lambda))$が存在して,
    \[U_t=\int^\infty_{-\infty}e^{it\lambda}dE(\lambda).\]
\end{theorem}
\begin{remarks}
    これはBochnerの定理から導くことも出来,逆もたどれる.よって,2つの主張は同等であると考えて良い\cite{吉田}.
\end{remarks}

\subsection{調和解析}

\begin{theorem}
    有界連続関数$f\in C_b(\R;\C)$が,ある右連続かつ有界変動な関数$v\in\BV(\R;\C)$を用いて
    \[f(t)=\int^\infty_{-\infty}e^{i\lambda t}dv(\lambda)\]
    と調和振動$e^{i\lambda t}$の畳み込みとして表されるための必要十分条件は,次が与える:
    \[\sup_{n\in\N}\int_\R\Abs{\int_\R\paren{\frac{\sin(t/n)}{t/n}}^2f(t)e^{-i\lambda t}dt}d\lambda.\]
\end{theorem}
\begin{remarks}
    絶対収束列$(a_n)\in l^1(\N;\C)$は,
    \[\sum_{n=\infty}^\infty a_ne^{i\lambda_nt}\]
    によって振動を定める.しかし,$(a_n)$が一様収束しかしなくても,上の条件を満たさない「振動」が定義出来る.
    このようなクラスを特徴付けるのが,H. Bohrの概周期関数である.
\end{remarks}

\section{概周期関数}

\subsection{定義と特徴付け}

\begin{definition}[almost periodic function]
    $f\in C(\R;\C)$と$\ep>0$について,
    \begin{enumerate}
        \item \textbf{$\ep$に属する概周期}とは,$P_\ep:=\Brace{\tau\in\R\mid\sup_{t\in\R}\abs{f(t+\tau)-f(t)}\le\ep}$をいう.
        \item $\forall_{\ep>0}\;\exists_{l>0}\;\forall_{\al\in\R}\;(\al,\al+l)\cap P_\ep\ne\emptyset$ならば,$f$を\textbf{概周期関数}という.
    \end{enumerate}
\end{definition}

\begin{theorem}
    概周期関数$f\in C(\R;\C)$について,
    \begin{enumerate}
        \item $f\in C_b(\R)$である.
        \item $f\in UC(\R)$である.
    \end{enumerate}
\end{theorem}

\begin{theorem}[Bochner-Favard]
    $f\in C(\R;\C)$について,次は同値:
    \begin{enumerate}
        \item $f$は概周期的である.
        \item 任意の$\{a_n\}\subset\R$について,関数列$\{f_n(t):=f(t+a_n)\}$を作ると,ある部分列が$\R$上一様収束する.
    \end{enumerate}
\end{theorem}

\subsection{Weierstrassの近似定理}

\begin{theorem}
    概周期関数$f$と任意の$\ep>0$について,ある三角多項式$P_\ep$が存在して,$\norm{f-P_\ep}<3\ep$.
\end{theorem}

\chapter{1次元のFourier変換}

\begin{quotation}
    局所コンパクトアーベル群$G$について,Banach代数$(L^1(G),*)$の(あるいはその単位化$L^1(G)+\C\delta_0$の)Gelfand変換$\Gamma:L^1(G)\to C(\wh{L^1(G)})$をFourier変換という.
    $G=S^1,\R$などの一次元多様体のとき,指標群は$\wh{L^1(S^1)}\simeq_{\Top\Grp}\Z,\wh{L^1(\R)}\simeq_{\Top\Grp}\R$となり,値域は$c_0(\Z),UC_0(\R)$となる.
    こうして,特殊関数としては指標$e_t:\R\to\C$としての指数関数$e_t(x)=e^{ixt}$しか出現しない.
    この場合の調和解析は古典調和解析,
    または(狭義の)Fourier解析と呼ばれる.

    $G=S^2,\Delta,\partial B$などの,
    より多次元の群上の調和解析に向けて,
    この古典的な設定を復習する.

    また,$S^1$は$SO(2)$の,$S^n$は$SO(n+1)$の等質空間である.
    ここに発展可能性がある.
\end{quotation}

\chapter{多次元のFourier解析}

\begin{quotation}
    球面は$SO(3)$の等質空間として,コンパクト型の対称Riemann空間の最も簡単な例であり,
    $SO(3)$の既約ユニタリ表現は全て有限次元である.
\end{quotation}

\section{歴史}

\subsection{興り}

\begin{tcolorbox}[colframe=ForestGreen, colback=ForestGreen!10!white,breakable,colbacktitle=ForestGreen!40!white,coltitle=black,fonttitle=\bfseries\sffamily,
title=]
    Fourier解析が構成出来る基礎群には,単位円$S^1$,実数$\R$,球面$S^2$,円板$\Delta$などがある.
    $S^2$上のFourier解析がBanach空間$X$の単位球面$\partial B$上のFourier解析に拡張できる.
    いずれも正則表現の規約分解という意味で表現論的に統一的にみれる.
\end{tcolorbox}

\begin{history}[Fourier (1811) の熱伝導の研究]
    現代では実解析での定式化がされるが,元来熱方程式を解くために生まれた理論である.
    ここから,Riemannが積分を定義した「関数の三角級数による表現可能性について」から,長き道のりが始まる.
    結局,$\Delta$上のFourier解析が,Laplacianのスペクトル分解を与えるために,容易に解ける,と函数解析的に解釈できる.
    しかしこの見方が明瞭になるのは,底空間への表現論を視野に入れたときである.
\end{history}

\begin{history}[KleinのErlangen program]
    幾何学とは,変換群の作用によって不変な幾何学的性質を調べることが目的である.
    Euclid幾何学は合同変換不変量,射影幾何学は射影変換不変量を調べる.
    Riemann多様体の射は発見に時間がかかったが,affine接続を通じて生まれる変換の不変量についての学問だと考え得る(例えば計量を不変にする変換).

    特に,変換群を調べるにあたって,これが作用する等質空間は標準的模型として重要だと考えられる.
    この作用は等質空間上の関数空間への作用・表現を引き起こし,この関数空間の不変量を調べるのは解析学である.
\end{history}

\subsection{表現論}

\begin{tcolorbox}[colframe=ForestGreen, colback=ForestGreen!10!white,breakable,colbacktitle=ForestGreen!40!white,coltitle=black,fonttitle=\bfseries\sffamily,
title=]
    \begin{enumerate}
        \item $S^1$上のFourier展開とは,等質空間$S^1$の正則表現を既約分解することである.
        \item $\R$の場合は既約な閉部分空間は存在しない,これによりFourier級数がFourier積分になる.
        \item $S^2$はRiemann計量から定まるLaplacianを持ち,これにスペクトル分解が,正則表現の規約分解を与える.
        \item $\Delta$上のFourier変換は,$\Delta$上の関数を$\partial\Delta$上の関数に対応させる写像であり,その逆変換が一般化されたPoisson積分と,スペクトルパラメータの積分とに分解される.こうして,$\Delta$上のFourier変換が与えるLaplacianのスペクトル分解を用いれば,熱方程式が容易に解ける.
    \end{enumerate}
\end{tcolorbox}

\begin{history}[表現論的立場]
    級数論,線型変換論,球関数展開などはここに扱われ「応用解析」などとまとめられて乱雑な印象がある.
    しかし,表現論まで立場を戻せば,指数関数の果たす中心的役割は,加法群から乗法群への連続な群準同型として特徴付けられ,
    Fourier解析の応用の多様性が初めて理解される.
    そして三角関数というのは,$S^1,\R$の群構造に引き起こされる特殊関数であり,$S^2$によって引き起こされる球関数はGaussの超幾何関数と関連が深い.
    いずれもユニタリ表現の成分として自然に現れる.
    ここで「解析学的自然」に突っ込んでいくことになる.
\end{history}

\begin{history}[表現論と函数解析が交差するとき]
    Poisson積分は,単位演習上の解析汎関数全体から単位円板上の調和関数(またはLaplacianの固有関数)全体への線型同型を与える\cite{岡本}.
\end{history}

\begin{history}[wavelet]
    $ax+b$群という位相群上の調和解析の枠組みでwavelet解析も理解できる.
    Fourier解析の,不確定性原理による縛りから逃れる理論として1980年以来注目を浴びている.
\end{history}

\chapter{参考文献}

\begin{thebibliography}{99}
    \bibitem{岡本・展望}
    岡本清郷 (1997) 『フーリエ解析の展望』(すうがくぶっくす,朝倉書店).
    \bibitem{岡本}
    岡本清郷 (1980) 『等質空間上の解析学』(紀伊國屋書店).
    \bibitem{小松}
    小松彦三郎 (1978) 『Fourier解析』(岩波講座,基礎数学).
    \bibitem{金子}
    金子晃 (19) 『定数係数線型偏微分方程式』(岩波講座,基礎数学).
    \bibitem{河添}
    河添健 (2000) 『群上の調和解析』(すうがくの風景,朝倉書店).
\end{thebibliography}

\end{document}