\documentclass[uplatex,dvipdfmx]{jsreport}
\title{Fourier解析}
\author{}
\pagestyle{headings} \setcounter{secnumdepth}{4}
\input{/Users/Hirofumi Shiba/NatureOfStatistics/preamble_no_fonts.tex}
%\input{/Users/hirofumi.shiba48/NatureOfStatistics/preamble_no_fonts.tex}
%\input{/Users/hirof/NatureOfStatistics/preamble_no_fonts.tex}
\usepackage[math]{anttor}
\begin{document}
\tableofcontents

\begin{quotation}
    Bochnar型の定理を理解したい.
    Bochnar型の定理とは,Fourier-Stieltjes変換の値域は,正定値性によって特徴付けられるという種のものである.
    \begin{theorem}
        $\varphi:\R\to\C$がある有界単調増加関数のFourier-Stieltjes変換であるための必要十分条件は,連続かつ正定値であることである.
    \end{theorem}
\end{quotation}

\chapter{$L^1(\T)$のFourier解析}

\begin{quotation}
    $\Z$の上の指標$\{e^{int}\}_{n\in\Z}=\Hom_\Grp(\Z,\C^*)$は$\bT$に同型である.
    $\F:L^1(\bT)\to l^\infty(\Z)$の像は$c_0(\Z)\subset l^\infty(\Z)$に収まり,縮小的な単射であるが,全射でない.
    $L^2(\bT)$への制限$\F:L^2(\bT)\to l^2(\Z)$は等長同型である.
\end{quotation}

\begin{notation}\mbox{}
    \begin{enumerate}
        \item $\bT:=\ocinterval{-\pi,\pi}$を$\R/(2\pi\Z)$と同一視する.測度空間としては$(\bT,\B(\bT),dx)$を考える.
        \item $\varphi_n=e_n:=e^{int}\in C(\bT)$とする.
        \item $\R$上とは違って,$\bT$はコンパクトであるから,常に測度は確率測度に規格化しておくとする.
        $\dx=\frac{dx}{2\pi}$を平均測度とし,$\dint dx:=\int\dx$を平均積分とする.
        \item 平均測度に関するLebesgue空間を$\dL^p(\bT):= L^p(\bT,\dx)$とする.
    \end{enumerate}
\end{notation}
\begin{remarks}[Fourier変換の逆としてのGelfand変換]
    $\bT$上のFourier変換は,$\SO_2(\Z)$の等質空間$\bT$上の正則表現を規約分解することである.
    \begin{enumerate}
        \item 両方向に総和可能な数列$l^1(\Z)$は畳み込みについて単位的な可換Banach代数をなす.
        これはGelfand変換$\Gamma:l^1(\Z)\mono C(\bT)$と埋め込まれるが,この像$\Gamma(l^1(\Z))$はFourier級数が絶対収束する$[0,2\pi]$上連続な周期関数に等しい.
    \end{enumerate}
\end{remarks}

\section{Fourier変換の定義と性質}

\subsection{Fourier係数の定義と性質}

\begin{tcolorbox}[colframe=ForestGreen, colback=ForestGreen!10!white,breakable,colbacktitle=ForestGreen!40!white,coltitle=black,fonttitle=\bfseries\sffamily,
title=]
    $L^2(T)$について,正規直交基底$\{e_n\}$の係数への対応がFourier変換$\F:L^2(T)\to l^\infty(\Z)$である.
    これは一般に$L^1(T)$上に定義できる.
\end{tcolorbox}

\begin{definition}\mbox{}
    \begin{enumerate}
        \item $\{e^{int}\}_{n\in\Z}\subset C(\bT)$の線型和で表される関数$P:=\sum_{n=-N}^Na_ne^{int}\in C(\bT)$を\textbf{$N$-次の三角多項式}という.
        \item 任意の$f\in L^1(\bT)$に関して,次の積分を\textbf{第$n$Fourier級数}という:
        \[\wh{f}(n):=\frac{1}{2\pi}\int^\pi_{-\pi}f(t)e^{-int}dt.\]
        \item $f\in L^1(\bT)$の\textbf{第$N$Fourier部分和}とは,次の関数$S_N(f)$をいう:
        \[S_N(f)(t):=\sum_{n=-N}^N\wh{f}(n)e^{int}.\]
        \item 次に対応$\F:L^1(\bT)\to l^\infty(\Z);f\mapsto(\wh{f}(n))_{n\in\Z}$を\textbf{Fourier変換}という.
    \end{enumerate}
\end{definition}

\begin{proposition}
    $f\in L^1(\bT)$について,
    \begin{enumerate}
        \item $f$が$\pi$も周期に持つとき,任意の奇数$n\in\Z$について$\wh{f}(n)=0$.
        \item $f$が実数値関数であることと,$\forall_{n\in\Z}\;\o{\wh{f}(n)}=\wh{f}(-n)$は同値.
    \end{enumerate}
\end{proposition}
\begin{Proof}\mbox{}
    \begin{enumerate}
        \item $\forall_{t\in\R}\;f(t)=f(t+\pi)$より,
        \[\wh{f}(n)=\dint_\bT f(t)e^{-int}dt=\dint_\bT f(t+\pi)e^{-int}dt=e^{in\pi}\wh{f}(n).\]
        これより$\wh{f}(n)(e^{in\pi}-1)$であるが,$n$が奇数のとき,$e^{in\pi}=-1\ne1$である.
        \item 一般に,
        \begin{align*}
            \wh{\o{f}}(n)&=\dint_\bT\o{f}(t)e^{-int}dt\\
            &=\dint_\bT\o{f(t)e^{int}}dt=\o{\wh{f}(-n)}.
        \end{align*}
        $f$が実数値関数のとき,さらにこれが$\wh{f}(n)$に等しい.
    \end{enumerate}
\end{Proof}

\subsection{単項式とその並行移動}

\begin{example}[恒等関数]\mbox{}
    \begin{enumerate}
        \item $f_1(t)=t=\pi-2\sum_{n=1}^\infty\frac{\sin(nt)}{n},\qquad (t\in(0,2\pi))$.
        \item $f_2(t)=t=2\sum_{n=1}^\infty\frac{(-1)^{n+1}}{n}\sin(nt),\qquad(t\in(-\pi,\pi))$.
        \item $f_1,f_2\in L^2(\bT)$に注目すると,この上でのFourier変換の等長性からBaselの等式
        \[\sum_{n=1}^\infty\frac{1}{n^2}=\frac{\pi^2}{6}\]
        を得る.
        \item $\wh{f_1},\wh{f_2}\in c_0(\Z)$はいずれも絶対収束しない.$\abs{f_i(n)}$はいずれも調和級数を含む.
    \end{enumerate}
    \begin{enumerate}[{注}1]
        \item これはそれぞれ,$f(0)=f(2\pi),f(-\pi)=f(\pi)$の値を適当に定めることで(どう定めても一意的に) $L^1(\bT)$の別々の元とみなせる.
        \item 上の等式は,$f_1,f_2$がそれぞれ開区間$(0,2\pi),(-\pi,\pi)$上で$C^1$-級であるから$S_N[f_i]$がこの上でコンパクト一様収束するために成り立つ.しかし,不連続点の周りでは収束が取り残される(Gibbs現象).
        \item 不連続点において,$f_1(0)=f_1(2\pi)=\frac{0+2\pi}{2}=\pi$と定めると,等式は任意の$t\in\bT$で成り立つ.
    \end{enumerate}
\end{example}
\begin{Proof}\mbox{}
    \begin{enumerate}
        \item Fourier変換は,$n\ne0$のとき部分積分より,
        \begin{align*}
            \wh{f}(n)&=\dint^{2\pi}_{0}e^{-int}tdt=\frac{1}{2\pi}\Square{-\frac{e^{-int}}{in}t}^{2\pi}_{0}+\frac{1}{2\pi}\int^{2\pi}_{0}\frac{e^{-int}}{in}dt\\
            &=-\frac{1}{in}=\frac{i}{n}.
        \end{align*}
        残る$n=0$の場合は,
        \[\wh{f}(0)=\dint^{2\pi}_0tdt=\frac{1}{2\pi}\SQuare{\frac{t^2}{2}}^{2\pi}_0=\pi.\]
        以上より,
        \begin{align*}
            S_\infty[f](t)&=\pi-\sum_{n\ne0}\frac{e^{int}}{in}\\
            &=\pi-\sum_{n=1}^\infty\frac{e^{int}-e^{-int}}{in}=\pi-2\sum_{n=1}^\infty\frac{\sin(nt)}{n}.
        \end{align*}
        \item なんと,$f(t)=t\;(t\in(-\pi,\pi))$と見ると結果が変わる!
        Fourier変換は,$n\ne0$のとき部分積分より,
        \begin{align*}
            \wh{f}(n)&=\dint^\pi_{-\pi}e^{-int}tdt=\frac{1}{2\pi}\Square{-\frac{e^{-int}}{in}t}^{\pi}_{-\pi}+\frac{1}{2\pi}\int^\pi_{-\pi}\frac{e^{-int}}{in}dt\\
            &=-\frac{1}{in}\cos(n\pi)=\frac{i}{n}(-1)^{n+1}.
        \end{align*}
        残る$n=0$の場合は,
        \[\wh{f}(0)=\dint^\pi_{-\pi}tdt=0.\]
        以上より,
        \begin{align*}
            S_\infty[f](t)&=\sum_{n\ne0}\frac{(-1)^{n+1}}{in}e^{int}\\
            &=\sum_{n=1}^\infty\frac{(-1)^{n+1}}{n}\frac{e^{int}-e^{-int}}{i}=2\sum_{n=1}^\infty\frac{(-1)^{n+1}}{n}\sin(nt).
        \end{align*}
        \item $f\in L^2(\bT)$より,いずれの場合も$\F(f)\in l^2(\Z)$ではある.よって$\F$の等長性から,
        \[\norm{f}^2_2=\frac{1}{2\pi}\int^{2\pi}_0t^2dt=\frac{4\pi^2}{3}\]
        \[\sum_{n\in\Z}\abs{\wh{f}(n)}^2=\pi^2+2\sum_{n=1}^\infty\frac{1}{n^2}\]
        より,$\sum_{n=1}^\infty\frac{1}{n^2}=\frac{\pi^2}{6}$を得る.
        \item いずれの場合も,$\F(f)\notin l^1(\Z)$である.実際,$f$に殆ど至る所等しい連続関数は存在しない.したがって,$S_N[f]$は,$f$に一様収束する訳ではない.
        また,$f(t)=t$は$\R$上絶対連続であるが,$\bT$上絶対連続ではないから,$S_N[f]$は$f$に一様収束しない.
        すなわち,次の式は各点で成り立つが,一様には成り立たない:
        \[S_N[f](t)=\pi+\sum_{n=1}^N\frac{i}{n}(e^{int}-e^{-int})=\pi-2\sum_{n=1}^N\frac{\sin nt}{n}\]
        より,
        \[\sum_{n=1}^\infty\frac{\sin nt}{n}=\frac{\pi-t}{2}\;\as\qquad\on\bT.\]
        \item しかしRiemannの局所性原理の系\ref{cor-Riemann-locality}より,$(0,2\pi)$上で$C^\infty$-級だから,ここではコンパクト一様収束するが,$t=0+0,2\pi-0$の近傍でのみ一様でないことがわかる.
    \end{enumerate}
\end{Proof}

\begin{example}[恒等関数の調整(saw function)]\label{exp-pi-t-2}
    上の観察から,
    \begin{enumerate}
        \item $f_1(t)=\frac{\pi-t}{2}=\sum_{n\in\N^+}\frac{\sin(nt)}{n}\qquad(t\in(0,2\pi))$.
        \item \[f_2(t)=\begin{cases}
            -\frac{\pi}{2}-\frac{x}{2}&-\pi\le x<0,\\
            0&x=0,\\
            \frac{\pi}{2}-\frac{x}{2}&0<x\le\pi.
        \end{cases}=\sum_{n=1}^\infty\frac{\sin(nt)}{n}\qquad(t\in(-\pi,\pi)).\]
        もまったく同様のFourier級数を持つ.
        \item $f_1,f_2$は$\bT$上では連続ではないが,$S_N[f]\to f$は$\bT$上の各点で起こる.
    \end{enumerate}
\end{example}
\begin{Proof}\mbox{}
    \begin{enumerate}
        \item  \[S_N[f](t)=\sum_{n=1}^N\frac{\sin nt}{n}\xrightarrow{N\to\infty}\sum_{n=1}^\infty\frac{\sin nt}{n}=\frac{\pi-t}{2}\quad\on\bT.\]
        しかし,$t=0+0$の近傍では一様収束しない.
        \item $f_1,f_2$は$L^1(\bT)$の元としては同一であるため.
        \item saw functionの不連続点は$f_2$では$t=0$に来ている.$f_2$の$t=0$では左右の極限と左右の微分係数を持つから,Diniの判定法より,$S_N[f_2](0)\to f_2(0)=0$を得る.
    \end{enumerate}
\end{Proof}
\begin{remarks}
    そもそも級数$\sum_{n=1}^\infty\frac{\sin(nt)}{n}$が任意の$t\in\R$について収束するということが非自明な事実である.
\end{remarks}

\begin{example}[$V$-関数]
    \[\tcboxmath{f(t)=\abs{t}=\frac{\pi}{2}-\frac{4}{\pi}\sum_{n\in\N^+}\frac{\cos(2n-1)t}{(2n-1)^2}=\frac{\pi}{2}-\frac{4}{\pi}\sum_{n:\text{正の奇数}}\frac{\cos(nt)}{n^2},\qquad(t\in[-\pi,\pi]).}\]
    は$f\in C(\bT)$を満たす例である.このことによって
    減衰速度は$O(n^{-2})$まで改善されているが,$t=0=2\pi$の近傍で可微分性はないので,
    やはり一様収束はしない.$f(0)=0$と併せて,
    \[\sum_{n:\text{正の奇数}}\frac{1}{n^2}=\frac{\pi^2}{8}\]
    を得る.
\end{example}
\begin{Proof}
    $f\in C(\bT)$は左右の微分係数をすべて持つので,$S_N[f]\to f$は各点収束はする.よって,$f(t)=S_\infty[f](t)$として上の等式を得るから,あとは$S_\infty[f]$を計算すれば良い.
    \begin{align*}
        \wh{f}(n)&=\frac{1}{2\pi}\paren{\int^\pi_0te^{-int}dt-\int^0_{-\pi}te^{-int}dt}\\
        &=\frac{1}{2\pi}\frac{1}{n^2}(e^{int}+e^{-int}-2)=\frac{1}{\pi n^2}(\cos(nt)-1).
    \end{align*}
    $\wh{f}(0)=\frac{\pi}{2}$と併せると,
    \[S_\infty[f](t)=\frac{\pi}{2}+\sum_{n:\text{奇数}}\frac{2}{\pi n^2}(e^{int}+e^{-int})=\frac{\pi}{2}+\frac{4}{\pi}\sum_{n:\text{奇数}}\frac{\cos(nt)}{n^2}.\]
\end{Proof}

\begin{example}[冪関数]
    \[\tcboxmath{f(t)=t^2=\frac{\pi^2}{3}+4\sum_{n=1}^\infty\frac{(-1)^n}{n^2}\cos(nt),\qquad(t\in[-\pi,\pi]).}\]
    も$f\in C(\bT)$であるが,
    $t=\pi$で可微分ではない.
    が,左右の微分係数を持つので,$t\in\bT$上で$S_N[f]\to f$は各点収束する.
    しかし実は,$\wh{f}(n)\in l^1(\Z)$から$S_N[f]\to f\;\In C(\bT)$を得る.
    特に,
    \[-\frac{\pi^2}{12}=\sum_{n\in\N^+}\frac{(-1)^n}{n^2},\quad\frac{\pi^2}{6}=\sum_{n\in\N^+}\frac{1}{n^2}.\]
\end{example}
\begin{Proof}\mbox{}
    \begin{enumerate}
        \item Fourier変換は,$n\ne0$のとき部分積分より,
        \begin{align*}
            \wh{f}(n)&=\int^\pi_{-\pi}e^{-int}t^2\dt=\frac{1}{2\pi}\Square{-\frac{e^{-int}}{in}t^2}^\pi_{-\pi}+\int^\pi_{-\pi}\frac{e^{-int}}{in}t\frac{dt}{\pi}\\
            &=\frac{\pi^2}{\pi in}(i\sin n\pi)+\frac{\pi}{\pi n^2}(e^{-in\pi}+e^{in\pi})-\frac{1}{\pi n^2}\int^\pi_{-\pi}e^{-int}dt\\
            &=\frac{2}{n^2}\cos(n\pi)=\frac{2}{n^2}(-1)^n.
        \end{align*}
        残る$n=0$の場合は
        \[\wh{f}(0)=\int^\pi_{-\pi}t^2\dt=\frac{\pi^2}{3}.\]
        \item 特に$\F(f)\in l^1(\Z)$より,命題\ref{thm-l1-series}から,$S_N[f]\to f$は一様収束である.
        よって,
        \[t^2=\frac{\pi^2}{3}+\sum_{n=1}^\infty\frac{2}{n^2}(-1)^2(e^{int}+e^{-int})=\frac{\pi^2}{3}+4\sum_{n\in\N^+}\frac{(-1)^n}{n^2}\cos(nt).\]
        \item 特に$t=0$の場合を見ると
        \[-\frac{\pi^2}{12}=\sum_{n=1}^\infty\frac{(-1)^n}{n^2}.\]
        $t=\pi$の場合を見ると
        \[\frac{\pi^2}{6}=\sum_{n=1}^\infty\frac{1}{n^2}.\]
    \end{enumerate}
\end{Proof}

\begin{example}[冪関数の修正]\mbox{}
    \begin{enumerate}
        \item $f(t)=\frac{(\pi-t)^2}{4}=\frac{\pi^2}{12}+\sum_{n=1}^\infty\frac{\cos(nt)}{n^2},\qquad(t\in[0,2\pi])$.
        \item $f(t)=\begin{cases}
            t(\pi-t),&t\in[0,\pi],\\
            -t(\pi-t),&t\in[-\pi,0]
        \end{cases}=\frac{8}{\pi}\sum_{n:\text{正の奇数}}\frac{\sin nt}{n^3}$.
    \end{enumerate}
    (2)から次の等式が得られる:
    \[\sum_{n=0}^\infty\frac{1}{(2n+1)^6}=\frac{\pi^6}{960},\quad\sum_{n=1}^\infty\frac{1}{n^6}=\frac{\pi^6}{945}.\]
\end{example}

\subsection{単関数の例}

\begin{example}[ジャンプ関数]
    $f(t):=1_{[0,\pi]}(t)\;(t\in[-\pi,\pi])$を考える.
    \begin{enumerate}
        \item $f*f(t)=t1_{[0,\pi]}$.
        \item $S_\infty[f](t)=\frac{1}{2}+\frac{2}{\pi}\sum_{n\in\N^+}\frac{\sin(2n-1)t}{2n-1}=\frac{1}{2}+\frac{2}{\pi}\sum_{n:\text{正の奇数}}\frac{\sin nt}{n}$.
    \end{enumerate}
\end{example}
\begin{Proof}\mbox{}
    \begin{enumerate}
        \item 畳み込み$f*f$は,連続な上昇関数になる:
        \[f*f(t)=\int^\pi_{-\pi}f(t-s)f(s)\frac{ds}{2\pi}=\frac{1}{2\pi}\int^\pi_01_{\Brace{t\ge s}}(s)ds=\begin{cases}
            t&t\in[0,\pi],\\
            0&t\in(-\pi,0).
        \end{cases}\]
        \item $\wh{f}$は
        \[\wh{f}(n)=\int^\pi_{-\pi}e^{-int}f(t)\dt=\int^\pi_0e^{-int}\dt=
        \begin{cases}
            \frac{i}{2\pi n}(e^{-in\pi}-1)&n\ne0,\\
            \frac{1}{2}&n=0.
        \end{cases}
        =\begin{cases}
            -\frac{i}{\pi n}&n\text{が奇数},\\
            0&n\ne0\text{が偶数},\\
            \frac{1}{2}&n=0.
        \end{cases}\]
        \item $\F$の等長性より,
        \[\norm{f}_2^2=\frac{1}{2}=\frac{1}{4}+2\sum_{n=1,3,5,\cdots}\frac{1}{\pi^2n^2}\]
        より,$n\ge1$が奇数のときの$\frac{1}{n^2}$のみを足していくと$\frac{\pi^2}{8}$を得ることになる.
    \end{enumerate}
\end{Proof}

\begin{example}[一様分布]
    $f(t):=1_{[a,b]}(t)\;(t\in\R,a<b)$を考える.
    \begin{enumerate}
        \item $S_\infty[f](t)=\frac{b-a}{2\pi}+\sum_{n\ne0}\frac{e^{-ina}-e^{-inb}}{2\pi in}e^{int}$は収束している.
        \item $a\ne-\pi$または$b\ne\pi$ならば,$S_N[f]$はいかなる$t\in\R$でも絶対収束しない.
    \end{enumerate}
\end{example}

\subsection{滑らかな例}

\begin{example}
    $\al\in\R\setminus\Z$について,
    \[\tcboxmath{f(t)=\frac{\pi}{\sin\pi\al}e^{i(\pi-t)\al}=\sum_{n\in\Z}\frac{e^{int}}{n+\al},\qquad(t\in[0,2\pi]).}\]
    特に,Parsevalの等式より,
    \[\sum_{n\in\Z}\frac{1}{(n+\al)^2}=\frac{\pi^2}{(\sin\pi\al)^2}.\]
\end{example}



\begin{example}[周期の合わない三角関数]
    $\al\in\R\setminus\Z$について,
    \[\tcboxmath{f(t)=\cos\al t=\frac{\sin\pi\al}{\pi\al}+\sum_{n=1}^\infty\frac{(-1)^n2\al\sin\pi\al}{\pi(\al^2-n^2)}\cos nt.}\]
    これは一様に絶対収束する.特に,次の等式が系として得られる:
    \[\frac{\pi}{\sin\pi\al}=\lim_{N\to\infty}\sum_{n=-N}^N\frac{1}{\al+n},\qquad \frac{\sin\pi x}{\pi x}=\prod^\infty_{n=1}\paren{1-\frac{x^2}{n^2}},\quad(x\in\R).\]
\end{example}
\begin{Proof}
    Fourier係数は$\wh{f}\in l^1(\Z)$を満たすから,$f\in C(\bT)$と合せて,任意の$t\in\bT$で等式が成り立つことに注意.
    Fourier係数は,$f$が偶関数であるために$\wh{f}(n)=\wh{f}(-n)$を満たすことに注意し,対称性に注意して計算すると,
    \begin{align*}
        \wh{f}(n)&=\frac{1}{2}(\wh{f}(n)+\wh{f}(-n))=\frac{1}{2\pi}\int^\pi_{-\pi}f(t)\frac{e^{-int}+e^{int}}{2}dt\\
        &=\frac{1}{2\pi}\int^\pi_{-\pi}\cos\al t\cos ntdt=\frac{1}{2\pi}\int^\pi_{-\pi}\frac{\cos(\al+n)t+\cos(\al-n)t}{2}dt\\
        &=\frac{1}{4\pi}\Square{\frac{\sin(\al+n)t}{\al+n}+\frac{2\sin(\al-n)\pi}{\al-n}}^\pi_{-\pi}=\frac{1}{2\pi}\paren{\frac{\sin(\al\pi)\cos(n\pi)}{\al+n}+\frac{\sin(\al\pi)\cos(n\pi)}{\al-n}}\\
        &=\frac{1}{2\pi}\frac{2\al\sin(\al\pi)\cos(n\pi)}{\al^2-n^2}=\frac{(-1)^n\al\sin(\al\pi)}{\pi(\al^2-n^2)}.
    \end{align*}
    これに$t=\pi$を代入すると,
    \[\cos\al\pi=\frac{\sin\pi\al}{\pi\al}+\sum_{n=1}^\infty\frac{(-1)^n2\al\sin\al\pi}{\pi(\al^2-n^2)}\cos n\pi\]
    の両辺に$\frac{\pi}{\sin\al\pi}$を乗じると,
    \begin{align*}
        \frac{\pi}{\tan\al\pi}&=\frac{1}{\al}+\sum_{n=1}^\infty\frac{2\al}{\al^2-n^2}=\frac{1}{\al}+\sum_{n=1}^\infty\paren{\frac{1}{\al-n}+\frac{1}{\al+n}}\\
        &=\lim_{N\to\infty}\sum_{n=-N}^N\frac{1}{\al+n}.
    \end{align*}
    さらに,この等式を
    \[\frac{\pi}{\tan\al\pi}-\frac{1}{\al}=\sum_{n=1}^\infty\frac{2\al}{\al^2-n^2}\]
    を,任意の$x\in(0,1)$について$\al$について積分して,
    \[\int^x_0\paren{\frac{\pi}{\tan\al\pi}-\frac{1}{\al}}d\al=\int^x_0\sum_{n=1}^\infty\frac{2\al}{\al^2-n^2}d\al.\]
    この計算は,
    \begin{align*}
        \LHS&=\SQuare{\log\abs{\sin\pi\al}-\log\abs{\al}}^x_0=\log\frac{\sin\pi x}{x}-\log\frac{\sin\pi 0}{0}=\log\frac{\sin\pi x}{\pi x}.\\
        \RHS&=\sum_{n=1}^\infty\SQuare{\log\abs{\al^2-n^2}}^x_0=\sum_{n=1}^\infty\Paren{\log\abs{x^2-n^2}-\log n^2}=\log\prod^\infty_{n=1}\Abs{1-\frac{x^2}{n^2}}.
    \end{align*}
    対数の中身を比較して,結論を得る.
    $x\in(0,1)$の他の値についても,同様に計算できる.
\end{Proof}


\begin{example}[三角関数の冪]
    $f(t):=\cos t\;(t\in[-\pi,\pi])$とする.
    \begin{enumerate}
        \item $f*f(t)=\frac{1}{2}\cos t$が成り立つ.したがって,$f^{n*}=2^{-n}f$.
        \item 次に$\wh{f}(n)=\frac{1}{2}1_{\Brace{\pm1}}$.
        \item $g(t):=(\cos t)^m$のFourier係数は,
        \[\wh{g}(n)=\frac{1}{2^m}\comb{m}{\abs{n}}1_{\abs{n}\le m}(n).\]
    \end{enumerate}
\end{example}
\begin{Proof}\mbox{}
    \begin{enumerate}
        \item \begin{align*}
            f*f(t)&=\frac{1}{2\pi}\int^\pi_{-\pi}f(t-s)f(s)ds\\
            &=\frac{1}{2\pi}\int^\pi_{-\pi}\cos(t-s)\cos(s)ds\\
            &=\frac{1}{2\pi}\int^\pi_{-\pi}\frac{1}{2}\Paren{\cos t+\cos(t-2s)}ds
            =\frac{\cos t}{2}.
        \end{align*}
        \item \begin{align*}
            \wh{f}(n)&=\frac{1}{2\pi}\int^\pi_{-\pi}\cos te^{-int}dt\\
            &=\frac{1}{2\pi}\int^\pi_{-\pi}\frac{e^{it}+e^{-it}}{2}e^{-int}dt=\frac{1}{2}(1_{\pm1})
        \end{align*}
        \item \begin{align*}
            \wh{g}(n)&=\frac{1}{2\pi}\int^\pi_{-\pi}(\cos t)^me^{-int}dt\\
            &=\frac{1}{2\pi}\int^\pi_{-\pi}\paren{\frac{e^{it}+e^{-it}}{2}}^me^{-int}dt.
        \end{align*}
    \end{enumerate}
\end{Proof}

\begin{example}[単位円周上の複素積分との繋がり]\mbox{}
    \begin{enumerate}
        \item $0$でのLaurent展開が$g(z)=\sum_{n\in\Z}a_nz^n$である$0$に特異点をもち得る正則関数$g$を用いて
        $f(t):=g(e^{it})$と表せるとする.このとき,
        \[\wh{f}(n)=a_n.\]
        \item $f(t):=\cos(e^{it})$について,
    \end{enumerate}
\end{example}


\begin{example}[局所的な比較]
    \[f(t):=\Abs{\sin\frac{t}{2}}\]
    と,$t=0$の近傍で$g(t)=\abs{t}$が成り立つ,$g\in C^\infty(\bT\setminus\{0\})$について,
    \begin{enumerate}
        \item $f$のFourier係数は,
        \[\wh{f}(n)=\frac{2}{\pi}\frac{1}{4n^2-1},\qquad n\ne0,\qquad \wh{f}(0)=\frac{2}{\pi}.\]
        \item $\wh{g}(n)=O(n^{-2})\;(\abs{n}\to0)$.
    \end{enumerate}
\end{example}
\begin{Proof}\mbox{}
    \begin{enumerate}
        \item \begin{align*}
            \wh{f}(n)&=\frac{1}{2\pi}\int^\pi_{-\pi}\abs{\sin t/2}e^{-int}dt\\
            &=\frac{1}{2\pi}\frac{1}{2i}\paren{\int^\pi_0(e^{it/2}-e^{-it/2})e^{-int}dt-\int^0_{-\pi}(e^{it/2}-e^{-it/2})e^{-int}dt}\\
            &=\frac{1}{\pi}\frac{2}{4n^2-1}.
        \end{align*}
        \item $g-2f$は$t=0$でも$C^2$-級であるから,全体として$\wh{g-2f}(n)=O(n^{-2})$.$\wh{f}(n)=O(n^{-2})$と併せて結論を得る.
    \end{enumerate}
\end{Proof}

\subsection{畳み込みに関する関手性}

\begin{tcolorbox}[colframe=ForestGreen, colback=ForestGreen!10!white,breakable,colbacktitle=ForestGreen!40!white,coltitle=black,fonttitle=\bfseries\sffamily,
title=]
    $(L^1(T),*)$は可換環の構造も持つから,Banach空間の構造と併せて,Banach代数をなす.
\end{tcolorbox}

\begin{proposition}[畳み込みが積をなす]
    $\forall_{f,g\in L^1(\bT)}\;\forall_{n\in\Z}\;\wh{f*g}(n)=\wh{f}(n)\wh{g}(n)$.すなわち,次の図式は可換:
    \[\xymatrix{
        L^1(\bT)\times L^1(\bT)\ar[r]^-{\F\times\F}\ar[d]_-{*}&l^\infty(\Z)\times l^\infty(\Z)\ar[d]^-{\cdot}\\
        L^1(\bT)\ar[r]^-{\F}&l^\infty(\Z).
    }\]
\end{proposition}
\begin{Proof}
    任意の$f,g\in L^1(T),n\in\Z$について,
    \begin{align*}
        \wh{(f*g)}(n)&=\int_Te^{-inz}(f*g)(z)\frac{dz}{2\pi}\\
        &=\int_T e^{-inz}\int_Tf(z-y)g(y)\frac{dydz}{(2\pi)^2}\\
        &=\int_T\int_Te^{-inx}f(x)e^{-iny}g(y)\frac{dxdy}{(2\pi)^2}
    \end{align*}
    と,$z-y=:x$という変数変換により書ける.最後の式は,Fubiniの定理より,$\wh{f}(n)\wh{g}(n)$に他ならない.
\end{Proof}

\begin{example}[Fourier部分和はDirichlet核との畳み込みである]
    係数列$(a_n)_{-N\le n\le N}$が最も自明な$N$次の三角多項式$D_N\in L^1(\bT)$を$\bT$上の\textbf{Dirichlet核}という:
    \[D_N(u):=\sum_{n=-N}^Ne^{inu}.\]
    すると,
    \begin{enumerate}
        \item $(D_N*f)(t)=S_N(f)(t)\in L^1(\bT)$の関係がある.
        実際,
        \begin{align*}
            (D_N*f)(u)&=\int_T\sum_{n=-N}^Ne^{in(u-t)}f(t)\frac{dt}{2\pi}=\sum_{n=-N}^Ne^{inu}\wh{f}(n).
        \end{align*}
        \item Fourier部分和を再びFourier変換すると,Fourier係数への対応に戻る:
        $\wh{D_N*f}(n)=\wh{f}(n)1_{\abs{n}\le N}$.
        実際,
        \begin{align*}
            \F(D_N*f)(m)&=\int_Te^{-imt}(D_N*f)(t)\frac{dt}{2\pi}\\
            &=\int_Te^{-imt}\sum_{n=-N}^N\wh{f}(n)e^{int}\dt
            =\sum_{n=-N}^N\wh{f}(n)\int e^{i(n-m)t}\dt=\wh{f}(m).
        \end{align*}
    \end{enumerate}
\end{example}

\begin{proposition}[三角多項式との畳み込みは三角多項式である]
    $P(t)=\sum_{j=-n}^na_je^{ijt}$を$n$次の三角多項式とする.このとき,任意の$f\in L^1(\bT)$について,
    \[(P*f)(t)=\sum_{j=-n}^na_j\wh{f}(j)e^{ijt}.\]
\end{proposition}
\begin{Proof}
    簡単な計算による.
\end{Proof}

\begin{proposition}[冪零元はただ一つである]\mbox{}
    \begin{enumerate}
        \item $f\in L^1(\bT)$が$f*f=f$を満たすならば$f=0\;\ae$である.
        \item $f\in L^1(\R)$が$f*f=f$を満たすならば$f=0\;\ae$である.
    \end{enumerate}
\end{proposition}
\begin{Proof}\mbox{}
    \begin{enumerate}
        \item 任意の$n\in\Z$について$(\wh{f}(n))^2=\wh{f}(n)$より,$\wh{f}(n)\in\{0,1\}$.
        よって,係数が$1$のみからなる三角多項式が条件を満たす.
        \item $f\in UC_0(\wh{\R})$より,$\wh{f}=0$が必要だから,$f=0\;\ae$である.
    \end{enumerate}
\end{Proof}

\subsection{Fourier変換の連続性}

\begin{tcolorbox}[colframe=ForestGreen, colback=ForestGreen!10!white,breakable,colbacktitle=ForestGreen!40!white,coltitle=black,fonttitle=\bfseries\sffamily,
title=]
    $\F:L^1(\bT)\to c_0(\Z)$はノルム減少的であるため,有界線型作用素を定める.
\end{tcolorbox}

\begin{proposition}[\cite{Stein-Shakarchi-03-Fourier} Exercise 2.11]
    $f_k\to f\;\In L^1(\bT)$のとき,$\wh{f_k}\to f\;\In c_0(\Z)$は$n\in\Z$に関して一様に起こる.
\end{proposition}

\subsection{Fourier級数の急減少性と可微分性の対応}

\begin{tcolorbox}[colframe=ForestGreen, colback=ForestGreen!10!white,breakable,colbacktitle=ForestGreen!40!white,coltitle=black,fonttitle=\bfseries\sffamily,
title=]
    次の極めて示唆的な定理により,$f\in L^1(\bT)$の滑らかさは,$\wh{f}(n)$の$\abs{n}\to\infty$における減衰の速さと相関していることが示唆される.
    これの$L^1(\R)$上での対応物を考えることにより,Schwartzの超関数の概念が自然に得られる.
    $C^\infty(\bT)$は$C_c^\infty(\R)$に対応するのだ.
\end{tcolorbox}

\begin{definition}[rapidly decreasing sequence]
    数列$(a_n)_{n\in\Z}$が\textbf{急減少}であるとは,$\forall_{k\in\N}\;a_n=o(\abs{n}^{-k})\;(\abs{n}\to\infty)$が成り立つことをいう.
\end{definition}

\begin{theorem}[急減少性による可微分関数の特徴付け]
    $f\in L^1(\bT)$について,次は同値:
    \begin{enumerate}
        \item $\exists_{g\in C^\infty(\bT)}\;f=g\;\ae\on\bT$.
        \item $\wh{f}$は急減少である.
    \end{enumerate}
\end{theorem}
\begin{Proof}\mbox{}
    \begin{description}
        \item[(1)$\Rightarrow$(2)] 系\ref{cor-for-Fourier-series-to-absolutely-convergent}と同様の発想で,部分積分により$\forall_{k\in\N}\;\wh{f}(n)=\frac{1}{(in)^k}\wh{f^{(k)}}(n)$.
        $\wh{f^{(k)}}(n)$は有界であるから,特に$\wh{f}(n)=o(\abs{n}^{-k})$.
        \item[(2)$\Rightarrow$(1)] $\wh{f}\in l^1(\Z)$であるから,Fourier部分和は$\bT$上一様収束する\ref{thm-l1-series}.この極限関数を$g(t):=\sum_{n\in\Z}\wh{f}(n)e^{int}$とすると,$g\in C^\infty(\bT)$を示せばよい.
        ここで,Fourier部分和の微分$S_N^{(k)}[f](t)=\sum_{n=-N}^N(in)^k\wh{f}(n)e^{int}$は,$((in)^k\wh{f}(n))_{n\in\Z}$がやはり絶対収束することより,一様収束するから,極限と微分の交換=高別微分が任意階数可能で,$g\in C^\infty(\bT)$.
    \end{description}
\end{Proof}
\begin{remarks}
    $\bT$上における急減少関数の空間は$\S(\bT)=C^\infty(\bT)$である.
\end{remarks}

\begin{corollary}
    $f\in L^1(\bT)$について,(1)$\Rightarrow$(2)が成り立つ.
    \begin{enumerate}
        \item $\exists{g\in C^k(\bT)}\;f=g\;\ae\on\bT$.
        \item $\wh{f}(n)=o(\abs{n}^{-k})$.
    \end{enumerate}
    任意の$m<k$について,$f$は$m$階微分可能で$f^{(m)}\in L^2(\bT)$を満たし,$m:=k-2$について$f\in C^{k}(\bT)$であり,これが最良の結果である\cite{Katznelson-Fourier} Exercise 4.2.
\end{corollary}
\begin{Proof}\mbox{}
    \begin{description}
        \item[(1)$\Rightarrow$(2)] 系\ref{cor-for-Fourier-series-to-absolutely-convergent}でやったように,部分積分を繰り返す.
        \item[(2)$\Rightarrow$(1)] 任意の$k-m>\frac{1}{2}$について,$f$は$m$階微分可能で$f^{(m)}\in L^2(\bT)$を満たすことは示せる\cite{Katznelson-Fourier} Exercise 5.5. 
    \end{description}
\end{Proof}

\begin{remarks}[Bernsteinの定理]
    $f$が$\al>1/2$について,$\al$-次Holder連続ならば,$\wh{f}\in l^1(\Z)$となる\cite{Stein-Shakarchi-03-Fourier} Section 2.3,\cite{Katznelson-Fourier} Th'm 6.3.

\end{remarks}

\begin{theorem}[\cite{Katznelson-Fourier} Exercise 4.3]
    $f\in C^\infty(\bT)$について,次は同値:
    \begin{enumerate}
        \item $f$は実解析的.
        \item Gervey類1である:ある$R\in\R$が存在して,
        \[\sup_{t\in\bT}\abs{f^{(n)}(t)}\le n!R^n,\qquad n\in\N^+.\]
        \item 指数減少である:ある$K,a>0$が存在して,
        \[\abs{\wh{f}(n)}\le Ke^{-a\abs{n}},\qquad n\in\Z.\]
    \end{enumerate}
\end{theorem}

\subsection{Fourier変換の非全射性:Fourier変換像の減衰速度に関する必要条件}

\begin{tcolorbox}[colframe=ForestGreen, colback=ForestGreen!10!white,breakable,colbacktitle=ForestGreen!40!white,coltitle=black,fonttitle=\bfseries\sffamily,
title=対数速度で減少する数列はFourier係数列たり得ない]
    Fourier級数の減衰速度に注目すれば,$\F:L^1(\bT)\to c_0(\Z)$が全射でないことも判る.
    しかしだからと言って終域を$c_0(\Z)$からさらに絞ることは出来ない.
    任意の$(b_n)\in c_0(\Z)$に対して,それよりも遅く収束するようなFourier級数が必ず存在してしまう\ref{cor-image-of-Fourier-transform}.
\end{tcolorbox}

\begin{theorem}[Fourier級数であるための必要条件]
    $f\in L^1(\bT)$に対して,$(\wh{f}(n))_{n\in\Z}$は次が必要:
    \[\sum_{0\ne n\in\Z}\frac{\wh{f}(n)}{n}<\infty.\]
\end{theorem}
\begin{Proof}\mbox{}
    \begin{enumerate}
        \item $\wh{f}(0)=0$を仮定して収束を示せば,一般の場合も$(f-\wh{f}(0))+\wh{f}(0)$と見ることで,すぐに収束が従うことがわかる.
        \item このとき,
        \[\dint_\bT f(t)dt=\dint_\bT f(t)e^{i0t}dt=\wh{f}(0)=0\]
        が成り立つから,原始関数を$F(t):=\dint^t_0f(s)ds$とおくと,これは$\bT$上の絶対連続関数となる:$F\in\AC(\bT)$.
        \item よって定理\ref{thm-uniform-convergence-of-Fourier-series}から,$S_N[F]\to F\;\In C(\bT)$が成り立つ.
        特に$S_N[F](0)\to F(0)=0$である.
        が,$F$のFourier係数は,部分積分\ref{lemma-partial-integral-law}を用いて
        \begin{align*}
            \forall_{n\ne0}\wh{F}(n)&=\int_\bT e^{-int}F(t)\dt\\
            &=\underbrace{\Square{-\frac{e^{-int}}{in}F(t)}^\pi_{-\pi}}_{=0}+\int_{\bT}\frac{e^{-int}}{in}f(t)\dt=-i\frac{\wh{f}(n)}{n}.
        \end{align*}
        という関係があるから,
        \[S_N[F](0)=\sum_{n=-N}^N\wh{F}(n)e^{in0}=\wh{F}(0)-i\sum_{1\le\abs{n}\le N}\frac{\wh{f}(n)}{n}\xrightarrow{N\to\infty}0\]
        を含意している.
    \end{enumerate}
\end{Proof}

\begin{example}
    $(b_n)_{n\in\Z}$を$b_n:=\frac{1}{\log n}\;(n\ge2),b_n=0\;(n\le 1)$とすると,$(b_n)\in c_0(\Z)$であるが,
    \[\sum_{n=2}^\infty\frac{b_n}{n}=\sum_{n=2}^\infty\frac{1}{n\log n}=\infty.\]
    よって,$(b_n)\in c_0(\Z)$はFourier変換の像に入らない.
\end{example}
\begin{remarks}
    このように,$\Im\F$を捉えることは難しいが,$\bT$上の複素Borel速度に枠組みを広げ,そのFourier変換を考えれば,きれいな対応がつくようになり,
    その先のユニタリ作用素のスペクトル分解定理につながる.
    また$\Im\F\subset c_0(\Z)$をより狭い空間としてとらえたいかもしれないが,
    これは出来ない.任意の元$(\ep_n)\in c_0(\Z)_+$に対して,これより遅く収束するFourier係数列を持つ$f\in C(\bT)$が存在する\ref{thm-image-of-Fourier-transform}.
\end{remarks}

\subsection{Fourier係数は任意に遅く減少させられる}

\begin{tcolorbox}[colframe=ForestGreen, colback=ForestGreen!10!white,breakable,colbacktitle=ForestGreen!40!white,coltitle=black,fonttitle=\bfseries\sffamily,
title=]
    任意の収束正数列$(\ep_n)\in c_0(\Z;\R^+)$に対して,$(\wh{f}_n)>(\ep_n)$を満たす$f\in L^1(\bT)$と,$\frac{\wh{f}(n)}{\ep_n}\to\infty$を満たす$f\in C(\bT)$とが存在する.
\end{tcolorbox}

\begin{theorem}[任意に遅く収束する凸数列の構成算譜]
    非負数列$\{\ep_n\}_{n\in\N}\subset\R_+$は$0$に収束するとする.このとき,凸数列$(a_n)_{n\in\N}\in c_0(\N)$が存在して,$\forall_{n\in\N}\;a_n>\ep_n$.
\end{theorem}

\begin{corollary}[任意に遅く収束するFourier係数を持つ偶関数]\label{cor-image-of-Fourier-transform}
    $\{\ep_n\}_{n\in\N}\subset\R_+$は$0$に収束するとする.このとき,偶関数$f\in L^1(\bT)$が存在して,$\forall_{n\in\N}\;\wh{f}(n)=\wh{f}(-n)>\ep_n$.
\end{corollary}
\begin{Proof}
    定理に従って,より遅く収束する凸数列$(a_n)\in c_0(\N)$を取る.
    これが定めるコサイン級数は,ある可積分関数$f\in L^1(\bT)$を定め,$\wh{f}(n)=a_n/2$を満たす\ref{thm-convergent-cosine-series}.
\end{Proof}

\begin{theorem}\label{thm-image-of-Fourier-transform}
    $0$に収束する正数列$\{\ep_n\}_{n\in\Z}$に対して,ある$f\in C(\bT)$が存在して$(\wh{f}(n)/\ep_n)_{n\in\Z}$は非有界になる.
\end{theorem}
\begin{Proof}
    一様有界性原理による.線型作用素$T_n:C(\bT)\to\C$を$T_nf:=\wh{f}(n)/\ep_n$で定めると,$\norm{T_n}=1/\ep_n$より非有界であることによる.
\end{Proof}

\begin{lemma}
    $X$をBanach空間,$Y$をノルム空間,$\{T_\lambda\}_{\lambda\in\Lambda}\subset B(X,Y)$を族とする.
    $\forall_{x\in X}\;\sup_{\lambda\in\Lambda}\norm{T_\lambda x}$ならば,$\sup_{\lambda\in\Lambda}\norm{T_\lambda}$.
\end{lemma}
\begin{corollary}
    列$\{T_n\}\subset B(X,Y)$の作用素ノルムが非有界ならば,ある点$x\in X$が存在して$\{T_nx\}\subset Y$は収束しない.
\end{corollary}

\section{収束しないFourier級数の例}

\begin{tcolorbox}[colframe=ForestGreen, colback=ForestGreen!10!white,breakable,colbacktitle=ForestGreen!40!white,coltitle=black,fonttitle=\bfseries\sffamily,
title=]
    Fourier部分和$S_N[f]$の算術平均$\sigma_N[f]$は$f\in L^p(\bT)\;(1\le p)$について$\sigma_N[f]\to f\;\In L^p(\bT)$かつ
    $f\in C(\bT)$について$\sigma_N[f]\to f\;\In C(\bT)$.
    しかし$S_N[f]$自体は$f\in\AC(\bT)$について$S_N[f]\to f\;\In C(\bT)$しか成り立たない.
    \begin{enumerate}
        \item 当然$S_N[f]\to f\;\In L^1(\bT)$には反例がある\ref{thm-function-with-L1-diverging-Fourier-series}.これは畳み込み作用素$f\mapsto S_N$が$L^1(\bT)$-有界でないため当然である.
        \item $f\in C(\bT)$なだけでは$\lim_{N\to\infty}S_N[f](t)$が存在するとは限らないが,反例はdu Bois Reymondが与える\ref{thm-du-Bois-Reymond}.
    \end{enumerate}
    ただし,$f\in L^2(\bT)$ならば$S_N[f]\to f\;\ae$である(Carleson).
\end{tcolorbox}

\subsection{Dirichlet核のノルム}

\begin{tcolorbox}[colframe=ForestGreen, colback=ForestGreen!10!white,breakable,colbacktitle=ForestGreen!40!white,coltitle=black,fonttitle=\bfseries\sffamily,
title=]
    $\{D_n\}\subset L^1(\bT)$は評価$\norm{D_N}_1=O(\log N)$より有界ではなく,
    $D_N*-:L^1(\bT)\to L^1(\bT)$は非有界作用素である.
    これが,$D_N*f$が必ずしも$f$に$L^1(\bT)$-収束しない本質的な要因となる.
    さらに,$\exists_{g\in\bT}\;(D_N*g)(x)\to\infty$を満たす$g\in C(\bT)$も構成出来る.
\end{tcolorbox}

\begin{observation}
    そもそも,等比数列の和としての表式
    \[D_N(t)=\sum_{n=-N}^Ne^{int}=\frac{\sin\paren{N+\frac{1}{2}}t}{\sin\frac{t}{2}}\]
    の分母に評価$\forall_{t\in\R_+}\;\sin(t/2)\le t/2$を用いると,$\frac{2\sin\paren{N+\frac{1}{2}}t}{t}$で抑えられてしまい,これは$L^1$-非有界であることを予感させる.
\end{observation}

\begin{proposition}[Dirichlet核は$L^1$-非有界]\mbox{}\label{prop-norm-of-Dirichlet-kernel}
    \begin{enumerate}
        \item $\forall_{N\in\N}\;\norm{D_N}_1\ge(4/\pi^2)\log N$.
        特に,$\{D_n\}\subset L^1(\bT)$は有界ではない.
        \item $\exists_{C>0}\;\forall_{N\in\N}\;\norm{D_N}_1\le(4/\pi^2)\log N+C$.
    \end{enumerate}
\end{proposition}
\begin{Proof}\mbox{}
    \begin{enumerate}
        \item 評価$\forall_{t\in\R_+}\;\sin(t/2)\le t/2$と,変数変換$x=\paren{N+\frac{1}{2}}t$と積分$\int^{(n+1)\pi}_{n\pi}\abs{\sin x}dx=2$より,次のように評価できる:
        \begin{align*}
            \norm{D_N}_1&=\frac{1}{2\pi}\int^\pi_{-\pi}\Abs{\frac{\sin\paren{N+\frac{1}{2}}t}{\sin\frac{t}{2}}}dt=\frac{1}{\pi}\int^\pi_{0}\Abs{\frac{\sin\paren{N+\frac{1}{2}}t}{\sin\frac{t}{2}}}dt\\
            &\ge\frac{2}{\pi}\int^\pi_{0}\frac{\Abs{\sin\paren{N+\frac{1}{2}}t}}{t}dt=\frac{2}{\pi} \int^{(N+1/2)\pi}_0 \frac{\abs{\sin x}}{x}dx\\
            &=\frac{2}{\pi}\paren{\sum_{n=0}^{N-1}\int^{(n+1)\pi}_{n\pi}\frac{\abs{\sin x}}{x}dx+\int^{(N+1/2)\pi}_{N\pi}\frac{\abs{\sin x}}{x}dx}\\
            &\ge\frac{2}{\pi}\sum_{n=0}^{N-1}\frac{1}{(n+1)\pi}\int^{(n+1)\pi}_{n\pi}\abs{\sin x}dx=\frac{4}{\pi^2}\sum_{n=1}^N\frac{1}{n}\ge\frac{4}{\pi^2}\log N.
        \end{align*}
        \item 関数$\frac{1}{\sin\frac{t}{2}}-\frac{2}{t}$は次の考察より$\ocinterval{0,\pi}$上で有界である.
        \[\frac{1}{\sin\frac{t}{2}}-\frac{2}{t}=\frac{2}{t}\paren{\frac{1}{1-\frac{1}{6}\paren{\frac{t}{2}}-^2+O(t^4)}-1}=\frac{2}{t}\paren{\frac{1}{6}\paren{\frac{t}{2}}^2+O(t^4)}=O(t),\qquad(t\to0).\]
        よって,次のように評価できる:
        \begin{align*}
        \norm{D_N}_1&=\frac{1}{2\pi}\int^\pi_{-\pi}\Abs{\frac{\sin\paren{N+\frac{1}{2}}t}{\sin\frac{t}{2}}}dt=\frac{1}{\pi}\int^\pi_{0}\Abs{\sin\paren{N+\frac{1}{2}}t}\paren{\frac{1}{\sin\frac{t}{2}}-\frac{2}{t}+\frac{2}{t}}dt\\
        &\le\frac{2}{\pi}\int^\pi_0\frac{\Abs{\sin\paren{N+\frac{1}{2}}t}}{t}dt+\underbrace{\frac{1}{\pi}\int^\pi_0\Abs{\sin\paren{N+\frac{1}{2}}t}\paren{\frac{1}{\sin\frac{t}{2}}-\frac{2}{t}}dt}_{\le C_1}\\
        &\le\frac{2}{\pi}\sum_{n=1}^N\int^{\frac{n}{N+\frac{1}{2}}\pi}_{\frac{n-1}{N+\frac{1}{2}}\pi}\frac{\Abs{\sin\paren{N+\frac{1}{2}}t}}{t}dt+\underbrace{\frac{2}{\pi}\int^{\pi}_{\frac{N}{N+\frac{1}{2}}\pi}\frac{\Abs{\sin\paren{N+\frac{1}{2}}t}}{t}dt}_{\le C_2-C_1}+C_1\\
        &\le\frac{2}{\pi}\sum_{n=1}^N\frac{1}{\frac{n-1}{N+\frac{1}{2}}\pi}\underbrace{\int^{\frac{n}{N+\frac{1}{2}}\pi}_{\frac{n-1}{N+\frac{1}{2}}\pi}\sin\paren{N+\frac{1}{2}}tdt}_{=\frac{2}{N+\frac{1}{2}}}+C_2\\
        &\le\frac{4}{\pi^2}\sum_{n=1}^N\frac{1}{n-1}+C_2\le\frac{4}{\pi^2}\log N+C_2.
        \end{align*}
    \end{enumerate}
\end{Proof}

\begin{proposition}[畳み込み作用素のノルムは$L^1$-ノルムに等しい]\label{prop-operator-norm-of-convolution}
    任意の$f\in L^1(\bT)$に対して,これが定める畳み込み作用素$f*:L^1(\bT)\to L^1(\bT)$のノルムは$\norm{f}_{L^1(\bT)}$に等しい.
\end{proposition}
\begin{Proof}
    Youngの不等式より,$\norm{f*g}\le\norm{f}\norm{g}$から,$\norm{f*}\le\norm{f}_{L^1(\bT)}$である.
    しかし,特に$\{K_N\}\subset L^1(\bT)$はすべて$L^1(\bT)$の単位円周上の元である\ref{thm-Fejer-kernel-on-T}が,
    ここでの値を考えると,$\norm{K_N*f}_{L^1(\bT)}\to\norm{f}_{L^1(\bT)}$であるから,$\norm{f}_{L^1(\bT)}$は近似的に達成される.
    よって,作用素ノルムは$\norm{f}_{L^1(\bT)}$に等しい.
\end{Proof}

\subsection{畳み込み作用素のノルム}

\begin{proposition}[一方で$L^2$-有界]
    一方で,作用素$S_N:=D_N*-:L^2(\bT)\to L^2(\bT)$は作用素ノルムが$1$で近似的単位元をなす.
\end{proposition}

\begin{proposition}[原点でのFourier係数を対応させる作用素の作用素ノルムもDirichlet核のノルムに一致する]
    $T_N:C(\bT)\to\C$を
    \[T_N[f]:=S_N[f](0)=\frac{1}{2\pi}\int_\bT D_N(-t)f(t)dt\]
    で定めると,$\norm{T_N}=\norm{D_N}_1$.
\end{proposition}

\subsection{Fourier級数が$L^1$-収束しない関数の構成}

\begin{tcolorbox}[colframe=ForestGreen, colback=ForestGreen!10!white,breakable,colbacktitle=ForestGreen!40!white,coltitle=black,fonttitle=\bfseries\sffamily,
title=Fejer核の級数として構成する]
    双対的な消息として,$f$に$L^\infty$-収束をしない$S_N[f]$が存在する.
\end{tcolorbox}

\begin{theorem}\label{thm-function-with-L1-diverging-Fourier-series}
    増大列$\{n_k\}\subset\N$が十分速く増加するとき,
    \[f(t):=\sum_{k=1}^\infty\frac{1}{2^k}K_{n_k}(t)\quad\in L^1(\bT)\]
    のFourier部分和$S_N[f]=D_N*f$は$L^1(\bT)$上非有界になる.
    特に,$S_N[f]$は$f$に$L^1$-収束しない.
\end{theorem}

\begin{remarks}[Riesz]
    実は,任意の$f\in L^p(\bT)\;(1<p<\infty)$については,$S_N[f]\to f$が$L^p(\bT)$の意味で成り立つ.
\end{remarks}

\subsection{Fourier級数がある点で収束しない関数の構成}

\begin{tcolorbox}[colframe=ForestGreen, colback=ForestGreen!10!white,breakable,colbacktitle=ForestGreen!40!white,coltitle=black,fonttitle=\bfseries\sffamily,
title=]
    歴史的には,$f\in C(\bT)$の存在はP. du Bois Reymond (1876)が,$f\in L^1(\bT)$の存在はKolmogorovが示した.
    Luzin (1913)の予想はCarleson (1966)が示した:任意の$f\in L^2(\bT)$についてFourier級数はa.s.の意味で収束する.
\end{tcolorbox}

\begin{theorem}\mbox{}\label{thm-du-Bois-Reymond}
    \begin{enumerate}
        \item 任意の$M>0$に対して,三角多項式$p$と自然数$n$であって,$\norm{p}_\infty\le 1,\abs{S_n[p](0)}>M$を満たすものが存在する.
        \item 次の関数$f\in C(\bT)$のFourier部分和について,$\abs{S_n[f](0)}$は非有界である.特に,$0$において収束しない:
        \[f(t):=\sum_{k\in\N^+}\frac{1}{2^k}q_k(t),\quad q_k(t)=e^{iL_kt}p_k(t),L_k=L_k+\deg(p_k)+1+\deg({p_{k+1}}),L_1=0,\abs{S_{n_k}[p_k](0)}>2^{k}k.\]
    \end{enumerate}
\end{theorem}

\begin{theorem}[対称性を破ることによる構成\cite{Stein-Shakarchi-03-Fourier} 2.2節]
    $N\in\N^+$について,次数$N$の三角多項式
    \[f_N(t):=\sum_{1\le\abs{n}\le N}\frac{e^{int}}{n},\quad \wt{f}_N(t):=\sum_{-N\le n\le -1}\frac{e^{int}}{n},\qquad(t\in[-\pi,\pi]).\]
    を考える.
    \begin{enumerate}
        \item $\abs{\wt{f}_N(0)}\ge c\log N$.
        \item $F_N(\theta)$は$N\in\N^+,t\in[-\pi,\pi]$に関して一様に有界.
    \end{enumerate}
    この周波数を変える:
    \[P_N(t):=e^{i(2N)t}f_N(t),\quad \wt{P}_N(t):=e^{i(2N)t}\wt{f}_N(t).\]
    そして収束する正項級数$\sum_{k\in\N}\al_k$と次を満たす正数列$\{N_k\}\subset\N$をとる:
    \begin{enumerate}[(i)]
        \item $N_{k+1}>3N_k$.
        \item $\al_k\log N_k\to\infty$.
    \end{enumerate}
    これについて,
    \[f(t):=\sum_{k=1}^\infty\al_kP_{N_k}(t).\]
    とすると,これは$C(\bT)$の元を定める.これについて,$S_N[f](0)$は収束しない.
\end{theorem}

\begin{theorem}[Kolmogorov]
    ある$f\in L^1(\bT)$が存在して,殆ど至るところFourier級数が発散する.
\end{theorem}

\section{総和法と積分核}

\begin{tcolorbox}[colframe=ForestGreen, colback=ForestGreen!10!white,breakable,colbacktitle=ForestGreen!40!white,coltitle=black,fonttitle=\bfseries\sffamily,
title=Dirichlet核を飼い慣らす試み]
    $D_N(t)=\frac{\sin(N+1/2)t}{\sin(t/2)}$は激しく振動し,非負ではなく,$\dint_TD_N(t)dt=1$にも拘らず$\dint_T\abs{D_N(t)}dt\xrightarrow{N\to\infty}\infty$.
    \begin{enumerate}
        \item 連続関数のFourier級数は,Cesaro総和可能かつAbel総和可能である!
        \item Fourier部分和の算術平均を取ることで,Fejer核$K_N$は非負になり,一様な評価も可能となっている.
        \item Fourier部分和をある整級数の$r=1$における値と見ることで計算可能になる.これはFourier係数を係数とする整級数は$(-1,1)$上で一様に絶対収束するためである\ref{prop-Abel-Poisson-sum}.
    \end{enumerate}
    いずれもFourier係数の減衰性を速める試みであり,$K_n$は線型減衰,$P_r$は幾何減衰をかけている:
    \[K_N(t)=\sum_{n=-N}^N\paren{1-\frac{\abs{n}}{N+1}}e^{int},\quad P_r(t)=\lim_{N\to\infty}\sum_{n=-N}^Nr^{\abs{n}}e^{int}.\]
\end{tcolorbox}

\subsection{Cesaro総和法}

\begin{tcolorbox}[colframe=ForestGreen, colback=ForestGreen!10!white,breakable,colbacktitle=ForestGreen!40!white,coltitle=black,fonttitle=\bfseries\sffamily,
title=]
    級数の収束について,Cesaro和の方法が有用である.
    ここで,Dirichlet核$D_N\in L^1(\bT)$のCesaro和を考える,という応用があり得る.
    これをFejer核という.
\end{tcolorbox}

\begin{lemma}[Cesaro和]
    列$(a_n)$について,
    \[S_N:=\sum_{n=1}^Na_n,\quad\sigma_N:=\frac{1}{N}\sum_{n=1}^NS_n\]
    とする.$\exists_{S\in\R}\;S_N\to S$ならば$\sigma_N\to S$だが,逆は成り立たない.
\end{lemma}
\begin{Proof}
    $\exists_{S\in\R}\;S_N\to S$のとき,$\forall_{\ep>0}\;\exists_{N'\in\N}\;\forall_{N\ge N'}\;\abs{S-S_N}<\ep$より,
    \begin{align*}
        \abs{\sigma_N-S}&\le\frac{1}{N}\sum_{n=1}^N\abs{S-S_N}\\
        &<\frac{1}{N}\sum_{n=1}^{N'}\abs{S-S_n}+\paren{1-\frac{N'}{N}}\ep\xrightarrow{N\to\infty}\ep.
    \end{align*}
\end{Proof}

\begin{theorem}[Fejer核との畳み込みとしてのCesaro和]\label{thm-Fejer-kernel-on-T}
    \[\sigma_N[f]:=\frac{1}{N+1}\sum_{n=0}^NS_n[f]\in L^1(\bT)\]
    を\textbf{Fourier Cesaro和}という.
    \[K_N(t):=\frac{1}{N+1}\sum_{n=0}^ND_n(t)=\sum_{n=-N}^N\paren{1-\frac{\abs{n}}{N+1}}e^{int}.\]
    を\textbf{Fejer核}という.
    \begin{enumerate}
        \item 畳み込み:$\sigma_N[f]=K_N*f\in L^1(\bT)$.
        \item 等比数列の和公式:$K_N(t)=\frac{1}{N+1}\paren{\frac{\sin\frac{N+1}{2}t}{\sin\frac{t}{2}}}^2$.特に,$K_N\ge0$.
        \item $K_N\dt$は$\bT$上の確率測度を定める:$\int_\bT K_N(t)\dt=1$.
        \item $\forall_{\delta\in(0,\pi)}\;\sup_{\delta\le\abs{t}\le\pi}K_N(t)\xrightarrow{N\to\infty}0$.
    \end{enumerate}
\end{theorem}
\begin{Proof}\mbox{}
    \begin{enumerate}\setcounter{enumi}{-1}
        \item まず,$K_N(t)$のシグマを1つに整理し直す.
        \begin{align*}
            K_N(t)&=\frac{1}{N+1}\sum_{n=0}^N\sum_{k=-n}^ne^{ikt}\\
            &=\sum_{n=-N}^N\frac{1}{N+1}(N-\abs{n}+1)e^{int}=\sum_{n=-N}^N\paren{1-\frac{\abs{n}}{N+1}}e^{int}.
        \end{align*}
        第一行の式で$e^{int}$という形の項が何回出現するかに注目すると,添え字$k$の動く範囲の最大値が
        $\abs{n}\le\abs{k}\le N$である間出現するから,$N-\abs{n}+1$回だとわかるのである.
        \item $K_N$は$D_N$から定義されているから,
        \begin{align*}
            (K_N*f)(u)&=\int_TK_N(u-t)f(t)\dt\\
            &=\int_T\frac{1}{N+1}\paren{\sum_{n=0}^ND_N(u-t)}f(t)\dt\\
            &=\frac{1}{N+1}\sum_{n=0}^N\int_TD_N(u-t)f(t)\dt=\frac{1}{N+1}\sum_{n=0}^NS[f](u).
        \end{align*}
        \item まず,等比数列の和と見ることで,
        \[D_N(t)=\sum_{n=-N}^Ne^{int}=\frac{\sin\paren{N+\frac{1}{2}}t}{\sin\paren{\frac{1}{2}t}}\]
        がわかる.実際,
        \begin{align*}
            \sum_{n=-N}^Ne^{int}&=e^{-iNt}\frac{1-e^{it(2N+1)}}{1-e^{it}}\times\frac{e^{-it/2}}{e^{-it/2}}\\
            &=\frac{e^{-it(N+1/2)}-e^{it(N+1/2)}}{e^{-it/2}-e^{it/2}}.
        \end{align*}
        続いて,これらを足し合わせると,
        \begin{align*}
            \sum_{n=0}^ND_n(t)&=\sum_{n=0}^N\frac{\sin\paren{n+\frac{1}{2}}t}{\sin(t/2)}\\
            &=\frac{1}{\sin(t/2)}\sum_{n=0}^N\sin\paren{n+\frac{1}{2}}t\\
            &=\frac{1}{(\sin(t/2))^2}\sum_{n=0}^N\sin\frac{t}{2}\sin\paren{n+\frac{1}{2}}t\\
            &=\frac{1}{2(\sin(t/2))^2}\sum_{n=0}^N(\cos nt-\cos(n+1)t)\\
            &=\frac{1}{(\sin(t/2))^2}\frac{1-\cos(N+1)t}{2}=\paren{\frac{\sin((N+1)t/2)}{\sin(t/2)}}^2.
        \end{align*}
        以上より,
        \[K_N(t)=\frac{1}{N+1}\sum_{n=0}^ND_n(t)=\frac{1}{N+1}\paren{\frac{\sin((N+1)t/2)}{\sin(t/2)}}^2.\]
        \item そもそも$D_N(t)$の$\bT$上の積分は,$e^{int}+e^{-int}=2\cos nt$より,
        \begin{align*}
            \int_TD_N(t)\dt&=\int_Te^{0}\dt+\sum_{n=1}^N\int_T(e^{int}+e^{-int})\dt
            =1+0=1.
        \end{align*}
        よってその算術平均である$K_N$の積分も$1$である.
        \item (2)の結果を用いて,次のように評価できる:
        \[(0\le)K_n(t)=\frac{1}{N+1}\paren{\frac{\sin\frac{N+1}{2}t}{\sin\frac{t}{2}}}^2\le\frac{1}{N+1}\frac{1}{(\sin t/2)^2}\le\frac{1}{N+1}\frac{1}{(\sin\delta/2)^2},\quad(t\in[-\pi,-\delta]\cup[\delta,\pi])\]
        よって,この上では一様に$0$に収束する. 
    \end{enumerate}
\end{Proof}
\begin{remarks}[平均によってDirichlet核を飼い慣らす]
    $D_N(t)=\frac{\sin(N+1/2)t}{\sin(t/2)}$は激しく振動し,非負ではなく,$\int_TD_N(t)\dt=1$にも拘らず$\int_T\abs{D_N(t)}\dt\xrightarrow{N\to\infty}\infty$.
    その原因は正負に激しく振動するためで,$N\to\infty$の極限で周波数が無限大になりためである.
    Cesaro和を取る過程で,
    \[\sum_{n=0}^ND_n(t)=\paren{\frac{\sin((N+1)t/2)}{\sin(t/2)}}^2.\]
    が成り立ち,そのような性質がなくなってふるまいがよくなるのである.
\end{remarks}

\subsection{総和核は$C(\bT)$の近似的単位元である}

\begin{tcolorbox}[colframe=ForestGreen, colback=ForestGreen!10!white,breakable,colbacktitle=ForestGreen!40!white,coltitle=black,fonttitle=\bfseries\sffamily,
title=]
    正の総和核とは,$0$に集中していく可積分な密度関数をいう.
\end{tcolorbox}

\begin{definition}[summability kernel]
    列$\{k_n\}\subset L^1(\bT)$であって次の3条件を満たすものを\textbf{総和核}または\textbf{近似的単位元}\cite{Stein-Shakarchi-03-Fourier}という:
    \begin{enumerate}[{[S}1{]}]
        \item 規格化:$\forall_{n\in\N}\;\dint_Tk_n(t)dt=1$.
        \item $L^1(\bT)$-有界:$\sup_{n\in\N}\norm{k_n}_1<\infty$.
        \item $\bT\setminus\{0\}$上で一様に消える:$\forall_{\delta\in(0,\pi)}\;\int_{\delta\le\abs{t}\le\pi}k_n(t)\dt\to0$.
    \end{enumerate}
    明らかにFejer核は総和核である.さらに正である.Dirichlet核は(1),(3)は満たす近似的単位元であるが,(2)を満たさない.
\end{definition}

\begin{theorem}[総和核は$C(\bT)$の近似的単位元である]\label{thm-kernel-is-approximate-unit-in-C}
    $\{k_n\}\subset L^1(\bT)$を総和核とする.このとき,$\forall_{f\in C(\bT)}\;k_n*f\in C(\bT)$で,$f\in C(\bT)$に一様ノルムについて収束する:
    \[\forall_{f\in C(\bT)}\;k_n*f\xrightarrow{n\to\infty}f\in C(\bT).\]
\end{theorem}
\begin{Proof}
    $k_n*f$の連続性は表示
    \[(k_n*f)(t)=\dint_Tf(t-s)k_n(s)ds\]
    から明らか.
    任意の$f\in C(T),t\in T$を取ると,
    \begin{align*}
        \forall_{\delta\in(0,\pi)}\;\abs{(k_n*f)(t)-f(t)}&=\Abs{\frac{1}{2\pi}\int_Tk_n(t-s)f(s)ds-f(t)}\\
        &=\frac{1}{2\pi}\Abs{\int_T(f(t-s)-f(t))k_n(s)ds}\\
        &\le\frac{1}{2\pi}\paren{\int^\delta_{-\delta}+\int_{\delta\le\abs{s}\le\pi}}\abs{f(t-s)-f(t)}\abs{k_n(s)}ds
    \end{align*}
    と表示できる.ここで,任意の$\ep>0$に対して,ある$\delta\in(0,\pi),n\in\N$が存在して右辺が$\ep$で抑えられることを論じる.
    \begin{enumerate}
        \item $f$は$\bT$上一様連続だから,$\exists_{\delta\in(0,\pi)}\;\forall_{t\in T}\;\forall_{s\in(-\delta,\delta)}\;\abs{f(t-s)-f(t)}\abs{k_n(s)}ds$.
        \item 総和核の性質(3)より,$\exists_{N\in\N}\;\forall_{n\ge N}\;\int_{\delta\le\abs{t}\le\pi}\abs{k_n(t)}\dt<\ep$.
    \end{enumerate}
    以上より,右辺は
    \begin{align*}
        &<\frac{1}{2\pi}\int^\delta_{-\delta}\ep\abs{k_n(s)}ds+\ep\sup_{\delta\le\abs{s}\le\pi}\abs{f(t-s)-f(t)}.
    \end{align*}
    これは有界な項と$\ep$の積より,$\ep\to0$の極限を考えることで$t\in T$に依らずに評価ができたことになる.
\end{Proof}

\begin{corollary}[Weierstrassの多項式近似]\label{cor-Weierstrass}
    $[a,b]\subset\R$をコンパクト集合とする.
    \begin{enumerate}
        \item $C(\bT)$の任意の元は三角多項式で一様近似できる.
        \item $C([a,b])$の任意の元は多項式で一様近似できる.
    \end{enumerate}
\end{corollary}
\begin{Proof}\mbox{}
    \begin{enumerate}
        \item Fejer核は総和核だから$\forall_{f\in C(\bT)}\;K_N*f\to f$であるが,$K_N*f$はFourier部分和の算術平均で,三角多項式である.
        \item \begin{enumerate}
            \item 次の対応$T$はBanの同型である:
            \[\xymatrix@R-2pc{
                T:C([a,b])\ar[r]&C([0,1])\\
                \rotatebox[origin=c]{90}{$\in$}&\rotatebox[origin=c]{90}{$\in$}\\
                f(t)\ar@{|->}[r]&f((b-a)t+a).
            }\]
            よって,$C([0,1])$について多項式が稠密であることを示す.
            \item $i:C([0,1])\mono C([0,2\pi])$を,$f(2\pi):=f(0)$とし,それ以外を線型に補間する対応とすると,これはBanの埋め込みである.
            \item $i(f)$は三角多項式で近似できる.Taylor展開を考えれば,$i(f)$は多項式で近似できる.$C([0,1])$に制限しても同様である.
        \end{enumerate}
    \end{enumerate}
\end{Proof}

\subsection{総和核は$L^p(\bT)$の近似的単位元である}

\begin{theorem}[総和核は$L^p(\bT)$の近似的単位元である]\label{thm-kernel-is-approximate-unit-in-Lp}
    $\{k_n\}\subset L^1(\bT)$を総和核とする.このとき,$\forall_{1\le p\in\R}\;\forall_{f\in L^p(\bT)}\;k_n*f\in L^p(\bT)$で,$f\in L^p(\bT)$に$L^p$-ノルムについて収束する:
    \[\forall_{f\in C(\bT)}\;k_n*f\xrightarrow{n\to\infty}f\in C(\bT).\]
\end{theorem}
\begin{Proof}
    任意の$f\in L^p(\bT)$を取る.
    \begin{enumerate}
        \item $C(T)$は$L^p(\bT)$で稠密だから,$\forall_{\ep>0}\;\exists_{g\in C(\bT)}\;\norm{f-g}_p<\ep$.
        \item 定理\ref{thm-kernel-is-approximate-unit-in-C}より,$\exists_{N\in\N}\;\forall_{n\ge N}\;\norm{k_n*g-g}_\infty<\ep$.
    \end{enumerate}
    以上より,
    \begin{align*}
        \forall_{n\ge N}\quad\norm{k_n*f-f}_p&=\norm{k_n*f-k_n*g}_p+\norm{k_n*g-g}p+\norm{f-g}_p\\
        &\le\norm{k_n}_1\norm{f-g}_p+\norm{k_n*g-g}_\infty+\norm{f-g}_p
        <\paren{\sup_{n\in\N}\norm{k_n}_1+2}\ep.
    \end{align*}
\end{Proof}
\begin{remarks}
    これは$C(\bT)\mono L^\infty(\bT)$の結果のみを用いて示しているが,
    途中で$\norm{k*g-g}_p\le\norm{k*g-g}_\infty$を使ったので,$\bT$特有の議論だと言うべきであろう.
    $\R$上では単関数近似とHolderの不等式を用いて直接議論する\ref{thm-property-of-summerbility-kernel-in-R}.
\end{remarks}

\begin{corollary}[Riemann-Lebesgue]\mbox{}\label{cor-Riemann-Lebesgue}
    \begin{enumerate}
        \item 任意の$L^p(\bT)\;(1\le p<\infty)$の元は三角多項式で$L^p$-近似可能である.
        \item $\F:L^1(\bT)\to l^\infty(\bT)$は単射である:$\Ker\F=0$.
        \item $\F:L^1(\bT)\to l^\infty(\bT)$はノルム減少的である.
        \item $\Im\F\subset c_0(\Z):=l^\infty(\Z)\cap C_0(\Z)$.
    \end{enumerate}
\end{corollary}
\begin{Proof}\mbox{}
    \begin{enumerate}
        \item Fejer核は総和核だから$\forall_{f\in L^p(\bT)}\;K_N*f\to f$であるが,$K_N*f$はFourier部分和の算術平均で,三角多項式である.
        \item $f\in\Ker\F$とすると,$f$のFourier係数は全て$0$である.このとき,Fourier変換のCesaro平均は
        \[(K_n*f)(u)=\sum_{n=-N}^N\paren{1-\frac{\abs{n}}{N+1}}\wh{f}(n)e^{inu}=0\]
        であるから,$(K_n*f)\to0=f\in L^1(\bT)$.よって,$\Ker\F=0$.
        \item 任意の$f,g\in L^1(\bT)$について,次のように評価できる:
        \begin{align*}
            \forall_{n\in\N}\quad\abs{\wh{f}(n)-\wh{g}(n)}&=\Abs{\int_T(f(t)-g(t))e^{-int}\dt}\\
            &\le\int_T\abs{f(t)-g(t)}e^{-int}\dt\le\norm{f-g}_1\norm{e^{-int}}_\infty=\norm{f-g}_1.
        \end{align*}
        よって,$\sup_{n\in\N}\abs{\wh{f}(n)-\wh{g}(n)}=\norm{\wh{f}-\wh{g}}_\infty\le\norm{f-g}_1$.
        \item 任意の$f\in L^1(\bT)$について,(1)より,$\forall_{\ep>0}\;\exists_{P(t)=\sum_{n=-N}^Na_ne^{int}}\;\norm{f-P}_1<\ep$.
        これより,$\forall_{\abs{n}\ge N+1}\;\wh{P}(n)=0$に注意すれば,$\F$のノルム減少性(3)より,
        \[\forall_{\abs{n}\ge N+1}\;\abs{\wh{f}(n)}\le\abs{\wh{f}(n)-\wh{P}(n)}+\abs{\wh{P}(n)}\le\norm{f-P}_1<\ep.\]
    \end{enumerate}
\end{Proof}

\begin{lemma}
    $1\le p<\infty$とする.
    \begin{enumerate}
        \item (Young) $\forall_{f\in L^p(\bT)}\;\forall_{g\in L^1(\bT)}\;f*g\in L^p(\bT)\land \norm{f*g}_p\le\norm{f}_p\norm{g}_1$.
        \item (Riesz) $C(\bT)\subset L^p(\bT)$は$L^p$-ノルムについて稠密である.
    \end{enumerate}
\end{lemma}

\subsection{Abelの連続性定理}

\begin{tcolorbox}[colframe=ForestGreen, colback=ForestGreen!10!white,breakable,colbacktitle=ForestGreen!40!white,coltitle=black,fonttitle=\bfseries\sffamily,
title=]
    収束級数を係数に持つ冪級数は,$r\nearrow1$の極限について連続になる.
    これを基礎として,Fourier部分和の極限が存在するならば,Poisson積分によっても得られる.
\end{tcolorbox}

\begin{definition}[Abel summability method]
    数列$(a_n)$について,
    次の値
    \[A_r:=\sum_{n=1}^\infty r^na_n,\quad 0\le r<1\]
    が存在し,極限$\lim_{r\to1}A_r=:\al\in\R$が存在するとき,これを\textbf{Abel総和可能}という.
\end{definition}

\begin{example}
    数列$a_n=(-1)^{n-1}n$の級数
    \[S_N:=\sum_{n=1}^N(-1)^{n-1}n\]
    はAbel総和可能だがCesaro総和可能ではない.
\end{example}
\begin{Proof}\mbox{}
    \begin{enumerate}
        \item Abel和$A_r=\sum_{n=1}^\infty r^n(-1)^{n-1}n$の収束は,隣項比が
        \[\frac{a_{n+1}}{a_n}=\frac{r^{n+1}(-1)^{n}(n+1)}{r^n(-1)^{n-1}n}=r\paren{1+\frac{1}{n}}\]
        で,$0\le r<1$に対して十分大きな$n$については$1$より小さいから,d'Alembertの判定法より分かる.
        \item 一方でCesaro和$\sigma_N=\frac{1}{N}\sum_{n=1}^NS_n$は,$S_n=(-1)^{n-1}\ceil{n/2}\;(n\in\N^+)$と表せることに注意すれば,
        \begin{align*}
            \sigma_{N+1}-\sigma_N&=\frac{1}{N+1}\sum_{n=1}^NS_n-\frac{1}{N}\sum_{n=1}^NS_n\\
            &=\frac{S_{N+1}}{N+1}+\sum_{n=1}^NS_n\paren{\frac{1}{N+1}-\frac{1}{N}}\\
            &=\frac{S_{N+1}}{n+1}-\sum_{n=1}^N\frac{S_n}{N(N+1)}=\frac{\sum_{n=1}^N(S_{n+1}-S_n)}{N(N+1)}
        \end{align*}
        より,$\sigma_N$は振動して収束しないことがわかる.
    \end{enumerate}
\end{Proof}

\begin{theorem}[Abelの連続性定理]
    数列$\{a_n\}\subset\R$について,級数$\al:=\sum_{n\in\N}a_n$は収束するとする.
    このとき,Abel総和可能でもあり,Abel和の極限と一致する:$\al=\lim_{r\to1}A_r$.
    すなわち,収束級数$(a_n)$を係数に持つ冪級数
    $\sum_{n\in\N}a_nx^n$は$[0,1]$上連続である.
\end{theorem}
\begin{Proof}\mbox{}
    \begin{enumerate}
        \item Abel和と部分和$S_n:=\sum_{k=1}^na_k$との関係は,$S_0=0$とすると,
        \begin{align*}
            \sum_{n=1}^Nr^na_n&=\sum_{n=1}^Nr^n(S_n-S_{n-1})\\
            &=\sum_{n=1}^Nr^nS_n-\sum_{n=0}^{N-1}r^{n+1}S_n\\
            &=\sum_{n=1}^{N-1}(r^n-r^{n+1})S_n+r^NS_N\\
            &=(1-r)\sum_{n=1}^{N-1}r^nS_n+r^NS_N=:A^{(1)}+A^{(2)}
        \end{align*}
        と判る.$S_N$が極限を持つことより,$0\le r<1$のとき,$A^{(2)}\xrightarrow{N\to\infty}0$.
        同様にして$A^{(1)}$もCauchy列であるから,極限$A^{(1)}\xrightarrow{N\to\infty}(1-r)\sum_{n=1}^\infty r^nS_n=A_r$を持つ.
        \item あとは,こうして定まる関数$f:(-1,1)\to\R;r\mapsto(1-r)\sum_{n=1}^\infty r^nS_n$が$f(1)=\al$とすると連続であることを示せば良い.
        任意に$\ep>0$を取る.$S_n$は収束するから,$\exists_{N\in\N}\;\forall_{n\ge N}\;\abs{\al-S_n}<\frac{\ep}{2}$.
        さらに,
        \[\forall_{r\in(-1,1)}\;(1-r)\sum_{n=0}^\infty r^n=1\Rightarrow(1-r)\sum_{n=0}^\infty r^n\al=\al\]
        であるから,$N\in\N$に応じて十分小さい$\delta>0$を取れば,
        \[\exists_{\delta>0}\;\forall_{r\in(1-\delta,1)}\;\abs{f(r)-\al}=\abs{(1-r)\sum_{n=0}^\infty(S_n-\al)r^n}\le(1-x)\sum_{n=0}^N\abs{S_n-\al}\abs{r}^n+\frac{\ep}{2}\le\ep.\]
    \end{enumerate}
\end{Proof}
\begin{remarks}[Abelの部分和公式]
    証明中で行った式変形(「$r$を乗じて引く」というやつ)を一般化すると,
    $\{a_n\},\{b_n\}\subset\R$に対して,$\sigma_n:=b_0+b_1+\cdots+b_n,\sigma_{-1}=0$とするとき,
    \[\forall_{N>M\ge 0}\quad\sum_{n=M}^Na_nb_n=\sum_{n=M}^{N-1}(a_n-a_{n+1})\sigma_n+(a_N\sigma_N-a_M\sigma_{M-1}).\]
    という部分積分公式が成り立つ.
    左辺を右辺に対応させる対応を\textbf{Abel変換}という.
\end{remarks}

\subsection{Abel-Poisson積分法}

\begin{tcolorbox}[colframe=ForestGreen, colback=ForestGreen!10!white,breakable,colbacktitle=ForestGreen!40!white,coltitle=black,fonttitle=\bfseries\sffamily,
title=]
    $\wh{f}(n)e^{int}$を係数に持つ級数は収束半径は1より大きいから,
    $r\in(-1,1)$上では$r\bT$上一様に絶対収束する.
\end{tcolorbox}

\begin{proposition}[Abel-Poisson sum]\label{prop-Abel-Poisson-sum}
    $f\in L^1(\bT)$と$r<1$に対して,$r^{\abs{n}}$で重みづけたFourier部分和の極限
    \[A_r[f](t):=\sum_{n\in\Z}r^{\abs{n}}\wh{f}(n)e^{int},\quad 0\le r<1\]
    は一様に絶対収束する.特に,$A_r[f]\in C(\bT)$.
    これを$f$の\textbf{Abel-Poisson和}または\textbf{Poisson積分}という.
\end{proposition}
\begin{Proof}
    Fourier変換のノルム減少性\ref{cor-Riemann-Lebesgue}より,$\forall_{n\in\Z}\;\abs{\wh{f}(n)}\le\norm{f}_1$.
    よって,任意の$t\in\bT$について,
    \[\sum_{n\in\Z}\abs{r^{\abs{n}}}\abs{\wh{f}(n)}\abs{e^{int}}\le\norm{f}_1\sum_{n\in\Z}\abs{r}^{\abs{n}}<\infty.\]
\end{Proof}

\begin{proposition}[Poisson核は連続に添え字づけられた総和核である]
    \[P_r(t):=\sum_{n\in\Z}r^{\abs{n}}e^{int}\in L^1(\bT)\]
    を\textbf{Poisson核}という.
    \begin{enumerate}
        \item 畳み込み:$A_r[f]=P_r*f$.
        \item 等比数列の和公式:$P_r(t)=\frac{1-r^2}{1-2r\cos t+r^2}$.特に,$P_r\ge0$.
        \item $P_r\dt$は$\bT$上の確率測度を定める$\int_TP_r(t)\dt=1$.
        \item $\forall_{\delta\in(0,\pi)}\;\sup_{\delta\le\abs{t}\le\pi}P_r(t)\to0\;(r\to1)$.
    \end{enumerate}
    $(P_r)_{r\in[0,1]}$は「連続な総和核」になる.適当な部分列$(P_{r_n})$を考えれば,厳密な総和核ともなろう.
\end{proposition}
\begin{Proof}\mbox{}
    \begin{enumerate}
        \item 次のように計算できる:
        \begin{align*}
            \forall_{t\in\bT}\;(P_r*f)(t)&=\int_\bT P_r(t-u)f(u)du\\
            &=\int_\bT\paren{\sum_{n\in\Z}r^{\abs{n}}e^{int}}e^{-inu}f(u)du\\
            &=\sum_{n\in\Z}r^{\abs{n}}\wh{f}(n)e^{int}=A_r[f](t).
        \end{align*}
        \item 等比級数の和と見ることで,次のように計算できる:
        \begin{align*}
            P_r(t)&=1+\sum_{n=1}^\infty r^ne^{int}+\sum_{n=1}^\infty r^ne^{-int}\\
            &=1+\frac{re^{it}}{1-re^{it}}+\frac{re^{-it}}{1-re^{-it}}\\
            &=1+\frac{re^{it}-r^2+re^{-it}-r^2}{1-2r\cos t+r^2}=\frac{1-r^2}{1-2r\cos t+r^2}.
        \end{align*}
        \item $P_r(t)=\sum_{n\in\Z}r^{\abs{n}}e^{int}$は絶対収束もし,かつ$\bT$上可積分だから,Lebesgueの優収束定理より,
        \[\int_\bT P_r(t)\dt=\int_\bT\sum_{n\in\Z}r^{\abs{n}}e^{int}\dt=\sum_{n\in\Z}\int_\bT r^{\abs{n}}e^{int}\dt=1.\]
        \item まず,$P_r(t)$の分母について,
        \begin{align*}
            \forall_{\delta\in(0,\pi/2)}\;\forall_{\delta\le\abs{t}\le\pi}\;1-2r\cos t+r^2&\ge1-2r\cos\delta+r^2\\
            &=(r-\cos\delta)^2+1-\cos^2\delta\ge 1-\cos^2\delta>0
        \end{align*}
        と評価出来るから,
        \[(0\le)\sup_{\delta\le\abs{t}\le\pi}P_r(t)\le\sup_{\delta\le\abs{t}\le\pi,\delta\in(0,\pi/2)}P_r(t)\le\frac{1-r^2}{1-\cos^2\delta}\xrightarrow{r\to1}0.\]
    \end{enumerate}
\end{Proof}

\begin{theorem}
    Abel-Poisson和についても次が成り立つ:
    \begin{enumerate}
        \item 任意の$1\le p<\infty$について,$\forall_{f\in L^p(\bT)}\;A_r[f]\xrightarrow{r\to1}f$が$L^p(\bT)$-の意味で成り立つ.
        \item $\forall_{f\in C(\bT)}\;A_r[f]\xrightarrow{r\to1}f$が$C(\bT)$の一様ノルムについて成り立つ.
        \item $\forall_{f\in L^\infty(\bT)}\;A_r[f]\xrightarrow{r\to 1}f$が$L^\infty(\bT)$の$w^*$-位相について成り立つ.
    \end{enumerate}
\end{theorem}

\subsection{円板上のLaplace方程式のD-境界値問題}

\begin{tcolorbox}[colframe=ForestGreen, colback=ForestGreen!10!white,breakable,colbacktitle=ForestGreen!40!white,coltitle=black,fonttitle=\bfseries\sffamily,
title=]
    そもそもFourierは熱方程式を解くために理論を作った.なお,Poisson方程式$\Lap u=f$について,$f=\pp{u}{t}$の場合が熱方程式,$f=\pp{^2u}{t^2}$の場合が波動方程式である.
    熱にしろ,振動にしろ,平衡状態を表すのが調和関数である.
\end{tcolorbox}

\begin{proposition}[SchwartzによるDirichlet境界値問題の解決]
    Poisson核$\{P_r\}_{r\in[0,1]}\subset L^1(\bT)$について,
    \begin{enumerate}
        \item Poisson核$P_r(\theta)$は$re^{i\theta}\in\Delta$上の調和関数である:$\Laplace P_r=0\;\on\Delta$.
        \item 任意の$f\in L^1(\bT)$に対して,Abel-Poisson和$u(re^{i\theta}):=(P_r*f)(\theta)=A_r[f](\theta)$は$\Delta$上の調和関数である:$\Laplace u=0\;\on\Delta$.
        \item 任意の$f\in C(\bT)$に対して,$u$は次のディリクレ境界値問題の解を与える:
        \[\begin{cases}
            \Laplace u=0&\on\Delta\\
            u=f&\on\partial\Delta
        \end{cases}\]
    \end{enumerate}
\end{proposition}
\begin{Proof}\mbox{}
    \begin{enumerate}
        \item 次のように表示すれば判るように,正則関数$\frac{1+z}{1-z}$の実部であるからである:
        \[P_r(\theta)=\sum_{n\in\Z}r^{\abs{n}}e^{in\theta}=1+\sum_{n=1}^\infty (r^ne^{int}+r^ne^{-int})=1+\sum_{n=1}^\infty(z^n+\o{z}^n)=1+2\Re\paren{\frac{z}{1-z}}.\]
        \item 次のように表示すれば判るように,$\Delta$上の正則関数と,正則関数の複素共役との和であるから,Laplace作用素$\Laplace$の線型性より,全体として$P_r*f(\theta)$も調和:
        \[(P_r*f)(\theta)=\sum_{n=0}^\infty\wh{f}(n)r^ne^{in\theta}+\sum_{n=1}^\infty\wh{f}(-n)r^ne^{-in\theta}=\sum_{n=0}^\infty\wh{f}(n)z^n+\sum_{n=1}^\infty\wh{f}(-n)\o{z}^n.\]
        \item Poisson核が総和核であるために,$u(re^{i\theta})=P_r*f(\theta)\xrightarrow{r\to1}f(\theta)$より判る.
    \end{enumerate}
\end{Proof}
\begin{remarks}[Abel-Poisson和の積分表示]
    \[P_r(\theta)=\Re\paren{\frac{1+re^{i\theta}}{1-re^{i\theta}}}\]
    という表示も得た.すると,Abel-Poisson和は
    \begin{align*}
        A_r[f](t)=(P_r*f)(t)&=\frac{1}{2\pi}\int_\bT P_r(t-\theta)f(\theta)d\theta\\
        &=\frac{1}{2\pi}\int_\bT\Re\paren{\frac{1+re^{it-i\theta}}{1-re^{it-i\theta}}}f(\theta)d\theta=\frac{1}{2\pi}\int_\bT\Re\paren{\frac{e^{i\theta}+re^{it}}{e^{i\theta}-re^{it}}}f(\theta)d\theta.
    \end{align*}
    と表せる,$(r,t)$を$re^{it}=:z\in\Delta$として同時に動かすと,
    \[P_f(z):=(P_r*f)(t)=\frac{1}{2\pi}\int_\bT\Re\paren{\frac{e^{i\theta}+z}{e^{i\theta}-z}}f(\theta)d\theta\quad z\in\Delta\]
    という積分を得る,これを$f\in L^1(\bT)$の\textbf{Poisson積分}という.これは最大値の原理から導かれる次の標識もあり,これを\textbf{Poissonの公式}という:
    \[\frac{1}{2\pi}\int_{\bT}\frac{Re^{i\theta}+z}{Re^{i\theta}-z}f(Re^{i\theta})d\theta=\frac{1}{2\pi}\int_\bT\frac{R^2-\abs{z}^2}{\abs{Re^{i\theta}-a}^2}f(Re^{i\theta})d\theta.\]
\end{remarks}

\begin{proposition}
    $f\in C^\infty(\bT)$に関するDirichlet境界値問題について,
    \begin{enumerate}
        \item $f(\theta)=\cos\theta$の場合の解は$u(z)=\Re(z)$が与える.
        \item $f(\theta)=\cos^2\theta$の場合の解は$u(z)=\frac{1}{2}(1+\Re(z^2))$が与える.
    \end{enumerate}
\end{proposition}
\begin{Proof}\mbox{}
    \begin{enumerate}
        \item 命題に従って,
        \begin{align*}
            u(re^{i\theta})&=(P_r*f)(\theta)=\sum_{n\in\Z}r^{\abs{n}}e^{in\theta}\int_\bT\cos te^{-int}\dt\\
            &=\frac{1}{2}\sum_{n\in\Z}r^{\abs{n}}e^{in\theta}\int^\pi_{-\pi}(e^{-(n-1)it}+e^{-(n+1)it})\dt\\
            &=\paren{\pi re^{i\theta}+\pi re^{-i\theta}}\frac{1}{2\pi}=r\cos\theta.
        \end{align*}
        と計算出来る.
        \item 計算
        \[\wh{f}(n)=\int_\bT\cos^2te^{-int}\dt=\frac{1}{4}\int^\pi_{-\pi}(e^{-(n-2)it}+e^{-(n+2)it}+2e^{-int})\dt\]
        に注意すれば,
        命題に従って,
        \begin{align*}
            u(re^{i\theta})&=(P_r*f)(\theta)=\frac{r^2}{4}(e^{2i\theta}+e^{-2i\theta})+\frac{1}{2}\\
            &=r^2\cos^2\theta+\frac{1-r^2}{2}
        \end{align*}
        これは,
        \[r^2\cos^2\theta+\frac{1-r^2}{2}=x^2+\frac{1-(x^2+y^2)}{2}=\frac{1+x^2-y^2}{2}=\frac{1+\Re(z^2)}{2}\]
        と表せる.
    \end{enumerate}
\end{Proof}

\subsection{Tauber型の定理}

\begin{tcolorbox}[colframe=ForestGreen, colback=ForestGreen!10!white,breakable,colbacktitle=ForestGreen!40!white,coltitle=black,fonttitle=\bfseries\sffamily,
title=]
    複素級数$\sum_{n=1}^\infty c_n$について,収束するならばCesaro総和可能ならばAbel総和可能である.
    逆は制約を付けないと成り立たず,このような結果をTauber型の定理という.
\end{tcolorbox}

\begin{theorem}[\cite{Stein-Shakarchi-03-Fourier} Exercise 2.14]
    複素数列$\{c_n\}\subset\C$について,
    \begin{enumerate}
        \item $\sum_{n=1}^\infty c_n$は$\sigma$にCesaro総和可能で,$c_n=o(1/n)$ならば,級数$\sum_{n=1}^\infty c_n$は収束して$\sigma$に等しい.
        \item Abel総和可能性についても同様の結果が成り立つ.
    \end{enumerate}
\end{theorem}

\begin{proposition}[Littlewoodによる精緻化]
    複素数列$\{c_n\}\subset\C$について,
    $\sum_{n=1}^\infty c_n$は$\sigma$にAbel総和可能で,$c_n=O(1/n)$ならば,級数$\sum_{n=1}^\infty c_n$は収束して$\sigma$に等しい.
\end{proposition}

\begin{corollary}\label{cor-speed-of-convergence-and-Fourier-series-convergence}
    $f\in L^1(\bT)$が
    $\wh{f}(n)=O(1/n)$を満たすとき,
    \begin{enumerate}
        \item $t\in\bT$上で$f$が跳躍不連続ならば,
        \[S_N[f](t)\to\frac{f(t^+)+f(t^-)}{2}.\]
        \item $f\in C(\bT)$ならば,$S_N[f]\to f\;\In C(\bT)$.
    \end{enumerate}
\end{corollary}

\subsection{総和核の例}

\begin{example}[Gauss kernel]
    $\bT$の熱核を
    \[G_t(x):=\sum_{n\in\Z}e^{-n^2t}e^{int}\quad(t>0),\qquad \wh{G}_t(n)=e^{-n^2t}\quad(n\in\Z).\]
    で定めると,これは正の総和核である.
\end{example}
\begin{Proof}\mbox{}
    \begin{enumerate}[{[S}1{]}]
        \item $n=0$以外の項は周期関数であるから.
        \item \[G_t(x)=\frac{2\pi}{\sqrt{4\pi t}}\sum_{n\in\Z}e^{\frac{(x+2\pi n)^2}{4t}}\]
        の表示から,正であるため.
        \item 次のように評価できる:
        \begin{align*}
            \sup_{\delta\le\abs{t}\le\pi}\abs{G_s(t)}&\le\frac{2\pi}{\sqrt{4\pi s}}\paren{e^{-\frac{\delta^2}{4s}}+\sum_{n\ne0}e^{-\frac{(\pi n)^2}{4s}}}\xrightarrow{s\to0}0.
        \end{align*}
    \end{enumerate}
\end{Proof}

\begin{example}
    \[K_N*K_N(t)=\sum_{n=-N}^N\paren{1-\frac{\abs{n}}{N+1}}^2e^{int}\]
    も総和核である.
\end{example}
\begin{Proof}\mbox{}
    \begin{enumerate}[{[S}1{]}]
        \item $n=0$以外の項は周期関数であるから,
        \begin{align*}
            \dint_\bT(K_N*K_N)(t)dt=\dint_\bT e^{i0t}dt=1.
        \end{align*}
        \item 周期化による表示:
        $\norm{K_N*K_N}_{L^1(\bT)}\le\norm{K_N}_{L^1(\bT)}\norm{K_N}=1$より\ref{prop-operator-norm-of-convolution}.
        \item $K_N*K_N$は$K_N$のCesaro和ととらえられる.任意の$\delta\in(0,\pi)$について,
        \begin{align*}
            \sup_{\delta\le\abs{t}\le\pi}K_N*K_N(t)&=\sup_{\delta\le\abs{t}\le\pi}\frac{1}{N+1}\sum_{n=0}^N\frac{1}{n+1}\paren{\frac{\sin\frac{n+1}{2}t}{\sin\frac{t}{2}}}^2\\
            &\le\frac{1}{N+1}\paren{\sin\frac{\delta}{2}}^{-2}\sum_{n=0}^N\frac{1}{n+1}.
        \end{align*}
        であるが,
        \[\frac{1}{N+1}\sum_{n=0}^N\frac{1}{n+1}\le\frac{1}{N+1}\paren{1+\int^N_1\frac{dx}{x}}=\frac{\log N+1}{N+1}\xrightarrow{N\to\infty}0.\]
    \end{enumerate}
\end{Proof}

\begin{proposition}[Jackson's kernel \cite{Katznelson-Fourier} Exercise 3.5]
    \[J_N(t):=\frac{K^2_N(t)}{\norm{K_N}^2_{L^2(\bT)}}\]
    と定めると,これは正の総和核になる.
\end{proposition}

\section{Fourier部分和の各点収束}

\begin{tcolorbox}[colframe=ForestGreen, colback=ForestGreen!10!white,breakable,colbacktitle=ForestGreen!40!white,coltitle=black,fonttitle=\bfseries\sffamily,
title=]
    以前までは$L^p$-収束を議論していた.
    これと各点での消息は全く別物である.
    特に各点ではひどいが,
    \begin{enumerate}
        \item $f\in L^1(\bT)$の連続点$t\in\bT$では,$S_N[f](t)$は収束するならば値は$f(t)$である(Fejerの定理の系\ref{cor-Fejer}).
        \item $f\in L^1(bT)$の可微分点$t\in\bT$では,$S_N[f](t)$は収束する(Diniの定理の系\ref{cor-Fourier-sum-converges-at-differentiable-points}).
        \item $f\in C(\bT)$が$\wh{f}\in l^1(\Z)$も満たすとき,$S_N[f]\to f\;\In C(\bT)$は一様収束である.
        この前件は特に$f\in C^2(\bT)$のとき満たされる.
        \item 実は,ある閉区間$I$上で連続かつその近傍で有界変動ならば,Fourier級数は$f$に一様収束する.
    \end{enumerate}
\end{tcolorbox}

\begin{remarks}
    実は,Fourier部分和$S_N[f]$が$f$に各点収束する$C(\bT)$の元は,$C(\bT)$のBaire第一類集合をなす.
    すなわち,$S_N[f]$が概発散するような$f\in C(\bT)$の全体は充満集合である.
    これは,$D_N$との畳み込みが定める作用素に対するBanach-Steinhausの定理からわかる.
\end{remarks}

\subsection{Fourier部分和の収束先:Cesaro和による予測}

\begin{tcolorbox}[colframe=ForestGreen, colback=ForestGreen!10!white,breakable,colbacktitle=ForestGreen!40!white,coltitle=black,fonttitle=\bfseries\sffamily,
title=Fourier級数は元の関数の情報を十分に含んでおり,算術平均によって取り出せる]
    \begin{enumerate}
        \item Fejerの定理より,両側極限が存在する点では,その平均にCesaro和の値が収束することが判る.
        \item したがって関数の連続点$t\in\bT$においては,Fourier部分和$S_N[f]$は収束するならばその値は$f(t)$である.
        \item ではFourier部分和の極限$\lim_{N\to\infty}S_N[f](t)$の存在条件は,両側の微分係数も存在すれば十分(後述のDiniの定理の系\ref{cor-Fourier-sum-converges-at-differentiable-points}).
    \end{enumerate}
\end{tcolorbox}

\begin{theorem}[Fejer]
    $f\in L^1(\bT),t_0\in\bT$について,次の$\al\in\C$が存在するならば,$\sigma_N[f](t_0)\to\al$が成り立つ:
    \[\al:=\frac{1}{2}\lim_{h\to\infty}(f(t_0+h)+f(t_0-h)).\]
    したがって,Fourier部分和$S_N[f](t_0)$の極限が$N\to\infty$で存在するならば,これも$\al$である.
\end{theorem}
\begin{Proof}\mbox{}
    \begin{description}
        \item[評価の対象] $K_N=\frac{1}{N+1}\sum_{n=0}^ND_n$は偶関数の和なので偶関数であることと,$\al=\frac{1}{2\pi}\int^\pi_02\al K_N(t)dt$に注意して,
        \begin{align*}
            \sigma_N[f](t_0)-\al&=\frac{1}{2\pi}\int^\pi_{-\pi}f(t_0-t)K_N(t)dt-\al\\
            &=\frac{1}{2\pi}\int^\pi_0(f(t_0-t)+f(t_0+t))K_N(t)dt=\frac{1}{2\pi}\int^\pi_0\paren{f(t_0+t)+f(t_0-t)-2\al}K_N(t)dt.\\
            \abs{\sigma_N[f](t_0)-\al}&=\frac{1}{2\pi}\paren{\int^\delta_0+\int^\pi_{\delta}}\abs{f(t_0+t)+f(t_0-t)-2\al}K_N(t)dt=:A^{(1)}+A^{(2)}.
        \end{align*}
        \item[評価] 任意の$\ep>0$を取る.
        \begin{enumerate}
            \item 極限$\al$についての仮定から,$\exists_{\delta\in(0,\pi)}\;\forall_{t\in(-\delta,\delta)}\;\abs{f(t_0+t)+f(t_0-t)-2\al}<\ep$.
            よって,
            \[A^{(1)}\le\int^\delta_0\ep K_N(t)\dt\le\ep.\]
            \item また$K_N$は総和核だから,$\exists_{N_1\in\N}\;\forall_{N\ge N_1}\;\forall_{\abs{t}\in[\delta,\pi]}\;K_N(t)<\ep$.
            よって,
            \[A^{(2)}\le\frac{1}{2\pi}\int^\pi_\delta\abs{f(t_0+t)+f(t_0-t)-2\al}\ep dt\le(\norm{f}_1+\abs{\al})\ep.\]
        \end{enumerate}
    \end{description}
\end{Proof}

\begin{corollary}\label{cor-Fejer}
    $f\in L^1(\bT)$とし,$t_0\in\bT$上で$f$は連続とする.
    \begin{enumerate}
        \item $\sigma_N[f](t_0)\to f(t_0)$.
        \item $S_N[f](t_0)$が$N\to\infty$の極限で収束するならば,その極限は$f(t_0)$である.
    \end{enumerate}
\end{corollary}


\begin{theorem}[Lebesgue \cite{Katznelson-Fourier} Th'm 3.2]
    $f\in L^1(\bT)$とする.
    \begin{enumerate}
        \item 点$t_0\in\bT$において次が成り立つならば,$\sigma_N[f](t_0)\to f(t_0)$:
        \[\exists_{\al\in\R}\quad\lim_{h\to0}\frac{1}{h}\int^h_0\Abs{\frac{f(t_0+s)+f(t_0-s)}{2}-\al}ds=0.\]
        \item 特に,殆ど至る所の$t\in\bT$において,$\sigma_N[f](t)\to f(t)$.
    \end{enumerate}
\end{theorem}
\begin{remarks}
    Fejerの定理の条件は(1)よりも強いので,一般化になっている.
    ここまでくると,可積分関数に関する主張としてとらえることができる.
    これがLebesgueの定理である.
\end{remarks}

\subsection{Fourier部分和の収束先:Abel-Poisson和による予測}

\begin{tcolorbox}[colframe=ForestGreen, colback=ForestGreen!10!white,breakable,colbacktitle=ForestGreen!40!white,coltitle=black,fonttitle=\bfseries\sffamily,
title=]
    Fejerの定理の類似物が成り立ち,またAbel-Poisson和はCesaro和よりも収束しやすいから$A_r[f]$も$f$に概収束するが,$A_r[f]$は$\Delta$上の調和関数であることからより強力な収束をする.
    同様にHardy-Littlewoodの理論による.
\end{tcolorbox}

\begin{theorem}
    $f\in L^1(\bT),t_0\in\bT$について,次の$\al\in\C$が存在するならば,$A_r[f](t_0)\to\al$が成り立つ:
    \[\al:=\frac{1}{2}\lim_{h\to\infty}(f(t_0+h)+f(t_0-h)).\]
    したがって,Fourier部分和$S_N[f](t_0)$の極限が$N\to\infty$で存在するならば,これも$\al$である.
\end{theorem}
\begin{Proof}
    Poisson核$P_r(t)=\sum_{n\in\Z}r^{\abs{n}}e^{int}$も偶関数であるため,同様に証明出来る.
\end{Proof}
\begin{remarks}
    特に
    $f\in L^1(\bT)$が$t_0\in\bT$で連続であるならば,
    $A_r[f](t_0):=\sum_{n\in\Z}r^{\abs{n}}\wh{f}(n)e^{int_0}$が$f(t_0)$に$r\to1$で収束する.
    $A_r[f](t_0)$は境界条件が$f$で与えられるDirichlet境界値問題の解$u$とも見れるから,境界付近の挙動に関する情報とも見れる.
\end{remarks}

\begin{theorem}[Fatou]
    $f\in L^1(\bT)$とする.$f$の狭義のLebesgue点$x^0\in\bT$において,$A_r[f](x^0)$は$f(x^0)$に等しい非接極限を持つ.
    すなわち,任意の角域
    \[\Gamma_\al(x^0):=\Brace{re^{ix}\in\Delta\mid\abs{\arg(1-re^{i(x-x^0)})}\le\al}.\quad 0\le\al<\frac{\pi}{2}\]
    においては,任意の近づき方$re^{ix}\to e^{ix^0}$に対して,$A_r[f](re^{ix})\to f(x^0)$.
\end{theorem}
\begin{remarks}
    $f\in L^1(\bT)$のPoisson積分$A_r[f]$は$\bT$上殆ど至る所で$f(x)$に等しい非接極限を持つ.
\end{remarks}

\subsection{Diniの判定条件}

\begin{tcolorbox}[colframe=ForestGreen, colback=ForestGreen!10!white,breakable,colbacktitle=ForestGreen!40!white,coltitle=black,fonttitle=\bfseries\sffamily,
title=]
    Diniの定理より,両側微分係数が存在する点では,その平均に値が収束することが判る.
    よって,可微分関数$f$のFourier級数は$f$に各点収束する.
\end{tcolorbox}

\begin{theorem}[Dini test]
    $f\in L^1(\bT),t_0\in\bT$について,次を満たす$\al\in\C$が存在するならば,$S_N[f](t_0)\to\al$が成り立つ:
    \[\int^\pi_0\Abs{\frac{f(t_0+t)+f(t_0-t)-2\al}{t}}dt<\infty.\]
    すなわち,$F_\al(t)=\frac{f(t_0+t)+f(t_0-t)-2\al}{t}\;(t\in\bT)$とおけば,$F_\al\in L^1(\bT)$の如何を問うており,これは$F_\al$の$t=0$での性質を測るための一つの十分条件である.
\end{theorem}
\begin{Proof}\mbox{}
    \begin{description}
        \item[問題の所在] 
        評価したい対象は,Dirichlet核$D_N$が偶関数であり,また$D_N(t)=\frac{\sin\paren{N+\frac{1}{2}}t}{\sin\frac{t}{2}}$と表せることに注意すれば,
        \begin{align*}
            S_N[f](t_0)-\al&=\int^\pi_{-\pi}f(t_0-t)D_N(t)\dt-\al\\
            &=\int^\pi_0(f(t_0+t)+f(t_0-t)-2\al)D_N(t)\dt\\
            &=\int^\pi_0\underbrace{\frac{f(t_0+t)+f(t_0-t)-2\al}{t}\frac{t}{\sin\frac{t}{2}}}_{=:\varphi(t)}\sin\paren{Nt+\frac{t}{2}}\dt\\
            &=\int^\pi_0\varphi(t)\cos(t/2)\sin Nt\dt+\int^\pi_0\varphi(t)\sin(t/2)\cos Nt\dt.
        \end{align*}
        と変形できる.これが$0$に収束することを示せば良い.
        \item[Riemann-Lebesgueの補題に拠る解決]
        $\frac{t}{\sin(t/2)}\in L^\infty(\bT)$より,定理の仮定から$\varphi\in L^1(\bT)$であることに注意すれば,次の補題から$0$に収束することが判る.
    \end{description}
\end{Proof}

\begin{lemma}[Riemann-Lebesgueの補題の系]\label{lemma-cor-of-Riemann-Lebesgue}
    任意の$g\in L^1(\bT)$について,
    \begin{enumerate}
        \item $\int_\bT g(t)\cos nt\dt\xrightarrow{n\to\infty}0$.
        \item $\int_\bT g(t)\sin nt\dt\xrightarrow{n\to\infty}0$.
    \end{enumerate}
\end{lemma}
\begin{Proof}
    任意の$g\in L^1(\bT)$について,
    \begin{align*}
        \frac{1}{2\pi}\int^{2\pi}_0g(t)\cos ntdt&=\frac{1}{2\pi}\int^{2\pi}_0g(t)\frac{e^{int}+e^{-int}}{2}dt\\
        &=\frac{\wh{g}(n)+\wh{g}(-n)}{2}\to0.
    \end{align*}
\end{Proof}

\begin{corollary}[可微分点においてFourier級数は収束する]\label{cor-Fourier-sum-converges-at-differentiable-points}
    $f\in L^1(\bT),t_0\in\bT$とする.
    \begin{enumerate}
        \item $f$が$t_0$で左右極限$f(t_0\pm0)$を持ち,さらに$f$は$t_0$で左右微分を持つならば,
        \[S_N[f](t_0)\xrightarrow{N\to\infty}\frac{f(t_0+0)+f(t_0-0)}{2}.\]
        \item 特に,$f$が$t_0$で微分可能ならば,$S_N[f](t_0)\xrightarrow{N\to\infty}f(t_0)$.
    \end{enumerate}
\end{corollary}
\begin{Proof}
    $\al$を左右の極限の平均$\al=\frac{f(t_0+0)+f(t_0-0)}{2}$とおくと,$\al\in\R$は定まっており,
    \[\Abs{\frac{f(t_0+t)+f(t_0-t)-2\al}{t}}\le\Abs{\frac{f(t_0+t)-f(t_0-0)}{t}}+\Abs{\frac{f(t_0-t)-f(t_0-0)}{t}}.\]
    右辺は$f$が$t_0$にて左右の微分係数を持つという仮定より,有限である.
\end{Proof}
\begin{history}[他の判定条件]
    Dirichlet (1829)は可微分関数より狭いクラスである単調関数について,各点収束の結果を得た\ref{thm-Dirichlet-Jordan-test}.
    なお,有界変動関数は単調関数の差で表せることはJordanがその後示した.
    なお,有界変動関数の不連続点は高々可算個で,そこでは元の値に収束するわけではないことに注意.
\end{history}

\begin{observation}[Dini testの射程の観察]
    \[f(t):=\begin{cases}
        t^{1/2}&t\in[0,\pi]\\
        \abs{t}^{-1/3}&t\in(-\pi,0)
    \end{cases}\]
    のFourier級数の極限$\lim_{N\to\infty}S_N[f](t)\;(t\in\bT)$はどうなるかを考える.
    $t\ne0$ならば明らかに微分可能であるから,極限は$f(t)$である.

    しかし$t$の冪は$1$より小さいから,例えば$\dd{\sqrt{t}}{t}=\frac{1}{2\sqrt{t}}$は$t\to+0$での極限を持たない.
    一方で,$F(t):=\frac{f(t)+f(-t)-2\al}{t}=t^{-1/2}+t^{-2/3}-\frac{2\al}{t}\;(0<t<\pi)$とおくと,これは$\al=0$のとき$[0,\pi]$上可積分である:$F\in L^1([0,\pi])$.
    よって,Diniの定理から直接$S_N[f](0)\to f(0)=0$が判る.
\end{observation}

\subsection{反転公式:Fourier係数が絶対収束することによる条件}

\begin{tcolorbox}[colframe=ForestGreen, colback=ForestGreen!10!white,breakable,colbacktitle=ForestGreen!40!white,coltitle=black,fonttitle=\bfseries\sffamily,
title=]
    $\F:L^1(\bT)\to c_0(\Z)$において,特に像が$l^1(\Z)$に入るとき,Fourier部分和は概収束する.
    よって,Fourier係数が絶対収束する連続関数$f$は,Fourier級数が$f$に一様収束する.
\end{tcolorbox}

\begin{theorem}[Fourier級数が絶対収束するならば部分和は一様収束する]\label{thm-l1-series}
    $f\in L^1(\bT)$のFourier変換$\wh{f}\in c_0(\Z)$は絶対収束するとする:$\sum_{n\in\Z}\abs{\wh{f}(n)}<\infty$.
    \begin{enumerate}
        \item このとき,ある連続関数$g\in C(\bT)$が存在して,一様ノルムについて$S_N[f]\to g\;\In C(\bT)$.
        \item さらにこのとき,$f=g\;\ae\;\on\bT$が成立する.すなわち,$f\in L^1(\bT)$のFourier部分和は概収束し,連続な修正$g$を持つ.
        \item $f\in C(\bT)$でもあるとき,Fourier部分和$S_N[f]$は$f$に一様収束する.
    \end{enumerate}
\end{theorem}
\begin{Proof}\mbox{}
    \begin{enumerate}
        \item $g(t):=\sum_{n\in\Z}\wh{f}(n)e^{int}$とすると,仮定$\wh{f}\in l^1(\Z)$より$g\in L^1(\bT)$,特に一様に絶対収束する.よって,$g$は$\bT$上連続.
        \item さらに,Fourier変換$\F:L^1(\bT)\to c_0(\Z)$の単射性\ref{cor-Riemann-Lebesgue}より,$f$は$g$と$L^1(\bT)$の元としては一致する.
        \item $F:=\Brace{x\in\bT\mid f(x)=g(x)}$とすると,これは測度1の集合である.仮に$N:=\bT\setminus F$が内点を持つならば,十分小さな開球を含むということだから,測度が$0$であることに矛盾する.
        したがって,$N\subset\partial F$.よって,$F$は$\bT$上稠密より,$f,g$の連続性より$F=\bT$.
    \end{enumerate}
\end{Proof}
\begin{remarks}
    実は,定理の条件を満たす空間を
    \[\A(\bT):=\Brace{f\in L^1(\bT)\mid\wh{f}\in l^1(\Z)}\]
    とすると,$\A(\bT)\simeq_\Ban l^1(\Z)$である\cite{Katznelson-Fourier} 6節.
    $l^1(\Z)$の方のノルムについてBanach代数としても同型である.
\end{remarks}

\begin{corollary}[Fourier級数が絶対収束するための条件]\label{cor-for-Fourier-series-to-absolutely-convergent}
    任意の$f\in C^2(\bT)$について,
    \[\wh{f}(n)=O(1/\abs{n}^2),\qquad\abs{n}\to\infty.\]
    したがって,$\wh{f}\in L^1(\bT)$が成り立つから,特に,Fourier部分和$S_N[f]$は$f$に一様収束する.
\end{corollary}
\begin{Proof}
    任意の$n\in\Z\setminus\{0\}$について,部分積分を2回実行することで,$f,f'$の周期性に注意すれば,
    \begin{align*}
        \wh{f}(n)&=\int^{2\pi}_0f(t)e^{-int}\dt\\
        &=\frac{1}{2\pi}\paren{\Square{-\frac{1}{in}f(t)e^{-int}}^{2\pi}_0+\frac{1}{in}\int^{2\pi}_0f'(t)e^{-int}\dt}\\
        &=\frac{1}{2\pi in}\paren{\Square{-\frac{1}{in}f'(t)e^{-int}}^{2\pi}_0+\frac{1}{in}\int^{2\pi}_0f''(t)e^{-int}\dt}\\
        &=-\frac{1}{n^2}\wh{f''}(n).
    \end{align*}
\end{Proof}
\begin{remarks}
    Fourier変換と部分積分は相性がよく,種々の非自明な式を生み出す.
    Tauber型の定理によれば,$\wh{f}\in l^1(\Z)$ではなくとも,$\wh{f}(n)=O(1/n)$で十分\ref{cor-speed-of-convergence-and-Fourier-series-convergence}.
\end{remarks}

\subsection{各点収束条件}

\begin{tcolorbox}[colframe=ForestGreen, colback=ForestGreen!10!white,breakable,colbacktitle=ForestGreen!40!white,coltitle=black,fonttitle=\bfseries\sffamily,
title=]
    Diniの判定条件の精緻化の形で特徴付けが得られ,また「有界変動」の言葉で全く違う十分条件も与えられる.
\end{tcolorbox}

\begin{theorem}[各点収束性の特徴付け]
    $f\in L^1(\bT),x\in\bT$について,次は同値:
    \begin{enumerate}
        \item $S_N[f](x)\xrightarrow{N\to\infty}f(x)$.
        \item $\exists_{0<\delta\le\pi}\;\int^\delta_0(f(x+y)+f(x-y)-2f(x))\frac{\sin ny}{y}dy\xrightarrow{n\to\infty}0$.
    \end{enumerate}
    また,ある区間$I\subset[-\pi,\pi]$について,次も同値:
    \begin{enumerate}
        \item $S_N[f]$が$f$に$I$上一様収束する.
        \item $f$は$I$上有界かつ$I$上一様に$\delta>0$が存在して(2)が成り立つ.
    \end{enumerate}
\end{theorem}
\begin{remarks}
    明らかにDiniの判定条件は(2)の十分条件になっている.
\end{remarks}

\begin{theorem}[Dirichlet-Jordan test]\label{thm-Dirichlet-Jordan-test}
    $f\in L^1(\bT)$が
    \begin{enumerate}
        \item $x\in\bT$を含む区間で有界変動ならば,
        \[S_N[f](x)\to\frac{1}{2}(f(x+0)+f(x-0)).\]
        \item 閉区間$I$上連続かつ$I$の近傍上有界変動ならば,$S_N[f]$は$I$上一様に$f$に収束する.
    \end{enumerate}
\end{theorem}

\section{絶対連続ならばFourier部分和は一様収束する}

\begin{tcolorbox}[colframe=ForestGreen, colback=ForestGreen!10!white,breakable,colbacktitle=ForestGreen!40!white,coltitle=black,fonttitle=\bfseries\sffamily,
title=]
    $S_N[f]\to f\;\In C(\bT)$の真の十分条件に迫る.
    \begin{enumerate}
        \item 一様収束の十分条件は$f\in C^1(\bT)$\ref{prop-idea-of-uniform-convergence-of-Fourier-series}だけでなく実は$f\in\AC(\bT)$でも十分.
        \item $f\in C(\bT)$の連続度が$\om(\delta)=o((\log\delta)^{-1})\;(\delta\to0)$を満たすならば,$S_N[f]\to f\;\In C(\bT)$.
    \end{enumerate}
\end{tcolorbox}

\subsection{絶対連続性による条件}

\begin{lemma}[絶対連続性の特徴付け]
    $f\in L^1(\bT)$について,
    次の2条件は同値.
    \begin{enumerate}
        \item $f\in\AC(\bT)$.
        \item ある$\varphi\in L^1(\bT)$が存在して,次を満たす:
        \[f(t)=\int^t_0\varphi(s)ds+f(0),\quad\dint_\bT\varphi(t)dt=0.\]
    \end{enumerate}
\end{lemma}

\begin{proposition}[証明のアイデア]\label{prop-idea-of-uniform-convergence-of-Fourier-series}
    $f\in\AC(\bT)$かつ,$\varphi\in L^2(\bT)$を満たすとする($f\in C^1(\bT)$ならば十分である).このとき,$S_N[f]\to f$は$\bT$上一様収束する.
\end{proposition}
\begin{Proof}
    部分積分\ref{lemma-partial-integral-law}より,
    \begin{align*}
        \wh{f}(n)&=\int_\bT e^{-int}f(t)\dt\\
        &=\Square{\frac{e^{-int}}{-in}f(t)}^\pi_{-\pi}+\frac{1}{in}\int_\bT e^{-int}\varphi(t)\dt=\frac{\wh{\varphi}(n)}{in}.
    \end{align*}
    が成り立つ.
    よって,$l^2(\Z)$上のCauchy-Schwarzの不等式より,
    \[\sum_{M+1\le\abs{n}\le N}\abs{\wh{f}(n)}\le\underbrace{\paren{\sum_{M+1\le\abs{n}\le N}\abs{\wh{\varphi}(n)}^2}^{1/2}}_{\le\norm{\varphi}_2}\paren{\sum_{M+1\le\abs{n}\le N}\frac{1}{n^2}}^{1/2}\xrightarrow{N,M\to\infty}0.\]
\end{Proof}


\begin{theorem}[一様収束の十分条件]\label{thm-uniform-convergence-of-Fourier-series}
    $f\in\AC(\bT)$について,$S_N[f]\to f$は$\bT$上一様収束する.
\end{theorem}
\begin{Proof}\mbox{}
    \begin{description}
        \item[問題の所在] Riemannの局所性原理と同様,\textbf{Fourier部分和の差の標準分解}
        \begin{align*}
            S_N[f](t)-f(t)&=\frac{1}{2\pi}\int^\pi_{-\pi}(f(t-s)-f(t))D_N(s)ds\\
            &=\frac{1}{2\pi}\paren{\int^\delta_{-\delta}+\int_{\delta\le\abs{s}\le\pi}}(f(t-s)-f(t))\frac{\sin\paren{N+\frac{1}{2}}s}{\sin\frac{s}{2}}ds=:A_1+A_2.
        \end{align*}
        と分解して考えると,第二項は
        \[A_2=\frac{1}{2\pi}\int^\pi_{-\pi}\underbrace{(f(t-s)-f(t))\frac{1}{\sin(s/2)}\chi_{\Brace{\abs{s}\ge\delta}}(s)}_{=:h_t(s)}\sin(N+1/2)sds\]
        とも表せ,$h:[-\pi,\pi]\to L^1(\bT)$は,$\chi_{\Brace{\abs{s}\ge\delta}}$の存在により,$h_t\in L^\infty(\bT)$より特にwell-definedで,平行移動をしているのみなので連続でもある.
        よって補題\ref{lemma-uniform-Riemann-Lebesgue}より,$t\in I$に依らずに$\wh{h_t}(n)\xrightarrow{\abs{n}\to\infty}0$.この系として$A_2\xrightarrow{N\to\infty}0$を得る.
        問題は第一項の収束である.
        \item[問題の解決] 第一項の収束は,$I_N(s)$は偶関数$D_N(s)$の原始関数より奇関数であることに注意すれば,
        \begin{align*}
            A_1&=\frac{1}{2\pi}\int^\delta_{-\delta}(f(t-s)-f(t))D_N(s)ds\\
            &=\frac{1}{2\pi}\Square{(f(t-s)-f(t))I_N(s)}^\delta_{-\delta}+\frac{1}{2\pi}\int^\delta_{-\delta}\varphi(t-s)I_N(s)ds\\
            &=\frac{1}{2\pi}\paren{(f(t-\delta)-f(t))I_N(\delta)-(f(t+\delta)-f(t))I_N(-\delta)}+\frac{1}{2\pi}\int^\delta_{-\delta}\varphi(t-s)I_N(s)ds\\
            &=\frac{1}{2\pi}\paren{I_N(\delta)(f(t-\delta)+f(t+\delta)-2f(t))+\int^\delta_{-\delta}\varphi(t-s)I_N(s)ds}.
        \end{align*}
        に注目して示せる.
        \begin{enumerate}
            \item $f\in\AC(\bT)$は特に一様連続なので,$\exists_{\delta\in(0,\pi)}\;\forall_{t\in\bT}\;\forall_{s\in[-\delta,\delta]}\;\abs{f(t-s)-f(t)}<\ep$.すると$\forall_{t\in\bT}\;\abs{f(t+\delta)+f(t-\delta)-2f(t)}<2\ep$.
            \item $\exists_{\varphi_0\in C(\bT)}\;\norm{\varphi-\varphi_0}_{L^1(\bT)}<\ep$.これに対してさらに$\delta\norm{\varphi_0}_\infty<\ep$も満たすように$\delta>0$をより小さくする.すると,$\abs{I_N(s)}\le C$\ref{lemma-boundedness-of-integral-of-Dirichlet-kernel}に注意して
            \begin{align*}
                \Abs{\int^\delta_{-\delta}\varphi(t-s)I_N(s)ds}&\le C\int^\delta_{-\delta}\abs{\varphi(t-s)}ds<C\int^\delta_{-\delta}\abs{\varphi_0(t-s)}ds+2\pi C\ep\\
                &\le 2C\delta\norm{\varphi_0}_\infty+2\pi C\ep<10C\ep.
            \end{align*}
        \end{enumerate}
        以上より,$\abs{A_1}<\frac{1}{2\pi}(2C\ep+10C\ep)$.
    \end{description}
\end{Proof}

\subsection{Riemann-Lebesgueの補題の一様評価}

\begin{lemma}\label{lemma-uniform-Riemann-Lebesgue}
    $[a,b]=I\subset\R$を有界閉区間,$h:I\to L^1(\bT)$を連続写像とする.
    すると,$\wh{h_t}(n)$は$\abs{n}\to\infty$の極限に関して,$t\in I$上一様に$0$に収束する.
\end{lemma}
\begin{Proof}
    任意に$\ep>0$をとる.
    \begin{enumerate}
        \item $h(I)\subset L^1(\bT)$はコンパクトだから,$I$の点列$a=t_0<\cdots<t_m=b$が存在して,$\forall_{j\in[m-1]}\;\forall_{t\in[t_j,t_{j+1}]}\;\norm{h_t-t_{t_j}}_{L^1(\bT)}<\ep/2$.
        \item さらに有限個選び出した点$h_{t_j}$についても,それぞれ三角多項式$P_j$が存在して,$\norm{h_{t_j}-P_j}_{L^1(\bT)}<\ep/2$を満たす(Weierstrassの定理\ref{cor-Weierstrass}).
        \item $N:=\max_{j\in[m-1]}\deg(P_j)$とすると,$\F:L^1(\bT)\to l^\infty(\Z)$は縮小写像である\ref{cor-Riemann-Lebesgue}から,
        \[\forall_{\abs{n}>N}\;\forall_{j\in[m-1]}\;\forall_{t\in[t_j,t_{j+1}]}\;\abs{\wh{h_t}(n)}=\abs{\wh{h_t}(n)-\wh{P_j}(n)}\le\norm{h_t-P_j}_{L^1(\bT)}<\ep.\]
    \end{enumerate}
\end{Proof}

\subsection{Dirichlet核の不定積分の有界性}

\begin{lemma}\mbox{}\label{lemma-boundedness-of-integral-of-Dirichlet-kernel}
    \begin{enumerate}
        \item $\int^\infty_0\frac{\sin x}{x}dx=\frac{\pi}{2}$.
        \item 不定積分$I_N(t):=\int^t_0D_N(s)ds\quad(t\in[-\pi,\pi])$について,$\exists_{C>0}\;\forall_{N\in\N}\;\forall_{t\in[-\pi,\pi]}\;\abs{I_N(t)}\le C$が成り立つ.
    \end{enumerate}
\end{lemma}
\begin{Proof}
    まず,Dirichlet核の積分を次のように分解する:
    \begin{align*}
        I_N(t)&=\int^t_0\frac{\sin\paren{N+\frac{1}{2}}s}{\sin\frac{s}{2}}ds\\
        &=\int^t_0\paren{\frac{1}{s/2}-\frac{1}{\sin(s/2)}}\sin\paren{N+\frac{1}{2}}sds+2\int^t_0\frac{\sin\paren{N+\frac{1}{2}}s}{s}ds=:A_1+A_2.
    \end{align*}
    さらに,この第二項は変数変換$x=\paren{N+\frac{1}{2}}s$により,
    \[A_2=2\int^{\paren{N+\frac{1}{2}}t}_0\frac{\sin x}{x}dx\]
    と表せる.
    $D_N$は偶関数だから,$t\in[0,\pi]$と仮定して示せば良い.
    \begin{enumerate}
        \item Dirichlet核は正規化されているから,$\forall_{N\in\N}\;I_N(\pi)=\frac{2\pi}{2}=\pi$.
        Riemann-Lebesgueの補題の系\ref{lemma-cor-of-Riemann-Lebesgue}より,$A_1\xrightarrow{N\to\infty}0$であることに注意すると,
        \[\lim_{N\to\infty}\int^{\paren{N+\frac{1}{2}}t}_0\frac{\sin x}{x}dx=\frac{\pi}{2}.\]
        あとは,これを連続化できること
        \[\forall_{\lambda\in((N+1/2)\pi,(N+3/2)\pi)}\;\Abs{\int^\lambda_{\paren{N+\frac{1}{2}}\pi}\frac{\sin x}{x}dx}\le\frac{1}{\paren{N+\frac{1}{2}}\pi}\xrightarrow{N\to\infty}0.\]
        より従う.
        \item 第1項$A_1$の被積分関数
        \[\frac{1}{s/2}-\frac{1}{\sin(s/2)}=\frac{\sin\frac{s}{2}-\frac{s}{2}}{\frac{s}{2}\sin\frac{s}{2}}=:\frac{f(s)}{g(s)}\]
        も$\ocinterval{0,\pi}$上有界である.
        実際,$f,g$は片側近傍$(0,\delta)$上で可微分かつ$g'$は消えず,l'Hospitalの定理を繰り返し適用することにより,
        \[\lim_{s\to0}\frac{f(s)}{g(s)}=\lim_{s\to0}\frac{f''(s)}{g''(s)}=\lim_{s\to0}\frac{-\frac{1}{4}\sin(s/2)}{\cos(s/2)/2-s\sin(s/2)/8}=0.\]
        を得る.
        よって,$t\in[0,\pi]$に依らず有界.第二項の有界性は(1)による.
    \end{enumerate}
\end{Proof}

\subsection{絶対連続関数の部分積分公式}

\begin{lemma}\label{lemma-partial-integral-law}
    $f,g\in\AC([-\pi,\pi])$について,$\varphi,\psi\in L^1([-\pi,\pi])$をその微分とする.このとき,
    \[\forall_{[a,b]\subset[-\pi,\pi]}\;\int^b_af(t)\psi(t)dt=f(b)g(b)-f(a)g(a)-\int^b_a\varphi(t)g(t)dt.\]
\end{lemma}
\begin{Proof}
    $\varphi,\psi\in C([-\pi,\pi])$の場合は成立.一般の場合は,$C([-\pi,\pi])$の$L^1([-\pi,\pi])$上の稠密性による.
\end{Proof}

\subsection{古典的な結果}

\begin{corollary}[Dini-Lipschitz (1878)]
    $f\in C(\bT)$の連続度$\om(\delta)=\sup_{\abs{x-y}<\delta}\abs{f(x)-f(y)}$について,
    \[\lim_{\delta\to0}\om(\delta)\log\delta=0\]
    ならば$S_n[f]\to f$は一様収束する.
\end{corollary}

\section{Riemannの局所性原理:一様収束と各点収束の狭間}

\begin{tcolorbox}[colframe=ForestGreen, colback=ForestGreen!10!white,breakable,colbacktitle=ForestGreen!40!white,coltitle=black,fonttitle=\bfseries\sffamily,
    title=]
    Fourier部分和$S_N[f]$の収束は一様性に取り残される点があり,これをGibbs現象という.
    どのような点で収束性が悪くなるのか?
    \begin{enumerate}
        \item Fourier級数$S_N[f]$の収束は,注目点$x\in\bT$の近傍での$f$の様子のみに依存する($f\mapsto S_N[f]$は畳み込み変換なので当然ではある).
        \item すなわち,ある開区間$I\osub\bT$上で$f=0\;\In I$ならば,$S_N[f]\to 0$は$I$上でコンパクト一様である\ref{thm-Riemann-locality}.
        \item 開区間$I\osub\bT$上で$C^1$-級だったら,$S_N[f]\to f$は$I$上でコンパクト一様である.
        \item 
    \end{enumerate}
\end{tcolorbox}    

\subsection{原理}

\begin{tcolorbox}[colframe=ForestGreen, colback=ForestGreen!10!white,breakable,colbacktitle=ForestGreen!40!white,coltitle=black,fonttitle=\bfseries\sffamily,
title=]
    $t_0\in\bT$の近傍で$f=g$ならば,$S_N[f](t_0)-S_N[g](t_0)\to0$である.
    これは$f-g$が$t_0$の近傍で零である際特に可微分であるため,Diniの定理\ref{cor-Fourier-sum-converges-at-differentiable-points}からすぐわかる.
    これだけでなく,局所的にコンパクト一様に収束することもわかる.
\end{tcolorbox}

\begin{theorem}\label{thm-Riemann-locality}
    $f,g\in L^1(\bT)$は開区間$J\osub\bT$上で等しいとする:$f=g\;\on\bT$.
    このとき,$S_N[f]-S_N[g]$は$J$上$0$にコンパクト一様収束する.
\end{theorem}
\begin{Proof}\mbox{}
    \begin{description}
        \item[問題の所在] 任意の$I=[a,b]\subset J$と$[a-\delta,b+\delta]\subset J$を満たす$\delta>0$を取る.
        このとき,
        \begin{align*}
            \forall_{t\in I}\quad S_N[f](t)-S_N[g](t)&=\frac{1}{2\pi}\int^\pi_{-\pi}(f(t-s)-g(t-s))D_N(s)ds\\
            &=\frac{1}{2\pi}\paren{\int^\delta_{-\delta}+\int_{\delta\le\abs{s}\le\pi}(f(t-s)-g(t-s))D_N(s)ds}=:A_1+A_2
        \end{align*}
        と分解できるが,$\delta>0$は十分小さくとったので,第一項は$A_1=0$になる.そこで,第二項$A_2$が$0$に$I$上一様収束することを示せば良い.
        \item[解決] $\chi_\delta:=\chi_{[-\pi,-\delta]\cup[\delta,\pi]}$とすると,
        \[A_2=\frac{1}{2\pi}\int^\pi_{-\pi}(f(t-s)-g(t-s))\frac{1}{\sin\frac{s}{2}}\chi_{\delta}(s)\sin\paren{N+\frac{1}{2}}ds.\]
        と表せるが,$h_t(s):=(f(t-s)-g(t-s))\frac{1}{\sin\frac{s}{2}}\chi_{\delta}(s)$は,対応$h:I\to L^1(\bT)$としては$t\in I$に関して平行移動させているのみであるから,連続になる(平行移動はいつでも位相群に位相同型を定める).
        よって補題\ref{lemma-uniform-Riemann-Lebesgue}より,$\wh{h_t}(s)$は$t\in I$について一様に$\wh{h_t}(s)\xrightarrow{N\to\infty}0$.
        この虚部に注目すれば従う.
    \end{description}
\end{Proof}

\subsection{応用}

\begin{tcolorbox}[colframe=ForestGreen, colback=ForestGreen!10!white,breakable,colbacktitle=ForestGreen!40!white,coltitle=black,fonttitle=\bfseries\sffamily,
title=]
    $\bT$上絶対連続(特に$C^1$-級)ならば一様収束することを見た.
    では局所的に$C^1$-級だったら,局所的には一様に収束することが期待される.
\end{tcolorbox}

\begin{corollary}[局所化された一様収束原理]\label{cor-Riemann-locality}
    $f\in L^1(\bT)$は開区間$J\osub\bT$上$C^1$-級とする.このとき,$S_N[f]$は$f$に$J$上コンパクト一様収束する.
\end{corollary}
\begin{Proof}
    任意の閉区間$I\subset J$をとる.ある開区間$I\subset J'\subset J$について,
    $f=g\;\on J'$を満たす$g\in C^1(\bT)$が存在する.
    このとき,定理\ref{thm-uniform-convergence-of-Fourier-series}より,$S_N[g]$は$g$に$I$上一様収束する.
    また,Riemannの局所性原理\ref{thm-Riemann-locality}より,$I$上一様に$S_N[f]-S_N[g]\to0$.
    2つ併せて,$I$上一様に$S_N[f]\to g=f$.
\end{Proof}

\section{$L^2(\bT)$上のFourier変換}

\begin{tcolorbox}[colframe=ForestGreen, colback=ForestGreen!10!white,breakable,colbacktitle=ForestGreen!40!white,coltitle=black,fonttitle=\bfseries\sffamily,
title=]
    Fourier変換の$L^2(\bT)\mono L^1(\bT)$への制限
    $\F:L^2(\bT)\to l^2(\Z)$はHilbの同型になる.
\end{tcolorbox}

\subsection{一般論:正規直交基底の定めるFourier変換}

\begin{definition}[CONS]
    正規直交系$E\subset H$が\textbf{完全}であるとは,次の意味で極大であることをいう:
    \[\forall_{f\in H}\;[\forall_{e\in E}\;(e|f)=0\Rightarrow f=0].\]
\end{definition}

\begin{proposition}
    $\{\varphi_n\}_{n\in\N}$をCONSとする.次が成り立つ:
    \begin{enumerate}
        \item Fourier展開:$\forall_{f\in H}\;f=\sum_{n\in\N}(f|\varphi_n)\varphi_n$.
        \item Fourier変換の等長性$\norm{f}^2=\sum_{n\in\N}\abs{(f|\varphi_n)}^2$.
        \item Fourier変換は内積を保つ:$\forall_{f,g\in H}\;(f|g)=\sum_{n\in\N}(f|\varphi_n)\o{g|\varphi_n}$.
    \end{enumerate}
\end{proposition}
\begin{Proof}\mbox{}
    \begin{enumerate}
        \item $S_N:=\sum_{n=0}^N(f|\varphi_n)\varphi_n\xrightarrow{N\to\infty}f\;\In H$を示せば良い.
        \begin{enumerate}
            \item 内積を用いた計算より,
            \[\norm{f-S_N}^2=\norm{f}^2-\sum_{n=0}^N\abs{(f|\varphi_n)}^2\ge0\]
            したがって,$\sum_{n\in\N}\abs{(f|\varphi_n)}^2\le\norm{f}^2$(Besselの不等式).
            \item $\{S_N\}\subset H$はCauchy列である.これは(1)に$\norm{S_N-S_M}^2=\sum_{n=M+1}^N\abs{(f|\varphi_n)}^2\to0$が含意されているためである.
            この極限を$g:=\sum_{n\in\N}(f|\varphi_n)\varphi_n\in H$とする.
            \item $(\varphi_n)$が正規直交系であるため,$\forall_{n\in\N}\;(f-g|\varphi_n)=0$.$(\varphi_n)$は完全であるため,$f-g=0$が必要.
        \end{enumerate}
        \item (1)(a)で用いた等式を$N\to\infty$とすると,
        \[0=\lim_{N\to\infty}\norm{f-S_N}^2=\norm{f}^2-\sum_{n\in\N}\abs{(f|\varphi_n)}^2.\]
        \item (1)より$f=\sum_{n\in\N}(f|\varphi_n)\varphi_n=\lim_{N\to\infty}S_N$であるが,これの内積の連続性による帰結である.
    \end{enumerate}
\end{Proof}

\subsection{特殊論:$L^2(\bT)$での消息}

\begin{corollary}
    $\{\varphi_n:=e^{int}\}_{n\in\Z}\subset L^2(\bT)$について,
    \begin{enumerate}
        \item $(\varphi_n)$はCONSである.
        \item $S_N[f]\to f\;\in L^2(\bT)$.
        \item $\F:L^2(\bT)\to l^2(\Z)$は内積を保つ線型同型である.
    \end{enumerate}
\end{corollary}
\begin{Proof}\mbox{}
    \begin{enumerate}
        \item 完全性とはFourier係数の一意性に他ならない\ref{cor-Riemann-Lebesgue}.
        \item 命題の(1)の$f=\sum_{n\in\N}(f|\varphi_n)\varphi_n$の具体的な場合である.
        \item 命題の(2)より,well-definedである.(3)より内積を保つ.また$\F$は単射であった\ref{cor-Riemann-Lebesgue}から,あとは全射性を示せば良い.
        これは,任意の$(a_n)\in l^2(\Z)$について,$S_N:=\sum_{n=-N}^Na_n\varphi_n$はCauchy列だから,
        \[\sum_{n\in\Z}a_n\varphi_n\in L^2(\bT),\quad\F\paren{\sum_{n\in\Z}a_n\varphi_n}=(a_n)\]
        より.
    \end{enumerate}
\end{Proof}

\begin{remarks}[\cite{Stein-Shakarchi-03-Fourier} Exercise 3.6]
    $a_k=\frac{1}{k}1_{\Brace{k\ge1}}(k)$が定める$L^2(\bT)$関数はRiemann可積分ではない.
\end{remarks}

\subsection{一様収束の十分条件への応用}

\begin{proposition}
    $f\in\AC(\bT)$は$L^2$-な導関数$\varphi$を持つとする:
    \[f(t)=\int^t_0\varphi(s)ds+f(0),\qquad t\in\bT.\]
    このとき,$\wh{f}\in l^1(\Z)$.
    特に,$S_N[f]\to f\;\In C(\bT)$.
\end{proposition}
\begin{Proof}
    $\wh{f}(n)=\frac{1}{in}wh{\varphi}(n)$に注意して,Cauchy-Schwarzの不等式より,任意の自然数$N>M$に対して,
    \[\sum_{M+1\le\abs{n}\le N}\abs{\wh{f}(n)}\le\underbrace{\paren{\sum_{M+1\le\abs{n}\le N}\abs{\wh{\varphi}(n)}^2}^{1/2}}_{\le\norm{\varphi}_2}\paren{\sum_{M+1\le\abs{n}\le N}\frac{1}{n^2}}^{1/2}.\]
    $1/n^2$が収束列であることより,$N,M\to\infty$において右辺は収束する.
\end{Proof}

\subsection{$L^2(\bT)$の基底}

\begin{proposition}[三角多項式]\mbox{}
    \begin{enumerate}
        \item $(e^{inx})_{n\in\Z}$,$\{\cos nx\}_{n\in\N}\cup\{\sin nx\}_{n\in\N^+}$はいずれも$L^2(\cointerval{-\pi,\pi})$の正規直交基底である.
        \item $\{\cos nx\}_{n\in\N}$,$\{\sin nx\}_{n\in\N^+}$はいずれも$L^2(\cointerval{0,\pi})$の正規直交基底である.
    \end{enumerate}
\end{proposition}

\begin{lemma}\mbox{}
    \begin{enumerate}
        \item 任意の$n,m\in\N^+$について,$\frac{1}{\pi}\int^\pi_{-\pi}\cos nx\cos mxdx=\delta(n-m)$.
        \item 任意の$n,m\in\N^+$について,$\frac{1}{\pi}\int^\pi_{-\pi}\sin nx\sin mxdx=\delta(n-m)$.
        \item 任意の$n,m\in\Z$について,$\int^\pi_{-\pi}\sin nx\cos mxdx=0$.
    \end{enumerate}
\end{lemma}
\begin{Proof}
    \begin{align*}
        e^{inx}e^{imx}&=\cos nx\cos mx-\sin nx\sin mx+i\Paren{\cos nx\sin mx+\cos mx\sin nx},\\
        e^{inx}e^{-imx}&=\cos nx\cos mx+\sin nx\sin mx+i\Paren{-\cos nx\sin mx+\cos mx\sin nx}.
    \end{align*}
    \begin{enumerate}
        \item $\cos nx\cos mx=\Re e^{inx}e^{imx}+e^{inx}e^{-imx}$であるから,
        \[(\cos nx|\cos mx)=(e^{inx}|e^{imx})+(e^{inx}|e^{-imx}).\]
        \item 同様.
        \item $\cos mx\sin nx=\Im e^{inx}e^{imx}+e^{inx}e^{-imx}$であるから,常に零である.
    \end{enumerate}
\end{Proof}

\subsection{補間定理による結果}

\begin{theorem}[Hausdorff-Young]
    $1\le p\le2$について,Fourier変換は線型作用素$\F:L^p(\bT)\to l^{p^*}(\Z)$を定め,ノルム減少的である.
\end{theorem}
\begin{Proof}
    Riesz-Thorinの補間定理による.
\end{Proof}

\section{三角級数による可積分関数の構成}

\begin{tcolorbox}[colframe=ForestGreen, colback=ForestGreen!10!white,breakable,colbacktitle=ForestGreen!40!white,coltitle=black,fonttitle=\bfseries\sffamily,
title=]
    三角級数とは,Fourier級数の奇関数・偶関数に対する特別な場合として得られる.
    偶関数と奇関数のFourier級数展開は,$(a_n),(b_n)\in c_0(\Z)$を用いて
    \[\frac{a_0}{2}+\sum_{n=1}^\infty a_n\cos nt,\qquad \sum_{n=1}^\infty b_n\sin nt\]
    の$2$と$2i$倍で表せる.これがある可積分関数$f\in L^1(\bT)$を定めるのはいつか,また$S_N[f]$が$N\to\infty$で$f$に戻ってくるくらいに正則性を持つのはいつか?
\end{tcolorbox}

\begin{remarks}\mbox{}
    \begin{enumerate}
        \item $(a_n),(b_n)\in c_0(\N)$が単調減少ならば,$\bT\setminus\{0\}$上の連続関数を定めるから,$\bT$上の関数を定める.
        が,$\bT$上可積分とは限らない\ref{prop-convergent-sine-series}.
        \item $(a_n)\in c_0(\N)$が凸数列であるとき,コサイン級数は$\bT\setminus\{0\}$上収束し,$L^1(\bT)_+$の元を定める.
        さらに$a_N\log N\to0$のとき,$S_N[f]\to f\;\In L^1(\bT)$も成り立つ\ref{thm-convergent-cosine-series}.
        \item 凸数列のうち,次の性質が,収束先として得るコサイン級数・サイン級数の可積分性を特徴づける:
        \[\sum_{n=1}^\infty(b_n-b_{n+1})\log n<\infty.\]
    \end{enumerate}
\end{remarks}

\begin{example}
    $b_n=\frac{1}{n}$とすると,$f(t)=\frac{\pi-t}{2}$となり,$S_N[f]$は$f$に戻るが$t=0$の近傍では一様ではない
    \ref{exp-pi-t-2}.
\end{example}

\subsection{共役Dirichlet核}

\begin{tcolorbox}[colframe=ForestGreen, colback=ForestGreen!10!white,breakable,colbacktitle=ForestGreen!40!white,coltitle=black,fonttitle=\bfseries\sffamily,
title=]
    Dirichlet核
    \[D_N(t)=\Re\paren{1+2\sum_{n=1}^Ne^{int}}\]
    の共役調和関数$\wt{D}_N$を考える.
\end{tcolorbox}

\begin{definition}\mbox{}
    \begin{enumerate}
        \item 次の表示から明らかに$D_N$は調和関数である:
        \[D_N(t)=\sum_{n=-N}^Ne^{int}=1+\sum_{n=1}^N(e^{int}+e^{-int})=1+2\sum_{n=1}^N\cos nt.\]
        \item 次を\textbf{共役Dirichlet核}という:
        \[\wt{D}_N(t):=\frac{1}{i}\sum_{n=1}^N(e^{int}-e^{-int})=2\sum_{i=1}^N\sin nt.\]
    \end{enumerate}
\end{definition}

\begin{lemma}\mbox{}
    \begin{enumerate}
        \item 等比数列の和公式:$\wt{D}_N(t)=\frac{\cos\frac{t}{2}-\cos\paren{N+\frac{1}{2}}t}{\sin\frac{t}{2}}$.
        \item $L^1$-ノルムの評価:$\exists_{C>0}\;\forall_{N\in\N}\;\norm{\wt{D}_N}_1\le 2\log N+C$.
    \end{enumerate}
\end{lemma}

\subsection{単調減少な係数を持つ三角級数}

\begin{tcolorbox}[colframe=ForestGreen, colback=ForestGreen!10!white,breakable,colbacktitle=ForestGreen!40!white,coltitle=black,fonttitle=\bfseries\sffamily,
title=]
    主な手法は数列に対するAbel変換=部分積分である.
    Fourier級数の問題は振動することであり,単調減少なもの(従って特に正なもの)に限れば全く怖くなく,
    必ず$\bT$上の関数を定める.が,これが可積分とは限らない.
\end{tcolorbox}

\begin{notation}
    数列$(a_n)$について,
    \begin{enumerate}
        \item 階差の差を$\De^2 a_{n+1}:=(a_n-a_{n+1})-(a_{n+1}-a_{n+2})$と表す.
        \item $\forall_{n\in\N}\;\De^2a_{n+1}\ge0$を満たすとき,\textbf{凸}であるという.
    \end{enumerate}
\end{notation}

\begin{proposition}[単調減少係数の三角級数は$\bT$上の関数を定める]\label{prop-convergent-sine-series}
    単調減少な正数列$(a_n),(b_n)\in c_0(\N)$について,級数
    \[\frac{a_0}{2}+\sum_{n=1}^\infty a_n\cos nt,\qquad \sum_{n=1}^\infty b_n\sin nt\]
    はいずれも$[-\pi,\pi]\setminus\{0\}$上コンパクト一様収束する.
    特に,これらは$\bT\setminus\{0\}$上の連続関数を定める.
\end{proposition}
\begin{Proof}
    コサイン部分和
    \[s_N:=\frac{a_0}{2}+\sum_{n=1}^Na_n\cos nt\]
    の極限について考える.
    本質は$\delta\le\abs{t}\le\pi$における一様な評価
    \[\abs{D_N(t)}\le\frac{1}{\abs{\sin t/2}},\quad \abs{\wt{D}_N(t)}\le\frac{2}{\abs{\sin t/2}}.\]
    に依拠するので,サイン部分和についてもまったく同様.
    \begin{enumerate}[{Step}1]
        \item $D_n(t)=1+2\cos t+2\cos 2t+\cdots+2\cos nt$に注意すれば,部分積分より,
        \begin{align*}
            2s_N(t)&=a_0+\sum_{n=1}^Na_n\cdot 2\cos nt=\sum_{n=0}^Na_n(D_n(t)-D_{n-1}(t))\\
            &=\sum_{n=0}^{N-1}(a_n-a_{n+1})D_n(t)+a_ND_N(t)-a_0\cdot 0.
        \end{align*}
        \item 任意の$t\in[-\pi,\pi]\setminus\{0\}$について$\abs{D_N(t)}\le\frac{1}{\abs{\sin t/2}}$と評価できるから,まず$a_ND_N(t)\xrightarrow{N\to\infty}0$.さらにこれは$\abs{t}\ge\delta$について一様.
        続いて,
        \[\Abs{\sum_{n=M}^{N-1}(a_n-a_{n+1})D_n(t)}\le\sum_{n=M}^{n-1}\frac{a_n-a_{n+1}}{\abs{\sin t/2}}=\frac{a_M-a_N}{\abs{\sin t/2}}\xrightarrow{N,M\to\infty}0.\]
        これも一様である.
    \end{enumerate}
\end{Proof}
\begin{remarks}[凸数列への注目]\label{remarks-convex-sequence}
    さらにもう一度Abel変換することで,
    \[2s_N(t)=\sum_{n=0}^{N-2}(n+1)\Delta^2a_{n+1}K_n(t)+N(a_{N-1}-a_N)K_{N-1}(t)+a_ND_N(t).\]
    を得る.
\end{remarks}

\begin{lemma}[凸数列は単調減少する]
    $(a_n)\in c_0(\N)$を凸数列とする.
    \begin{enumerate}
        \item 単調減少な正数列である.
        \item $\lim_{n\to\infty}n(a_n-a_{n+1})=0$.
        \item $\sum_{n\in\N}(n+1)\Delta^2a_{n+1}=a_0$.
    \end{enumerate}
\end{lemma}
\begin{Proof}\mbox{}
    \begin{enumerate}
        \item 階差が減少していくのであるから,
        \[a_n-a_{n+1}\ge a_{n+1}-a_{n+2}\ge\cdots\to0.\]
        \item $n$が偶数のとき,階差は減少していくのであるから,
        \begin{align*}
            0\le 2n(a_{2n}-a_{2n+1})&\le 2\Paren{(a_{2n}-a_{2n+1})+(a_{2n-1}-a_{2n})+\cdots+(a_n-a_{n+1})}\\
            &=2(a_n-a_{2n+1}).
        \end{align*}
        と評価できる.$(a_n)$は収束列と仮定したから,最右辺は$0$に収束する.
        $n$が奇数の場合も同様.
        \item $N\ge2$について,部分積分より,
        \[a_0-a_N=\sum_{n=0}^{N-1}(a_n-a_{n+1})cdot1=\sum_{n=0}^{N-2}\Delta^2a_{n+1}(n+1)+(a_{N-1}-a_N)N-(a_0-a_1)0.\]
        両辺の$N\to\infty$の極限を考えると,左辺は$a_0$で,右辺は(2)より第一項しか残らない.
    \end{enumerate}
\end{Proof}
\begin{remarks}
    これは$f''\ge0,\lim_{x\to\infty}f(x)=0$を満たす$f\in C^2(\R^+)$が満たす次の性質の対応物である:
    \begin{enumerate}
        \item $f$は単調減少である.
        \item $\lim_{x\to\infty}xf'(x)=0$.
        \item $\int_{\R_+}xf''(x)dx=f(0)$.
    \end{enumerate}
\end{remarks}

\begin{proposition}[さらに$\bT$上の連続関数を定めるための必要十分条件 猪 定理6.2]
    $(a_n),(b_n)\in c_0(\N)$を単調減少列とする.
    \begin{enumerate}
        \item コサイン級数$\sum_{n=1}^\infty a_n\cos nt$について,次は同値:
        \begin{enumerate}
            \item $\bT$上一様に収束する.
            \item $(a_n)\in l^1(\N)$である.
        \end{enumerate}
        \item サイン級数$\sum_{n=1}^\infty b_n\sin nt$について,次は同値:
        \begin{enumerate}
            \item $\bT$上一様に収束する.
            \item $nb_n\to0$.
        \end{enumerate}
    \end{enumerate}
\end{proposition}

\subsection{三角級数が可積分関数を定めるための条件}

\begin{tcolorbox}[colframe=ForestGreen, colback=ForestGreen!10!white,breakable,colbacktitle=ForestGreen!40!white,coltitle=black,fonttitle=\bfseries\sffamily,
title=]
    係数が単調減少ならば$\bT$上の関数を定める.その際の収束は$\bT\setminus\{0\}$上の広義一様収束であるから,
    極限が可積分であるためにはあとは$L^1(\bT)$-収束するための条件に等しい.
    logの速度よりも速く収束する凸数列ならば,Fourier部分和は$L^1$-収束する.
\end{tcolorbox}

\begin{theorem}[コサイン級数が可積分関数を定めるための条件]\label{thm-convergent-cosine-series}
    $(a_n)\in c_0(\N)$を単調減少な正数列とし,これが定める$\bT$上の関数/$\bT\setminus\{0\}$上の連続関数
    \[f(t):=\frac{a_0}{2}+\sum_{n=1}^\infty a_n\cos nt,\qquad (t\in\bT\setminus\{0\}).\]
    を考える.
    \begin{enumerate}
        \item $(a_n)$が凸ならば$f\in L^1(\bT)_+$であり,
        任意の$N\in\N$について,
        \[S_N[f](t)=\frac{a_0}{2}+\sum_{n=1}^Na_n\cos nt,\qquad(t\in\bT).\]
        \item さらに$a_N\log N\to0$ならば,そのときに限り,$S_N[f]\to f\;\in L^1(\bT)$.
        \item $(a_n)$が凸でなくても,
        \[\sum_{n=1}^\infty(a_n-a_{n+1})\log n<\infty\]
        を満たすならば$f\in L^1(\bT)$であり,$S_N[f](t)=s_N(t)\to f\;\In L^1(\bT)$が成り立つ.
    \end{enumerate}
\end{theorem}
\begin{Proof}\mbox{}
    \begin{enumerate}
        \item 
        \begin{enumerate}[{Step}1]
            \item コサイン部分和のFejer核による表示\ref{remarks-convex-sequence}
            \[2s_N(t)=\sum_{n=0}^{N-2}(n+1)\Delta^2a_{n+1}K_n(t)+N(a_{N-1}-a_N)K_{N-1}(t)+a_ND_N(t).\]
            について,$N\to\infty$の極限を考えると,
            右辺第一項は収束列を係数に持つ$K_n(t)$の線型和であるが,$\norm{K_n}=1$より$L^1$-収束する(これは$L^1(\bT)$の完備性の特徴付けである).
            さらに,任意の$t\ne0$についても,評価$0\le K_n(t)\le\frac{\sin^2\frac{t}{2}}{1+n}$より各点収束もわかる.
            第二項も補題より,$0$に収束する列を係数に持つ$K_n(t)$の線型和であるから,$0$に$L^1(\bT)$-収束かつ$t\ne0$で各点収束する.
            第三項は任意の$t\ne0$で$0$に収束するから,総じて
            \[2f(t)=\sum_{n=0}^\infty(n+1)\Delta^2a_{n+1}K_n(t)(\ge0).\]
            このとき,$\partial B\subset L^1(\bT)$の元の収束列に関する線型和であるから,
            右辺は$\bT$上可積分である.
            \item 任意の$n\in\N$について,$f$のFourier係数は
            補題より
            \begin{align*}
                2\wh{f}(n)&=\sum_{j=0}^\infty(j+1)\Delta^2a_{j+1}\wh{K}_j(n)=\sum_{j=n}^\infty(j+1)\Delta^2a_{j+1}\paren{1-\frac{n}{j+1}}\\
                &=\sum_{j=n}^\infty(j+1-n)\Delta^2a_{j+1}=\sum_{j=0}^\infty(j+1)\Delta^2a_{j+n+1}=a_n.
            \end{align*}
            $f$は偶関数であるから,定理の主張を得る.
        \end{enumerate}
        \item (1)より,$S_N[f]=s_N$と戻ってくるから,あとはStep1の第3項が$0$に$L^1$-収束することをいえばよく,
        そのための必要十分条件が$a_N\log N\to0$である\ref{prop-norm-of-Dirichlet-kernel}.
    \end{enumerate}
\end{Proof}

\begin{theorem}
    $(b_n)\in c_0(\N)$を単調減少な正数列とし,これが定める$\bT$上の関数/$\bT\setminus\{0\}$上の連続関数
    \[g(t):=\sum_{n=1}^\infty b_n\sin nt,\qquad (t\in\bT\setminus\{0\}).\]
    を考える.
    \begin{enumerate}
        \item 次を満たすとき,(そしてその時に限り\ref{thm-characterization-of-integrability-of-sine-series},)$g\in L^1(\bT)$かつ$S_N[g](t)=\sum_{n=1}^Nb_n\sin nt$が成り立つ:
        \[\sum_{n=1}^\infty(b_n-b_{n+1})\log n<\infty.\]
        \item $S_N[g]\to g\;\In L^1(\bT)$.
    \end{enumerate}
\end{theorem}

\subsection{可積分性の特徴付け}

\begin{proposition}[\cite{Katznelson-Fourier} Exercise 4.6]
    \[\sum_{n=2}^\infty\frac{\sin nt}{\log n}\]
    は任意の$t\in\bT$について収束するが,$L^1(\bT)$の元のFourier級数とはなりえない.
\end{proposition}
\begin{Proof}
    係数は単調減少であるから,任意の$t\in\bT\setminus\{0\}$で収束して,$\bT\setminus\{0\}$上の連続関数を定める.
    なお,$t=0$の場合は$0$に収束する.
    しかし,実はこの関数は可積分ではない.
    実際,次の命題の要件$\sum_{n=1}^\infty(b_n-b_{n+1})\log n<\infty$を満たすためには,
    $b_n\log n\to0$が必要であるが,これは満たされない.
\end{Proof}

\begin{proposition}[\cite{Stein-Shakarchi-03-Fourier} Exercise 3.7]
    \[\sum_{n=2}^\infty\frac{\sin nt}{n^\al},\qquad 0<\al<1.\]
    は任意の$t\in\bT$について収束するが,$L^1(\bT)$の元のFourier級数とはなりえない.
\end{proposition}

\begin{theorem}\label{thm-characterization-of-integrability-of-sine-series}
    $(b_n)_{n\in\N}\in c_0(\N)$は単調減少列とすると,
    \[f(t)=\sum_{n\in\N^+}b_n\sin nt\]
    は任意の$t\in\bT$で収束する($f(0)=0$に注意).これについて,次は同値:
    \begin{enumerate}
        \item $f\in L^1(\bT)$.
        \item $\sum_{n=1}^\infty(b_n-b_{n+1})\log n<\infty$.
        \item $\sum_{n=1}^\infty\frac{b_n}{n}<\infty$.
    \end{enumerate}
\end{theorem}
\begin{Proof}
    (2)が成り立つならば$b_n\log n\to0$が成り立ち,これは上の定理による.
    よって,(1)$\Rightarrow$(2)を示せばよい.サイン部分和
    \[\wt{s}_N(t)=\sum_{n=1}^Nb_n\sin nt\]
    の部分積分
    \[2\wt{s}_N(t)=\sum_{n=1}^N(b_n-b_{n+1})\wt{D}_n(t)+b_n\wt{D}_N(t).\]
    に対して,さらに変形
    \[\wt{s}^*_N(t):=\frac{wt{s}_{N-1}(t)+\wt{s}_N(t)}{2}\]
    を考えると,
    \[2\wt{s}^*_N(t)=\sum_{n=1}^{N-1}(b_n-b_{n+1})\wt{D}^*_n(t)+b_N\wt{D}^*_N(t).\]
    これが,$t\ne0$のとき$f(t)$に収束するから,
    \[2f(t)=\sum_{n=1}^\infty(b_n-b_{n+1})\wt{D}^*_n(t).\]
    を得る.補題より,
    $f\in L^1(\bT)$ならば
    \[\sum_{n=1}^\infty(b_n-b_{n+1})\log n<\infty.\]
    が必要であることがわかる.

    (2)$\Leftrightarrow$(3)は一般の単調減少な正数列について成り立つ性質である.
\end{Proof}
\begin{remarks}
    実はこの命題のコサイン級数版は正しくない.
\end{remarks}

\begin{lemma}
    変形共役Dirichlet核
    \[\wt{D^*}_N(t)=\frac{\wt{D}_{N-1}(t)+\wt{D}_N(t)}{2}=\frac{1-\cos Nt}{\tan\frac{t}{2}}\ge0.\]
    について,任意の$N\in\N$に対して,
    \[\int^\pi_0\wt{D^*}_N(t)dt\ge2\log N-2.\]
\end{lemma}

\section{Hardy空間}

\begin{tcolorbox}[colframe=ForestGreen, colback=ForestGreen!10!white,breakable,colbacktitle=ForestGreen!40!white,coltitle=black,fonttitle=\bfseries\sffamily,
title=]
    $f\in L^1(\bT)$が$f\in C(\bT)$であるための,必要条件は$\wh{f}\in l^2(\Z)$であり,十分条件は$\wh{f}\in l^1(\Z)$で,いずれもこれ以上改善できない.
    Fourier係数の減衰速度によって$f$の滑らかさを統制できるのは,$f\in L^2(\bT)$であることのみである.

    Abel-Poisson和は調和である.これが正則になるための必要十分条件の観察を通じて,Hilbの同型
    $\A_r=P_r*:{}^+\!L^2(\bT)\to H^2(\Delta)$を構築する.
\end{tcolorbox}

\subsection{Abel-Poisson和が正則であるための条件}

\begin{notation}
    Poisson各$P_r(\theta)\;(re^{i\theta}\in\Delta)$自身も$\Delta$上の調和関数であると同時に,
    畳み込みによって,調和関数の境界値からの対応$\A:=P_r*:L^1(\bT)\to H(\Delta)$を与え,これをPoisson積分というのであった.
\end{notation}

\begin{proposition}
    $f\in L^1(\bT)$とそのPoisson積分$u:=\A[f]$について,
    次は同値:
    \begin{enumerate}
        \item $u\in\O(\Delta)$.
        \item $f$の負のFourier係数が消えている:$\forall_{n\in\N^+}\;\wh{f}(-n)=0$.
    \end{enumerate}
\end{proposition}
\begin{Proof}\mbox{}
    \begin{description}
        \item[(2)$\Rightarrow$(1)] Abel-Poisson和は
        \[\A_r[f](\theta)=\sum_{n\in\Z}r^{\abs{n}}\wh{f}(n)e^{in\theta}=\sum_{n\in\N}\wh{f}(n)z^n+\sum_{n\in\N^+}\wh{f}(-n)\o{z}^n,\qquad z:=re^{i\theta}\]
        と表せる\ref{prop-Abel-Poisson-sum}ため.
        \item[(1)$\Rightarrow$(2)] 仮定より,
        \[\sum_{n\in\N^+}\wh{f}(-n)\o{z}^n=\A_r[f]-\sum_{n\in\N}\wh{f}(n)z^n\]
        は正則関数の差であるから,左辺も正則.しかし,明らかに左辺はその複素共役が正則である.
        これは,左辺が定数であることを意味する.$z=0$の値を考えることで,左辺は実は$0$である.すなわち,$\forall_{n\in\N^+}\;\wh{f}(-n)=0$.
    \end{description}
\end{Proof}

\subsection{Abel-Poisson作用素の特徴付け}

\begin{observation}
    $f\in L^2(\bT)$を考える.調和関数に関する最大値の定理,あるいは,$\A_r=P_r*:L^2(\bT)\to H(\Delta)$がノルム減少的であることより,
    \[\sup_{0\le r<1}\int_\bT\abs{F(re^{i\theta})}^2d\theta<\infty.\]
    が成り立つ.実はこの条件は$L^2$関数のPoisson積分として得られる$\Delta$上の調和関数を特徴付ける.
\end{observation}

\begin{definition}[Hardy space]
    \[H^2(\Delta):=\Brace{u\in\O(\bT)\;\middle|\; \sup_{r\in(0,1)}\int_\bT\abs{u(re^{i\theta})}^2d\theta<\infty}\]
    を単位円板上の\textbf{Hardy空間}という.また,$L^2(\bT)$の閉部分空間を
    \[{}^+\!L^2(\bT):=\Brace{f\in L^2(\bT;\C)\mid\forall_{n\in\N^+}\;\wh{f}(-n)=0}\]
    とする.
\end{definition}

\begin{theorem}[Hardy空間の特徴付け]
    関数$F:\Delta\to\C$について,次の条件は同値:
    \begin{enumerate}
        \item $F\in H^2(\Delta)$.
        \item ある$\{a_n\}\subset l^2(\Z)$が存在して,$F(z)=\sum_{n\in\N}a_nz^n$と冪級数展開される.
        \item $F$はある$f\in{}^+\!L^2(\bT;\C)$のPoisson積分である.
    \end{enumerate}
    このとき,次が成り立つ:
    \[\lim_{r\nearrow1}\int_\bT\abs{F(re^{i\theta})}^2d\theta=2\pi\sum_{n\in\N}\abs{a}^2=\int_\bT\abs{f(\theta)}^2d\theta=:\norm{F}_{H^2}.\]
    特に,$f\in{}^+\!L^2(\bT)$は一意に定まる.
\end{theorem}
\begin{remarks}
    これはすなわち,$H^2(\Delta),l^2(\Z),{}^+\!L^2(\bT)$がいずれもBanach空間として同型であることを言っている.
    さらに$\A_r=P_r*:{}^+\!L^2(\bT)\to H^2(\Delta)$はユニタリである.
\end{remarks}

\subsection{Cauchy核}

\begin{definition}
    $f\in{}^+\!L^2(\bT)$について,Cauchyの積分公式より,
    \[F(z)=\frac{1}{2\pi i}\int_{\partial\Delta}\frac{f(\zeta)}{\zeta-z}d\zeta=\frac{1}{2\pi}\int_\bT\frac{f(\varphi)}{1-re^{i(\theta-\varphi)}}d\varphi=:\frac{1}{2\pi}(C_r*f)(\theta),\quad z=re^{i\theta}.\]
    で定まる積分核
    \[C_r(\theta):=\frac{1}{1-re^{i\theta}}=\sum_{n\in\N}r^ne^{in\theta},\quad0\le r<1.\]
    を\textbf{Cauchy核}という.
\end{definition}

\begin{theorem}[Cauchy核の性質]
    射影$\pi:L^2(\bT)\epi{}^+\!L^2(\bT)$について,
    \begin{enumerate}
        \item 任意の$f\in L^2(\bT)$に対して,$(C_r*f)(\theta)=(P_r*\pi f)(\theta)$.
        \item $\pi f$は次の$L^2$-極限に等しい:
        \[(\pi f)(\theta)=\frac{1}{2\pi}\lim_{r\nearrow 1}\int_\bT\frac{f(\varphi)}{1-re^{i(\theta-\varphi)}}d\varphi.\]
    \end{enumerate}
\end{theorem}

\subsection{調和関数の場合}

\begin{notation}
    Hardy空間の調和関数における対応物を
    \[\H^2(\Delta):=\Brace{F\in\H(\bT)\;\middle|\; \sup_{r\in(0,1)}\int_\bT\abs{F(re^{i\theta})}^2d\theta<\infty}\]
    と表す.
\end{notation}

\begin{theorem}
    関数$F:\Delta\to\R$について,次は同値:
    \begin{enumerate}
        \item $F\in\H^2(\Delta)$.
        \item ある$f\in L^2(\bT)$が存在して,そのPoisson積分として表せる:$F(r\cos\theta,r\sin\theta)=\A_r[f](\theta)$.
    \end{enumerate}
    このとき,次が成り立ち,$f$は一意的に存在する:
    \[\lim_{r\nearrow1}\int_\bT\abs{F(re^{i\theta})}^2d\theta=\int_\bT\abs{f(\theta)}^2d\theta.\]
\end{theorem}

\subsection{一般の$p$について}

\begin{tcolorbox}[colframe=ForestGreen, colback=ForestGreen!10!white,breakable,colbacktitle=ForestGreen!40!white,coltitle=black,fonttitle=\bfseries\sffamily,
title=]
    $p\ge1$について,閉部分空間への埋め込み$H^p(\Delta)\mono L^p(\bT)$が存在する.
\end{tcolorbox}

\begin{definition}[Nevanlinna space]\mbox{}
    \begin{enumerate}
        \item $0<p<\infty$について,
        \[\norm{f}_{H^p}^p:=\sup_{r\in(0,1)}\dint_{\bT}\abs{f(re^{i\theta})}^pd\theta=\lim_{r\to1}\dint_{\bT}\abs{f(re^{i\theta})}^pd\theta,\qquad f\in\O(\Delta).\]
        と定め,これが有限になる正則関数の全体を$H^p(\Delta)\subset\O(\Delta)$とする.
        \item $p=\infty$のときは,単に一様ノルム$\norm{f}_{H^\infty}:=\sup_{z\in\Delta}\abs{f(z)}$とする.
        \item 次の値が有限になる$\Delta$上の有理型関数の全体を$\cN$で表す:
        \[\norm{f}_\cN:=\sup_{r\in(0,1)}\dint\log^+\abs{f(re^{i\theta})}d\theta,\qquad\log^+(z):=\log(z)\lor0.\]
    \end{enumerate}
    このとき,$p\le q\in{(0,\infty]}$について,$H^q\subset H^p$となることは,確率空間上の$L^p$-空間に似ている.なお,任意の$p\in{(0,\infty]}$に対して$H^p\subset\cN$.
\end{definition}
\begin{remark}
    $p<1$について,$\norm{-}_{H^p}^p$は三角不等式と$p$-次の斉次性を満たすため,計量を定めるが,$\norm{-}_{H^p}$では三角不等式が成り立たない.
\end{remark}

\begin{theorem}[\cite{Katznelson-Fourier} Th'm 3.12]
    任意の$p\ge1$と関数$f:\Delta\to\C$に対して,次は同値:
    \begin{enumerate}
        \item $f\in H^p(\Delta)$.
        \item ある$f\in{}^+\!L^p(\bT)$のPoisson積分である.
    \end{enumerate}
\end{theorem}

\chapter{$L^1(\R)$のFourier解析}

\begin{quotation}
    \begin{description}
        \item[Fourier変換] $\F:L^1(\R)\to C_0(\wh{\R})$の像は$C_0(\wh{\R})$の真の部分代数となる.
        \item[Fourier-Stieltjes変換] $L^1(\R)\mono M(\R)$に沿った連続延長を持つ.このときの手法で,$L^1(\R)\mono\S'(\R)$に最終的に延長される.
    \end{description}
\end{quotation}

\begin{notation}\mbox{}
    \begin{enumerate}
        \item $x\in\R,\xi\in\wh{\R}=\R$として座標を区別する.実際,位相群の表現論によれば,$\wh{f}$の定義域は$f$の定義域$\R$の双対として捉えられ,$\R^*=\R$の別記法である.
    \end{enumerate}
\end{notation}

\section{定義と基本性質}

\subsection{Fourier変換の定義}

\begin{tcolorbox}[colframe=ForestGreen, colback=ForestGreen!10!white,breakable,colbacktitle=ForestGreen!40!white,coltitle=black,fonttitle=\bfseries\sffamily,
title=]
    一般の関数を「周期無限大の周期関数」と見ることで$\bT$の場合の極限として対応する.
    $L_c^\infty(\R)$の元については反転公式が成り立つから,この消息を$L^1(\R)$の全体に拡張することが理論の目標である.
    Fourier変換の定義には$\frac{1}{2\pi}$を付けないが,反転公式($\wR$の世界から$\R$に戻す時)には出現することとなる.
\end{tcolorbox}

\begin{observation}[$L_c^\infty(\R)$の元については反転公式が成り立つ]
    $L^1(\R)$の元は$C^\infty_c(\R)$による近似が可能である.$C^\infty_c(\R)$の関数は$\bT$上の関数とみなせる.
    実際,$f\in C_c^\infty(\R)$に対して,$\supp f\subset[-L\pi,L\pi]$のとき,$f_L(t):=f(Lt)$とすれば,そのFourier係数は
    \[\wh{f_L}(n)=\frac{1}{2\pi}\int^\pi_{-\pi}f_L(t)e^{-int}dt=\frac{1}{2\pi L}\int^{L\pi}_{-L\pi}f(y)e^{-in\frac{y}{L}}dy=\frac{1}{2\pi L}\int^\infty_{-\infty}f(y)e^{-i\frac{n}{L}y}dy.\]
    よって,$f\in C_c^\infty(\R)$は次のようにFourier展開出来る:
    \[f(x)=f_L(x/L)=\frac{1}{2\pi}\sum_{n=-\infty}^\infty\paren{\underbrace{\int^\infty_{-\infty}f(y)e^{-i\frac{n}{L}y}dy}_{=\wh{f}_L(n/L)}}e^{i\frac{n}{L}x}\frac{1}{L}.\]
    右辺は,$\R$上の関数$\xi\mapsto\wh{f}(\xi)e^{i\xi x}$の等分割$\xi=n/L$に関するRiemann和の形をしている.
    $L>0$は任意に取ったから$L\to\infty$とすれば,$f\in C_c^\infty(\R)$の場合はたしかに積分に収束しており,より一般の$f$についても,積分表示
    \[f(x)=\frac{1}{2\pi}\int^\infty_{-\infty}\wh{f}(\xi)e^{i\xi x}d\xi=\frac{1}{2\pi}\int^\infty_{-\infty}\paren{\int^\infty_{-\infty}f(y)e^{-i\xi y}dy}e^{i\xi x}d\xi.\]
    が期待できる.これを情報の復元という意味で\textbf{反転公式}という.
    実はこの等式は,右辺がLebesgue積分の意味で定義出来さえすれば,a.e.で成り立つ!(反転公式\ref{cor-inversion-theorem-in-R1}).
    よって,Fourier変換$\wh{f}\in C_0(\R)$の可積分性の研究に勤しむのである.
\end{observation}

\begin{definition}[Fourier transform]
    $f\in L^1(\R)$について,次の関数$\wh{f}\in UC_0(\R)$を\textbf{Fourier変換}という:
    \[\wh{f}(\xi):=\int_\R f(x)e^{-i\xi x}dx\quad\xi\in\wR.\]
    次の関数$\wc{f}\in UC_0(\R)$を\textbf{反転Fourier変換}という:
    \[\wc{f}(\xi):=\wh{\wt{f}}(\xi)=\int_\R f(x)e^{i\xi x}dx\quad\xi\in\wR.\]
\end{definition}

\begin{remarks}
    代数的性質としては,$f,\wh{f}\in L^1(\R)$である限り,
    \begin{enumerate}
        \item $\wc{\wh{f}}=2\pi f$(反転公式\ref{cor-inversion-theorem-in-R1}).
        \item $\wh{\wh{f}}=2\pi\wt{f}$(反転公式の系\ref{thm-inversion-theorem-in-Rd}).
    \end{enumerate}
\end{remarks}

\begin{remarks}[指標群について]\tiny
    まず,Vectの双対空間に当たるGrp上での概念,$\C$の乗法群$\C^*=\bT$による群の表現の群を\textbf{指標群}$\wh{G}$という:
    \[\wh{G}:=\Hom_\Grp(G,\C^*)\]
    全く同様な概念を$C^*$-代数についても考えると,$\wh{\A}\subset\A^*$が成り立つ(群準同型ならば,台となるBanach空間の構造についても連続線型).
    すると何故か$\wh{L^1(G)}=\wh{G}$が成り立つ.
    \begin{quote}
        \begin{enumerate}
            \item 任意の可換で単位的なBanach代数$\A$は,その指標のなすコンパクトハウスドルフ空間上の連続関数の代数$C(\wh{\A})$に埋め込める.
            \item $\A$がさらに$C^*$-代数ならば,$*$-等長同型である.
            非単位的な$C^*$-代数$\A$について,指標のなす空間は局所コンパクトであり,$C_0(\wh{\A})$に$*$-等長同型である.
        \end{enumerate}
    \end{quote}
    $G$を$\R^n$や$\bT$などの局所コンパクトAbel郡とし,畳み込みによるBanach代数$L^1(G)$を考える.
    $\A:=L^1(\R)+\C\delta$とすると単位化される.
    \begin{enumerate}
        \item Gelfand変換$\Gamma:\A\to C(\wh{\A})$は$\wh{\A}=\R\cup\{\infty\}$となり,$L^1(\R)$への制限はFourier変換に他ならない.
        \item なお,$\A:=L^1(\R_+)$を考えるとこれはGelfand変換はLaplace変換である.
        \item $f^*(x):=\o{f}(-x)$で対合を定めると,等長であるが$C^*$-性は持たない.
    \end{enumerate}
    一般の局所コンパクトAbel群$G$に対して,双対群$\wh{G}$も局所コンパクトAbel群となる.
    このような視点から,一般の局所コンパクトAbel群$G$上の群環$L^1(G)$にFourier変換$L^1(G)\to C_0(\wh{G})$を考えることが出来る.
    $G$が離散群ならば,離散Fourier変換を得る.
\end{remarks}

\subsection{変換としての性質}

\begin{tcolorbox}[colframe=ForestGreen, colback=ForestGreen!10!white,breakable,colbacktitle=ForestGreen!40!white,coltitle=black,fonttitle=\bfseries\sffamily,
title=]
    Fourier変換はノルム減少的な写像$\F:L^1(\R)\to UC_0(\wR)$を定める.
    また,畳み込みと各点積とに対して,群準同型でもある.
\end{tcolorbox}

\begin{proposition}[Fourier変換の値域]\label{prop-basic-property-of-Fourier-transform}
    $\F:L^1(\R)\to C_0(\wR)$について,
    \begin{enumerate}
        \item 次の意味でノルム減少的である:$\norm{\wh{f}}_\infty\le\norm{f}_1$.
        \item $\wh{f}$は$\wh{\R}$上一様連続である:$\Im\F\subset UC(\wh{\R})$.
    \end{enumerate}
\end{proposition}
\begin{Proof}\mbox{}
    \begin{enumerate}
        \item 任意の$\xi\in\wR$について,
        \begin{align*}
            \abs{\wh{f}(\xi)}&=\Abs{\int_\R f(x)e^{-i\xi x}dx}\le\int_\R\abs{f(x)}dx=\norm{f}_1.
        \end{align*}
        \item 任意の$\xi,\eta\in\wR$について,
        \[\abs{\wh{f}(\xi+\eta)-\wh{f}(\xi)}=\Abs{\int_\R f(x)e^{-i\xi x}(e^{-i\eta x}-1)dx}\le\int_\R\abs{f(x)}\abs{e^{-i\eta x}-1}dx.\]
        まず,右辺は$\forall_{x\in\R}\;\abs{e^{-i\eta x}-1}\le2$より可積分である.
        従ってLebesgueの優収束定理より,$\eta\to0$のとき,$\xi\in\wR$に依らずに$0$に収束する.
    \end{enumerate}
\end{Proof}

\subsection{畳み込みに関する関手性}

\begin{proposition}[畳み込みに関する関手性]
    $f,g\in L^1(\R),\xi\in\wR$について,$\wh{f*g}(\xi)=\wh{f}(\xi)\wh{g}(\xi)$.
\end{proposition}
\begin{Proof}
    Fubiniの定理を用いた次の変形から明らか:
    \begin{align*}
        \wh{f*g}(\xi)&=\F\Square{\int_\R f(x-y)g(y)dy}(\xi)\\
        &=\int_\R e^{-i\xi x}\int_\R f(x-y)g(y)dydx\\
        &=\iint_{\R^2}\paren{e^{-i\xi(x-y)}f(x-y)dx}e^{-i\xi y}g(y)dy.
    \end{align*}
\end{Proof}

\begin{proposition}
    $f,h\in L^1(\R)$であって,ある$H\in L^1(\wR)$について,
    \[h(x)=\frac{1}{2\pi}\int_\R H(\xi)e^{i\xi x}d\xi\]
    と表せているとする.このとき,
    \[(h*f)(x)=\frac{1}{2\pi}\int_\R H(\xi)\wh{f}(\xi)e^{i\xi x}d\xi.\]
\end{proposition}
\begin{Proof}
    積$H(\xi)f(y)$は$\R^2$上で可積分であるから,Fubiniの定理により次のように計算できる:
    \begin{align*}
        (h*f)(x)&=\int_\R h(x-y)f(y)dy=\frac{1}{2\pi}\iint_{\R^2}H(\xi)e^{i\xi x}e^{-i\xi y}f(y)d\xi dy\\
        &=\frac{1}{2\pi}\int_\R H(\xi)e^{i\xi x}\int e^{-i\xi y}f(y)dyd\xi=\frac{1}{2\pi}\int_\R H(\xi)\wh{f}(\xi)e^{i\xi x}d\xi.
    \end{align*}
\end{Proof}

\begin{theorem}[Youngの不等式]
    $f\in L^p(\R),g\in L^1(\R)\;(1\le p<\infty)$について,
    \begin{enumerate}
        \item 殆ど至る所の$x\in\R$に対して,$y\mapsto f(x-y)g(y)$は$\R$上可積分である.
        \item $\norm{f*g}_p\le\norm{f}_p\norm{g}_1$.
        \item $f*g=g*f$かつ$f*(g*h)=(f*g)*h$.
    \end{enumerate}
\end{theorem}

\begin{corollary}
    $\{k_\lambda\}_{\lambda>0}\subset L^1(\R)_+$を正の総和核とする.
    総和核との畳み込みは$L^p(\R^n)$-ノルム減少的である:$\forall_{p\in[1,\infty]}\;\norm{k_\lambda*f}_{L^p(\R^n)}\le\norm{f}_{L^p(\R^n)}$.
\end{corollary}
\begin{Proof}
    非負な総和核の場合,$\norm{k_\lambda}=\norm{k}=1$より,Youngの不等式から.
\end{Proof}

\subsection{自己相関関数}

\begin{tcolorbox}[colframe=ForestGreen, colback=ForestGreen!10!white,breakable,colbacktitle=ForestGreen!40!white,coltitle=black,fonttitle=\bfseries\sffamily,
title=]
    定常的な信号について,自己相関解析という分野がある.
    この一般化されたFourier変換の理論は,Einstein (1914)がアイデアを説明しており,Wiener (1930)が決定的論関数について示し,Khinchin (1934)が定常過程について示した.
\end{tcolorbox}

\begin{notation}
    関数$f,g\in L^1(\R)$に対して,
    \begin{enumerate}
        \item ${}^-\!f(x):=f(-x)$とする.
        \item $f,g$の\textbf{相関関数}を次で定める:
        \[[f:g](x):={}^-\!({}^-\!f*g)(x)=\int_\R{}^-\!f((-x)-y)g(y)dy=\int_\R f(y+x)g(y)dy.\]
        \item 特に$[f:f]$を\textbf{自己相関関数}といい,$f$を強定常な実過程の確率密度関数とみたときの自己相関関数に一致する.変数$x$は,注目している2時点のズレを表す変数である.
    \end{enumerate}
\end{notation}

\begin{proposition}
    自己相関関数$[f:f]$について,
    \begin{enumerate}
        \item 対称である:$[f:f](x)=[f:f](-x)$.
        \item 任意の$t\in\R$について,$\abs{[f:f](t)}\le [f:f](0)$.
    \end{enumerate}
\end{proposition}
\begin{Proof}\mbox{}
    \begin{enumerate}
        \item 変数変換$z:=y+x$について,
        \[[f:f](x)=\int_\R f(y+x)f(y)dy=\int_\R f(z)f(z-x)dz=[f:f](-x).\]
        \item 
    \end{enumerate}
\end{Proof}

\begin{theorem}[Wiener-Khinchin]
    $f\in L^1(\R)$に自己相関関数$[f:f]$が$\R$上で存在するとする.このとき,ある単調増加関数$F:\R\to\R$が存在して,
    \[[f:f](t)=\int_\R e^{2\pi itx}dF(x).\]
    このときの$F$を\textbf{パワースペクトル密度(関数)}という.
\end{theorem}
\begin{remarks}
    この定理の主張は,$f$にFourier変換が存在するかわからないために歯切れが悪くなっている.

\end{remarks}

\section{総和核の定義と例}

\begin{tcolorbox}[colframe=ForestGreen, colback=ForestGreen!10!white,breakable,colbacktitle=ForestGreen!40!white,coltitle=black,fonttitle=\bfseries\sffamily,
title=]
    $\R$上のDirichlet核もFejer核もある関数
    \[D_1(x)=\frac{1}{\pi}\frac{\sin x}{x},\quad K_1(x)=\frac{1}{2\pi}\paren{\frac{\sin x/2}{x/2}}^2.\]
    の相似変換の列とみれる.
    同様にして,$\R$上の確率密度関数も総和核を定める.
\end{tcolorbox}

\subsection{総和核の定義と構成}

\begin{tcolorbox}[colframe=ForestGreen, colback=ForestGreen!10!white,breakable,colbacktitle=ForestGreen!40!white,coltitle=black,fonttitle=\bfseries\sffamily,
title=]
    $\R$上の積分が$1$関数$k\in L^1(\R)$に対して,その$y$軸に向けた圧縮$k_\lambda(x):=\lambda k(\lambda x)$を考えれば,これは総和核になる.
\end{tcolorbox}

\begin{definition}[summability kernel]
    次を満たす可積分関数の族
    $\{k_\lambda\}_{\lambda>0}\subset L^1(\R)$を\textbf{$\lambda\to\infty$についての総和核}という:
    \begin{enumerate}
        \item 規格化:$\forall_{\lambda>0}\;\int_\R k_\lambda(x)dx=1$.
        \item $L^1$-有界性:$\exists_{C>0}\;\forall_{\lambda>0}\;\norm{k_\lambda}\le C$.
        \item 集中性:$\forall_{\delta>0}\;\int_{\abs{x}\ge\delta}\abs{k_\lambda(x)}dx\xrightarrow{\lambda\to\infty}0$.
    \end{enumerate}
\end{definition}

\begin{proposition}[総和核の構成法]\label{prop-construction-of-summability-kernel}
    $\R$上の関数
    $k\in L^1(\R),\int_\R k(x)dx=1$について,$k_\lambda(x):=\lambda k(\lambda x)$とすれば,これは総和核である.
\end{proposition}
\begin{Proof}\mbox{}
    \begin{enumerate}
        \item 代入と変数変換より,
        \begin{align*}
            \int_\R k_\lambda(x)dx=\lambda\int_\R k(\lambda x)dx=\lambda\int_\R k(y)\frac{dy}{\lambda}=1.
        \end{align*}
        \item 絶対値がついたのみで,全く同様に$y=\lambda x$を考えることで,
        \begin{align*}
            \norm{k_\lambda}_{L^1(\R)}=\int_\R \abs{k_\lambda(x)}dx=\int_\R\abs{k(y)}dy=\norm{k}_{L^1(\R)}.
        \end{align*}
        \item 同様の変数変換$y=x\lambda$により積分区間に変数$y$を移すことができ,
        \[\int_{\abs{x}\ge\delta}\abs{k_\lambda(x)}dx=\int_{\abs{x}\ge\delta}\lambda\abs{k(\lambda x)}dx=\int_{\abs{y}\ge\lambda\delta}\abs{k(y)}dy\xrightarrow{\lambda\to\infty}0.\]
        と議論出来る.
    \end{enumerate}
\end{Proof}

\subsection{Dirichlet核とFejer核}

\begin{tcolorbox}[colframe=ForestGreen, colback=ForestGreen!10!white,breakable,colbacktitle=ForestGreen!40!white,coltitle=black,fonttitle=\bfseries\sffamily,
title=]
    $\bT,\R$上のDirichlet核とFejer核は,双対$\Z,\wh{\R}$上で考えると明らかに同一物である:
    \[\wh{D}_\lambda(\xi)=1_{[-\lambda,\lambda]}(\xi),\quad\wh{K}_\lambda(\xi)=\paren{1-\frac{\abs{\xi}}{\lambda}}1_{[-\lambda,\lambda]}(\xi)\]
    この$\wh{\R}$上の関数または$\Z$への制限をFourier係数/変換とするような関数がDirichlet核とFejer核である.
    すると,分母$N+1,\lambda$の違いに混乱せずに済む.
\end{tcolorbox}

\begin{definition}[Dirichlet kernel, Fejer kernel on $\R$]\label{def-Fejer-kernel-on-R}
    任意の$\lambda>0$について,
    \begin{enumerate}
        \item Dirichlet核を
        \[D_\lambda(x):=\frac{1}{2\pi}\int^\lambda_{-\lambda}e^{i\xi x}d\xi\]
        と定める.
        \item Fejer核を
        \[K_\lambda(x):=\frac{1}{2\pi}\int^\lambda_{-\lambda}\paren{1-\frac{\abs{\xi}}{\lambda}}e^{i\xi x}d\xi\]
        と定める.
    \end{enumerate}
\end{definition}

\begin{proposition}[Dirichlet核の性質]\label{prop-property-of-Dirichlet-kernel-on-R1}
    Dirichlet核$\{D_\lambda\}$について,
    \begin{enumerate}
        \item $\R\setminus\{0\}$上での表現:$\forall_{x\ne0}\;D_\lambda(x)=\frac{\sin\lambda x}{\pi x}$.
        \item 自己相似構造:$D_\lambda(x)=\lambda D_1(\lambda x)$.
        \item 非可積分性:$D_\lambda\notin L^1(\R)$.特に,Dirichlet核は総和核でない.
        \item (3)の一般化:$\wh{1_{[-\lambda,\lambda]}}(\xi)\notin L^1(\wR)$.
    \end{enumerate}
\end{proposition}
\begin{Proof}\mbox{}
    \begin{enumerate}
        \item $D_\lambda$を素直に積分してみれば,
        \begin{align*}
            D_\lambda(x)&=\frac{1}{2\pi}\int^\lambda_{-\lambda}e^{i\xi x}d\xi=\frac{1}{2\pi}\Square{\frac{1}{ix}e^{i\xi x}}^\lambda_{-\lambda}\\
            &=\frac{e^{i\lambda x}-e^{-i\lambda x}}{2\pi ix}=\frac{\sin \lambda x}{\pi x}.
        \end{align*}
        \item 積分の変数変換による性質である.
        \item (1)の表示を見れば判る通り,Lebesgue可積分ではない.
        \item 次の計算の通り,やはりLebesgue可積分ではない:
        \begin{align*}
            \wh{1_{[-\lambda,\lambda]}}(\xi)&=\int^\lambda_{-\lambda}e^{-i\xi x}dx=\frac{2\sin(\lambda\xi)}{\xi}.
        \end{align*}
    \end{enumerate}
\end{Proof}

\begin{proposition}[Fejer核の性質]\label{prop-property-of-Fejer-kernel-on-R1}
    Fejer核$\{K_\lambda\}$について,
    \begin{enumerate}
        \item $\R\setminus\{0\}$上での表現:任意の$x\ne0$について,\[K_\lambda(x)=\frac{1-\cos\lambda x}{\pi\lambda x^2}=\frac{2\sin^2(\lambda x/2)}{\pi\lambda x^2}=\frac{\lambda}{2\pi}\paren{\frac{\sin(\lambda x/2)}{\lambda x/2}}^2\ge0.\]
        \item 自己相似構造:$K_\lambda(x)=\lambda K_1(\lambda x)$.
        \item 総和核である:$\int_\R K_1(x)dx=1$.
        特に,$K_\lambda\in L^1(\R)$.
    \end{enumerate}
    特に,Fejer核は総和核である.
\end{proposition}
\begin{Proof}\mbox{}
    \begin{enumerate}
        \item 次のように計算出来る:
        \begin{align*}
            K_\lambda(x)&=\frac{1}{2\pi}\int^\lambda_0\paren{1-\frac{\xi}{\lambda}}e^{i\xi x}d\xi+\frac{1}{2\pi}\int^0_{-\lambda}\paren{1+\frac{\xi}{\lambda}}e^{i\xi x}d\xi\\
            &=\frac{1}{2\pi}\int^\lambda_0\paren{1-\frac{\xi}{\lambda}}(e^{i\xi x}+e^{-i\xi x})d\xi=\frac{1}{\pi}\int^\lambda_0\paren{1-\frac{\xi}{\lambda}}\cos(\xi x)d\xi\\
            &=\frac{1}{\pi}\Square{\paren{1-\frac{\xi}{x}}\frac{\sin(\xi x)}{x}}^\lambda_0+\frac{1}{\pi\lambda}\int^\lambda_0\frac{\sin(\xi x)}{x}d\xi\\
            &=\frac{1}{\pi\lambda x^2}\Square{-\cos(\xi x)}^\lambda_0=\frac{1-\cos(\lambda x)}{\pi\lambda x^2}.
        \end{align*}
        \item 積分の変数変換による.
        \item $\lambda=1$として示せば良い.
        また$0$に特異性があるので,$(-\infty,0),(0,\infty)$に分割して計算すると,
        \begin{align*}
            \int^\infty_0\paren{\frac{\sin(x/2)}{x/2}}^2dx&=2\int^\infty_0\paren{\frac{\sin y}{y}}^2dy\\
            &=2\paren{\Square{-\frac{\sin^2y}{y}}^\infty_0+\int^\infty_0\frac{2\sin y\cos y}{y}dy}\\
            &=2\int^\infty_0\frac{\sin 2y}{y}dy=2\int^\infty_0\frac{\sin u}{u}du=\pi.
        \end{align*}
        ただし,$\int^\infty_0\frac{\sin u}{u}=\frac{\pi}{2}$はRiemann広義積分で,正則関数$e^z/z$の留数計算の虚部の比較,またはDirichlet核の積分\ref{lemma-boundedness-of-integral-of-Dirichlet-kernel}などから判る.
        \item (2),(3)より,命題\ref{prop-construction-of-summability-kernel}の構成法より,たしかにFejer核は総和核になる.
    \end{enumerate}
\end{Proof}

\subsection{Poisson核の定義}

\begin{tcolorbox}[colframe=ForestGreen, colback=ForestGreen!10!white,breakable,colbacktitle=ForestGreen!40!white,coltitle=black,fonttitle=\bfseries\sffamily,
    title=]
    確率密度関数は総和核を定める\ref{prop-construction-of-summability-kernel}.
    \begin{enumerate}
        \item $\r{C}(0,1)$の確率密度関数をPoisson核,$\rN(0,1)$の確率密度関数をGauss核といい,
        いずれも再生性を持つ分布族である.
        \item $(\r{C}(0,\sigma))_{\sigma>0}$の確率密度関数の族がPoisson核が定める総和核,
        $(\rN(0,\sigma))_{\sigma>0}$の確率密度関数の族がGauss核が定める総和核になる.
        \item 前者は減衰が遅いことが知られているが,実際特性関数$\varphi(u)=e^{-\abs{u}}$は原点で特異である.
        一方Gauss核は急減少関数である.
        なお,確率論の意味での特性関数とFourier変換が一致するのは,いずれも対称な分布であるためである\ref{thm-inversion-theorem-in-Rd}(3).
    \end{enumerate}
\end{tcolorbox}

\begin{definition}[Poisson kernel]\label{def-Poisson-kernel}
    $C(0,1)$の確率密度関数
    \[P(x):=\frac{1}{\pi}\frac{1}{1+x^2}\]
    を\textbf{Poisson核}という.これは,特性関数$\varphi(u)=e^{-\abs{u}}$のFourier変換の$\frac{1}{2\pi}$倍になっている.
\end{definition}
\begin{remarks}[$\bT$上のPoisson核とFourier変換を通じて緩く対応している]
    $\bT$上のPoisson核とは,
    \[P_r(t)=\sum_{n\in\Z}r^{\abs{n}}e^{int},\quad\wh{P_r}(n)=r^{\abs{n}}\]
    を満たすものであった.これを,$\wh{P}(n):=\wh{P_{e^{-1}}}(n)$と受け継いで,補間したものである.
\end{remarks}

\begin{theorem}[$\R$上のPoisson核のFourier変換]
    関数$\varphi(x):=e^{-\abs{x}}$について,
    \begin{enumerate}
        \item $e^{-\abs{x}}$のFourier変換:$\wh{\varphi}(\xi)=\frac{2}{1+\xi^2}=2\pi P(\xi)$.
        \item Cauchy分布の特性関数:任意の$x\in\R$について,
        \[\varphi(x)=\frac{1}{\pi}\int_{\wR}\frac{e^{i\xi x}}{1+\xi^2}d\xi=\int_{\wR} P(\xi)e^{i\xi x}d\xi=\wc{P}(\xi).\]
        \item 特に,$\wh{P}(\xi)=e^{-\abs{\xi}}$.
    \end{enumerate}
    以上より,$P$のFourier変換も,逆Fourier変換も$e^{-\abs{x}}$である.
\end{theorem}
\begin{Proof}\mbox{}
    \begin{enumerate}
        \item 次のように計算できる:
        \begin{align*}
            \wh{\varphi}(\xi)&=\int_\R e^{-\abs{x}}e^{-i\xi x}dx=\int^\infty_0e^{-x-i\xi x}dx+\int^0_{-\infty}e^{x-i\xi x}dx\\
            &=\Square{-\frac{e^{-x-i\xi x}}{1+i\xi}}^\infty_0+\Square{\frac{e^{x-i\xi x}}{1-i\xi}}^0_{-\infty}=\frac{1}{1+i\xi}+\frac{1}{1-i\xi}=\frac{2}{1+\xi^2}.
        \end{align*}
        \item $\wh{\varphi}\in L^1(\wR)$がわかったので,$\varphi\in C(\R)$と反転公式\ref{cor-inversion-theorem-in-R1}による.
        \item $\wh{P}(\xi)=\frac{1}{2\pi}(\F^2\varphi)(\xi)=\varphi(-\xi)=e^{-\abs{\xi}}$.
    \end{enumerate}
\end{Proof}

\subsection{$\R^d$上のGauss核}

\begin{tcolorbox}[colframe=ForestGreen, colback=ForestGreen!10!white,breakable,colbacktitle=ForestGreen!40!white,coltitle=black,fonttitle=\bfseries\sffamily,
title=]
    $N(0,1)$の確率密度関数をGauss核といい,$(N(0,\sigma^2))_{\sigma>0}$の確率密度関数の族がGauss核が定める総和核となる.
    Gauss核の周期化は熱核の定数倍になる.
\end{tcolorbox}

\begin{definition}[Gauss kernel]
    $N_d(0,I_d)$の確率密度関数
    \[G(x):=\frac{1}{(2\pi)^{d/2}}e^{-\frac{\abs{x}^2}{2}}\]
    を$\R^d$上のGauss核という.
\end{definition}
\begin{remarks}\label{remarks-Gauss-kernel-on-T}
    $\bT$の熱核を
    \[W_t(x):=\sum_{n\in\Z}e^{-n^2t}e^{int}\quad(t>0),\qquad \wh{W}_t(n)=e^{-n^2t}\quad(n\in\Z).\]
    で定める.
\end{remarks}

\begin{theorem}[Gauss核のFourier変換]\mbox{}\label{thm-Fourier-transform-of-Gaussian-kernel-on-R}
    \begin{enumerate}
        \item Schwartz急減少関数である:$G\in\S$.
        \item Fourier変換は次で表せる:$\wh{G}(\xi)=e^{-\frac{\abs{\xi}^2}{2}}$.
    \end{enumerate}
\end{theorem}
\begin{Proof}
    \[\wh{G}(\xi)=\frac{1}{\sqrt{2\pi}}\int_\R e^{-\frac{x^2}{2}}e^{-i\xi\cdot x}dx\]
    の計算であるが,次のようにSteinの微分方程式による現象である.
    \begin{enumerate}
        \item $\wh{G}$を微分すると,
        \begin{align*}
            \dd{}{\xi}\wh{G}(\xi)&=\frac{1}{\sqrt{2\pi}}\int_\R e^{-\frac{x^2}{2}}(-ix)e^{-i\xi\cdot  x}dx\\
            &=\frac{1}{\sqrt{2\pi}}\paren{\Square{ie^{-\frac{x^2}{2}}e^{-\xi\cdot  x}}^\infty_{-\infty}-i\int_\R e^{-\frac{x^2}{2}}(-i\xi)e^{-i\x\cdot i x}dx}\\
            &=-\frac{\xi}{\sqrt{2\pi}}\int_\R e^{-\frac{x^2}{2}}e^{-\xi\cdot  x}dx=-\xi\wh{G}(\xi).
        \end{align*}
        よって,任意階の微分も$G$が多項式倍されるだけで,やはり急減少である.
        \item よって,$\wh{G}(\xi)$は$e^{-\frac{\xi^2}{2}}$の定数倍であるが,$\wh{G}(0)=\int_\R G(x)dx=1$の初期条件を考えると,$\wh{G}(\xi)=e^{-\frac{\xi^2}{2}}$.
    \end{enumerate}
\end{Proof}

\begin{remarks}[Gauss核が定める総和核]
    $G_\lambda(x):=\lambda^nG(\lambda x)$の代わりに,変数変換$\lambda=\frac{1}{\sqrt{2\ep}}$を考えて,$(\rN(0,2\ep))$の密度関数の族と考える:
    \[G_\ep(x):=\frac{1}{(2\ep)^{n/2}}G\paren{\frac{x}{\sqrt{2\ep}}}=\frac{1}{(4\pi\ep)^{n/2}}e^{-\frac{\abs{x}^2}{4\ep}}.\]
    これを熱方程式の基本解または\textbf{熱核}という.
\end{remarks}

\subsection{$\R^d$上のFriedrichs軟化子}

\begin{tcolorbox}[colframe=ForestGreen, colback=ForestGreen!10!white,breakable,colbacktitle=ForestGreen!40!white,coltitle=black,fonttitle=\bfseries\sffamily,
title=]
    Poisson核やGauss核を反転させたものを適切に切り取ることで,隆起関数を構成出来る.
    超関数論の基本言語である.
    軟化性能については,そもそも任意のコンパクト台を持つ超関数$T\in\D'_c$について,$f\in C^\infty(\R^d)$ならば$T*f\in C^\infty(\R^d)$である.
\end{tcolorbox}

\begin{example}[総和核としてのFriedrichs軟化子]
    $\R^n$上の関数
    \[\eta(x):=Ce^{-\frac{1}{1-\abs{x}^2}}1_{\Brace{\abs{x}<1}}\quad C:=\paren{\int_{\R^n} e^{-\frac{1}{1-\abs{x}^2}}dx}^{-1},x\in\R^n\]
    は,次の性質を満たす:
    \begin{enumerate}
        \item $\eta\in C_c^\infty(\R^n)_+$.これは,$\rho(r)=e^{-\frac{1}{1-r^2}}1_{\abs{x}<r}$という関数が,$r=1$において内側から任意階数の微分係数が存在し$0$であることから判る.
        \item $\supp \eta\subset B(0,1)$.
        \item $\int_{\R^n}\eta(x)dx=1$.
    \end{enumerate}
    したがって,$\eta_\lambda(x):=\lambda^n\eta(\lambda x)$とするとこれは総和核である.
    だが軟化子の文脈では$\eta_\ep(x):=\frac{1}{\ep^n}\eta\paren{\frac{x}{\ep}}$として$\eta\to0$の極限を考える.
\end{example}
\begin{remarks}[Gauss核の変換としての隆起関数]\label{remarks-transformed-Gauss-kernel}
    Gauss関数$e^{-x^2}$に対して,変換
    \[x^2\mapsto\frac{1}{1-x^2}\]
    を考えると,$\pm1$が無限大へ飛び,無限大が$\pm1$に来る.
    したがって,$C^\infty$-級のまま$x=\pm1$で一度零を取る.
    そこで,$1_{\norm{x}<1}$だけ残して他を切り取ると,$C_c^\infty(\R^d)$の元を得る.
\end{remarks}

\begin{lemma}[軟化子の性質]
    $\eta_\ep$との畳み込み${}^\ep:L^1_\loc(U)\to C^\infty(U_\ep)\;(U\osub\R^n)$は
    \begin{enumerate}
        \item $L^1_\loc(\R^n)$上で定義でき,値域は$C^\infty(U_\ep)$である.ただし,
        \[U_\ep:=\Brace{U+x\in\R^n\mid\norm{x}\le\ep}.\]
        \item $\supp u\subset K$ならば,$\supp J_\lambda u\subset \Brace{K+x\in\R^n\mid\norm{x}\le\ep}$.
        \item $f\in C(U)$でもあるならば,収束$f^\ep\to f$は広義一様である.
        \item $1\le p<\infty$について,$f^\ep\to f$は$L^p_\loc(U)$の意味でも収束する.
    \end{enumerate}
\end{lemma}
\begin{Proof}\mbox{}
    \begin{enumerate}
        \item $j_\lambda$の台がコンパクトであるために$L^1_\loc(\R^n)$上で定義出来る.
        またこれに付随して,微分と積分の交換が可能であるから,やはり任意階微分可能になる.
        \item 被積分関数$\eta_\ep(x-y)u(y)$が$y\in\Brace{K+x\in\R^n\mid\norm{x}\le\ep}$上にしか台を持たないためである.
        \item \cite{Brezis-FunctionalAnalysis} Prop.4.21.$f$の連続性から,
        任意にコンパクト集合$K\compsub U$を取る.任意の$\ep>0$に対して,ある$\delta>0$が存在して,
        \[\abs{f(x-y)-f(x)}<\ep,\qquad x\in K,y\in B(0,\delta).\]
        が成り立っているから,任意の$x\in U$に対して,
        \begin{align*}
            (\eta_\ep*f)(x)-f(x)&=\int_{\R^n}\Paren{f(x-y)-f(x)}\eta_\ep(y)dy\\
            &=\int_{B(0,r)}\Paren{f(x-y)-f(x)}\eta_\ep(y)dy.
        \end{align*}
        が,任意の$r<\delta$について成り立ち,特に任意の$x\in K$の場合は
        \[\abs{(\eta_\ep*f)(x)-f(x)}\le\ep\int\eta_\ep=\ep.\]
        と評価できる.
        \item $\{\eta_\ep\}\subset C_c^\infty(\R^n)$が総和核であるためである.
    \end{enumerate}
\end{Proof}

\begin{observation}[1の分割を与える隆起関数]\mbox{}
    \begin{enumerate}
        \item Poisson核の反転$e^{-\frac{1}{x}}1_{\R^+}$は$\R$上$C^\infty$-級である.
        \item 次の関数は,$x\le 0$では相変わらず$0$で,$1\le x$では分母と分子が一致することより常に$1$であるような$C^\infty$-級関数である:
        \[\frac{e^{-\frac{1}{x}}}{e^{-\frac{1}{x}}+e^{-\frac{1}{1-x}}}.\]
        \item これを切り貼りすることで,2つの開集合$U\subset V$について,$V$上で1で,$U$内に台を持つものが作れる.
    \end{enumerate}
\end{observation}

\subsection{De la Vallee Poussinの核}

\begin{definition}
    \[V_\lambda(x):=2K_{2\lambda}(x)-K_\lambda(x),\qquad x\in\R.\]
\end{definition}

\begin{proposition}
    \[\wh{V}_\lambda(\xi)=\begin{cases}
        1&\abs{\xi}\le1,\\
        2-\frac{\abs{\xi}}{\lambda}&\lambda\le\abs{\xi}\le 2\lambda,\\
        0&2\lambda\le\abs{\xi}.
    \end{cases}\]
\end{proposition}

\section{Fourier部分和の総和的性質}

\subsection{総和核の性質}

\begin{theorem}\label{thm-property-of-summerbility-kernel-in-R}
    $\{k_\lambda\}_{\lambda>0}\subset L^1(\R)$を総和核とする.
    \begin{enumerate}
        \item 有界な一様連続関数$f\in UC_b(\R)$について,$k_\lambda*f\xrightarrow{\lambda\to\infty}f$は一様収束する.
        \item 可積分関数$f\in L^p(\R)\;(1\le p<\infty)$について,$k_\lambda*f\xrightarrow{\lambda\to\infty}f$は$L^p$-収束する.
    \end{enumerate}
\end{theorem}
\begin{Proof}\mbox{}
    \begin{enumerate}
        \item 任意の$\ep>0$を取る.
        \begin{enumerate}
            \item $f$の一様連続性の仮定より,$\exists_{\delta>0}\;\forall_{x\in\R}\;\forall_{y\in(-\delta,\delta)}\;\abs{f(x-y)-f(x)}<\ep$.
            \item 総和核の性質より,
            \[\abs{(k_\lambda*f)(x)-f(x)}=\Abs{\int_\R f(x-y)k_\lambda(y)dy-f(x)\int_\R k_\lambda(y)dy}\le\int_\R\abs{f(x-y)-f(x)}\abs{k_\lambda(y)}dy.\]
        \end{enumerate}
        この2点より,
        \begin{align*}
            \abs{(k_\lambda*f)(x)-f(x)}&\le\ep\int_{\abs{y}<\delta}\abs{k_\lambda(y)}dy+2\norm{f}_1\int_{\abs{y}\ge\delta}\abs{k_\lambda(y)}dy.
        \end{align*}
        第一項は$\lambda$に依らず$C\ep$で抑えられる.第二項は$\lambda\to\infty$の極限で$0$に収束する,特に$\ep$よりも小さく出来る.
        そしてこの評価は$x\in\R$に依っていない.
        \item \begin{description}
            \item[$p>1$の場合] 任意に$\ep>0$を取る.
            \begin{enumerate}
                \item $f\in L^p(\R)$を単関数近似,または$C_c(\R)$の元で近似すれば判ることに,$\exists_{\delta>0}\;\forall_{\abs{y}\le\delta}\;\int_\R\abs{f(x-y)-f(x)}^pdx<\ep$.
                \item Holderの不等式より,
                \begin{align*}
                    \norm{k_\lambda*f-f}^p_p&\le\int_\R\paren{\int_\R\abs{f(x-y)-f(x)}\underbrace{\abs{k_\lambda(y)}}_{=\abs{k_\lambda(y)}^{1/p}\abs{k_\lambda(y)}^{1/q}}dy}^pdx\\
                    &\le\int_\R\paren{\int_\R\abs{f(x-y)-f(x)}^p\abs{k_\lambda(y)}dy}\underbrace{\paren{\int_\R\abs{k_\lambda(y)}dy}^{p/q}}_{\le C^{p/q}}dx.
                \end{align*}
            \end{enumerate}
            この2点より,$x$で先に積分すると,
            \begin{align*}
                \norm{k_\lambda*f-f}^p_p&\le C^{p/q}\paren{\ep\int_{\abs{y}<\delta}\abs{k_\lambda(y)}dy+(2\norm{f}_p)^{p}\int_{\abs{y}\ge\delta}\abs{k_\lambda(y)}dy}.
            \end{align*}
            十分大きい$\lambda>0$に対して,第二項は十分小さく,特に$(2\norm{f}_p)^p\ep$より小さく出来る.
            これで$p>1$の場合は示せた.
            \item[$p=1$の場合] (b)のステップと飛ばしても(a)によって同様な評価が成功する.
        \end{description}
    \end{enumerate}
\end{Proof}
\begin{remarks}
    (2)の証明に,$\bT$の場合は$C(\bT)$が$L^p(\bT)$で稠密であることを用いたために非常に簡潔に終わった\ref{thm-kernel-is-approximate-unit-in-Lp}が,$\R$上では成り立たないので,より精密な$L^p$-ノルムの評価を必要とした.
\end{remarks}

\begin{corollary}[試験関数の空間の稠密性]
    $1\le p<\infty$のとき,$L^p(\R)$の元は$C_c^\infty(\R)$の元で$L^p$-近似出来る.
\end{corollary}
\begin{Proof}任意の$f\in L^p(\R)$を取る.
    \begin{enumerate}
        \item $\norm{f-f1_{[-M,M]}}_p\xrightarrow{M\to\infty}0$より,初めから$\supp f\subset[-M,M]$と仮定して良い.
        \item $\varphi\in C_c^\infty(\R)$で$\int_\R\varphi(x)dx=1$を満たすものが定める総和核$\{\varphi_\lambda\}\subset L^1(\R)$を考えると,$\varphi_\lambda*f\to f$が$L^p(\R)$-ノルムについて成り立つ.
        $\varphi_\lambda*f$は$C^\infty_c$の元である.
    \end{enumerate}
\end{Proof}

\subsection{Fourier変換の単射性}

\begin{tcolorbox}[colframe=ForestGreen, colback=ForestGreen!10!white,breakable,colbacktitle=ForestGreen!40!white,coltitle=black,fonttitle=\bfseries\sffamily,
title=]
    Fejer核,Gauss核,Poisson核のいずれを用いても,次のようにFourier変換の単射性が導ける.
    が,まだ$L^1(\R^n)$の元としての同一性が得られるのみで,各点の反転公式はより複雑な消息である.
\end{tcolorbox}

\begin{definition}[Cesaro和]
    $\sigma_\lambda:L^1(\R)\to L^1(\R)\;(\lambda>0)$を$\sigma_\lambda(f):=K_\lambda*f$と定める.
    $K_\lambda$は有界であるため,$\sigma_\lambda[f]$は任意の$x\in\R$上で定まる.また,
    Youngの不等式から$\norm{\sigma_\lambda[f]}_1\le\norm{f}_1$で,これは有界線型作用素である.
\end{definition}

\begin{corollary}[Fourier変換の単射性]\mbox{}\label{cor-expression-of-Cesaro-sum}
    \begin{enumerate}
        \item Fejer和との畳み込みのFourier変換による表示:Cesaro和$\sigma_\lambda[f]=K_\lambda*f$は次のように表せる:
        \[\forall_{x\in\R}\quad \sigma_\lambda[f](x)=\frac{1}{2\pi}\int^\lambda_{-\lambda}\paren{1-\frac{\abs{\xi}}{\lambda}}\wh{f}(\xi)e^{i\xi x}d\xi.\]
        \item $L^1$-での反転公式:任意の$f\in L^1(\R)$について,$\sigma_\lambda(f)$は$f$に$\lambda\to\infty$で$L^1$-収束する.
        \item Fourier変換の単射性:Fourier変換$\F:L^1(\R)\to UC_0(\wR)$は単射である.
    \end{enumerate}
\end{corollary}
\begin{Proof}\mbox{}
    \begin{enumerate}
        \item Fejer核の定義から
        \[\sigma_\lambda[f](x)=\int_\R K_\lambda(x-y)f(y)dy=\frac{1}{2\pi}\int_\R\paren{\int^\lambda_{-\lambda}\paren{1-\frac{\abs{\xi}}{\lambda}}e^{i\xi(x-y)}d\xi}f(y)dy.\]
        あとはFubiniの定理による.
        \item $K_\lambda$は総和核であるためである\ref{thm-property-of-summerbility-kernel-in-R}.
        \item $\Ker\F=0$を示せば良い.$f\in L^1(\R)$は$\wh{f}=0$を満たすとすると,
        (1)の表示から,
        $\forall_{\lambda>0}\;\sigma_\lambda[f]=0$である.よって,$\sigma_\lambda[f]\xrightarrow{\lambda\to\infty}0=f$.
    \end{enumerate}
\end{Proof}

\begin{proposition}[\cite{黒田成俊-関数解析}定理5.8,5.9]\mbox{}
    \begin{enumerate}
        \item Gauss核との畳み込みは次のように表せる:
        \[(G_\ep*f)(x)=\frac{1}{(2\pi)^{n/2}}\int_{\R^n}\wh{f}(\xi)e^{-\ep\abs{\xi}^2+i\xi x}d\xi\]
        \item 任意の$f\in L^1(\R^n)\cap L^p(\R^n)\;(p\in\cointerval{1,\infty})$は$L^p$-収束の意味で反転公式を満たす:
        \[f(x)=\lim_{\ep\to0}(G_\ep*f)(x)=\lim_{\ep\to0}\frac{1}{(2\pi)^{n/2}}\int_{\R^n}\wh{f}(\xi)e^{-\ep\abs{\xi}^2+i\xi x}d\xi.\]
        \item $\F:L^1(\R^n)\to UC_0(\wh{\R^n})$は単射である.
    \end{enumerate}
\end{proposition}

\subsection{反転公式}

\begin{tcolorbox}[colframe=ForestGreen, colback=ForestGreen!10!white,breakable,colbacktitle=ForestGreen!40!white,coltitle=black,fonttitle=\bfseries\sffamily,
title=]
    \[f(x)=\lim_{\lambda\to\infty}\frac{1}{2\pi}\int^\lambda_{-\lambda}\paren{1-\frac{\abs{\xi}}{\lambda}}\wh{f}(\xi)e^{i\xi x}d\xi\]
    が$L^1$-収束の意味で成り立つことはわかったが,各点の意味で成り立つための十分条件を考える.
    これは$\wh{f}\in L^1(\wR)$が与える.
    $\bT$の場合の$\wh{g}\in l^1(\Z)$に当たる.
    $g\in C^2(\bT)$はこの十分条件を与えるのであった.対応する$\R$上の条件は$f\in C^2_b(\R)$である!
\end{tcolorbox}

\begin{corollary}[反転公式]\label{cor-inversion-theorem-in-R1}
    $f\in L^1(\R)$が$\wh{f}\in L^1(\wR)$を満たせば,次が成り立つ:
    \[f(x)\overset{\as}{=}\frac{1}{2\pi}\int_{\wR}\wh{f}(\xi)e^{i\xi x}d\xi=\frac{1}{2\pi}\wc{\wh{f}}.\]
    特に,右辺は$\R$上連続であることに注意すれば,$f$には連続な修正が存在する.
\end{corollary}
\begin{Proof}
    Cesaro和の表示\ref{cor-expression-of-Cesaro-sum}(1)
    \[\sigma_\lambda[f](x)=\frac{1}{2\pi}\int_\R1_{[-\lambda,\lambda]}(\xi)\paren{1-\frac{\abs{\xi}}{\lambda}}\wh{f}(\xi)e^{i\xi x}d\xi\]
    に対して,
    \begin{enumerate}
        \item Lebesgueの優収束定理より,任意の$x\in\R$に対して,右辺は
        \[\frac{1}{2\pi}\int_\R1_{[-\lambda,\lambda]}(\xi)\paren{1-\frac{\abs{\xi}}{\lambda}}\wh{f}(\xi)e^{i\xi x}d\xi\xrightarrow{\lambda\to\infty}\frac{1}{2\pi}\int_{\wR}\wh{f}(\xi)e^{i\xi x}d\xi=:g(\xi).\]
        \item 一方で,Fejer核$K_\lambda$は総和核だから,左辺は$\sigma_\lambda[f]\to f\in L^1(\R)$.
    \end{enumerate}
    するとこのとき,$f=g\;\ae$である.
    実際,Fatouの補題より,
    \[(0\le)\int_\R\abs{f-g}dx=\int_\R\lim_{\lambda\to\infty}\abs{\sigma_\lambda[f]-g}dx\le\liminf_{\lambda\to\infty}\int_\R\abs{\sigma_\lambda[f]-g}dx=0.\]
\end{Proof}

\begin{corollary}[Fourier変換の2回適用]\label{cor-twice-Fourier-transform-in-R1}
    $f,\wh{f}\in L^1(\R)$のとき,$(\F^2 f)(x)=(2\pi)f(-x)\;(\ae x\in\R)$.
\end{corollary}
\begin{Proof}
    \[\wh{f}(\xi)=\int_{\R}f(x)e^{-ix\xi}dx=\int_\R f(-x)e^{ix\xi}.\]
    という関係に注意して,
    $\wh{f}\in L^1(\R)$にもう一度反転公式を用いることで,
    \[\wh{f}(x)=\frac{1}{2\pi}\int_\R\wh{\wh{f}}(\xi)e^{i\xi x}d\xi.\]
    Fourier変換の単射性から,$\wh{\wh{f}}(x)=f(-x)\;\ae$
\end{Proof}

\begin{proposition}[反転可能性の十分条件]\label{prop-sufficient-condition-to-invert-Fourier-transform}
    $f\in C^2(\R)$かつ$f,f',f''\in L^1(\R)$とする(例えば$f\in C_b^2(\R)$ならばこれを満たす).
    このとき,$\wh{f}\in L^1(\wh{\R})$で,反転公式は全ての$x\in\R$で成立する.
\end{proposition}
\begin{Proof}
    Fourier変換と微分との関係\ref{lemma-differential-and-Fourier-transform}より,$\wh{f}(\xi)=\frac{\wh{f''}(\xi)}{-\xi^2}$.
    よって,$\abs{\wh{f}(\xi)}\le\Abs{\frac{\wh{f''}(\xi)}{\abs{\xi}^2}}\le\frac{\norm{f''}_1}{\abs{\xi}^2}$より,両辺を積分すれば,$\wh{f}\in L^1(\wR)$が判る.
\end{Proof}

\subsection{Riemann-Lebesgueの補題}

\begin{tcolorbox}[colframe=ForestGreen, colback=ForestGreen!10!white,breakable,colbacktitle=ForestGreen!40!white,coltitle=black,fonttitle=\bfseries\sffamily,
title=]
    $\bT$上のRiemann-Lebesgueの補題は$L^1(\bT)$の元の三角多項式による近似を用いた.
    $\R$上では$C_c^\infty(\R)$の元で近似することを考える.
    そこで,Fourier変換の微分に関する性質を用いる!
\end{tcolorbox}

\begin{lemma}[微分とFourier変換]\label{lemma-differential-and-Fourier-transform}
    $f\in C^1(\R)$について,$f,f'\in L^1(\R)$ならば,
    \[\forall_{\xi\in\wR}\;\wh{f'}(\xi)=i\xi\wh{f}(\xi).\]
\end{lemma}
\begin{Proof}
    Lebesgueの定理より,可微分な連続関数$f$が可積分な導関数を持つことと,$f$が有界変動で
    \[f(x)=f(0)+\int_0^xf'(t)dt\]
    と表せることは同値.特に,$f(x)\xrightarrow{\abs{x}\to\infty}0$が必要.
    これに注意すれば,部分積分により,
    \[\wh{f'}(\xi)=\int_\R f'(x)e^{-i\xi x}dx=\Square{f(x)e^{-i\xi x}}^\infty_{-\infty}+i\xi\int_\R f(x)e^{-i\xi x}dx=i\xi\wh{f}(\xi).\]
\end{Proof}

\begin{corollary}[Riemann-Lebesgue]\label{cor-Riemann-Lebesgue-R}
    任意の$f\in L^1(\R)$に対して,$\wh{f}(\xi)\xrightarrow{\abs{\xi}\to\infty}0$.
\end{corollary}
\begin{Proof}
    任意の$\ep>0$を取る.
    \begin{enumerate}
        \item $C_c^\infty(\R)$の$L^1(\R)$上の稠密性より,$\exists_{g\in C_c^\infty(\R)}\;\norm{f-g}_1<\ep$.
        \item $\abs{\wh{g'}(\xi)}=\abs{\xi\wh{g}(\xi)}$であるから,
        \[\abs{\wh{g}(\xi)}\le\Abs{\frac{\wh{g'}(\xi)}{\xi}}\le\frac{\norm{g'}_1}{\abs{\xi}}\]
        特に,$\abs{\xi}$が十分大きくなるように$\xi\in\wR$を取れば,$\abs{\wh{g}(\xi)}<\ep$.
    \end{enumerate}
    以上の2点より,$\abs{\xi}$が十分大きくなるように$\xi\in\wR$を取れば,
    \[\abs{\wh{f}(\xi)}\le\abs{\wh{f}(\xi)-\wh{g}(\xi)}+\abs{\wh{g}(\xi)}<\norm{f-g}_1+\ep<2\ep.\]
\end{Proof}
\begin{Proof}\mbox{}
    \begin{enumerate}[{Step}1]
        \item 一様分布の密度のFourier変換は
        \begin{align*}
            \wh{1_{[a,b]}}(\xi)&=\int_\R 1_{[a,b]}(x)e^{-ix\xi}dx=\int^b_ae^{-ix\xi}dx\\
            &=\SQuare{-\frac{e^{-ix\xi}}{i\xi}}^b_a=\frac{e^{-i\xi b}-e^{-i\xi a}}{i\xi}.
        \end{align*}
        と計算できる.
        \item 任意の$f\in L^1(\R)$は一様分布の密度の線型和で$L^1$-近似される.よって,任意の$\ep>0$に対して,単関数$g$と$L\in\R$が存在して,
        \[\abs{\wh{f}(\xi)}\le\abs{\wh{f}(\xi)-\wh{g}(\xi)}+\abs{\wh{g}(\xi)}\le\norm{f-g}_{L^1(\R)}+\ep\le 2\ep,\qquad\abs{\xi}\ge L.\]

    \end{enumerate}
\end{Proof}

\section{Fourier変換による減衰・可微分性対応}

\begin{tcolorbox}[colframe=ForestGreen, colback=ForestGreen!10!white,breakable,colbacktitle=ForestGreen!40!white,coltitle=black,fonttitle=\bfseries\sffamily,
title=]
    \begin{enumerate}
        \item そもそも反転公式が,$f$の不連続点の存在は,$\wh{f}$を可積分でなくすほどの減衰の遅さをもたらすことを示唆している.
        \item 反転公式の要件はシーソーのようなものであり,$f\in C_b^2(\R)$が一つの十分条件となるのであった.
        \item $f$も$\wh{f}$もコンパクト台を持つような$L^1(\R)$の元は$0$に限る.
    \end{enumerate}
\end{tcolorbox}

\subsection{Fejer核の設計意図}

\begin{tcolorbox}[colframe=ForestGreen, colback=ForestGreen!10!white,breakable,colbacktitle=ForestGreen!40!white,coltitle=black,fonttitle=\bfseries\sffamily,
title=]
    反転公式を通じて,Fejer核の定義\ref{def-Fejer-kernel-on-R}の設計意図が実現される.
\end{tcolorbox}

\begin{corollary}[Fejer核のFourier変換]\label{cor-Fourier-transform-of-Fejer-kernel}
    Fejer核のFourier変換は次のように表示出来る:
    \[\wh{K_\lambda}(\xi)=1_{[-\lambda,\lambda]}(\xi)\paren{1-\frac{\abs{\xi}}{\lambda}}.\]
    特に,コンパクト台を持つ.
\end{corollary}
\begin{Proof}
    右辺を$\varphi_\lambda(\xi)$とおくと,
    \begin{enumerate}
        \item Fejer核$K_\lambda$は
        \[K_\lambda(x)=\frac{1}{2\pi}\int_\R\varphi_\lambda(\xi)e^{i\xi x}d\xi=\frac{1}{2\pi}\wh{\varphi_\lambda}(-x)\]
        と定義したのであった.よって特に,$\wh{\varphi_\lambda}\in L^1(\wR)$.
        \item $\varphi_\lambda$の連続性に注意すれば,反転公式\ref{cor-inversion-theorem-in-R1}より,
        \[\forall_{x\in\R}\quad\varphi_\lambda(x)=\frac{1}{2\pi}\int_{\wR}\wh{\varphi_\lambda}(\xi)e^{i\xi x}d\xi.\]
        これは,$x,\xi$の違いを正すだけで,
        \[\varphi_\lambda(\xi)=\frac{1}{2\pi}\int_\R\wh{\varphi_\lambda}(x)e^{i\xi x}dx=\int_\R K_\lambda(-x)e^{i\xi x}dx=\wh{K_\lambda}(\xi)\]
        を意味している.
    \end{enumerate}
\end{Proof}

\subsection{Fourier変換はコンパクト台を持つものが殆どである}

\begin{tcolorbox}[colframe=ForestGreen, colback=ForestGreen!10!white,breakable,colbacktitle=ForestGreen!40!white,coltitle=black,fonttitle=\bfseries\sffamily,
title=]
    \begin{enumerate}
        \item $\F:L^1(\R)\to UC_0(\wR)$であるが,実際は殆どが$C_c(\wR)$に収まる.
        \item $f\in L^1(\R)$がコンパクト台を持つならば,$\wh{f}$は$\C$上で正則である.
    \end{enumerate}
    以上より,$f,\wh{f}$のいずれもコンパクト台を持つならば,$f=\wh{f}=0$.
\end{tcolorbox}

\begin{corollary}[Fourier変換は大抵コンパクト台を持つ]
    $\Brace{g\in L^1(\R)\mid\wh{g}\in C_c(\wR)}$は$L^1(\R)$上稠密である.
\end{corollary}
\begin{Proof}
    任意の$f\in L^1(\R)$は$K_\lambda*f$により$L^1$-近似されるのであるが,$K_\lambda*f$のFourier変換はみなコンパクト台を持つ.
\end{Proof}

\begin{proposition}[コンパクト台を持つ関数のFourier変換]\label{prop-fourier-transform-of-functions-with-compact-support}
    $f\in L^1(\R)$がコンパクトな台を持つとする.このとき,
    \begin{enumerate}
        \item 次の式は$\xi\in\C$上でも定義される:
        \[\wh{f}(\xi)=\int_\R f(x)e^{-i\xi x}dx\]
        \item $\wh{f}$は$\C$上正則である.
    \end{enumerate}
\end{proposition}

\subsection{可微分性と減衰速度の対応}

\begin{tcolorbox}[colframe=ForestGreen, colback=ForestGreen!10!white,breakable,colbacktitle=ForestGreen!40!white,coltitle=black,fonttitle=\bfseries\sffamily,
title=]
    反転公式により,$f$の不連続点の存在が,$\wh{f}$を$L^1(\wR)$から外すほどの減衰の遅さをもたらし得ることが分かる.
\end{tcolorbox}

\begin{proposition}\mbox{}
    \begin{enumerate}
        \item $f\in C^k(\R)\;(k\in\N)$について,$f,f',\cdots,f^{(k)}\in L^1(\R)$ならば,$\wh{f}(\xi)=o(\abs{\xi}^{-k})\;(\abs{\xi}\to\infty)$.
        \item $f\in L^1(\R)$が$\xi^k\wh{f}(\xi)\in L^1(\wR)$ならば,$f$のある修正が$f\in C^k(\R)$を満たす.
    \end{enumerate}
\end{proposition}
\begin{Proof}\mbox{}
    \begin{enumerate}
        \item 命題\ref{prop-sufficient-condition-to-invert-Fourier-transform}同様に部分積分を繰り返す.
        \item 
    \end{enumerate}
\end{Proof}
\begin{example}[可微分性と減衰速度の対応の例]\mbox{}
    \begin{enumerate}
        \item 定義関数$1_{[-1,1]}$のFourier変換$2\frac{\sin\xi}{\xi}$はLebesgue可積分でない\ref{prop-property-of-Dirichlet-kernel-on-R1}.
        \item Poisson核は$x=0$で可微分性がなく,$P\notin C^1(\wh{\R})$より,Cauchy分布の確率密度関数は1次の積率も持たない.
        \item コンパクト微分可能性が減衰速度に対応する.(1)の要件は$f\in C_c^k(\R)$ならば満たす.
    \end{enumerate}
\end{example}

\section{Fourier変換の各点収束と局所性原理}

\begin{tcolorbox}[colframe=ForestGreen, colback=ForestGreen!10!white,breakable,colbacktitle=ForestGreen!40!white,coltitle=black,fonttitle=\bfseries\sffamily,
title=]
    $S_\lambda[f]$が$f$にいつ各点収束するかの問題は,$\bT$上の問題に帰着し,まったく同様のふるまいをする.
    \begin{enumerate}
        \item 点$x_0\in\R$で可微分ならば,$S_\lambda[f](x_0)\to f(x_0)$.
        \item 一様収束の性質についても,まったく同様に$\bT$上の場合に帰着される.
    \end{enumerate}
\end{tcolorbox}

\subsection{Cesaro和とFejer核}

\begin{proposition}[Fejer核はDirihlet核のCesaro平均である]
    任意の$\lambda>0,x\in\R$に対して,
    \[\frac{1}{\lambda}\int^\lambda_0D_\eta(x)d\eta=K_\lambda(x).\]
\end{proposition}
\begin{Proof}
    次のように計算できる:
    \begin{align*}
        \frac{1}{\lambda}\int^\lambda_0D_\eta(x)d\eta&=\frac{1}{\lambda}\int^\lambda_0\frac{1}{2\pi}\int^\eta_{-\eta}e^{ix\xi}d\xi d\eta\\
        &=\frac{1}{2\pi\lambda}\int^\lambda_0\SQuare{\frac{e^{ix\xi}}{ix}}^\eta_{-\eta}d\eta\\
        &=\frac{1}{\pi\lambda x}\int^\lambda_0\frac{e^{ix\eta}-e^{-ix\eta}}{2i}d\eta\\
        &=\frac{1}{\pi\lambda x}\int^\lambda_0\sin(x\eta)d\eta=\frac{1-\cos(x\lambda)}{\pi\lambda x^2}.
    \end{align*}
\end{Proof}

\begin{proposition}[$\R$上のFejerの定理]
    $f\in L^1(\R)$が$x=0$を連続点にもつならば,
    \[\sigma_\lambda[f](0)\xrightarrow{\lambda\to\infty}f(0).\]
\end{proposition}
\begin{Proof}\mbox{}
    \begin{enumerate}[{Step}1]
        \item $K_\lambda$は確率密度であるから,
        \[\sigma_\lambda[f](0)-f(0)=\int_\R\Paren{f(-\xi)-f(0)}K_\lambda(\xi)d\xi.\]
        と表せる.
        \item $f$は$0$で連続との仮定から,ある$\delta>0$が存在して,
        \[\int_{\abs{\delta}\ge0}\abs{f(-\xi)-f(0)}K_\lambda(\xi)d\xi<\ep.\]
        \item $K_\lambda$は総和核であるから,$\lambda>0$を十分大きくとれば,残りの$\delta\le\abs{x}$での積分も任意に小さくできる.
    \end{enumerate}
\end{Proof}

\begin{proposition}[Fejer核の軟化性]
    任意の$f\in L^1(\R),\lambda>0$に対して,$K_\lambda*f\in C^\infty(\R)$である.
\end{proposition}
\begin{Proof}
    Fubiniの定理より,次のように計算できる.
    \begin{align*}
        K_\lambda*f(x)&=\int_\R K_\lambda(x-y)f(y)dy\\
        &=\int_\R f(y)\frac{1}{2\pi}\int^\lambda_{-\lambda}\paren{1-\frac{\abs{\xi}}{\lambda}}e^{i(x-y)\xi}d\xi dy\\
        &=\frac{1}{2\pi}\int^\lambda_{-\lambda}\paren{1-\frac{\abs{\xi}}{\lambda}}e^{ix\xi}\int_\R f(y)e^{-iy\xi}dyd\xi\\
        &=\frac{1}{2\pi}\int^\lambda_{-\lambda}\paren{1-\frac{\abs{\xi}}{\lambda}}e^{ix\xi}\wh{f}(\xi)d\xi.
    \end{align*}
    被積分関数の$x$に関する任意階の導関数は再び$\xi$に関する連続関数を定め,$\xi$に関して$[-\lambda,\lambda]$可積分であるため,
    任意階微分可能である.
\end{Proof}

\subsection{Fourier部分和とDirichlet核}

\begin{notation}
    \[S_\lambda[f](x):=\frac{1}{2\pi}\int^\lambda_{-\lambda}\wh{f}(\xi)e^{i\xi x}d\xi\]
\end{notation}

\begin{proposition}
    任意の$f\in L^1(\R),x\in\R$に対して,Fourier部分和は$\R$上のDirichlet核を用いて$(D_\lambda*f)(x)=S_\lambda[f](x)$と表せる.
\end{proposition}
\begin{Proof}
    Fubiniの定理より,次のように表せる:
    \begin{align*}
        D_\lambda*f(x)&=\int_\R f(y)D_\lambda(x-y)dy=\int_\R f(y)\frac{1}{2\pi}\int^\lambda_{-\lambda}e^{i(x-y)\xi}d\xi dy\\
        &=\frac{1}{2\pi}\int^\lambda_{-\lambda}e^{ix\xi}\int_\R f(y)e^{-iy\xi}dyd\xi\\
        &=\frac{1}{2\pi}\int^\lambda_{-\lambda}e^{ix\xi}\wh{f}(\xi)d\xi.
    \end{align*}
\end{Proof}

\subsection{$L^1(\bT)$上の局所性原理}

\begin{tcolorbox}[colframe=ForestGreen, colback=ForestGreen!10!white,breakable,colbacktitle=ForestGreen!40!white,coltitle=black,fonttitle=\bfseries\sffamily,
title=局所的に周期関数で近似すればそこでのFourier級数の収束先に一致する]
    $f\in L^1(\R)$の$x_0\in\R$でのFourier部分和の収束の如何は,その近傍を切り取って無理やり$\bT$上に植えた関数$f|_{(x_0-\pi,x_0+\pi]}\in L^1(\bT)$の$t=0$でのFourier部分和の収束に一致する.
\end{tcolorbox}

\begin{theorem}[$\R$上の局所性原理]
    $f\in L^1(\R)$と$x_0\in\R$について,2つの極限
    \[\lim_{\lambda\to\infty}S_\lambda[f](x_0),\qquad\lim_{N\to\infty}S_N[\varphi](0),\quad\varphi(t):=f(x_0+t)\in L^1(\bT)\]
    は,片方が存在すればもう片方も存在し,その値は一致する.
\end{theorem}
\begin{Proof}
    3つ目の補題より,$f$を平行移動した関数を$F(x):=f(x_0+x)$とすると,$S_\lambda[f](x_0)=S_\lambda[F](0)$.
    1,2つ目の補題より,$\abs{\lim_{\lambda\to\infty}S_\lambda[F](0)-\lim_{N\to\infty}S_N[\varphi](0)}=0$.
\end{Proof}
\begin{remarks}[Riemannの局所性原理の引き戻し]
    $S_\lambda[f](x_0)$が$\lambda\to\infty$の極限で収束するか否かは,$f|_{(x_0-\pi,x_0+\pi]}\in L^1(\bT)$のFourier部分和の$t=0$での収束性に連動しており,
    したがって局所性の原理\ref{thm-Riemann-locality}から,$f$の$x_0$の近傍での振る舞いのみに依存する.
\end{remarks}

\begin{corollary}
    $f\in L^1(\R)$について,
    \begin{enumerate}
        \item $f$が$x_0\in\R$で左右の極限を持ち,さらに左右の微分係数を持つならば,
        \[S_\lambda[f](x_0)\xrightarrow{\lambda\to\infty}\frac{f(x_0+0)+f(x_0-0)}{2}.\]
        \item $f\in C(\R)$であって,$\lim_{\lambda\to\infty}S_\lambda[f](0)$が存在しないものが存在する.
    \end{enumerate}
\end{corollary}
\begin{remark}
    反転公式の系より,連続な修正が存在しない$f\in L^1(\R)$のFourier変換$\wh{f}$は$\wR$上可積分ではないが,
    特定の$x\in\R$に関して$S_\lambda[f](x)$の極限は存在し得る.
\end{remark}

\subsection{証明}

\begin{tcolorbox}[colframe=ForestGreen, colback=ForestGreen!10!white,breakable,colbacktitle=ForestGreen!40!white,coltitle=black,fonttitle=\bfseries\sffamily,
title=]
    証明ではRiemann-Lebesgueの補題が肝要になる.
\end{tcolorbox}

\begin{lemma}[Fourier部分和の差の収束1]
    $f\in L^1(\R)$に対して,$\varphi:=f|_{[-\pi,\pi]}\in L^1(\bT)$とする.このとき,
    \[S_N[f](0)-S_N[\varphi](0)\xrightarrow{N\to\infty}0.\]
\end{lemma}
\begin{Proof}
    それぞれのFourier部分和は,
    \begin{enumerate}
        \item $\int^N_{-N}e^{-i\xi x}d\xi\frac{2}{x}\sin(Nx)$に注意して,Fubiniの定理より,
        \begin{align*}
            S_N[f](0)&=\frac{1}{2\pi}\int^N_{-N}\wh{f}(\xi)d\xi=\frac{1}{2\pi}\int^N_{-N}\paren{\int_\R f(x)e^{-i\xi x}dx}d\xi\\
            &=\frac{1}{2\pi}\int_\R f(x)\paren{\int^N_{-N}e^{-i\xi x}d\xi}dx=\frac{1}{2\pi}\int_\R f(x)\frac{\sin(Nx)}{x/2}dx.
        \end{align*}
        \item \[S_N[\varphi](0)=(D_N*\varphi)(0)=\frac{1}{2\pi}\int^\pi_{-\pi}D_N(-s)\varphi(s)ds=\frac{1}{2\pi}\int^\pi_{-\pi}f(x)\frac{\sin\paren{N+\frac{1}{2}}x}{\sin\frac{x}{2}}dx.\]
    \end{enumerate}
    以上より,
    \[S_N[f](0)-S_N[\varphi](0)=\frac{1}{2\pi}\int_{\abs{x}\le\pi}f(x)\paren{\frac{\sin Nx}{x/2}-\frac{\sin\paren{N+\frac{1}{2}}x}{\sin\frac{x}{2}}}dx+\frac{1}{\pi}\int_{\abs{x}>\pi}\frac{f(x)}{x}\sin (Nx)dx.\]
    であるが,この第2項は,$\R$上のRiemann-Lebesgueの補題\ref{cor-Riemann-Lebesgue-R}の系より,$N\to\infty$のとき$0$に収束する.
    続いて第1項も,$\sin\paren{N+\frac{1}{2}}x=\sin Nx\cos\frac{x}{x}+\cos Nx\sin\frac{x}{2}$より,
    \[\frac{1}{2\pi}\int_{\abs{x}\le\pi}f(x)\paren{\paren{\frac{1}{x/2}-\frac{\cos(x/2)}{\sin(x/2)}}\sin(Nx)-\cos (Nx)}dx\]
    と変形すると,実は$h(x):=\frac{1}{x/2}-\frac{\cos(x/2)}{\sin(x/2)}$は$[-\pi,\pi]$上で有界であることより,$\bT$上のRiemann-Lebesgueの補題の系\ref{lemma-cor-of-Riemann-Lebesgue}より,やはり$0$に収束する.
    $h(x)$の有界性は,$0$の近傍で有界であることを確認すれば十分で,$(\tan x)^{-1}$の$x=0$での極の主要部は$\frac{1}{x}$のみであるから,これは判る.
\end{Proof}

\begin{lemma}[Fourier部分和の差の収束2]
    $f\in L^1(\R)$について,
    \[\forall_{\ep>0}\;\exists_{n\in\N}\;\forall_{N\ge n}\;\forall_{\lambda\in\cointerval{N,N+1}}\;\abs{S_N[f](0)-S_\lambda[f](0)}<\ep.\]
\end{lemma}
\begin{Proof}
    任意の$\ep>0$を取ると,Riemann-Lebesgueの補題\ref{cor-Riemann-Lebesgue-R}より,
    十分大きな$\xi$について$\abs{\wh{f}(\xi)}<\ep/2$だから,
    \[\exists_{N\in\N}\;\forall_{\lambda\in[N,N+1]}\quad\abs{S_N[f](0)-S_\lambda[f](0)}\le\paren{\int^{-N}_{-\lambda}+\int^\lambda_{N}}\abs{\wh{f}(\xi)}d\xi<\ep.\]
\end{Proof}

\begin{lemma}[平行移動させた関数のFourier部分和]
    $f\in L^1(\R)$と$x_0\in\R$について,平行移動させた関数を$F(x):=f(x_0+x)$とすると,$S_\lambda[F](0)=S_\lambda[f](x_0)$.
\end{lemma}
\begin{Proof}
    \begin{align*}
        \wh{F}(\xi)&=\int_\R f(x_0+x)e^{-i\xi x}dx=\int_\R f(y)e^{-i\xi y}e^{-\xi x_0}dy=e^{i\xi x_0}\wh{f}(\xi)
    \end{align*}
    だから,
    \[S_\lambda[F](0)=\frac{1}{2\pi}\int^\lambda_{-\lambda}\wh{F}(\xi)d\xi=\frac{1}{2\pi}\int^\lambda_{-\lambda}e^{i\xi x_0}\wh{f}(\xi)d\xi=S_\lambda[f](x_0).\]
\end{Proof}

\section{関数の周期化:$\bT,\R$の往来}

\begin{tcolorbox}[colframe=ForestGreen, colback=ForestGreen!10!white,breakable,colbacktitle=ForestGreen!40!white,coltitle=black,fonttitle=\bfseries\sffamily,
title=]
    核関数は$\wR$の離散部分空間$\Z\subset\wR$上の挙動を中心に一般化した.
    $L^1(\R)$の元は$\Z$上でのFourier変換の値を通じて$L^1(\bT)$の元と対応がつき,
    この上で反転公式の消息にアクセスすることができる(Poissonの和公式).
\end{tcolorbox}

\subsection{周期化の定義}

\begin{tcolorbox}[colframe=ForestGreen, colback=ForestGreen!10!white,breakable,colbacktitle=ForestGreen!40!white,coltitle=black,fonttitle=\bfseries\sffamily,
title=]
    $f\in L^1(\R)$について,
    \[\wh{f}(n)=\int_\R f(x)e^{-inx}dx=\sum_{k\in\Z}\int^{2\pi}_0f(x+2\pi k)e^{-inx}dx=\dint_\bT\Pe[f]e^{-inx}dx=\wh{\Pe[f]}(n).\]
    というアイデアである.
    $\Pe:L^1(\R)\to L^1(\bT)$はノルム減少的である.
\end{tcolorbox}

\begin{definition}[periodization]
    $f\in L^1(\R)$に対して,
    Fourier変換$\wh{f}\in L^1(\wR)$の整数上での値
    $(\wh{f}(n))_{n\in\Z}$をFourier係数に持つ$\bT$上の関数
    \[\Pe[f](t)=F(t):=2\pi\sum_{k\in\Z}f(t+2\pi k),\qquad(t\in[0,2\pi])\]
    を\textbf{周期化}という.
\end{definition}
\begin{remarks}
    $x\in[0,2\pi]$に対して,$f$の$x+2\pi\Z$での値の和の$2\pi$倍を返す関数が$\Pe[f]$である.
\end{remarks}

\begin{lemma}[定義の成功]
    $f\in L^1(\bT)$の周期化$\Pe[f]:\bT\to\R$について,
    \begin{enumerate}
        \item $\Pe[f](t)=2\pi\sum_{k\in\Z}f(t+2\pi k)$は$\bT$上概収束し,$L^1(\bT)$-収束もし,$\norm{\Pe[f]}_{L^1(\bT)}\le\norm{f}_{L^1(\R)}$.特に,$\Pe[f]\in L^1(\bT)$.
        \item たしかにFourier係数は$\forall_{n\in\N}\;\wh{f}(n)=\wh{\Pe[f]}(n)$を満たす.
    \end{enumerate}
\end{lemma}
\begin{Proof}\mbox{}
    \begin{enumerate}
        \item $\Pe[f]$の$L^1$-ノルムは
        \[\norm{\Pe[f]}_{L^1(\bT)}=\Norm{2\pi\sum_{k\in\Z}f(t+2\pi k)}_{L^1(\bT)}\le\frac{1}{2\pi}\int_\bT2\pi\sum_{k\in\Z}\abs{f(t+2\pi k)}dt=\int_\R\abs{f(x)}dx=\norm{f}_{L^1(\R)}<\infty\]
        を満たすから,$\sum_{k\in\Z}\abs{f(t+2\pi k)}dt$は殆ど至る所有限値である.
        また,$f\in L^1(\R)$より,$\sum_{k=-N}^Nf(t+2\pi k)$は$N$についての$L^1$-Cauchy列を定めるから,$L^1(\bT)$-収束もする.
        \item 積分区間の分割$\R=\cup_{k\in\Z}[2\pi k,2\pi (k+1)]$を通じて,
        Lebesgueの優収束定理より,
        次のように計算できる:
        \begin{align*}
            \wh{F}(n)&=\frac{1}{2\pi}\int_\bT F(t)e^{-int}dt=\frac{1}{2\pi}\int_\bT e^{-int}\paren{2\pi\sum_{k\in\Z}f(t+2\pi k)}dt\\
            &=\sum_{k\in\Z}\int_\bT e^{-int}f(t+2\pi k)dt=\int_\R f(t)e^{-int}dt=\wh{f}(n).
        \end{align*}
    \end{enumerate}
\end{Proof}

\subsection{Poissonの和公式}

\begin{tcolorbox}[colframe=ForestGreen, colback=ForestGreen!10!white,breakable,colbacktitle=ForestGreen!40!white,coltitle=black,fonttitle=\bfseries\sffamily,
title=]
    $f$と$\Pe[f]$で,Fourier変換の$n\in\Z$での値は等しいから,
    これらがいずれも反転可能であるとき,$t\in\bT$での値は一致する.
    特にFourier係数の和$\sum_{n\in\Z}\wh{f}(n)$は周期化の$0$での値$\Pe[f](0)$に等しい.
    これをPoissonの和公式という.
\end{tcolorbox}

\begin{proposition}[周期化を通じた$\bT$上での反転公式]
    $f\in L^1(\R)$は次を満たすとする:
    \begin{enumerate}[{[}1{]}]
        \item $\wh{f}|_\Z\in l^1(\Z)$を満たす.
        \item 任意の$t\in\bT$について$\Pe[f](t)=2\pi\sum_{k\in\Z}f(t+2\pi k)$は収束する.
        \item $\Pe[f]\in C(\bT)$を満たす
    \end{enumerate}
    このとき,任意の$t\in[0,2\pi]$について,
    \[\sum_{n\in\Z}\wh{f}(n)e^{int}=2\pi\sum_{k\in\Z}f(t+2\pi k).\]
    特に$t=0$のときをPoissonの和公式という;
    \[\sum_{n\in\Z}\wh{f}(n)=2\pi\sum_{k\in\Z}f(2\pi k).\]
\end{proposition}
\begin{Proof}\mbox{}
    \begin{description}
        \item[Step1] [1]の条件$\sum_{n\in\Z}\abs{\wh{f}(n)}<\infty$と$\bT$上の反転公式\ref{thm-l1-series}
        より左辺は存在し,$f(t)$に等しい.
        \[f(t)\overset{\as}{=}\sum_{n\in\Z}\wh{f}(n)e^{int}=\LHS.\]
        [2]の条件より任意の$t\in\bT$について級数$F(t)=\sum_{k\in\Z}f(t+2\pi k)$が収束するという条件より,右辺も収束し,
        \[\RHS=2\pi\sum_{k\in\Z}f(t+2\pi k)\overset{\as}{=}\Pe[f](t).\]
        \item[Step2] $f,\Pe[f]$は等しいFourier係数を持つので,$L^1(\bT)$の元としては等しい.
        さらに,
        [3]の条件$\Pe[f]\in C(\bT)$より,任意の$t\in[0,2\pi]$で等号が成り立つ.
    \end{description}
\end{Proof}

\subsection{Poisson核}

\begin{corollary}[$\R$上のPoisson核に関するPoissonの和公式]
    \[\sum_{n\in\Z}e^{-s\abs{n}}=2\sum_{k\in\Z}\frac{s}{s^2+4\pi^2k^2}.\]
\end{corollary}
\begin{Proof}
    Poisson核$P(x)=\frac{1}{\pi}\frac{1}{1+x^2}$が定める総和核
    \[P_s(x):=\frac{1}{s}P\paren{\frac{x}{s}}=\frac{1}{\pi}\frac{s}{s^2+x^2}\quad(s>0)\]
    を考える.
    $\wh{P}(\xi)=e^{-\abs{\xi}}$であったから,変数変換$x=sy$より,
    \[\wh{P_s}(\xi)=\int_\R P_s(x)e^{-i\xi x}dx=\int_\R P(y)e^{-i\xi sy}dy=\wh{P}(\xi s)=e^{-s\abs{\xi}}.\]
    \begin{enumerate}
        \item $\wh{P}(n)=e^{-\abs{n}}$は絶対収束する.
        \item $\sum_{k\in\Z}P_s(t+2\pi k)=\sum_{k\in\Z}\frac{1}{s}e^{-\Abs{\frac{t+2\pi k}{s}}}$は任意の$t\in\Z$について収束し,
        \item $t\in\bT$について連続である.
    \end{enumerate}
    から,Poissonの和公式より,Fourier係数の和は$\Pe[P_s](0)$の値に等しく,
    \[\sum_{n\in\Z}\wh{P_s}(\xi)=\sum_{n\in\Z}e^{-\abs{n}}=2\sum_{k\in\Z}\frac{s}{s^2+(2\pi k)^2}=2\pi\sum_{k\in\Z}P_s(2\pi k).\]
\end{Proof}

\subsection{Fejer核}

\begin{notation}
    $\bT,\R$上のFejer核を
    \[\wh{K_N^\bT}(n)=\paren{1-\frac{\abs{n}}{N+1}}1_{\abs{n}\le N},\quad\wh{K_\lambda^\R}(\xi)=\paren{1-\frac{\abs{\xi}}{\lambda}}1_{[-\lambda,\lambda]}.\]
    と定めたのであった.これは次のように表示できるのであった:
    \[K_N^\bT(t)=\frac{1}{N+1}\paren{\frac{\sin(N+1)t/2}{\sin(t/2)}}^2,\quad K_\lambda^\R(x)=\frac{\lambda}{2\pi}\paren{\frac{\sin(\lambda x/2)}{\lambda x/2}}^2.\]
\end{notation}

\begin{proposition}\mbox{}
    \begin{enumerate}
        \item $\forall_{n\in\Z}\;\wh{K^\R_{N+1}}(n)=\wh{K^\bT_N}(n)$.
        \item $\forall_{t\in\bT}\;K_N^\bT(t)=2\pi\sum_{k\in\Z}K^\R_{N+1}(t+2\pi k)$,特に,
        \[\forall_{t\in\bT}\quad \frac{1}{(\sin(t/2))^2}=\sum_{k\in\Z}\frac{4}{(t+2\pi k)^2}.\]
        \item 特に,次が得られる:
        \[\forall_{t\in\R\setminus2\pi\Z}\;\forall_{x\in\R\setminus\pi\Z}\quad\frac{1}{\sin^2x}=\sum_{k\in\Z}\frac{1}{(x+\pi k)^2}.\]
    \end{enumerate}
\end{proposition}
\begin{Proof}\mbox{}
    \begin{enumerate}
        \item 明らか.
        \item Fourier係数が一致し,その上いずれも連続であるため.そしてこれに表示を代入すれば良い.
    \end{enumerate}
\end{Proof}

\subsection{Dirichlet核}

\begin{tcolorbox}[colframe=ForestGreen, colback=ForestGreen!10!white,breakable,colbacktitle=ForestGreen!40!white,coltitle=black,fonttitle=\bfseries\sffamily,
title=]
    $D^\R_\lambda\notin L^1(\R)$であるため,$D^\R_\lambda$の周期化がすぐ$D^\bT_N$になるわけではない.
\end{tcolorbox}

\begin{proposition}
    $D_\lambda^\R$の周期化は次で与えられる:
    \begin{enumerate}
        \item $\lambda>0$が整数でないとき,$D_N^\bT$.
        \item $\lambda>0$が整数であるとき,$\frac{D^\bT_{N-1}+D_N^\bT}{2}$.
    \end{enumerate}
\end{proposition}

\subsection{Gauss核}

\begin{tcolorbox}[colframe=ForestGreen, colback=ForestGreen!10!white,breakable,colbacktitle=ForestGreen!40!white,coltitle=black,fonttitle=\bfseries\sffamily,
title=]
    Gauss核の周期化は熱核の定数倍になる.
    Gauss核に再生性があったのと同様,熱核には半群性がある.
\end{tcolorbox}

\section{$M(\R)$上のFourier変換}

\begin{tcolorbox}[colframe=ForestGreen, colback=ForestGreen!10!white,breakable,colbacktitle=ForestGreen!40!white,coltitle=black,fonttitle=\bfseries\sffamily,
title=]
    Fourier変換は$L^1(\R)\mono M(\R)$に沿って延長されるが,Riemann-Lebesgueの補題は成り立たなくなり,
    $\F:M(\R)\to UC(\R)$しか得られない.
    そして$\Im\F\subset UC(\R)$の特定は極めて困難な問題になる\cite{Katznelson-Fourier} Section VI.2.
\end{tcolorbox}

\begin{remarks}\mbox{}
    \begin{enumerate}
        \item $M(\R)$を$\R$上の符号付きBorel測度の空間とすると,次のペアリングを通じた双対空間$M(\R)=(C_0(\R))^*$と同一視出来る:
        \[(f|\mu):=\int f\o{d\mu},\qquad f\in C_0(\R),\mu\in M(\R).\]
        \item $M(\R)$上の全変動ノルム
        \[\norm{\mu}_{M(\R)}:=\int\abs{d\mu}\]
        はペアリングを通じた$\sigma(M(\R),C_0(\R))$-位相と同値になる.
        \item $f\mapsto f(x)dx$は埋め込み$L^1(\R)\mono M(\R)$を定める.
        \item $M(\R)$上の畳み込み
        \[(\mu*\nu)(E):=\int\mu(E-y)d\nu(y),\qquad E\in\B(\R)\]
        は,$M(\R)$と$C_0(\R)$の間の畳み込みの随伴
        \[(\mu*\nu|\varphi):=(\mu|\nu*\wt{\varphi})\]
        としての定義(超関数の畳み込みとしての定義)と同値になる.
    \end{enumerate}
\end{remarks}

\subsection{Fourier-Stieltjes変換の定義}

\begin{definition}
    $\mu\in M(\R)$の\textbf{Fourier-Stieltjes変換}とは,
    \[\wh{\mu}(\xi):=\o{\int e^{i\xi x}\o{d\mu(x)}}=\int e^{-i\xi x}d\mu(x),\qquad\xi\in\wh{\R}.\]
    をいう.$L^1(\R)\mono M(\R)$上では通常のFourier変換に一致し,その延長となっている.
\end{definition}

\subsection{Fourier-Stieltjes変換の単射性}

\begin{theorem}[Parseval's formula]
    $\mu\in M(\R)$,$f\in L^1(\R)\cap C(\R)$は$\wh{f}\in L^1(\wh{\R})$も満たすとする.このとき,
    \[\int f(x)d\mu(x)=\frac{1}{2\pi}\int\wh{f}(\xi)\wh{\mu}(-\xi)d\xi.\]
\end{theorem}
\begin{Proof}
    反転公式\ref{cor-inversion-theorem-in-R1}より,
    \[f(x)=\frac{1}{2\pi}\int\wh{f}(\xi)e^{i\xi x}=\frac{1}{2\pi}\int\wh{f}(\xi)\o{\wh{\mu}(\xi)}d\mu.\]
    これを代入して,
    \[\int f(x)d\mu(x)=\frac{1}{2\pi}\iint\wh{f}(\xi)e^{i\xi x}d\mu(x)d\xi=\frac{1}{2\pi}\int\wh{f}(\xi)\wh{\mu}(-\xi).\]
\end{Proof}

\begin{corollary}[uniqueness theorem]
    $\F:M(\R)\to UC(\wh{\R})$は単射である.
\end{corollary}

\subsection{像の特徴付け1}

\begin{theorem}
    $\varphi\in C_b(\wh{\R})$について,次は同値:
    \begin{enumerate}
        \item ある測度のFourier-Stieltjes変換である:$\varphi\in\F(M(\R)_+)$.
        \item 任意の$f\in C^\infty_c(\R)_+$に対して,
        \[\int\wh{f}(\xi)\varphi(-\xi)\ge0.\]
    \end{enumerate}
\end{theorem}
\begin{Proof}\mbox{}
    \begin{description}
        \item[(1)$\Rightarrow$(2)] 
    \end{description}
\end{Proof}

\subsection{像の特徴付け2:Bochnerの定理}

\begin{definition}
    $\varphi:\wh{\R}\to\C$が\textbf{正定値}であるとは,任意の$\xi_1,\cdots,\xi_N\in\wh{\R}$と$z_1,\cdots,z_N\in\C$に対して,
    \[\sum_{j,k\in[N]}\varphi(\xi_j-\xi_k)z_j\o{z_k}\ge0\]
    を満たすことをいう.
\end{definition}

\begin{proposition}
    $\varphi$を正定値とする.このとき,
    \begin{enumerate}
        \item $\varphi(-\xi)=\o{\varphi(\xi)}$が成り立つ.
        \item $\abs{\varphi(\xi)}\le\varphi(0)$.
        \item 原点で連続であるならば,$\R$上で一様連続である.
    \end{enumerate}
\end{proposition}
\begin{Proof}
    $N=2,z_1=z,z_2=z$と取り,$\xi:=\xi_2-\xi_1$とすると,
    \begin{align*}
        \sum_{j,k=1}^2\varphi(\xi_j-\xi_k)z_j\o{z_k}&=\varphi(0)1\o{1}+\varphi(0)z\o{z}+\varphi(\xi_1-\xi_2)\o{z}+\varphi(\xi_2-\xi_1)z\\
        &=\varphi(0)(1+\abs{z}^2)+\varphi(-\xi)\o{z}+\varphi(\xi)z\ge0.
    \end{align*}
    \begin{enumerate}
        \item $z:=1$とすると$\varphi(\xi)+\varphi(-\xi)\ge-\varphi(0)2$.特に,$\varphi(\xi)+\varphi(-\xi)\in\R$.
        $z:=i$とすると,$i(\varphi(\xi)-\varphi(-\xi))\in\R$.よって,$\varphi(-\xi)=\o{\varphi(\xi)}$.
        \item $z\varphi(\xi)=-\abs{\varphi(\xi)}$を満たす$z\in\C$を取ると,$2\varphi(0)-2\abs{\varphi(\xi)}\ge0$.
    \end{enumerate}
\end{Proof}

\begin{theorem}[Bochner]
    関数$\varphi:\wh{\R}\to\C$について,次は同値:
    \begin{enumerate}
        \item ある測度のFourier-Stieltjes変換である:$\varphi\in\F(M(\R)_+)$.
        \item $\varphi$は正定値な連続関数である.
    \end{enumerate}
\end{theorem}

\subsection{可微分性の対応}

\begin{lemma}
    $\varphi=\wh{\mu}\;(\mu\in M(\R)_+)$は次を満たすとする:
    \[2\varphi(0)-\varphi(h)-\varphi(-h)=O(h^2).\]
    原点で2階微分可能ならばこれを満たす.
    このとき,
    \begin{enumerate}
        \item 2次の絶対積率が存在する:$\int x^2d\mu<\infty$.
        \item 2階微分$\varphi''$が存在し,$\wh{\R}$上一様連続である.
    \end{enumerate}
\end{lemma}

\begin{proposition}
    $\varphi=\wh{\mu}\;(\mu\in M(\R)_+)$は原点で$2m$階微分可能とする.
    このとき,
    \begin{enumerate}
        \item $2m$次の絶対積率が存在する:$\int x^{2m}d\mu<\infty$.
        \item 2階微分$\varphi^{(2m)}$が存在し,$\wh{\R}$上一様連続である.
        \item $\varphi^{(2m)}(0)=0$ならば,$\mu=\varphi(0)\delta_0$である.
    \end{enumerate}
\end{proposition}

\subsection{原点で解析的な正定値関数}

\begin{tcolorbox}[colframe=ForestGreen, colback=ForestGreen!10!white,breakable,colbacktitle=ForestGreen!40!white,coltitle=black,fonttitle=\bfseries\sffamily,
title=]
    原点で解析的な正定値関数は実軸の近傍$\Brace{\zeta\in\C\mid\abs{\Im\zeta}<a}$上に解析接続され,したがってBochnerの定理からある測度のFourier-Stieltjes変換である.
    さらに虚軸上でも正則ならば,整関数を定める必要がある.
\end{tcolorbox}

\begin{lemma}
    $F:=\wh{\mu}\;(\mu\in M(\R)_+)$は原点で解析的とする.このとき,
    \begin{enumerate}
        \item ある$b>0$について$\int e^{b\abs{x}}d\mu<\infty$.
        \item $\wh{\mu}$の$\wh{\R}$への制限はFourier-Stieltjes変換
        \[F(\xi)=\int e^{-\xi x}d\mu(x)\]
        である.
    \end{enumerate}
\end{lemma}

\begin{theorem}[Marcinkiewiczの定理の特別な場合]
    ある多項式$P$について,
    $e^{P(\xi)}$はある測度のFourier-Stieltjes変換であるとする:$e^{P(\xi)}\in \F(M(\R)_+)$.
    このとき,$\deg P\le 2$.
\end{theorem}

\chapter{$L^1(\R^d)$のFourier解析}

\begin{quotation}
    \begin{description}
        \item[$L^2(\R^d)$理論では$L^1(\R^d)\cap L^2(\R^d)$の稠密部分空間に注目する] $L^1(\R)$の議論を$L^1(\bT)$から始める際には$C_c^\infty(\R)$に注目し,
        $C_c^\infty(\R)$に成り立つ反転公式の消息の一般化が理論の大局であったが(最終的に反転積分が定義可能な全ての場合について成り立つことがわかった),
        $L^1(\R^d)$ではより広い$\S(\R^d)$と$\S'(\R^d)$を開発することを考える.
        \item[超関数理論ではペアリングに関する随伴に沿って各演算を延長する] 
        $L^p(\R^d)\;(p\in[1,2])$までは上述の方法でうまく行ったが,$L^p(\R^d)\;(p>2)$以降は,Fourier変換は関数にはならなくなる.そこで,反転公式を得るには,分布の理論(特に緩増加分布の理論)を必要とする.
        超関数は$\D:=C_c^\infty(\R^n)$をFrechet空間とし,その双対空間$\D'$の元とする.
        \begin{enumerate}
            \item 任意の線型写像$T:\D(\R^n)\to\D(\R^n)$が$w^*$-位相について連続ならば,延長$\D'(\R^n)\to\D'(\R^n)$を持つ.
            \item また,双対$T^*:\D(\R^n)\to\D(\R^n)$とそのペアリングに関する随伴$T^*:\D'(\R^n)\to\D'(\R^n)$の言葉でも定義出来る.(2つは一致するかは未確認).
        \end{enumerate}
        微分や$C^\infty(\R^d)$-倍,畳み込みが(2)の例である.
        しかし畳み込みは$\D'(\R^n)\otimes\D'(\R^n)$全域上には延長しない.
        これによりFourier変換も$\D'(\R^n)$全域上には延長しない.
        \item[超関数の層] 超関数には直接各点ではアクセスできないが,任意の開集合上に$\D(U)$が定まっており,それから$\D(\R^d)$が組み立てられている.
        \item[緩増加関数] $\D(\R^d)\subset\S(\R^d)$をテスト関数として取り直すことで,$\D'(\R^d)$の有用な部分空間を得る.これを\textbf{緩増加分布}という.$\S(\R^d)$はやはりFourier変換の騎手であり,$\D'(\R^d)$上には延長しないが$\S'(\R^d)$上にはFourier変換は延長する.
    \end{description}
\end{quotation}

\begin{notation}\mbox{}
    \begin{enumerate}
        \item $\D(\R^d):=C_c^\infty(\R^d)$と表す.
        \item $\xi\cdot x$で$\R^d$の標準内積を表し,ノルムを$\abs{x}$で表す.
        \item このノルムの記法を用いて,$f(x)=o(\abs{x}^{-k})\;(\abs{x}\to\infty):\Leftrightarrow\abs{x}^kf(x)\xrightarrow{\abs{x}\to\infty}0$と定める.
        \item $\wh{\R^d}=\R^d$を混用する.
    \end{enumerate}
\end{notation}

\section{$\S(\R^d)$上のFourier変換}

\begin{tcolorbox}[colframe=ForestGreen, colback=ForestGreen!10!white,breakable,colbacktitle=ForestGreen!40!white,coltitle=black,fonttitle=\bfseries\sffamily,
title=]
    Fourier変換の滑らかさと減衰速度の対応を考えると,滑らかで減衰の速い関数のクラスは$\F$の不変部分空間になることが期待できる.
\end{tcolorbox}

\subsection{Fourier変換の定義}

\begin{definition}
    $f\in L^1(\R^d)$の\textbf{Fourier変換}とは,次の関数$\wh{f}\in UC_0(\R^d)$
    \[\wh{f}(\xi):=\int_{\R^d}f(x)e^{-i\xi\cdot x}dx\qquad\xi\in\R^d\]
    をいう.\textbf{反転Fourier変換}とは,
    \[\wc{f}(\xi):=\int_{\R^d}f(x)e^{i\xi\cdot x}dx\qquad\xi\in\R^d\]
    をいう.
\end{definition}
\begin{remarks}\mbox{}
    \begin{enumerate}
        \item 変数分離型$f(x)=f_1(x_1)\cdots f_d(x_d)$であったならば,そのFourier変換はやはり変数分離型で,$\wh{f}(\xi)=\wh{f_1}(\xi_1)\cdots\wh{f_d}(\xi_d)$である.
        \item そして,一般の関数$f\in L^1(\R^d)$は,変数分離型の関数の線型和で近似出来る\cite{木田良才-Fourier}.このために,一般の$d\in\N^+$での理論も$L^1(\R)$に似通う.
    \end{enumerate}
\end{remarks}

\subsection{Schwartz急減少関数の定義と性質}

\begin{definition}[rapidly decreasing, Schwartz function]\mbox{}
    \begin{enumerate}
        \item $f:\R^d\to\R$が\textbf{急減少}であるとは,$\forall_{k\in\N}\;f(x)=o(\abs{x}^{-k})\;(\abs{x}\to\infty)$を満たすことをいう.
        \item \textbf{Schwartz急減少関数}とは,$f\in C^\infty(\R^d)$を満たす急減少関数であって,任意の多重指数$\al\in\N^d$について$\partial^\al f$も急減少になるものをいう.
        \item Schwartz急減少関数の全体を$\S:=\S(\R^d)$で表す.
    \end{enumerate}
\end{definition}

\begin{lemma}[$\S$は多項式環上の代数になる]\mbox{}
    \begin{enumerate}
        \item $C_c^\infty(\R^d)\subset\S$.よって特に$\S$は$L^p(\R^d)\;(1\le p<\infty)$上稠密である.
        \item $\forall_{f\in\S}\;\forall_{\al\in\N^d}\;\partial^\al f\in\S$.
        \item $\forall_{p\in\C[X]}\;\forall_{f\in\S}\;pf\in\S$.
        \item $\forall_{f,g\in\S}\;fg\in\S$.
    \end{enumerate}
    (4)だけでなく,実は$\S$は畳み込みについても閉じるが,これを示すには反転公式から従う$\F$の$\S$上の全単射性が必要\ref{cor-Fourier-transform-is-bijection-on-S}.
    これは$\S$が$\F$の不変部分空間であるためには必要な性質である.
\end{lemma}
\begin{Proof}\mbox{}
    \begin{enumerate}
        \item $C_c^\infty(\R^d)$の元は任意階の微分も十分遠くからは消えているから,当然急減少である.
        \item 明らか.
        \item 微分の線形性から,Schwartz急減少関数の線型和は再びSchwartz急減少関数であるため.
        \item Leibniz則より,$fg$の任意階の微分も$\S$の元の積と和で書かれる.
    \end{enumerate}
\end{Proof}

\subsection{Fourier変換と微分と多項式}

\begin{tcolorbox}[colframe=ForestGreen, colback=ForestGreen!10!white,breakable,colbacktitle=ForestGreen!40!white,coltitle=black,fonttitle=\bfseries\sffamily,
title=微分と乗算の対応する世界]
    $\partial_{x_i}$と$i\xi_i$,$x_i\times$と$i\partial_{\xi_i}$とは可換になる:
    $\wh{\partial^\al f}=i^{\abs{\al}}\xi^\al\wh{f}$,$\wh{x^\al f}=i^{\abs{\al}}\partial^\al\wh{f}$
    \[\xymatrix{
        \S\ar[r]^-{\F}\ar[d]_-{\partial^\al_x}&\S\ar[d]^-{i^{\abs{\al}}\xi^\al\times}&&\S\ar[r]^-{\F}\ar[d]_-{x^\al\times}&\S\ar[d]^-{i^{\abs{\al}}\partial^\al_\xi}\\
        \S\ar[r]^-{\F}&\S&&\S\ar[r]^-\F&\S
    }\]
    この消息を通じて,$\F(\S)\subset\S$が判る.
\end{tcolorbox}

\begin{proposition}[$\partial_{x_i}$-微分により対応する$i\xi_j$が出てくる]
    任意の$f\in\S$と$\al\in\N^d$に対して,$\wh{\partial^\al f}(\xi)=i^{\abs{\al}}\xi^\al\wh{f}(\xi)$.
\end{proposition}
\begin{Proof}
    $\al=(1,0,\cdots,0)$の場合について示す.他の場合も全く同様である.
    まず,Fubiniの定理より,積分を
    \begin{align*}
        \wh{f_{x_1}}(\xi)&=\int_{\R^d}f_{x_1}e^{-i\xi\cdot x}dx\\
        &=\int_{\R^{d-1}}\paren{\int_\R f_{x_1}e^{-i\xi_1x_1}}e^{-i(\xi_2x_2+\cdots+\xi_dx_d)}dx_2\cdots dx_d.
    \end{align*}
    このとき,$f(x)e^{-i\xi_1x_1}\xrightarrow{\abs{x}\to\infty}0$に注意すれば,部分積分より,
    \begin{align*}
        \int_\R f_{x_1}e^{-i\xi_1x_1}dx_1&=\Square{fe^{-i\xi_1x_1}}^\infty_{-\infty}-\int_\R fe^{-i\xi_1x_1}(-i\xi_1)dx_1\\
        &=i\xi_1\int_\R fe^{-i\xi_1x_1}dx_1.
    \end{align*}
    これを元の式に代入すると,$\wh{f_{x_1}}(\xi)=i\xi_1\wh{f}(\xi)$.
\end{Proof}

\begin{proposition}[$x^\al$を乗じると導関数が出てくる]
    任意の$f\in\S$に対して,
    \begin{enumerate}
        \item $\wh{f}\in C^\infty(\R^d)$.
        \item $\forall_{\al\in\N^d}\;\wh{x^\al f}(\xi)=i^{\abs{\al}}(\partial^\al\wh{f})(\xi)$.
    \end{enumerate}
\end{proposition}
\begin{Proof}
    $\al=(1,0,\cdots,0)$とする.他の場合も同様.
    \begin{enumerate}
        \item $\partial^\al\wh{f}=\wh{f}_{\xi_1}$が存在することは,
        \[\forall_{\xi\in\R^d}\;\forall_{h\ne0\in\R}\quad\frac{\wh{f}(\xi+(h,0,\cdots,0))-\wh{f}(\xi)}{h}=\int_{\R^d}f(x)e^{-i\xi\cdot x}\frac{e^{-ihx_1}-1}{h}dx\]
        という変形と,$\forall_{h\in\R\setminus\{0\}}\;\Abs{\frac{e^{-ihx_1}-1}{h}}\le\abs{x_1}$の評価により,Lebesgueの優収束定理から従う.なお,この評価は,
        \[e^{-ihx_1}-1=e^{-ihx_1/2}(e^{-ihx_1/2}-e^{ihx_1/2})=-2ie^{-ihx_1/2}\sin(hx_1/2)\]
        と変形出来ることより,
        \[\sup_{h\ne0}\Abs{\frac{e^{-ihx_1}-1}{h}}\le\sup_{h\ne0}\Abs{\frac{2\sin(hx_1/2)}{h}}\le\abs{x_1}.\]
        \item 上式の右辺は$h\to0$の極限で,
        \[\lim_{h\to0}\frac{\wh{f}(\xi+(h,0,\cdots,0))-\wh{f}(\xi)}{h}=\int_{\R^d}f(x)e^{-i\xi\cdot x}(-ix_1)e^{-i\xi\cdot x}dx=-i\wh{(x_1f)}(\xi).\]
        故に,$\wh{f_{\xi_1}}(\xi)=-i\wh{(x_1f)}(\xi)$.
    \end{enumerate}
\end{Proof}

\begin{theorem}
    $f\in\S$ならば$\wh{f}\in\S$である.
\end{theorem}
\begin{Proof}
    任意の$f\in\S$を取ると,$\wh{f}\in C^\infty(\R^d)$である.
    \begin{enumerate}
        \item まず$f$の急減少性を示す.任意の$\al\in\N^d$を取ると,
        \[\abs{\xi^\al\wh{f}(\xi)}=\abs{\wh{(\partial^\al f)}(\xi)}\le\norm{\partial^\al f}_1.\]
        最後は$\F:L^1(\R^d)\to C_0(\R^d)$のノルム減少性\ref{prop-basic-property-of-Fourier-transform}を用いた.
        よって,$\xi^\al\wh{f}$は$\R^d$上有界である.
        よって,$\wh{f}$は急減少である.
        \item 続いて,$\wh{(x^\al f)}(\xi)=i^{\abs{\al}}(\partial^\al\wh{f})(\xi)$より,右辺の急減少性も従う.
    \end{enumerate}
\end{Proof}

\section{総和的性質と反転公式}

\begin{tcolorbox}[colframe=ForestGreen, colback=ForestGreen!10!white,breakable,colbacktitle=ForestGreen!40!white,coltitle=black,fonttitle=\bfseries\sffamily,
title=]
    $\R$上の議論とまったく同様に進む.
\end{tcolorbox}

\subsection{$\R^d$上のRiemann-Lebesgueの補題}

\begin{proposition}[一般次元のRiemann-Lebesgueの補題]\label{cor-Riemann-Lebesgue-Rd}
    任意の$f\in L^1(\R^d)$に対して,$\wh{f}\in UC_0(\R^d)$である.
\end{proposition}
\begin{Proof}\mbox{}
    \begin{enumerate}
        \item $\wh{f}\in UC(\R^d)$はLebesgueの収束定理から従う.
        \item 無限遠で消えることは,$f$が$C_c^\infty(\R^d)$の元で$L^1(\R^d)$-近似でき,したがって$\wh{f}$は$\S$の元で$UC_0(\R^d)$-近似出来ることによる.
    \end{enumerate}
\end{Proof}

\subsection{$\R^d$上の総和核}

\begin{definition}
    $\R^d$上の\textbf{総和核}$\{k_\lambda\}\subset L^1(\R^d)$とは,次の3条件を満たすものである:
    \begin{enumerate}
        \item 正規化:$\int_{\R^d}k_\lambda(x)dx=1$.
        \item $L^1$-有界性:$\sup_{\lambda>0}\norm{k_\lambda}_{L^1(\R^d)}<\infty$.
        \item $L^1$-集約性:任意の$\delta>0$について,$\lim_{\lambda\to\infty}\int_{\abs{x}\ge\delta}abs{k_\lambda(x)}dx=0$.
    \end{enumerate}
\end{definition}

\begin{proposition}[総和核の構成]\label{prop-construction-of-summability-kernel-on-Rd}
    $k\in L^1(\R^d)\cap C(\R^d)$は$\int_{\R^d}k(x)dx=1$を満たすならば,
    \[k_\lambda(x):=\lambda^dk(\lambda x),\qquad \lambda>0,x\in\R^d.\]
    は総和核をなす.
\end{proposition}

\begin{observation}\label{observation-summability-kernel-in-Rd}
    Fejer核(またはGauss核)を用いた$L^1(\R)$上の反転公式の議論\ref{cor-inversion-theorem-in-R1}
    が一般の$d\in\N^+$でも通用するためには,次の組$(\kappa,k)\in L^1(\wh{\R^d})\times L^1(\R^d)$を用意すれば良い:
    \begin{enumerate}[{[S}1{]}]
        \item $\kappa\in L^1(\wh{\R^d})$は有界,原点で連続,$\kappa(0)=1$.
        \item $k:=\wc{\kappa}$とFourier逆変換で定義すると,$\wh{k}(0)=\int_{\R^d}k(x)dx=1$.
    \end{enumerate}
\end{observation}


\begin{example}[$\R^d$上のFejer核]
    $\kappa(\xi):=1_{[-1,1]}(\xi)(1-\abs{\xi})$として,$\kappa^d(x):=\kappa(x_1)\cdots\kappa(x_d)$とすると,$\R^d$上のFejer核を定義できる.
\end{example}

\begin{proposition}
    議論\ref{observation-summability-kernel-in-Rd}を満たす$k$について,$k_\lambda(x):=\lambda^dk(\lambda x)\;(\lambda>0)$は総和核を定める.
\end{proposition}

\subsection{反転公式}

\begin{tcolorbox}[colframe=ForestGreen, colback=ForestGreen!10!white,breakable,colbacktitle=ForestGreen!40!white,coltitle=black,fonttitle=\bfseries\sffamily,
title=]
    $\R^d$上の総和核の理論も全く同様に成り立ち,$L^p(\R^d)\;(1\le p<\infty)$の元は$\D(\R^d)$の元で$L^p$-近似出来る.
    特に,$\S$は$L^p(\R^d)$上稠密である.
\end{tcolorbox}

\begin{theorem}[$\R^d$上の反転公式]\label{thm-inversion-theorem-in-Rd}
    $f\in L^1(\R^d)$について,$\wh{f}\in L^1(\R^d)$を満たすならば,次が成り立つ:
    \begin{enumerate}
        \item 反転公式:
        \[f(x)=\frac{1}{(2\pi)^d}\int_{\R^d}\wh{f}(\xi)e^{i\xi\cdot x}d\xi\quad \ae x\in\R^d.\]
        \item 特に,$f$は連続な修正が存在する.
        \item $\F$の2回適用は,元の関数の左右反転の$2\pi$倍になる:
        \[(\F^2f)(x)=(\F\wh{f})(x)=(2\pi)^df(-x)\quad\ae x\in\R^d.\]
        特に,$\F^4=(2\pi)^{2d}$.
    \end{enumerate}
\end{theorem}
\begin{Proof}
    まず,反転公式を$L^1(\R^d)$の意味で示す.
    $k\in L^1(\R^d)$を議論\ref{observation-summability-kernel-in-Rd}を満たす関数とする(Fejer核またはGauss核など).
    すると,$k_\lambda(x):=\lambda^dk(\lambda x)\;(\lambda>0)$は総和核を定める.
    これについて,
    \begin{align*}
        (k_\lambda *f)(x)&=\int_{\R^d}k_\lambda(x-y)f(y)dy\\
        &=\frac{\lambda^d}{(2\pi)^d}\int_{\R^d}f(y)\paren{\int_{\R}\kappa(\xi)e^{i\xi\cdot\lambda(x-y)}d\xi}dy\\
        &=\frac{1}{(2\pi)^d}\int_{\R^d}\paren{\int_{\R^d}\kappa(\eta/\lambda)e^{i\eta\cdot(x-y)}d\eta}f(y)dy\\
        &=\frac{1}{(2\pi)^d}\int_{\R^d}\wh{f}(\eta)\kappa(\eta/\lambda)e^{i\eta\cdot x}d\eta\\
        &\xrightarrow{\lambda\to\infty}\frac{1}{(2\pi)^d}\int_{\R^d}\wh{f}(\eta)e^{i\eta\cdot x}d\eta=:g(\xi).
    \end{align*}
    と計算できる.最後の収束は,$\kappa$の$0$での連続性と有界性(S1)により,Lebesgueの収束定理から従う.
    左辺の収束は総和核であることにより,$k_\lambda*f\to f$が$L^1(\R^d)$の意味で成り立つ.
    \begin{enumerate}
        \item 最後にFatouの補題から,
        \[(0\le)\int_{\R^d}\abs{f-g}dx=\int_{\R^d}\lim_{\lambda\to\infty}\abs{k_\lambda*f-g}dx\le\liminf_{\lambda\to\infty}\int_{\R^d}\abs{k_\lambda*f-g}dx=0.\]
        より,$f=g\;\ae$
        \item (1)の右辺が連続であるため.
        \item $\wh{f}$を2通りで計算する.
        まず,
        \[\wh{f}(y)=\int_{\R^d}f(x)e^{-ix\cdot y}dx=\int_{\R^d}f(-x)e^{ix\cdot y}dx.\]
        次に,反転公式の両辺をFourier変換して,
        \begin{align*}
            \wh{f}(y)&=\frac{1}{(2\pi)^d}\iint_{\R^d\times\R^d}\wh{f}(\xi)e^{i\xi\cdot x}e^{-ix\cdot y}d\xi dx\\
            &=\frac{1}{(2\pi)^d}\int_{\R^d}(\F^2f(x))e^{ix\cdot y}dx.
        \end{align*}
        あとは,Fourier変換の単射性より,$\F^2f(x)=(2\pi)^df(-x)$.
    \end{enumerate}
\end{Proof}

\subsection{Fourier変換の$\S$上の全単射性}

\begin{corollary}[$\S$上の全単射性と畳み込みに対する閉性]\mbox{}\label{cor-Fourier-transform-is-bijection-on-S}
    \begin{enumerate}
        \item $f\in L^1(\R^d)$に対して,$\wh{f}=0$ならば$f=0$である.
        \item $\F:\S\to\S$は全単射.
        \item $\forall_{f,g\in\S}\;f*g\in\S$.
    \end{enumerate}
\end{corollary}
\begin{Proof}\mbox{}
    \begin{enumerate}
        \item 反転公式より.
        \item (1)より$\F$は単射である.全射性を示す.$f\in\S$を任意に取ると,
        \[\forall_{x\in\R^d}\quad(\F^4f)(x)=(2\pi)^d(\F^2f)(-x)=(2\pi)^{2d}f(x)\]
        であるから,$\F^{-1}(f)$の元として$\F^3((2\pi)^{-2d}f)$が取れる.
        \item $f,g\in\S$を任意に取ると,$f*g\in L^1(\R^d)$で,$\wh{(f*g)}=\wh{f}\wh{g}\in\S$.(2)より,$\exists_{h\in\S}\;\wh{(f*g)}=\wh{h}$であるが,単射性より,$f*g=h\in\S$.
    \end{enumerate}
\end{Proof}

\begin{definition}
    Fourier変換$\F:\S\iso\S$の逆変換$\F^{-1}:\S\to\S$を\textbf{Fourier逆変換}という.
    $\F^{-1}\varphi=(2\pi)^{-2d}\F^3\varphi$の関係がある.
\end{definition}

\section{$L^2(\R^d)$上のFourier変換}

\begin{tcolorbox}[colframe=ForestGreen, colback=ForestGreen!10!white,breakable,colbacktitle=ForestGreen!40!white,coltitle=black,fonttitle=\bfseries\sffamily,
title=]
    $f\in L^2(\R^d)$に対するFourier変換は次の$L^2(\wh{\R^d})$-極限として定義される:
    \[\F[f]:=\lim_{L\to\infty}\frac{1}{(2\pi)^{d/2}}\int_{\abs{x}\le L}f(x)e^{-i\xi x}dx\]
    すると等長であり,内積を保つ.
\end{tcolorbox}

\subsection{Fourier-Plancherel変換の定義}

\begin{tcolorbox}[colframe=ForestGreen, colback=ForestGreen!10!white,breakable,colbacktitle=ForestGreen!40!white,coltitle=black,fonttitle=\bfseries\sffamily,
title=]
    同型$\F:L^2(\R^d)\iso L^2(\R^d)$の構成を考える.
    $\F:L^2(\bT)\to l^2(\Z)$の場合との違いは,$L^2(\R^d)\not\subset L^1(\R^d)$である点であるので,途中で$\F$の延長が必要になる.
    しかし,$C_c^\infty(\R^d)\subset\S\subset L^2(\R^d)$は稠密である.
\end{tcolorbox}

\begin{lemma}
    任意の$f\in L^1(\R^d)\cap L^2(\R^d)$に対して,列$\{f_n\}\subset C_c^\infty(\R^d)$が存在して,$L^1(\R^d)$と$L^2(\R^d)$のいずれの位相についても$f_n\to f$.
\end{lemma}
\begin{Proof}
    $f\in L^1(\R^d)\cap L^2(\R^d)$には,$L^1$の意味でも$L^2$の意味でも近似列$\{f_n\}\subset C_c^\infty(\R^d)$が存在するが,これが一般に$1\le p<\infty$に依らないことを示せる.
\end{Proof}

\begin{theorem}[Plancherel: Banの同型となる延長の存在]
    線型同型$\F:L^2(\R^d)\iso L^2(\R^d)$で次を満たすものが唯一存在する:
    \begin{enumerate}
        \item $\forall_{f\in L^1(\R^d)\cap L^2(\R^d)}\;\F f=\wh{f}$.
        \item $\forall_{f\in L^2(\R^d)}\;\norm{\F f}_2=(2\pi)^{d/2}\norm{f}_2$.
    \end{enumerate}
\end{theorem}
\begin{Proof}\mbox{}
    \begin{description}
        \item[$\S$上で条件を満たす] すでにある$\F:\S\iso\S$は,(2)を満たすことを確認する.
        $f,g\in\S$について,$h(x)=\o{\wh{g}(x)}$とおくと,反転公式\ref{thm-inversion-theorem-in-Rd}より,
        \[wh{h}(\xi)=\int_{\R^d}\o{\wh{g}(x)}e^{-i\xi\cdot x}dx=\o{\int_{\R^d}\wh{g}(x)e^{i\xi\cdot x}dx}=(2\pi)^d\o{g(\xi)}.\]
        これを踏まえると,$(x,y)\mapsto f(x)h(y)e^{-ix\cdot y}$は可積分であるから,Fubiniの定理より,
        \[\int_{\R^d}f(x)\wh{h}(x)dx=\int_{\R^d\times\R^d}f(x)h(y)e^{-ix\cdot y}dxdy=\int_{\R^d}\wh{f}(y)h(y)dy.\]
        を得る.以上を併せて,
        \[(2\pi)^d(f|g)=(f|\o{\wh{h}})=(\wh{f}|\o{h})=(\wh{f}|\wh{g}).\]
        \item[$\F$の延長] 任意の$f\in L^2(\R^d)$に対して,$\S$上の収束列$\{f_n\}\subset\S$を取り,$\F f:=\lim_{n\to\infty}\F f_n$と定める.
        この定義が$\S$上の収束列の取り方に依らない.
        これは,$f_n,g_n$がいずれも$f$に$L^2(\R^d)$-収束するならば,
        \[\norm{\wh{f_n}-\wh{g_n}}_{L^2(\R^d)}=(2\pi)^{d/2}\norm{f_n-g_n}_{L^2(\R^d)}\to0\]
        より,それぞれのFourier変換の極限は等しい.$\F$は$\S$上全単射であるから,$f_n,g_n$の極限も等しい.
        続いて,$\S$上の等式$\norm{\wh{f_n}}_2=(2\pi)^{d/2}\norm{f_n}_2$は$n\to\infty$の極限でも成り立つ.
        \item[(1)の証明] 任意の$f\in L^1(\R^d)\cap L^2(\R^d)$を取る.$\{f_n\}\subset\S$は$L^1,L^2$両方の意味で$f$に収束するとする.
        このとき,
        $\F:\S\to\S$のノルム減少性$\norm{\wh{f_n}-\wh{f}}_\infty\le\norm{f_n-f}_1$より,$\wh{f}_n\to\wh{f}$は一様収束であることに注意すれば,
        $\F f=\wh{f}$,すなわち,$\wh{f_n}$の$L^2$-極限と一様収束極限が一致することを示せばよい.
        これは,任意の球$B\subset\R^d$について,
        \[\int_B\abs{\F f(\xi)-\wh{f}(\xi)}d\xi\le\int_B\abs{\F f(\xi)-\F f_n(\xi)}d\xi+\int_B\abs{\wh{f_n}(\xi)-\wh{f}(\xi)}d\xi.\]
        であるが,第2項は$\wh{f_n}\to\wh{f}\;\In C_0(\R)$より,第1項はCauchy-Schwarzの不等式より$\norm{\F f-\F f_n}_2$の定数倍で抑えられる.
        よって,この右辺は$n\to\infty$で$0$に収束する.
        \item[$\F$の全射性] あとは$\F$が全射であることを示せば,$\F$が線型同型であることが従う.
        $\S\subset\Im\F$は$L^2(\R^d)$の閉集合であるため,全射であることがわかる.
        \item[$\F$の一意性] $L^1(\R^d)\cap L^2(\R^d)$は$L^2(\R^d)$で稠密であるため,延長は一意である.
    \end{description}
\end{Proof}

\subsection{$L^2(\R^d)$上のFourier変換はHilbert空間の同型を定める}

\begin{corollary}[Parsevalの等式]
    任意の$f,g\in L^2(\R^d)$について,ユニタリ性$(2\pi)^d(f|g)=(\wh{f}|\wh{g})$が成り立つ.
\end{corollary}
\begin{Proof}
    上の証明の第一段より,$f,g\in\S$については成り立つ.
    内積の連続性より,引き続き$L^2(\R^d)$全域で成り立つ.
\end{Proof}

\subsection{$L^2$-上の反転公式}

\begin{proposition}
    $f\in L^2(\R)$について,$S_\lambda[f]\in L^2(\R)$であり,
    \[S_\lambda[f](x)=\frac{1}{2\pi}\int^\lambda_{-\lambda}(\F f)(\xi)e^{i\xi x}d\xi\xrightarrow{\lambda\to\infty}f\;\In L^2(\R).\]
\end{proposition}

\subsection{$L^2(\R^d)$上のFourier変換と畳み込み}

\begin{proposition}
    $f,g\in L^2(\R)$に対して,$f*g$は任意の$x\in\R$について定まり,
    $f*g\in C_b(\R)$を満たす.
\end{proposition}

\begin{theorem}[\cite{黒田成俊-関数解析}定理5.17]
    任意の$f\in L^2(\R^n),g\in L^1(\R^n)$に対して,
    \[\wh{f*g}(\xi)=\wh{f}(\xi)\wh{g}(\xi).\]
\end{theorem}

\subsection{$L^2(\R^d)$上のFourier変換の例}

\begin{tcolorbox}[colframe=ForestGreen, colback=ForestGreen!10!white,breakable,colbacktitle=ForestGreen!40!white,coltitle=black,fonttitle=\bfseries\sffamily,
title=]
    Dirichlet核などは$L^1$ではないが$L^2$であるものの代表例であるから,これによりFourier変換が計算できるようになったものは多数ある.
\end{tcolorbox}

\begin{proposition}[代表例]
    $f(x)=\frac{1}{x+i}$は$f\in L^2(\R)\setminus L^1(\R)$を満たすが,このFourier変換は
    \[\wh{f}(\xi)=-\sqrt{2\pi}ie^{-\xi}1_{\Brace{\xi>0}}.\]
\end{proposition}

\section{総和核の例と偏微分方程式}

\subsection{Poisson核のFourier変換}

\begin{theorem}[$\R^d$上のPoisson核のFourier変換]\mbox{}\label{thm-Fourier-transform-of-Poisson-kernel-on-Rd}
    \begin{enumerate}
        \item \[\F[e^{-\abs{x}}](\xi)=2^d\pi^{\frac{d-1}{2}}\Gamma\paren{\frac{d+1}{2}}\frac{1}{(1+\abs{\xi}^2)^{\frac{d+1}{2}}}.\]
        \item \[\forall_{t>0}\quad\F[e^{-t\abs{x}}](\xi)=2^d\pi^{\frac{d-1}{2}}\Gamma\paren{\frac{d+1}{2}}\frac{t}{(t^2+\abs{\xi}^2)^{\frac{d+1}{2}}}.\]
    \end{enumerate}
\end{theorem}
\begin{Proof}\mbox{}
    \begin{enumerate}
        \item 仮に等式
        \[e^{-\abs{x}}=\int^\infty_0g(t)e^{-\frac{\abs{x}^2}{4t}}dt\]
        があれば,Fubiniの定理より,
        \begin{align*}
            \F\Square{e^{-\abs{x}}}(\xi)&=\int_{\R^d}\paren{\int^\infty_0g(t)e^{-\frac{\abs{x}^2}{4t}}dt}e^{-ix\cdot\xi}d\xi\\
            &=\int^\infty_0g(t)\F\Square{e^{-\frac{\abs{x}^2}{4t}}}(\xi)dt=(4\pi t)^{d/2}\int^\infty_0g(t)e^{-t\xi^2}dt.
        \end{align*}
        と計算できる.ただし,Gauss核のFourier変換
        \[\F^2[(\pi/t)^{-d/2}e^{-x^2t}]=\F[e^{-\frac{x^2}{4t}}]=(2\pi)^d(\pi/t)^{-d/2}e^{-x^2t}=(4\pi t)^{d/2}e^{-x^2t}.\]
        を用いた.補題から,$g$の表示はわかっているから,これを代入して,変数変換$s:=(1+\abs{\xi}^2)t$により,
        \begin{align*}
            \F\Square{e^{-\abs{x}}}(\xi)&=\int^\infty_0\paren{(4\pi t)^{d/2}\frac{e^{-t}}{\sqrt{\pi t}}}e^{-t\xi^2}dt=2^d\pi^{\frac{d-1}{2}}\int^\infty_0t^{\frac{d-1}{2}}e^{-(1+\abs{\xi}^2)t}dt\\
            &=2^d\pi^{\frac{d-1}{2}}\frac{1}{(1+\abs{\xi}^2)^{\frac{d+1}{2}}}\int^\infty_0s^{\frac{d-1}{2}}e^{-s}ds=2^d\pi^{\frac{d-1}{2}}\Gamma\paren{\frac{d+1}{2}}\frac{1}{(1+\abs{\xi}^2)^{\frac{d+1}{2}}}.
        \end{align*}
        \item 変数変換による.
    \end{enumerate}
\end{Proof}
\begin{remarks}
    Gamma関数と単位球の表面積の関係
    \[\frac{\Gamma(n/2)}{\pi^{n/2}}=\frac{2}{n\om_n}=\frac{2}{\sigma_n}\]
    を用いると,
    \begin{align*}
        \F[e^{-\abs{x}}](\xi)&=(2\pi)^d\frac{\Gamma\paren{\frac{d+1}{2}}}{\pi^{\frac{d+1}{2}}}\frac{1}{(1+\abs{\xi}^2)^{\frac{d+1}{2}}}\\
        &=\frac{(2\pi)^d}{\sigma_{d+1}}\frac{2}{\sqrt{1+\abs{\xi}^2}^{d+1}}=\frac{2(2\pi)^d}{\abs{\partial B(0,\sqrt{1+\abs{\xi}^2})}}.
    \end{align*}
    と,$d+1$次元単位球の境界(多様体としては$d$次元)の表面積が出てくる.
\end{remarks}

\begin{lemma}[$\varphi$のGauss核のLaplace変換としての表示 (Bochner's method of subordination \cite{Pinsky09-Wavelet} 2.2.2.3)]
    $\varphi(u)=e^{-\abs{u}}$は,Gauss核$e^{-\abs{x}^2/4t}$と
    \[g(t):=\frac{1}{\sqrt{\pi t}}e^{-t}\]
    との畳み込みとして理解できる:
    \[e^{-\abs{x}}=\frac{1}{\pi^{1/2}}\int^\infty_0e^{-t}\frac{e^{-\frac{\abs{x}^2}{4t}}}{\sqrt{t}}dt.\]
\end{lemma}
\begin{Proof}
    $\varphi(u)=e^{-\abs{u}}$について,
    $\wh{P}(u)=\varphi(u)$であるから,
    \[\varphi(x)=e^{-\abs{x}}=\int_\R P(\xi)e^{-i\xi x}d\xi=\frac{1}{\pi}\int_\R\frac{e^{-i\xi x}}{1+\xi^2}d\xi.\]
    これと,等式
    \[\frac{1}{1+\xi^2}=\int^\infty_0e^{-(1+\xi^2)t}dt=\SQuare{-\frac{e^{-(1+\xi^2)t}}{1+\xi^2}}^\infty_0\]
    に注意すると,Fubiniの定理から,
    \begin{align*}
        e^{-\abs{x}}&=\frac{1}{\pi}\int_\R\frac{e^{-i\xi x}}{1+\xi^2}d\xi=\frac{1}{\pi}\int_\R\paren{\int^\infty_0e^{-(1+\xi^2)t}dt}e^{-i\xi x}d\xi\\
        &=\frac{1}{\pi}\int^\infty_0e^{-t}\paren{\int_\R e^{-t\xi^2}e^{-i\xi x}d\xi}dt=\frac{1}{\pi}\int^\infty_0e^{-t}\sqrt{\frac{\pi}{t}}e^{-\frac{\abs{x}^2}{4t}}dt.
    \end{align*}
    ただし,最後は$\rN(0,1/2t)$の密度関数のFourier変換が
    \[\varphi(u)=e^{-\frac{u^2}{2}\frac{1}{2t}}=e^{-\frac{u^2}{4t}}.\]
    であることを用いた.
\end{Proof}
\begin{remarks}
    これにより,任意の$x\in\R^d$について,
    \[e^{-\abs{x}}=\frac{1}{\sqrt{\pi}}\int^\infty_0\frac{e^{-t}}{t^{1/2}}e^{-\frac{\abs{x}^2}{4t}}dt.\]
    を得る.
\end{remarks}

\begin{proposition}[Poisson核の半群性]
    $P_y(x):=y^{-d}P_1(x/y)$について,
    $P_y*P_w=P_{y+w}$.
\end{proposition}
\begin{Proof}
    \begin{align*}
        \wh{P_y*P_w}(\xi)&=\wh{P_y}(\xi)\wh{P_w}(\xi)=e^{-y\abs{\xi}}e^{-w\abs{\xi}}\\
        &=e^{-(x+y)\abs{\xi}}=\wh{P_{y+w}}(\xi).
    \end{align*}
\end{Proof}

\subsection{Poisson核の畳み込み}

\begin{proposition}
    $\varphi(x)=e^{-\abs{x}}$について,
    \begin{enumerate}
        \item $\varphi*\varphi(x)=(1+\abs{x})e^{-\abs{x}}$.
        \item $g(x)=\frac{1}{(1+x^2)^2}$について,
        \[\wh{g}(\xi)=\frac{\pi}{2}(1+\abs{\xi})e^{-\abs{\xi}}.\]
    \end{enumerate}
\end{proposition}
\begin{Proof}\mbox{}
    \begin{enumerate}
        \item \begin{align*}
            f*f(x)&=\int_\R e^{-\abs{x-y}}e^{-\abs{y}}dy=\int^\infty_0e^{-\abs{x-y}}e^{-y}dy+\int^0_{-\infty}e^{-\abs{x-y}}e^{y}dy.
        \end{align*}
        さらに$x\ge0$を仮定すると,
        \begin{align*}
            f*f(x)&=e^{-x}\int^x_0dy+\int^\infty_xe^{x-2y}dy+\int^0_{-\infty}e^{-x+2y}dy\\
            &=xe^{-x}+\SQuare{-\frac{e^{x-2y}}{2}}^\infty_x+\SQuare{\frac{e^{-x+2y}}{2}}^0_{-\infty}=(1+x)e^{-x}.
        \end{align*}
        同様にして,$x\le0$のとき$(1-x)e^{x}$を得る.
        \item $\wh{\varphi}(\xi)=\frac{2}{1+\xi^2}$より,
        \[\frac{1}{4}\wh{\varphi*\varphi}(\xi)=\frac{1}{4}(\wh{\varphi}(\xi))^2=\frac{1}{(1+\xi^2)^2}=g(\xi).\]
        よって,
        \[\wh{g}(\xi)=\frac{1}{4}\wh{\wh{\varphi*\varphi}}(\xi)=\frac{\pi}{2}\varphi*\varphi(\xi)=\frac{\pi}{2}(1+\abs{\xi})e^{-\abs{\xi}}.\]
    \end{enumerate}
\end{Proof}

\subsection{上半平面上のPoisson核}

\begin{theorem}[Poisson核の表示]
    $\wh{P_y}(\xi)=e^{-y\abs{\xi}}$によって$\R^d$上の関数$P_y$を定めると,次のように表示できる:
    \[P_y(x)=\frac{\Gamma\paren{\frac{d+1}{2}}}{\pi^{\frac{d+1}{2}}}\frac{y}{(y^2+\abs{x}^2)^{\frac{d+1}{2}}}.\]
    さらに,$(P_y)_{y>0}$は$y\to0$の極限において総和核をなす.
\end{theorem}
\begin{Proof}\mbox{}
    \begin{enumerate}
        \item 定義式$\wh{P_y}(\xi)=e^{-y\abs{\xi}}$の両辺をFourier変換すると,
        \[(2\pi)^dP_y(-x)=\F\Square{e^{-y\abs{\xi}}}(x)\]
        を得る.あとは定理による.
        \item (1)より,関係$P_y(x)=y^{-d}_1\paren{\frac{x}{y}}$を満たすため.
    \end{enumerate}
\end{Proof}

\begin{observation}[Laplace方程式の空間についてのFourier変換]
    上半平面$\R^{d+1}_+:=\Brace{(x_1,\cdots,x_d,y)\in\R^{d+1}\mid y>0}$におけるLaplace方程式の境界値問題
    \[\begin{cases}
        \paren{\Lap_x+\pp{^2}{y^2}}u(x,y)=0&\In\R^{d+1}_+,\\
        u=f&\on\partial\R^{d+1}_+.
    \end{cases}\]
    を考える.2つの式をいずれも,$u_y(x):=u(x,y)$と$y\in\R_+$を固定してみて,$x\in\R^d$についてFourier変換すると,部分積分を通じて
    \[\begin{cases}
        0=-\abs{\xi}^2\wh{u_y}(\xi)+\pp{^2\wh{u_y}}{y^2}(\xi)&\xi\in\R^d,\\
        \wh{u_0}(\xi)=\wh{f}(\xi)&\xi\in\R^d.
    \end{cases}\]
    これを今度は$y\in\R$についてのODEとみると,有界な解は$\wh{u_y}(\xi)=e^{-y\abs{\xi}}\wh{f}(\xi)$に限る.
    $\wh{u_y}\in C_0(\R^d)$である必要があるので,自然な制約である.
    以上より,$u_y$が一般解であるための必要条件は,は$\wh{P_y}(\xi)=e^{-y\abs{\xi}}$を満たす核について,
    $u_y=P_y*f$という条件を満たすことがわかる.
\end{observation}

\subsection{Gauss核のFourier変換}

\begin{theorem}[\cite{Strichartz03-Distribution} 3.5節]\mbox{}\label{thm-Fourier-transform-of-Gaussian-density}
    \begin{enumerate}
        \item $\R$上のGauss核について,\[\F[e^{-tx^2}](\xi)=\sqrt{\frac{\pi}{t}}e^{-\frac{\xi^2}{4t}}.\]
        \item $\R^d$上のGauss核について,
        \[\F[e^{-t\abs{x}^2}]=\paren{\frac{\pi}{t}}^{d/2}e^{-\frac{\abs{\xi}^2}{4t}}.\]
    \end{enumerate}
\end{theorem}
\begin{Proof}\mbox{}
    \begin{enumerate}[{Step}1]
        \item 平方完成により,
        \begin{align*}
            \int^\infty_{-\infty}e^{-tx^2}e^{ix\xi}dx=e^{-\frac{\xi^2}{4t}}\int^\infty_{-\infty}e^{-t\paren{x-\frac{i\xi}{2t}}^2}dx
        \end{align*}
        を得る.
        \item これは$z=x-\frac{i\xi}{2t}$の変数変換により計算可能である.
        まず$\xi>0$の場合には,
        $\C$上の積分路$\partial([-R,R]\times[0,i\xi/2t])$を考えることにより,これを計算できる.
        左端と右端の積分は,
        \[\abs{e^{-tz^2}}=\abs{e^{-t(x^2-y^2)-2itxy}}\le e^{-t(x^2-y^2)}\]
        より,$R\to\infty$の極限で$0$に収束する.
        よって,被積分関数が$\C$上の正則関数であることより,
        \[\sqrt{\frac{\pi}{t}}=\int_\R e^{-tx^2}dx=\int_\R \exp\paren{-t\paren{x-\frac{i\xi}{2t}}^2}dx.\]
        \item $\xi<0$の場合と,元の式と併せて,
        \[\F[e^{-tx^2}](\xi)=\sqrt{\frac{\pi}{t}}e^{-\frac{\xi^2}{4t}}.\]
    \end{enumerate}
\end{Proof}

\subsection{Gauss核の畳み込み}

\begin{proposition}
    \[\paren{e^{-s\abs{x}^2}*e^{-t\abs{x}^2}}(x)=\paren{\frac{\pi}{s+t}}^{d/2}e^{-\frac{st}{s+t}\abs{x}^2}.\]
\end{proposition}
\begin{Proof}\mbox{}
    \begin{enumerate}[{Step}1]
        \item まず,左辺のFourier変換は
        \begin{align*}
            \wh{e^{-s\abs{x}^2}*e^{-t\abs{x}^2}}(\xi)&=\frac{\pi^d}{(st)^{d/2}}e^{-\frac{\abs{\xi}^2}{4s}}e^{-\frac{\abs{\xi}^2}{4t}}=\frac{\pi^d}{(st)^{d/2}}e^{-\frac{s+t}{4st}\abs{\xi}^2}.
        \end{align*}
        \item 再び両辺のFourier変換を考えると,
        \begin{align*}
            (2\pi)^d\paren{e^{-s\abs{x}^2}*e^{-t\abs{x}^2}}(x)&=\frac{\pi^d}{(st)^{d/2}}\paren{\frac{4st\pi}{s+t}}^{d/2}\exp\paren{-\frac{st}{s+t}\abs{\xi}^2}.
        \end{align*}
    \end{enumerate}
\end{Proof}

\begin{proposition}[熱核の半群性]
    $H_t(x):=t^{-d}H(x/t)$について,
    $H_s*H_t=H_{s+t}$.
\end{proposition}
\begin{Proof}
    \begin{align*}
        \wh{H_s*H_y}(\xi)&=\wh{H_s}(\xi)\wh{H_t}(\xi)=e^{-s\abs{\xi}^2}e^{-t\abs{\xi}^2}\\
        &=e^{-(s+y)\abs{\xi}^2}=\wh{H_{s+t}}(\xi).
    \end{align*}
\end{Proof}

\subsection{Gauss核と回転}


\begin{proposition}[熱核の特徴付け Maxwell (1860) \cite{Pinsky09-Wavelet} Prop. 2.2.51]
    $f\in L^1(\R^d)\;(d\ge2)$について,ある$f_0,f_1,\cdots,f_d$が存在して
    \[f(x_1,\cdots,x_d)=f_1(x_1)\cdots f_d(x_d)=f_0(\sqrt{x_1^2+\cdots+x_d^2}),\qquad x\in\R^.\]
    を満たすとする.このとき,ある$A\in\R,B>0$が存在して,
    \[f(x)=Ae^{-B\abs{x}^2}.\]
\end{proposition}
\begin{remarks}
    さらに,Lebesgue測度について絶対連続なものに限らず,一般の$\R^d$上の確率測度について,上の条件を満たすものは正規分布に限ることが導ける.
    周辺分布$e^{-tx_1^2}$の積として得られ,さらに動径の二乗の関数であるような密度は,独立な多変量正規分布に限る.
    さらに,この2つの性質はFourier変換によって保たれる.Gauss核が$\F$の不動点を与えるのはこのことによる.
\end{remarks}

\begin{proposition}[動径の関数のFourier変換は動径の関数である]
    \[R_\theta:=\mtrx{\cos\theta}{-\sin\theta}{\sin\theta}{\cos\theta},\qquad\theta\in[0,2\pi]\]
    とすると,$\F[f\circ R_\theta]=(\F f)\circ R_\theta$が成り立つ.
\end{proposition}

\subsection{上半平面上の熱核}

\begin{theorem}
    $\wh{H_t}(\xi)=e^{-t\abs{\xi}^2}$によって$\R^d$上の関数$H_t$を定めると,次のように表示できる:
    \[H_t(x)=\frac{1}{(4\pi t)^{d/2}}e^{-\frac{\abs{x}^2}{4t}}.\]
    さらに,$(H_t)_{t>0}$は$t\to0$の極限において総和核をなす.
    なお,これは$\rN_d(0,2tI_d)$の密度関数でもある.
\end{theorem}

\begin{observation}
    上半平面$\R^{d+1}_+:=\Brace{(x_1,\cdots,x_d,t)\in\R^{d+1}\mid t\ge0}$における熱方程式の境界値問題
    \[\begin{cases}
        u_t=\Lap u&\In \R^{d+1}_+,\\
        u=f&\on\partial\R^{d+1}_+.
    \end{cases}\]
    を考える.$t\in\R_+$を固定して$x\in\R^d$についてFourier変換すると,$t>0$についてのODE
    \[\begin{cases}
        \pp{\wh{u_t}}{t}(\xi)=-\abs{\xi}^2\wh{u_t}(\xi),\\
        \wh{u_0}(\xi)=\wh{f}(\xi).
    \end{cases}\]
    を得る.これは1階線型ODEの初期値問題であるから,ただ一つの解$\wh{u_t}(\xi)=e^{-t\abs{\xi}^2}\wh{f}(\xi)$を持つ.
    すなわち,$u$が一般解であるための必要条件は,
    $\wh{H_t}(\xi)=e^{-t\abs{\xi}^2}$を満たす関数$H_t$について,
    $u_t=H_t*f$である.
\end{observation}

\subsection{単位円周上の熱方程式}

\begin{proposition}
    $\bT$上の熱核\ref{remarks-Gauss-kernel-on-T}
    \[W_t(x)=\sum_{n\in\Z}e^{-n^2t}e^{inx},\qquad x\in\bT,t>0.\]
    について,
    \begin{enumerate}
        \item $\wh{W_t}(n)=e^{-n^2t}$.
        \item $(W_t)_{t>0}$は総和核である.
        \item 任意の境界条件$f\in L^1(\bT)$に対して,$u(x,t):=(W_t*f)(x)$は熱方程式を満たし,$W_t*f\to f\;\In C(\bT)$.
    \end{enumerate}
\end{proposition}

\section{Poisson積分とHardy空間}

\subsection{上半平面のPoisson核}

\begin{tcolorbox}[colframe=ForestGreen, colback=ForestGreen!10!white,breakable,colbacktitle=ForestGreen!40!white,coltitle=black,fonttitle=\bfseries\sffamily,
title=]
    上半平面上の調和関数とその$\R$上の境界値との関係を考える.
    $L^1(\R)$だと制限が強いので,$L^2(\R)$-理論があることは大事なことである.
\end{tcolorbox}

\begin{definition}\mbox{}
    \begin{enumerate}
        \item $\P_y(x)=\P(x,y):=\frac{1}{\pi}\frac{y}{x^2+y^2}$を\textbf{上半平面上のPoisson核}という.
        \item これは$y\to0$に関する総和核になっている\ref{def-Poisson-kernel}:$\P_y(x)=\frac{1}{y}P\paren{\frac{x}{y}}$.
        \item $\wh{\P_y}(\xi)=\frac{1}{\sqrt{2\pi}}e^{-y\abs{\xi}}$が成り立つ.
    \end{enumerate}
\end{definition}

\begin{theorem}
    $f\in L^2(\R)$に対して,Poisson積分を$F(x,y):=(\P_y*f)(x)$と定める.
    \begin{enumerate}
        \item $F$は$f$を境界値に持つ上半平面上の調和関数である.
        \item 次は同値:
        \begin{enumerate}
            \item $F$は正則でもある.
            \item $(-\infty,0)$上殆ど至る所$\wh{f}=0\;\ae(-\infty,0)$
        \end{enumerate}
    \end{enumerate}
\end{theorem}

\subsection{上半平面上のHardy空間}

\begin{tcolorbox}[colframe=ForestGreen, colback=ForestGreen!10!white,breakable,colbacktitle=ForestGreen!40!white,coltitle=black,fonttitle=\bfseries\sffamily,
title=]
    Poisson積分$\P_y*f$は境界が$f$である調和関数になる.
    これがさらに正則関数になるためには,$\wh{f}$が$\R^-$上殆ど至る所消えることが必要十分である.
\end{tcolorbox}

\begin{definition}
    上半平面$H^+\subset\C$に対して,
    \begin{enumerate}
        \item $\O(H^2)$の部分空間\textbf{Hardy空間}を
        \[H^2:=\Brace{F\in\O(H^+)\;\middle|\;\exists_{M>0}\;\forall_{y>0}\;\int_\R\abs{F(x,y)}^2dx\le M^2}.\]
        と定める.また,調和関数の部分空間としての対応物を$\H^2(H^+)$とする.
        \item $L^2(\R)$の閉部分空間を次のように定める:
        \[L^+(\R):=\Brace{f\in L^2(\R)\mid\wh{f}=0\;\ae(-\infty,0)}.\]
    \end{enumerate}
\end{definition}

\begin{theorem}
    関数$F:H^+\to\C$について,次の3条件は同値:
    \begin{enumerate}
        \item $F\in H^2$.
        \item ある$h\in L^2(\R^+)$が存在して,
        \[F(z)=\frac{1}{\sqrt{2\pi}}\int^\infty_0h(\xi)e^{iz\xi}\xi.\]
        \item $F$はある$f\in L^+(\R)$のPoisson積分である:$F(x+iy)=(\P_y*f)(x)$.
    \end{enumerate}
    また,このとき次の等式が成り立つ:
    \[\lim_{y\searrow0}\int_\R\abs{F(x+yi)}^2dx=\int_\R\abs{f(x)}^2dx=\int^\infty_0\abs{h(t)}^2dt.\]
    特に,$f,h$は$F$に対して一意的である.
\end{theorem}
\begin{remarks}[Paley-Wienerへの布石]
    (2)において,$h$がコンパクト台を持つように取れたとする.
    するとこのとき$F$は$\C$上への延長を持ち,さらに次の評価が成り立つ:
    \[\abs{F(z)}\le Ae^{\sigma\abs{z}},\qquad A:=(2\pi)^{-1/2}\sigma^{1/2}\norm{h}_{L^2(\R^+)}.\]
    こうして,$h$の減衰と$F$の減衰が対応している!?
\end{remarks}

\begin{theorem}
    関数$F:H^+\to\C$について,次の2条件は同値:
    \begin{enumerate}
        \item $F\in\H^2(H^+;\R)$.
        \item ある$f\in L^2(\R)$のPoisson積分である.
    \end{enumerate}
    このとき,次の等式が成り立つ:
    \[\lim_{y\searrow0}\int_\R\abs{F(x,y)}^2dx=\int_\R\abs{f(x)}^2dx.\]
\end{theorem}

\section{Fourier-Laplace変換}

\subsection{Paley-Wienerの定理}

\begin{definition}
    整関数$F$が\textbf{指数型$\sigma$}であるとは,ある$\sigma>0$について,
    任意の$\ep>0$に対して$A_\ep>0$が存在して$\forall_{z\in\C}\;\abs{F(z)}\le A_\ep e^{(\sigma+\ep)\abs{z}}$を満たすことをいう.
\end{definition}

\begin{theorem}
    関数$F:\C\to\C$に対して,
    次は同値:
    \begin{enumerate}
        \item 指数型$\sigma$であり,かつ,$F|_{\R}\in L^2(\R)$.
        \item ある$h\in L^2(-\sigma,\sigma)$が存在して,
        \[F(z)=\frac{1}{\sqrt{2\pi}}\int^\sigma_{-\sigma}h(\xi)e^{iz\xi}d\xi,\qquad(z\in\C).\]
    \end{enumerate}
\end{theorem}

\begin{definition}
    定理の状況において,$F$を$h$の\textbf{Fourier-Laplace変換}ともいう\cite{Rudin-FunctionalAnalysis}Th'm 7.23.
    これはFourier変換像の$\C$上への延長になっている.
\end{definition}

\chapter{$\R^d$上のSchwartz超関数}



\section{超関数の定義と例}

\begin{tcolorbox}[colframe=ForestGreen, colback=ForestGreen!10!white,breakable,colbacktitle=ForestGreen!40!white,coltitle=black,fonttitle=\bfseries\sffamily,
title=]
    「ペアリングによって捕まえられる対象」として,関数を超える概念を定義する.
    本質的には,$\D:=C_c^\infty(\R^d)$上のある自然な局所凸かつ完備な位相$\tau$について連続な線型汎関数$T:\D\to\C$を\textbf{超関数}という.
    よって特に,$\R^d$上のRadon積分(=Radon測度)は超関数である.
\end{tcolorbox}

\subsection{関数のペアリングによる動機}

\begin{tcolorbox}[colframe=ForestGreen, colback=ForestGreen!10!white,breakable,colbacktitle=ForestGreen!40!white,coltitle=black,fonttitle=\bfseries\sffamily,
title=]
    関数という対象は,種々のペアリングから復元される.圏論的精神である.
    このときの相手としてどんな関数空間を取るべきだろうか?
\end{tcolorbox}

\begin{example}[Gauss核との畳み込みを通じたペアリング]
    $G\in C^\infty(\R^d)$をGauss核とすると,$G_\lambda(x):=\lambda^dG(\lambda x)$は$\R^d$上の総和核を定める\ref{prop-construction-of-summability-kernel-on-Rd}.
    すなわち,$L^p(\R^d)$とのペアリングを
    \[(G_\lambda(x-\cdot)|f):=(f*G_\lambda)(x)=\int_{\R^d}G_\lambda(x-y)f(y)dy.\]
    とすると,$\lambda\to\infty$で$f$に$L^p(\R^d)$-収束する.
    これはペアリングの情報から復元に成功しているとも見做せる.
\end{example}

\begin{example}[正規直交基底が定めるペアリング]
    $f,g\in L^2(\R^d)$間には標準的なペアリング
    \[(f|g)=\int_{\R^d}f(x)\o{g(x)}dx\]
    が存在し,任意の$f\in L^2(\R^d)$は$((f|\varphi_n))_{n\in\N}\in l^2(\N)$から復元される.
    これが$L^2(\R^d)$のFourier理論で,実際,正規直交基底は同型$L^2(\R^d)\simeq_\Hilb l^2(\N)$を定めるのであった.
\end{example}

\subsection{Schwartz超関数の定義}

\begin{tcolorbox}[colframe=ForestGreen, colback=ForestGreen!10!white,breakable,colbacktitle=ForestGreen!40!white,coltitle=black,fonttitle=\bfseries\sffamily,
title=]
    $K\compsub\R^d$について,半ノルム
    \[\norm{\varphi}_{N,\D(K)}:=\sum_{\abs{\al}\le N}\sup_{x\in K}\abs{\partial^\al\varphi(x)},\qquad\varphi\in\D(K).\]
    の族を考える.
    $T\in\D'(U)$であることは,
    任意の$K\compsub U$について,ある$N\in\N$が存在して,半ノルム$\norm{-}_{N,\D(K)}$について有界であることをいう.
\end{tcolorbox}

\begin{definition}[distribution / generalized function]\mbox{}\label{def-topology-in-D}
    \begin{enumerate}
        \item $\{\varphi_n\}\subset\D$が$\varphi\in\D$に\textbf{$\D$-収束}するとは,次の2条件を満たすことをいう:
        \begin{enumerate}
            \item $\bigcup_{n\in\N}\supp\varphi_n\cup\supp\varphi\subset\R^d$はコンパクトである.
            \item 任意の多重指数$\al\in\N^d$に対して,$\R^d$上の一様収束$\partial^\al\varphi_n\to\partial^\al\varphi\in C_c^\infty(\R^d)$が成り立つ.
        \end{enumerate}
        \item 線型汎関数$T:\D\to\C$が\textbf{超関数}であるとは,$\D$の(1)の収束に関して連続であることをいう.
        \item $T(\varphi)$は$(T|\varphi)$とも書く.
    \end{enumerate}
\end{definition}

\begin{remarks}[試験関数の空間に応じて,それぞれの分布論が存在する]
    同様の発想で,$\D=C_c^\infty(\R^d)$よりも小さい空間を用意すれば,その双対としてより広い超関数の空間を得る.
    そのような発想の例として,C. RoumieuやA. Beurlingのultradistributionがある.
\end{remarks}

\begin{remark}[一般化への途]
    $\D(\Om)\;(\Om\osub\R^n)$も同様に定義できるが,一般性のために議論が煩雑になる.
    前節は$\Om=\R^n$に議論を限定することで見通しを良くしている.
    $\D(\Om)$も,同様に定義できる位相について,
    Frechet空間の狭義帰納極限として得られるFrechet空間となる\cite{小松超関数}.
\end{remark}

\subsection{超関数の連続性の特徴付け}

\begin{proposition}\label{prop-characterization-of-distribution}
    線型写像$T:\D\to\C$に対して,次は同値:
    \begin{enumerate}
        \item $T$は連続である.
        \item 任意のコンパクト集合$K\compsub\R^d$に対して,ある$C>0,N\in\N$が存在して,$\supp\varphi\subset K$を満たす任意の$\varphi\in\D$に対して,
        \[\abs{(T|\varphi)}\le C\sum_{\abs{\al}\le N}\norm{\partial^\al\varphi}_\infty.\]
    \end{enumerate}
    この$N\in\N$がコンパクト集合$K\compsub\R^d$に依らずに取れるとき($C$も$K$に依らない必要はない\cite{Rudin-FunctionalAnalysis}Def.6.8),$N$を超関数$T$の\textbf{階数}という.
\end{proposition}
\begin{Proof}\mbox{}
    \begin{description}
        \item[(2)$\Rightarrow$(1)] 任意の$\D$の収束列$\{\varphi_n\}\subset\D$に対して,$\cup_{n\in\N}\supp\varphi_n\cup\supp\varphi\subset K$を満たすコンパクト集合$K\compsub\R^d$に対して
        \[\abs{(T|\varphi)-(T|\varphi_n)}\le C\sum_{\abs{\al}\le N}\norm{\partial^\al(\varphi-\varphi_n)}_{\infty}\]
        が成り立つ.$\varphi_n\to\varphi$は一様収束であるから,右辺は$0$に収束する.
        \item[(1)$\Rightarrow$(2)] ある$K\compsub\R^d$について,条件を満たす$C>0,N\in\N$が存在しないと仮定して矛盾を示す.
        \begin{enumerate}[{Step}1]
            \item 特に,任意の$n\in\N$に対して,ある$\varphi_n\in\D$が存在して,$\supp\varphi_n\subset K$かつ
            \[\abs{(T|\varphi_n)}>n\sum_{\abs{\al}\le n}\norm{\partial^\al\varphi_n}_\infty.\]
            \item $\psi_n:=\frac{\varphi_n}{n\sum_{\abs{\al}\le n}\norm{\partial^\al\varphi_n}_\infty}$と定めると,やはり$\supp\psi_n\subset K$であるが,
            \[\Abs{\paren{T\middle|\frac{\varphi_n}{n\sum_{\abs{\al}\le n}\norm{\partial^\al\varphi_n}_\infty}}}=\frac{1}{n\sum_{\abs{\al}\le n}\norm{\partial^\al\varphi_n}_\infty}\abs{(T|\varphi_n)}>\frac{1}{n\sum_{\abs{\al}\le n}\norm{\partial^\al\varphi_n}_\infty}n\sum_{\abs{\al}\le n}\norm{\partial^\al\varphi_n}_\infty=1\]
            より$\abs{(T|\psi_n)}>1$で,特に$0$には収束しない.
            \item しかし,
            任意の多重指数$\al\in\N^d$に対して,$\abs{\al}\le n$ならば$\norm{\partial^\al\varphi_n}_\infty\le1/n$を満たすので,$\partial^\al\psi_n$は$\R^d$上一様に$0$に収束する.
            よって,$\psi_n\to0\;\In\D$.
        \end{enumerate}
        以上より,$T$が連続であることに矛盾.
    \end{description}
\end{Proof}

\subsection{超関数の例}

\begin{tcolorbox}[colframe=ForestGreen, colback=ForestGreen!10!white,breakable,colbacktitle=ForestGreen!40!white,coltitle=black,fonttitle=\bfseries\sffamily,
title=]
    超関数には$T:L^1_\loc(\R^d)\mono\D'$が埋め込めて,
    それ以外の元には測度$T:M(\R^d)\mono\D'$やHeavisideの単位関数がある.
\end{tcolorbox}

\begin{example}[局所可積分関数は超関数である]
    任意の$f\in L^1_\loc(\R^d)$は$\D$上に線型汎関数
    \[T_f(\varphi):=\int_{\R^d}f(x)\varphi(x)dx,\qquad\varphi\in\D.\]
    を定める.これは超関数である:$T:L^1_\loc(\R^d)\to\D'(\R^d)$.
    実際,$\D$上の収束列$(\varphi_n)$を取ると,コンパクト集合$\supp\varphi\cup\bigcup_{n\in\N}\supp\varphi_n\subset K$が取れて,
    $f$は$K$上可積分で,$\varphi_n$は$K$上一様収束するために,
    Lebesgueの収束定理から
    \[T(\varphi_n)=\int_Kf(x)\varphi_n(x)dx\xrightarrow{n\to\infty}\int_Kf(x)\varphi(x)dx=T(\varphi).\]
    この$T_f(\varphi)$を$(T_f|\varphi),f(\varphi)$とも書く.
\end{example}

\begin{proposition}[変分法の基本補題]\label{prop-fundamental-lemma-of-variational-calculus}
    局所可積分関数の超関数への埋め込み
    $T:L^1_\loc(\R^d)\to\D'(\R^d)$は単射である.
\end{proposition}
\begin{Proof}
    $T_f=0$ならば$f=0$を示せば良い.
    \begin{enumerate}[{Step}1]
        \item 任意の有界開集合$U\osub\R^d$に対して$\int_Uf(x)dx=0$である.
        
        実際,まず開球$U\subset B\osub\R^d$を取ると,$\{\varphi_n\}\subset\D$であって$1_U$に下から収束する列が取れる.するとLebesgueの収束定理より,
        \[0=\int_Bf(x)\varphi_n(x)dx\xrightarrow{n\to\infty}\int_Bf(x)1_U(x)dx=\int_Uf(x)dx=0.\]
        \item 任意の測度正の集合$A$において,$f=0\;\as\;\on A$である.
        
        ある測度正の集合$A$が存在して$f>0\;\on A$と仮定し,矛盾を導く.
        \begin{enumerate}
            \item Lebesgue測度の正則性より,ある測度正のコンパクト集合$K\compsub A$が存在して,$\int_Kfdx>0$が必要.
            \item 一方で,開球$A\subset B\osub\R^d$について,$B\setminus K$は開集合であるから,Step1の結論より,
            \[\int_Kfdx=\int_Bfdx-\int_{B\setminus K}fdx=0-0=0.\]
        \end{enumerate}
    \end{enumerate}
\end{Proof}

\begin{example}[測度は超関数である]\mbox{}
    \begin{enumerate}
        \item $0$での評価$(\delta|\varphi):=\varphi(0)$で定まる写像$\delta:\D\to\C$は超関数である.
        線形性は明らかで,連続性も,$\D$のコンパクト開位相についてすでに連続であるため,$\D$の位相についても連続である.
        また$\delta$は局所可積分でない.$\delta$が関数とするならば,
        変分法の基本補題と同様の議論より,$\delta=0\;\as\R^d\setminus\{0\}$が従うから,$T_\delta=0\ne\delta$が必要である.
        \item 一般に,任意の複素Borel測度$\mu\in M(\R^d)$に対して,これが定める積分
        \[(T_\mu|\varphi):=\int_{\R^d}\varphi d\mu\]
        は超関数である.$T_\mu$の連続性はやはりLebesgueの収束定理から従う.
        が,$T_\mu$が関数に拠るかどうかは,$\mu$のLebesgue測度に対する絶対連続性に依存する.
    \end{enumerate}
\end{example}

\subsection{超関数と偏微分方程式}

\begin{tcolorbox}[colframe=ForestGreen, colback=ForestGreen!10!white,breakable,colbacktitle=ForestGreen!40!white,coltitle=black,fonttitle=\bfseries\sffamily,
title=]
    \begin{enumerate}
        \item 超関数は波動方程式の自然な解空間を与える
        \item Poisson方程式の基本解の意味が明確になり,同様の手続きで,偏微分作用素に対する基本解が組織的に得られる(Malgrange-Ehrenpreisの定理).
    \end{enumerate}
\end{tcolorbox}

\begin{example}[超関数は波動方程式の自然な解空間を与える]
    $\R$上の波動方程式
    \[u_{tt}=k^2u_{xx}\qquad(x,t)\in\R\times\R,k>0\]
    に対して,任意の$f\in C^2(\R)$が定める$u(x,t):=f(x-kt)$は古典解である.
    これは$\R$上の滑らかな波形$f$が速さ$k$で移動していく解である.
    これなら,もっと尖った$f$に対しても,$u(x,t):=f(x-kt)$は解であっても良いはずである.
    実際,非斉次項が滑らかでない場合は,このような解しか見つからないことがある.
    超関数によってこれを解と認めることが出来る.事実,適切に微分を定義することにより,
    \[\pp{^2}{t^2}T_u=k^2\pp{^2}{x^2}T_u\quad\on\D'\]
    は成り立っている.
\end{example}

\begin{example}[Poisson方程式の基本解の意味が明確になる]
    ある超関数$T:\D\to\C$が$\D'$上で$\Lap T=\delta$を満たすとする.
    すると,任意の$f\in\D$に対して$u:=T*f$は実は
    \[\Lap u=\Lap(T*f)=(\Lap T)*f=\delta *f=f\]
    を満たす.
    \begin{enumerate}
        \item まず$T*f$は必ず$C^\infty$-級になる.
        \item $\Delta(T*f)=(\Delta T)*f$が成り立つ.
    \end{enumerate}
    そして肝心の基本解についても次が成り立つ:
\end{example}

\begin{theorem}[Malgrange-Ehrenpreis (1954-56)]
    零でない多項式$p\in\C[X_1,\cdots,X_d]$に対して,偏微分作用素
    \[P:=p\paren{\partial_{x_1},\cdots,\partial_{x_d}}\]
    には基本解が存在する:$\exists_{E\in\D'(\R^d)}\;PE=\delta$.
\end{theorem}

\section{超関数の微分}

\begin{tcolorbox}[colframe=ForestGreen, colback=ForestGreen!10!white,breakable,colbacktitle=ForestGreen!40!white,coltitle=black,fonttitle=\bfseries\sffamily,
title=]
    超関数は常に任意階微分可能である.
    通常の意味では有限階しか微分可能でない関数も,超関数微分を通じて導関数が見つかることがある.
\end{tcolorbox}

\subsection{微分の定義}

\begin{tcolorbox}[colframe=ForestGreen, colback=ForestGreen!10!white,breakable,colbacktitle=ForestGreen!40!white,coltitle=black,fonttitle=\bfseries\sffamily,
title=微分作用素の随伴を通じた延長]
    内積によるペアリング$(-|-):C^\infty(\R^d)\times\D(\R^d)\to\C$では,偏微分作用素$\partial_{x_i}$を1つ移動させると符号が変わる.
    すなわち,微分作用素$\partial_{x_i}$の随伴は$-\partial_{x_i}$である.これを用いる.
\end{tcolorbox}

\begin{observation}[微分作用素の移動]\label{observation-derivative-through-pairing}
    任意の$f\in C^\infty(\R^d)$と$\al\in\N^d$に対して,
    \[\forall_{\varphi\in\D}\quad(\partial^\al f|\varphi)=(-1)^{\abs{\al}}(f|\partial^\al\varphi).\]
\end{observation}
\begin{Proof}
    部分積分を一度行うことで,
    \[(\partial_{x_i} f|\varphi)=\int_{\R^d}(\partial_{x_i}f)(x)\varphi(x)dx=0+(-1)\int_{\R^d}f(x)(\partial_{x_i}\varphi)(x)dx=(-1)(f|\partial_{x_i}\varphi).\]
    が成り立つ.
\end{Proof}

\begin{definition}
    超関数$T\in\D'$の微分$\partial^\al T\in\D'$とは,
    \[(\partial^\al T|\varphi):=(-1)^{\abs{\al}}(T|\partial^\al\varphi)\quad(\varphi\in\D)\]
    で定まる元$\partial^\al T:\D\to\C$をいう.
\end{definition}
\begin{Proof}
    $\partial^\al T\in\D'$を示す.
    線型性は明らかであるから,連続性を示す.
    任意に収束列$\{\varphi_n\}\subset\D$を取る.
    収束の定義より$\partial^\al\varphi_n\to\partial^\al\varphi\;\In\D$でもあるから,
    \[(\partial^\al T|\varphi_n)=(-1)^{\abs{\al}}(T|\partial^\al\varphi_n)\xrightarrow{n\to\infty}(-1)^{\al}(T|\partial^\al\varphi)=(\partial^\al T|\varphi).\]
\end{Proof}

\subsection{超関数微分の例}

\begin{example}[$\D'\setminus L^1_\loc$の微分]\mbox{}
    \begin{enumerate}
        \item \textbf{Heaviside関数}$H:=1_{\Brace{x\ge0}}\in L^1_\loc(\R)$の微分は$T_H'=\delta$である.すなわち,微分作用素$\dd{}{x}$の基本解である.
        
        実際,
        \[(T_H'|\varphi)=-(T_H|\varphi')=-\int_\R H(x)\varphi'(x)dx=-\int_{\R_+}\varphi'(x)dx=\varphi(0).\]
        特に,$H*f$は微分方程式
        $u'=f$
        の解を与える.
        \item \textbf{Delta分布}の微分は,「高階のDelta分布」というべきものになる.任意の$\al\in\N^d$について,
        \[(\partial^\al\delta|\varphi)=(-1)^{\abs{\al}}(\delta|\partial^\al\varphi)=(-1)^{\abs{\al}}(\partial^\al\varphi)(0).\]
        であるから,微分は$\al$階導関数の$0$での微分係数(の適切な符号変化)を与える評価写像$\varphi\mapsto(-1)^{\abs{\al}}(\partial^\al\varphi)(0)$となる.
    \end{enumerate}
\end{example}

\begin{proposition}[整合性(設計意図の成功)]\mbox{}\label{prop-compatibility-of-distributional-derivative}
    \begin{enumerate}
        \item $f\in C^k(\R^d)$と$\abs{\al}\le k$について,$\partial^\al T_f=T_{\partial^\al f}$.
        \item $f\in L^1_\loc(\R^d)$は,ある$g\in L^1_\loc(\R^d)$の原始関数であるとする:$f(x)=\int^x_0g(t)dt+f(0)$.このとき,$(T_f)'=T_g$.
    \end{enumerate}
\end{proposition}
\begin{Proof}\mbox{}
    \begin{enumerate}
        \item 観察\ref{observation-derivative-through-pairing}で確認した通り,超関数微分の設計意図である.
        \item 任意の$\varphi\in\D$に対して,
        \[((T_f)'|\varphi)=-(T_f|\varphi')=-\int_\R f(x)\varphi'(x)dx=-\paren{\SQuare{f(x)\varphi(x)}^\infty_{-\infty}-\int_\R g(x)\varphi(x)dx}=(g|\varphi)=T_g(\varphi).\]
    \end{enumerate}
\end{Proof}

\begin{example}[対数関数の超関数微分]\label{exp-principal-value-as-distribution}
    $f(x):=\log\abs{x}\in L^1_\loc(\R)$は原点に特異性を持ち,
    自身は局所可積分であるが,微分は局所可積分とはならず,したがって命題(2)に該当しない.
    \begin{enumerate}
        \item 従来の微分では,$\R\setminus\{0\}$上での導関数$\frac{1}{x}$を持つが,$\R$上での導関数は存在しない,とするところだろう.
        極限$\lim_{\ep\searrow0}\frac{1}{x}1_{\Brace{\abs{x}\ge\ep}}$も何かの意味で存在するわけではない.
        \item しかし,命題(2)に該当しないだけで超関数微分は可能であり,
        \[(T_f'|\varphi):=-\int_\R\varphi'(x)\log\abs{x}dx=\lim_{\ep\searrow0}\int_{\abs{x}\ge\ep}\frac{\varphi(x)}{x}dx.\]
        と捉えることが出来る.このCauchyの主値を通じた対応$T'_f:\D\to\C$は超関数であり,
        これを$\PV\paren{\frac{1}{x}}$と表し,$1/x$の\textbf{主値}という:
        \[(\log\abs{x})'=\PV\paren{\frac{1}{x}}=\lim_{r\searrow0}\int_{\abs{x}\ge r}\frac{\varphi(x)}{x}dx,\qquad\varphi\in\D.\]
        \item 次の超関数方程式が成り立つ:
        \[x\PV\paren{\frac{1}{x}}=1\quad\in\D'(\R).\]
        \item このような発想は,対数関数$f(x)=\log\abs{x}$と$F(x):=1/x$に限らず,主値積分が定義可能な任意の$F$について考えられる.
    \end{enumerate}
\end{example}
\begin{Proof}
    任意の$\varphi\in\D(\R)$に対して,
    \[\paren{x\PV\paren{\frac{1}{x}}\middle|\varphi}=\lim_{\ep\to0}\int_\R\frac{x\varphi(x)}{x}dx=\int_\R\varphi(x)dx=(1|\varphi).\]
\end{Proof}

\begin{example}[主値超関数の微分を続ける]
    \[\paren{\PV\paren{\frac{1}{x}}\middle|\varphi}=\PV\int_\R\varphi(x)\frac{dx}{x}.\]
    は$\log\abs{x}$の超関数微分である.この超関数微分は,
    \[\paren{D\PV\paren{\frac{1}{x}}\middle|\varphi}=-\PV\int_\R\frac{\varphi(x)-\varphi(0)}{x^2}dx.\]
\end{example}
\begin{Proof}
    部分積分による.
\end{Proof}

\begin{example}
    $f:\R\to\R$を右連続かつ有限変動な関数とすると,$\mu{((a,b])}=f(b)-f(a)$を満たすRadon測度$\mu\in M(\R)$が存在する.
    このとき,
    \begin{enumerate}
        \item $Df\in L^1(\R)$である.
        \item $DT_f=T_\mu$.
        \item $DT_f=T_{Df}$であることと$f$が絶対連続であることは同値.
    \end{enumerate}
\end{example}
\begin{Proof}
    一般に,有限変動関数$f$について,ある有限変動関数$g$であって$g'=0\;\ae$を満たすものが存在して,
    \[f(x)=g(x)+\int^x_{-\infty}f'(y)dy.\]
    このとき,$g=0$であることが$f$が絶対連続であることに同値.
    このとき,$\mu=f'dx$である.
\end{Proof}

\subsection{定数関数の特徴付け}

\begin{proposition}\label{prop-characterization-of-constants-through-distributional-derivative}
    $T\in\D'(\R)$が$T'=0$を満たすならば,$T\in L_\loc^1(\R)$かつ定数関数である.
\end{proposition}
\begin{Proof}
    $T\in\D'(\R)$は$T'=0$を満たすとする.
    \begin{enumerate}[{Step}1]
        \item $\varphi_0\in\D$は$\int_\R\varphi_0(x)dx=1$を満たすとすると,任意の$\varphi\in\D$は,$\varphi=\psi'+\al\varphi_0\;(\psi\in\D,\al\in\C)$と表せる.
        
        実際,$\al:=(1|\varphi)=\int_\R\varphi(x)dx,\;\psi(x):=\int^x_{-\infty}(\varphi(t)-\al\varphi_0(t))dt$とすると,たしかに$\varphi\in\D$である.
        \item $c:=(T|\varphi_0)$とおくと,$T=c$である.
        
        実際,
        \[(T|\varphi)=(T|\psi'+\al\varphi_0)=-(T'|\psi)+\al(T|\varphi_0)=-0+\al c=c(1|\varphi).\]
    \end{enumerate}
\end{Proof}

\subsection{超関数微分の特徴付け}

\begin{proposition}[商の極限としての理解]
    任意の超関数$u\in\D'(\R)$について,
    \[\frac{u-\tau_xu}{x}\xrightarrow{x\to0}Du\quad\In\D'(\R).\]
\end{proposition}

\subsection{Laplace作用素の基本解}

\begin{proposition}[2次元でのLaplacianの基本解]
    \[E(x):=\frac{1}{2\pi}\log\abs{x}\in L^1_\loc(\R^2)\]はLaplacian $\Lap$の基本解である:$\Lap T_E=\delta$.
\end{proposition}
\begin{Proof}
    任意の$\varphi\in\D$を取る.$(\Lap T_E|\varphi)=\varphi(0)$,すなわち,
    \[\int_{\R^2}E\cdot \Lap\psi dxdy=\varphi(0,0)\]
    を示せば良い.
    \begin{enumerate}[{Step}1]
        \item $\Om_\ep:=\Brace{(x,y)\in\R^2\mid x^2+y^2\ge\ep^2}$上での積分を考える.
        微分形式の関係$d(\varphi_xEdy)=(\varphi_{xx}E+\varphi_xE_x)dx\wedge dy$と
        Stokesの定理より,
        \[\int_\Om E\Lap\varphi dxdy=\int_{\partial\Om}(\varphi_xE-\varphi E_x)dy-\int_{\partial\Om}(\varphi_yE-\varphi E_y)dx+\int_\Om\varphi\Lap Edxdy.\]
        第一項と第二項は,極座標を通じた積分計算より$0,\varphi(0,0)$に収束する.第三項は常に$0$である.
    \end{enumerate}
\end{Proof}

\begin{proposition}[状況の抽象化]
    局所可積分関数$E\in L^1_\loc(\R^2)$の関数としての微分$E_x$は$\R^2\setminus\{(0,0)\}$上存在し,
    やはり$\R^2$上局所可積分とする.
    このとき,超関数としての偏導関数も,$E_x$と一致する:$\pp{}{x}T_E=E_x$.
\end{proposition}
\begin{Proof}
    任意の$\varphi\in\D$について,$(E|\varphi_x)=-(E_x|\varphi)$を示せば良い.
    実際,部分積分より,
    \[\int_\R\varphi_xEdx=\SQuare{\varphi(x,y)E(x,y)}^\infty_{-\infty}-\int_\R\varphi E_xdx\]
    が任意の$y\in\R\setminus\{0\}$で成り立つから,総じて
    \[\int_{\R^2}\varphi_xEdxdy=\int_\R\paren{\int_\R\varphi_xEdx}dy=\int_\R\paren{\SQuare{\varphi(x,y)E(x,y)}^\infty_{-\infty}-\int_\R\varphi E_xdx}dy=-\int_{\R^2}\varphi E_xdxdy.\]
    と結論付けることが出来る.
\end{Proof}
\begin{remarks}
    一般の$\R^d\;(d\ge3)$で成り立つが,$d=1$の場合はHeaviside関数が反例になる.
    $d=1$のときは,$E$の原点での連続性を課す必要がある.
\end{remarks}

\subsection{Cauchy-Riemann作用素の基本解}

\begin{definition}[Cauchy-Riemann operator]
    \[\partial:=\pp{}{x}-i\pp{}{y},\quad\o{\partial}:=\pp{}{x}+i\pp{}{y}\]
    について,後者をCauchy-Riemann作用素という.
\end{definition}

\begin{proposition}[Cauchy-Riemann作用素の基本解]
    $f(z):=\frac{1}{2\pi z}\in L^1_\loc(\R^2)$は$\o{\partial}$の基本解である.
\end{proposition}
\begin{Proof}
    $\Lap=\o{\partial}\partial$が成り立つから,$\Lap$の基本解$E$について,$\partial E$を求めればこれが$\o{\partial}$の基本解になる.
    $E(z)=\frac{1}{2\pi}\log\abs{z}$の関数としての微分は,原点以外で
    \[E_x=\frac{1}{2\pi}\frac{x}{x^2+y^2},\quad E_y=\frac{1}{2\pi}\frac{y}{x^2+y^2},\qquad(x,y)\in\R^2\setminus\{(0,0)\}.\]
    いずれも$\R^2$上で局所可積分であるから,命題\ref{prop-compatibility-of-distributional-derivative}より,これらは超関数微分でもある.
    よって,
    \[\partial E=\frac{1}{2\pi}\frac{x-yi}{x^2+y^2}=\frac{1}{2\pi}\frac{1}{z}.\]
    実際,任意の$\varphi\in\D$に対して,
    \[(\delta|\varphi)=(\Lap E|\varphi)=-(\partial E|\o{\partial}\varphi)=-\paren{\frac{1}{2\pi z}\middle|\o{\partial}\varphi}.\]
\end{Proof}

\section{畳み込み1:超関数と試験関数}

\begin{tcolorbox}[colframe=ForestGreen, colback=ForestGreen!10!white,breakable,colbacktitle=ForestGreen!40!white,coltitle=black,fonttitle=\bfseries\sffamily,
title=]
    $\psi\in\D$と$f\in L^1_\loc(\R^d)$に対して,
    \[(f*\psi)(x)=\int_{\R^d}f(y)\psi(x-y)dy=\int_{\R^d}f(y)(\tau_x\wt{\psi})(y)dy=(f|\tau_x\wt{\psi}).\]
    が成り立っている.この性質を用いて定義を拡張する.
\end{tcolorbox}

\begin{notation}
    $\varphi\in\D$について,
    \begin{enumerate}
        \item 反転:$\wt{\varphi}(x):=\varphi(-x)$.
        \item シフト:$(\tau_x\varphi)(y):=\varphi(y-x)$.
    \end{enumerate}
\end{notation}

\subsection{畳み込みの定義と性質}

\begin{definition}[超関数と試験関数との畳み込み]
    $\psi\in\D$を試験関数,$T\in\D'$を超関数とする.関数$T*\psi\in C^\infty(\R^d)$を
    \[(T*\psi)(x):=(T|\tau_x\wt{\psi})\qquad(x\in\R^d).\]
    で定める.
\end{definition}

\begin{proposition}[シフトと微分と可換である]\label{prop-property-of-convolution-with-functions}
    任意の$\psi\in\D,T\in\D'$について,
    \begin{enumerate}
        \item シフトとの可換性:任意の$t\in\R^d$に対して,$\tau_t(T*\psi)=T*(\tau_t\psi)$.
        \item 微分との結合性:任意の$\al\in\N^d$に対して,
        \[\partial^\al(T*\psi)=(\partial^\al T)*\psi=T*(\partial^\al\psi).\]
        \item 可微分性:$T*\psi\in C^\infty(\R^d)$である.
    \end{enumerate}
\end{proposition}
\begin{Proof}\mbox{}
    \begin{enumerate}
        \item 右辺と左辺は,任意の$x\in\R^d$に対して,
        \begin{align*}
            \LHS&=\tau_t(T*\psi)(x)=T*\psi(x-t)=(T|\tau_{x-t}\wt{\psi})\\
            \RHS&=T*(\tau_t\psi)(x)=(T|\tau_x(\wt{\tau_t\psi}))
        \end{align*}
        と計算出来るから,あとは$\tau_{x-t}\wt{\psi}=\tau_x\wt{\tau_t\psi}$を示せば良い.これは,
        \begin{align*}
            \LHS&=(\tau_{x-t}\wt{\psi})(y)=\wt{\psi}(y-x+t)=\psi(x-y-t),\\
            \RHS&=(\tau_x\wt{\tau_t\psi})(y)=\wt{\tau_t\psi}(y-x)=\tau_t\psi(x-y)=\psi(x-y-t).
        \end{align*}
        \item \begin{enumerate}[{Step}1]
            \item $(\partial^\al T)*\psi(x)=(\partial^\al T|\tau_x\wt{\psi})=(-1)^{\abs{\al}}(T|\partial^\al(\tau_x\wt{\psi}))$であるが,
            \[\partial^\al(\tau_x\wt{\psi})(y)=\partial^\al\Paren{y\mapsto\psi(x-y)}(y)=(-1)^{\abs{\al}}(\partial^\al\psi)(x-y)=(-1)^{\abs{\al}}\tau_x\wt{\partial^\al\psi}(y)\]
            であるから,引き続き$(-1)^{\abs{\al}}(T|\partial^\al(\tau_x\wt{\psi}))=(T|\tau_x\wt{\partial^\al\psi})=T*(\partial^\al\psi)$と計算できる.
            \item 任意の単位ベクトル$e\in\R^d$に対して,$T*\psi$が$e$方向に微分可能で$D_e(T*\psi)=T*(D_e\psi)$が成り立つことを示す.
            \begin{enumerate}
                \item $\eta_r:=\frac{\tau_{-re}-\tau_0}{r}$とすると,
                \[(\eta_r\psi)(y)=\frac{\psi(y+re)-\psi(y)}{r}\xrightarrow{r\to0}(D_e\psi)(y).\]
                さらに,任意の多重指数$\beta\in\N^d$について,
                \[\partial^\beta(\eta_r\psi)(y)=\frac{(\partial^\beta\psi)(y+re)-(\partial^\beta\psi)(y)}{r}\xrightarrow{r\to0}(D_e\partial^\beta\psi)(y)=\partial^\beta(D_e\psi)(y).\]
                これは,$\psi$の台がコンパクトであることと平均値の定理より,$y\in\R^d$について一様に起こるために,微分の交換が可能である.
                さらに,$\eta_r\psi\to D_e\psi\;\In\D$かつ$\wt{\eta_r\psi}\to\wt{D_e\psi}\;\In\D$.
                特に,$\tau_x\wt{\eta_r\psi}\to\tau_x\wt{D_e\psi}\;\In\D$.
                \item よって,$T\in\D'$について,
                \[(T*(\eta_r\psi))(x)=(T|\tau_x\wt{\eta_r\psi})\xrightarrow{r\to0}(T|\tau_x\wt{D_e\psi})=(T*(D_e\psi))(x).\]
                \item 一方で,(1)から
                \[(T*(\eta_r\psi))(x)=(\eta_r(T*\psi))(x)=\frac{(T*\psi)(x+re)-(T*\psi)(x)}{r}\xrightarrow{r\to0}D_e(T*\psi)(x).\]
            \end{enumerate}
            以上より,$D_e(T*\psi)=T*(D_e\psi)$.
            \item (2)の議論は繰り返すことが出来るから,$C^\infty$-級であることが判る.
        \end{enumerate}
    \end{enumerate}
\end{Proof}

\subsection{畳み込みの例}

\begin{proposition}[Delta分布が畳み込みの単位元である]
    任意の$\psi\in C^\infty(\R^d)$に対して,$\delta*\psi=\psi$である.
\end{proposition}
\begin{Proof}
    次のように計算できる:
    \[(\delta*\psi)(x)=(\delta|\tau_x\wt{\psi})=(\tau_x\wt{\psi})(0)=\wt{\psi}(-x)=\psi(x).\]
\end{Proof}

\subsection{畳み込みの随伴}

\begin{tcolorbox}[colframe=ForestGreen, colback=ForestGreen!10!white,breakable,colbacktitle=ForestGreen!40!white,coltitle=black,fonttitle=\bfseries\sffamily,
title=]
    $\psi$と畳み込みを取るという作用素を$C_\psi$とすると,その随伴は$C_{\wt{\psi}}$である.
\end{tcolorbox}

\begin{observation}[関数の畳み込みと内積]
    そもそも$f\in L^1_\loc(\R^d),\varphi,\psi\in\D$に対して,
    \begin{align*}
        (f*\psi|\varphi)&=\iint_{\R^d}\psi(x-y)f(y)\varphi(x)dxdy\\
        &=\int_{\R^d}f(y)(\wt{\psi}*\varphi)(y)dy=(f|\wt{\psi}*\varphi).
    \end{align*}
    あとは$\wt{\psi}*\varphi\in\D$に注意.この等式は$L^1_\loc(\R^d)\mono\D(\R^d)$に沿って一般化出来る.
\end{observation}

\begin{proposition}[畳み込み作用素の随伴作用素]\label{prop-adjoint-of-convolution}
    任意の$T\in\D'$と$\varphi,\psi\in\D$に対して,
    \[(T*\psi|\varphi)=(T|\wt{\psi}*\varphi).\]
\end{proposition}
\begin{Proof}\mbox{}
    \begin{enumerate}[{Step}1]
        \item 関数$\wt{\psi}*\varphi$は次のRiemann和の極限として表せる:
        \[\wt{\psi}*\varphi(x)=\int_{\R^d}\wt{\psi}(x-y)\varphi(y)dy=\lim_{\ep\to0}S_\ep(x),\qquad S_\ep(x):=\ep^d\sum_{v\in\Z^d}\wt{\psi}(x-\ep v)\varphi(\ep v).\]
        実は,最後の収束は$\D$の位相についても成り立つ.

        $S_\ep(x)$は$\ep\to0$について一様収束するため,$\cup_{\ep>0}S_\ep$はコンパクト台を持つ.
        よって,任意の多重指数$\al\in\N^d$に対して$\partial^\al S_\ep(x)$が$\partial^\al(\wt{\psi}*\varphi)$に一様収束することを示せば良い.
        \begin{align*}
            \partial^\al S_\ep(x)&=\ep^d\sum_{v\in\Z^d}\partial^\al(\wt{\psi})(x-\ep v)\varphi(\ep v)\\
            &\xrightarrow{\ep\to0}\int_{\R^d}\partial^\al(\wt{\psi})(x-y)\varphi(y)dy=((\partial^\al(\wt{\psi}))*\varphi)(x).
        \end{align*}
        であるが,$\partial^\al(\wt{\psi}),\varphi$は再びコンパクト台を持ち一様連続であるから,やはりRiemann和は一様に収束する.
        \item よって,$(T|S_\ep)\xrightarrow{\ep>0}(T|\wt{\psi}*\varphi)$を得たから,あとは$(T|S_\ep)\to(T*\psi|\varphi)$でもあることを示せば良い.
        \begin{align*}
            (T|S_\ep)&=\ep^d\sum_{v\in\Z^d}(T|\tau_{\ep v}\wt{\psi})\varphi(\ep v)=\ep^d\sum_{v\in\Z^d}(T*\psi)(\ep v)\varphi(\ep v)\\
            &\xrightarrow{\ep\to0}\int_{\R^d}(T*\psi)(y)\varphi(y)=(T*\psi|\varphi).
        \end{align*}
    \end{enumerate}
\end{Proof}
\begin{remarks}[Riemann和としての表示に戻る]
    観察にあった$T\in L^1_\loc(\R^d)$の場合と同じように,積分として現して変形を得ることが出来ない.
    しかし,$\wt{\psi}*\varphi$をRiemann和として表す地点まで戻ると,$T*-$の線形性を用いて同じように式変形が出来る!
\end{remarks}

\subsection{畳み込み作用素の特徴付け}

\begin{theorem}[\cite{Rudin-FunctionalAnalysis} Th'm 6.33]
    作用素$L:\D\to C^\infty(\R)$について,次は同値:
    \begin{enumerate}
        \item ある$u\in\D'$について,$L(\varphi)=u*\varphi\;(\varphi\in\D)$である.
        \item $L:\D\to C^\infty(\R)$は線型であり,平行移動と可換である:$\forall_{x\in\R^d}\;\tau_xL=L\tau_x$.
    \end{enumerate}
\end{theorem}

\begin{theorem}[\cite{Rudin-FunctionalAnalysis} Exxercise 6.25]
    作用素$L:\D\to C^\infty(\R)$について,次は同値:
    \begin{enumerate}
        \item ある$u\in\D'$について,$L(\varphi)=u*\varphi\;(\varphi\in\D)$である.
        \item $L:\D\to C^\infty(\R)$は線型であり,微分と可換である:
        \[LD^\al\varphi=D^\al L\varphi,\qquad\varphi\in\D,\al\in\N^d.\]
    \end{enumerate}
\end{theorem}

\section{超関数の台}

\subsection{超関数の台の定義}

\begin{tcolorbox}[colframe=ForestGreen, colback=ForestGreen!10!white,breakable,colbacktitle=ForestGreen!40!white,coltitle=black,fonttitle=\bfseries\sffamily,
title=]
    $U\osub\R^d$上に台を持たないとは,$U$上に台を持つ任意の試験関数$\varphi\in\D(U)$に対して$(T|\varphi)=0$を満たすことをいう.
\end{tcolorbox}

\begin{definition}[support of distribution]
    $U\osub\R^d$と$T\in\D'(\R^d)$について,
    \begin{enumerate}
        \item $\D(U):=\Brace{\varphi\in\D(\R^d)\mid\supp\varphi\subset U}$とする.
        \item $\forall_{\varphi\in\D(U)}\;(T|\varphi)=0$を満たす開集合$U\in\O(\R^d)$の全体
        \[\U(T):=\Brace{U\in\O(\R^d)\mid\forall_{\varphi\in\D(U)}\;(T|\varphi)=0}\]
        を考える.
        \item 超関数$T$の台とは,閉集合$\supp T:=\R^d\setminus\cup\U(T)$をいう.
    \end{enumerate}
\end{definition}

\begin{proposition}[整合性]
    任意の$f\in L^1_\loc(\R^d)$について,
    \begin{enumerate}
        \item 任意の開集合$U\in\O(\R^d)$について,$f|_{U}=0\Leftrightarrow U\in\U(T_f)$.
        \item $\supp f=\supp T_f$.
    \end{enumerate}
\end{proposition}
\begin{Proof}\mbox{}
    \begin{enumerate}
        \item $\Rightarrow$は$f|_{U}=0$のとき,任意の$\varphi\in\D(U)$について$f|_{\supp\varphi}=0$であるから,当然$(T_f|\varphi)=0$,すなわち$U\in\U(T)$.
        $\Leftarrow$は,仮定の$\forall_{\varphi\in\D(U)}\;(T_f|\varphi)=0$は$\forall_{\varphi\in\D(\R^d)}\;(T_{f|_U}|\varphi)=0$を含意するために,変分法の基本補題\ref{prop-fundamental-lemma-of-variational-calculus}から$f|_U=0$が従う.
        \item そもそも$\supp f$は$f|_{U}=0$を満たす$U\osub\R^d$で最大のものの補集合として特徴付けられることに注意すれば,(1)から従う.
    \end{enumerate}
\end{Proof}

\subsection{超関数の台の特徴付け}

\begin{tcolorbox}[colframe=ForestGreen, colback=ForestGreen!10!white,breakable,colbacktitle=ForestGreen!40!white,coltitle=black,fonttitle=\bfseries\sffamily,
title=]
    一般の関数の場合と同様,$f|_{U}=0\Leftrightarrow\forall_{\varphi\in\D(U)}\;(T_f|\varphi)=0$を満たす$U\osub\R^d$のうち\textbf{最大のものの補集合}として特徴付けられる.
\end{tcolorbox}

\begin{lemma}[任意の開集合族に対してそれに属する局所有限な1の分割が存在する]
    $\V\subset\O(\R^d)$について,$\Om:=\cup\V$とする.このとき,列$\{(\psi_i)\}_{i\in\N}\subset\D(\Om)$であって,次の2条件を満たすものが存在する:
    \begin{enumerate}
        \item 任意の$i\in\N$に対して,$\Im(\psi_i)\subset[0,1]$かつ$\exists_{V\in\V}\;\supp\psi_i\subset V$.
        \item 任意のコンパクト集合$K\subset\Om$に対して,ある$m\in\N$が存在して
        \[\sum_{i\in[m]}\psi_i(x)=1\;\on K.\]
    \end{enumerate}
\end{lemma}
\begin{Proof}\mbox{}
    \begin{enumerate}[{Step}1]
        \item $\Om$は可分だから稠密な可算部分集合$S\subset\Om$が取れる.
        $S$の元を中心とし,半径が有理数であって$\V$のある元に含まれるような$\R^d$に開球の全体を$(B_i)_{i\in\N}$とする.
        さらにそれぞれの半径を半分にしたものを$(C_i)_{i\in\N}$とする.すると,$\Om=\cup_{i\in\N}C_i=\cup_{i\in\N}B_i$.
        \item $\{\varphi_i\}\subset\D(\Om)$であって,$\Im\varphi_i\subset[0,1]$で,$\varphi_i=1\;\on C_i$かつ$\supp\varphi_i\subset B_i$を満たすものが取れる.
        これに対して
        \[\psi_1:=\varphi_1,\quad\psi_{i+1}:=(1-\varphi_1)\cdots(1-\varphi_i)\varphi_{i+1}\quad(i\in\N).\]
        と定めると,やはり$\supp\psi_i\subset B_i$であるから(1)を満たす.さらに,帰納的に
        \[\psi_1+\cdots+\psi_i=1-(1-\varphi_1)\cdots(1-\varphi_i)\]
        が成り立つため,任意のコンパクト集合$K\compsub\Om$に対してこれを被覆するのに十分な$(C_i)_{i\in[m]}$を取れば良い.
    \end{enumerate}
\end{Proof}

\begin{proposition}[台の補集合は最大性で特徴付けられる]
    $T\in\D'(\R^d)$を超関数とする.
    \begin{enumerate}
        \item 任意の族$\{U_\lambda\}_{\lambda\in\Lambda}\subset\U(T)$に対して,$\cup_{\lambda\in\Lambda}U_\lambda\in\U(T)$.
        \item $\U(T)$は$\cup\U(T)$を最大元とする.
    \end{enumerate}
\end{proposition}
\begin{Proof}
    (1)が示せれば,$\U(T)$の全体を族として取ることで(2)が成り立つ.
    族$\{U_\lambda\}_{\lambda\in\Lambda}\subset\U(T)$を任意に取り,$U:=\cup_{\lambda\in\Lambda}U_\lambda\osub\R^d$とする.
    任意の$\varphi\in\D(U)$について$(T|\varphi)=0$を示せば良い.
    \begin{enumerate}[{Step}1]
        \item $(U_\lambda)$に属する局所有限な1の分割$(\psi)_{i\in\N}$を取る.
        $\varphi$の台はコンパクトであるから,ある$m\in\N$について有限和$\varphi=\sum_{i\in[m]}\psi_i\varphi$で分解出来る.
        \item よって,各$\psi_i\varphi$はある$\lambda\in\Lambda$について$\D(U_\lambda)$に含まれるから,特に$(T|\psi_i\varphi)=0$.よって,
        次のように計算できる:
        \[(T|\varphi)=\sum_{i\in[m]}(T|\psi_i\varphi)=0.\]
    \end{enumerate}
\end{Proof}

\subsection{超関数の台の例}

\begin{tcolorbox}[colframe=ForestGreen, colback=ForestGreen!10!white,breakable,colbacktitle=ForestGreen!40!white,coltitle=black,fonttitle=\bfseries\sffamily,
title=]
    実際の超関数の台を証明するには,
    $\forall_{\varphi\in\D(U)}\;(T|\varphi)=0$を満たす最大の$U\osub\R^d$を見つける,という方針による.
\end{tcolorbox}

\begin{proposition}[Delta分布の台]
    $\supp\delta=\{0\}$.
\end{proposition}
\begin{Proof}
    $U:=\R\setminus\{0\}$とすると,任意の$\varphi\in\D(U)$について$(\delta|\varphi)=0$である.
    一方で,隆起関数$\Psi(x):=e^{-\frac{1}{1-x^2}}1_{\norm{x}<1}$\ref{remarks-transformed-Gauss-kernel}は$\D(\R^d)$の元であるが$(\delta|\Psi)=e^{-1}\ne0$である.
\end{Proof}

\begin{proposition}[零分布の特徴付け]
    任意の超関数$T\in\D'$について,$\supp T=\emptyset\Leftrightarrow T=0$.
\end{proposition}
\begin{Proof}
    変分法の基本補題\ref{prop-fundamental-lemma-of-variational-calculus}の換言である.
\end{Proof}

\subsection{一点に台を持つ超関数}

\begin{tcolorbox}[colframe=ForestGreen, colback=ForestGreen!10!white,breakable,colbacktitle=ForestGreen!40!white,coltitle=black,fonttitle=\bfseries\sffamily,
title=]
    $\delta$とその微分,そしてその線型和は原点のみからなる台を持つが,この性質がこのクラスを特徴付ける.
    実は,台が原点のみか空集合である分布は,ある多項式のFourier変換としても特徴付けられる.
\end{tcolorbox}

\begin{theorem}[原点のみを台に持つ超関数はデルタ分布の微分の線型結合である \cite{Rudin-FunctionalAnalysis} Th'm 6.25]
    超関数$T\in\D'$は$\supp T=\{0\}$を満たすとする.
    このとき,非負整数$N\in\N$と複素数$a_\al\in\C$が存在して,
    \[T=\sum_{\abs{\al}\le N}a_\al\partial^\al\delta.\]
\end{theorem}
\begin{Proof}\mbox{}
    \begin{enumerate}[{Step}1]
        \item 特に$T$はコンパクト台を持つので,ある$N\in\N$について階数$N$である\ref{prop-order-of-distribution-with-compact-support}:
        ある$C>0$が存在して,
        \[\abs{(T|\varphi)}\le C\sum_{\abs{\al}\le N}\norm{\partial^\al\varphi}_\infty,\qquad\varphi\in\D.\]
        \item $\eta\in\D$を$\{0\}$の近傍で$1$となるものとし,任意にとった$\varphi\in\D$に対して,関数
        \[\eta(x)\paren{\varphi(x)-\sum_{\abs{\al}\le N}\frac{(\partial^\al\varphi)(0)}{\al!}x^\al}\]
        を考えると,これは$\D$の元で,補題の仮定を満たすから,$\Ker T$の元である.
        ゆえに,
        \begin{align*}
            (T|\varphi)&=(T|\eta\varphi)=\paren{T\;\middle|\;x\mapsto\eta(x)\sum_{\abs{\al}\le N}\frac{(\partial^\al\varphi)(0)}{\al!}x^\al}\\
            &=\sum_{\abs{\al}\le N}\underbrace{\paren{T\;\middle|\;x\mapsto\eta(x)\frac{x^\al}{\al!}}}_{=:b_\al}(\partial^\al\varphi)(0)=:sum_{\abs{\al}\le N}(-1)^{\abs{\al}}b_\al(\partial^\al\delta|\varphi).
        \end{align*}
        $a_\al:=(-1)^{\abs{\al}}b_\al$とおくことで,定理の表示を得る.
    \end{enumerate}
\end{Proof}

\begin{lemma}[超関数の核に属するための条件]
    $T$を階数$N\in\N$を持つコンパクト台を持つ超関数とする.
    $\varphi\in\D$が,任意の$\abs{\al}\le N$を満たす多重指数$\al\in\N^d$に対して$(\partial^\al\varphi)(0)=0$を満たすならば,$(T|\varphi)=0$である.
\end{lemma}

\subsection{超関数の台の演算}

\begin{tcolorbox}[colframe=ForestGreen, colback=ForestGreen!10!white,breakable,colbacktitle=ForestGreen!40!white,coltitle=black,fonttitle=\bfseries\sffamily,
title=]
    $\supp(T*\psi)\subset\supp T+\supp\psi$が嬉しい.
\end{tcolorbox}

\begin{proposition}[畳み込みの台は和で抑えられる]\label{prop-property-of-support-of-distribution}
    $T\in\D'$を超関数とする.
    \begin{enumerate}
        \item 試験関数$\varphi\in\D$が$\supp T$の近傍で$0$ならば,$(T|\varphi)=0$.
        \item 微分は台を狭める:任意の多重指数$\al\in\N^d$に対して,$\supp(\partial^\al T)\subset\supp T$.
        \item 任意の$\psi\in\D$に対して,$\supp(T*\psi)\subset\supp T+\supp\psi$.
    \end{enumerate}
\end{proposition}
\begin{Proof}
    $U:=\cup\U(T)=\R^d\setminus\supp T$と表す.
    \begin{enumerate}
        \item $\varphi\in\D$が$\supp T$の近傍で$0$であるならば,$\varphi\in\D(U)$である.よって,$U=\cup\U(T)$より,$(T|\varphi)=0$.
        \item 任意の$\varphi\in\D(U)$について,$U\in\U(\partial^\al T)$を示せば良い.
        $\partial^\al\varphi\in\D(U)$であることに注意すれば,$(\partial^\al T|\varphi)=(-1)^{\abs{\al}}(T|\partial^\al\varphi)=0$.
        \item $\Brace{x\in\R^d\mid T*\psi(x)\ne0}\subset\supp T+\supp\psi$であることを示せば,$\supp T+\supp\psi$が閉集合であることより結論が成り立つ.
        
        $T*\psi(x)\ne0$を満たす$x\in\R^d$を任意に取る.$(T*\psi)(x)=(T|\tau_x\wt{\psi})$であるから,(1)より,$\tau_x\wt{\psi}$は$\supp T$の任意の近傍上で$0$以外の値も取る.
        すなわち,$\supp T\cap \supp\tau_x\wt{\psi}\ne\emptyset$.
        元$y\in\supp T\cap \supp\tau_x\wt{\psi}$を取って議論すると,
        $x-y\in\supp\psi$であるから,
        \[x\in y+\supp\psi\subset\supp T+\supp\psi.\]
    \end{enumerate}
\end{Proof}
\begin{remark}[超関数の台の上で零でも値は非零になり得る]
    (1)について,$\varphi|_{\supp T}=0$としても,$(T|\varphi)=0$とは限らない.実際,$\varphi(0)=0$を満たす$\varphi\in\D(\R^d)$について,$\supp\delta'=\{0\}$よりこの上で零であるが,
    $(\delta'|\varphi)=-\varphi'(0)$は零とは限らない.
\end{remark}

\begin{lemma}[閉集合のコンパクトな移動は閉]\label{lemma-compact-translation-of-closed-set}
    $A,B\subset\R^d$について,$A$が閉で$B$はコンパクトとする.$A+B$は閉である.
\end{lemma}
\begin{Proof}
    任意の点列$\{a_n+b_n\}\subset A+B$が収束点$x\in\R^d$を持つとし,$x\in A+B$を示す.
    部分列を取り直すことで,$b_n\to b\in B$として良い.すると残った値は
    \[\abs{a_n-(x-b)}\le\abs{a_b+b_b-x}+\abs{b-b_n}\]
    より,$a_n\to x-b$.すると,$A$は閉集合だから$x-b\in A$.以上より,$x=(x-b)+b\in A+B$.
\end{Proof}
\begin{remark}[コンパクトじゃない移動は開き得る]
    $B$は閉集合であるが有界とは限らないとする.
    すると,$A:=\R\times\{0\},B:=\Brace{x_2\ge e^{x_1}}$とすると,$A+B=\R\times\R^+$.
\end{remark}

\section{畳み込み2:超関数とコンパクト台を持つ超関数の畳み込み}

\begin{tcolorbox}[colframe=ForestGreen, colback=ForestGreen!10!white,breakable,colbacktitle=ForestGreen!40!white,coltitle=black,fonttitle=\bfseries\sffamily,
title=]
    今度は随伴によって定義を拡張する.
    \begin{enumerate}
        \item $T,S\in\D'(\R^d)$について,$S$がコンパクト台を持つとき,
        \[(T*S|\varphi):=(T|\wt{S}*\varphi),\qquad\varphi\in\D.\]
        で定まるのは,$\supp(\wt{S}*\varphi)\subset\supp(\wt{S})+\supp(\varphi)$による.
        \item 一方で$T$がコンパクト台を持つとき,
        \[(T*S|\varphi):=(\o{T}|\wt{S}*\varphi),\qquad\varphi\in\D.\]
        とするには,$\wt{S}*\varphi\in C^\infty(\R^d)$に過ぎないため,
        $T$の定義域を$C^\infty(\R^d)$上にまで延長する必要がある.
    \end{enumerate}
    この2つの定義は一致し,可換になる.
\end{tcolorbox}

\begin{notation}[超関数の反転]
    $T\in\D'$の反転$\wt{T}\in\D'$を随伴により$(\wt{T}|\varphi):=(T|\wt{\varphi})\;(\varphi\in\D)$と定める.
\end{notation}

\subsection{コンパクト台を持つ超関数の延長}

\begin{tcolorbox}[colframe=ForestGreen, colback=ForestGreen!10!white,breakable,colbacktitle=ForestGreen!40!white,coltitle=black,fonttitle=\bfseries\sffamily,
title=]
    コンパクト台を持つ超関数$T\in\D'_c(\R^d)$について,$\supp T$の近傍で$1$な元$\eta\in\D(\R^d)$が存在するから,これを軸として
    \[(\o{T}|\varphi):=(T|\eta\varphi),\qquad(\varphi\in C^\infty(\R^d))\]
    と延長出来る.
\end{tcolorbox}

\begin{proposition}
    $T\in\D'$はコンパクト台を持つとする.
    \begin{enumerate}
        \item $T$は線型な延長$\o{T}:C^\infty(\R^d)\to\C$を持つ:
        \[(\o{T}|\varphi):=(T|\eta\varphi),\qquad(\varphi\in C^\infty(\R^d)).\]
        ただし,$\eta\in\D$は$\supp T$の近傍で$\eta=1$を満たすならば,取り方に依らない.
        \item $\varphi,\psi\in C^\infty(\R^d)$が$\supp T$の近傍で一致するならば,$(\o{T}|\varphi)=(\o{T}|\psi)$を満たす.
    \end{enumerate}
\end{proposition}
\begin{Proof}\mbox{}
    \begin{enumerate}
        \item 
        \begin{description}
            \item[延長であること] 任意の$\varphi\in\D$に対して,$\varphi-\eta\varphi$は$\supp T$の近傍で$0$であるから,$(T|\varphi)=(T|\eta\varphi)$\ref{prop-property-of-support-of-distribution}.
            \item[well-definedness] $\zeta\in\D$も$\supp T$の近傍で$\zeta=1$を満たすとする.これも任意の$\varphi\in C^\infty(\R^d)$に対して$\eta\varphi-\zeta\varphi$は$\supp T$の近傍で$0$となるから,これが一意性を導く.
        \end{description}
        \item $\varphi,\psi\in C^\infty(\R^d)$が$\supp T$の近傍で一致するとする.$\eta\varphi-\eta\psi$は$\supp T$の近傍で$0$になることによる.
    \end{enumerate}
\end{Proof}

\subsection{畳み込みの定義}

\begin{definition}
    $T,S\in\D'$とし,いずれか一方はコンパクト台を持つとする.
    \begin{enumerate}
        \item $S$がコンパクト台を持つとき,$\wt{S}*\varphi\in\D$を利用して,
        \[(T*S|\varphi):=(T|\wt{S}*\varphi)\quad(\varphi\in\D)\]
        で定める.
        \item $T$がコンパクト台を持つとき,$\wt{S}*\varphi\in C^\infty(\R^d)$に過ぎないが,$T$の$C^\infty(\R^d)$への延長$\o{T}$を用いて,
        \[(T*S|\varphi):=(\o{T}|\wt{S}*\varphi)\quad(\varphi\in\D)\]
        で定める.
    \end{enumerate}
\end{definition}
\begin{remark}
    $S,T$のいずれもコンパクト台を持つとき,$\o{T}$は$T$の延長であるから$\D(\R^d)$上では同じ対応を定めており,2つの定義は一致する.
\end{remark}

\begin{lemma}[well-definedness:畳み込みはたしかに超関数を与える]\label{lemma-well-definedness-of-convolution-between-distributions}
    $\psi_n\to\psi\;\In\D$とする.
    \begin{enumerate}
        \item 畳み込みの連続性:コンパクト台を持つ任意の超関数$R\in\D'$に対して,$R*\psi_n\to R*\psi\;\In\D$.
        \item $\D$-作用の連続性:任意の$R\in\D'$と$\eta\in\D$に対して,$\eta(R*\psi_n)\to\eta(R*\psi)\;\In\D$.
        \item 畳み込みのwell-definedness:$T*S\in\D'$はたしかに超関数である.
    \end{enumerate}
\end{lemma}
\begin{Proof}
    $\psi_n\to\psi\;\In\D$なので,あるコンパクト集合$K\csub\R^d$が存在して,$\cup_{n\in\N}\supp\varphi_n\cup\supp\varphi\subset K$.
    \begin{enumerate}
        \item 
        \begin{enumerate}[{Step}1]
            \item 
            よって,$\cup_{n\in\N}\supp(R*\psi_n)\cup\supp(R*\psi)\subset\supp R+K$.
            \item 任意の$\al\in\N^d$を取って,$R*\psi_n$の$\al$-偏導関数が一様収束することを示せばよい.任意の$x\in\supp R+K$に対して,
            \begin{align*}
                \abs{\partial^\al(R*\psi_n)(x)-\partial^\al(R*\psi)(x)}&=\abs{(R*(\partial^\al\psi_n))(x)-(R*(\partial^\al\psi))(x)}\\
                &=\abs{(R|\tau_x(\wt{\partial^\al(\psi_n-\psi)}))}\\
                &\le C\sum_{\abs{\beta}\le N}\norm{\partial^\beta(\tau_x(\wt{\partial^\al(\psi_n-\psi)}))}_\infty\\
                &=C\sum_{\abs{\beta}\le N}\norm{\partial^{\al+\beta}\psi_n-\partial^{\al+\beta}\psi}_\infty\to0.
            \end{align*}
            が,$R$の超関数としての連続性の特徴付け\ref{prop-characterization-of-distribution}から従う.
            この評価はたしかに$x\in\supp R+K$の取り方に依存していない.
        \end{enumerate}
        \item 台については(1)とまったく同様.あとは任意の$\al\in\N^d$に対して,$\eta(R*\psi_n)$の$\al$-偏導関数の一様収束を示せばよい.
        任意に$x\in\supp\eta$を取ると,Leibniz則より,
        \begin{align*}
            \abs{\partial^\al(\eta(R*\psi_n))(x)-\partial^\al(\eta(R*\psi))(x)}&=\sum_{\al_1+\al_2=\al}c_{\al_1\al_2}\abs{(\partial^{\al_1}\eta)(x)}\abs{\partial^{\al_2}(R*\psi_n)(x)-\partial^{\al_2}(R*\psi)(x)}\\
            &\le\sum_{\al_1+\al_2=\al}c_{\al_1\al_2}\norm{\partial^{\al_1}\eta}_\infty C\sum_{\abs{\beta}\le N}\norm{\partial^{\al_2+\beta}\psi_n-\partial^{\al_2+\beta}\psi}_\infty\to0.
        \end{align*}
        この評価はたしかに$x\in\supp\eta$に依存しない.
        \item 定義(1)については,(1)より,
        \[(T*S|\psi_n)=(T|\wt{S}*\psi_n)\to(T|\wt{S}*\psi)=(T*S|\psi).\]
        定義(2)については,(2)も併せて,任意の$\supp T$の近傍で$\eta=1$を満たす$\eta\in\D$について,
        \[(T*S|\psi_n)=(\o{T}|\wt{S}*\psi_n)=(T|\eta\wt{S}*\psi_n)\to(T|\eta\wt{S}*\psi)=(T*S|\psi).\]
    \end{enumerate}
\end{Proof}

\subsection{畳み込みの代数的性質}

\begin{proposition}[結合性]\label{prop-associativity-of-convolution-between-distributions}
    任意の$\varphi,\psi\in\D$と$T\in\D'$に対して,
    \begin{enumerate}
        \item 反転の畳み込みに対する関手性:$\wt{T}*\wt{\psi}=\wt{T*\psi}$.
        \item 結合性:$(T*\psi)*\varphi=T*(\psi*\varphi)$.
    \end{enumerate}
\end{proposition}
\begin{Proof}\mbox{}
    \begin{enumerate}
        \item まず等式$\wt{\tau_x\psi}=\tau_{-x}\wt{\psi}$を示す.任意の$y\in\R^d$に対して,
        \[(\wt{\tau_x*\psi})(y)=(\tau_x*\psi)(-y)=\psi(-y-x)=\wt{\psi}(x+y)=(\tau_{-x}\wt{\psi})(y).\]
        したがって,次のようにわかる:
        \[(\wt{T}*\wt{\psi})(x)=(\wt{T}|\tau_x\psi)=(T|\tau_{-x}\wt{\psi})=(T*\psi)(-x)=(\wt{T*\psi})(x).\]
        \item 任意の$x\in\R^d$に対して,
        \begin{align*}
            \LHS=((T*\psi)*\varphi)(x)&=(T*\psi|\tau_x\wt{\varphi})=(T|\wt{\psi}*(\tau_x\wt{\varphi})).
        \end{align*}
        であるが,
        \begin{align*}
            \wt{\psi}*(\tau_x\wt{\varphi})&=\int_{\R^d}\wt{\psi}(\bullet-y)\wt{\varphi}(y-x)dy=\int_{\R^d}\wt{\varphi}(\bullet-x-y)\wt{\varphi}(y)dy\\
            &=\tau_x(\wt{\psi}*\wt{\varphi})=\tau_x(\wt{\psi*\varphi}).
        \end{align*}
        より,最初の計算は次のように続けることができる:
        \[(T|\wt{\psi}*(\tau_x\wt{\varphi}))=(T|\tau_x(\wt{\psi*\varphi}))=(T*(\psi*\varphi))(x).\]
    \end{enumerate}
\end{Proof}

\begin{example}
    うち2つが超関数である場合,結合性は成り立たない:
    \[0=(1*\delta')*H=1*(\delta'*H)=1.\]
    これは,
    \begin{enumerate}
        \item $1*\delta'=0$.
        \item $\delta'*H=\delta$.
    \end{enumerate}
    による.
\end{example}
\begin{Proof}\mbox{}
    \begin{enumerate}
        \item 可換性より,
        \[(1*\delta'|\varphi)=(\delta'*1|\varphi)=(\delta'|1*\varphi)\]
        であるが,
        \[1*\varphi(x)=\int_\R 1(x-y)\varphi(y)dy=\int_\R\varphi(y)dy\]
        は定値関数であるから,微分は零である.
        \item 微分と$*$の可換性から,
        \[\delta'*H=(\delta*H)'=H'=\delta.\]
    \end{enumerate}
\end{Proof}

\begin{proposition}[畳み込みの可換性]
    $T,S\in\D'$のいずれかがコンパクト台を持つとする.このとき,
    \[T*S=S*T.\]
\end{proposition}
\begin{Proof}
    $S$がコンパクト台を持つと仮定して一般性は失われない.
    \begin{description}
        \item[方針] 任意の$\varphi,\psi\in\D$について,
        \[(T*S|\varphi*\psi)=(S*T|\varphi*\psi)\]
        が成り立つことを示す.すると,$\varphi*\psi$の形をした元が$\D$内で稠密であることと,$T*S,S*T$が連続であることから,
        延長が一意であることより結論が従う.
        実際,任意の$\int_{\R^d}\varphi(x)dx=1$を満たす$\varphi\in\D$に対して$\varphi_n(x)=n^d\varphi(nx)$とおくと,これは$\R^d$上の総和核となる\ref{prop-construction-of-summability-kernel-on-Rd}.
        したがって,任意の$\psi\in\D\subset UC_b(\R^d)$に対して,$\varphi_n*\psi\to\psi\;\In\D$.議論は次の系\ref{cor-dense-subspace-of-distribution}でしている.
        \item[証明] 
        \begin{enumerate}[{Step}1]
            \item 関数との畳み込みの可換性から,
            \begin{align*}
                (T*S|\varphi*\psi)&=(T|\wt{S}*(\varphi*\psi))=(T|(\wt{S}*\varphi)*\psi)=(T|\psi*(\wt{S}*\varphi))\\
                &=(T*\wt{\psi}|\wt{S}*\varphi)=(\wt{T}*\psi|S*\wt{\psi}).
            \end{align*}
            \item $\wt{T}*\psi\in C^\infty(\R^d)$としかわからないから,$S*\wt{\varphi}$の超関数としての$C^\infty(\R^d)$上への延長を考える必要がある.
            任意のコンパクト集合$\supp S\subset K\compsub\R^d$と$\eta\in\D$であって$K\setminus\supp\varphi$の近傍で$\eta=1$となるものについて,$\supp(S*\wt{\varphi})\subset K\setminus\supp\varphi$が成り立つから,
            \[(S*\wt{\varphi}|\eta(\wt{T}*\psi))=(S|\varphi*(\eta(\wt{T}*\psi)))=(S|(\eta(\wt{T}*\psi))*\varphi).\]
            \item この右辺は任意の$x\in K$に対して,
            \[((\eta(\wt{T}*\psi))*\varphi)(x)=\int_{\supp\varphi}\underbrace{\eta(x-y)}_{=1}(\wt{T}*\psi)(x-y)\varphi(y)dy=((\wt{T}*\psi)*\varphi)(x).\]
            特に,$\supp S$の近傍で$(\eta(\wt{T}*\psi))*\varphi=(\wt{T}*\psi)*\varphi$.
            よって,命題\ref{prop-property-of-support-of-distribution}(2)より,
            \[(S|(\eta(\wt{T}*\psi))*\varphi)=(\o{S}|(\wt{T}*\psi)*\varphi)=(\o{S}|\wt{T}*(\psi*\varphi))=(S*T|\psi*\varphi).\]
            以上より結論を得る.
        \end{enumerate}
    \end{description}
\end{Proof}

\subsection{超関数の総和的性質}

\begin{corollary}[超関数の空間で可微分関数の空間は稠密 \cite{Rudin-FunctionalAnalysis} Th'm 6.32]\label{cor-dense-subspace-of-distribution}
    証明の中で次の(1)を示した.$\{k_\lambda\}\subset L^1(\R^d)$を総和核とする.
    \begin{enumerate}
        \item 任意の$\varphi\in\D$に対して,$\varphi*k_\lambda\to\varphi\;\In\D$.
        \item 任意の超関数$T\in\D'$に対して,$T*k_\lambda\to T\;\In\D'$.
    \end{enumerate}
    特に,超関数の空間$\D'$で,可微分関数の全体$C^\infty$は稠密である.
\end{corollary}
\begin{Proof}\mbox{}
    \begin{enumerate}
        \item 命題\ref{prop-convolution-to-be-continuous}の証明中で,連続関数$f\in C(\R^d)$について,$f*k_\lambda\to f$は広義一様収束することを議論した.
        したがって,任意の$\varphi\in\D$と$\al\in\N^d$について,$D^\al(\varphi*k_\lambda)\to D^\al\varphi$は一様収束する.
        さらに,$\varphi*k_\lambda$の台についても,$k_\lambda$の台が$\{0\}$に縮んでいくために,条件を満たす.
        \item 任意の$\varphi\in\D$に対して,
        \begin{align*}
            (u|\wt{\varphi})&=(u|\wt{\varphi}*\delta)=(u*\delta)(0)=\lim_{\lambda\to\infty}\Paren{u*(k_\lambda*\varphi)}(0)\\
            &=\lim_{\lambda\to\infty}\Paren{(u*k_\lambda)*\varphi}(0)=\lim_{\lambda\to\infty}(u*k_\lambda|\wt{\varphi}).
        \end{align*}
        よりわかる.畳み込みの作用素としての連続性と,超関数1つ試験関数2つの場合の結合性\ref{prop-associativity-of-convolution-between-distributions},ペアリングの連続性を用いた.
    \end{enumerate}
\end{Proof}

\subsection{畳み込みと微分・台の関係}

\begin{proposition}[微分との可換性]
    $T,S\in\D'$のいずれかがコンパクト台を持つとする.このとき,任意の多重指数$\al\in\N^d$に対して,
    \[\partial^\al(T*S)=(\partial^\al T)*S=T*(\partial^\al S).\]
\end{proposition}
\begin{Proof}
    可換性より,$S$がコンパクト台を持つと仮定して一般性を失わない.
    \begin{enumerate}[{Step}1]
        \item 左の等式を示す.任意の$\varphi\in\D$に対して,
        \[(\partial^\al(T*S)|\varphi)=(-1)^{\abs{\al}}(T*S|\partial^\al\varphi)=(-1)^{\abs{\al}}(T|\wt{S}*(\partial^\al\varphi)).\]
        すると関数との畳み込みに対する微分の可換性より,
        $\wt{S}*(\partial^\al\varphi)=\partial^\al(\wt{S}*\varphi)$であるから,次のように計算を進めることができる:
        \[(-1)^{\abs{\al}}(T|\wt{S}*(\partial^\al\varphi))=(-1)^{\abs{\al}}(T|\partial^\al(\wt{S}*\varphi))=(\partial^\al T|\wt{S}*\varphi)=((\partial^\al T)*S|\varphi).\]
        \item 右の等式を示す.まず,
        任意の$\psi\in\D$に対して,
        \[(\partial^\al\wt{S}|\psi)=(-1)^{\abs{\al}}(\wt{S}|\partial^\al\varphi)=(-1)^{\abs{\al}}(S|\wt{\partial^\al\psi})=(-1)^{\abs{\al}}(\partial^\al S|\wt{\psi})=(-1)^{\abs{\al}}(\wt{\partial^\al S}|\psi).\]
        より,$\partial^\al\wt{S}=(-1)^{\abs{\al}}(\wt{\partial^\al S})$.
        これを関数との畳み込みに対する微分の移動$\wt{S}*(\partial^\al\varphi)=(\partial^\al\wt{S})*\varphi$\ref{prop-property-of-convolution-with-functions}と併せて,
        \[(-1)^{\abs{\al}}(T|\wt{S}*(\partial^\al\varphi))=(-1)^{\abs{\al}}(T|(\partial^\al\wt{S})*\varphi)=(T|\wt{\partial^\al S}*\varphi)=(T*(\partial^\al S)|\varphi).\]
    \end{enumerate}
\end{Proof}

\begin{proposition}[台の演算]
    $T,S\in\D'$のいずれかがコンパクト台を持つとする.このとき,
    \[\supp(T*S)\subset\supp T+\supp S.\]
\end{proposition}
\begin{Proof}
    $\supp T+\supp S$は閉集合のコンパクトな移動であるから,閉集合である\ref{lemma-compact-translation-of-closed-set}.
    よって,任意の$\varphi\in\D$であって,台が$\supp T+\supp S$と交わらないものについて,$(T*S|\varphi)=0$が成り立つことを示せば,$\U(T*S)$が$\R^d\setminus(\supp T+\supp S)$よりも小さいことがわかる.
    いま,
    \[\supp(\wt{S}*\varphi)\subset\supp\wt{S}+\supp\varphi=\supp\varphi-\supp S.\]
    であることに注意すると,右辺は$\supp T$と交わらない閉集合なので,$\wt{S}*\varphi$は$\supp T$の近傍で$0$である.
    故に,$(T*S|\varphi)=(\o{T}|\wt{S}*\varphi)=0$が命題\ref{prop-property-of-support-of-distribution}(1)による.
\end{Proof}

\subsection{畳み込みの例}

\begin{proposition}
    任意の$T\in\D'$に対して,$T*\delta=T=\delta*T$.
\end{proposition}
\begin{Proof}
    任意の$\varphi\in\D$に対して,
    \begin{align*}
        (T*\delta|\varphi)&=(T|\wt{\delta}*\varphi)=(T|\delta*\varphi)=(T|\varphi).
    \end{align*}
    あとは可換性$T*\delta=\delta*T$によると考えてよい.
\end{Proof}

\section{コンパクト台を持つ超関数の導関数としての表示}

\begin{tcolorbox}[colframe=ForestGreen, colback=ForestGreen!10!white,breakable,colbacktitle=ForestGreen!40!white,coltitle=black,fonttitle=\bfseries\sffamily,
title=]
    \begin{enumerate}
        \item 任意の超関数$T\in\D'(\R^d)$は,局所的には連続関数$f\in C(\R^d)$と多重指数$\al\in\N^d$を用いて$\partial^\al f$の形で表せる.
        \item そしてその深さの指標が階数である.
        \item これを,コンパクト台を持つ超関数$T\in\D'_c(\R^d)$について,有限個のコンパクト台を持つ連続関数$f_1,\cdots,f_n\in C_c(\R^d)$と$\al_1,\cdots,\al_n\in\N^d$を用いて,
        \[T=\sum_{k\in[n]}\partial^{\al_k}f_k.\]
        と表せる,という定式化で示す.
    \end{enumerate}
\end{tcolorbox}

\begin{remarks}[本質は超関数微分にあり]
    この事実より,超関数の性質はひとえに超関数微分のみから来るもので,元々は本当に関数であったことが解る.
    あるいは,関数は超関数を認めれば只管に微分可能である.
\end{remarks}

\subsection{連続関数の導関数としての超関数の局所表示}

\begin{example}[Delta分布の局所関数表示]
    Delta分布はコンパクト台を持ち,階数$0$である.
    \begin{enumerate}
        \item $g(x):=x1_{\R_+}$とすると連続である.
        $g'(x)=H(x)$はHeaviside関数,$g''(x)=\delta(x)$がDelta分布である.
        \item 一方で,$g$の台はコンパクトではない.
        そこで$h$を,$\R_+$上にコンパクトな台を持ち,原点の近くで$h(x)=x$であり,$\R^+$上で$C^\infty$-級である関数とする.
        すると,この関数の二階微分$h''$は,原点の近傍では超関数$\delta$と関数$0$の和で,原点から遠い場所では普通の$C^\infty$-級関数$\varphi\in\D$である.
        すなわち,ある$\varphi\in\D$について,$h''=\delta+\varphi\Leftrightarrow\delta=h''-\varphi$.
        \item 単独のコンパクト台を持つ連続関数の微分として理解することは出来ない.
        
        まず,1階の微分について$T'=\delta\;(T\in\D')$が成り立つならば,そもそも$H'=\delta$であったから,$T=H+c_0$に限る\ref{prop-characterization-of-constants-through-distributional-derivative}.
        これは連続でないしコンパクト台でもない.
        続いて,2階の微分について$S''=\delta\;(S\in\D')$が成り立つとすると,$g+c_0x+x_1$と表せ,$\R$上の連続関数であるが,コンパクト台は持たない.
        高階の微分でも同様である.
    \end{enumerate}
\end{example}

\subsection{超関数の階数}

\begin{tcolorbox}[colframe=ForestGreen, colback=ForestGreen!10!white,breakable,colbacktitle=ForestGreen!40!white,coltitle=black,fonttitle=\bfseries\sffamily,
title=]
    $\D(U)$の半ノルム$\norm{\varphi}_N$について有界であるとき,$T$は階数$N$を持つという.
    $C^\infty(U)$の位相も半ノルムの列$(\norm{\varphi}_N)_{N\in\N}$によって生成されるから,これは$C^\infty(U)$上への連続線型延長を持つことを意味する.
\end{tcolorbox}

\begin{definition}[order]
    超関数$T\in\D'$が\textbf{階数$N$}を持つとは,ある$C>0$が存在して,任意の$\varphi\in\D$に対して,
    \[\abs{(T|\varphi)}\le C\sum_{\abs{\al}\le N}\norm{\partial^\al\varphi}_\infty.\]
    を満たすことをいう.
\end{definition}
\begin{remarks}[複素Borel測度の一般化]
    \[\norm{\varphi}_N:=\sum_{\abs{\al}\le N}\norm{\partial^\al\varphi}\]
    は$\D(\R^d)$上にノルムを定める.$N=0$のときが一様ノルムであり,$N$と共に単調増大する.
    実は,Riesz-Markov-Kakutaniの定理によれば,階数$0$の超関数=一様ノルムについての有界線型汎関数がちょうど複素Borel測度に対応する.
\end{remarks}

\begin{example}[微分の階数の一般化である]\mbox{}
    \begin{enumerate}
        \item 可積分関数$f\in L^1(\R^d)$について,$\partial^\al f$は階数$\abs{\al}$である.
        \item Delta分布$\delta$は階数$0$を持つ.
        その$\al$-偏微分$\partial^\al\delta$は階数$\abs{\al}$である.
    \end{enumerate}
\end{example}
\begin{Proof}\mbox{}
    \begin{enumerate}
        \item 任意の$\al\in\N^d,\varphi\in\D$に対して,
        \[\abs{(\partial^\al f|\varphi)}\le\norm{f}_{L^1(\R^d)}\norm{\partial^\al\varphi}_\infty.\]
        と評価出来る.
        \item 任意の$\al\in\N^d,\varphi\in\D$に対して,
        \[\abs{(\partial^\al\delta|\varphi)}=\abs{(\delta|\partial^\al\varphi)}=\abs{\partial^\al\varphi(0)}\le\norm{\partial^\al\varphi}_\infty.\]
    \end{enumerate}
\end{Proof}

\subsection{階数の例}

\begin{proposition}
    $\PV\paren{\frac{1}{x}}$は,$\D$上でも$\S$上でも階数$1$である.
\end{proposition}
\begin{Proof}\mbox{}
    \begin{description}
        \item[$\S$上での階数] 任意の$\varphi\in\S(\R)$に対して,
        Taylorの定理を念頭に,
        \[\varphi(x)=\varphi(0)+x\varphi_1(x),\qquad\varphi_1(x):=\frac{\varphi(x)-\varphi(0)}{x}.\]
        と変形すると,中間値の定理より,
        \[\sup_{\abs{x}\le R}\abs{\varphi_1(x)}=\sup_{\abs{x}\le R}\Abs{\frac{\varphi(x)-\varphi(0)}{x}}\le\sup_{\abs{x}\le R}\abs{\varphi'(x)}.\]
        これを踏まえて,
        \begin{align*}
            \paren{\PV\frac{1}{x}\middle|\varphi}&=\lim_{\ep\to0}\int_{\abs{x}\ge\ep}\frac{\varphi(x)}{x}dx\\
            &=\lim_{\ep\to0}\int_{\ep\le\abs{x}\le1}\paren{\frac{\varphi(0)}{x}+\varphi_1(x)}dx+\int_{\abs{x}\ge1}\frac{\varphi(x)}{x}dx\\
            &=\lim_{\ep\to0}\int_{\ep\le\abs{x}\le1}\varphi_1(x)dx+\int_{\abs{x}\ge1}\frac{x\varphi(x)}{x^2}dx\\
            &\le2\sup_{\abs{x}\le 1}\abs{\varphi'(x)}+\sup_{x\in\R}\abs{x\varphi(x)}\int_{\abs{x}\ge1}\frac{dx}{x^2}\\
            &\le2\sup_{x\in\R}\abs{\varphi'(x)}+2\sup_{x\in\R}\abs{x\varphi(x)}\le2\norm{\varphi}_1.
        \end{align*}
        第二項の評価は\href{https://math.stackexchange.com/questions/2708497/principal-value-of-1-x}{stackexchange},第一項の評価は\cite{Grubb09-Distributions}による.
        \item[$\D$上での階数] 全く同様の評価が行えて(第二項の評価は必要なくなるが),$\D$上でも階数$1$である.
    \end{description}
\end{Proof}
\begin{remarks}
    こうやって評価するのか.特異点($0$での発散と無限大での発散)は分解して一つ一つ和として処理するために,$\D,\S$のノルムの定義が一様ノルムと和で2つを組み合わせる形で与えていたのかもしれない.
\end{remarks}

\subsection{コンパクト台を持つ超関数の性質}

\begin{proposition}[コンパクト台を持つ超関数は階数を持つ]\label{prop-order-of-distribution-with-compact-support}
    $T\in\D'_c(\R^d)$はコンパクト台を持つとする.ある$N\in\N$について,階数$N$である.
\end{proposition}
\begin{Proof}
    コンパクト台を持つ超関数は,$\supp T$を含む任意のコンパクト集合上で一定の$N\in\N$を用いて,$\norm{\bullet}_{N}$について有界になる.
    \begin{enumerate}[{Step}1]
        \item $K\compsub\R^d$を,台$\supp T$を距離$1$だけ膨らませたものとする:$\supp T\subset K$.
        すると,$T\in\D'$の連続性の特徴付け\ref{prop-characterization-of-distribution}より,
        ある$C>0,N\in\N$が存在して,
        \[\abs{(T|\varphi)}\le C\sum_{\abs{\al}\le N}\norm{\partial^\al\varphi}_\infty.\]
        \item 任意の$\varphi\in\D$について,同様の評価が成功することを示せばよい.そのために,$\eta\in\D$であって,$\supp T$の近傍で$\eta=1$かつ$\R^d\setminus K$上で$\eta=0$となるものを取ると,$\supp(\eta\varphi)\subset K$を満たすから,上の評価から,
        \[\abs{(T|\varphi)}=\abs{(T|\eta\varphi)}\le C\sum_{\abs{\al}\le N}\norm{\partial^\al(\eta\varphi)}_\infty\le C\sum_{\abs{\al}\le N}\sum_{\al_1+\al_2=\al}c_{\al_1\al_2}\norm{\partial^{\al_1}\eta}_\infty\norm{\partial^{\al_2}\varphi}_\infty.\]
    \end{enumerate}
\end{Proof}

\begin{proposition}[畳み込みが連続になるための十分条件]\label{prop-convolution-to-be-continuous}
    $T\in\D'_c(\R^d)$はコンパクト台を持ち,階数$N$であるとする.
    このとき,任意の$f\in C^N(\R^d)$について,畳み込み$T*f\in\D'$は$\R^d$上の連続関数である.
\end{proposition}
\begin{Proof}\mbox{}
    \begin{description}
        \item[方針] 
        そもそも$T*f$は$f\in L^1_\loc(\R^d)\mono\D'(\R^d)$と見ての定義であることに注意.
        これを,$(T|\tau_x\wt{f})$と見て証明したい.
        
        いま$f\in C^N(\R^d)$でしかないので,$f$の軟化を考える.
        $\psi\in\D$であって$\int_{\R^d}\psi(x)dx=1$を満たすものを用いて,総和核$\psi_n(x):=n^d\psi(nx)$を構成する.これについて,
        \[F_n(x):=(\o{T}|\tau_x(\wt{\psi_n*f}))=\Paren{T\;\bigg|\;\eta(\tau_x(\wt{\psi_n*f}))}.\]
        と定め,極限$F(x):=\lim_{n\to\infty}F_n(x)$を考えると,$T*f=F$となっているはずである.これを示す.
        %\begin{enumerate}
        %    \item そもそも$\psi_n*f\in C^\infty(\R^d)$までしかわからないため,延長$\o{T}$を考えていることに注意.
        %    \item まず,これは$T*f=F$を満たす.実際,$\eta(\tau_x(\wt{\psi_n*f}))\to\eta(\tau_x\wt{f})\in C_c^\infty(\R^d)$であること\ref{lemma-well-definedness-of-convolution-between-distributions}による.
        %    \item まったく同様に$x\to a\in\R^d$について,$\eta(\tau_x(\wt{\psi_n*f}))\to\eta(\tau_a(\wt{\psi_n*f}))$より,$F_n$は連続.
        %\end{enumerate}
        \item[$F$の存在と連続性] まず,$F_n$が連続であることは,任意の収束列$x\to a\in\R^d$について,$\eta(\tau_x(\wt{\psi_n*f}))\to\eta(\tau_a(\wt{\psi_n*f}))$よりわかる.あとは,
        任意の$x\in\R^d$について,$(F_n(x))_{n\in\N}$はCauchy列になることと,その際の収束が広義一様であることを示す.
        \begin{enumerate}[{Step}1]
            \item $T$は階数$N$であるから,ある$C>0$が存在して,
            \[\abs{F_n(x)-F_m(x)}\le C\sum_{\abs{\al}\le N}\norm{\partial^\al(\eta\tau_x(\wt{\psi_n*f}))-\partial^\al(\eta\tau_x(\wt{\psi_m*f}))}_\infty.\]
            $\wt{}$は$*$上に関手性を持つ\ref{prop-associativity-of-convolution-between-distributions}こと,Leibnitz則,微分と$*$の可換性より,
            \[\abs{F_n(x)-F_m(x)}\le C\sum_{\abs{\al}\le N}\sum_{\al_1+\al_2=\al}c_{\al_1\al_2}\norm{(\partial^{\al_1}\eta)\tau_x(\wt{\psi_n}*\partial^{\al_2}\wt{f})-\wt{\psi_m}*\partial^{\al_2}\wt{f}}_\infty.\]
            \item 右辺の$\partial^{\al_2}\wt{f}=:g$は連続関数であるから,あとは$\wt{\psi_n}g$が$\R^d$上広義一様収束することを示せばよい.
            これは,
            \[(\wt{\psi_n}*g)(x)-g(x)=\int_B\wt{\psi_n}(y)(g(x-y)-g(x))dy.\]
            が任意の閉球$\supp\wt{\psi}\subset B\compsub\R^d$について成り立つことがわかる.
        \end{enumerate}
        \item[$T*f=F$である] すなわち,
        \[(T*f|\varphi)=(\o{T}|\wt{f}*\varphi)=(T|\eta(\wt{f}*\varphi))=(F|\varphi).\]
        を示せばよい.これは畳み込みの随伴関係の一般化と見れる.証明は\ref{prop-adjoint-of-convolution}を修正する形で進む.
    \end{description}
\end{Proof}

\subsection{超関数の連続関数の導関数としての表現}

\begin{theorem}[コンパクト台を持つ超関数は連続関数の微分である]
    $T\in\D'_c(\R^d)$はコンパクト台を持ち,階数$N$であるとする.
    このとき,$\R^d$上の連続関数$g\in C(\R^d)$が存在して,
    \[\pp{^{N+2}}{x_1^{N+2}}\pp{^{N+2}}{x_2^{N+2}}\cdots\pp{^{N+2}}{x_d^{N+2}}g=T.\]
\end{theorem}
\begin{Proof}\mbox{}
    \begin{description}
        \item[基本解の構成] $E\in C^N(\R^d)$を
        \[E(x):=\frac{x_1^{N+1}\cdots x_d^{N+1}}{((N+1)!)^d}1_{(\R^+)^d}(x).\] 
        と定めると,超関数の偏微分方程式
        \[\pp{^{N+2}}{x_1^{N+2}}\pp{^{N+2}}{x_2^{N+2}}\cdots\pp{^{N+2}}{x_d^{N+2}}E=\delta.\]
        を満たす.実際,$E$は通常の意味の偏微分方程式
        \[\pp{^{N+1}}{x_1^{N+1}}\pp{^{N+1}}{x_2^{N+1}}\cdots\pp{^{N+1}}{x_d^{N+1}}E=1_{(\R^+)^d}(x).\]
        を満たし,さらに各変数$x_i$で微分して$\delta$を得る.
        \item[証明] $g:=T*E$とおくと,これは主張の偏微分方程式を満たす.そして,命題\ref{prop-convolution-to-be-continuous}より$g$は連続である.
    \end{description}
\end{Proof}

\begin{corollary}[一般の超関数も局所的には連続関数の微分である]
    任意の超関数$T\in\D'$と有界開集合$U\osub\R^d$に対して,
    連続関数$f\in C(\R^d)$と多重指数$\al\in\N^d$が存在して,
    \[(T|\varphi)=(\partial^\al f|\varphi),\qquad\varphi\in\D(U).\]
\end{corollary}
\begin{Proof}
    $U\osub\R^d$上で$1$となる元$\eta\in\D$を取り,超関数$S\in\D'$を
    \[(S|\varphi):=(T|\eta\varphi),\qquad\varphi\in\D.\]
    で定めると,$\supp S\subset\supp\eta$よりコンパクト台を持つ.
    すると定理より,ある$f\in C(\R^d)$と$\al\in\N^d$が存在して,
    \[S=\pp{^\al f}{x^\al}.\]
    すべて併せると,任意の$\varphi\in\D(U)$に対しては,
    \[(T|\varphi)=(T|\eta\varphi)=(S|\varphi)=(\partial^\al f|\varphi).\]
\end{Proof}
\begin{remarks}
    証明の中で,$T\in\D'$のふるまいを$U$上で完全に再現する$S\in\D'_c$を構成して示したが,これを$S=\eta T$で表す.
\end{remarks}

\begin{corollary}[コンパクト台を持つ超関数はコンパクト台を持つ連続関数の微分の有限和である]
    コンパクト台を持つ任意の超関数$T\in\D'$と,その台の任意の開近傍$\supp T\subset U\osub\R^d$に対して,有限個のコンパクト台を持つ連続関数$f_1,\cdots,f_n\in C_c(\R^d)$と多重指数$\al_1,\cdots,\al_n\in\N^d$が存在して,$\supp f_k\subset U\;(k\in[n])$かつ
    \[T=\sum_{k\in[n]}\pp{^{\al_k}}{x^{\al_k}}f_k.\]
\end{corollary}
\begin{Proof}\mbox{}
    \begin{enumerate}[{Step}1]
        \item 一般に,任意の可微分関数$\eta\in C^\infty(\R^d)$と超関数$T\in\D'$に対して,Leibnitz則から次が成り立つ:
        \[\eta(\partial^\beta T)=\sum_{\al\le\beta}(-1)^{\abs{\al}}\frac{\beta!}{(\beta-\al)!\al!}\partial^{\beta-\al}((\partial^\al\eta)T).\]
        \item 定理から,ある$g\in C(\R^d)$と$\beta\in\N^d$とが存在して,$T=\partial^\beta g$.
        
        $\eta\in\D$であって,$\supp T$の近傍で$1$であるものであり,$\supp\eta\subset U$を満たすものが取れる.これについて,
        \[T=\eta(\partial^\beta g)=\sum_{\al\le\beta}c_{\al\beta}\partial^{\beta-\al}((\partial^\al\eta)g).\]
        が成り立ち,$(\partial^\al\eta)g\in C_c(\R^d)$かつ$\supp(\partial^\al\eta)g\subset \supp\eta\subset U$である.
    \end{enumerate}
\end{Proof}

\subsection{超関数と可微分関数の積}

\begin{tcolorbox}[colframe=ForestGreen, colback=ForestGreen!10!white,breakable,colbacktitle=ForestGreen!40!white,coltitle=black,fonttitle=\bfseries\sffamily,
title=]
    前節で,$T\in\D'$のコンパクト化$S=\eta T\;(\eta\in\D)$を定義したが,一般の$\eta\in C^\infty(\R^d)$について,この演算$\eta T\in\D'$は随伴を通じて定義される.
\end{tcolorbox}

\chapter{$\S'(\R^d)$上のFourier変換}

\begin{quotation}
    Fourier変換は
    $L^1(\R^d)$上から,自然な定義域として$\S'(\R^d)$まで延長することができ,この範囲で可逆であるが,
    Foureir変換像は関数とは限らなくなる.
    その定義は,$L^2(\R^d)$への延長の際に用いた空間$\S(\R^d)\subset L^1(\R^d)$との随伴により,延長する.
\end{quotation}

\section{緩増加超関数}

\begin{tcolorbox}[colframe=ForestGreen, colback=ForestGreen!10!white,breakable,colbacktitle=ForestGreen!40!white,coltitle=black,fonttitle=\bfseries\sffamily,
title=]
    緩増加分布がFourier変換が延長する超関数のクラスである.
    多くの局所可積分関数は緩増加である.
    このクラスまで辿り着けば,Fourier変換は逆変換を持つ.
\end{tcolorbox}

\subsection{緩増加分布の定義と特徴付け}

\begin{tcolorbox}[colframe=ForestGreen, colback=ForestGreen!10!white,breakable,colbacktitle=ForestGreen!40!white,coltitle=black,fonttitle=\bfseries\sffamily,
title=]
    $\D$上のノルムの要請に似た要請を$\S$上に課し,$\D$上で収束する列は$\S$の意味でも収束するようにする.
    $\D$上では同じことだが,$\S$のノルムには,多項式倍$\norm{x^\al(\partial^\beta\varphi)}_\infty$の値も考慮に入れることによる.
\end{tcolorbox}

\begin{definition}[tempered distribution]
    Schwartz急減少関数の空間$\S(\R^d)$を考える.
    \begin{enumerate}
        \item $\N\in\N$について,$\S$上のノルム
        \[\norm{\varphi}_N:=\sum_{\abs{\al},\abs{\beta}\le N}\sup_{x\in\R^d}\abs{x^\al(\partial^\beta\varphi)(x)},\qquad\varphi\in\S.\]
        を考える.
        \item $\{\varphi_n\}\subset\S$が$\S$上で収束するとは,任意の$N\in\N$に対して$\norm{\varphi_n-\varphi}_N\xrightarrow{n\to\infty}0$を満たすことをいう.
        \item 双対空間$\S'(\R^d)$の元を\textbf{緩増加分布}という.
    \end{enumerate}
\end{definition}

\begin{proposition}\label{prop-characterization-of-tempered-distribution}
    線型写像$T:\S\to\C$について,次は同値:
    \begin{enumerate}
        \item $T\in\S'$.
        \item ある$N\in\N,C>0$が存在して,$\forall_{\varphi\in\S}\;\abs{(T|\varphi)}\le C\norm{\varphi}_N$.
    \end{enumerate}
\end{proposition}
\begin{Proof}\mbox{}
    \begin{description}
        \item[(2)$\Rightarrow$(1)] $\exists_{N\in\N,C>0}\;\abs{(T|\varphi)}\le C\norm{\varphi}_N$であるから,任意の収束列$\{\varphi_n\}\subset\S$に対して,
        \[\abs{(T|\varphi_n-\varphi)}\le C\norm{\varphi_n-\varphi}_N\xrightarrow{n\to\infty}0.\]
        \item[(1)$\Rightarrow$(2)] (2)が成り立たないと仮定して矛盾を導く.特に,任意の$n\in\N$について,ある$\varphi_n\in\S$が存在して,$\abs{(T|\varphi_n)}>n\norm{\varphi_n}_n$を満たす.
        $\psi_n:=\frac{\varphi_n}{n\norm{\varphi_n}_n}$とおくと,$\abs{(T|\psi_n)}>1$を満たす.
        一方で,任意の$N\in\N$について,十分大きい$n\ge N$については,
        \[\norm{\psi_n}_N\le\norm{\psi_n}_n=\frac{1}{n}\xrightarrow{n\to\infty}0\]
        これは$T$が連続であることに矛盾.
    \end{description}
\end{Proof}

\subsection{緩増加分布は超関数である}

\begin{tcolorbox}[colframe=ForestGreen, colback=ForestGreen!10!white,breakable,colbacktitle=ForestGreen!40!white,coltitle=black,fonttitle=\bfseries\sffamily,
title=]
    単射$\D\mono\S$は稠密な埋め込みで,
    これにより$\S'\mono\D'$も単射である.
\end{tcolorbox}

\begin{proposition}[試験関数はSchwartz超関数の空間に稠密に埋め込める]\mbox{}
    \begin{enumerate}
        \item $\D\subset\S$は稠密である.
        \item 包含$\D\mono\S$は連続である.
    \end{enumerate}
\end{proposition}
\begin{Proof}\mbox{}
    \begin{enumerate}
        \item 任意の$\varphi\in\S$に対して,$\D$の列$\{\varphi_n\}\subset\D$が存在して$\varphi_n\to\varphi\;\In\S$を示せば良い.
        \begin{description}
            \item[方針] $\psi\in\D$を,$\Im\psi\subset[0,1]$であって立方体$[-1,1]^d$上で$\psi=1$を満たすものとする.
            $\psi_n(x):=\psi(x/n)$と定め,$\varphi_n:=\psi_n\varphi$とすると,$\varphi_n\in\D$である.
            このように構成した列$\{\varphi_n\}\subset\D$は$\varphi_n\to\varphi\;\In\S$を満たすことを示す.
            \item[証明] 一般に多重指数$\beta\in\N^d$と点$x\in\R^d$に対して,積の微分則より,
            \[(\partial^\beta\varphi_n)(x)-(\partial^\beta\varphi)(x)=\partial^\beta((\psi_n-1)\varphi)(x)=\sum_{\beta_1+\beta_2=\beta}c_{\beta_1\beta_2}\partial^{\beta_1}(\psi_n-1)(x)(\partial^\beta\varphi)(x).\]
            よって,$\al\in\N^d$も多重指数にして,
            \[\abs{x^\al\partial^\beta(\varphi_n-\varphi)(x)}=\sum_{\beta_1+\beta_2=\beta}c_{\beta_1\beta_2}\abs{\partial^{\beta_1}(\psi_n-1)(x)}\abs{x^\al(\partial^{\beta_2}\varphi)(x)}.\]
            \begin{enumerate}
                \item $\norm{\partial^{\beta_1}(\psi_n-1)}_\infty$は$n$について有界である.実際,$\norm{\psi_n-1}_\infty\le1$が成り立ち,$\beta_1\ne0$については,$\norm{\partial^{\beta_1}(\psi_n-1)}_\infty=n^{-\abs{\beta_1}}\norm{\partial^{\beta_1}\psi}_\infty$と表せる.
                \item $\varphi\in\S$の急減少性より,$\abs{x^\al(\partial^{\beta_2}\varphi)(x)}$は$x$が大きくなるにつれて小さくなる.
            \end{enumerate}
            以上2点より,右辺は$n$に依らずに小さくなる.一方で$\abs{x}<n$ならば,任意の多重指数$\beta_1\in\N^d$に対して$\partial^{\beta_1}(\psi_n-1)(x)=0$を満たすように作ったので,右辺は$0$である.
            以上より,右辺は$\sup_{x\in\R^d}$を取っても,$n\to\infty$で$0$に収束する.
        \end{description}
        \item 次の系の証明内で議論されるが,次の通りである.$\varphi_n\to\varphi\;\In\D$のとき,任意の$x^\al\partial^\beta\varphi_n$はコンパクト台を持つから,$\partial^\beta\varphi_n\to\partial^\beta\varphi$が$\R^d$上で一様収束することより,
        $\norm{\varphi_n-\varphi}_N\to0$.よって,$\S$の意味でも収束する.
    \end{enumerate}
\end{Proof}

\begin{corollary}[緩増加分布は超関数に埋め込まれる]
    制限が定める線型写像$\S'\to\D'$
    \[\xymatrix@R-2pc{
        \S'\ar[r]&\D'\\
        \rotatebox[origin=c]{90}{$\in$}&\rotatebox[origin=c]{90}{$\in$}\\
        T\ar@{|->}[r]&T|_{\D}
    }\]
    は単射である.
\end{corollary}
\begin{Proof}\mbox{}
    \begin{description}
        \item[well-definedness] まず,任意の$T\in\S'$に対して$T|_\D\in\D'$を示す.これは,汎関数$\D\to\C$として連続であることを示せばよい.
        
        任意に収束列$\{\varphi_n\}\subset\D$を取る.
        任意のコンパクト集合$\cup_{n\in\N}\supp\varphi_n\cup\supp\varphi\subset K$上で,任意階の導関数が一様収束するから,任意の$N\in\N$について$\norm{\varphi_n-\varphi}_N\xrightarrow{n\to\infty}0$.
        $x^\al$を乗じてもコンパクト台を持つことには変わらないから,
        $\S$でも$\varphi_n\to\varphi\;\In\S$である.
        よって,$(T|\varphi_n)\to(T|\varphi)$.
        \item[単射性の証明] $\D\subset\S$が稠密であるから,$T$が$\D$上で零ならば,$\S$上でも零である.
    \end{description}
\end{Proof}

\subsection{緩増加関数の例1:構成法}

\begin{tcolorbox}[colframe=ForestGreen, colback=ForestGreen!10!white,breakable,colbacktitle=ForestGreen!40!white,coltitle=black,fonttitle=\bfseries\sffamily,
title=]
    \begin{enumerate}
        \item 多項式増大な$L(\R^d)$の元は緩増加関数である.
        \item $L^p(\R^d)$の元も緩増加関数である.
        \item コンパクト台を持つ超関数は緩増加である:$\D'_c(\R^d)\mono\S'$.
    \end{enumerate}
\end{tcolorbox}

\begin{proposition}[裾が十分に軽い局所可積分関数は緩増加である]\mbox{}\label{prop-sufficient-condition-for-Borel-measure-to-be-tempered}
    \begin{enumerate}
        \item $f\in L^1_\loc(\R^d)$がある$N\in\N$について$fx^{-N}\in L^1(\R^d\setminus B)$,すなわち,
        \[\int_{\abs{x}\ge1}\frac{\abs{f(x)}}{\abs{x}^N}dx<\infty\]
        を満たすならば,$f\in\S'$である.
        \item Borel測度$\mu\in M(\R^d)_+$は局所有限であるとする.ある$N\in\N$が存在して
        \[\int_{\abs{x}\ge1}\frac{1}{\abs{x}^N}d\mu(x)<\infty\]
        を満たすならば,$\mu\in\S'$である.
    \end{enumerate}
\end{proposition}
\begin{Proof}\mbox{}
    \begin{enumerate}
        \item 任意の$\varphi\in\S$を取る.
        \begin{align*}
            \abs{(f|\varphi)}&\le\int_{\abs{x}\le1}\abs{f(x)}\abs{\varphi(x)}dx+\int_{\abs{x}\ge1}\frac{\abs{f(x)}}{\abs{x}^N}\abs{x}^N\abs{\varphi(x)}dx\le C\norm{\varphi}_N.
        \end{align*}
        最後の不等式評価は,第一項が有界で,第二項については$\abs{x}^N\abs{\varphi(x)}$が$\norm{\varphi}_N$で抑えられることとHolderの不等式による.
        \item まったく同様に,任意の$\varphi\in\S$に対して,
        \[\abs{(\mu|\varphi)}\le\int_{\abs{x<1}}\abs{\varphi}d\mu+\int_{\abs{x}\ge1}\frac{1}{\abs{x}^N}\abs{x}^N\abs{\varphi(x)}d\mu(x)\le C\norm{\varphi}_N.\]
        より.
    \end{enumerate}
\end{Proof}

\begin{definition}[function with at most polynomial growth]
    関数$f:\R^d\to\C$が\textbf{多項式増大}であるとは,ある$N\in\N,C>0$が存在して,
    任意の$x\in\R^d$に対して
    \[\abs{f(x)}\le C(1+\abs{x}^N)\]
    を満たすことをいう.
\end{definition}

\begin{corollary}[緩増加分布の重要なクラス2つ]\mbox{}
    \begin{enumerate}
        \item 多項式増大な可測関数は緩増加分布である.
        \item $L^p(\R^d)\;(p\in[1,\infty])$の元は緩増加分布である.
    \end{enumerate}
\end{corollary}
\begin{Proof}\mbox{}
    \begin{enumerate}
        \item 次数$n$の多項式増大関数$f$に対しては,$N\ge n+2$などと取っておけば,$\abs{x^{-N}f}$は$\R^d\setminus B$上可積分である.
        \item $f\in L^p(\R^d)$に対して,$\abs{x}^{-N}$を$N\ge 2p^*$と取れば$L^{p^*}(\R^d\setminus B)$の元であるから,Holderの不等式より,$fx^{-N}\in L^1(\R^d\setminus B)$.
    \end{enumerate}
\end{Proof}

\begin{example}[緩増加分布は微分について閉じているが,微分により多項式増大関数のクラスから飛び出ることがある]
    $f(x):=\sin(e^x)$は有界なので特に多項式増大であるが,$f'(x)=e^x\cos(e^x)$は多項式増大ではない.
    しかし,$\S'$の元ではある.
\end{example}

\begin{proposition}[コンパクト台を持つ超関数は緩増加分布]\label{prop-compactly-supported-distribution-is-tempered}
    $T\in\D'_c(\R^d)$の$\S$への延長$\o{T}|_{\S}$は$\S'$の元である.
\end{proposition}
\begin{Proof}
    任意の収束列$\{\varphi_n\}\subset\S$を取る.
    $(\o{T}|\varphi_n)\to(\o{T}|\varphi)$を示せばよい.
    $\supp T$の近傍で$\eta=1$な任意の元$\eta\in\D$に対して,$\eta\varphi_n\to\eta\varphi\;\In\D$である.
    よって,
    \[(\o{T}|\varphi_n)=(T|\eta\varphi_n)\xrightarrow{n\to\infty}(T|\eta\varphi)=(\o{T}|\varphi).\]
\end{Proof}

\subsection{緩増加分布の例2:局所可積分関数の緩増加性の判定}

\begin{example}[緩増加でない例]
    指数関数$f(x)=e^x$は局所可積分関数であるから超関数であるが,緩増加分布ではない.
\end{example}
\begin{Proof}\mbox{}
    \begin{description}
        \item[要点] $(f|-)$が$\S$上には連続線型汎関数を定めないことを示せば良い.
        \item[証明] $\varphi\in C^\infty(\R)_+$を$\varphi(x):=e^{-\frac{x}{2}}\;(x>1)$を満たす$\R_+$上に台を持つ関数とすると,$\varphi\in\S$であるが,$(f|\varphi)$は定まらない.
    \end{description}
\end{Proof}

\subsection{緩増加分布の例3:微分と多項式倍に関する閉性}

\begin{tcolorbox}[colframe=ForestGreen, colback=ForestGreen!10!white,breakable,colbacktitle=ForestGreen!40!white,coltitle=black,fonttitle=\bfseries\sffamily,
title=]
    $\S$が微分と多項式倍について閉じているために,$\S'$も閉じている.
    設計意図かのように感じるきれいな議論である.
    逆に,任意の緩増加分布は,ある多項式増大の連続関数の微分として得られるものである.
\end{tcolorbox}

\begin{proposition}[緩増加分布の微分・多項式倍は再び緩増加である]
    緩増加分布$T\in\S'$と多重指数$\al\in\N^d$について,
    \begin{enumerate}
        \item $x^\al T\in\S'$.
        \item $\partial^\al T\in\S'$.
    \end{enumerate}
\end{proposition}
\begin{Proof}
    $T$はある$N\in\N$について,$\S$のノルム$\norm{-}_N$について作用素ノルム$C>0$を持つとする.
    このとき,
    \begin{align*}
        \abs{(x^\al T|\varphi)}&=\abs{(T|x^\al\varphi)}\le C\norm{x^\al\varphi}_N\le C\norm{\varphi}_{N+\abs{\al}}.\\
        \abs{(\partial^\al T|\varphi)}&=\abs{(T|\partial^\al\varphi)}\le C\norm{\partial^\al\varphi}_N\le C\norm{\varphi}_{N+\abs{\al}}.
    \end{align*}
    から従う.
\end{Proof}

\begin{example}[$1/x$の主値超関数は緩増加である]
    $f(x)=x\log\abs{x}-x$は多項式増大関数であるから,特に緩増加分布.
    よって,その1階微分と2階微分
    \[f'(x)=\log\abs{x},\quad (T_{f'})'=\PV\paren{\frac{1}{x}}.\]
    も緩増加である.
\end{example}

\section{緩増加分布のFourier変換}

\begin{tcolorbox}[colframe=ForestGreen, colback=ForestGreen!10!white,breakable,colbacktitle=ForestGreen!40!white,coltitle=black,fonttitle=\bfseries\sffamily,
title=]
    次のようにして,$\F:L^1(\R^d)\to L^1(\R^d)$には全単射な延長$\F:\S'(\R^d)\to\S'(\R^d)$を持つ.
\end{tcolorbox}

\subsection{Fourier変換の随伴}

\begin{observation}[Fourier変換の随伴作用素]\label{observation-adjoint-of-Fourier-transform}
    任意の$f\in L^1(\R^d),\varphi\in\S$に対して,
    \[(\wh{f}|\varphi)=(f|\wt{\varphi}).\]
\end{observation}
\begin{Proof}
    Fubiniの定理より,
    \[(\wh{f}|\varphi)=\int_{\R^d}\wh{f}(y)\varphi(y)dy=\iint_{\R^d\times\R^d}f(x)\varphi(y)e^{-iy\cdot x}dxdy=\int_{\R^d}f(x)\wt{\varphi}(x)dx=(f|\wt{\varphi}).\]
\end{Proof}

\begin{remarks}[Fourier変換の延長における障碍]
    Fourier変換の随伴作用素は変わらないことが判ったから,$(\wt{T}|\varphi):=(T|\wt{\varphi})$と定めたいが,$\varphi\in\D$のFourier変換が$\wh{\varphi}\in\D$でもあるならば,$\varphi=0$が従うのであった\ref{prop-fourier-transform-of-functions-with-compact-support}.
\end{remarks}

\subsection{定義と全単射性}

\begin{definition}[Fourier transform of tempered distribution]
    緩増加分布$T,S\in\S'$に対して,
    \begin{enumerate}
        \item 写像$\wt{T}:\S\to\C$を
        \[(\wh{T}|\varphi):=(T|\wt{\varphi}),\qquad(\varphi\in\S)\]
        で定める.これを\textbf{Fourier変換}という.
        これは$L^1(\R^d)\mono\S(\R^d)$上の延長となっている\ref{observation-adjoint-of-Fourier-transform}.
        \item 写像$\check{S}:\S\to\C$を
        \[(\wc{S}|\varphi):=(S|\F^{-1}\varphi),\qquad(\varphi\in\S)\]
        で定める.これを\textbf{Fourier逆変換}という.
    \end{enumerate}
\end{definition}

\begin{proposition}[well-definedness]\mbox{}
    \begin{enumerate}
        \item $\S$上のFourier変換の連続性:Fourier変換$\F:\S\to\S$は連続である.
        \item 像も緩増加である:任意の緩増加分布$T\in\S'$のFourier変換は緩増加分布である:$\wt{T}\in\S'$.
    \end{enumerate}
\end{proposition}
\begin{Proof}\mbox{}
    \begin{enumerate}
        \item \begin{description}
            \item[一様ノルムに関する有界性] ある$C>0$が存在して,$\forall_{\varphi\in\S}\;\norm{\wh{\varphi}}_\infty\le C\norm{\varphi}_{d+1}$が成り立つ.
            
            実際,任意の$\xi\in\R^d$に対して,
            \begin{align*}
                \abs{\wh{\varphi}(\xi)}&\le\int_{\R^d}\abs{\varphi(x)}dx=\int_{\R^d}\frac{1}{1+\abs{x}^{d+1}}(1+\abs{x}^{d+1})\abs{\varphi(x)}dx\le\Norm{\frac{1}{1+\abs{x}^{d+1}}}_{L^1(\R^d)}\norm{\varphi}_{d+1}.
            \end{align*}
            \item[$\S$の一般の半ノルムについて減少的である] 任意の$N\in\N$と$\varphi\in\S$に対して,$\norm{\wh{\varphi}}_N\le C_2\norm{\varphi}_{2N+d+1}$.
            
            実際,任意の$\al,\beta\in\N^d$に対して,
            \begin{align*}
                \abs{\xi^\al(\partial^\beta\wh{\varphi})(\xi)}&=\abs{\xi^\al(\wh{x^\beta\varphi})(\xi)}=\abs{\wh{(\partial^\al(x^\beta\varphi))}(\xi)}\\
                &\le C\norm{\partial^\al(x^\beta\varphi)}_{d+1}\le C_1\norm{\varphi}_{\abs{\al}+\abs{\beta}+d+1}.
            \end{align*}
        \end{description}
        \item $\wh{T}(\varphi)=T(\F(\varphi))$としてたしかに$\S'$を終域とする写像である.
    \end{enumerate}
\end{Proof}

\begin{proposition}[全単射性]
    Fourier変換$\F:\S'\to\S'$は全単射であり,$S\mapsto\wc{S}$が逆を与える.
\end{proposition}
\begin{Proof}
    任意の$\varphi\in\S$に対して,
    \[(\wc{\wh{T}}|\varphi)=(\wh{T}|\F^{-1}\varphi)=(T|\F\F^{-1}\varphi)=(T|\varphi).\]
    逆もまったく同様.
\end{Proof}

\begin{proposition}[Fourier変換の2回適用は反転の$(2\pi)^d$倍である]
    緩増加分布$T\in\S'$に対して,
    \begin{enumerate}
        \item $\F^2T=(2\pi)^d\wt{T}$である.
        \item 特に,$\F^4=(2\pi)^{2d}$.
    \end{enumerate}
\end{proposition}
\begin{Proof}\mbox{}
    \begin{enumerate}
        \item 任意の$\varphi\in\S$に対して,$\S(\R^d)$上のFourier変換の反転公式\ref{thm-inversion-theorem-in-Rd}から,
        \[(\F^2T|\varphi)=(T|\F^2\varphi)=(T|(2\pi)^d\wt{\varphi})=(2\pi)^d(\wt{T}|\varphi).\]
        \item 明らか.
    \end{enumerate}
\end{Proof}

\subsection{Fourier変換と微分}

\begin{tcolorbox}[colframe=ForestGreen, colback=ForestGreen!10!white,breakable,colbacktitle=ForestGreen!40!white,coltitle=black,fonttitle=\bfseries\sffamily,
title=]
    任意の$d$変数多項式$P$について,
    \[\wh{P(D)f}=P\wh{f},\qquad\wh{Pf}=P(-D)\wh{f}\]
    と表せる\cite{Rudin-FunctionalAnalysis}Th'm 7.4.
\end{tcolorbox}

\begin{proposition}
    緩増加分布$T\in\S'$と多重指数$\al\in\N^d$について,
    \begin{enumerate}
        \item 微分は多項式倍に移る:$\wh{\partial^\al T}=i^{\abs{\al}}x^\al\wh{T}$.
        \item 多項式倍は微分に移る:$\wh{x^\al T}=i^{\abs{\al}}\partial^\al\wh{T}$.
    \end{enumerate}
\end{proposition}
\begin{Proof}
    任意の$\varphi\in\S$に対して,
    \begin{enumerate}
        \item $\wh{x^\al f}(\xi)=i^{\abs{\al}}(\partial^\al\wh{f})(\xi)$より,
        \[(\wh{\partial^\al T}|\varphi)=(\partial^\al T|\wh{\varphi})=(-1)^{\abs{\al}}(T|\partial^\al\wh{\varphi})=i^{\abs{\al}}(T|\wh{x^\al\varphi})=i^{\abs{\al}}(x^\al\wh{T}|\varphi).\]
        \item $\wh{\partial^\al\varphi}=i^{\abs{\al}}x^\al\wh{\varphi}$より,
        \[(\wh{x^\al T}|\varphi)=(T|x^\al\wh{\varphi})=i^{-\abs{\al}}(T|\wh{\partial^\al\varphi})=i^{-\abs{\al}}(\wh{T}|\partial^\al\varphi)=i^{\abs{\al}}(\partial^\al\wh{T}|\varphi).\]
    \end{enumerate}
\end{Proof}

\subsection{Fourier変換の連続性}

\begin{tcolorbox}[colframe=ForestGreen, colback=ForestGreen!10!white,breakable,colbacktitle=ForestGreen!40!white,coltitle=black,fonttitle=\bfseries\sffamily,
title=]
    $\S'$にはペアリングを通じた$w^*$-位相を入れる.
\end{tcolorbox}

\begin{proposition}
    $\F:\S'\to\S'$は位相同型である.
\end{proposition}
\begin{Proof}
    $\{T_n\}\subset\S'$を収束列とする.すなわち,任意の$\varphi\in\S$について,$(T_n|\varphi)\to(T|\varphi)$.
    このとき,
    \[(\wh{T_n}|\varphi)=(T_n|\wh{\varphi})\to(T|\wh{\varphi})=(\wh{T}|\varphi).\]
\end{Proof}

\section{Fourier変換の例}

\begin{remarks}[計算テクニックをまとめてみる]
    $f\in\S'(\R^d)$のFourier変換を求めたい.
    \begin{enumerate}
        \item Fourier変換が既知の関数の積分変換としての表示が得られたら,Fubiniの定理からそこを起点として計算できる.Bochnerのsubordinationを用いたPoisson核のFourier変換\ref{thm-Fourier-transform-of-Poisson-kernel-on-Rd}や$f(x)=\abs{x}^\al$の計算など.
        \item 微分をして満たすべきODEを求める.Gauss核の計算\ref{thm-Fourier-transform-of-Gaussian-kernel-on-R}など.
        \item 複素積分とみて,留数計算・解析接続などを用いる.$\R^d$上のGauss核の計算\ref{thm-Fourier-transform-of-Gaussian-density},$f(x)=e^{is\abs{x}^2}$のFourier変換など.
        \item Heaviside関数は,超関数微分方程式$H'=\delta$にFourier変換を繰り返し,Gauss核における値を初期条件と併せてこれを解いた.
    \end{enumerate}
\end{remarks}

\subsection{例1:デルタ測度の変換}

\begin{tcolorbox}[colframe=ForestGreen, colback=ForestGreen!10!white,breakable,colbacktitle=ForestGreen!40!white,coltitle=black,fonttitle=\bfseries\sffamily,
title=]
    Fourier変換,Fourier-Plancherel変換,Fourier-Stieltjes変換の延長となっており,新たにDelta分布や多項式増加関数のFourier変換も計算可能になっている.
    $L^\infty(\R)$の像を擬測度ともいう.
\end{tcolorbox}

\begin{example}[Delta分布は定数関数に移る]\mbox{}
    \begin{enumerate}
        \item $\wh{\delta}=1$.
        \item $\wh{1}=(2\pi)^d\delta$.
        \item 線型汎関数$\delta_a(\varphi):=\varphi(a)\;(a\in\R^d)$はコンパクト台を持つ超関数なので緩増加分布である\ref{prop-compactly-supported-distribution-is-tempered}.
        そのFourier変換は可積分関数で,$\wh{\delta_a}(\xi)=\exp(-ia\cdot\xi)\;(\xi\in\R^d)$.
        \item $\F(e^{-ia\cdot x})=(2\pi)^d\delta_{-a}$.
    \end{enumerate}
\end{example}
\begin{Proof}\mbox{}
    \begin{enumerate}
        \item 任意の$\varphi\in\S$に対して,
        \[(\wh{\delta}|\varphi)=(\delta|\wh{\varphi})=\wh{\varphi}(0)=\int_{\R^d}\varphi(x)dx=(1|\varphi).\]
        \item $\wh{1}=\F^2\delta=(2\pi)^d\wt{\delta}=(2\pi)^d\delta$.
        \item 全く同様に,任意の$\varphi\in\S$に対して,
        \[(\wh{\delta_a}|\varphi)=(\delta_a|\wh{\varphi})=\int_{\R^d}\varphi(x)e^{-ix\cdot a}dx=(e^{-ia\cdot-}|\varphi).\]
        \item $\wh{e^{-ia\cdot x}}=\F^2\delta_a=(2\pi)^d\wt{\delta_a}=(2\pi)^d\delta_{-a}$.
    \end{enumerate}
\end{Proof}

\begin{example}[Delta分布の無限和]
    \[\sum_{n\in\Z}\delta_n\]
    は多項式増加な局所有限Borel測度であるから,緩増加分布である.このFourier変換は,
    \[\F\paren{\sum_{n\in\Z}\delta_n}=2\pi\sum_{k\in\Z}\delta_{2\pi k},\quad\F\paren{\sum_{k\in\Z}\delta_{2\pi k}}=\sum_{n\in\Z}\delta_n.\]
\end{example}
\begin{Proof}
    まずPoissonの和公式より,$\varphi\in\S$の周期化$\Pe[\varphi]$の$t=0$での値はFourier展開での$t=0$と等しいから,
    \[\sum_{n\in\Z}\wh{\varphi}(n)=2\pi\sum_{k\in\Z}\varphi(2\pi k),\qquad\varphi\in\S.\]
    これより,次のように計算できる:
    \begin{align*}
        \paren{\F\paren{\sum_{n\in\Z}\delta_n}\;\middle|\varphi}&=\paren{\sum_{n\in\Z}\delta_n\;\middle|\;\wh{\varphi}}=\sum_{n\in\Z}\wh{\varphi}(n)\\
        &=2\pi\sum_{k\in\Z}\varphi(2\pi k)=\paren{2\pi\sum_{k\in\Z}\delta_{2\pi k}\;\middle|\;\varphi}.
    \end{align*}
\end{Proof}

\begin{example}[多項式はDelta分布の微分に移る]
    多重指数$\al\in\N^d$について,
    \begin{enumerate}
        \item $\wh{x^\al}=i^{\abs{\al}}\partial^\al\wh{1}=(2\pi)^di^{\abs{\al}}\partial^\al\delta$.
        \item $\wh{\partial^\al\delta}(\xi)=i^{\abs{\al}}\xi^\al$.
    \end{enumerate}
\end{example}

\subsection{例2:一般の$L^p$関数の変換}

\begin{proposition}[整合性]\mbox{}
    \begin{enumerate}
        \item $L^1(\R^d)$上のFourier変換の延長である:$f\in L^1(\R^d)$について,$\wh{T_f}=\wh{f}$.
        \item $L^2(\R^d)$上のFourier-Plancherel変換の延長である:$f\in L^2(\R^d)$について,$\wh{T_f}=T_{\F f}$である.
    \end{enumerate}
\end{proposition}
\begin{Proof}\mbox{}
    \begin{enumerate}
        \item 任意の$\varphi\in\S$に対して,Fubiniの定理により
        \[(\wh{f}|\varphi)=\iint_{\R^d\times\R^d}f(x)e^{-ix\cdot \xi}\varphi(\xi)dxd\xi=(f|\wh{\varphi}).\]
        であることから従う.
        \item $f_n:=f1_{[-n,n]^d}$とすると,$f_n\to f\;\In L^2(\R^d)$かつ$f_n\to f\;\In L^1(\R^d)$である.Fourier-Plancherel変換の連続性から,
        $\F f_n\to\F f\;\In L^2(\R^d)$でもある.
        Cauchy-Schwarzの不等式より,内積は$L^2(\R^d)$-収束に関して連続であることに注意すれば,任意の$\varphi\in\S$に対して,
        \[(\wh{T_f}|\varphi)=(f|\wh{\varphi})=\lim_{n\to\infty}(f_n|\wh{\varphi})=\lim_{n\to\infty}(\wh{f_n}|\varphi)=(\F f|\varphi).\]
    \end{enumerate}
\end{Proof}

\begin{definition}[pseudo-measure]
    $L^p(\R^d)\subset\S'(\R^d)$上のFourier変換を考えることが出来ている.
    \begin{enumerate}
        \item $\F(L^p(\wh{\R}))\;(p\in[1,\infty])$は再びBanach空間であり,双対関係$\F(L^p(\wh{\R}))\simeq(\F(L^{p^*}(\wh{\R})))^*\;{(p\in(1,\infty])}$も引き継がれる.
        \item 任意の$\F(L^p(\wh{\R}))\;(p<\infty)$について,$\S(\R)$は稠密部分空間である.したがって,この上の連続線型汎関数(例えばRadon測度)は全て緩増加分布と見做せる.
        \item 一方で,$\F(L^\infty(\wh(\R)))$の双対空間は$\F(L^1(\wh{\R}))$ではない.
        $\F(L^\infty(\wh{\R}))$の元を\textbf{擬測度}という.
        \item Bore測度は有界関数のFourier変換であるから,埋め込み$M(\wh{\R})\mono\F(L^\infty)$が存在する.
    \end{enumerate}
\end{definition}

\begin{example}
    \[f(x):=e^{is\abs{x}^2},\qquad 0\ne s\in\R.\]
    は$f\in L^\infty(\R)$であるが,可積分ではない.
    \[\F[e^{is\abs{x}^2}]=-\]
\end{example}
\begin{Proof}
    定義から直接計算することが決してできない例である\cite{Strichartz03-Distribution}4.2.3.
    \begin{description}
        \item[予想] Gauss核のFourier変換\ref{thm-Fourier-transform-of-Gaussian-density}より,
        \[\F[e^{-t\abs{x}^2}](\xi)=\paren{\frac{\pi}{t}}^{d/2}e^{-\frac{\abs{\xi}^2}{4t}}.\]
        これに$t=-is$への置換を考えて,
        \[\F[e^{is\abs{x}^2}](\xi)=\paren{\frac{i\pi}{s}}^{d/2}\exp\paren{-\frac{i\abs{\xi}^2}{4s}}.\]
        を予想したい.
        \item[検証] 複素変数への置換というのは,$\varphi\in\S$について
        \[(e^{-t\abs{x}^2}|\wh{\varphi})=\int_{\R^d}e^{-t\abs{x}^2}\wh{\varphi}(x)dx=\paren{\frac{\pi}{t}}^{d/2}\int_{\R^d}e^{-\frac{\abs{\xi}^2}{4t}}\varphi(x)dx=(\F[e^{-t\abs{x}^2}]|\varphi).\]
        左辺の積分が$t>0$から一般の$\Re z=0$に解析接続できるかどうかの問題に等しいから,このことを示せばよい.
        そもそも,左辺の積分は$\Re z>0$の場合には定義できており(収束する),複素微分可能であるから正則である.
        そして,$s\ne0$の場合に限り,$\ep+is\searrow is$の極限も存在することが示せるから,上のFourier変換の結果を得る.
    \end{description}
\end{Proof}

\subsection{例3:符号関数と$1/x$の主値超関数のFourier変換}

\begin{tcolorbox}[colframe=ForestGreen, colback=ForestGreen!10!white,breakable,colbacktitle=ForestGreen!40!white,coltitle=black,fonttitle=\bfseries\sffamily,
title=]
    \[\wh{H}=\pi\delta-i\PV\paren{\frac{1}{x}}.\]
    の第1項を含めた$\pi\delta-iQ$は$x\to\infty$にて一切減衰しないという特異性から来るもので,
    第2項の$-ip$が原点での特異性がもたらす影響だと捉えられる.
    $p$は任意の$\R\setminus(-1,1)$上で$1/x$に等しい$C^\infty$-関数としてこの分解が成り立つ.
\end{tcolorbox}

\begin{example}[Heaviside関数と$1/x$の主値超関数]\mbox{}
    \begin{enumerate}
        \item Heaviside関数$H:=1_{[0,\infty]}$のFourier変換はDelta分布と主値超関数の線型和である:
        \[\wh{H}=\pi\delta-i\PV\paren{\frac{1}{x}}.\]
        \item 特に,$\sgn(x):=2H-1$について,
        \[\wh{\sgn}=-2i\PV\paren{\frac{1}{x}},\quad \wh{\PV\paren{\frac{1}{x}}}=i\pi\sgn.\]
        \item 主値超関数のFourier変換はHeaviside関数の反転の平行移動である:
        \[\wh{\PV\paren{\frac{1}{x}}}=\begin{cases}
            -\pi i&x>0,\\
            \pi i&x\le 0.
        \end{cases}\]
    \end{enumerate}
\end{example}
\begin{Proof}\mbox{}
    \begin{enumerate}
        \item 
        \begin{description}
            \item[近似列を取る] $H_\ep(x):=e^{-\ep x}H(x)=e^{-\ep x}1_{\R_+}$と定めると,
            きわめて減少速度が速いために
            $\{H_\ep\}\subset L^1(\R^d)$で,Lebesgueの優収束定理から$\forall_{\varphi\in\S}\;(H_\ep|\varphi)\to(H|\varphi)$より,$H_\ep\to H\;\In\S'$である.
            すると,
            Fourier変換の連続性より,$\wh{H_\ep}\to\wh{H}\;\In\S'$でもある.
            \item[方針] Heaviside関数には緩増加分布として$H+\wt{H}=1$の関係式があるから,これに注目する.
            実際,各点では$x\ne0$ならば$\wt{H}(x)=H(-x)=1-H(x)$であるため,超関数としては$H+\wt{H}=1$を得る.

            この超関数方程式は簡単に両辺をFourier変換出来て,
            $\wh{H}+\wh{\wt{H}}=2\pi\delta$を得る.$\wh{H}$を知るには,あとは$\wh{H}-\wh{\wt{H}}$の値を知ればよい.
            これは,近似列
            \[\wh{H_\ep}-\wh{\wt{H_\ep}}\to \wh{H}-\wh{\wt{H}}\;\In\S'\]
            を考えることにより計算できる可能性がある.
            \item[$\wh{H_\ep}-\wh{\wt{H_\ep}}$は計算できる] 
            \[\wh{H_\ep}-\wh{\wt{H_\ep}}=\frac{1}{\ep+i\xi}-\frac{1}{\ep-i\xi}=\frac{-2i\xi}{\ep^2+\xi^2}.\]
            この$\S'$における収束先は,$\wh{H_\ep}-\wh{\wt{H_\ep}}$が偶関数であるために
            \[\int^r_{-r}\frac{\xi}{\ep^2+\xi^2}\varphi(0)d\xi=0.\]
            を満たすことに注意すれば,
            任意の$\varphi\in\S$について
            \begin{align*}
                (-2i)^{-1}((\wh{H_\ep}-\wh{\wt{H_\ep}})|\varphi)&=\int_\R\frac{\xi}{\ep^2+\xi^2}\varphi(\xi)d\xi\\
                &=\int_{\abs{\xi}\ge r}\frac{\xi}{\ep^2+\xi^2}\varphi(\xi)d\xi+\int^r_{-r}\frac{\xi}{\ep^2+\xi^2}(\varphi(\xi)-\varphi(0))d\xi\\
                &\xrightarrow{\ep\searrow0}\int_{\abs{\xi}\ge r}\frac{\varphi(\xi)}{\xi}d\xi+\int^r_{-r}\frac{\varphi(\xi)-\varphi(0)}{\xi}d\xi\\
                &\xrightarrow{r\searrow0}\paren{\PV\paren{\frac{1}{x}}\middle|\varphi}+0.
            \end{align*}
            と計算できる.まず,$\ep\searrow0$の極限は,第一項は$\varphi$の急減少性から$\abs{\xi}\ge\ep$において可積分であるため,第二項は
            平均値の定理から
            \[\sup_{\abs{\xi}\le r}\Abs{\frac{\varphi(\xi)-\varphi(0)}{\xi}}\le\sup_{\abs{\xi}\le r}\abs{\varphi'(\xi)}\le\norm{\varphi}_1\]
            と評価できることより,有界であるためである.
            これにより,第二項はそのまま$r\searrow0$の極限で$0$に収束する.
            \item[結論] 以上の計算より,
            $\wh{H}-\wh{\wt{H}}=-2i\PV\paren{\frac{1}{x}}$.故に,
            \[\wh{H}=\frac{1}{2}\Paren{(\wh{H}+\wh{\wt{H}})+(\wh{H}-\wh{\wt{H}})}=\frac{1}{2}\Paren{2\pi\delta-2i\PV\paren{\frac{1}{x}}}=\pi\delta-i(\log\abs{x})'.\]
        \end{description}
        \item 前述の関係式
        \[\PV\paren{\frac{1}{x}}=-\pi i\delta+i\wh{H}\]
        のFourier変換を考えると,
        \[\wh{\PV(1/x)}=-\pi i+2\pi i\wt{H}=\begin{cases}
            -\pi i&x>0,\\
            \pi i&x\le 0.
        \end{cases}\]
    \end{enumerate}
\end{Proof}
\begin{remark}[素直な計算は悉く失敗する]\mbox{}
    \begin{enumerate}
        \item 素直な計算
        \[\wh{H}(\xi)=\int^\infty_0e^{-i\xi x}dx=\SQuare{-\frac{e^{-i\xi x}}{i\xi}}^\infty_0\]
        は,$e^{-i\xi x}$は$x\to\infty$の極限で収束しないので,この道は通れない.
        もっと減少の強い$e^{-\ep x}$を乗じた$H_\ep(x):=e^{-\ep x}H(x)$は
        超関数微分の随伴関係より,
        \[\wh{H_\ep}(\xi)=\int_{\R}e^{-i\xi\cdot x}e^{-\ep x}H(x)dx=\int_\R\frac{e^{-x(i\xi+\ep)}}{\ep+i\xi}\delta(x)dx=\frac{1}{\ep+i\xi}.\]
        と計算可能であるから,この極限$\ep\to0$としてとらえたい.
        \item しかし,$\wh{H_\ep}\to\wh{H}$は$\S$の意味であるから,単純に右辺の各点収束極限を考えても正しくない.
        そこで,$\varphi\in\S$に作用させて,
        \[(\wh{H_\ep}|\varphi)=\int_\R\frac{\varphi(\xi)}{\ep+i\xi}d\xi\]
        を得るが,$\ep\to0$の極限で右辺が可積分のままであるかは分からない.
        実際,今回は主値積分として(たまたま)値が収束するだけであるから,そもそもLebesgue収束定理を使う場所ではない.
        証明では,$(\wh{H_\ep}-\wh{\wt{H_\ep}}|\varphi)$を考えることで,回避している.
        その回避法は,微分の定義式$\frac{\varphi(\xi)-\varphi(0)}{\xi}$を積分の中に作って,平均値の定理によって
        \[\sup_{\abs{\xi}\le r}\Abs{\frac{\varphi(\xi)-\varphi(0)}{\xi}}\le\sup_{\abs{\xi}\le r}\abs{\varphi'(\xi)}\le\norm{\varphi}_1\]
        と評価することによる.
    \end{enumerate}
\end{remark}
\begin{Proof}[\textbf{\underline{[別証明]}}]
    次は$1/x$の主値超関数が満たす超関数方程式に注目する.
    \begin{enumerate}[{Step}1]
        \item 等式$H'=\delta$をFourier変換すると$ix\wh{H}=1$である.
        主値超関数は$x\PV\paren{\frac{1}{x}}=1$を満たす\ref{exp-principal-value-as-distribution}から,辺々を引くと
        \[x\paren{i\wh{H}-\PV\paren{\frac{1}{x}}}=0.\]
        再びFourier変換すると,
        \[\paren{2\pi i\wt{H}-\wh{\PV\paren{\frac{1}{x}}}}'=0.\]
        という超関数微分方程式を得る.これは1階なので簡単に解ける.命題\ref{prop-characterization-of-constants-through-distributional-derivative}より,
        \[2\pi i\wt{H}-\wh{\PV\paren{\frac{1}{x}}}=c.\]
        これをFourier変換して,反転$\wt{}$を考えることで,
        \[2\pi i\wh{H}-2\pi\PV\paren{\frac{1}{x}}=2\pi c\delta\quad\Leftrightarrow\quad\wh{H}=-ic\delta-i\PV\paren{\frac{1}{x}}.\]
        よって,あとは$c\in\C$を求めれば良い.
        \item Gauss核$G(x)=\frac{1}{\sqrt{2\pi}}e^{-\frac{x^2}{2}}$
        での値を考える.すると,
        \[(\wh{H}|G)=(H|\wh{G})=\int^\infty_0e^{-\frac{x^2}{2}}dx=\sqrt{\frac{\pi}{2}}.\]
        かつ,$G$は偶関数なので,
        \[\paren{\PV\paren{\frac{1}{x}}\middle|G}=0.\]
        以上より,
        \[\sqrt{\frac{\pi}{2}}=-ic\frac{1}{\sqrt{2\pi}}\qquad\therefore c=i\pi.\]
    \end{enumerate}
\end{Proof}

\subsection{例4:漸近挙動に向けて}

\begin{tcolorbox}[colframe=ForestGreen, colback=ForestGreen!10!white,breakable,colbacktitle=ForestGreen!40!white,coltitle=black,fonttitle=\bfseries\sffamily,
title=]
    \begin{enumerate}
        \item $f(x)=\abs{x}^m$のFourier変換は$\wh{f}(x)=O(\abs{x}^{-(m+1)})\;(\abs{x}\to\infty)$のオーダーで減衰する.
        \item $\al\in(-d,0)$の範囲において,$f(x)=(\abs{x}^{\al})\;(\abs{x}\to\infty)$ならば,$\wh{f}(x)=(\abs{x}^{-\al-d})\;(\abs{x}\to\infty)$となる.
    \end{enumerate}
    (1)もたしかに$m+(-(m+1))=-1$となっている.
\end{tcolorbox}

\begin{example}[絶対値の処理]\mbox{}
    \begin{enumerate}
        \item 局所可積分関数$f(x)=\abs{x}^m\in L^1_\loc(\R)$のFourier変換は
        \[\F(\abs{x}^m)=\begin{cases}
            -2i^{m+1}(\log\abs{x})^{(m+1)}&m\text{は奇数},\\
            (2\pi)^di^{m}\delta^{(m)}&m\text{は偶数}.
        \end{cases}\]
        $m$が奇数の際,$\F(\abs{x}^m)$は原点以外で関数$\frac{2i^{m+1}m!}{x^{m+1}}$に等しい.
        \item 局所可積分関数$f(x)=x^m1_{\R_+}=x^mH(x)\in L^1_\loc(\R)$のFourier変換は,
        \[\F(x^mH(x))=i^m\pi\delta^{(m)}-i^{m+1}(\log\abs{x})^{(m+1)},\qquad(m\in\N).\]
        この超関数は原点以外では関数$\frac{(-i)^{m+1}m!}{x^{m+1}}$に等しい.
    \end{enumerate}
\end{example}
\begin{Proof}\mbox{}
    \begin{enumerate}
        \item 偶数の場合は$f(x)=x^m$であるため.奇数の場合は$f(x)=x^m\sgn x$であるが,
        $\sgn x=2H-1$の関係式に注意すれば,
        \[\wh{\sgn x}=2(\pi\delta-i(\log\abs{x})')-(2\pi)^1\delta=-2i(\log\abs{x})'\]
        であるから,
        \[\F(\abs{x}^m)=i^m(\wh{\sgn x})^{(m)}=-2i^{m+1}(\log\abs{x})^{(m+1)}.\]
        \item 次のように計算できる:
        \[\F(x^mH(x))=i^m(\wh{H})^{(m)}=i^m\pi\delta^{(m)}-i^{m+1}(\log\abs{x})^{(m+1)}.\]
    \end{enumerate}
\end{Proof}

\begin{example}[$d$次元での計算]\label{exp-Fourier-transform-of-absxalpha}
    \[f(x):=\abs{x}^{\al},\qquad-d<\Re\al<0,x\in\R^d.\]
    は$\Re\al<0$より減少関数である.
    $-d<\Re\al$であるから局所可積分であり,緩増加である\ref{prop-sufficient-condition-for-Borel-measure-to-be-tempered}.
    原点または無限遠点で特異性を持つため,この形の関数は決して可積分ではない.
    Fourier変換は,
    \[\F[\abs{x}^{\al}](\xi)=\frac{\pi^{d/2}2^{\al+d}\Gamma\paren{\frac{\al}{2}+\frac{d}{2}}}{\Gamma\paren{-\frac{\al}{2}}}\abs{\xi}^{-\al-d}.\]
    が与える.再び$-\al-d\in(-d,0)$であることに注意.
\end{example}
\begin{Proof}\mbox{}
    \begin{enumerate}[{Step}1]
        \item Poisson核のFourier変換\ref{thm-Fourier-transform-of-Poisson-kernel-on-Rd}を求める際に用いたBochner's method of subordinationによる.
        まず,Gauss核のLaplace変換としての表示を考える.
        これは,いま$\al<0$としたから,次のGamma関数の計算式によって与えられる(Gamma関数は有理型関数に延長されるが,積分が収束するのは右半平面においてである).
        $t=s\abs{x}^2$の変数変換を考えることで,
        \begin{align*}
            \int^\infty_0s^{-\frac{\al}{2}-1}e^{-s\abs{x}^2}ds&=\int^\infty_0\paren{\frac{t}{\abs{x}^2}}^{-\frac{\al}{2}-1}e^{-t}\frac{dt}{\abs{x}^2}\\
            &=\abs{x}^\al\int^\infty_0t^{-\frac{\al}{2}-1}e^{-t}dt=\abs{x}^\al\Gamma\paren{-\frac{\al}{2}}.
        \end{align*}
        を得る.
        \item 上の式にFubiniの定理を用いると,
        置換$t:=\frac{\abs{\xi}^2}{4s}$の計算に注意力を使うが,
        次のように計算できる:
        \begin{align*}
            \F[\abs{x}^{\al}](\xi)&=\int_{\R^d}\frac{1}{\Gamma\paren{-\frac{\al}{2}}}\int^\infty_0s^{-\frac{\al}{2}-1}e^{-s\abs{x}^2}dse^{-ix\cdot\xi}d\xi\\
            &=\frac{1}{\Gamma\paren{-\frac{\al}{2}}}\int^\infty_0s^{-\frac{\al}{2}-1}\F[e^{-s\abs{x}^2}](\xi)ds\\
            &=\frac{1}{\Gamma\paren{-\frac{\al}{2}}}\int^\infty_0s^{-\frac{\al}{2}-1}\paren{\frac{\pi}{s}}^{d/2}e^{-\frac{\abs{\xi}^2}{4s}}ds\\
            &=\frac{\pi^{d/2}}{\Gamma\paren{-\frac{\al}{2}}}\int^\infty_0s^{-\frac{\al}{2}-\frac{d}{2}-1}e^{-\frac{\abs{\xi}^2}{4s}}ds\\
            &=\frac{\pi^{d/2}}{\Gamma\paren{-\frac{\al}{2}}}\int^\infty_0\paren{\frac{\abs{\xi}^2}{4t}}^{-\frac{\al}{2}-\frac{d}{2}+1}e^{-t}\frac{4}{\abs{\xi}^2}dt\\
            &=\frac{\pi^{d/2}2^{\al+d}}{\Gamma\paren{-\frac{\al}{2}}\abs{\xi}^{\al+d}}\int^\infty_0t^{\frac{\al}{2}+\frac{d}{2}-1}e^{-t}dt=\frac{\pi^{d/2}2^{\al+d}\Gamma\paren{\frac{\al}{2}+\frac{d}{2}}}{\Gamma\paren{-\frac{\al}{2}}}\abs{\xi}^{-\al-d}.
        \end{align*}
    \end{enumerate}
\end{Proof}

\subsection{例5:Laplacianの基本解}

\begin{tcolorbox}[colframe=ForestGreen, colback=ForestGreen!10!white,breakable,colbacktitle=ForestGreen!40!white,coltitle=black,fonttitle=\bfseries\sffamily,
title=]
    $\delta$も,そのFourier変換も,その微分とFourier変換も調べた.
    $d\ge2$の場合について,これを引き続き調べる.
\end{tcolorbox}

\begin{theorem}
    $d\ge3$のとき,$\Lap$の基本解は,局所可積分関数
    \[E(x)=-\frac{1}{c_{d-2}}\frac{1}{\abs{x}^{d-2}},\qquad x\in\R^d,c_\al:=\frac{2^{d-\al}\pi^{d/2}\Gamma\paren{\frac{d}{2}-\frac{\al}{2}}}{\Gamma\paren{\frac{\al}{2}}}.\]
    が与える.
\end{theorem}
\begin{Proof}
    $\F[\Lap f]=-\abs{\xi}^2\wh{f}$に注意すれば,
    上の例により,
    \[\F\Square{\Lap\paren{\frac{1}{c_{d-2}}\frac{1}{\abs{x}^{d-2}}}}(\xi)=\frac{\abs{\xi}^2c_{d-2}\abs{\xi}^{-2}}{c_{d-2}}=1.\]
    であるため.
\end{Proof}
\begin{remark}
    単位球の体積が
    \[\om_n=\frac{\pi^{n/2}}{\frac{n}{2}\Gamma\paren{\frac{n}{2}}}.\]
    であるため,規格化定数は
    \[c_{d-2}=\frac{\pi^{d/2}2^2\Gamma(1)}{\Gamma\paren{\frac{d}{2}-1}}=\frac{\pi^{d/2}}{\Gamma\paren{\frac{d}{2}}}2^2\frac{d-2}{2}=\om_d4\frac{d}{2}\frac{d-2}{2}=\om_d(d-2)d.\]
    と表せる.
\end{remark}

\begin{proposition}[Liouvilleの定理の一般化]
    $T\in\S'$が$\Lap T=0$を満たすならば,$T$は多項式である.特に,$\R^d$上の調和関数が多項式増大ならば,多項式である.
\end{proposition}

\subsection{例6:対数関数のFourier変換}

\begin{definition}
    \[\gamma:=-\Gamma'(1)=\lim_{n\to\infty}\paren{1+\frac{1}{2}+\cdots+\frac{1}{n}-\log n}.\]
    を\textbf{Euler定数}という.
\end{definition}

\begin{proposition}
    \[\F\Square{\PV\frac{1}{\abs{x}}}=2\Gamma'(1)-2\log\abs{\xi},\quad\F[\log\abs{x}](\xi)=-2\pi\gamma\delta(\xi)-\pi\PV\frac{1}{\abs{\xi}}.\]
\end{proposition}

\begin{proposition}
    $f(x)=\log\abs{x}\;(x\in\R^2)$について,
    \[\frac{1}{2\pi}(\F(\log\abs{x})|\varphi)=-\int_{\abs{x}\le1}\frac{\varphi(x)-\varphi(0)}{\abs{x}^2}dx-\int_{\abs{x}>1}\frac{\varphi(x)}{\abs{x}^2}dx+2\pi(\Gamma'(1)+\log 2)\varphi(0),\qquad\varphi\in\S.\]
\end{proposition}

\section{Fourier変換の漸近挙動}

\begin{tcolorbox}[colframe=ForestGreen, colback=ForestGreen!10!white,breakable,colbacktitle=ForestGreen!40!white,coltitle=black,fonttitle=\bfseries\sffamily,
title=]
    原点で特異性を持つ関数$f\in L(\R)$に対して,$\wh{f}$の無限大での漸近挙動を知ることが出来る.
\end{tcolorbox}

\begin{example}
    $f(x):=e^{-\abs{x}}\in L^1(\R)$について,
    \[\wh{f}(\xi)=\frac{2}{\xi^2}-\frac{2}{\xi^4}+o(\abs{\xi}^{-5})\quad(\abs{\xi}\to\infty).\]
    実際,$\wh{f}(\xi)=\frac{2}{1+\xi^2}$であることに注意.
\end{example}
\begin{Proof}\mbox{}
    \begin{enumerate}[{Step}1]
        \item $f$の原点での漸近展開
        \[e^{-\abs{x}}=\sum_{n\in\N}(-1)^n\frac{\abs{x}^n}{n!}\]
        を考える.原点で特異性を持つのは奇数次の項で,例えば最初の2項は
        \[g(x):=-\abs{x}-\frac{\abs{x}^3}{3!}\]
        である.残ったのは$f-g\in C^4(\R)$.
        4階微分$(f-g)^{(4)}\in C(\R)$は連続であるから,もう一度(例えば超関数の意味で)微分が出来て,$(f-g)^{(5)}\in L^1_\loc(\R)$と表すとしよう.
        実は$(f-g)^{(5)}\in\S(\R)\subset L^1(\R)$でもあるのは,$g^{(5)}$は原点のみに台を持つことと,$f^{(5)}$の急減少性による.
        \item するとRiemann-Lebesgueの補題より,$\F((f-g)^{(5)})(\xi)\xrightarrow{\abs{\xi}\to\infty}0$.
        これは,超関数方程式,特に,原点以外での方程式
        \[\F((f-g)^{(5)})(\xi)=(i\xi)^5(\wh{f}(\xi)-\wh{g}(\xi)),\qquad \xi\in\R\setminus\{0\}\]
        を満たす(原点の近傍で$0$を取る$\S$の元での値を見ることで従う).
        故に,$\wh{f}(\xi)=\wh{g}(\xi)+o(\abs{\xi}^{-5})$という評価を得た.
        \item そして$\wh{g}$は計算出来る:
        \[\F(\abs{x}^m)=\frac{2i^{m+1}m!}{\xi^{m+1}},\qquad(x\in\R\setminus\{0\}).\]
        \[\wh{g}(\xi)=\frac{2}{\xi^2}-\frac{2}{\xi^4}.\]
    \end{enumerate}
\end{Proof}

\section{緩増加分布の多項式増大連続関数の導関数としての表示}

\begin{tcolorbox}[colframe=ForestGreen, colback=ForestGreen!10!white,breakable,colbacktitle=ForestGreen!40!white,coltitle=black,fonttitle=\bfseries\sffamily,
title=]
    \[E(x):=\frac{x_1^{N+1}\cdots x^{N+1}_d}{((N+1)!)^d}1_{(\R^+)^d}\]
    は$\C^N(\R^d)$の元で,等式$\partial_1^{N+2}\cdots\partial^{N+2}_dE=\delta$を満たす.これについて$f:=T*E$と定めれば,$f\in C_p(\R^d)$となり,
    \[\partial^{N+2}_1\cdots\partial^{N+2}_df=T\]
    を満たす.
    この理論を立てるために,
    \begin{enumerate}
        \item 台が正錐に含まれる超関数同士の畳み込みを定義し,
        \item $f\in C_p(\R^d)$となることを示す
    \end{enumerate}
    必要がある.
\end{tcolorbox}

\subsection{定理と例}

\begin{theorem}
    任意の緩増加分布$T\in\S'$に対して,多項式増大の連続関数$f\in C_p(\R^d)$と多重指数$\al\in\N^d$が存在して,
    \[T=\partial^\al f.\]
\end{theorem}

\begin{example}
    主値超関数$\PV\paren{\frac{1}{x}}$について,$f(x):=x\log\abs{x}-x$とすると,$f\in C_p(\R)$である.これについて,
    \[(T_f)''=\PV\paren{\frac{1}{x}}.\]
\end{example}

\subsection{緩増加分布の階数の定義}

\begin{definition}
    緩増加分布$T\in\S'(\R^d)$が\textbf{$\S'$で階数$N$である}とは,ある$N\in\N,C>0$が存在して$\forall_{\varphi\in\S}\;\abs{(T|\varphi)}\le C\norm{\varphi}_N$を満たすことをいう.
    緩増加分布は必ず階数を持つ\ref{prop-characterization-of-tempered-distribution}.
\end{definition}

\subsection{台が正錐に含まれる超関数の間の畳み込み}

\begin{tcolorbox}[colframe=ForestGreen, colback=ForestGreen!10!white,breakable,colbacktitle=ForestGreen!40!white,coltitle=black,fonttitle=\bfseries\sffamily,
title=]
    \[E(x):=\frac{x_1^{N+1}\cdots x^{N+1}_d}{((N+1)!)^d}1_{(\R^+)^d}\]
    は$\C^N(\R^d)$の元で,等式$\partial_1^{N+2}\cdots\partial^{N+2}_dE=\delta$を満たす.これについて$f:=T*E$と定めれば,$f\in C_p(\R^d)$となることが期待できるが,
    $E,T$はコンパクト台を持つとは限らないのでこのままでは畳み込みが定義出来ない.
\end{tcolorbox}

\begin{notation}
    点$a=(a_1,\cdots,a_d)\in\R^d$を頂点とする正錐を
    \[\Sigma_a:=\prod^d_{i=1}{\cointerval{a_i,\infty}}\]
    で表す.
\end{notation}

\begin{proposition}\mbox{}
    \begin{enumerate}
        \item 任意の$a,b\in\R^d$に対して,$\Sigma_a\cap(-\Sigma_b)$は空集合か$\R^d$の閉矩形である.特に,コンパクトである.
        \item 任意の$a\in\R^d$とコンパクト集合$K\compsub\R^d$に対して,ある$b\in\R^d$が存在して$\Sigma_a+K\subset\Sigma_b$.
    \end{enumerate}
\end{proposition}

\subsection{緩増加分布の階数の性質}

\begin{proposition}
    $T\in\S'$は$\S'$で階数$N$,$f\in C^N(\R^d)$,$\supp T\subset\Sigma_a,\supp f\subset\Sigma_b$とする.このとき,
    \begin{enumerate}
        \item 畳み込み$T*f\in\D'$は$\R^d$上の連続関数である.
        \item ある$\abs{\al}\le N$を満たす多重指数$\al\in\N^d$に対して$\partial^\al f$が多項式増大ならば,$T*f$も多項式増大である.
    \end{enumerate}
\end{proposition}

\section{畳み込み3:緩増加分布と急減少関数の畳み込み}

\begin{tcolorbox}[colframe=ForestGreen, colback=ForestGreen!10!white,breakable,colbacktitle=ForestGreen!40!white,coltitle=black,fonttitle=\bfseries\sffamily,
title=]
    $\S',\S$の元の間の畳み込みは,$\D',\D$の間の畳み込みと同様に,ペアリングを用いて進む.
    しかし,多項式増大連続関数の導関数としての表示をヘビーに用いる.
\end{tcolorbox}

\subsection{畳み込みの定義}

\begin{definition}
    $T\in\S',\psi\in\S$に対して,畳み込み$T*\psi\in C^\infty(\R^d)$を
    \[(T*\psi)(x):=(T|\tau_x\wt{\psi}),\qquad(x\in\R^d).\]
    で定める.
\end{definition}

\begin{lemma}[多項式増大関数に関するwell-definedness]
    $f\in C(\R^d)$は多項式増大,$\psi\in\S$とする.このとき,
    \begin{enumerate}
        \item $f*\psi$は再び多項式増大.
        \item $f*\psi\in C^\infty(\R^d)$
        \item $\forall_{\al\in\N^d}\;\partial^\al(f*\psi)=f*(\partial^\al\psi)$.
    \end{enumerate}
    (1)の証明は,一般の多項式増大な可測関数$f\in L(\R^d)$と急減少な可測関数$\psi\in L(\R^d)$とに対して行う.
\end{lemma}

\begin{proposition}[畳み込みのwell-definedness]
    任意の$T\in\S'$と$\psi\in\S$に対して,
    \begin{enumerate}
        \item $T$を表現する多項式増大の連続関数$f\in C(\R^d),T=\partial^\al f$について,$T*\psi=f*(\partial^\al\psi)$.
        \item $T*\psi\in C^\infty(\R^d)$かつ多項式増大である.よって特に$T*\psi\in\S'$.
    \end{enumerate}
\end{proposition}

\subsection{畳み込みの随伴}

\begin{proposition}
    任意の$T\in\S'$と$\psi,\varphi\in\S$に対して,
    \[(T*\psi|\varphi)=(T|\wt{\psi}*\varphi).\]
\end{proposition}

\subsection{畳み込みと微分の可換性}

\begin{proposition}
    任意の$T\in\S',\psi\in\S,\al\in\N^d$に対して,
    \[\partial^\al(T*\psi)=(\partial^\al T)*\psi=T*(\partial^\al\psi).\]
\end{proposition}

\subsection{畳み込みとFourier変換}

\begin{lemma}
    $T\in\D_c'$はコンパクト台を持つとする(特に$T\in\S'$である).
    \begin{enumerate}
        \item $\wh{T}\in\S'$は多項式増大の$C^\infty$-級関数である.
        \item 任意の$\xi\in\R^d$に対して,
        \[\wh{T}(\xi)=(\o{T}|e_{\xi}).\]
        ただし,$e_\xi:=e^{-ix\cdot\xi}\in\S(\R^d)$とし,$\o{T}$は$C^\infty(\R^d)$上への延長とした.
    \end{enumerate}
\end{lemma}

\begin{proposition}
    任意の$T\in\S',\psi\in\S$に対して,
    \[\wh{(T*\psi)}=\wh{\psi}\wh{T}.\]
\end{proposition}

\begin{proposition}
    $T,S\in\S'$とし,$S$はコンパクト台を持つとする.このとき,
    \begin{enumerate}
        \item $T*S\in\S'$.
        \item $\wh{(T*S)}=\wh{S}\wh{T}$.
    \end{enumerate}
\end{proposition}

\section{$\R$上の周期を持つ超関数}

\begin{tcolorbox}[colframe=ForestGreen, colback=ForestGreen!10!white,breakable,colbacktitle=ForestGreen!40!white,coltitle=black,fonttitle=\bfseries\sffamily,
title=]
    $\bT$上のすべての超関数は緩増加である.
    したがって,$\D'(\bT)$上までFourier変換の定義域は延長し,
    きわめて理想的な解析の舞台となっている.
\end{tcolorbox}

\subsection{定義}

\begin{definition}\mbox{}
    \begin{enumerate}
        \item $T\in\D'(\R)$が\textbf{周期$2\pi$を持つ}とは,次を満たすことをいう:
        \[\tau_{2\pi}T=T.\]
        \item \textbf{単位関数}$\chi\in\D$とは,$\Im\chi\subset[0,1]$かつ
        \[\sum_{k\in\Z}\chi(x+2\pi k)=1,\qquad x\in\R.\]
        を満たすものをいう.
    \end{enumerate}
    $\chi\in\D$での値$T(\chi)$を,$T$の$[0,2\pi]$上での積分とみなす.
\end{definition}

\begin{lemma}\mbox{}
    \begin{enumerate}
        \item 値$(T|\chi)$は,単位関数$\chi\in\D$の取り方に依らない.
        \item $2\pi$周期を持つ超関数は,緩増加である.
    \end{enumerate}
\end{lemma}

\subsection{周期超関数のFourier展開}

\begin{tcolorbox}[colframe=ForestGreen, colback=ForestGreen!10!white,breakable,colbacktitle=ForestGreen!40!white,coltitle=black,fonttitle=\bfseries\sffamily,
title=]
    周期$2\pi$を持つ超関数は常にFourier展開可能である.
\end{tcolorbox}

\begin{theorem}
    $T\in\S'$を周期$2\pi$を持つ超関数とする.
    多項式増大の数列$(c_n)_{n\in\Z}$が存在して,
    \[T=\sum_{n\in\Z}c_ne^{inx}.\]
    特にこのとき,
    \[\wh{T}=2\pi\sum_{n\in\Z}c_n\delta_n.\]
    これを,超関数$T$の\textbf{Fourier級数展開}という.
\end{theorem}

\subsection{例}

\begin{example}
    $S:=\sum_{k\in\Z}\delta_{2\pi k}$について,
    \begin{enumerate}
        \item $\wh{S}=\sum_{n\in\Z}\delta_n$.
        \item $S$は次のようにFourier展開できる:
        \[S=\frac{1}{2\pi}\sum_{n\in\Z}e^{inx}.\]
    \end{enumerate}
\end{example}

\subsection{$\bT$上の超関数としての特徴付け}

\begin{theorem}
    $\chi$を単位関数とする.
    \begin{enumerate}
        \item $T\in\S'(\R)$は周期$2\pi$を持つとする.$\al(T)\in\D'(\bT)$を,
        \[(\al(T)|\psi):=(T|\chi\o{\psi}),\qquad\psi\in C^\infty(\bT),\o{\psi}\in C^\infty(\R)\text{はその持ち上げ}\]
        \item $S\in\D'(\bT)$に対して,$\beta(T)\in\S'(\R)$を
        \[(\beta(S)|\varphi):=(S|\Pe[\varphi]),\qquad\varphi\in\D.\]
    \end{enumerate}
    このとき,$\al,\beta$は連続で,互いに逆写像である.
\end{theorem}

\subsection{$\bT$上の微分方程式}

\begin{problem}
    $\R$上の微分作用素
    \[P:=\sum_{l=0}^ma_l\dd{^l}{x^l},\qquad a_l\in\C\]
    と$2\pi$周期を持つ超関数$f\in\S'(\R)$に対して,超関数ODE
    \[Pu=f,\qquad u\in\S'(\R)\]
    を考える.するとこれは$\bT$上のODEとも見れる.
    $\bT$上では超関数は常にFourier展開可能であるから,これによって完全に決定できる.
    \[u=\sum_{n\in\Z}u_ne^{inx},\quad f=\sum_{n\in\Z}f_ne^{inx}\]
    とFourier展開されるとする
\end{problem}

\begin{proposition}
    $P$が定める多項式
    \[p(\xi):=\sum_{l=0}^ma_l(i\xi)^l\in\C[\xi]\]
    と
    \[\Om:=\Brace{n\in\Z\mid p(n)=0}\]
    \begin{enumerate}
        \item $\Om=\emptyset$ならば,任意の$f$に対して解$u$がただ一つ存在する.
        \item $\forall_{n\in\Om}\;f_n=0$ならば,解$u$は無数に存在する.
        \item $\exists_{n\in\Om}\;f_n\ne0$ならば,解$u$は存在しない.
    \end{enumerate}
\end{proposition}
\begin{Proof}\mbox{}
    \begin{enumerate}[{Step}1]
        \item 
        \[\wh{Pu}=p(\xi)\wh{u}=2\pi p(\xi)\sum_{n\in\Z}u_n\delta_n=2\pi\sum_{n\in\Z}p(n)u_n\delta_n.\]
        が$\wh{f}=2\pi\sum_{n\in\Z}f_n\delta_n$に等しいことが必要.
        すなわち,
        \[p(n)u_n=f_n,\qquad n\in\Z\]
        が必要十分.
        \item 上が満たされるとき,$u:=\sum_{n\in\Z}p(n)^{-1}f_ne^{inx}$は解を与える.実際,
        係数$p(n)^{-1}f_n$は多項式増大な数列であるから,$u$は周期$2\pi$を持つ$\S'$の元である.
        \item (2)については,
        \[u:=\sum_{n\in\Z\setminus\Om}p(n)^{-1}f_ne^{inx}\]
        は解であり,さらに,係数が多項式増大の数列をなす限り,$e^{inx}\;(n\in\Om)$の線型結合との和は解である.
    \end{enumerate}
\end{Proof}

\section{試験関数のなすFrechet空間}

\begin{tcolorbox}[colframe=ForestGreen, colback=ForestGreen!10!white,breakable,colbacktitle=ForestGreen!40!white,coltitle=black,fonttitle=\bfseries\sffamily,
title=]
    $\D(\R^d),\S(\R^d)$のいずれも,基本的には一様ノルムを深くして行った半ノルムの族を考える.
    $\D(\R^d)$上の位相には,よりわかりやすい特徴付けがある.
    \begin{enumerate}
        \item $\D$上の半ノルム
        \[\norm{\varphi}_N:=\sum_{\abs{\al}\le N}\sup_{x\in\R^d}\abs{\partial^\al\varphi(x)},\qquad\varphi\in\D.\]
        を考える.
        \item $\D(\R^d)$上で連続であることは,任意の$K\comp\R^d$に対して,$\D(K)$上で有限階数を持つことに同値.
        なお,$\D(\R^d)$上で階数を持つとは,任意の$K\comp\R^d$に対して一様な階数$N\in\N$が取れることをいう.
        \item $\S$上の半ノルム
        \[\norm{\varphi}_N:=\sum_{\abs{\al},\abs{\beta}\le N}\sup_{x\in\R^d}\abs{x^\al(\partial^\beta\varphi)(x)},\qquad\varphi\in\S.\]
        を考え,収束とは任意の$N\in\N$について$\norm{\varphi_n-\varphi}_N\to0$であることをいう.
        \item $\S(\R^d)$上で連続であることは,$\S(\R^d)$上で有限階数を持つことに同値.
        \item 明らかに同じ$N\in\N$については,$\S$上の半ノルムの方が大きい.よって,緩増加分布は$\D$上で有限階数を持つ.
    \end{enumerate}
\end{tcolorbox}

\subsection{試験関数の空間について}

\begin{definition}[distribution / generalized function]\mbox{}
    \begin{enumerate}
        \item $N\in\N$について,$\D$上のノルム
        \[\norm{\varphi}_N:=\sum_{\abs{\al}\le N}\sup_{x\in\R^d}\abs{\partial^\al\varphi(x)},\qquad\varphi\in\D.\]
        を考える.
        \item $\{\varphi_n\}\subset\D$が$\varphi\in\D$に\textbf{$\D$-収束}するとは,次の2条件を満たすことをいう:
        \begin{enumerate}
            \item $\bigcup_{n\in\N}\supp\varphi_n\cup\supp\varphi\subset\R^d$はコンパクトである.
            \item 任意の多重指数$\al\in\N^d$に対して,$\partial^\al\varphi_n\to\partial^\al\varphi$が一様ノルムについて成り立つ.
        \end{enumerate}
        \item 線型汎関数$T:\D\to\C$が\textbf{超関数}であるとは,$\D$の(2)の収束に関して連続であることをいう.
    \end{enumerate}
\end{definition}

\begin{proposition}
    線型写像$T:\D\to\C$に対して,次は同値:
    \begin{enumerate}
        \item $T$は連続である.
        \item 任意のコンパクト集合$K\compsub\R^d$に対して,ある$C>0,N\in\N$が存在して,$\varphi\in\D(K)$に対して,
        \[\abs{(T|\varphi)}\le C\sum_{\abs{\al}\le N}\norm{\partial^\al\varphi}_\infty.\]
    \end{enumerate}
    この$N\in\N$がコンパクト集合$K\compsub\R^d$に依らずに取れるとき($C$も$K$に依らない必要はない\cite{Rudin-FunctionalAnalysis}Def.6.8),$N$を超関数$T$の\textbf{階数}という.
\end{proposition}

\subsection{急減少関数の空間について}

\begin{definition}[tempered distribution]\mbox{}
    \begin{enumerate}
        \item $\N\in\N$について,$\S$上のノルム
        \[\norm{\varphi}_N:=\sum_{\abs{\al},\abs{\beta}\le N}\sup_{x\in\R^d}\abs{x^\al(\partial^\beta\varphi)(x)},\qquad\varphi\in\S.\]
        を考える.
        \item $\{\varphi_n\}\subset\S$が$\varphi\in\S$に\textbf{$\S$-収束}するとは,任意の$N\in\N$に対して$\norm{\varphi_n-\varphi}_N\xrightarrow{n\to\infty}0$を満たすことをいう.
        \item 線型汎関数$T:\S\to\C$が\textbf{緩増加超関数}であるとは,$\S$の(2)の収束について連続であることをいう.
    \end{enumerate}
\end{definition}
\begin{remarks}
    $f\in C^\infty(\R^n)$が緩増加であるための条件は
    \begin{enumerate}
        \item $\forall_{N\in\N}\;\max_{\abs{\al}\le N}\sup_{x\in\R^n}(1+\abs{x}^2)^N\abs{(\partial^\al f)(x)}<\infty$.
        \item $\forall_{P\in\C[X_1,\cdots,X_d]}\;\forall_{\al\in\N^d}\;P\cdot\partial^\al f\in L^\infty(\R^d)$.
        \item $\forall_{P\in\C[X_1,\cdots,X_d]}\;\forall_{\al\in\N^d}\;P\cdot\partial^\al f\in L^1(\R^d)$.
    \end{enumerate}
    とも表せる.$(1+\abs{x}^2)^NP\cdot\partial^\al f\in L^\infty(\R^d)$が$P\cdot\partial^\al f\in L^\infty(\R^d)\in L^1(\R^d)$に同値であることに注意.
\end{remarks}

\begin{proposition}
    線型写像$T:\S\to\C$について,次は同値:
    \begin{enumerate}
        \item $T\in\S'$.
        \item ある$N\in\N,C>0$が存在して,$\forall_{\varphi\in\S}\;\abs{(T|\varphi)}\le C\norm{\varphi}_N$.
    \end{enumerate}
    特に,$\S$上の$\norm{-}_N$の方が$\D$上の$\norm{-}_N$よりも大きくなることに注意すれば,緩増加分布は有限な階数を持つ.
\end{proposition}

\subsection{可微分関数の空間について}

\begin{tcolorbox}[colframe=ForestGreen, colback=ForestGreen!10!white,breakable,colbacktitle=ForestGreen!40!white,coltitle=black,fonttitle=\bfseries\sffamily,
title=]
    $C^\infty(\Om),\D(K)$は共通の位相によりFrechet空間となるが,$\D(U)$は同様の方法では完備にならない.
    別の位相を用いて完備な局所凸空間とするが,今度は距離化可能性を失う.
\end{tcolorbox}

\begin{definition}
    $\Om\osub\R^d$とそのコンパクト集合の増大列$\Om=\cup_{i\in\N}K_i$について,
    \begin{enumerate}
        \item $K\compsub\Om$について,$\D(K):=\Brace{f\in C^\infty(\R^d)\mid\supp f\subset K}$とする.
        \item $C^\infty(\Om)$上の半ノルム$p_N\;(N\in\N)$を,
        \[p_N(f):=\sum_{\abs{\al}\le N}\max_{x\in K_N}\abs{\partial^\al f(x)}.\]
        と定める.
        \item この半ノルムの列$(p_N)$について,
        \[V(p_N,n):=\Brace{f\in C^\infty(\Om)\;\middle|\;p_N(f)<\frac{1}{n}},\qquad(n,N\in\N).\]
        が生成する位相を考える.なお,今回は$p_N$が$N$について単調増加であることに注意すれば,
        開球の列
        \[V_N:=\Brace{f\in C^\infty(\Om)\;\middle|\;p_N(f)<\frac{1}{N}},\qquad(N\in\N)\]
        が基本系をなすので,この列のみに注目すれば十分であることに注意.
    \end{enumerate}
\end{definition}

\begin{lemma}
    半ノルムの列$(p_N)_{N\in\N}$は$C^\infty(\Om)$上に局所凸な位相を定め,それは
    \begin{enumerate}
        \item 距離化可能である.
        \item 評価$\ev_x:C^\infty(\Om)\to\R$は連続である.
    \end{enumerate}
\end{lemma}
\begin{Proof}\mbox{}
    \begin{enumerate}
        \item 半ノルムの列$(p_N)$が$C^\infty(\Om)$の分離族であることを確認すれば,
        $(V_N)$の形で生成される位相は局所凸で,かつ距離化可能であることは一般論による.\cite{Rudin-FunctionalAnalysis}定理1.37.
        \item コンパクト開位相よりも濃い位相であるため.
    \end{enumerate}
\end{Proof}

\begin{proposition}\mbox{}
    \begin{enumerate}
        \item $C^\infty(\Om)$はFrechet空間である.
        \item $C^\infty(\Om)$はHeine-Borel性を持つ.したがって一般論より,局所有界ではなく,特にノルム付け可能でない.
        \item $\D(K)\subset C^\infty(\Om)$は閉部分空間である.特に,再びFrechet空間である.
    \end{enumerate}
\end{proposition}
\begin{Proof}\mbox{}
    \begin{enumerate}
        \item あとは$C^\infty(\Om)$が完備であることを確認すれば良い.
        Cauchy列$\{f_i\}\subset C^\infty(\Om)$を取る.
        任意の$N\in\N$について,十分大きな$i,j\in\N$について$f_i-f_j\in V_N$が成り立つから,特に$\forall_{\abs{\al}\le N}\;\abs{D^\al f_i-D^\al f_j}<\frac{1}{N}\;\on K_N$.
        特に,$D^\al f_i$は$\Om$上コンパクト一様にある$g_\al\in C(\Om)$に収束する.
        よって特に,$f_i\to g_0\in C^\infty(\Om)$かつ$g_\al=\partial^\al g_0$.
        以上より,$C^\infty(\Om)$の位相で$f_i\to g$.
        \item $E\subset C^\infty(\Om)$が閉かつ有界とし,これがコンパクトになることを示せば良い.
    \end{enumerate}
\end{Proof}

\begin{proposition}
    $\D(\Om)$について,
    \begin{enumerate}
        \item 帰納系$(\D(K))_{K\in\cC(\Om)}$の極限として特徴付けられる.
        特に,再びFrechet空間である.
        \item 半ノルム
        \[\norm{\varphi}_N:=\sum_{\abs{\al}\le N}\sup_{x\in\Om}\abs{D^\al\varphi(x)},\qquad\varphi\in\D(\Om),N\in\N\]
        の$\D(K)$への制限は$p_N$と同値である.
        \item よって,$\norm{\varphi}_N$を用いても,同様に距離化可能な局所凸線型空間$\D(\Om)$を考えることが出来る.
        しかし,この位相について$\D(\Om)$は完備にはならない!
    \end{enumerate}
\end{proposition}
\begin{remarks}
    一般に,Frechet空間$X$の双対空間$X^*$もFrechet空間になることと$X$がBanach空間であることに同値.
    よって,$\D^*(K)$はFrechet空間ではない.
\end{remarks}

\subsection{試験関数の空間再論}

\begin{tcolorbox}[colframe=ForestGreen, colback=ForestGreen!10!white,breakable,colbacktitle=ForestGreen!40!white,coltitle=black,fonttitle=\bfseries\sffamily,
title=]
    $\D$の位相の定義\ref{def-topology-in-D}のもう一つの\cite{Rudin-FunctionalAnalysis}による定義を見る.
\end{tcolorbox}

\begin{definition}
    $\D(\Om)$の位相$\tau$を次のように定める:
    \begin{enumerate}
        \item $\D(K)\;(K\in\cC(\Om))$のFrechet空間としての位相を$\tau_K$とする.
        \item 各$K\in\cC(\Om)$への制限が開になるような絶対凸集合の全体
        \[\beta:=\Brace{W\subset\D(\Om)\mid W\text{は絶対凸で}\forall_{K\in\cC(\Om)}\;\D(K)\cap W\in\tau_K}.\]
        を考える.
        \item $\varphi+W\;(\varphi\in\D(\Om),W\in\beta)$という形の集合の全体が生成する位相を$\tau$とする.
    \end{enumerate}
\end{definition}

\begin{theorem}\mbox{}
    \begin{enumerate}
        \item $\D(\Om)$の部分空間としての$\D(K)$の位相と,Frechet空間としての$\D(K)$の位相とは一致する.
        \item 次は同値:
        \begin{enumerate}
            \item $\{\varphi_i\}\subset\D(\Om)$がCauchy列とする.
            \item あるコンパクト集合$K\compsub\Om$について$\{\varphi_i\}\subset\D(K)$であり,かつ任意の$N\in\N$について$\lim_{i,j\to\infty}\norm{\varphi_i-\varphi_j}=0$.
        \end{enumerate}
        \item 次は同値:
        \begin{enumerate}
            \item $\varphi_i\to0$である.
            \item あるコンパクト集合$K\subset\Om$が存在して,$\cup_{i\in\N}\supp\varphi_i\subset K$であり,かつ任意の多重指数$\al\in\N^d$について$D^\al\varphi_i\to0$が一様に成り立つ.
        \end{enumerate}
        \item $\D(\Om)$はこの位相について完備である.
    \end{enumerate}
\end{theorem}

\begin{theorem}[試験関数の空間上の線形作用素の連続性の特徴付け]
    $Y$を局所凸空間,$\Lambda:\D(\Om)\to Y$を線型作用素とする.次の4条件は同値:
    \begin{enumerate}
        \item $\Lambda$は連続.
        \item $\Lambda$は有界.
        \item $\varphi_i\to0$ならば,$\Lambda\varphi_i\to0$.
        \item 任意の$K\compsub\Om$について,制限$\Lambda|_{D_K}$は連続.
    \end{enumerate}
\end{theorem}
\begin{corollary}
    微分作用素$D^\al:\D(\Om)\to\D(\Om)$は連続である.
\end{corollary}
\begin{remarks}
    実は全射でもある(Ehrenpreis, Malgrange).
    偏微分方程式$Du=f$に対して,
    $DT=\delta_0$を満たす解$T\in\D'(\Om)$を\textbf{基本解}または\textbf{Green関数}といい,これを求めた後,$u:=f*T$は滑らかな解である.
\end{remarks}

\begin{corollary}
    線型汎関数$\Lambda:\D(\Om)\to\C$について,次は同値:
    \begin{enumerate}
        \item $\Lambda\in\D'(\Om)$.
        \item 任意のコンパクト集合$K\compsub\Om$について,ある$N\in\N,C>0$が存在して,
        \[\forall_{\varphi\in\D(K)}\;\abs{\Lambda\varphi}\le C\norm{\varphi}_N.\]
    \end{enumerate}
    ある$N\in\N$が存在して,任意の$K\in\cC(\Om)$について一様に取れるとき,$N$を超関数$\Lambda$の\textbf{階数}という.
\end{corollary}

\subsection{急減少関数の空間再論}

\begin{definition}
    $\S(\R)$に半ノルムの族
    \[\norm{f}_{j,n}:=\sup_{x\in\R}\abs{x^nf^{(j)}(x)},\qquad j,n\in\N.\]
    によって位相を入れると,距離化可能で完備な局所凸空間,すなわちFrechet空間を得る.
\end{definition}
\begin{Proof}
    距離
    \[d(f,g):=\sum_{j,n\ge0}\frac{1}{2^{j+n}}\frac{\norm{f-g}_{j,n}}{1+\norm{f-g}_{j,n}}\]
    が同値な位相を与える.
\end{Proof}

\begin{proposition}[\cite{Rudin-FunctionalAnalysis} Th'm 7.4]
    次の変換は全て$\S(\R^n)$上の連続な線型変換である:
    \begin{enumerate}
        \item 多項式$P$について,$f\mapsto Pf$.
        \item $g\in\S(\R^n)$について,$f\mapsto gf$.
        \item $f\mapsto D^\al f$.
        \item $f\mapsto\wh{f}$.
    \end{enumerate}
\end{proposition}

\chapter{$L^p(\R^d)$上のFourier変換}

\begin{quotation}
    $L^p(\R^d)$上のFourier変換の理論を補間理論という.\cite{Pinsky09-Wavelet}第3章など参照.
\end{quotation}

\section{Riesz-Thorinの補完定理}

\section{平均関数}

\begin{definition}[rearrangement, average function]
    $(X,\M,\mu)$を完備測度空間,$f\in L(X)$とする.
    \begin{enumerate}
        \item $\mu_f(s):=\mu(\Brace{x\in X\mid\abs{f(x)}>s})\;(s>0)$で定まる$\mu:\R^+\to[0,\infty]$を\textbf{分布関数}という.
        \item $f^*(t):=\inf\Brace{s>0\mid\mu_f(s)\le t}$で定まる$f^*:\R^+\to[0,\infty]$を\textbf{再配列}という.
        \item $f^{**}(t):=\frac{1}{t}\int^t_0f^*(s)ds$で定まる$f^{**}:\R^+\to[0,\infty]$を\textbf{平均関数}という.
    \end{enumerate}
\end{definition}

\begin{theorem}\mbox{}
    \begin{enumerate}
        \item 分布関数$\mu_f$は$\R^+$上右連続な単調減少関数である.
        \item 再配列$f^*$も$\R^+$上右連続な単調減少関数であり,Lebesgue測度$m$について$\mu_f=m_{f^*}$を満たす.
        \item 逆に上の2つの性質を持つ関数$f^*$が正値ならば,再配列$f^*$に限る.
        \item $\forall_{f,g\in L(X)}\;\forall_{t_1,t_2\in\R^+}\;(f+g)^*(t_1+t_2)\le f^*(t_1)+g^*(t_2)$.
        \item 平均関数$f^{**}$は$\R^+$上連続な単調減少関数であり,$f\in L^\infty(X)+L^1(X)$であるときにかぎって有限値である.
        \item 平均関数は$f^{**}=\inf\Brace{\norm{f_0}_{L^\infty(X)}+t^{-1}\norm{f_1}_{L^1(X)}\mid f=f_0+f_1}$と特徴付けられる.
        \item $\forall_{f,g\in L^\infty(X)+L^1(X)}\;\forall_{t\in\R^+}\;(f+g)^{**}(t)\le f^{**}(t)+g^{**}(t)$.
    \end{enumerate}
\end{theorem}

\section{極大関数}

\subsection{定義とHardy-Littlewoodの定理}

\begin{definition}
    $-\infty\le a<b\le\infty$,$f\in L^1_\loc((a,b))$とする.次を右・左・対称\textbf{極大関数}という.
    \begin{enumerate}
        \item $\Theta_rf(x):=\sup_{x<x+t\le b}\frac{1}{t}\int^t_0\abs{f(x+y)}dy$.
        \item $\Theta_lf(x):=\sup_{a\le x-t<x}\frac{1}{t}\int^t_0\abs{f(x-y)}dy$.
        \item $\Theta f(x):=\sup_{a\le x-t<x+t\le b}\frac{1}{2t}\int^t_{-t}\abs{f(x+y)}dy$.
    \end{enumerate}
\end{definition}

\begin{theorem}[Hardy-Littlewood]
    $\Theta_r,\Theta_l,\Theta$は次を満たす劣線型作用素である:
    \begin{enumerate}
        \item $(\Theta_rf)^*(t)\le f^{**}(t)$.
    \end{enumerate}
\end{theorem}

\section{Lorentz空間}

\chapter{Fourier変換と超関数論}

\begin{quotation}
    前節で,Fourier変換を通じれば,関数をその成分$\al e^{i\lambda x}$に分解できることが解った.
    物理的にはそれぞれの成分は調和振動子に当たる.
    今後は群$\R$は可換であってもコンパクトではないので,
    これを一般的に展開するには,創作が必要になり,超関数論が要請される.
\end{quotation}

\section{Formal Property}

\begin{tcolorbox}[colframe=ForestGreen, colback=ForestGreen!10!white,breakable,colbacktitle=ForestGreen!40!white,coltitle=black,fonttitle=\bfseries\sffamily,
title=]
    周期関数$f\in L^2(T)$や$f\in L^1(T)$に関するFourier変換を,まずFourier係数列への対応$\F:L^1(T)\to C_0(\Z)=c_0$として考えた.では$\Z$上から$\R$上へと拡張しようとすると,ほとんどの$f\in C(T)$では失敗するが,$f\in L^p(T)\;(1<p\le\infty)$では殆ど至る所の$x\in\R$で成功する.
    一般の関数$f\in L^1(\R)$については,Fourier変換を実数上の関数として調べる.
\end{tcolorbox}

\subsection{特性関数}

\begin{tcolorbox}[colframe=ForestGreen, colback=ForestGreen!10!white,breakable,colbacktitle=ForestGreen!40!white,coltitle=black,fonttitle=\bfseries\sffamily,
title=]
    指標とは,加法から乗法への群準同型をいう.
    指標との畳み込み$\wh{f}(t)=(f*e_t)(0)$を\textbf{Fourier変換}という.
\end{tcolorbox}

\begin{notation}[normalized Lebesgue measure]\mbox{}
    \begin{enumerate}
        \item $dx$を$\R$上のLebesgue測度,$m$をそれを$\sqrt{2\pi}$で割ったものとする:
        $\int_\R f(x)dm(x)=\frac{1}{\sqrt{2\pi}}\int_\R f(x)dx$.
        
        一般に,$dm_n(x)=(2\pi)^{-n/2}dx$.
        \item 内積のスケール変化に合わせて,\textbf{$p$-ノルム}の定義も変わる
        $\norm{f}_p:=\paren{\int_\R\abs{f(x)}^pdm(x)}^{1/p}\quad(1\le p<\infty)$.
        \item \textbf{畳み込み}もリスケールする:
        $(f*g)(x):=\int_\R f(x-y)g(y)dm(y)\quad(x\in\R)$.
        \item $f\in L^1$に対して,指標$e_t(x)=e^{itx}$との畳み込みでの$u=0$の値
        (各積分核$e^{-ixt}\;(t\in\R)$に関する積分変換)を\textbf{Fourier変換}$\wh{-}:L^1\to L^1$といい,次のように表す:
        \[\hat{f}(t):=(f*e_t)(0)=\int_\R f(x)e_{-t}(x)dm(x)=\int_\R f(x)e_t(0-x)dm(x)=\int_\R f(x)e^{-ixt}dm(x)\quad(t\in\R).\]
        \item $L^p(\R),C_0(\R)$を$L^p,C_0$と略記する.
        \item 多重指数$\al\in\N^n$に対して,\textbf{偏微分作用素}$D_\al$を$D_\al:=(i)^{-n}D^\al=\paren{\frac{1}{i}\pp{}{x_1}}^{\al_1}\cdots\paren{\frac{1}{i}\pp{}{x_n}}^{\al_n}$で表す.このとき,$D_\al e_t=t^\al e_t$が成り立つ.これがFourier変換の偏微分方程式論への応用可能性を支える.
        \item \textbf{遷移作用素}$\tau_x\;(x\in\R^n)$を$(\tau_xf)(y):=f(y-x)$で定める.
    \end{enumerate}
\end{notation}

\begin{definition}[character, characteristic function]\mbox{}
    \begin{enumerate}
        \item 加法群$\R$から$\C$の乗法部分群$S^1$への群準同型,すなわち可換群$\R$の連続な1次元ユニタリ表現を\textbf{指標}という:$\abs{\varphi(t)}=1,\varphi(s+t)=\varphi(s)\varphi(t)$.
        \item 特に,$t\in\R^n$の指標$e_t$とは,指数関数$e_t(x):=e^{it\cdot x}\;(x\in\R^n)$をいう.2つの引数について全く対称であることに注意.
        \item 指標$t\in\R^n$の,$\R^n$上の確率測度に関する積分(との合成)$\varphi(u)=\int_\R e^{iut}dt$を特性関数という.
    \end{enumerate}
\end{definition}

\begin{tcolorbox}[colframe=ForestGreen, colback=ForestGreen!10!white,breakable,colbacktitle=ForestGreen!40!white,coltitle=black,fonttitle=\bfseries\sffamily,
    title=]
    基底$x:=(x^1,\cdots,x^n)$に対して,これらの組み合わせの冪を簡潔に表記する記法を導入する.
\end{tcolorbox}

\begin{definition}[multi-index]
    集合$\N^d$に各点和と$l_1$-ノルム$\abs{\al}:=\al_1+\cdots+\al_d$,階乗$\al!=\al_1!\cdots\al_d!$を備えた区間
    の元を\textbf{$d$次の多重指数}という.
    \begin{enumerate}
        \item 二項係数について$\begin{pmatrix}\al\\\beta\end{pmatrix}=\begin{pmatrix}\al_1\\\beta_1\end{pmatrix}\cdots\begin{pmatrix}\al_d\\\beta_d\end{pmatrix}$とする.
        \item $f\in C^d(\R^d)$に対する偏微分について,$\partial^\al f=\frac{\partial^{\al_1+\cdots+\al_d}}{\partial x_1^{\al_1}\cdots\partial x_d^{\al_d}}f$とする.
        \item $x\in\R^d$に対して,$x^\al=x_1^{\al_1}\cdots x_d^{\al_d}$とする.
    \end{enumerate}
\end{definition}

\subsection{Fourier変換の関手性}

\begin{tcolorbox}[colframe=ForestGreen, colback=ForestGreen!10!white,breakable,colbacktitle=ForestGreen!40!white,coltitle=black,fonttitle=\bfseries\sffamily,
title=]
    Fourier変換は,群準同型$(L_1,\delta_0,*)\to(L_1,1,\cdot)$を与える.また,
    指標による積を平行移動に,平行移動を指標による積に変換する.
\end{tcolorbox}

\begin{theorem}[Fourier変換の関手性]
    $f\in L^1,\al,\lambda\in\R$とする.
    \begin{enumerate}
        \item $\wh{\tau_xf}=e_{-x}\wh{f}$.すなわち,$\wh{f(x-\al)}=\wh{f}(t)e^{-i\al t}$.
        \item $\wh{e_xf}=\tau_x\wh{f}$.すなわち,$\wh{f(x)e^{i\al x}}=\wh{f}(t-\al)$.
        \item $\wh{(f*g)}=\wh{f}\wh{g}$.
        \item $h(x)=f\paren{\frac{x}{\lambda}}\;(\lambda>0)$のFourier変換は,$\wh{h}(t)=\lambda^n\wh{f}(\lambda t)$.ただし$f\in L^1(\R^n)$とした.
    \end{enumerate}
    また,次が成り立つ:
    \begin{enumerate}\setcounter{enumi}{4}
        \item $g(x)=\o{f(-x)}\Rightarrow\wh{g}(t)=\o{\wh{f}(t)}$.
        \item $g(x)=-ixf(x)\land g\in L^1\Rightarrow\wh{f}$は微分可能で$\wh{f}'(t)=\wh{g}(t)$.
    \end{enumerate}
\end{theorem}
\begin{proof}\mbox{}
    \begin{enumerate}
        \item \begin{align*}
            \wh{\tau_xf}(t)&=\int(\tau_xf)\cdot e_{-t}(y)dm(y)\\
            &=\int f(y)(\tau_{-x}e_{-t})(y)dm(y)\\
            &=\int f(y)e_{-t}(y+x)dm(y)\\
            &=e_{-t}(x)\wh{f}(t)=e_{-x}(t)\wh{f}(t).
        \end{align*}
        \item \begin{align*}
            \wh{e_xf}(t)&=\int(e_xf)(y)e_{-t}(y)dm(y)\\
            &=\int f(y)e_{-(t-x)}(y)dm(y)=(f*e_{t-x})(0)=(\tau_x\wh{f})(t).
        \end{align*}
        \item Fubiniの定理より,
        \begin{align*}
            \wh{f*g}(t)&=\int_{\R\times\R}\paren{f(y-x)g(x)dx}e_{-t}(y)dm(y)\\
            &=\int_\R g(x)e_{-t}(x)dx\int_\R f(y-x)e_{-t}(y-x)dm(y)=\wh{g}(t)\wh{f}(t).
        \end{align*}
        Fibiniの定理より,$\mu\dae x$について$\int_\R f(y-x)e_{-t}(y-x)dm(y)$は可積分で,その値は一定であることに注意.
        \item 簡単な変数変換により分かる.1次元の場合は次のようになり,$n$次元の場合は$\lambda I$の行列式$\lambda^n$がかかる:
        \begin{align*}
            \wh{h}(t)&=\int_\R h(x)e_{-t}(x)dm(x)=\int_\R f\paren{\frac{x}{\lambda}}e_{-t}\paren{\frac{x}{\lambda}}dm(x)\\
            &=\lambda\int_\R f(y)e_{-t}(y)dx=\lambda\wh{f}(t).
        \end{align*}
    \end{enumerate}
\end{proof}
\begin{remarks}
    以上の結果は,ひとえに$m$の平行移動不変性と,任意の$t\in\R$について積分核$x\mapsto e^{itx}$は$\R$の指標であることによる.
    なお,$\R$の連続な指標はすべて指数関数によって表現される.
\end{remarks}

\begin{remark}[(6)の逆]
    微分を$ti$による積に変換することから,Fourier変換は常微分方程式論でも応用される.
    $f,f'\in L^1$のとき,$\wh{f'}=it\wh{f}(t)$である.
\end{remark}

\section{The Inversion Theorem}

\begin{tcolorbox}[colframe=ForestGreen, colback=ForestGreen!10!white,breakable,colbacktitle=ForestGreen!40!white,coltitle=black,fonttitle=\bfseries\sffamily,
title=]
    元の世界に戻す方法があって初めて,Formal Propertyが応用上の意味を持つ.
    まず$\F:L^1\to C_0$がノルム減少的な線型写像であることが分かる.このうち,$L_1\subset\Im\F$の上では可逆である.
    特に$\S$上では同型を定める.
\end{tcolorbox}

\subsection{ノルム減少性}

\begin{discussion}[Fourier級数に関する逆転公式]
    Fourier係数列が
    \[c_n=(f,e^{inx})=\frac{1}{2\pi}\int^\pi_{-\pi}f(x)e^{-inx}dx\quad(n\in\Z)\]
    であった場合,元の関数は$f(x)=\sum^\infty_{n=-\infty}c_ne^{inx}\in L^2(T)$となるのであった.
    しかし,$f$の級数の収束は各点においては定まらない.
    しかし,追加で$c_n\in L^1(\mu)=l^\infty$を仮定すれば,級数は一様収束するから,Fourier級数は$f$に殆ど至る所収束する.
\end{discussion}

\begin{notation}
    任意の関数$f\in\Map(\R,\C)$と点$y\in\R$について,$f$の$y$-平行移動を
    \[f_y(x)=f(x-y)\]
    で表す.
\end{notation}

\begin{lemma}
    $f\in L^p\;(1\le p<\infty)$に対する
    平行移動写像
    \[\xymatrix@R-2pc{
        \R\ar[r]&L^p(\R)\\
        \rotatebox[origin=c]{90}{$\in$}&\rotatebox[origin=c]{90}{$\in$}\\
        y\ar@{|->}[r]&f_y=(f\cdot-y)
    }\]
    は一様連続である.
\end{lemma}

\begin{theorem}
    $\F:L^1\to C_0$がノルム減少的な写像(short map)であることを定立する.
    \begin{enumerate}
        \item $f\in L^1$ならば$\wh{f}\in C_0$である.
        \item $\norm{\wh{f}}_\infty\le\norm{f}_1$.
    \end{enumerate}
\end{theorem}

\subsection{Poisson核}

\begin{tcolorbox}[colframe=ForestGreen, colback=ForestGreen!10!white,breakable,colbacktitle=ForestGreen!40!white,coltitle=black,fonttitle=\bfseries\sffamily,
title=]
    正関数$H$と,そのFourier変換$h_\lambda$であって積分が明らかなものの組を把握しておく.
    これはなんでも良いが,Poisson核とする.
\end{tcolorbox}

\begin{proposition}
    $f\in L^1$ならば,
    \[(f*h_\lambda)(x)=\int^\infty_{-\infty}H(\lambda t)\wh{f}(t)e^{ixt}dm(t).\]
\end{proposition}

\begin{proposition}
    $g\in L^\infty$は$x\in\R$において連続であるとする.このとき,
    \[\lim_{\lambda\to 0}(g*h_\lambda)(x)=g(x).\]
\end{proposition}

\begin{proposition}
    $f\in L^p\;(1\le p<\infty)$について,$\lim_{\lambda\to0}\norm{f*h_\lambda-f}_p=0$.
\end{proposition}

\subsection{Gauss核}

\begin{lemma}
    $\phi_n(x):=\exp\paren{-\frac{1}{2}\abs{x}^2}\;(x\in\R^n)$とすると,
    \begin{enumerate}
        \item $\phi_n\in\S_n$.
        \item $\wh{\phi}_n=\phi_n$.
        \item $\phi_n(0)=\int_{\R_n}\wh{\phi_n}dm_n$.
    \end{enumerate}
\end{lemma}

\subsection{逆転公式}

\begin{theorem}[The Inversion Theorem]\mbox{}
    \begin{enumerate}
        \item $g\in\S_n$のとき,$g(x)=\int_{\R^n}\wh{g}e_xdm_n\;(x\in\R^n)$.
        \item $f,\wh{f}\in L^1(\R^n)$のとき,$f_0(x):=\int_{\R^n}\wh{f}e_xdm_n\;(x\in\R^n)$と定めると$f_0\in C_0(\R^n)$で,$f=f_0\;\ae x$.
    \end{enumerate}
    $f,\wh{f}\in L^1$で,
    \[g(x)=\int^\infty_{-\infty}\wh{f}(t)e^{ixt}dm(t)\quad(x\in\R)\]
    を満たすとする.このとき,$g\in C_0$かつ$f=g\;\ae$
\end{theorem}

\begin{theorem}[The Uniqueness Theorem]
    $f\in L^1$かつ$\wh{f}=0$ならば,$f(x)=0\;\ae$
\end{theorem}

\section{The Plancherel Theorem}

\begin{tcolorbox}[colframe=ForestGreen, colback=ForestGreen!10!white,breakable,colbacktitle=ForestGreen!40!white,coltitle=black,fonttitle=\bfseries\sffamily,
title=$L^1$に入る前に,$L^2$上での完成された議論を見る]
    $\S_n$は$L_1$上でも$L_2$上でも稠密であるため,$L^1\cap L^2$上のPlancherel作用素は$L^2$上に延長し,\textbf{等長なままである}:$\norm{f_2}=\norm{\wh{f}_2}$.
    Lebesgue測度は有界でないから$L_2$は$L_1$の部分集合ではないが,$L_2$に「制限」するとFourier-Plancherel変換はすごく振る舞いが良い.
\end{tcolorbox}

\begin{theorem}
    次の条件を満たす線型等長同型$\Psi:L^2(\R^n)\mono L^2(\R^n)$が一意的に存在する:$\forall_{f\in\S_n}\;\Psi f=\wh{f}$.
\end{theorem}

\section{The Banach Algebra $L^1$}

\section{分布}

\begin{tcolorbox}[colframe=ForestGreen, colback=ForestGreen!10!white,breakable,colbacktitle=ForestGreen!40!white,coltitle=black,fonttitle=\bfseries\sffamily,
title=]
    解析学にぴったりな議論空間を,代数のことばで得ることを考える.
\end{tcolorbox}

\subsection{試験関数の空間の定義}

\begin{notation}
    $\D=\D(\R)=C_c^\infty(\R)$を隆起関数という.
\end{notation}

\begin{definition}[test function space]
    非空な開集合$\Om\subset\R^n$に対して,台がコンパクト集合$K\subset\Om$に含まれる関数のなすFrechet空間$\D_K$の合併
    $\D(\Om):=\cup_{K\compsub\Om}\D_K$を\textbf{試験関数の空間}という.
    すなわち,$\D(\Om):=C^\infty_c(\Om)=\Brace{\phi\in C^\infty(\Om)\mid\supp\phi\compsub\Om}$.
\end{definition}

\begin{lemma}
    ノルムの列$\norm{\phi}_N:=\max\Brace{\abs{D^\al\phi(x)}\in\R_+\mid x\in\Om,\abs{\al}\le N}$を考える.
    \begin{enumerate}
        \item このノルムの列は,任意の部分空間$\D_K\subset\D(\Om)$に元々のFrechet空間としての局所凸位相$\tau_K$を定める.
        \item このノルムの列は,$\D(\Om)$に局所凸で距離化可能な位相を定めるが,完備でない.
    \end{enumerate}
\end{lemma}

\subsection{完備な位相の導入}

\begin{tcolorbox}[colframe=ForestGreen, colback=ForestGreen!10!white,breakable,colbacktitle=ForestGreen!40!white,coltitle=black,fonttitle=\bfseries\sffamily,
title=]
    完備だが,距離化可能でない位相$\tau$を代わりに導入し,以後は$\D(\R^n)$にはこの位相を入れ,こちらを研究の道具とする.
\end{tcolorbox}

\begin{definition}\mbox{}
    \begin{enumerate}
        \item 絶対凸集合(均衡凸集合)$W\subset\D(\Om)$であって,$\forall_{K\compsub\Om}\;\D_K\cap W\in\tau_K$を満たすもの全体の集合を$\beta$と表す.
        \item $\beta$の元の平行移動$\phi+W\;(\varphi\in\D(\Om),W\in\beta)$で表される集合の任意合併で得られる集合全体の集合を$\tau$で表す.
    \end{enumerate}
\end{definition}

\begin{theorem}\mbox{}\label{thm-topology-on-the-space-of-test-functions}
    \begin{enumerate}
        \item $\tau$は$\D(\Om)$に位相を定め,$\beta$は$\tau$の局所基を定める.
        \item $\tau$は局所凸である.
    \end{enumerate}
\end{theorem}

\section{急減少関数空間}

\begin{tcolorbox}[colframe=ForestGreen, colback=ForestGreen!10!white,breakable,colbacktitle=ForestGreen!40!white,coltitle=black,fonttitle=\bfseries\sffamily,
title=]
    可積分関数の$\F$による像は可積分とは限らない.
    そこで,$\F$が保存する性質を探したい.
    そのうち一つは急減少性である.
    $\F$は複素Frechet空間$S(\R^n)\subset C^\infty(\R^n)$上に線型自己同型を定め,標準的なFourier変換$\F:L^2(\R^n)\to L^2(\R^n)$の自然な延長となる.
    この性質により,突然この空間が調和解析の中心へと躍り出る.
    緩増加関数の空間$D'(X)$は,線型偏微分方程式を解く自然な場となる.
\end{tcolorbox}

\subsection{定義}

\begin{tcolorbox}[colframe=ForestGreen, colback=ForestGreen!10!white,breakable,colbacktitle=ForestGreen!40!white,coltitle=black,fonttitle=\bfseries\sffamily,
title=]
    一般に$\R^n$上の急減少関数とは,$\R$の座標写像の任意の冪との積写像が有界写像であることをいう.
    が,主に議論の対象となる,無限階微分可能な急減少関数とは,任意階の(偏)導関数が急減少であることをいう.
\end{tcolorbox}

\begin{definition}[rapidly decreasing function, the Schwartz space]
    $f\in C^\infty(\R^n)$が\textbf{急減少偏導関数を持つ滑らかな関数}であるとは,次の同値な2条件を満たすことをいう:
    \begin{enumerate}
        \item $\forall_{N\in\N}\;\sup_{\abs{\al}\le N}\sup_{x\in\R^n}(1+\abs{x}^2)N\abs{(D_\al f)(x)}<\infty$.
        \item $\forall_{n\in\N,\al\in\N^n}\;\forall_{P\in\R[x]}\;P\cdot D_\al f\in l^\infty(\R^n)$.
    \end{enumerate}
    この関数がなす位相線形空間を\textbf{Schwartz空間}といい,$\S_n=\S(\R^n)$で表す.
\end{definition}
\begin{remarks}
    無限回微分可能な関数であるが,$\abs{x}\to\infty$を考えたときに,任意階の導関数が,$x$の任意の負冪よりも速くゼロに収束するものをいう.
    これは確率分布との関連で捉えると見通しが良い.どうして収束の速さがFourier変換と関係があるのだろうか.
\end{remarks}

\begin{lemma}[局所凸位相線形空間である]\mbox{}
    \begin{enumerate}
        \item 線型空間$X$上の分離的な半ノルム列$\P$が誘導する始位相は第2可算な局所凸位相で,$E\subset X$が有界であることと$p:E\to\R$が有界であることとは同値になる.
        \item ノルムの列$\sup_{\abs{\al}\le N}\sup_{x\in\R^n}(1+\abs{x}^2)N\abs{(D_\al f)(x)}\;(N\in\N)$は,$\S_n$上に局所凸な位相を定める.
    \end{enumerate}
\end{lemma}

\begin{example}\mbox{}
    \begin{enumerate}
        \item 隆起関数はSchwartz関数である:$C^\infty_c(\R^n)\mono\cS(\R^n)$.
        \item Gauss関数も急減少する:$\forall_{i\in\N^n}\;\forall_{a>0}\;x^ie^{-a\abs{x}^2}\in S(\R^n)$.
    \end{enumerate}
\end{example}

\subsection{性質}

\begin{tcolorbox}[colframe=ForestGreen, colback=ForestGreen!10!white,breakable,colbacktitle=ForestGreen!40!white,coltitle=black,fonttitle=\bfseries\sffamily,
title=]
    $(\S_n,\delta,*)$は「部分群」をなし,
    Fourier変換は$\S_n$の自己同型で,$*$の構造と両立する.
\end{tcolorbox}

\begin{theorem}\mbox{}
    \begin{enumerate}
        \item $\S_n$はFrechet空間である.
        \item 次が定める線型写像$\S_n\to\S_n$は連続である:$f\mapsto Pf,f\mapsto gf,f\mapsto D_\al f$.
        \item $f\in\S_n,P\in \R[x]$について,$\wh{P(D)f}=P\wh{f}$かつ$\wh{Pf}=P(-D)\wh{f}$.
        \item (逆転定理) $\F|_{\S_n}$は周期$4$の線型同型(全単射な連続線型写像)$\S_n\to\S_n$である.
    \end{enumerate}
\end{theorem}

\begin{theorem}
    $f,g\in\S_n$とする.
    \begin{enumerate}
        \item $f*g\in\S_n$.
        \item $\wh{fg}=\wh{f}*\wh{g}$.
    \end{enumerate}
\end{theorem}

\section{緩増加分布空間}

\begin{tcolorbox}[colframe=ForestGreen, colback=ForestGreen!10!white,breakable,colbacktitle=ForestGreen!40!white,coltitle=black,fonttitle=\bfseries\sffamily,
title=]
    \cite{Rudin-FunctionalAnalysis}の7章.
\end{tcolorbox}

\subsection{定義と例}

\begin{theorem}\mbox{}
    \begin{enumerate}
        \item $\D(\R^n)\subset\S_n$は稠密である.
        \item 包含$i:\D(\R^n)\to\S_n$は連続である.
    \end{enumerate}
\end{theorem}

\begin{definition}[tempered distribution]
    $L\in\S_n^*$に対して,定理より$u_L:=L\circ i\in\D'(\R^n)$となる.
    また,$\D(\R^n)$の稠密性より,$L\mapsto u_L$の対応は単射である.こうして対応$\S'_n\ni L\mapsto u_L\in\D'(\R^n)$が定まる.
    この像に入る$\D'(\R^n)$の元を\textbf{緩増加分布}という.
    すなわち,$u\in\D'(\R^n)$であって,$\S_n$上への連続な延長を持つものをいう.
\end{definition}

\begin{example}[任意の可積分関数は緩増加である]\mbox{}
    \begin{enumerate}
        \item コンパクトな台を持つ超関数は緩増加である.
        \item $\R^n$上のBorel測度$\mu$であって,$\exists_{k\in\N}\;\int_{\R^n}(1+\abs{x}^2)^{-k}d\mu(x)<\infty$を満たすものは緩増加である.
        \item $g\in\L(\R^n)$が$\exists_{N>0,p\in\cointerval{1,\infty}}\;\int_{\R^n}\abs{(1+\abs{x}^2)^{-N}g(x)}^pdm_n(x)=C<\infty$を満たすならば緩増加である.
        \item 任意の$g\in L^p(\R^n)\;(p\in[1,\infty])$は緩増加である.
    \end{enumerate}
    $e^x\cos(e^x)$は緩増加であるが,$e^x$はそうではない.
\end{example}

\subsection{閉性}

\begin{theorem}
    $\al\in\N^m,P\in\R[x],g\in\S_n$とする.$u\in\S'_n$ならば,$D^\al u,Pu,gu\in\S'_n$である.
\end{theorem}

\subsection{Fourier変換}

\begin{tcolorbox}[colframe=ForestGreen, colback=ForestGreen!10!white,breakable,colbacktitle=ForestGreen!40!white,coltitle=black,fonttitle=\bfseries\sffamily,
title=]
    $\S_n$にはFourier変換が定まっているので,超関数の方法で$\S'_n$上にも定める.
\end{tcolorbox}

\begin{definition}
    $u\in\S'_n$のFourier変換を,
    $\wh{u}(\phi):=u(\wh{\phi})\;(\phi\in\S_n)$で定める.
\end{definition}

\begin{theorem}\mbox{}
    \begin{enumerate}
        \item $\F:\S'_n\mono\S'_n$は周期4の線型同型(全単射な連続線型写像であり,逆も連続線型)である.
        \item $u\in\S'_n,P\in\R[x]$のとき,$\wh{P(D)u}=P\wh{u}$かつ$\wh{Pu}=P(-D)\wh{u}$.
    \end{enumerate}
\end{theorem}

\begin{example}[多項式のFourier変換]
    
\end{example}

\begin{corollary}
    超関数$u$について,次の2条件は同値.
    \begin{enumerate}
        \item $u$はある多項式$P\in\S'_n$のFourier変換である.
        \item $\supp u$は$\{0\}$または$\emptyset$である.
    \end{enumerate}
\end{corollary}

\subsection{畳み込み}

\begin{tcolorbox}[colframe=ForestGreen, colback=ForestGreen!10!white,breakable,colbacktitle=ForestGreen!40!white,coltitle=black,fonttitle=\bfseries\sffamily,
title=]
    超関数の方法で畳み込みを定義する.
\end{tcolorbox}

\begin{lemma}[Schwartz空間での収束の特徴付け]
    $w:=(1,0,\cdots,0)\in\R^n,\phi\in\S_n$とし,
    \[\phi_\ep(x):=\frac{\phi(x+\ep w)-\phi(x)}{\ep}\quad(x\in\R^n,\ep>0)\]
    とすると,$\phi_\ep\xrightarrow{\ep\to0}\pp{\phi}{x_1}$.
\end{lemma}

\begin{definition}
    $u\in\S'_n,\phi\in\S_n$について,$(u*\phi)(x):=u(\tau_x\widecheck{\phi})\;(x\in\R^n)$と定める.
\end{definition}

\begin{theorem}
    $\phi\in\S_n,u\in\S'_n$とする.
    \begin{enumerate}
        \item $u*\phi\in C^\infty(\R^n)$かつ$\forall_{m\in\N,\al\in\N^m}\;D^\al(u*\phi)=(D^\al u)*\phi=u*(D^\al\phi)$.
        \item $u*\phi$は多項式のオーダーで増加し,特に緩増加である.
        \item $\wh{u*\phi}=\wh{\phi}\wh{u}$.
        \item $\forall_{\phi\in\S_n}\;(u*\phi)*\psi=u*(\phi*\psi)$.
        \item $\wh{u}*\wh{\phi}=\wh{\phi u}$.
    \end{enumerate}
\end{theorem}

\section{複素Borel測度のFourier変換}

\begin{theorem}
    $\mu:\B(\R^n)\to\C$を複素Borel測度とすると,特に緩増加分布でもある.このFourier変換
    \[\wh{\mu}(x):=\int_{\R^n}e^{-ix\cdot t}d\mu(t)\quad(x\in\R^n)\]
    について,
    \begin{enumerate}
        \item たしかに,緩増加分布としてのFourier変換に一致する.
        \item $\wh{\mu}$は有界かつ一様連続である.
    \end{enumerate}
\end{theorem}

\section{Paley-Wienerの定理}

\begin{tcolorbox}[colframe=ForestGreen, colback=ForestGreen!10!white,breakable,colbacktitle=ForestGreen!40!white,coltitle=black,fonttitle=\bfseries\sffamily,
title=コンパクト台を持つ緩増加関数のFourier変換は整関数を定める]
    関数の無限大での減衰挙動と,そのFourier変換の解析性との間に対応がつく.このクラスの結果をPaley-Wienerの定理という.
    古典的な結果は,$\R$への制限が$L^2$に属するような指数型整関数$\C\to\C$の特徴付けを与える定理である.
\end{tcolorbox}

\subsection{古典的結果}

\begin{notation}
    $rB=\Delta(0,r)$とする.
\end{notation}

\begin{lemma}
    整関数$f:\C^n\to\C$が$\R^n$にて消えるならば,$f=0$である.
\end{lemma}

\begin{theorem}
    試験関数$\phi\in\D(\R^n)$とそのFourier変換$f:\C^n\to\C$について,次の2条件は同値.
    \begin{enumerate}
        \item ある試験関数$\phi\in\D(\R^n)$が存在して$\exists_{r>0}\;\supp\phi\subset rB$を満たし,
        \[f(z)=\int_{\R^n}\phi(t)e^{-iz\cdot t}dm_n(t)\quad(z\in\C^n)\]
        と表せる.
        \item $f$は整関数で,$\forall_{N\in\N}\;\exists_{\gamma_N<\infty}\;\forall_{z\in\C^n}\abs{f(z)}\le\gamma_N(1+\abs{z})^{-N}e^{r\abs{\Im z}}$.
    \end{enumerate}
\end{theorem}

\subsection{緩増加超関数への一般化}

\begin{discussion}
    $u$を$\R^n$上のコンパクト台を持つ超関数とすると,緩増加である:$u\in\S'_n$.
    よって,$\wh{u}\in\S'_n$が$\wh{u}(\phi):=u(\wh{\phi})\;(\phi\in\S_n)$と定まるのであった.

    このとき,$\wh{u}(x)=u(e_{-x})\;(x\in\R^n)$となる.
    実際,$\wh{f}(x)=\int fe_xdm_n$に注意すると,
    \[\wh{u}(\phi)=\int\wh{u}(x)\phi(x)dx=\int u(x)\wh{\phi}(x)dx=u(\wh{\phi}).\]
    すなわち,超関数$\wh{u}:\R^n\to\R$は関数としてしっかり定まっている.
    これは解析接続によって$\C^n$上に延長する.
\end{discussion}

\begin{theorem}
    $u\in\D'(\R^n)$と$f:\C^n\to\C$について,次の2条件は同値.
    \begin{enumerate}
        \item $u\in\D'(\R^n)$であって$\exists_{r>0}\;\supp u\subset rB$を満たすものが存在し,$u$は次数$N$を持ち,$\forall_{z\in\C^n}\;f(z)=u(e_{-z})$を満たす.
        \item $f$は整関数で,$f|_{\R^n}$は$u$のFourier変換であり,また$\exists_{\gamma<\infty}\;\forall_{z\in\C^n}\;\abs{f(z)}\le\gamma(1+\abs{z})^Ne^{r\abs{\Im z}}$を満たす.
    \end{enumerate}
\end{theorem}

\begin{definition}
    この定理によって与えられる整関数$\wh{u}(z)=u(e_{-z})$を,$u$の\textbf{Fourier-Laplace変換}ともいう.
\end{definition}

\subsection{Laplace変換}

\begin{tcolorbox}[colframe=ForestGreen, colback=ForestGreen!10!white,breakable,colbacktitle=ForestGreen!40!white,coltitle=black,fonttitle=\bfseries\sffamily,
title=]
    普段は$\R_+$へ制限した退化した姿を見るが,$f$が有界なとき$\L f$は右半平面に正則な関数を定め,
    そして$f$が緩増加関数であるとき$\L f$は整関数を定める.
    そのTaylor級数展開は,関数をモーメントの線型和として表示する(積率母関数).
    その一部を取り出したものがMellin変換である.
\end{tcolorbox}

\begin{definition}
    関数$f:(0,\infty)\to\R$について,
    \begin{enumerate}
        \item $\L f(x):=\int^\infty_0f(t)e^{-tp}dt$を\textbf{Laplace変換}という.
        \item $\M f(x):=\int^\infty_0f(t)t^{x-1}dt$を\textbf{Mellin変換}という.
    \end{enumerate}
\end{definition}

\section{Sobolevの補題}

\chapter{時系列解析}

\section{弱定常確率過程}

\begin{enumerate}[({A}1)]
    \item 
\end{enumerate}

\begin{definition}
    過程$X:\Om\to L(T)$について,
    \begin{enumerate}
        \item 次の3条件を満たすとき,\textbf{弱定常過程}といい,$\rho$を\textbf{共分散過程}という.
        \begin{enumerate}[({W}1)]
            \item $T\to L^2(\Om)$を定める:$\forall_{t\in T}\;E[X^2_t]<\infty$.
            \item 定常平均性:$\forall_{t\in T}\;E[X_t]=m$.
            \item 定常性:$\forall_{s,t\in T}\;E[(X_s-m)\o{X_t-m}]=\rho(s-t)$.
        \end{enumerate}
        \item さらに次を満たすとき,\textbf{強定常過程}という.
        \begin{enumerate}[({S}1)]
            \item 任意の$t_1,\cdots,t_n\in T$について,確率変数族$(X_{t_1+t},\cdots,X_{t_n+t})_{t\in T}$は同分布である.
        \end{enumerate}
    \end{enumerate}
\end{definition}

\begin{theorem}[弱定常過程の構成]
    $c_n\in l^2(\Z;\C)$と,同一の分散$\sigma^2$を持つ直交確率変数列$\{X_n\}_{n\in\Z}\subset L_0^2(\Om)$について,
    \[Y_n:=\sum_{k\in\Z}c_{n-k}X_k\]
    は$L^2$-収束し,中心化された弱定常過程を定める.その共分散列は$\rho_n=\sigma^2\sum_{\nu\in\Z}c_{n+\nu}\o{c_\nu}$と表せる.
\end{theorem}

\begin{theorem}
    $X:\Om\to L(\R)$を弱定常過程,$\rho$を共分散関数とする.
    \begin{enumerate}
        \item $\rho$が$u=0$で連続ならば,$\rho$は全域で連続である.
        \item $X:\R\to L^2(\Om)$が$t=0$で連続ならば,$\R$全域で連続である.
        \item $X:\R\to L^2(\Om)$が連続であることと,$\rho:\R\to\R$が連続であることとは同値.
    \end{enumerate}
\end{theorem}

\begin{theorem}[Crum]
    可測な弱定常過程は,メイト$X:\R\to L^2(\Om)$が$L^2$-連続である.特に,その共分散関数は連続である.
\end{theorem}

\section{共分散関数のスペクトル表現}

\begin{theorem}
    $X:\R\to L^2_0(\Om)$を中心化された可測な弱定常過程とする.
    ある分布関数$F$が存在して,その共分散関数$\rho$は特性関数
    \[\rho(u)=\int_\R e^{iu\lambda}dF(\lambda)\]
    で表せる.この$F$を$X$の\textbf{スペクトル分布関数}という.
\end{theorem}
\begin{remark}
    $\R=\Z$と変更した場合は,積分区間は$[-\pi,\pi]$になる.
\end{remark}

\begin{theorem}
    逆に,任意の分布関数$F$は,1つの可測弱定常過程が存在して,$F$の特性関数がその共分散関数となる.
\end{theorem}

\section{彷徨測度と確率積分}

\begin{tcolorbox}[colframe=ForestGreen, colback=ForestGreen!10!white,breakable,colbacktitle=ForestGreen!40!white,coltitle=black,fonttitle=\bfseries\sffamily,
title=]
    ホワイトノイズを彷徨測度といっているのか??
\end{tcolorbox}

\section{弱定常過程のスペクトル分解}

\section{移動平均過程}

\section{大数の法則}

\begin{definition}
    $\frac{1}{T}\int^T_0X_tdt$が$T\to\infty$の極限で$L^2$-収束するとき,\textbf{弱大数の法則に従う}という.
    さらに極限が$E[X_t](=m)$に等しいとき,\textbf{エルゴード的}であるという.
\end{definition}

\begin{theorem}
    可測弱定常過程$\R\to L^2(\Om)$について,次の2条件は同値:
    \begin{enumerate}
        \item エルゴード的である.
        \item $X$のスペクトル分布関数$F$について,$F(0+)-F(0)=0$.
    \end{enumerate}
\end{theorem}

\chapter{参考文献}

\bibliography{../StatisticalSciences.bib,../SocialSciences.bib,../mathematics.bib,../statistics.bib}
\begin{thebibliography}{99}
    \item{大島}
    大島利雄,小松彦三郎 (1977). 『1階偏微分方程式』(岩波基礎数学,岩波講座基礎数学,解析学(II) iii).
    \item{小松-Distribution}
    小松彦三郎 (1978). 『超関数論入門』(岩波講座基礎数学,解析学(II) ix).
    \item{金子}
    金子晃 (1976). 『定数係数線型偏微分方程式』(岩波講座基礎数学,解析学(II) v).
    \item{小松-Fourier}
    小松彦三郎 (1978). 『Fourier解析』(岩波講座基礎数学,解析学(II) vi).
    \item{Katznelson}
    Y. Katznelson (2004). \textit{An introduction to harmonic analysis}. Third edition, Cambridge Math. Lib., Cambridge University Press, Cambridge.

    \item{猪狩}
    猪狩惺 (1975) 『フーリエ級数』(岩波書店)
    \item{Korevaar}
    J. Korevaar \textit{Fourier Analysis and Related Topics}
    \item{河田}
    河田龍夫 (1985) 『フーリエ解析と統計』(応用統計数学シリーズ,共立出版)
    \item{河添}
    河添健 (2000) 『群上の調和解析』(すうがくの風景,朝倉書店).
    \item{岡本・展望}
    岡本清郷 (1997) 『フーリエ解析の展望』(すうがくぶっくす,朝倉書店).
    \item{高橋陽一郎}
    高橋陽一郎『実関数とFourier解析』
    \item{Rudin}
    Rudin \textit{Real and Complex Analysis} 3rd
    \item{Rudin-FA}
    Rudin \textit{Functional Analysis}
\end{thebibliography}

\end{document}