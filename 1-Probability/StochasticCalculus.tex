\documentclass[uplatex,dvipdfmx]{jsreport}
\title{確率解析}
\author{司馬博文}
\date{\today}
\pagestyle{headings} \setcounter{secnumdepth}{4}
%%%%%%%%%%%%%%% 数理文書の組版 %%%%%%%%%%%%%%%

\usepackage{mathtools} %内部でamsmathを呼び出すことに注意.
%\mathtoolsset{showonlyrefs=true} %labelを附した数式にのみ附番される設定.
\usepackage{amsfonts} %mathfrak, mathcal, mathbbなど.
\usepackage{amsthm} %定理環境.
\usepackage{amssymb} %AMSFontsを使うためのパッケージ.
\usepackage{ascmac} %screen, itembox, shadebox環境.全てLATEX2εの標準機能の範囲で作られたもの.
\usepackage{comment} %comment環境を用いて,複数行をcomment outできるようにするpackage
\usepackage{wrapfig} %図の周りに文字をwrapさせることができる.詳細な制御ができる.
\usepackage[usenames, dvipsnames]{xcolor} %xcolorはcolorの拡張.optionの意味はdvipsnamesはLoad a set of predefined colors. forestgreenなどの色が追加されている.usenamesはobsoleteとだけ書いてあった.
\setcounter{tocdepth}{2} %目次に表示される深さ.2はsubsectionまで
\usepackage{multicol} %\begin{multicols}{2}環境で途中からmulticolumnに出来る.
\usepackage{mathabx}\newcommand{\wc}{\widecheck} %\widecheckなどのフォントパッケージ

%%%%%%%%%%%%%%% フォント %%%%%%%%%%%%%%%

\usepackage{textcomp, mathcomp} %Text Companionとは,T1 encodingに入らなかった文字群.これを使うためのパッケージ.\textsectionでブルバキに!
\usepackage[T1]{fontenc} %8bitエンコーディングにする.comp系拡張数学文字の動作が安定する.

%%%%%%%%%%%%%%% 一般文書の組版 %%%%%%%%%%%%%%%

\definecolor{花緑青}{cmyk}{1,0.07,0.10,0.10}\definecolor{サーモンピンク}{cmyk}{0,0.65,0.65,0.05}\definecolor{暗中模索}{rgb}{0.2,0.2,0.2}
\usepackage{url}\usepackage[dvipdfmx,colorlinks,linkcolor=花緑青,urlcolor=花緑青,citecolor=花緑青]{hyperref} %生成されるPDFファイルにおいて、\tableofcontentsによって書き出された目次をクリックすると該当する見出しへジャンプしたり、さらには、\label{ラベル名}を番号で参照する\ref{ラベル名}やthebibliography環境において\bibitem{ラベル名}を文献番号で参照する\cite{ラベル名}においても番号をクリックすると該当箇所にジャンプする.囲み枠はダサいので,colorlinksで囲み廃止し,リンク自体に色を付けることにした.
\usepackage{pxjahyper} %pxrubrica同様,八登崇之さん.hyperrefは日本語pLaTeXに最適化されていないから,hyperrefとセットで,(u)pLaTeX+hyperref+dvipdfmxの組み合わせで日本語を含む「しおり」をもつPDF文書を作成する場合に必要となる機能を提供する
\usepackage{ulem} %取り消し線を引くためのパッケージ
\usepackage{pxrubrica} %日本語にルビをふる.八登崇之(やとうたかゆき)氏による.

%%%%%%%%%%%%%%% 科学文書の組版 %%%%%%%%%%%%%%%

\usepackage[version=4]{mhchem} %化学式をTikZで簡単に書くためのパッケージ.
\usepackage{chemfig} %化学構造式をTikZで描くためのパッケージ.
\usepackage{siunitx} %IS単位を書くためのパッケージ

%%%%%%%%%%%%%%% 作図 %%%%%%%%%%%%%%%

\usepackage{tikz}\usetikzlibrary{positioning,automata}\usepackage{tikz-cd}\usepackage[all]{xy}
\def\objectstyle{\displaystyle} %デフォルトではxymatrix中の数式が文中数式モードになるので,それを直す.\labelstyleも同様にxy packageの中で定義されており,文中数式モードになっている.

\usepackage{graphicx} %rotatebox, scalebox, reflectbox, resizeboxなどのコマンドや,図表の読み込み\includegraphicsを司る.graphics というパッケージもありますが,graphicx はこれを高機能にしたものと考えて結構です(ただし graphicx は内部で graphics を読み込みます)
\usepackage[top=15truemm,bottom=15truemm,left=10truemm,right=10truemm]{geometry} %足助さんからもらったオプション

%%%%%%%%%%%%%%% 参照 %%%%%%%%%%%%%%%
%参考文献リストを出力したい箇所に\bibliography{../mathematics.bib}を追記すると良い.

%\bibliographystyle{jplain}
%\bibliographystyle{jname}
\bibliographystyle{apalike}

%%%%%%%%%%%%%%% 計算機文書の組版 %%%%%%%%%%%%%%%

\usepackage[breakable]{tcolorbox} %加藤晃史さんがフル活用していたtcolorboxを,途中改ページ可能で.
\tcbuselibrary{theorems} %https://qiita.com/t_kemmochi/items/483b8fcdb5db8d1f5d5e
\usepackage{enumerate} %enumerate環境を凝らせる.

\usepackage{listings} %ソースコードを表示できる環境.多分もっといい方法ある.
\usepackage{jvlisting} %日本語のコメントアウトをする場合jlistingが必要
\lstset{ %ここからソースコードの表示に関する設定.lstlisting環境では,[caption=hoge,label=fuga]などのoptionを付けられる.
%[escapechar=!]とすると,LaTeXコマンドを使える.
  basicstyle={\ttfamily},
  identifierstyle={\small},
  commentstyle={\smallitshape},
  keywordstyle={\small\bfseries},
  ndkeywordstyle={\small},
  stringstyle={\small\ttfamily},
  frame={tb},
  breaklines=true,
  columns=[l]{fullflexible},
  numbers=left,
  xrightmargin=0zw,
  xleftmargin=3zw,
  numberstyle={\scriptsize},
  stepnumber=1,
  numbersep=1zw,
  lineskip=-0.5ex
}
%\makeatletter %caption番号を「[chapter番号].[section番号].[subsection番号]-[そのsubsection内においてn番目]」に変更
%    \AtBeginDocument{
%    \renewcommand*{\thelstlisting}{\arabic{chapter}.\arabic{section}.\arabic{lstlisting}}
%    \@addtoreset{lstlisting}{section}
%    }
%\makeatother
\renewcommand{\lstlistingname}{算譜} %caption名を"program"に変更

\newtcolorbox{tbox}[3][]{%
colframe=#2,colback=#2!10,coltitle=#2!20!black,title={#3},#1}

% 証明内の文字が小さくなる環境.
\newenvironment{Proof}[1][\bf\underline{[証明]}]{\proof[#1]\color{darkgray}}{\endproof}

%%%%%%%%%%%%%%% 数学記号のマクロ %%%%%%%%%%%%%%%

%%% 括弧類
\newcommand{\abs}[1]{\lvert#1\rvert}\newcommand{\Abs}[1]{\left|#1\right|}\newcommand{\norm}[1]{\|#1\|}\newcommand{\Norm}[1]{\left\|#1\right\|}\newcommand{\Brace}[1]{\left\{#1\right\}}\newcommand{\BRace}[1]{\biggl\{#1\biggr\}}\newcommand{\paren}[1]{\left(#1\right)}\newcommand{\Paren}[1]{\biggr(#1\biggl)}\newcommand{\bracket}[1]{\langle#1\rangle}\newcommand{\brac}[1]{\langle#1\rangle}\newcommand{\Bracket}[1]{\left\langle#1\right\rangle}\newcommand{\Brac}[1]{\left\langle#1\right\rangle}\newcommand{\bra}[1]{\left\langle#1\right|}\newcommand{\ket}[1]{\left|#1\right\rangle}\newcommand{\Square}[1]{\left[#1\right]}\newcommand{\SQuare}[1]{\biggl[#1\biggr]}
\renewcommand{\o}[1]{\overline{#1}}\renewcommand{\u}[1]{\underline{#1}}\newcommand{\wt}[1]{\widetilde{#1}}\newcommand{\wh}[1]{\widehat{#1}}
\newcommand{\pp}[2]{\frac{\partial #1}{\partial #2}}\newcommand{\ppp}[3]{\frac{\partial #1}{\partial #2\partial #3}}\newcommand{\dd}[2]{\frac{d #1}{d #2}}
\newcommand{\floor}[1]{\lfloor#1\rfloor}\newcommand{\Floor}[1]{\left\lfloor#1\right\rfloor}\newcommand{\ceil}[1]{\lceil#1\rceil}
\newcommand{\ocinterval}[1]{(#1]}\newcommand{\cointerval}[1]{[#1)}\newcommand{\COinterval}[1]{\left[#1\right)}


%%% 予約語
\renewcommand{\iff}{\;\mathrm{iff}\;}
\newcommand{\False}{\mathrm{False}}\newcommand{\True}{\mathrm{True}}
\newcommand{\otherwise}{\mathrm{otherwise}}
\newcommand{\st}{\;\mathrm{s.t.}\;}

%%% 略記
\newcommand{\M}{\mathcal{M}}\newcommand{\cF}{\mathcal{F}}\newcommand{\cD}{\mathcal{D}}\newcommand{\fX}{\mathfrak{X}}\newcommand{\fY}{\mathfrak{Y}}\newcommand{\fZ}{\mathfrak{Z}}\renewcommand{\H}{\mathcal{H}}\newcommand{\fH}{\mathfrak{H}}\newcommand{\bH}{\mathbb{H}}\newcommand{\id}{\mathrm{id}}\newcommand{\A}{\mathcal{A}}\newcommand{\U}{\mathfrak{U}}
\newcommand{\lmd}{\lambda}
\newcommand{\Lmd}{\Lambda}

%%% 矢印類
\newcommand{\iso}{\xrightarrow{\,\smash{\raisebox{-0.45ex}{\ensuremath{\scriptstyle\sim}}}\,}}
\newcommand{\Lrarrow}{\;\;\Leftrightarrow\;\;}

%%% 注記
\newcommand{\rednote}[1]{\textcolor{red}{#1}}

% ノルム位相についての閉包 https://newbedev.com/how-to-make-double-overline-with-less-vertical-displacement
\makeatletter
\newcommand{\dbloverline}[1]{\overline{\dbl@overline{#1}}}
\newcommand{\dbl@overline}[1]{\mathpalette\dbl@@overline{#1}}
\newcommand{\dbl@@overline}[2]{%
  \begingroup
  \sbox\z@{$\m@th#1\overline{#2}$}%
  \ht\z@=\dimexpr\ht\z@-2\dbl@adjust{#1}\relax
  \box\z@
  \ifx#1\scriptstyle\kern-\scriptspace\else
  \ifx#1\scriptscriptstyle\kern-\scriptspace\fi\fi
  \endgroup
}
\newcommand{\dbl@adjust}[1]{%
  \fontdimen8
  \ifx#1\displaystyle\textfont\else
  \ifx#1\textstyle\textfont\else
  \ifx#1\scriptstyle\scriptfont\else
  \scriptscriptfont\fi\fi\fi 3
}
\makeatother
\newcommand{\oo}[1]{\dbloverline{#1}}

% hslashの他の文字Ver.
\newcommand{\hslashslash}{%
    \scalebox{1.2}{--
    }%
}
\newcommand{\dslash}{%
  {%
    \vphantom{d}%
    \ooalign{\kern.05em\smash{\hslashslash}\hidewidth\cr$d$\cr}%
    \kern.05em
  }%
}
\newcommand{\dint}{%
  {%
    \vphantom{d}%
    \ooalign{\kern.05em\smash{\hslashslash}\hidewidth\cr$\int$\cr}%
    \kern.05em
  }%
}
\newcommand{\dL}{%
  {%
    \vphantom{d}%
    \ooalign{\kern.05em\smash{\hslashslash}\hidewidth\cr$L$\cr}%
    \kern.05em
  }%
}

%%% 演算子
\DeclareMathOperator{\grad}{\mathrm{grad}}\DeclareMathOperator{\rot}{\mathrm{rot}}\DeclareMathOperator{\divergence}{\mathrm{div}}\DeclareMathOperator{\tr}{\mathrm{tr}}\newcommand{\pr}{\mathrm{pr}}
\newcommand{\Map}{\mathrm{Map}}\newcommand{\dom}{\mathrm{Dom}\;}\newcommand{\cod}{\mathrm{Cod}\;}\newcommand{\supp}{\mathrm{supp}\;}


%%% 線型代数学
\newcommand{\vctr}[2]{\begin{pmatrix}#1\\#2\end{pmatrix}}\newcommand{\vctrr}[3]{\begin{pmatrix}#1\\#2\\#3\end{pmatrix}}\newcommand{\mtrx}[4]{\begin{pmatrix}#1&#2\\#3&#4\end{pmatrix}}\newcommand{\smtrx}[4]{\paren{\begin{smallmatrix}#1&#2\\#3&#4\end{smallmatrix}}}\newcommand{\Ker}{\mathrm{Ker}\;}\newcommand{\Coker}{\mathrm{Coker}\;}\newcommand{\Coim}{\mathrm{Coim}\;}\DeclareMathOperator{\rank}{\mathrm{rank}}\newcommand{\lcm}{\mathrm{lcm}}\newcommand{\sgn}{\mathrm{sgn}\,}\newcommand{\GL}{\mathrm{GL}}\newcommand{\SL}{\mathrm{SL}}\newcommand{\alt}{\mathrm{alt}}
%%% 複素解析学
\renewcommand{\Re}{\mathrm{Re}\;}\renewcommand{\Im}{\mathrm{Im}\;}\newcommand{\Gal}{\mathrm{Gal}}\newcommand{\PGL}{\mathrm{PGL}}\newcommand{\PSL}{\mathrm{PSL}}\newcommand{\Log}{\mathrm{Log}\,}\newcommand{\Res}{\mathrm{Res}\,}\newcommand{\on}{\mathrm{on}\;}\newcommand{\hatC}{\widehat{\C}}\newcommand{\hatR}{\hat{\R}}\newcommand{\PV}{\mathrm{P.V.}}\newcommand{\diam}{\mathrm{diam}}\newcommand{\Area}{\mathrm{Area}}\newcommand{\Lap}{\Laplace}\newcommand{\f}{\mathbf{f}}\newcommand{\cR}{\mathcal{R}}\newcommand{\const}{\mathrm{const.}}\newcommand{\Om}{\Omega}\newcommand{\Cinf}{C^\infty}\newcommand{\ep}{\epsilon}\newcommand{\dist}{\mathrm{dist}}\newcommand{\opart}{\o{\partial}}\newcommand{\Length}{\mathrm{Length}}
%%% 集合と位相
\renewcommand{\O}{\mathcal{O}}\renewcommand{\S}{\mathcal{S}}\renewcommand{\U}{\mathcal{U}}\newcommand{\V}{\mathcal{V}}\renewcommand{\P}{\mathcal{P}}\newcommand{\R}{\mathbb{R}}\newcommand{\N}{\mathbb{N}}\newcommand{\C}{\mathbb{C}}\newcommand{\Z}{\mathbb{Z}}\newcommand{\Q}{\mathbb{Q}}\newcommand{\TV}{\mathrm{TV}}\newcommand{\ORD}{\mathrm{ORD}}\newcommand{\Tr}{\mathrm{Tr}}\newcommand{\Card}{\mathrm{Card}\;}\newcommand{\Top}{\mathrm{Top}}\newcommand{\Disc}{\mathrm{Disc}}\newcommand{\Codisc}{\mathrm{Codisc}}\newcommand{\CoDisc}{\mathrm{CoDisc}}\newcommand{\Ult}{\mathrm{Ult}}\newcommand{\ord}{\mathrm{ord}}\newcommand{\maj}{\mathrm{maj}}\newcommand{\bS}{\mathbb{S}}\newcommand{\PConn}{\mathrm{PConn}}

%%% 形式言語理論
\newcommand{\REGEX}{\mathrm{REGEX}}\newcommand{\RE}{\mathbf{RE}}
%%% Graph Theory
\newcommand{\SimpGph}{\mathrm{SimpGph}}\newcommand{\Gph}{\mathrm{Gph}}\newcommand{\mult}{\mathrm{mult}}\newcommand{\inv}{\mathrm{inv}}

%%% 多様体
\newcommand{\Der}{\mathrm{Der}}\newcommand{\osub}{\overset{\mathrm{open}}{\subset}}\newcommand{\osup}{\overset{\mathrm{open}}{\supset}}\newcommand{\al}{\alpha}\newcommand{\K}{\mathbb{K}}\newcommand{\Sp}{\mathrm{Sp}}\newcommand{\g}{\mathfrak{g}}\newcommand{\h}{\mathfrak{h}}\newcommand{\Exp}{\mathrm{Exp}\;}\newcommand{\Imm}{\mathrm{Imm}}\newcommand{\Imb}{\mathrm{Imb}}\newcommand{\codim}{\mathrm{codim}\;}\newcommand{\Gr}{\mathrm{Gr}}
%%% 代数
\newcommand{\Ad}{\mathrm{Ad}}\newcommand{\finsupp}{\mathrm{fin\;supp}}\newcommand{\SO}{\mathrm{SO}}\newcommand{\SU}{\mathrm{SU}}\newcommand{\acts}{\curvearrowright}\newcommand{\mono}{\hookrightarrow}\newcommand{\epi}{\twoheadrightarrow}\newcommand{\Stab}{\mathrm{Stab}}\newcommand{\nor}{\mathrm{nor}}\newcommand{\T}{\mathbb{T}}\newcommand{\Aff}{\mathrm{Aff}}\newcommand{\rsub}{\triangleleft}\newcommand{\rsup}{\triangleright}\newcommand{\subgrp}{\overset{\mathrm{subgrp}}{\subset}}\newcommand{\Ext}{\mathrm{Ext}}\newcommand{\sbs}{\subset}\newcommand{\sps}{\supset}\newcommand{\In}{\mathrm{in}\;}\newcommand{\Tor}{\mathrm{Tor}}\newcommand{\p}{\b{p}}\newcommand{\q}{\mathfrak{q}}\newcommand{\m}{\mathfrak{m}}\newcommand{\cS}{\mathcal{S}}\newcommand{\Frac}{\mathrm{Frac}\,}\newcommand{\Spec}{\mathrm{Spec}\,}\newcommand{\bA}{\mathbb{A}}\newcommand{\Sym}{\mathrm{Sym}}\newcommand{\Ann}{\mathrm{Ann}}\newcommand{\Her}{\mathrm{Her}}\newcommand{\Bil}{\mathrm{Bil}}\newcommand{\Ses}{\mathrm{Ses}}\newcommand{\FVS}{\mathrm{FVS}}
%%% 代数的位相幾何学
\newcommand{\Ho}{\mathrm{Ho}}\newcommand{\CW}{\mathrm{CW}}\newcommand{\lc}{\mathrm{lc}}\newcommand{\cg}{\mathrm{cg}}\newcommand{\Fib}{\mathrm{Fib}}\newcommand{\Cyl}{\mathrm{Cyl}}\newcommand{\Ch}{\mathrm{Ch}}
%%% 微分幾何学
\newcommand{\rE}{\mathrm{E}}\newcommand{\e}{\b{e}}\renewcommand{\k}{\b{k}}\newcommand{\Christ}[2]{\begin{Bmatrix}#1\\#2\end{Bmatrix}}\renewcommand{\Vec}[1]{\overrightarrow{\mathrm{#1}}}\newcommand{\hen}[1]{\mathrm{#1}}\renewcommand{\b}[1]{\boldsymbol{#1}}

%%% 函数解析
\newcommand{\HS}{\mathrm{HS}}\newcommand{\loc}{\mathrm{loc}}\newcommand{\Lh}{\mathrm{L.h.}}\newcommand{\Epi}{\mathrm{Epi}\;}\newcommand{\slim}{\mathrm{slim}}\newcommand{\Ban}{\mathrm{Ban}}\newcommand{\Hilb}{\mathrm{Hilb}}\newcommand{\Ex}{\mathrm{Ex}}\newcommand{\Co}{\mathrm{Co}}\newcommand{\sa}{\mathrm{sa}}\newcommand{\nnorm}[1]{{\left\vert\kern-0.25ex\left\vert\kern-0.25ex\left\vert #1 \right\vert\kern-0.25ex\right\vert\kern-0.25ex\right\vert}}\newcommand{\dvol}{\mathrm{dvol}}\newcommand{\Sconv}{\mathrm{Sconv}}\newcommand{\I}{\mathcal{I}}\newcommand{\nonunital}{\mathrm{nu}}\newcommand{\cpt}{\mathrm{cpt}}\newcommand{\lcpt}{\mathrm{lcpt}}\newcommand{\com}{\mathrm{com}}\newcommand{\Haus}{\mathrm{Haus}}\newcommand{\proper}{\mathrm{proper}}\newcommand{\infinity}{\mathrm{inf}}\newcommand{\TVS}{\mathrm{TVS}}\newcommand{\ess}{\mathrm{ess}}\newcommand{\ext}{\mathrm{ext}}\newcommand{\Index}{\mathrm{Index}\;}\newcommand{\SSR}{\mathrm{SSR}}\newcommand{\vs}{\mathrm{vs.}}\newcommand{\fM}{\mathfrak{M}}\newcommand{\EDM}{\mathrm{EDM}}\newcommand{\Tw}{\mathrm{Tw}}\newcommand{\fC}{\mathfrak{C}}\newcommand{\bn}{\boldsymbol{n}}\newcommand{\br}{\boldsymbol{r}}\newcommand{\Lam}{\Lambda}\newcommand{\lam}{\lambda}\newcommand{\one}{\mathbf{1}}\newcommand{\dae}{\text{-a.e.}}\newcommand{\das}{\text{-a.s.}}\newcommand{\td}{\text{-}}\newcommand{\RM}{\mathrm{RM}}\newcommand{\BV}{\mathrm{BV}}\newcommand{\normal}{\mathrm{normal}}\newcommand{\lub}{\mathrm{lub}\;}\newcommand{\Graph}{\mathrm{Graph}}\newcommand{\Ascent}{\mathrm{Ascent}}\newcommand{\Descent}{\mathrm{Descent}}\newcommand{\BIL}{\mathrm{BIL}}\newcommand{\fL}{\mathfrak{L}}\newcommand{\De}{\Delta}
%%% 積分論
\newcommand{\calA}{\mathcal{A}}\newcommand{\calB}{\mathcal{B}}\newcommand{\D}{\mathcal{D}}\newcommand{\Y}{\mathcal{Y}}\newcommand{\calC}{\mathcal{C}}\renewcommand{\ae}{\mathrm{a.e.}\;}\newcommand{\cZ}{\mathcal{Z}}\newcommand{\fF}{\mathfrak{F}}\newcommand{\fI}{\mathfrak{I}}\newcommand{\E}{\mathcal{E}}\newcommand{\sMap}{\sigma\textrm{-}\mathrm{Map}}\DeclareMathOperator*{\argmax}{arg\,max}\DeclareMathOperator*{\argmin}{arg\,min}\newcommand{\cC}{\mathcal{C}}\newcommand{\comp}{\complement}\newcommand{\J}{\mathcal{J}}\newcommand{\sumN}[1]{\sum_{#1\in\N}}\newcommand{\cupN}[1]{\cup_{#1\in\N}}\newcommand{\capN}[1]{\cap_{#1\in\N}}\newcommand{\Sum}[1]{\sum_{#1=1}^\infty}\newcommand{\sumn}{\sum_{n=1}^\infty}\newcommand{\summ}{\sum_{m=1}^\infty}\newcommand{\sumk}{\sum_{k=1}^\infty}\newcommand{\sumi}{\sum_{i=1}^\infty}\newcommand{\sumj}{\sum_{j=1}^\infty}\newcommand{\cupn}{\cup_{n=1}^\infty}\newcommand{\capn}{\cap_{n=1}^\infty}\newcommand{\cupk}{\cup_{k=1}^\infty}\newcommand{\cupi}{\cup_{i=1}^\infty}\newcommand{\cupj}{\cup_{j=1}^\infty}\newcommand{\limn}{\lim_{n\to\infty}}\renewcommand{\l}{\mathcal{l}}\renewcommand{\L}{\mathcal{L}}\newcommand{\Cl}{\mathrm{Cl}}\newcommand{\cN}{\mathcal{N}}\newcommand{\Ae}{\textrm{-a.e.}\;}\newcommand{\csub}{\overset{\textrm{closed}}{\subset}}\newcommand{\csup}{\overset{\textrm{closed}}{\supset}}\newcommand{\wB}{\wt{B}}\newcommand{\cG}{\mathcal{G}}\newcommand{\Lip}{\mathrm{Lip}}\DeclareMathOperator{\Dom}{\mathrm{Dom}}\newcommand{\AC}{\mathrm{AC}}\newcommand{\Mol}{\mathrm{Mol}}
%%% Fourier解析
\newcommand{\Pe}{\mathrm{Pe}}\newcommand{\wR}{\wh{\mathbb{\R}}}\newcommand*{\Laplace}{\mathop{}\!\mathbin\bigtriangleup}\newcommand*{\DAlambert}{\mathop{}\!\mathbin\Box}\newcommand{\bT}{\mathbb{T}}\newcommand{\dx}{\dslash x}\newcommand{\dt}{\dslash t}\newcommand{\ds}{\dslash s}
%%% 数値解析
\newcommand{\round}{\mathrm{round}}\newcommand{\cond}{\mathrm{cond}}\newcommand{\diag}{\mathrm{diag}}
\newcommand{\Adj}{\mathrm{Adj}}\newcommand{\Pf}{\mathrm{Pf}}\newcommand{\Sg}{\mathrm{Sg}}

%%% 確率論
\newcommand{\Prob}{\mathrm{Prob}}\newcommand{\X}{\mathcal{X}}\newcommand{\Meas}{\mathrm{Meas}}\newcommand{\as}{\;\mathrm{a.s.}}\newcommand{\io}{\;\mathrm{i.o.}}\newcommand{\fe}{\;\mathrm{f.e.}}\newcommand{\F}{\mathcal{F}}\newcommand{\bF}{\mathbb{F}}\newcommand{\W}{\mathcal{W}}\newcommand{\Pois}{\mathrm{Pois}}\newcommand{\iid}{\mathrm{i.i.d.}}\newcommand{\wconv}{\rightsquigarrow}\newcommand{\Var}{\mathrm{Var}}\newcommand{\xrightarrown}{\xrightarrow{n\to\infty}}\newcommand{\au}{\mathrm{au}}\newcommand{\cT}{\mathcal{T}}\newcommand{\wto}{\overset{w}{\to}}\newcommand{\dto}{\overset{d}{\to}}\newcommand{\pto}{\overset{p}{\to}}\newcommand{\vto}{\overset{v}{\to}}\newcommand{\Cont}{\mathrm{Cont}}\newcommand{\stably}{\mathrm{stably}}\newcommand{\Np}{\mathbb{N}^+}\newcommand{\oM}{\overline{\mathcal{M}}}\newcommand{\fP}{\mathfrak{P}}\newcommand{\sign}{\mathrm{sign}}\DeclareMathOperator{\Div}{Div}
\newcommand{\bD}{\mathbb{D}}\newcommand{\fW}{\mathfrak{W}}\newcommand{\DL}{\mathcal{D}\mathcal{L}}\renewcommand{\r}[1]{\mathrm{#1}}\newcommand{\rC}{\mathrm{C}}
%%% 情報理論
\newcommand{\bit}{\mathrm{bit}}\DeclareMathOperator{\sinc}{sinc}
%%% 量子論
\newcommand{\err}{\mathrm{err}}
%%% 最適化
\newcommand{\varparallel}{\mathbin{\!/\mkern-5mu/\!}}\newcommand{\Minimize}{\text{Minimize}}\newcommand{\subjectto}{\text{subject to}}\newcommand{\Ri}{\mathrm{Ri}}\newcommand{\Cone}{\mathrm{Cone}}\newcommand{\Int}{\mathrm{Int}}
%%% 数理ファイナンス
\newcommand{\pre}{\mathrm{pre}}\newcommand{\om}{\omega}

%%% 偏微分方程式
\let\div\relax
\DeclareMathOperator{\div}{div}\newcommand{\del}{\partial}
\newcommand{\LHS}{\mathrm{LHS}}\newcommand{\RHS}{\mathrm{RHS}}\newcommand{\bnu}{\boldsymbol{\nu}}\newcommand{\interior}{\mathrm{in}\;}\newcommand{\SH}{\mathrm{SH}}\renewcommand{\v}{\boldsymbol{\nu}}\newcommand{\n}{\mathbf{n}}\newcommand{\ssub}{\Subset}\newcommand{\curl}{\mathrm{curl}}
%%% 常微分方程式
\newcommand{\Ei}{\mathrm{Ei}}\newcommand{\sn}{\mathrm{sn}}\newcommand{\wgamma}{\widetilde{\gamma}}
%%% 統計力学
\newcommand{\Ens}{\mathrm{Ens}}
%%% 解析力学
\newcommand{\cl}{\mathrm{cl}}\newcommand{\x}{\boldsymbol{x}}

%%% 統計的因果推論
\newcommand{\Do}{\mathrm{Do}}
%%% 応用統計学
\newcommand{\mrl}{\mathrm{mrl}}
%%% 数理統計
\newcommand{\comb}[2]{\begin{pmatrix}#1\\#2\end{pmatrix}}\newcommand{\bP}{\mathbb{P}}\newcommand{\compsub}{\overset{\textrm{cpt}}{\subset}}\newcommand{\lip}{\textrm{lip}}\newcommand{\BL}{\mathrm{BL}}\newcommand{\G}{\mathbb{G}}\newcommand{\NB}{\mathrm{NB}}\newcommand{\oR}{\o{\R}}\newcommand{\liminfn}{\liminf_{n\to\infty}}\newcommand{\limsupn}{\limsup_{n\to\infty}}\newcommand{\esssup}{\mathrm{ess.sup}}\newcommand{\asto}{\xrightarrow{\as}}\newcommand{\Cov}{\mathrm{Cov}}\newcommand{\cQ}{\mathcal{Q}}\newcommand{\VC}{\mathrm{VC}}\newcommand{\mb}{\mathrm{mb}}\newcommand{\Avar}{\mathrm{Avar}}\newcommand{\bB}{\mathbb{B}}\newcommand{\bW}{\mathbb{W}}\newcommand{\sd}{\mathrm{sd}}\newcommand{\w}[1]{\widehat{#1}}\newcommand{\bZ}{\boldsymbol{Z}}\newcommand{\Bernoulli}{\mathrm{Ber}}\newcommand{\Ber}{\mathrm{Ber}}\newcommand{\Mult}{\mathrm{Mult}}\newcommand{\BPois}{\mathrm{BPois}}\newcommand{\fraks}{\mathfrak{s}}\newcommand{\frakk}{\mathfrak{k}}\newcommand{\IF}{\mathrm{IF}}\newcommand{\bX}{\mathbf{X}}\newcommand{\bx}{\boldsymbol{x}}\newcommand{\indep}{\raisebox{0.05em}{\rotatebox[origin=c]{90}{$\models$}}}\newcommand{\IG}{\mathrm{IG}}\newcommand{\Levy}{\mathrm{Levy}}\newcommand{\MP}{\mathrm{MP}}\newcommand{\Hermite}{\mathrm{Hermite}}\newcommand{\Skellam}{\mathrm{Skellam}}\newcommand{\Dirichlet}{\mathrm{Dirichlet}}\newcommand{\Beta}{\mathrm{Beta}}\newcommand{\bE}{\mathbb{E}}\newcommand{\bG}{\mathbb{G}}\newcommand{\MISE}{\mathrm{MISE}}\newcommand{\logit}{\mathtt{logit}}\newcommand{\expit}{\mathtt{expit}}\newcommand{\cK}{\mathcal{K}}\newcommand{\dl}{\dot{l}}\newcommand{\dotp}{\dot{p}}\newcommand{\wl}{\wt{l}}\newcommand{\Gauss}{\mathrm{Gauss}}\newcommand{\fA}{\mathfrak{A}}\newcommand{\under}{\mathrm{under}\;}\newcommand{\whtheta}{\wh{\theta}}\newcommand{\Em}{\mathrm{Em}}\newcommand{\ztheta}{{\theta_0}}
\newcommand{\rO}{\mathrm{O}}\newcommand{\Bin}{\mathrm{Bin}}\newcommand{\rW}{\mathrm{W}}\newcommand{\rG}{\mathrm{G}}\newcommand{\rB}{\mathrm{B}}\newcommand{\rN}{\mathrm{N}}\newcommand{\rU}{\mathrm{U}}\newcommand{\HG}{\mathrm{HG}}\newcommand{\GAMMA}{\mathrm{Gamma}}\newcommand{\Cauchy}{\mathrm{Cauchy}}\newcommand{\rt}{\mathrm{t}}
\DeclareMathOperator{\erf}{erf}

%%% 圏
\newcommand{\varlim}{\varprojlim}\newcommand{\Hom}{\mathrm{Hom}}\newcommand{\Iso}{\mathrm{Iso}}\newcommand{\Mor}{\mathrm{Mor}}\newcommand{\Isom}{\mathrm{Isom}}\newcommand{\Aut}{\mathrm{Aut}}\newcommand{\End}{\mathrm{End}}\newcommand{\op}{\mathrm{op}}\newcommand{\ev}{\mathrm{ev}}\newcommand{\Ob}{\mathrm{Ob}}\newcommand{\Ar}{\mathrm{Ar}}\newcommand{\Arr}{\mathrm{Arr}}\newcommand{\Set}{\mathrm{Set}}\newcommand{\Grp}{\mathrm{Grp}}\newcommand{\Cat}{\mathrm{Cat}}\newcommand{\Mon}{\mathrm{Mon}}\newcommand{\Ring}{\mathrm{Ring}}\newcommand{\CRing}{\mathrm{CRing}}\newcommand{\Ab}{\mathrm{Ab}}\newcommand{\Pos}{\mathrm{Pos}}\newcommand{\Vect}{\mathrm{Vect}}\newcommand{\FinVect}{\mathrm{FinVect}}\newcommand{\FinSet}{\mathrm{FinSet}}\newcommand{\FinMeas}{\mathrm{FinMeas}}\newcommand{\OmegaAlg}{\Omega\text{-}\mathrm{Alg}}\newcommand{\OmegaEAlg}{(\Omega,E)\text{-}\mathrm{Alg}}\newcommand{\Fun}{\mathrm{Fun}}\newcommand{\Func}{\mathrm{Func}}\newcommand{\Alg}{\mathrm{Alg}} %代数の圏
\newcommand{\CAlg}{\mathrm{CAlg}} %可換代数の圏
\newcommand{\Met}{\mathrm{Met}} %Metric space & Contraction maps
\newcommand{\Rel}{\mathrm{Rel}} %Sets & relation
\newcommand{\Bool}{\mathrm{Bool}}\newcommand{\CABool}{\mathrm{CABool}}\newcommand{\CompBoolAlg}{\mathrm{CompBoolAlg}}\newcommand{\BoolAlg}{\mathrm{BoolAlg}}\newcommand{\BoolRng}{\mathrm{BoolRng}}\newcommand{\HeytAlg}{\mathrm{HeytAlg}}\newcommand{\CompHeytAlg}{\mathrm{CompHeytAlg}}\newcommand{\Lat}{\mathrm{Lat}}\newcommand{\CompLat}{\mathrm{CompLat}}\newcommand{\SemiLat}{\mathrm{SemiLat}}\newcommand{\Stone}{\mathrm{Stone}}\newcommand{\Mfd}{\mathrm{Mfd}}\newcommand{\LieAlg}{\mathrm{LieAlg}}
\newcommand{\Sob}{\mathrm{Sob}} %Sober space & continuous map
\newcommand{\Op}{\mathrm{Op}} %Category of open subsets
\newcommand{\Sh}{\mathrm{Sh}} %Category of sheave
\newcommand{\PSh}{\mathrm{PSh}} %Category of presheave, PSh(C)=[C^op,set]のこと
\newcommand{\Conv}{\mathrm{Conv}} %Convergence spaceの圏
\newcommand{\Unif}{\mathrm{Unif}} %一様空間と一様連続写像の圏
\newcommand{\Frm}{\mathrm{Frm}} %フレームとフレームの射
\newcommand{\Locale}{\mathrm{Locale}} %その反対圏
\newcommand{\Diff}{\mathrm{Diff}} %滑らかな多様体の圏
\newcommand{\Quiv}{\mathrm{Quiv}} %Quiverの圏
\newcommand{\B}{\mathcal{B}}\newcommand{\Span}{\mathrm{Span}}\newcommand{\Corr}{\mathrm{Corr}}\newcommand{\Decat}{\mathrm{Decat}}\newcommand{\Rep}{\mathrm{Rep}}\newcommand{\Grpd}{\mathrm{Grpd}}\newcommand{\sSet}{\mathrm{sSet}}\newcommand{\Mod}{\mathrm{Mod}}\newcommand{\SmoothMnf}{\mathrm{SmoothMnf}}\newcommand{\coker}{\mathrm{coker}}\newcommand{\Ord}{\mathrm{Ord}}\newcommand{\eq}{\mathrm{eq}}\newcommand{\coeq}{\mathrm{coeq}}\newcommand{\act}{\mathrm{act}}

%%%%%%%%%%%%%%% 定理環境(足助先生ありがとうございます) %%%%%%%%%%%%%%%

\everymath{\displaystyle}
\renewcommand{\proofname}{\bf\underline{[証明]}}
\renewcommand{\thefootnote}{\dag\arabic{footnote}} %足助さんからもらった.どうなるんだ?
\renewcommand{\qedsymbol}{$\blacksquare$}

\renewcommand{\labelenumi}{(\arabic{enumi})} %(1),(2),...がデフォルトであって欲しい
\renewcommand{\labelenumii}{(\alph{enumii})}
\renewcommand{\labelenumiii}{(\roman{enumiii})}

\newtheoremstyle{StatementsWithUnderline}% ?name?
{3pt}% ?Space above? 1
{3pt}% ?Space below? 1
{}% ?Body font?
{}% ?Indent amount? 2
{\bfseries}% ?Theorem head font?
{\textbf{.}}% ?Punctuation after theorem head?
{.5em}% ?Space after theorem head? 3
{\textbf{\underline{\textup{#1~\thetheorem{}}}}\;\thmnote{(#3)}}% ?Theorem head spec (can be left empty, meaning ‘normal’)?

\usepackage{etoolbox}
\AtEndEnvironment{example}{\hfill\ensuremath{\Box}}
\AtEndEnvironment{observation}{\hfill\ensuremath{\Box}}

\theoremstyle{StatementsWithUnderline}
    \newtheorem{theorem}{定理}[section]
    \newtheorem{axiom}[theorem]{公理}
    \newtheorem{corollary}[theorem]{系}
    \newtheorem{proposition}[theorem]{命題}
    \newtheorem{lemma}[theorem]{補題}
    \newtheorem{definition}[theorem]{定義}
    \newtheorem{problem}[theorem]{問題}
    \newtheorem{exercise}[theorem]{Exercise}
\theoremstyle{definition}
    \newtheorem{issue}{論点}
    \newtheorem*{proposition*}{命題}
    \newtheorem*{lemma*}{補題}
    \newtheorem*{consideration*}{考察}
    \newtheorem*{theorem*}{定理}
    \newtheorem*{remarks*}{要諦}
    \newtheorem{example}[theorem]{例}
    \newtheorem{notation}[theorem]{記法}
    \newtheorem*{notation*}{記法}
    \newtheorem{assumption}[theorem]{仮定}
    \newtheorem{question}[theorem]{問}
    \newtheorem{counterexample}[theorem]{反例}
    \newtheorem{reidai}[theorem]{例題}
    \newtheorem{ruidai}[theorem]{類題}
    \newtheorem{algorithm}[theorem]{算譜}
    \newtheorem*{feels*}{所感}
    \newtheorem*{solution*}{\bf{[解]}}
    \newtheorem{discussion}[theorem]{議論}
    \newtheorem{synopsis}[theorem]{要約}
    \newtheorem{cited}[theorem]{引用}
    \newtheorem{remark}[theorem]{注}
    \newtheorem{remarks}[theorem]{要諦}
    \newtheorem{memo}[theorem]{メモ}
    \newtheorem{image}[theorem]{描像}
    \newtheorem{observation}[theorem]{観察}
    \newtheorem{universality}[theorem]{普遍性} %非自明な例外がない.
    \newtheorem{universal tendency}[theorem]{普遍傾向} %例外が有意に少ない.
    \newtheorem{hypothesis}[theorem]{仮説} %実験で説明されていない理論.
    \newtheorem{theory}[theorem]{理論} %実験事実とその(さしあたり)整合的な説明.
    \newtheorem{fact}[theorem]{実験事実}
    \newtheorem{model}[theorem]{模型}
    \newtheorem{explanation}[theorem]{説明} %理論による実験事実の説明
    \newtheorem{anomaly}[theorem]{理論の限界}
    \newtheorem{application}[theorem]{応用例}
    \newtheorem{method}[theorem]{手法} %実験手法など,技術的問題.
    \newtheorem{test}[theorem]{検定}
    \newtheorem{terms}[theorem]{用語}
    \newtheorem{solution}[theorem]{解法}
    \newtheorem{history}[theorem]{歴史}
    \newtheorem{usage}[theorem]{用語法}
    \newtheorem{research}[theorem]{研究}
    \newtheorem{shishin}[theorem]{指針}
    \newtheorem{yodan}[theorem]{余談}
    \newtheorem{construction}[theorem]{構成}
    \newtheorem{motivation}[theorem]{動機}
    \newtheorem{context}[theorem]{背景}
    \newtheorem{advantage}[theorem]{利点}
    \newtheorem*{definition*}{定義}
    \newtheorem*{remark*}{注意}
    \newtheorem*{question*}{問}
    \newtheorem*{problem*}{問題}
    \newtheorem*{axiom*}{公理}
    \newtheorem*{example*}{例}
    \newtheorem*{corollary*}{系}
    \newtheorem*{shishin*}{指針}
    \newtheorem*{yodan*}{余談}
    \newtheorem*{kadai*}{課題}

\raggedbottom
\allowdisplaybreaks
%%%%%%%%%%%%%%%% 数理文書の組版 %%%%%%%%%%%%%%%

\usepackage{mathtools} %内部でamsmathを呼び出すことに注意.
%\mathtoolsset{showonlyrefs=true} %labelを附した数式にのみ附番される設定.
\usepackage{amsfonts} %mathfrak, mathcal, mathbbなど.
\usepackage{amsthm} %定理環境.
\usepackage{amssymb} %AMSFontsを使うためのパッケージ.
\usepackage{ascmac} %screen, itembox, shadebox環境.全てLATEX2εの標準機能の範囲で作られたもの.
\usepackage{comment} %comment環境を用いて,複数行をcomment outできるようにするpackage
\usepackage{wrapfig} %図の周りに文字をwrapさせることができる.詳細な制御ができる.
\usepackage[usenames, dvipsnames]{xcolor} %xcolorはcolorの拡張.optionの意味はdvipsnamesはLoad a set of predefined colors. forestgreenなどの色が追加されている.usenamesはobsoleteとだけ書いてあった.
\setcounter{tocdepth}{2} %目次に表示される深さ.2はsubsectionまで
\usepackage{multicol} %\begin{multicols}{2}環境で途中からmulticolumnに出来る.
\usepackage{mathabx}\newcommand{\wc}{\widecheck} %\widecheckなどのフォントパッケージ

%%%%%%%%%%%%%%% フォント %%%%%%%%%%%%%%%

\usepackage{textcomp, mathcomp} %Text Companionとは,T1 encodingに入らなかった文字群.これを使うためのパッケージ.\textsectionでブルバキに!
\usepackage[T1]{fontenc} %8bitエンコーディングにする.comp系拡張数学文字の動作が安定する.

%%%%%%%%%%%%%%% 一般文書の組版 %%%%%%%%%%%%%%%

\definecolor{花緑青}{cmyk}{1,0.07,0.10,0.10}\definecolor{サーモンピンク}{cmyk}{0,0.65,0.65,0.05}\definecolor{暗中模索}{rgb}{0.2,0.2,0.2}
\usepackage{url}\usepackage[dvipdfmx,colorlinks,linkcolor=花緑青,urlcolor=花緑青,citecolor=花緑青]{hyperref} %生成されるPDFファイルにおいて、\tableofcontentsによって書き出された目次をクリックすると該当する見出しへジャンプしたり、さらには、\label{ラベル名}を番号で参照する\ref{ラベル名}やthebibliography環境において\bibitem{ラベル名}を文献番号で参照する\cite{ラベル名}においても番号をクリックすると該当箇所にジャンプする.囲み枠はダサいので,colorlinksで囲み廃止し,リンク自体に色を付けることにした.
\usepackage{pxjahyper} %pxrubrica同様,八登崇之さん.hyperrefは日本語pLaTeXに最適化されていないから,hyperrefとセットで,(u)pLaTeX+hyperref+dvipdfmxの組み合わせで日本語を含む「しおり」をもつPDF文書を作成する場合に必要となる機能を提供する
\usepackage{ulem} %取り消し線を引くためのパッケージ
\usepackage{pxrubrica} %日本語にルビをふる.八登崇之(やとうたかゆき)氏による.

%%%%%%%%%%%%%%% 科学文書の組版 %%%%%%%%%%%%%%%

\usepackage[version=4]{mhchem} %化学式をTikZで簡単に書くためのパッケージ.
\usepackage{chemfig} %化学構造式をTikZで描くためのパッケージ.
\usepackage{siunitx} %IS単位を書くためのパッケージ

%%%%%%%%%%%%%%% 作図 %%%%%%%%%%%%%%%

\usepackage{tikz}\usetikzlibrary{positioning,automata}\usepackage{tikz-cd}\usepackage[all]{xy}
\def\objectstyle{\displaystyle} %デフォルトではxymatrix中の数式が文中数式モードになるので,それを直す.\labelstyleも同様にxy packageの中で定義されており,文中数式モードになっている.

\usepackage{graphicx} %rotatebox, scalebox, reflectbox, resizeboxなどのコマンドや,図表の読み込み\includegraphicsを司る.graphics というパッケージもありますが,graphicx はこれを高機能にしたものと考えて結構です(ただし graphicx は内部で graphics を読み込みます)
\usepackage[top=15truemm,bottom=15truemm,left=10truemm,right=10truemm]{geometry} %足助さんからもらったオプション

%%%%%%%%%%%%%%% 参照 %%%%%%%%%%%%%%%
%参考文献リストを出力したい箇所に\bibliography{../mathematics.bib}を追記すると良い.

%\bibliographystyle{jplain}
%\bibliographystyle{jname}
\bibliographystyle{apalike}

%%%%%%%%%%%%%%% 計算機文書の組版 %%%%%%%%%%%%%%%

\usepackage[breakable]{tcolorbox} %加藤晃史さんがフル活用していたtcolorboxを,途中改ページ可能で.
\tcbuselibrary{theorems} %https://qiita.com/t_kemmochi/items/483b8fcdb5db8d1f5d5e
\usepackage{enumerate} %enumerate環境を凝らせる.

\usepackage{listings} %ソースコードを表示できる環境.多分もっといい方法ある.
\usepackage{jvlisting} %日本語のコメントアウトをする場合jlistingが必要
\lstset{ %ここからソースコードの表示に関する設定.lstlisting環境では,[caption=hoge,label=fuga]などのoptionを付けられる.
%[escapechar=!]とすると,LaTeXコマンドを使える.
  basicstyle={\ttfamily},
  identifierstyle={\small},
  commentstyle={\smallitshape},
  keywordstyle={\small\bfseries},
  ndkeywordstyle={\small},
  stringstyle={\small\ttfamily},
  frame={tb},
  breaklines=true,
  columns=[l]{fullflexible},
  numbers=left,
  xrightmargin=0zw,
  xleftmargin=3zw,
  numberstyle={\scriptsize},
  stepnumber=1,
  numbersep=1zw,
  lineskip=-0.5ex
}
%\makeatletter %caption番号を「[chapter番号].[section番号].[subsection番号]-[そのsubsection内においてn番目]」に変更
%    \AtBeginDocument{
%    \renewcommand*{\thelstlisting}{\arabic{chapter}.\arabic{section}.\arabic{lstlisting}}
%    \@addtoreset{lstlisting}{section}
%    }
%\makeatother
\renewcommand{\lstlistingname}{算譜} %caption名を"program"に変更

\newtcolorbox{tbox}[3][]{%
colframe=#2,colback=#2!10,coltitle=#2!20!black,title={#3},#1}

% 証明内の文字が小さくなる環境.
\newenvironment{Proof}[1][\bf\underline{[証明]}]{\proof[#1]\color{darkgray}}{\endproof}

%%%%%%%%%%%%%%% 数学記号のマクロ %%%%%%%%%%%%%%%

%%% 括弧類
\newcommand{\abs}[1]{\lvert#1\rvert}\newcommand{\Abs}[1]{\left|#1\right|}\newcommand{\norm}[1]{\|#1\|}\newcommand{\Norm}[1]{\left\|#1\right\|}\newcommand{\Brace}[1]{\left\{#1\right\}}\newcommand{\BRace}[1]{\biggl\{#1\biggr\}}\newcommand{\paren}[1]{\left(#1\right)}\newcommand{\Paren}[1]{\biggr(#1\biggl)}\newcommand{\bracket}[1]{\langle#1\rangle}\newcommand{\brac}[1]{\langle#1\rangle}\newcommand{\Bracket}[1]{\left\langle#1\right\rangle}\newcommand{\Brac}[1]{\left\langle#1\right\rangle}\newcommand{\bra}[1]{\left\langle#1\right|}\newcommand{\ket}[1]{\left|#1\right\rangle}\newcommand{\Square}[1]{\left[#1\right]}\newcommand{\SQuare}[1]{\biggl[#1\biggr]}
\renewcommand{\o}[1]{\overline{#1}}\renewcommand{\u}[1]{\underline{#1}}\newcommand{\wt}[1]{\widetilde{#1}}\newcommand{\wh}[1]{\widehat{#1}}
\newcommand{\pp}[2]{\frac{\partial #1}{\partial #2}}\newcommand{\ppp}[3]{\frac{\partial #1}{\partial #2\partial #3}}\newcommand{\dd}[2]{\frac{d #1}{d #2}}
\newcommand{\floor}[1]{\lfloor#1\rfloor}\newcommand{\Floor}[1]{\left\lfloor#1\right\rfloor}\newcommand{\ceil}[1]{\lceil#1\rceil}
\newcommand{\ocinterval}[1]{(#1]}\newcommand{\cointerval}[1]{[#1)}\newcommand{\COinterval}[1]{\left[#1\right)}


%%% 予約語
\renewcommand{\iff}{\;\mathrm{iff}\;}
\newcommand{\False}{\mathrm{False}}\newcommand{\True}{\mathrm{True}}
\newcommand{\otherwise}{\mathrm{otherwise}}
\newcommand{\st}{\;\mathrm{s.t.}\;}

%%% 略記
\newcommand{\M}{\mathcal{M}}\newcommand{\cF}{\mathcal{F}}\newcommand{\cD}{\mathcal{D}}\newcommand{\fX}{\mathfrak{X}}\newcommand{\fY}{\mathfrak{Y}}\newcommand{\fZ}{\mathfrak{Z}}\renewcommand{\H}{\mathcal{H}}\newcommand{\fH}{\mathfrak{H}}\newcommand{\bH}{\mathbb{H}}\newcommand{\id}{\mathrm{id}}\newcommand{\A}{\mathcal{A}}\newcommand{\U}{\mathfrak{U}}
\newcommand{\lmd}{\lambda}
\newcommand{\Lmd}{\Lambda}

%%% 矢印類
\newcommand{\iso}{\xrightarrow{\,\smash{\raisebox{-0.45ex}{\ensuremath{\scriptstyle\sim}}}\,}}
\newcommand{\Lrarrow}{\;\;\Leftrightarrow\;\;}

%%% 注記
\newcommand{\rednote}[1]{\textcolor{red}{#1}}

% ノルム位相についての閉包 https://newbedev.com/how-to-make-double-overline-with-less-vertical-displacement
\makeatletter
\newcommand{\dbloverline}[1]{\overline{\dbl@overline{#1}}}
\newcommand{\dbl@overline}[1]{\mathpalette\dbl@@overline{#1}}
\newcommand{\dbl@@overline}[2]{%
  \begingroup
  \sbox\z@{$\m@th#1\overline{#2}$}%
  \ht\z@=\dimexpr\ht\z@-2\dbl@adjust{#1}\relax
  \box\z@
  \ifx#1\scriptstyle\kern-\scriptspace\else
  \ifx#1\scriptscriptstyle\kern-\scriptspace\fi\fi
  \endgroup
}
\newcommand{\dbl@adjust}[1]{%
  \fontdimen8
  \ifx#1\displaystyle\textfont\else
  \ifx#1\textstyle\textfont\else
  \ifx#1\scriptstyle\scriptfont\else
  \scriptscriptfont\fi\fi\fi 3
}
\makeatother
\newcommand{\oo}[1]{\dbloverline{#1}}

% hslashの他の文字Ver.
\newcommand{\hslashslash}{%
    \scalebox{1.2}{--
    }%
}
\newcommand{\dslash}{%
  {%
    \vphantom{d}%
    \ooalign{\kern.05em\smash{\hslashslash}\hidewidth\cr$d$\cr}%
    \kern.05em
  }%
}
\newcommand{\dint}{%
  {%
    \vphantom{d}%
    \ooalign{\kern.05em\smash{\hslashslash}\hidewidth\cr$\int$\cr}%
    \kern.05em
  }%
}
\newcommand{\dL}{%
  {%
    \vphantom{d}%
    \ooalign{\kern.05em\smash{\hslashslash}\hidewidth\cr$L$\cr}%
    \kern.05em
  }%
}

%%% 演算子
\DeclareMathOperator{\grad}{\mathrm{grad}}\DeclareMathOperator{\rot}{\mathrm{rot}}\DeclareMathOperator{\divergence}{\mathrm{div}}\DeclareMathOperator{\tr}{\mathrm{tr}}\newcommand{\pr}{\mathrm{pr}}
\newcommand{\Map}{\mathrm{Map}}\newcommand{\dom}{\mathrm{Dom}\;}\newcommand{\cod}{\mathrm{Cod}\;}\newcommand{\supp}{\mathrm{supp}\;}


%%% 線型代数学
\newcommand{\vctr}[2]{\begin{pmatrix}#1\\#2\end{pmatrix}}\newcommand{\vctrr}[3]{\begin{pmatrix}#1\\#2\\#3\end{pmatrix}}\newcommand{\mtrx}[4]{\begin{pmatrix}#1&#2\\#3&#4\end{pmatrix}}\newcommand{\smtrx}[4]{\paren{\begin{smallmatrix}#1&#2\\#3&#4\end{smallmatrix}}}\newcommand{\Ker}{\mathrm{Ker}\;}\newcommand{\Coker}{\mathrm{Coker}\;}\newcommand{\Coim}{\mathrm{Coim}\;}\DeclareMathOperator{\rank}{\mathrm{rank}}\newcommand{\lcm}{\mathrm{lcm}}\newcommand{\sgn}{\mathrm{sgn}\,}\newcommand{\GL}{\mathrm{GL}}\newcommand{\SL}{\mathrm{SL}}\newcommand{\alt}{\mathrm{alt}}
%%% 複素解析学
\renewcommand{\Re}{\mathrm{Re}\;}\renewcommand{\Im}{\mathrm{Im}\;}\newcommand{\Gal}{\mathrm{Gal}}\newcommand{\PGL}{\mathrm{PGL}}\newcommand{\PSL}{\mathrm{PSL}}\newcommand{\Log}{\mathrm{Log}\,}\newcommand{\Res}{\mathrm{Res}\,}\newcommand{\on}{\mathrm{on}\;}\newcommand{\hatC}{\widehat{\C}}\newcommand{\hatR}{\hat{\R}}\newcommand{\PV}{\mathrm{P.V.}}\newcommand{\diam}{\mathrm{diam}}\newcommand{\Area}{\mathrm{Area}}\newcommand{\Lap}{\Laplace}\newcommand{\f}{\mathbf{f}}\newcommand{\cR}{\mathcal{R}}\newcommand{\const}{\mathrm{const.}}\newcommand{\Om}{\Omega}\newcommand{\Cinf}{C^\infty}\newcommand{\ep}{\epsilon}\newcommand{\dist}{\mathrm{dist}}\newcommand{\opart}{\o{\partial}}\newcommand{\Length}{\mathrm{Length}}
%%% 集合と位相
\renewcommand{\O}{\mathcal{O}}\renewcommand{\S}{\mathcal{S}}\renewcommand{\U}{\mathcal{U}}\newcommand{\V}{\mathcal{V}}\renewcommand{\P}{\mathcal{P}}\newcommand{\R}{\mathbb{R}}\newcommand{\N}{\mathbb{N}}\newcommand{\C}{\mathbb{C}}\newcommand{\Z}{\mathbb{Z}}\newcommand{\Q}{\mathbb{Q}}\newcommand{\TV}{\mathrm{TV}}\newcommand{\ORD}{\mathrm{ORD}}\newcommand{\Tr}{\mathrm{Tr}}\newcommand{\Card}{\mathrm{Card}\;}\newcommand{\Top}{\mathrm{Top}}\newcommand{\Disc}{\mathrm{Disc}}\newcommand{\Codisc}{\mathrm{Codisc}}\newcommand{\CoDisc}{\mathrm{CoDisc}}\newcommand{\Ult}{\mathrm{Ult}}\newcommand{\ord}{\mathrm{ord}}\newcommand{\maj}{\mathrm{maj}}\newcommand{\bS}{\mathbb{S}}\newcommand{\PConn}{\mathrm{PConn}}

%%% 形式言語理論
\newcommand{\REGEX}{\mathrm{REGEX}}\newcommand{\RE}{\mathbf{RE}}
%%% Graph Theory
\newcommand{\SimpGph}{\mathrm{SimpGph}}\newcommand{\Gph}{\mathrm{Gph}}\newcommand{\mult}{\mathrm{mult}}\newcommand{\inv}{\mathrm{inv}}

%%% 多様体
\newcommand{\Der}{\mathrm{Der}}\newcommand{\osub}{\overset{\mathrm{open}}{\subset}}\newcommand{\osup}{\overset{\mathrm{open}}{\supset}}\newcommand{\al}{\alpha}\newcommand{\K}{\mathbb{K}}\newcommand{\Sp}{\mathrm{Sp}}\newcommand{\g}{\mathfrak{g}}\newcommand{\h}{\mathfrak{h}}\newcommand{\Exp}{\mathrm{Exp}\;}\newcommand{\Imm}{\mathrm{Imm}}\newcommand{\Imb}{\mathrm{Imb}}\newcommand{\codim}{\mathrm{codim}\;}\newcommand{\Gr}{\mathrm{Gr}}
%%% 代数
\newcommand{\Ad}{\mathrm{Ad}}\newcommand{\finsupp}{\mathrm{fin\;supp}}\newcommand{\SO}{\mathrm{SO}}\newcommand{\SU}{\mathrm{SU}}\newcommand{\acts}{\curvearrowright}\newcommand{\mono}{\hookrightarrow}\newcommand{\epi}{\twoheadrightarrow}\newcommand{\Stab}{\mathrm{Stab}}\newcommand{\nor}{\mathrm{nor}}\newcommand{\T}{\mathbb{T}}\newcommand{\Aff}{\mathrm{Aff}}\newcommand{\rsub}{\triangleleft}\newcommand{\rsup}{\triangleright}\newcommand{\subgrp}{\overset{\mathrm{subgrp}}{\subset}}\newcommand{\Ext}{\mathrm{Ext}}\newcommand{\sbs}{\subset}\newcommand{\sps}{\supset}\newcommand{\In}{\mathrm{in}\;}\newcommand{\Tor}{\mathrm{Tor}}\newcommand{\p}{\b{p}}\newcommand{\q}{\mathfrak{q}}\newcommand{\m}{\mathfrak{m}}\newcommand{\cS}{\mathcal{S}}\newcommand{\Frac}{\mathrm{Frac}\,}\newcommand{\Spec}{\mathrm{Spec}\,}\newcommand{\bA}{\mathbb{A}}\newcommand{\Sym}{\mathrm{Sym}}\newcommand{\Ann}{\mathrm{Ann}}\newcommand{\Her}{\mathrm{Her}}\newcommand{\Bil}{\mathrm{Bil}}\newcommand{\Ses}{\mathrm{Ses}}\newcommand{\FVS}{\mathrm{FVS}}
%%% 代数的位相幾何学
\newcommand{\Ho}{\mathrm{Ho}}\newcommand{\CW}{\mathrm{CW}}\newcommand{\lc}{\mathrm{lc}}\newcommand{\cg}{\mathrm{cg}}\newcommand{\Fib}{\mathrm{Fib}}\newcommand{\Cyl}{\mathrm{Cyl}}\newcommand{\Ch}{\mathrm{Ch}}
%%% 微分幾何学
\newcommand{\rE}{\mathrm{E}}\newcommand{\e}{\b{e}}\renewcommand{\k}{\b{k}}\newcommand{\Christ}[2]{\begin{Bmatrix}#1\\#2\end{Bmatrix}}\renewcommand{\Vec}[1]{\overrightarrow{\mathrm{#1}}}\newcommand{\hen}[1]{\mathrm{#1}}\renewcommand{\b}[1]{\boldsymbol{#1}}

%%% 函数解析
\newcommand{\HS}{\mathrm{HS}}\newcommand{\loc}{\mathrm{loc}}\newcommand{\Lh}{\mathrm{L.h.}}\newcommand{\Epi}{\mathrm{Epi}\;}\newcommand{\slim}{\mathrm{slim}}\newcommand{\Ban}{\mathrm{Ban}}\newcommand{\Hilb}{\mathrm{Hilb}}\newcommand{\Ex}{\mathrm{Ex}}\newcommand{\Co}{\mathrm{Co}}\newcommand{\sa}{\mathrm{sa}}\newcommand{\nnorm}[1]{{\left\vert\kern-0.25ex\left\vert\kern-0.25ex\left\vert #1 \right\vert\kern-0.25ex\right\vert\kern-0.25ex\right\vert}}\newcommand{\dvol}{\mathrm{dvol}}\newcommand{\Sconv}{\mathrm{Sconv}}\newcommand{\I}{\mathcal{I}}\newcommand{\nonunital}{\mathrm{nu}}\newcommand{\cpt}{\mathrm{cpt}}\newcommand{\lcpt}{\mathrm{lcpt}}\newcommand{\com}{\mathrm{com}}\newcommand{\Haus}{\mathrm{Haus}}\newcommand{\proper}{\mathrm{proper}}\newcommand{\infinity}{\mathrm{inf}}\newcommand{\TVS}{\mathrm{TVS}}\newcommand{\ess}{\mathrm{ess}}\newcommand{\ext}{\mathrm{ext}}\newcommand{\Index}{\mathrm{Index}\;}\newcommand{\SSR}{\mathrm{SSR}}\newcommand{\vs}{\mathrm{vs.}}\newcommand{\fM}{\mathfrak{M}}\newcommand{\EDM}{\mathrm{EDM}}\newcommand{\Tw}{\mathrm{Tw}}\newcommand{\fC}{\mathfrak{C}}\newcommand{\bn}{\boldsymbol{n}}\newcommand{\br}{\boldsymbol{r}}\newcommand{\Lam}{\Lambda}\newcommand{\lam}{\lambda}\newcommand{\one}{\mathbf{1}}\newcommand{\dae}{\text{-a.e.}}\newcommand{\das}{\text{-a.s.}}\newcommand{\td}{\text{-}}\newcommand{\RM}{\mathrm{RM}}\newcommand{\BV}{\mathrm{BV}}\newcommand{\normal}{\mathrm{normal}}\newcommand{\lub}{\mathrm{lub}\;}\newcommand{\Graph}{\mathrm{Graph}}\newcommand{\Ascent}{\mathrm{Ascent}}\newcommand{\Descent}{\mathrm{Descent}}\newcommand{\BIL}{\mathrm{BIL}}\newcommand{\fL}{\mathfrak{L}}\newcommand{\De}{\Delta}
%%% 積分論
\newcommand{\calA}{\mathcal{A}}\newcommand{\calB}{\mathcal{B}}\newcommand{\D}{\mathcal{D}}\newcommand{\Y}{\mathcal{Y}}\newcommand{\calC}{\mathcal{C}}\renewcommand{\ae}{\mathrm{a.e.}\;}\newcommand{\cZ}{\mathcal{Z}}\newcommand{\fF}{\mathfrak{F}}\newcommand{\fI}{\mathfrak{I}}\newcommand{\E}{\mathcal{E}}\newcommand{\sMap}{\sigma\textrm{-}\mathrm{Map}}\DeclareMathOperator*{\argmax}{arg\,max}\DeclareMathOperator*{\argmin}{arg\,min}\newcommand{\cC}{\mathcal{C}}\newcommand{\comp}{\complement}\newcommand{\J}{\mathcal{J}}\newcommand{\sumN}[1]{\sum_{#1\in\N}}\newcommand{\cupN}[1]{\cup_{#1\in\N}}\newcommand{\capN}[1]{\cap_{#1\in\N}}\newcommand{\Sum}[1]{\sum_{#1=1}^\infty}\newcommand{\sumn}{\sum_{n=1}^\infty}\newcommand{\summ}{\sum_{m=1}^\infty}\newcommand{\sumk}{\sum_{k=1}^\infty}\newcommand{\sumi}{\sum_{i=1}^\infty}\newcommand{\sumj}{\sum_{j=1}^\infty}\newcommand{\cupn}{\cup_{n=1}^\infty}\newcommand{\capn}{\cap_{n=1}^\infty}\newcommand{\cupk}{\cup_{k=1}^\infty}\newcommand{\cupi}{\cup_{i=1}^\infty}\newcommand{\cupj}{\cup_{j=1}^\infty}\newcommand{\limn}{\lim_{n\to\infty}}\renewcommand{\l}{\mathcal{l}}\renewcommand{\L}{\mathcal{L}}\newcommand{\Cl}{\mathrm{Cl}}\newcommand{\cN}{\mathcal{N}}\newcommand{\Ae}{\textrm{-a.e.}\;}\newcommand{\csub}{\overset{\textrm{closed}}{\subset}}\newcommand{\csup}{\overset{\textrm{closed}}{\supset}}\newcommand{\wB}{\wt{B}}\newcommand{\cG}{\mathcal{G}}\newcommand{\Lip}{\mathrm{Lip}}\DeclareMathOperator{\Dom}{\mathrm{Dom}}\newcommand{\AC}{\mathrm{AC}}\newcommand{\Mol}{\mathrm{Mol}}
%%% Fourier解析
\newcommand{\Pe}{\mathrm{Pe}}\newcommand{\wR}{\wh{\mathbb{\R}}}\newcommand*{\Laplace}{\mathop{}\!\mathbin\bigtriangleup}\newcommand*{\DAlambert}{\mathop{}\!\mathbin\Box}\newcommand{\bT}{\mathbb{T}}\newcommand{\dx}{\dslash x}\newcommand{\dt}{\dslash t}\newcommand{\ds}{\dslash s}
%%% 数値解析
\newcommand{\round}{\mathrm{round}}\newcommand{\cond}{\mathrm{cond}}\newcommand{\diag}{\mathrm{diag}}
\newcommand{\Adj}{\mathrm{Adj}}\newcommand{\Pf}{\mathrm{Pf}}\newcommand{\Sg}{\mathrm{Sg}}

%%% 確率論
\newcommand{\Prob}{\mathrm{Prob}}\newcommand{\X}{\mathcal{X}}\newcommand{\Meas}{\mathrm{Meas}}\newcommand{\as}{\;\mathrm{a.s.}}\newcommand{\io}{\;\mathrm{i.o.}}\newcommand{\fe}{\;\mathrm{f.e.}}\newcommand{\F}{\mathcal{F}}\newcommand{\bF}{\mathbb{F}}\newcommand{\W}{\mathcal{W}}\newcommand{\Pois}{\mathrm{Pois}}\newcommand{\iid}{\mathrm{i.i.d.}}\newcommand{\wconv}{\rightsquigarrow}\newcommand{\Var}{\mathrm{Var}}\newcommand{\xrightarrown}{\xrightarrow{n\to\infty}}\newcommand{\au}{\mathrm{au}}\newcommand{\cT}{\mathcal{T}}\newcommand{\wto}{\overset{w}{\to}}\newcommand{\dto}{\overset{d}{\to}}\newcommand{\pto}{\overset{p}{\to}}\newcommand{\vto}{\overset{v}{\to}}\newcommand{\Cont}{\mathrm{Cont}}\newcommand{\stably}{\mathrm{stably}}\newcommand{\Np}{\mathbb{N}^+}\newcommand{\oM}{\overline{\mathcal{M}}}\newcommand{\fP}{\mathfrak{P}}\newcommand{\sign}{\mathrm{sign}}\DeclareMathOperator{\Div}{Div}
\newcommand{\bD}{\mathbb{D}}\newcommand{\fW}{\mathfrak{W}}\newcommand{\DL}{\mathcal{D}\mathcal{L}}\renewcommand{\r}[1]{\mathrm{#1}}\newcommand{\rC}{\mathrm{C}}
%%% 情報理論
\newcommand{\bit}{\mathrm{bit}}\DeclareMathOperator{\sinc}{sinc}
%%% 量子論
\newcommand{\err}{\mathrm{err}}
%%% 最適化
\newcommand{\varparallel}{\mathbin{\!/\mkern-5mu/\!}}\newcommand{\Minimize}{\text{Minimize}}\newcommand{\subjectto}{\text{subject to}}\newcommand{\Ri}{\mathrm{Ri}}\newcommand{\Cone}{\mathrm{Cone}}\newcommand{\Int}{\mathrm{Int}}
%%% 数理ファイナンス
\newcommand{\pre}{\mathrm{pre}}\newcommand{\om}{\omega}

%%% 偏微分方程式
\let\div\relax
\DeclareMathOperator{\div}{div}\newcommand{\del}{\partial}
\newcommand{\LHS}{\mathrm{LHS}}\newcommand{\RHS}{\mathrm{RHS}}\newcommand{\bnu}{\boldsymbol{\nu}}\newcommand{\interior}{\mathrm{in}\;}\newcommand{\SH}{\mathrm{SH}}\renewcommand{\v}{\boldsymbol{\nu}}\newcommand{\n}{\mathbf{n}}\newcommand{\ssub}{\Subset}\newcommand{\curl}{\mathrm{curl}}
%%% 常微分方程式
\newcommand{\Ei}{\mathrm{Ei}}\newcommand{\sn}{\mathrm{sn}}\newcommand{\wgamma}{\widetilde{\gamma}}
%%% 統計力学
\newcommand{\Ens}{\mathrm{Ens}}
%%% 解析力学
\newcommand{\cl}{\mathrm{cl}}\newcommand{\x}{\boldsymbol{x}}

%%% 統計的因果推論
\newcommand{\Do}{\mathrm{Do}}
%%% 応用統計学
\newcommand{\mrl}{\mathrm{mrl}}
%%% 数理統計
\newcommand{\comb}[2]{\begin{pmatrix}#1\\#2\end{pmatrix}}\newcommand{\bP}{\mathbb{P}}\newcommand{\compsub}{\overset{\textrm{cpt}}{\subset}}\newcommand{\lip}{\textrm{lip}}\newcommand{\BL}{\mathrm{BL}}\newcommand{\G}{\mathbb{G}}\newcommand{\NB}{\mathrm{NB}}\newcommand{\oR}{\o{\R}}\newcommand{\liminfn}{\liminf_{n\to\infty}}\newcommand{\limsupn}{\limsup_{n\to\infty}}\newcommand{\esssup}{\mathrm{ess.sup}}\newcommand{\asto}{\xrightarrow{\as}}\newcommand{\Cov}{\mathrm{Cov}}\newcommand{\cQ}{\mathcal{Q}}\newcommand{\VC}{\mathrm{VC}}\newcommand{\mb}{\mathrm{mb}}\newcommand{\Avar}{\mathrm{Avar}}\newcommand{\bB}{\mathbb{B}}\newcommand{\bW}{\mathbb{W}}\newcommand{\sd}{\mathrm{sd}}\newcommand{\w}[1]{\widehat{#1}}\newcommand{\bZ}{\boldsymbol{Z}}\newcommand{\Bernoulli}{\mathrm{Ber}}\newcommand{\Ber}{\mathrm{Ber}}\newcommand{\Mult}{\mathrm{Mult}}\newcommand{\BPois}{\mathrm{BPois}}\newcommand{\fraks}{\mathfrak{s}}\newcommand{\frakk}{\mathfrak{k}}\newcommand{\IF}{\mathrm{IF}}\newcommand{\bX}{\mathbf{X}}\newcommand{\bx}{\boldsymbol{x}}\newcommand{\indep}{\raisebox{0.05em}{\rotatebox[origin=c]{90}{$\models$}}}\newcommand{\IG}{\mathrm{IG}}\newcommand{\Levy}{\mathrm{Levy}}\newcommand{\MP}{\mathrm{MP}}\newcommand{\Hermite}{\mathrm{Hermite}}\newcommand{\Skellam}{\mathrm{Skellam}}\newcommand{\Dirichlet}{\mathrm{Dirichlet}}\newcommand{\Beta}{\mathrm{Beta}}\newcommand{\bE}{\mathbb{E}}\newcommand{\bG}{\mathbb{G}}\newcommand{\MISE}{\mathrm{MISE}}\newcommand{\logit}{\mathtt{logit}}\newcommand{\expit}{\mathtt{expit}}\newcommand{\cK}{\mathcal{K}}\newcommand{\dl}{\dot{l}}\newcommand{\dotp}{\dot{p}}\newcommand{\wl}{\wt{l}}\newcommand{\Gauss}{\mathrm{Gauss}}\newcommand{\fA}{\mathfrak{A}}\newcommand{\under}{\mathrm{under}\;}\newcommand{\whtheta}{\wh{\theta}}\newcommand{\Em}{\mathrm{Em}}\newcommand{\ztheta}{{\theta_0}}
\newcommand{\rO}{\mathrm{O}}\newcommand{\Bin}{\mathrm{Bin}}\newcommand{\rW}{\mathrm{W}}\newcommand{\rG}{\mathrm{G}}\newcommand{\rB}{\mathrm{B}}\newcommand{\rN}{\mathrm{N}}\newcommand{\rU}{\mathrm{U}}\newcommand{\HG}{\mathrm{HG}}\newcommand{\GAMMA}{\mathrm{Gamma}}\newcommand{\Cauchy}{\mathrm{Cauchy}}\newcommand{\rt}{\mathrm{t}}
\DeclareMathOperator{\erf}{erf}

%%% 圏
\newcommand{\varlim}{\varprojlim}\newcommand{\Hom}{\mathrm{Hom}}\newcommand{\Iso}{\mathrm{Iso}}\newcommand{\Mor}{\mathrm{Mor}}\newcommand{\Isom}{\mathrm{Isom}}\newcommand{\Aut}{\mathrm{Aut}}\newcommand{\End}{\mathrm{End}}\newcommand{\op}{\mathrm{op}}\newcommand{\ev}{\mathrm{ev}}\newcommand{\Ob}{\mathrm{Ob}}\newcommand{\Ar}{\mathrm{Ar}}\newcommand{\Arr}{\mathrm{Arr}}\newcommand{\Set}{\mathrm{Set}}\newcommand{\Grp}{\mathrm{Grp}}\newcommand{\Cat}{\mathrm{Cat}}\newcommand{\Mon}{\mathrm{Mon}}\newcommand{\Ring}{\mathrm{Ring}}\newcommand{\CRing}{\mathrm{CRing}}\newcommand{\Ab}{\mathrm{Ab}}\newcommand{\Pos}{\mathrm{Pos}}\newcommand{\Vect}{\mathrm{Vect}}\newcommand{\FinVect}{\mathrm{FinVect}}\newcommand{\FinSet}{\mathrm{FinSet}}\newcommand{\FinMeas}{\mathrm{FinMeas}}\newcommand{\OmegaAlg}{\Omega\text{-}\mathrm{Alg}}\newcommand{\OmegaEAlg}{(\Omega,E)\text{-}\mathrm{Alg}}\newcommand{\Fun}{\mathrm{Fun}}\newcommand{\Func}{\mathrm{Func}}\newcommand{\Alg}{\mathrm{Alg}} %代数の圏
\newcommand{\CAlg}{\mathrm{CAlg}} %可換代数の圏
\newcommand{\Met}{\mathrm{Met}} %Metric space & Contraction maps
\newcommand{\Rel}{\mathrm{Rel}} %Sets & relation
\newcommand{\Bool}{\mathrm{Bool}}\newcommand{\CABool}{\mathrm{CABool}}\newcommand{\CompBoolAlg}{\mathrm{CompBoolAlg}}\newcommand{\BoolAlg}{\mathrm{BoolAlg}}\newcommand{\BoolRng}{\mathrm{BoolRng}}\newcommand{\HeytAlg}{\mathrm{HeytAlg}}\newcommand{\CompHeytAlg}{\mathrm{CompHeytAlg}}\newcommand{\Lat}{\mathrm{Lat}}\newcommand{\CompLat}{\mathrm{CompLat}}\newcommand{\SemiLat}{\mathrm{SemiLat}}\newcommand{\Stone}{\mathrm{Stone}}\newcommand{\Mfd}{\mathrm{Mfd}}\newcommand{\LieAlg}{\mathrm{LieAlg}}
\newcommand{\Sob}{\mathrm{Sob}} %Sober space & continuous map
\newcommand{\Op}{\mathrm{Op}} %Category of open subsets
\newcommand{\Sh}{\mathrm{Sh}} %Category of sheave
\newcommand{\PSh}{\mathrm{PSh}} %Category of presheave, PSh(C)=[C^op,set]のこと
\newcommand{\Conv}{\mathrm{Conv}} %Convergence spaceの圏
\newcommand{\Unif}{\mathrm{Unif}} %一様空間と一様連続写像の圏
\newcommand{\Frm}{\mathrm{Frm}} %フレームとフレームの射
\newcommand{\Locale}{\mathrm{Locale}} %その反対圏
\newcommand{\Diff}{\mathrm{Diff}} %滑らかな多様体の圏
\newcommand{\Quiv}{\mathrm{Quiv}} %Quiverの圏
\newcommand{\B}{\mathcal{B}}\newcommand{\Span}{\mathrm{Span}}\newcommand{\Corr}{\mathrm{Corr}}\newcommand{\Decat}{\mathrm{Decat}}\newcommand{\Rep}{\mathrm{Rep}}\newcommand{\Grpd}{\mathrm{Grpd}}\newcommand{\sSet}{\mathrm{sSet}}\newcommand{\Mod}{\mathrm{Mod}}\newcommand{\SmoothMnf}{\mathrm{SmoothMnf}}\newcommand{\coker}{\mathrm{coker}}\newcommand{\Ord}{\mathrm{Ord}}\newcommand{\eq}{\mathrm{eq}}\newcommand{\coeq}{\mathrm{coeq}}\newcommand{\act}{\mathrm{act}}

%%%%%%%%%%%%%%% 定理環境(足助先生ありがとうございます) %%%%%%%%%%%%%%%

\everymath{\displaystyle}
\renewcommand{\proofname}{\bf\underline{[証明]}}
\renewcommand{\thefootnote}{\dag\arabic{footnote}} %足助さんからもらった.どうなるんだ?
\renewcommand{\qedsymbol}{$\blacksquare$}

\renewcommand{\labelenumi}{(\arabic{enumi})} %(1),(2),...がデフォルトであって欲しい
\renewcommand{\labelenumii}{(\alph{enumii})}
\renewcommand{\labelenumiii}{(\roman{enumiii})}

\newtheoremstyle{StatementsWithUnderline}% ?name?
{3pt}% ?Space above? 1
{3pt}% ?Space below? 1
{}% ?Body font?
{}% ?Indent amount? 2
{\bfseries}% ?Theorem head font?
{\textbf{.}}% ?Punctuation after theorem head?
{.5em}% ?Space after theorem head? 3
{\textbf{\underline{\textup{#1~\thetheorem{}}}}\;\thmnote{(#3)}}% ?Theorem head spec (can be left empty, meaning ‘normal’)?

\usepackage{etoolbox}
\AtEndEnvironment{example}{\hfill\ensuremath{\Box}}
\AtEndEnvironment{observation}{\hfill\ensuremath{\Box}}

\theoremstyle{StatementsWithUnderline}
    \newtheorem{theorem}{定理}[section]
    \newtheorem{axiom}[theorem]{公理}
    \newtheorem{corollary}[theorem]{系}
    \newtheorem{proposition}[theorem]{命題}
    \newtheorem{lemma}[theorem]{補題}
    \newtheorem{definition}[theorem]{定義}
    \newtheorem{problem}[theorem]{問題}
    \newtheorem{exercise}[theorem]{Exercise}
\theoremstyle{definition}
    \newtheorem{issue}{論点}
    \newtheorem*{proposition*}{命題}
    \newtheorem*{lemma*}{補題}
    \newtheorem*{consideration*}{考察}
    \newtheorem*{theorem*}{定理}
    \newtheorem*{remarks*}{要諦}
    \newtheorem{example}[theorem]{例}
    \newtheorem{notation}[theorem]{記法}
    \newtheorem*{notation*}{記法}
    \newtheorem{assumption}[theorem]{仮定}
    \newtheorem{question}[theorem]{問}
    \newtheorem{counterexample}[theorem]{反例}
    \newtheorem{reidai}[theorem]{例題}
    \newtheorem{ruidai}[theorem]{類題}
    \newtheorem{algorithm}[theorem]{算譜}
    \newtheorem*{feels*}{所感}
    \newtheorem*{solution*}{\bf{[解]}}
    \newtheorem{discussion}[theorem]{議論}
    \newtheorem{synopsis}[theorem]{要約}
    \newtheorem{cited}[theorem]{引用}
    \newtheorem{remark}[theorem]{注}
    \newtheorem{remarks}[theorem]{要諦}
    \newtheorem{memo}[theorem]{メモ}
    \newtheorem{image}[theorem]{描像}
    \newtheorem{observation}[theorem]{観察}
    \newtheorem{universality}[theorem]{普遍性} %非自明な例外がない.
    \newtheorem{universal tendency}[theorem]{普遍傾向} %例外が有意に少ない.
    \newtheorem{hypothesis}[theorem]{仮説} %実験で説明されていない理論.
    \newtheorem{theory}[theorem]{理論} %実験事実とその(さしあたり)整合的な説明.
    \newtheorem{fact}[theorem]{実験事実}
    \newtheorem{model}[theorem]{模型}
    \newtheorem{explanation}[theorem]{説明} %理論による実験事実の説明
    \newtheorem{anomaly}[theorem]{理論の限界}
    \newtheorem{application}[theorem]{応用例}
    \newtheorem{method}[theorem]{手法} %実験手法など,技術的問題.
    \newtheorem{test}[theorem]{検定}
    \newtheorem{terms}[theorem]{用語}
    \newtheorem{solution}[theorem]{解法}
    \newtheorem{history}[theorem]{歴史}
    \newtheorem{usage}[theorem]{用語法}
    \newtheorem{research}[theorem]{研究}
    \newtheorem{shishin}[theorem]{指針}
    \newtheorem{yodan}[theorem]{余談}
    \newtheorem{construction}[theorem]{構成}
    \newtheorem{motivation}[theorem]{動機}
    \newtheorem{context}[theorem]{背景}
    \newtheorem{advantage}[theorem]{利点}
    \newtheorem*{definition*}{定義}
    \newtheorem*{remark*}{注意}
    \newtheorem*{question*}{問}
    \newtheorem*{problem*}{問題}
    \newtheorem*{axiom*}{公理}
    \newtheorem*{example*}{例}
    \newtheorem*{corollary*}{系}
    \newtheorem*{shishin*}{指針}
    \newtheorem*{yodan*}{余談}
    \newtheorem*{kadai*}{課題}

\raggedbottom
\allowdisplaybreaks
%%%%%%%%%%%%%%%% 数理文書の組版 %%%%%%%%%%%%%%%

\usepackage{mathtools} %内部でamsmathを呼び出すことに注意.
%\mathtoolsset{showonlyrefs=true} %labelを附した数式にのみ附番される設定.
\usepackage{amsfonts} %mathfrak, mathcal, mathbbなど.
\usepackage{amsthm} %定理環境.
\usepackage{amssymb} %AMSFontsを使うためのパッケージ.
\usepackage{ascmac} %screen, itembox, shadebox環境.全てLATEX2εの標準機能の範囲で作られたもの.
\usepackage{comment} %comment環境を用いて,複数行をcomment outできるようにするpackage
\usepackage{wrapfig} %図の周りに文字をwrapさせることができる.詳細な制御ができる.
\usepackage[usenames, dvipsnames]{xcolor} %xcolorはcolorの拡張.optionの意味はdvipsnamesはLoad a set of predefined colors. forestgreenなどの色が追加されている.usenamesはobsoleteとだけ書いてあった.
\setcounter{tocdepth}{2} %目次に表示される深さ.2はsubsectionまで
\usepackage{multicol} %\begin{multicols}{2}環境で途中からmulticolumnに出来る.
\usepackage{mathabx}\newcommand{\wc}{\widecheck} %\widecheckなどのフォントパッケージ

%%%%%%%%%%%%%%% フォント %%%%%%%%%%%%%%%

\usepackage{textcomp, mathcomp} %Text Companionとは,T1 encodingに入らなかった文字群.これを使うためのパッケージ.\textsectionでブルバキに!
\usepackage[T1]{fontenc} %8bitエンコーディングにする.comp系拡張数学文字の動作が安定する.

%%%%%%%%%%%%%%% 一般文書の組版 %%%%%%%%%%%%%%%

\definecolor{花緑青}{cmyk}{1,0.07,0.10,0.10}\definecolor{サーモンピンク}{cmyk}{0,0.65,0.65,0.05}\definecolor{暗中模索}{rgb}{0.2,0.2,0.2}
\usepackage{url}\usepackage[dvipdfmx,colorlinks,linkcolor=花緑青,urlcolor=花緑青,citecolor=花緑青]{hyperref} %生成されるPDFファイルにおいて、\tableofcontentsによって書き出された目次をクリックすると該当する見出しへジャンプしたり、さらには、\label{ラベル名}を番号で参照する\ref{ラベル名}やthebibliography環境において\bibitem{ラベル名}を文献番号で参照する\cite{ラベル名}においても番号をクリックすると該当箇所にジャンプする.囲み枠はダサいので,colorlinksで囲み廃止し,リンク自体に色を付けることにした.
\usepackage{pxjahyper} %pxrubrica同様,八登崇之さん.hyperrefは日本語pLaTeXに最適化されていないから,hyperrefとセットで,(u)pLaTeX+hyperref+dvipdfmxの組み合わせで日本語を含む「しおり」をもつPDF文書を作成する場合に必要となる機能を提供する
\usepackage{ulem} %取り消し線を引くためのパッケージ
\usepackage{pxrubrica} %日本語にルビをふる.八登崇之(やとうたかゆき)氏による.

%%%%%%%%%%%%%%% 科学文書の組版 %%%%%%%%%%%%%%%

\usepackage[version=4]{mhchem} %化学式をTikZで簡単に書くためのパッケージ.
\usepackage{chemfig} %化学構造式をTikZで描くためのパッケージ.
\usepackage{siunitx} %IS単位を書くためのパッケージ

%%%%%%%%%%%%%%% 作図 %%%%%%%%%%%%%%%

\usepackage{tikz}\usetikzlibrary{positioning,automata}\usepackage{tikz-cd}\usepackage[all]{xy}
\def\objectstyle{\displaystyle} %デフォルトではxymatrix中の数式が文中数式モードになるので,それを直す.\labelstyleも同様にxy packageの中で定義されており,文中数式モードになっている.

\usepackage{graphicx} %rotatebox, scalebox, reflectbox, resizeboxなどのコマンドや,図表の読み込み\includegraphicsを司る.graphics というパッケージもありますが,graphicx はこれを高機能にしたものと考えて結構です(ただし graphicx は内部で graphics を読み込みます)
\usepackage[top=15truemm,bottom=15truemm,left=10truemm,right=10truemm]{geometry} %足助さんからもらったオプション

%%%%%%%%%%%%%%% 参照 %%%%%%%%%%%%%%%
%参考文献リストを出力したい箇所に\bibliography{../mathematics.bib}を追記すると良い.

%\bibliographystyle{jplain}
%\bibliographystyle{jname}
\bibliographystyle{apalike}

%%%%%%%%%%%%%%% 計算機文書の組版 %%%%%%%%%%%%%%%

\usepackage[breakable]{tcolorbox} %加藤晃史さんがフル活用していたtcolorboxを,途中改ページ可能で.
\tcbuselibrary{theorems} %https://qiita.com/t_kemmochi/items/483b8fcdb5db8d1f5d5e
\usepackage{enumerate} %enumerate環境を凝らせる.

\usepackage{listings} %ソースコードを表示できる環境.多分もっといい方法ある.
\usepackage{jvlisting} %日本語のコメントアウトをする場合jlistingが必要
\lstset{ %ここからソースコードの表示に関する設定.lstlisting環境では,[caption=hoge,label=fuga]などのoptionを付けられる.
%[escapechar=!]とすると,LaTeXコマンドを使える.
  basicstyle={\ttfamily},
  identifierstyle={\small},
  commentstyle={\smallitshape},
  keywordstyle={\small\bfseries},
  ndkeywordstyle={\small},
  stringstyle={\small\ttfamily},
  frame={tb},
  breaklines=true,
  columns=[l]{fullflexible},
  numbers=left,
  xrightmargin=0zw,
  xleftmargin=3zw,
  numberstyle={\scriptsize},
  stepnumber=1,
  numbersep=1zw,
  lineskip=-0.5ex
}
%\makeatletter %caption番号を「[chapter番号].[section番号].[subsection番号]-[そのsubsection内においてn番目]」に変更
%    \AtBeginDocument{
%    \renewcommand*{\thelstlisting}{\arabic{chapter}.\arabic{section}.\arabic{lstlisting}}
%    \@addtoreset{lstlisting}{section}
%    }
%\makeatother
\renewcommand{\lstlistingname}{算譜} %caption名を"program"に変更

\newtcolorbox{tbox}[3][]{%
colframe=#2,colback=#2!10,coltitle=#2!20!black,title={#3},#1}

% 証明内の文字が小さくなる環境.
\newenvironment{Proof}[1][\bf\underline{[証明]}]{\proof[#1]\color{darkgray}}{\endproof}

%%%%%%%%%%%%%%% 数学記号のマクロ %%%%%%%%%%%%%%%

%%% 括弧類
\newcommand{\abs}[1]{\lvert#1\rvert}\newcommand{\Abs}[1]{\left|#1\right|}\newcommand{\norm}[1]{\|#1\|}\newcommand{\Norm}[1]{\left\|#1\right\|}\newcommand{\Brace}[1]{\left\{#1\right\}}\newcommand{\BRace}[1]{\biggl\{#1\biggr\}}\newcommand{\paren}[1]{\left(#1\right)}\newcommand{\Paren}[1]{\biggr(#1\biggl)}\newcommand{\bracket}[1]{\langle#1\rangle}\newcommand{\brac}[1]{\langle#1\rangle}\newcommand{\Bracket}[1]{\left\langle#1\right\rangle}\newcommand{\Brac}[1]{\left\langle#1\right\rangle}\newcommand{\bra}[1]{\left\langle#1\right|}\newcommand{\ket}[1]{\left|#1\right\rangle}\newcommand{\Square}[1]{\left[#1\right]}\newcommand{\SQuare}[1]{\biggl[#1\biggr]}
\renewcommand{\o}[1]{\overline{#1}}\renewcommand{\u}[1]{\underline{#1}}\newcommand{\wt}[1]{\widetilde{#1}}\newcommand{\wh}[1]{\widehat{#1}}
\newcommand{\pp}[2]{\frac{\partial #1}{\partial #2}}\newcommand{\ppp}[3]{\frac{\partial #1}{\partial #2\partial #3}}\newcommand{\dd}[2]{\frac{d #1}{d #2}}
\newcommand{\floor}[1]{\lfloor#1\rfloor}\newcommand{\Floor}[1]{\left\lfloor#1\right\rfloor}\newcommand{\ceil}[1]{\lceil#1\rceil}
\newcommand{\ocinterval}[1]{(#1]}\newcommand{\cointerval}[1]{[#1)}\newcommand{\COinterval}[1]{\left[#1\right)}


%%% 予約語
\renewcommand{\iff}{\;\mathrm{iff}\;}
\newcommand{\False}{\mathrm{False}}\newcommand{\True}{\mathrm{True}}
\newcommand{\otherwise}{\mathrm{otherwise}}
\newcommand{\st}{\;\mathrm{s.t.}\;}

%%% 略記
\newcommand{\M}{\mathcal{M}}\newcommand{\cF}{\mathcal{F}}\newcommand{\cD}{\mathcal{D}}\newcommand{\fX}{\mathfrak{X}}\newcommand{\fY}{\mathfrak{Y}}\newcommand{\fZ}{\mathfrak{Z}}\renewcommand{\H}{\mathcal{H}}\newcommand{\fH}{\mathfrak{H}}\newcommand{\bH}{\mathbb{H}}\newcommand{\id}{\mathrm{id}}\newcommand{\A}{\mathcal{A}}\newcommand{\U}{\mathfrak{U}}
\newcommand{\lmd}{\lambda}
\newcommand{\Lmd}{\Lambda}

%%% 矢印類
\newcommand{\iso}{\xrightarrow{\,\smash{\raisebox{-0.45ex}{\ensuremath{\scriptstyle\sim}}}\,}}
\newcommand{\Lrarrow}{\;\;\Leftrightarrow\;\;}

%%% 注記
\newcommand{\rednote}[1]{\textcolor{red}{#1}}

% ノルム位相についての閉包 https://newbedev.com/how-to-make-double-overline-with-less-vertical-displacement
\makeatletter
\newcommand{\dbloverline}[1]{\overline{\dbl@overline{#1}}}
\newcommand{\dbl@overline}[1]{\mathpalette\dbl@@overline{#1}}
\newcommand{\dbl@@overline}[2]{%
  \begingroup
  \sbox\z@{$\m@th#1\overline{#2}$}%
  \ht\z@=\dimexpr\ht\z@-2\dbl@adjust{#1}\relax
  \box\z@
  \ifx#1\scriptstyle\kern-\scriptspace\else
  \ifx#1\scriptscriptstyle\kern-\scriptspace\fi\fi
  \endgroup
}
\newcommand{\dbl@adjust}[1]{%
  \fontdimen8
  \ifx#1\displaystyle\textfont\else
  \ifx#1\textstyle\textfont\else
  \ifx#1\scriptstyle\scriptfont\else
  \scriptscriptfont\fi\fi\fi 3
}
\makeatother
\newcommand{\oo}[1]{\dbloverline{#1}}

% hslashの他の文字Ver.
\newcommand{\hslashslash}{%
    \scalebox{1.2}{--
    }%
}
\newcommand{\dslash}{%
  {%
    \vphantom{d}%
    \ooalign{\kern.05em\smash{\hslashslash}\hidewidth\cr$d$\cr}%
    \kern.05em
  }%
}
\newcommand{\dint}{%
  {%
    \vphantom{d}%
    \ooalign{\kern.05em\smash{\hslashslash}\hidewidth\cr$\int$\cr}%
    \kern.05em
  }%
}
\newcommand{\dL}{%
  {%
    \vphantom{d}%
    \ooalign{\kern.05em\smash{\hslashslash}\hidewidth\cr$L$\cr}%
    \kern.05em
  }%
}

%%% 演算子
\DeclareMathOperator{\grad}{\mathrm{grad}}\DeclareMathOperator{\rot}{\mathrm{rot}}\DeclareMathOperator{\divergence}{\mathrm{div}}\DeclareMathOperator{\tr}{\mathrm{tr}}\newcommand{\pr}{\mathrm{pr}}
\newcommand{\Map}{\mathrm{Map}}\newcommand{\dom}{\mathrm{Dom}\;}\newcommand{\cod}{\mathrm{Cod}\;}\newcommand{\supp}{\mathrm{supp}\;}


%%% 線型代数学
\newcommand{\vctr}[2]{\begin{pmatrix}#1\\#2\end{pmatrix}}\newcommand{\vctrr}[3]{\begin{pmatrix}#1\\#2\\#3\end{pmatrix}}\newcommand{\mtrx}[4]{\begin{pmatrix}#1&#2\\#3&#4\end{pmatrix}}\newcommand{\smtrx}[4]{\paren{\begin{smallmatrix}#1&#2\\#3&#4\end{smallmatrix}}}\newcommand{\Ker}{\mathrm{Ker}\;}\newcommand{\Coker}{\mathrm{Coker}\;}\newcommand{\Coim}{\mathrm{Coim}\;}\DeclareMathOperator{\rank}{\mathrm{rank}}\newcommand{\lcm}{\mathrm{lcm}}\newcommand{\sgn}{\mathrm{sgn}\,}\newcommand{\GL}{\mathrm{GL}}\newcommand{\SL}{\mathrm{SL}}\newcommand{\alt}{\mathrm{alt}}
%%% 複素解析学
\renewcommand{\Re}{\mathrm{Re}\;}\renewcommand{\Im}{\mathrm{Im}\;}\newcommand{\Gal}{\mathrm{Gal}}\newcommand{\PGL}{\mathrm{PGL}}\newcommand{\PSL}{\mathrm{PSL}}\newcommand{\Log}{\mathrm{Log}\,}\newcommand{\Res}{\mathrm{Res}\,}\newcommand{\on}{\mathrm{on}\;}\newcommand{\hatC}{\widehat{\C}}\newcommand{\hatR}{\hat{\R}}\newcommand{\PV}{\mathrm{P.V.}}\newcommand{\diam}{\mathrm{diam}}\newcommand{\Area}{\mathrm{Area}}\newcommand{\Lap}{\Laplace}\newcommand{\f}{\mathbf{f}}\newcommand{\cR}{\mathcal{R}}\newcommand{\const}{\mathrm{const.}}\newcommand{\Om}{\Omega}\newcommand{\Cinf}{C^\infty}\newcommand{\ep}{\epsilon}\newcommand{\dist}{\mathrm{dist}}\newcommand{\opart}{\o{\partial}}\newcommand{\Length}{\mathrm{Length}}
%%% 集合と位相
\renewcommand{\O}{\mathcal{O}}\renewcommand{\S}{\mathcal{S}}\renewcommand{\U}{\mathcal{U}}\newcommand{\V}{\mathcal{V}}\renewcommand{\P}{\mathcal{P}}\newcommand{\R}{\mathbb{R}}\newcommand{\N}{\mathbb{N}}\newcommand{\C}{\mathbb{C}}\newcommand{\Z}{\mathbb{Z}}\newcommand{\Q}{\mathbb{Q}}\newcommand{\TV}{\mathrm{TV}}\newcommand{\ORD}{\mathrm{ORD}}\newcommand{\Tr}{\mathrm{Tr}}\newcommand{\Card}{\mathrm{Card}\;}\newcommand{\Top}{\mathrm{Top}}\newcommand{\Disc}{\mathrm{Disc}}\newcommand{\Codisc}{\mathrm{Codisc}}\newcommand{\CoDisc}{\mathrm{CoDisc}}\newcommand{\Ult}{\mathrm{Ult}}\newcommand{\ord}{\mathrm{ord}}\newcommand{\maj}{\mathrm{maj}}\newcommand{\bS}{\mathbb{S}}\newcommand{\PConn}{\mathrm{PConn}}

%%% 形式言語理論
\newcommand{\REGEX}{\mathrm{REGEX}}\newcommand{\RE}{\mathbf{RE}}
%%% Graph Theory
\newcommand{\SimpGph}{\mathrm{SimpGph}}\newcommand{\Gph}{\mathrm{Gph}}\newcommand{\mult}{\mathrm{mult}}\newcommand{\inv}{\mathrm{inv}}

%%% 多様体
\newcommand{\Der}{\mathrm{Der}}\newcommand{\osub}{\overset{\mathrm{open}}{\subset}}\newcommand{\osup}{\overset{\mathrm{open}}{\supset}}\newcommand{\al}{\alpha}\newcommand{\K}{\mathbb{K}}\newcommand{\Sp}{\mathrm{Sp}}\newcommand{\g}{\mathfrak{g}}\newcommand{\h}{\mathfrak{h}}\newcommand{\Exp}{\mathrm{Exp}\;}\newcommand{\Imm}{\mathrm{Imm}}\newcommand{\Imb}{\mathrm{Imb}}\newcommand{\codim}{\mathrm{codim}\;}\newcommand{\Gr}{\mathrm{Gr}}
%%% 代数
\newcommand{\Ad}{\mathrm{Ad}}\newcommand{\finsupp}{\mathrm{fin\;supp}}\newcommand{\SO}{\mathrm{SO}}\newcommand{\SU}{\mathrm{SU}}\newcommand{\acts}{\curvearrowright}\newcommand{\mono}{\hookrightarrow}\newcommand{\epi}{\twoheadrightarrow}\newcommand{\Stab}{\mathrm{Stab}}\newcommand{\nor}{\mathrm{nor}}\newcommand{\T}{\mathbb{T}}\newcommand{\Aff}{\mathrm{Aff}}\newcommand{\rsub}{\triangleleft}\newcommand{\rsup}{\triangleright}\newcommand{\subgrp}{\overset{\mathrm{subgrp}}{\subset}}\newcommand{\Ext}{\mathrm{Ext}}\newcommand{\sbs}{\subset}\newcommand{\sps}{\supset}\newcommand{\In}{\mathrm{in}\;}\newcommand{\Tor}{\mathrm{Tor}}\newcommand{\p}{\b{p}}\newcommand{\q}{\mathfrak{q}}\newcommand{\m}{\mathfrak{m}}\newcommand{\cS}{\mathcal{S}}\newcommand{\Frac}{\mathrm{Frac}\,}\newcommand{\Spec}{\mathrm{Spec}\,}\newcommand{\bA}{\mathbb{A}}\newcommand{\Sym}{\mathrm{Sym}}\newcommand{\Ann}{\mathrm{Ann}}\newcommand{\Her}{\mathrm{Her}}\newcommand{\Bil}{\mathrm{Bil}}\newcommand{\Ses}{\mathrm{Ses}}\newcommand{\FVS}{\mathrm{FVS}}
%%% 代数的位相幾何学
\newcommand{\Ho}{\mathrm{Ho}}\newcommand{\CW}{\mathrm{CW}}\newcommand{\lc}{\mathrm{lc}}\newcommand{\cg}{\mathrm{cg}}\newcommand{\Fib}{\mathrm{Fib}}\newcommand{\Cyl}{\mathrm{Cyl}}\newcommand{\Ch}{\mathrm{Ch}}
%%% 微分幾何学
\newcommand{\rE}{\mathrm{E}}\newcommand{\e}{\b{e}}\renewcommand{\k}{\b{k}}\newcommand{\Christ}[2]{\begin{Bmatrix}#1\\#2\end{Bmatrix}}\renewcommand{\Vec}[1]{\overrightarrow{\mathrm{#1}}}\newcommand{\hen}[1]{\mathrm{#1}}\renewcommand{\b}[1]{\boldsymbol{#1}}

%%% 函数解析
\newcommand{\HS}{\mathrm{HS}}\newcommand{\loc}{\mathrm{loc}}\newcommand{\Lh}{\mathrm{L.h.}}\newcommand{\Epi}{\mathrm{Epi}\;}\newcommand{\slim}{\mathrm{slim}}\newcommand{\Ban}{\mathrm{Ban}}\newcommand{\Hilb}{\mathrm{Hilb}}\newcommand{\Ex}{\mathrm{Ex}}\newcommand{\Co}{\mathrm{Co}}\newcommand{\sa}{\mathrm{sa}}\newcommand{\nnorm}[1]{{\left\vert\kern-0.25ex\left\vert\kern-0.25ex\left\vert #1 \right\vert\kern-0.25ex\right\vert\kern-0.25ex\right\vert}}\newcommand{\dvol}{\mathrm{dvol}}\newcommand{\Sconv}{\mathrm{Sconv}}\newcommand{\I}{\mathcal{I}}\newcommand{\nonunital}{\mathrm{nu}}\newcommand{\cpt}{\mathrm{cpt}}\newcommand{\lcpt}{\mathrm{lcpt}}\newcommand{\com}{\mathrm{com}}\newcommand{\Haus}{\mathrm{Haus}}\newcommand{\proper}{\mathrm{proper}}\newcommand{\infinity}{\mathrm{inf}}\newcommand{\TVS}{\mathrm{TVS}}\newcommand{\ess}{\mathrm{ess}}\newcommand{\ext}{\mathrm{ext}}\newcommand{\Index}{\mathrm{Index}\;}\newcommand{\SSR}{\mathrm{SSR}}\newcommand{\vs}{\mathrm{vs.}}\newcommand{\fM}{\mathfrak{M}}\newcommand{\EDM}{\mathrm{EDM}}\newcommand{\Tw}{\mathrm{Tw}}\newcommand{\fC}{\mathfrak{C}}\newcommand{\bn}{\boldsymbol{n}}\newcommand{\br}{\boldsymbol{r}}\newcommand{\Lam}{\Lambda}\newcommand{\lam}{\lambda}\newcommand{\one}{\mathbf{1}}\newcommand{\dae}{\text{-a.e.}}\newcommand{\das}{\text{-a.s.}}\newcommand{\td}{\text{-}}\newcommand{\RM}{\mathrm{RM}}\newcommand{\BV}{\mathrm{BV}}\newcommand{\normal}{\mathrm{normal}}\newcommand{\lub}{\mathrm{lub}\;}\newcommand{\Graph}{\mathrm{Graph}}\newcommand{\Ascent}{\mathrm{Ascent}}\newcommand{\Descent}{\mathrm{Descent}}\newcommand{\BIL}{\mathrm{BIL}}\newcommand{\fL}{\mathfrak{L}}\newcommand{\De}{\Delta}
%%% 積分論
\newcommand{\calA}{\mathcal{A}}\newcommand{\calB}{\mathcal{B}}\newcommand{\D}{\mathcal{D}}\newcommand{\Y}{\mathcal{Y}}\newcommand{\calC}{\mathcal{C}}\renewcommand{\ae}{\mathrm{a.e.}\;}\newcommand{\cZ}{\mathcal{Z}}\newcommand{\fF}{\mathfrak{F}}\newcommand{\fI}{\mathfrak{I}}\newcommand{\E}{\mathcal{E}}\newcommand{\sMap}{\sigma\textrm{-}\mathrm{Map}}\DeclareMathOperator*{\argmax}{arg\,max}\DeclareMathOperator*{\argmin}{arg\,min}\newcommand{\cC}{\mathcal{C}}\newcommand{\comp}{\complement}\newcommand{\J}{\mathcal{J}}\newcommand{\sumN}[1]{\sum_{#1\in\N}}\newcommand{\cupN}[1]{\cup_{#1\in\N}}\newcommand{\capN}[1]{\cap_{#1\in\N}}\newcommand{\Sum}[1]{\sum_{#1=1}^\infty}\newcommand{\sumn}{\sum_{n=1}^\infty}\newcommand{\summ}{\sum_{m=1}^\infty}\newcommand{\sumk}{\sum_{k=1}^\infty}\newcommand{\sumi}{\sum_{i=1}^\infty}\newcommand{\sumj}{\sum_{j=1}^\infty}\newcommand{\cupn}{\cup_{n=1}^\infty}\newcommand{\capn}{\cap_{n=1}^\infty}\newcommand{\cupk}{\cup_{k=1}^\infty}\newcommand{\cupi}{\cup_{i=1}^\infty}\newcommand{\cupj}{\cup_{j=1}^\infty}\newcommand{\limn}{\lim_{n\to\infty}}\renewcommand{\l}{\mathcal{l}}\renewcommand{\L}{\mathcal{L}}\newcommand{\Cl}{\mathrm{Cl}}\newcommand{\cN}{\mathcal{N}}\newcommand{\Ae}{\textrm{-a.e.}\;}\newcommand{\csub}{\overset{\textrm{closed}}{\subset}}\newcommand{\csup}{\overset{\textrm{closed}}{\supset}}\newcommand{\wB}{\wt{B}}\newcommand{\cG}{\mathcal{G}}\newcommand{\Lip}{\mathrm{Lip}}\DeclareMathOperator{\Dom}{\mathrm{Dom}}\newcommand{\AC}{\mathrm{AC}}\newcommand{\Mol}{\mathrm{Mol}}
%%% Fourier解析
\newcommand{\Pe}{\mathrm{Pe}}\newcommand{\wR}{\wh{\mathbb{\R}}}\newcommand*{\Laplace}{\mathop{}\!\mathbin\bigtriangleup}\newcommand*{\DAlambert}{\mathop{}\!\mathbin\Box}\newcommand{\bT}{\mathbb{T}}\newcommand{\dx}{\dslash x}\newcommand{\dt}{\dslash t}\newcommand{\ds}{\dslash s}
%%% 数値解析
\newcommand{\round}{\mathrm{round}}\newcommand{\cond}{\mathrm{cond}}\newcommand{\diag}{\mathrm{diag}}
\newcommand{\Adj}{\mathrm{Adj}}\newcommand{\Pf}{\mathrm{Pf}}\newcommand{\Sg}{\mathrm{Sg}}

%%% 確率論
\newcommand{\Prob}{\mathrm{Prob}}\newcommand{\X}{\mathcal{X}}\newcommand{\Meas}{\mathrm{Meas}}\newcommand{\as}{\;\mathrm{a.s.}}\newcommand{\io}{\;\mathrm{i.o.}}\newcommand{\fe}{\;\mathrm{f.e.}}\newcommand{\F}{\mathcal{F}}\newcommand{\bF}{\mathbb{F}}\newcommand{\W}{\mathcal{W}}\newcommand{\Pois}{\mathrm{Pois}}\newcommand{\iid}{\mathrm{i.i.d.}}\newcommand{\wconv}{\rightsquigarrow}\newcommand{\Var}{\mathrm{Var}}\newcommand{\xrightarrown}{\xrightarrow{n\to\infty}}\newcommand{\au}{\mathrm{au}}\newcommand{\cT}{\mathcal{T}}\newcommand{\wto}{\overset{w}{\to}}\newcommand{\dto}{\overset{d}{\to}}\newcommand{\pto}{\overset{p}{\to}}\newcommand{\vto}{\overset{v}{\to}}\newcommand{\Cont}{\mathrm{Cont}}\newcommand{\stably}{\mathrm{stably}}\newcommand{\Np}{\mathbb{N}^+}\newcommand{\oM}{\overline{\mathcal{M}}}\newcommand{\fP}{\mathfrak{P}}\newcommand{\sign}{\mathrm{sign}}\DeclareMathOperator{\Div}{Div}
\newcommand{\bD}{\mathbb{D}}\newcommand{\fW}{\mathfrak{W}}\newcommand{\DL}{\mathcal{D}\mathcal{L}}\renewcommand{\r}[1]{\mathrm{#1}}\newcommand{\rC}{\mathrm{C}}
%%% 情報理論
\newcommand{\bit}{\mathrm{bit}}\DeclareMathOperator{\sinc}{sinc}
%%% 量子論
\newcommand{\err}{\mathrm{err}}
%%% 最適化
\newcommand{\varparallel}{\mathbin{\!/\mkern-5mu/\!}}\newcommand{\Minimize}{\text{Minimize}}\newcommand{\subjectto}{\text{subject to}}\newcommand{\Ri}{\mathrm{Ri}}\newcommand{\Cone}{\mathrm{Cone}}\newcommand{\Int}{\mathrm{Int}}
%%% 数理ファイナンス
\newcommand{\pre}{\mathrm{pre}}\newcommand{\om}{\omega}

%%% 偏微分方程式
\let\div\relax
\DeclareMathOperator{\div}{div}\newcommand{\del}{\partial}
\newcommand{\LHS}{\mathrm{LHS}}\newcommand{\RHS}{\mathrm{RHS}}\newcommand{\bnu}{\boldsymbol{\nu}}\newcommand{\interior}{\mathrm{in}\;}\newcommand{\SH}{\mathrm{SH}}\renewcommand{\v}{\boldsymbol{\nu}}\newcommand{\n}{\mathbf{n}}\newcommand{\ssub}{\Subset}\newcommand{\curl}{\mathrm{curl}}
%%% 常微分方程式
\newcommand{\Ei}{\mathrm{Ei}}\newcommand{\sn}{\mathrm{sn}}\newcommand{\wgamma}{\widetilde{\gamma}}
%%% 統計力学
\newcommand{\Ens}{\mathrm{Ens}}
%%% 解析力学
\newcommand{\cl}{\mathrm{cl}}\newcommand{\x}{\boldsymbol{x}}

%%% 統計的因果推論
\newcommand{\Do}{\mathrm{Do}}
%%% 応用統計学
\newcommand{\mrl}{\mathrm{mrl}}
%%% 数理統計
\newcommand{\comb}[2]{\begin{pmatrix}#1\\#2\end{pmatrix}}\newcommand{\bP}{\mathbb{P}}\newcommand{\compsub}{\overset{\textrm{cpt}}{\subset}}\newcommand{\lip}{\textrm{lip}}\newcommand{\BL}{\mathrm{BL}}\newcommand{\G}{\mathbb{G}}\newcommand{\NB}{\mathrm{NB}}\newcommand{\oR}{\o{\R}}\newcommand{\liminfn}{\liminf_{n\to\infty}}\newcommand{\limsupn}{\limsup_{n\to\infty}}\newcommand{\esssup}{\mathrm{ess.sup}}\newcommand{\asto}{\xrightarrow{\as}}\newcommand{\Cov}{\mathrm{Cov}}\newcommand{\cQ}{\mathcal{Q}}\newcommand{\VC}{\mathrm{VC}}\newcommand{\mb}{\mathrm{mb}}\newcommand{\Avar}{\mathrm{Avar}}\newcommand{\bB}{\mathbb{B}}\newcommand{\bW}{\mathbb{W}}\newcommand{\sd}{\mathrm{sd}}\newcommand{\w}[1]{\widehat{#1}}\newcommand{\bZ}{\boldsymbol{Z}}\newcommand{\Bernoulli}{\mathrm{Ber}}\newcommand{\Ber}{\mathrm{Ber}}\newcommand{\Mult}{\mathrm{Mult}}\newcommand{\BPois}{\mathrm{BPois}}\newcommand{\fraks}{\mathfrak{s}}\newcommand{\frakk}{\mathfrak{k}}\newcommand{\IF}{\mathrm{IF}}\newcommand{\bX}{\mathbf{X}}\newcommand{\bx}{\boldsymbol{x}}\newcommand{\indep}{\raisebox{0.05em}{\rotatebox[origin=c]{90}{$\models$}}}\newcommand{\IG}{\mathrm{IG}}\newcommand{\Levy}{\mathrm{Levy}}\newcommand{\MP}{\mathrm{MP}}\newcommand{\Hermite}{\mathrm{Hermite}}\newcommand{\Skellam}{\mathrm{Skellam}}\newcommand{\Dirichlet}{\mathrm{Dirichlet}}\newcommand{\Beta}{\mathrm{Beta}}\newcommand{\bE}{\mathbb{E}}\newcommand{\bG}{\mathbb{G}}\newcommand{\MISE}{\mathrm{MISE}}\newcommand{\logit}{\mathtt{logit}}\newcommand{\expit}{\mathtt{expit}}\newcommand{\cK}{\mathcal{K}}\newcommand{\dl}{\dot{l}}\newcommand{\dotp}{\dot{p}}\newcommand{\wl}{\wt{l}}\newcommand{\Gauss}{\mathrm{Gauss}}\newcommand{\fA}{\mathfrak{A}}\newcommand{\under}{\mathrm{under}\;}\newcommand{\whtheta}{\wh{\theta}}\newcommand{\Em}{\mathrm{Em}}\newcommand{\ztheta}{{\theta_0}}
\newcommand{\rO}{\mathrm{O}}\newcommand{\Bin}{\mathrm{Bin}}\newcommand{\rW}{\mathrm{W}}\newcommand{\rG}{\mathrm{G}}\newcommand{\rB}{\mathrm{B}}\newcommand{\rN}{\mathrm{N}}\newcommand{\rU}{\mathrm{U}}\newcommand{\HG}{\mathrm{HG}}\newcommand{\GAMMA}{\mathrm{Gamma}}\newcommand{\Cauchy}{\mathrm{Cauchy}}\newcommand{\rt}{\mathrm{t}}
\DeclareMathOperator{\erf}{erf}

%%% 圏
\newcommand{\varlim}{\varprojlim}\newcommand{\Hom}{\mathrm{Hom}}\newcommand{\Iso}{\mathrm{Iso}}\newcommand{\Mor}{\mathrm{Mor}}\newcommand{\Isom}{\mathrm{Isom}}\newcommand{\Aut}{\mathrm{Aut}}\newcommand{\End}{\mathrm{End}}\newcommand{\op}{\mathrm{op}}\newcommand{\ev}{\mathrm{ev}}\newcommand{\Ob}{\mathrm{Ob}}\newcommand{\Ar}{\mathrm{Ar}}\newcommand{\Arr}{\mathrm{Arr}}\newcommand{\Set}{\mathrm{Set}}\newcommand{\Grp}{\mathrm{Grp}}\newcommand{\Cat}{\mathrm{Cat}}\newcommand{\Mon}{\mathrm{Mon}}\newcommand{\Ring}{\mathrm{Ring}}\newcommand{\CRing}{\mathrm{CRing}}\newcommand{\Ab}{\mathrm{Ab}}\newcommand{\Pos}{\mathrm{Pos}}\newcommand{\Vect}{\mathrm{Vect}}\newcommand{\FinVect}{\mathrm{FinVect}}\newcommand{\FinSet}{\mathrm{FinSet}}\newcommand{\FinMeas}{\mathrm{FinMeas}}\newcommand{\OmegaAlg}{\Omega\text{-}\mathrm{Alg}}\newcommand{\OmegaEAlg}{(\Omega,E)\text{-}\mathrm{Alg}}\newcommand{\Fun}{\mathrm{Fun}}\newcommand{\Func}{\mathrm{Func}}\newcommand{\Alg}{\mathrm{Alg}} %代数の圏
\newcommand{\CAlg}{\mathrm{CAlg}} %可換代数の圏
\newcommand{\Met}{\mathrm{Met}} %Metric space & Contraction maps
\newcommand{\Rel}{\mathrm{Rel}} %Sets & relation
\newcommand{\Bool}{\mathrm{Bool}}\newcommand{\CABool}{\mathrm{CABool}}\newcommand{\CompBoolAlg}{\mathrm{CompBoolAlg}}\newcommand{\BoolAlg}{\mathrm{BoolAlg}}\newcommand{\BoolRng}{\mathrm{BoolRng}}\newcommand{\HeytAlg}{\mathrm{HeytAlg}}\newcommand{\CompHeytAlg}{\mathrm{CompHeytAlg}}\newcommand{\Lat}{\mathrm{Lat}}\newcommand{\CompLat}{\mathrm{CompLat}}\newcommand{\SemiLat}{\mathrm{SemiLat}}\newcommand{\Stone}{\mathrm{Stone}}\newcommand{\Mfd}{\mathrm{Mfd}}\newcommand{\LieAlg}{\mathrm{LieAlg}}
\newcommand{\Sob}{\mathrm{Sob}} %Sober space & continuous map
\newcommand{\Op}{\mathrm{Op}} %Category of open subsets
\newcommand{\Sh}{\mathrm{Sh}} %Category of sheave
\newcommand{\PSh}{\mathrm{PSh}} %Category of presheave, PSh(C)=[C^op,set]のこと
\newcommand{\Conv}{\mathrm{Conv}} %Convergence spaceの圏
\newcommand{\Unif}{\mathrm{Unif}} %一様空間と一様連続写像の圏
\newcommand{\Frm}{\mathrm{Frm}} %フレームとフレームの射
\newcommand{\Locale}{\mathrm{Locale}} %その反対圏
\newcommand{\Diff}{\mathrm{Diff}} %滑らかな多様体の圏
\newcommand{\Quiv}{\mathrm{Quiv}} %Quiverの圏
\newcommand{\B}{\mathcal{B}}\newcommand{\Span}{\mathrm{Span}}\newcommand{\Corr}{\mathrm{Corr}}\newcommand{\Decat}{\mathrm{Decat}}\newcommand{\Rep}{\mathrm{Rep}}\newcommand{\Grpd}{\mathrm{Grpd}}\newcommand{\sSet}{\mathrm{sSet}}\newcommand{\Mod}{\mathrm{Mod}}\newcommand{\SmoothMnf}{\mathrm{SmoothMnf}}\newcommand{\coker}{\mathrm{coker}}\newcommand{\Ord}{\mathrm{Ord}}\newcommand{\eq}{\mathrm{eq}}\newcommand{\coeq}{\mathrm{coeq}}\newcommand{\act}{\mathrm{act}}

%%%%%%%%%%%%%%% 定理環境(足助先生ありがとうございます) %%%%%%%%%%%%%%%

\everymath{\displaystyle}
\renewcommand{\proofname}{\bf\underline{[証明]}}
\renewcommand{\thefootnote}{\dag\arabic{footnote}} %足助さんからもらった.どうなるんだ?
\renewcommand{\qedsymbol}{$\blacksquare$}

\renewcommand{\labelenumi}{(\arabic{enumi})} %(1),(2),...がデフォルトであって欲しい
\renewcommand{\labelenumii}{(\alph{enumii})}
\renewcommand{\labelenumiii}{(\roman{enumiii})}

\newtheoremstyle{StatementsWithUnderline}% ?name?
{3pt}% ?Space above? 1
{3pt}% ?Space below? 1
{}% ?Body font?
{}% ?Indent amount? 2
{\bfseries}% ?Theorem head font?
{\textbf{.}}% ?Punctuation after theorem head?
{.5em}% ?Space after theorem head? 3
{\textbf{\underline{\textup{#1~\thetheorem{}}}}\;\thmnote{(#3)}}% ?Theorem head spec (can be left empty, meaning ‘normal’)?

\usepackage{etoolbox}
\AtEndEnvironment{example}{\hfill\ensuremath{\Box}}
\AtEndEnvironment{observation}{\hfill\ensuremath{\Box}}

\theoremstyle{StatementsWithUnderline}
    \newtheorem{theorem}{定理}[section]
    \newtheorem{axiom}[theorem]{公理}
    \newtheorem{corollary}[theorem]{系}
    \newtheorem{proposition}[theorem]{命題}
    \newtheorem{lemma}[theorem]{補題}
    \newtheorem{definition}[theorem]{定義}
    \newtheorem{problem}[theorem]{問題}
    \newtheorem{exercise}[theorem]{Exercise}
\theoremstyle{definition}
    \newtheorem{issue}{論点}
    \newtheorem*{proposition*}{命題}
    \newtheorem*{lemma*}{補題}
    \newtheorem*{consideration*}{考察}
    \newtheorem*{theorem*}{定理}
    \newtheorem*{remarks*}{要諦}
    \newtheorem{example}[theorem]{例}
    \newtheorem{notation}[theorem]{記法}
    \newtheorem*{notation*}{記法}
    \newtheorem{assumption}[theorem]{仮定}
    \newtheorem{question}[theorem]{問}
    \newtheorem{counterexample}[theorem]{反例}
    \newtheorem{reidai}[theorem]{例題}
    \newtheorem{ruidai}[theorem]{類題}
    \newtheorem{algorithm}[theorem]{算譜}
    \newtheorem*{feels*}{所感}
    \newtheorem*{solution*}{\bf{[解]}}
    \newtheorem{discussion}[theorem]{議論}
    \newtheorem{synopsis}[theorem]{要約}
    \newtheorem{cited}[theorem]{引用}
    \newtheorem{remark}[theorem]{注}
    \newtheorem{remarks}[theorem]{要諦}
    \newtheorem{memo}[theorem]{メモ}
    \newtheorem{image}[theorem]{描像}
    \newtheorem{observation}[theorem]{観察}
    \newtheorem{universality}[theorem]{普遍性} %非自明な例外がない.
    \newtheorem{universal tendency}[theorem]{普遍傾向} %例外が有意に少ない.
    \newtheorem{hypothesis}[theorem]{仮説} %実験で説明されていない理論.
    \newtheorem{theory}[theorem]{理論} %実験事実とその(さしあたり)整合的な説明.
    \newtheorem{fact}[theorem]{実験事実}
    \newtheorem{model}[theorem]{模型}
    \newtheorem{explanation}[theorem]{説明} %理論による実験事実の説明
    \newtheorem{anomaly}[theorem]{理論の限界}
    \newtheorem{application}[theorem]{応用例}
    \newtheorem{method}[theorem]{手法} %実験手法など,技術的問題.
    \newtheorem{test}[theorem]{検定}
    \newtheorem{terms}[theorem]{用語}
    \newtheorem{solution}[theorem]{解法}
    \newtheorem{history}[theorem]{歴史}
    \newtheorem{usage}[theorem]{用語法}
    \newtheorem{research}[theorem]{研究}
    \newtheorem{shishin}[theorem]{指針}
    \newtheorem{yodan}[theorem]{余談}
    \newtheorem{construction}[theorem]{構成}
    \newtheorem{motivation}[theorem]{動機}
    \newtheorem{context}[theorem]{背景}
    \newtheorem{advantage}[theorem]{利点}
    \newtheorem*{definition*}{定義}
    \newtheorem*{remark*}{注意}
    \newtheorem*{question*}{問}
    \newtheorem*{problem*}{問題}
    \newtheorem*{axiom*}{公理}
    \newtheorem*{example*}{例}
    \newtheorem*{corollary*}{系}
    \newtheorem*{shishin*}{指針}
    \newtheorem*{yodan*}{余談}
    \newtheorem*{kadai*}{課題}

\raggedbottom
\allowdisplaybreaks
\usepackage[math]{anttor}
\begin{document}
\tableofcontents

\begin{tcolorbox}[colframe=ForestGreen, colback=ForestGreen!10!white,breakable,colbacktitle=ForestGreen!40!white,coltitle=black,fonttitle=\bfseries\sffamily,
title=Brown運動を駆動過程とした確率解析を導入し,適宜一般化する.]
    \begin{description}
        \item[積分の方が基本概念である] 確率論ははじめから偏微分方程式論と密接に関係していたことを思うと,確率論自体が測度論で基礎付けられることは自然であった.
        さらに,確率積分なる概念も自然に定義できるはずである.
        \item[確率積分の定義] Brown運動$B_t$は微分可能ではないし,有界変動にもならないので,Stieltjes積分としては定義できる道はない.
        \begin{enumerate}
            \item 伊藤清はKolmogov (1931,\cite{Kolmogorov-31})から発想した.一般の連続Markov過程を,連続関数$a,b$を用いて
            \[E[X_{t+\Delta}-X|X_t=x]=a(t,x)\Delta+o(\Delta),\quad \Var[X_{t+\Delta}-X_t|X_t=x]=b(t,x)\Delta+o(\Delta)\]
            で定める,という出発点が前文に書いてあった.これはBrown運動の場合から次の変換
            \[X_{t+\Delta}-X=a(t,X_t)\Delta+\sqrt{b(t,X_t)}(B_{t+\Delta}-B_t)+o(\Delta)\]
            を経て得られる,ということでもあるが,この式は意味を持たない.基礎付けることができれば,Brown運動の見本路に関する知識を,Markov過程の見本路を調べることに応用できそうである.
            連鎖律は伊藤の公式と呼び,微分は出来ないから積分の言葉で定式化される.余分な右辺第3項が特徴である.
            \item この手法は,一般の$L^2$-有界なマルチンゲール$B\in\M^2$に関して使える.
            これはDoobが初めに指摘し,渡辺信三,国田寛 (67,\cite{Kunita-Watanabe})が理論を作り,変数変換公式を得た.
            さらにMeyer (67,\cite{Meyer-Seminaire})が精緻な理論を組み立てる.
            \item 余分な右辺第3項を修正するのがStratonovich積分/対称確率積分であるが,$X$に対してより強い条件を必要とすることとなる.
        \end{enumerate}
        \item[確率微分方程式へ] 確率微分方程式を解くには,例えば同値な確率積分方程式をPicardの逐次近似法によって解くことが考えられる.
    \end{description}
\end{tcolorbox}

\chapter{Brown運動}

\begin{quotation}
    Brown (1828)が自然界の現象として発見し,随分遅れて数理モデルとなった.
    \begin{enumerate}
        \item 微粒子が液体分子と衝突することによって起こる運動だと予想されてから,これが数理模型として妥当なものであることはEinstein (05)が確認した.
        \item さらにEinsteinの模型を用いて,J. Perrin (26)はAvogadro数のかなり正確な測定に成功している.
        \item Levyは1910sには$L^2([0,1])$上の汎関数の解析に取り組んでいた.$L^2([0,1])$は可分であるから,有限部分空間からの近似が可能で,そこでのLebesgue測度に関する平均の系列の極限として,$L^2([0,1])$上の汎関数の平均を考えた.これはホワイトノイズによる積分に一致する.Levy (51)では$L^2([0,1])$上のLaplace作用素や調和汎関数など,無限次元調和解析の研究に進んでいる.
        \item Wiener (23)はこの研究に触発されて,Wiener空間を定義し,この無限次元測度空間を駆使してCybernetics (48)として理論体系を打ち立てた.その重要な手法の一つが調和解析であった.
        \item Levy (37)で独立確率変数列の和を,「無限小の確率変数の連続和」という方向への一般化を考える文脈でBrown運動に到達する(Levyの構成).ここからの研究は余人の追随を許さない速度で進み,1948年に最初の集大成となる著書を発表する.
    \end{enumerate}
    Levyが関数解析の文脈から自然にBrown運動の研究に至ったことは感嘆させる事実である.
    Wiener (23)はGateauxやLevyの関数解析のしごとに大きな刺激を受け,最終的に論文を書かせた動機はLevyとの無限次元空間上の積分に関する会話だったと言っている.
\end{quotation}

\section{Brown運動の定義と特徴付け}

\begin{tcolorbox}[colframe=ForestGreen, colback=ForestGreen!10!white,breakable,colbacktitle=ForestGreen!40!white,coltitle=black,fonttitle=\bfseries\sffamily,
title=]
    Brown運動は,
    \begin{enumerate}
        \item 平均$0$で分散が$\Gamma=\min$であるようなGauss過程であって,殆ど確実に連続であるような過程である.
        \item 多次元標準正規分布が有限周辺分布を与えるようなLevy過程である($C$-過程でもあるLevy過程はGauss過程であることに注意).
        \item 熱核が遷移確率密度を与えるようなMarkov過程である.
    \end{enumerate}
\end{tcolorbox}

\begin{definition}[(standard) Brownian motion / Wiener process]
    $(\Om,\F,P)$上の実確率過程$(B_t)_{t\in\R_+}$が\textbf{標準Brown運動}であるとは,次の4条件を満たすことをいう:
    \begin{enumerate}[({B}1)]
        \item 中心化:$B_0=0\;\as$
        \item 加法性:$\forall_{n=2,3,\cdots}\;\forall_{0\le t_1<\cdots<t_n}\;B_{t_n}-B_{t_{n-1}},\cdots,B_{t_2}-B_{t_1}$は独立.
        \item 周辺分布の正規性:増分について,$B_t-B_s\sim N(0,t-s)\;(0\le s<t)$.
        \item 広義$C$-過程:$\Om\to\Meas(\R_+,\R);\om\mapsto(t\mapsto B_t(\om))$について,殆ど確実に$t\mapsto B_t(\om)$は連続.
    \end{enumerate}
    $d$個の独立なBrown運動$B^1,\cdots,B^d$の積$B=(B^1_t,\cdots,B^d_t)_{t\in\R_+}$を\textbf{$d$次元Brown運動}という.
\end{definition}
\begin{example}[Brown運動ではない例]
    Brown運動$(B_t)_{t\in\R_+}$に対して,独立でランダムな時刻$U:\Om\to[0,1]$を考える.$U$は$[0,1]$上の一様分布に従うとする.
    \[\wt{B}_t:=\begin{cases}
        B_t,&t\ne U,\\
        0,&t=U.
    \end{cases}\]
    とすると,$P[s\ne U\land t\ne U]=0$に注意して,
    \[E[\wt{B}_t\wt{B}_s]=E[B_tB_s|s\ne U\land t\ne U]+0=E[B_tB_s1_{\Brace{s\ne U,t\ne U}}]P[s\ne U\land t\ne U]=s\land t.\]
    だから,これは平均$0$で共分散$s\land t$のGauss過程である.
    特に,Brown運動のバージョンである.
    一方で,$U=0$である場合を除いて見本道は連続でない,すなわち殆ど確実に見本道は連続でない.
\end{example}

\subsection{多次元正規分布の性質}

\begin{lemma}[多次元正規分布の特徴付け]
    $X_1,\cdots,X_n\in L(\Om)$について,
    次の2条件は同値:
    \begin{enumerate}
        \item $X_1,\cdots,X_n$の任意の線形結合は正規分布に従う.
        \item $(X_1,\cdots,X_n)$は多次元正規分布に従う.
    \end{enumerate}
\end{lemma}

\begin{lemma}[多次元正規確率変数の成分間の独立性の特徴付け]
    $Y_1:=(X_1,\cdots,X_{k_1})^\top,\cdots,Y_l=(X_{k_{l-1}},\cdots,X_d)^\top\;(l\ge 2)$のように,$X$を$l$個の確率変数$Y_1,\cdots,Y_l$に分ける.
    この分割に対して,$\Sigma$のブロック$\Sigma_{a,b}:=\Cov[Y_a,Y_b]\;(a,b\in[l])$を考える.
    \begin{enumerate}
        \item $Y_1,\cdots,Y_l$は独立.
        \item $\forall_{a,b\in[l]}\;a\ne b\Rightarrow\Sigma_{a,b}=O$.
    \end{enumerate}
\end{lemma}

\subsection{Brown運動のGauss過程としての特徴付け}

\begin{proposition}[Brown運動の特徴付け]
    実確率過程$B=(B_t)_{t\in\R_+}$について,次の2条件は同値.
    \begin{enumerate}
        \item $B$は(B1),(B2),(B3)を満たす.
        \item $B$は平均$0$共分散$\Gamma(s,t):=\min(s,t)$のGauss過程である.
    \end{enumerate}
\end{proposition}
\begin{Proof}\mbox{}
    \begin{description}
        \item[(1)$\Rightarrow$(2)] Gauss過程であることを示すには,任意の$0<t_1<\cdots<t_n\in\R_+$を取り,$B:=(B_{t_1},\cdots,B_{t_n}):\Om\to\R^n$が$n$次元の正規分布に従うことを示せば良い.
        まず,仮定(B1),(B2)より,$B_{t_1},B_{t_2}-B_{t_1},\cdots,B_{t_n}-B_{t_{n-1}}$は独立であり,それぞれが正規分布に従う.
        したがって,積写像$A:=(B_{t_1},B_{t_2}-B_{t_1},\cdots,B_{t_n}-B_{t_{n-1}}):\Om\to\R^n$は$n$次元正規分布に従う(補題(2)).
        行列
        \[J:=\begin{bmatrix}1&0&\cdots&\cdots&0\\1&1&0&\cdots&0\\\vdots&\ddots&\ddots&\ddots&\vdots\\1&\cdots&\cdots&1&0\\1&\cdots&\cdots&1&1\end{bmatrix}\]
        が定める線形変換を$f:\R^n\to\R^n$とおくと,これは明らかに可逆で,$B=f(A)$が成り立つ.
        ここで,任意の線型汎関数$q\in(\R^n)^*$に関して,$q(B)=q(f(A))$は,$q\circ A:\R^n\to\R$が線型であることから,正規分布に従う.
        よって,$B$も正規分布に従う.

        また,仮定(B3)より平均は$m(t)=E[B_t]=0$で,共分散は,$s\ge t$のとき,増分の独立性(B2)に注意して
        \[\Gamma(s,t)=\Cov[B_s,B_t]=E[B_sB_t]=E[B_s(B_t-B_s+B_s)]=E[B_s(B_t-B_s)]+E[B_s^2]=s.\]
        $\Gamma$の対称性より,これは$\Gamma(s,t)=\min(s,t)$を意味する.
        \item[(2)$\Rightarrow$(1)]
        \begin{enumerate}[({B}1)]
            \item 平均と分散を考えると,$m(0)=E[B_0]=0$かつ$\Gamma(0,0)=E[B_0^2]=0$.よって,$E[\abs{B_0}]=0$より,$B_0=0\;\as$
            \item Gauss過程であることより,組$A:=(B_{t_n}-B_{t_{n-1}},\cdots,B_{t_2}-B_{t_1})$は$n$次元正規分布に従う.これらが独立であることを示すには,補題より$A$の分散共分散行列$\Sigma_A$の非対角成分がすべて$0$であることを示せば良い.
            平均が$0$で$\Gamma(s,t)=\min(s,t)$であることより,
            任意の$1<i< j\in[n]$について,共分散の双線型性に注意して,
            \begin{align*}
                \Cov[B_{t_i}-B_{t_{i-1}},B_{t_j}-B_{t_{j-1}}]&=E[(B_{t_i}-B_{t_{i-1}})(B_{t_j}-B_{t_{j-1}})]\\
                &=E[B_{t_i}B_{t_j}]-E[B_{t_{i-1}}B_j]-E[B_{t_i}B_{t_{j-1}}]+E[B_{t_{i-1}}B_{t_{j-1}}]=i-(i-1)-i+(i-1)=0.
            \end{align*}
            \item 平均が$0$で$\Gamma(s,t)=\min(s,t)$であることより,
            \[E[(B_t-B_s)^2]=E[B_t^2]+E[B_s^2]-2E[B_tB_s]=t+s-2s=t-s.\]
        \end{enumerate}
    \end{description}
\end{Proof}

\subsection{条件付き特性関数による特徴付け}

\begin{proposition}
    $X\in\bF\cap C$は$X_0=0$を満たすとする.このとき,次は同値:
    \begin{enumerate}
        \item $X$は$d$-次元Brown運動である.
        \item $\forall_{0\le s\le t}\;\forall_{\xi\in\R^d}\;E[e^{i(\xi|X_t-X_s)}|\F_s]=e^{-\frac{1}{2}\abs{\xi}^2(t-s)}$.
    \end{enumerate}
\end{proposition}

\section{Brown運動の構成}

\begin{tcolorbox}[colframe=ForestGreen, colback=ForestGreen!10!white,breakable,colbacktitle=ForestGreen!40!white,coltitle=black,fonttitle=\bfseries\sffamily,
    title=]
    構成のアイデアは複数ある.
    \begin{enumerate}
        \item Gauss過程としての構成.
        \item Fourier級数としての構成.
        \item 対称な酔歩の分布極限としての構成.
    \end{enumerate}
\end{tcolorbox}

\begin{shishin}
    $T>0$として,区間$[0,T]$上の過程$B=(B_t)_{t\in[0,T]}$を条件(B1)から(B4)を満たすように構成すれば,
    過程の列$(Y^{(i)})_{i\in\N}$を「つなげた」過程を
    \[W_t:=\paren{\sum^{\floor{t/T}}_{i=1}Y_T^{(i)}}+Y^{[t/T]+1}_{t-[t/T]T}\]
    と定めると,これが目標の$\R_+$上のBrown運動となる.
    よって以降の構成の議論では,有界区間$[0,T]$上に構成することを目指す.
\end{shishin}

\subsection{Gauss過程の連続修正としての構成}

\begin{theorem}[Kolmogorov's continuity criterion / 正規化定理 / 連続変形定理]\label{thm-Kolmogorov-continuity-criterion}
    実過程$X=(X_t)_{t\in[0,T]}$が
    \[\exists_{\al,\beta,K>0}\;\forall_{s,t\in[0,T]}\;E[\abs{X_{t}-X_s}^\beta]\le K\abs{t-s}^{1+\al}\]
    を満たすならば,$X$のある連続な修正$\wt{X}$が識別不可能な違いを除いて一意的に存在して,
    任意の$0\le\gamma<\frac{\al}{\beta}$について次を満たす:
    \[\exists_{G_\gamma\in\L(\Om';\R)}\;\forall_{s,t\in[0,T]}\qquad\abs{\wt{X}_t-\wt{X}_s}\le G_\gamma\abs{t-s}^\gamma.\]
    特に,$\wt{X}$の見本道は$\gamma$-次Holder連続である.
\end{theorem}
\begin{Proof}
    \cite{Revuz-Yor}I.2 Thm (2.1)は,一般の添字集合$t\in[0,1)^d$上のBanach空間値過程$(X_t)$について示している.
\end{Proof}

\begin{construction}\label{construction-Kolmogorov}
    $\Gamma(s,t)=\min(s,t)$は明らかに対称である.
    さらに半正定値であることを示せば,これを共分散とする平均$0$のGauss過程が存在する.
    まず,$\min(s,t)=\int_{\R_+}1_{[0,s]}(r)1_{[s,t]}(r)dr$であることに注意すると,
    任意の$n\in\N_{\ge1}$と$a_1,\cdots,a_n\in\R$について,
    \begin{align*}
        \sum_{i,j\in[n]}a_ia_j\min(t_i,t_j)&=\sum_{i,j\in[n]}\int^\infty_01_{[0,t_i]}(r)1_{[0,t_j]}(r)dr\\
        &=\int^\infty_0\paren{\sum_{i=1}^na_i1_{[0,t_i]}(r)}^2dr\ge0.
    \end{align*}
    最後に,この過程が条件(B4)を満たすことを見れば良い.
    これはKolmogorovの正規化定理\ref{thm-Kolmogorov-continuity-criterion}による.
    任意の$0\le s\le  t$について,$B_t-B_s\sim N(0,t-s)$であり,正規分布の奇数次の中心積率は$0$で偶数次の中心積率は
    \[\forall_{k\in\N}\quad E[\abs{B_t-B_s}^{2k}]=\frac{(2k)!}{2^kk!}(t-s)^k\]
    と表せるから,ある$B$の修正が存在して,任意の閉区間$[0,T]$上で$\gamma<\frac{k-1}{2k}$について$\gamma$-次Holder連続である.
    特に,任意の$\gamma<\frac{1}{2}$について,$B$の見本道$t\mapsto B_t(\om)$は殆ど至る所$\gamma$-Holder連続である.
\end{construction}

\subsection{Fourier級数の極限としての構成}

\begin{tcolorbox}[colframe=ForestGreen, colback=ForestGreen!10!white,breakable,colbacktitle=ForestGreen!40!white,coltitle=black,fonttitle=\bfseries\sffamily,
title=]
    Wienerは調和解析の手法を重用した.
    Wienerが1923年に初めてBrown運動を構成したのはこの方法により,複素Brown運動を構成した.
\end{tcolorbox}

\subsubsection{Ito-Nishioの理論}

\begin{notation}
    $T>0$とし,$L^2([0,T])$の正規直交系$(e_n)_{n\ge0}$を取る.
    $\{Z_n\}_{n\ge0}\subset L^2(\Om)$を$N(0,1)$に従う独立同分布確率変数列とする.
\end{notation}

\begin{proposition}[Ito-Nishio]
    $\{\xi_i\}\subset L^2(\Om)$を$N(0,1)$に従う独立同分布列,$\{e_i\}$を可分Hilbert空間$L^2([0,1])$の正規直交基底とする.
    \[X_n(t,\om):=\sum_{i=1}^n\xi_i(\om)\int^t_0e_i(u)du\]
    について,次が成り立つ.
    \begin{enumerate}
        \item $\forall_{t\in[0,1]}\;\lim_{n\to\infty}X_n(t)$は$L^2$と概収束の意味で収束する.
        \item 極限過程$X$は条件(B2),(B3)を満たす.
        \item $X_n$は一様収束の位相についても,殆ど至る所収束する.とくに,$X$はBrown運動である.
    \end{enumerate}
\end{proposition}
\begin{proof}\mbox{}
    \begin{enumerate}
        \item \begin{description}
            \item[$L^2(\Om)$-収束] 任意に$t\in[0,1]$を取る.
            このとき,$E[\xi_i\xi_j]=\delta_{ij}$に注意すると,
            \begin{align*}
                \Norm{\int^M_{i=N}\xi_i(\om)\int^t_0e_i(u)du}_{L^2(\Om)}&=E\Square{\paren{\sum_{i=N}^M\xi_i\int^t_0e_i(u)du}^2}\\
                &=E\Square{\sum_{i=N}^M\xi_i^2\paren{\int^t_0e_i(u)du}^2}\\
                &=\sum_{i=N}^M\paren{\int^t_0e_i(u)du}^2\\
                &=\sum_{i=N}^M(1_{[0,t]}|e_n)^2_{L^2([0,1])}\\
                &=\sum^M_{i=0}(1_{[0,t]}|e_n)^2_{L^2([0,1])}-\sum^N_{i=0}(1_{[0,t]}|e_n)^2_{L^2([0,1])}\xrightarrow{N,M\to\infty}0.
            \end{align*}
            となるから,$(X_n)$は$L^2(\Om)$上のCauchy列である.$L^2(\Om)$の完備性より,これは収束する.
            \item[概収束] 各$Z_i:=\xi_i\int^t_0e_i(u)du$は平均$0$の,独立確率変数で,
            \[\sum_{i=1}^\infty E\Square{\xi^2_i\paren{\int^t_0e_i(u)du}^2}=\sum_{i=1}^\infty\paren{\int^t_0e_i(u)du}^2=1\]
            より,概収束する.
        \end{description}
        \item 補題より,$X$の平均と共分散を確かめれば良い.
        \begin{description}
            \item[平均] $E:L^2(\Om)\to\R$の$L^2(\Om)$-連続性より,
            \begin{align*}
                E[X]=E\Square{\lim_{n\to\infty}\sum^n_{i=0}\xi_i\int^t_0e_i(u)du}&=\lim_{n\to\infty}E\Square{\sum^n_{i=0}\xi_i\int^t_0e_i(u)du}=0.
            \end{align*}
            \item[分散] 任意の$s,t\in\R_+$と$n\ge1$について,$E[\xi_i\xi_j]=\delta_{ij}$とParsevalの等式より,
            \begin{align*}
                \Cov[X_s,X_t]&=E\Square{\paren{\sum_{i=1}^n\xi_i\int^s_0e_i(u)du}\paren{\sum_{i=1}^n\xi_i\int^t_0e_i(u)du}}\\
                &=\sum_{i=1}^n\paren{\int^s_0e_i(u)du}\paren{\int^t_0e_i(u)du}\\
                &=\sum_{i=1}^n(1_{[0,s]}|e_i)_{L^2([0,1])}(1_{[0,t]}|e_i)_{L^2([0,1])}\xrightarrow{n\to\infty}(1_{[0,s]}|1_{[0,t]})_{L^2([0,1])}=\min(s,t).
            \end{align*}
        \end{description}
        \item (Ito and Nishio, 1968 \cite{Ito-Nishio})より,殆ど至る所一様収束もする.よって,$X$は連続過程でもあり,従ってBrown運動である.
    \end{enumerate}
\end{proof}

\subsubsection{Wienerの構成}

\begin{tcolorbox}[colframe=ForestGreen, colback=ForestGreen!10!white,breakable,colbacktitle=ForestGreen!40!white,coltitle=black,fonttitle=\bfseries\sffamily,
title=]
    Wienerが1923年に初めてBrown運動を構成したのは,基底として三角級数を取った場合である\cite{Bass}(Thm 6.1).
\end{tcolorbox}

\begin{notation}
    $X_k,Y_k\sim N(0,1)\;(k\in\Z)$を独立列とし,
    \[Z_k:=\frac{1}{\sqrt{2}}(X_k+iY_k)\]
    を考える.これは2次元標準正規分布とみなせ,$\{Z_k\}$は中心化された独立Gauss系である.
    すなわち,$L^2_0(\Om;\C)$の正規直交系とみなせる.
    しかし,
    \[\sum_{k\in\Z}Z_k(\om)e^{ikt}\]
    は$\om\in\Om$毎に見ると収束しない.
    \[Z_1(t,\om):=tZ_0(\om)+\sum_{n\in\N^+}\frac{Z_n(\om)(e^{int}-1)}{in}+\sum_{n\in\N^+}\frac{Z_{-n}(\om)(e^{-int}-1)}{-in}\]
    に注目する.
\end{notation}

\begin{proposition}[$Z_1$はBrown運動の複素形とみなせる]
    $Z(t,\om):=\frac{1}{\sqrt{2\pi}}Z_1(t,\om)$は,中心化された確率変数族で,$E[Z(t)\o{Z(s)}]=t\land s$を満たす.
\end{proposition}

\begin{proposition}[収束問題の解決]
    \[Z_{m,n}(t,\om):=\sum_{k=m+1}^nZ_k(\om)\frac{e^{ikt}}{ik}\quad(m<n\in\Z)\]
    について,級数$\sum_{n\in\N^+}\abs{Z_{2^n,2^{n+1}}(t,\om)}$は殆ど全ての$\om\in\Om$について$t\in[0,1]$について一様収束する.
    特に,次のように定義した$Z$は広義の$C$-過程である:
    \[Z(t,\om):=\frac{t}{\sqrt{2\pi}}Z_0(\om)+\sum_{n\in\N}\paren{\sum_{k=2^n+1}^{2^{n+1}}\paren{\frac{Z_k(\om)(e^{ikt}-1)}{i\sqrt{2\pi}k}+\frac{Z_{-k}(\om)(e^{-ikt}-1)}{-i\sqrt{2\pi}k}}}.\]
\end{proposition}

\begin{proposition}
    \[\frac{1}{\sqrt{2\pi}}Z_1(t,\om)=Z(t,\om)=\frac{1}{\sqrt{2}}(X(t,\om)+Y(t,\om))\]
    について,$X,Y$は互いに独立なBrown運動である.
\end{proposition}

\begin{remarks}
    $t\in[0,\pi]$を固定した関数$f(s)=s\land t\in\cS^*([0,\pi])$のFourier級数を計算することで,
    \[s\land t=\frac{st}{\pi}+\frac{2}{\pi}\sum^\infty_{k=1}\frac{\sin ks\sin kt}{k^2}\]
    を得る.すると,$Z_n\sim N(0,1)$について,
    \[W_t:=\frac{t}{\sqrt{\pi}}Z_0+\sqrt{\frac{2}{\pi}}\sum^\infty_{k=1}Z_k\frac{\sin kt}{k}\]
    と定めれば,$(W_t)_{t\in[0,\pi]}$は平均$0$で共分散$E[W_sW_t]=s\land t$のGauss過程であることが期待できる.
    あとは,上記のFourier級数の収束の問題が残るのみである.
    これは以下の議論で$e_n(r)=\frac{1}{\sqrt{\pi}}\cos nt$と取った場合であり,このときのBrown運動$W_t$の表示を\textbf{Paley-Wiener表現}という.
\end{remarks}



\subsection{線形近似列の極限としての構成}

\begin{tcolorbox}[colframe=ForestGreen, colback=ForestGreen!10!white,breakable,colbacktitle=ForestGreen!40!white,coltitle=black,fonttitle=\bfseries\sffamily,
title=]
    次の命題は,$B_t$の値が$(a,b)$の外で知られたとき,内部$t\in(a,b)$では線形補間$\mu(t)$を取ってから,これに独立な量$\sigma(t)X_t$だけ振動させることで,Brown運動を再現できることを示唆する.
    これはHaar基底を取ったときの,Ito-Nishioの方法の例でもある.
\end{tcolorbox}

\begin{theorem}
    $(a,b)\subset\R_+$を有界区間,$t\in(a,b)$とする.このとき,$X_t$を$\{B_s\}_{s\in\R_+\setminus(a,b)}$と独立な標準正規確率変数とし,
    \[B_t=\mu(t)+\sigma(t)X_t,\quad\mu(t):=\frac{1}{b-a}((b-t)B_a+(t-a)B_b),\sigma^2(t):=\frac{1}{b-a}(t-a)(b-t)\]
    と表せる.
    特にこのとき,$\mu(t)=E[B_t|B_a,B_b]$が成り立つ.
\end{theorem}

\subsubsection{Levyの構成}

\begin{tcolorbox}[colframe=ForestGreen, colback=ForestGreen!10!white,breakable,colbacktitle=ForestGreen!40!white,coltitle=black,fonttitle=\bfseries\sffamily,
title=]
    LevyとCiesielskiは基底としてウェーブレットの一例であるHaar関数系を取った.
    まずは,明示的にHaar基底を使わずに構成し,線形補間としての意味を重視する.
    極めて具体的で,「こうすればBrown運動を構成できる」というのがよく分かる.
    $D_n:=\Brace{\frac{k}{2^n}\in[0,1]\;\middle|\;0\le k\le 2^n}$上に点をうち,その線型補間を行い,$C([0,1])$上での極限を取る.
    この場合も,見本道の連続性は自然に従い,また$\R_+\times\Om\to\R$の積空間上での可測性もわかりやすい.
\end{tcolorbox}

\begin{lemma}\mbox{}
    \begin{enumerate}
        \item $X_1,X_2\sim N(0,\sigma^2)$を独立同分布確率変数とする.
        このとき,$X_1+X_2$と$X_1-X_2$も独立で,いずれも$N(0,2\sigma^2)$に従う.
        \item $X\sim N(0,1)$とする.このとき,
        \[\forall_{x>0}\quad\frac{x}{x^2+1}\frac{1}{\sqrt{2\pi}}e^{-\frac{x^2}{2}}\le P[X>x]\le\frac{1}{x}\frac{1}{\sqrt{2\pi}}e^{-\frac{x^2}{2}}.\]
        \item $(X_n)$を正規確率ベクトルの列で,$\lim_{n\to\infty}X_n=X\;\as$を満たすとする.$b:=\lim_{n\to\infty}E[X_n],C:=\lim_{n\to\infty}\Cov[X_n]$が存在するならば,$X$は$b,C$が定める正規確率変数である.
    \end{enumerate}
\end{lemma}
\begin{Proof}\mbox{}
    \begin{enumerate}
        \item $\frac{1}{\sqrt{2}\sigma}(X_1+X_2,X_1-X_2)^\top$は,標準正規分布$\frac{1}{\sigma}(X_1,X_2)^\top$の直交行列$\frac{1}{\sqrt{2}}(1,1;1,-1)$による変換であるから,再び標準正規分布である.
        \item 右辺の不等式については,
        \begin{align*}
            P[X>x]&=\frac{1}{\sqrt{2\pi}}\int^\infty_xe^{-\frac{u^2}{2}}du\\
            &\le\frac{1}{\sqrt{2\pi}}\frac{u}{x}\int^\infty_xe^{-\frac{u^2}{2}}du
            =\frac{1}{\sqrt{2\pi}}\frac{1}{x}e^{-\frac{x^2}{2}}.
        \end{align*}
    \end{enumerate}
\end{Proof}

\begin{construction}
    $[0,1]$内の小数第$n$位までの2進有理数の集合を
    \[\D_n:=\Brace{\frac{k}{2^n}\in[0,1]\;\middle|\;0\le k\le 2^n}\]
    で表し,$\D:=\cup_{n\in\N}\D_n$とする.
    各$\D_n$上に点を打ち,その線型補完として$C([0,1])$の列を得て,その一様収束極限が(存在して)Gauss過程であることを導けば良い.
    \begin{description}
        \item[列の構成] $N(0,1)$に従う独立同分布確率変数列が,ある確率空間$(\Om,\A,P)$上に存在する:$\{Z_n\}\subset L^2(\Om)$.これを用いて,
        $\D_n$上の確率過程で,
        \begin{enumerate}
            \item $B_0=0,B_1=Z_1$,
            \item $\forall_{r<s<t\in\D_n}\;B_s-B_r\perp B_t-B_s\sim N(0,t-s)$,
            \item $(B_d)_{d\in\D_n}\perp(Z_t)_{t\in\D\setminus\D_n}$,
        \end{enumerate}
        の3条件を満たすものが任意の$n\in\N$について存在することを示す.

        $n=0$のとき,(1)で定まる$B_0,B_1$は,$B_1-B_0=Z_1\sim N(0,1)$かつ$(0,Z_1)$は$(Z_t)_{t\in\D\setminus\{0,1\}}$と独立である.
        $n>0$のとき,$d\in\D_n\setminus\D_{n-1}$について
        \[B_d:=\frac{B_{d-1/2^n}+B_{d+1/2^n}}{2}+\frac{Z_d}{2^{(n+1)/2}}\]
        とおけば良い.実際,このとき$\D_{n-1}\ni B_{d-1/2^n},B_{d+1/2^n}\perp (Z_t)_{t\in\D\setminus\D_{n-1}}$より,帰納法の仮定から$B_d\perp(Z_t)_{t\in\D\setminus\D_n}\subset(Z_t)_{t\in\D\setminus\D_{n-1}}$が成り立つ.
        よって後は条件(2)の成立を示せば良い.
        帰納法の仮定から$\frac{B_{d-1/2^n}-B_{d+1/2^n}}{2},\frac{Z_d}{2^{(n+1)/2}}$は独立に$N(0,1/2^{n+1})$に従う.
        よって,和と差$B_d-B_{d+1/2^n},B_d-B_{d-1/2^n}$も独立に$N(0,1/2^n)$に従う.このとき,$(B_d-B_{d-1/2^n})_{d\in\D_n\setminus\{0\}}$が独立であることを示せば十分である.
        これらは多次元正規分布を定めるから,対独立性を示せば十分である.
        
        これら新たな点が定める新たな$1/2^n$増分は,$\D_{n-1}$の点の凸結合としたから,$\D_{n-1}$の言葉で表せて,独立性は帰納法の仮定から従う.
        \item[一様収束極限の存在]
        $(B_d)_{d\in\D_n\setminus\D_{n-1}}$の線型補間を$F_n:[0,1]\to\R$と表す($D_{-1}=\emptyset$とする)と,
        \[\forall_{n\in\N}\;\forall_{d\in\D_n}\quad B_d=\sum^n_{i=0}F_i(d)=\sum^\infty_{i=0}F_i(d)\]
        が成り立つ.このとき,$\sum^\infty_{i=0}F_i(t)$は$C([0,1])$の一様ノルムについて収束すること,従って$\norm{F_n}_\infty$が$0$に収束することを示せば良い.

        任意の$c>1$と($\frac{1}{c\sqrt{n}}\le\sqrt{\frac{\pi}{2}}$を満たすくらい)十分大きな$n$について,
        \[P[\abs{Z_d}\ge c\sqrt{n}]\le2\frac{1}{c\sqrt{n}}\frac{1}{\sqrt{2\pi}}e^{-\frac{c^2n}{2}}\le e^{-\frac{c^2n}{2}}.\]
        よって,ある$N\in\N$について,
        \begin{align*}
            \sum^\infty_{n=N}P[\exists_{d\in\D_n}\;\abs{Z_d}\ge c\sqrt{n}]&<\sum^\infty_{n=N}\sum_{d\in\D_n}P[\abs{Z_d}\ge c\sqrt{n}]\\
            &\le\sum^\infty_{n=N}(2^n+1)\exp\paren{-\frac{c^2n}{2}}
        \end{align*}
        が成り立つから,$c>\sqrt{2\log 2}$を満たす$c$を取れば,級数$\sum^\infty_{n=0}P[\exists_{d\in\D_n}\;\abs{Z_d}\ge c\sqrt{n}]<\infty$が成り立つ.
        これを満たす$c$を一つ任意に取ると,Borel-Cantelliの補題から,
        \[P[\limsup_{n\to\infty}\Brace{\exists_{d\in\D_n}\;\abs{Z_d}\ge c\sqrt{n}}]\le\lim_{N\to\infty}\sum_{n=N}^\infty P[\exists_{d\in\D_n}\;\abs{Z_d}\ge c\sqrt{n}]=0.\]
        すなわち,$\exists_{N\in\N}\;\forall_{n\ge N}\;\forall_{d\in D_n}\;\abs{Z_d}<c\sqrt{n}$.
        これは,$\forall_{n\ge N}\;\norm{F_n}_\infty<c\sqrt{n}2^{-n/2}$を意味する.
        よって,$B_t:=\sum^\infty_{i=0}F_i(t)$は確率$1$で一様収束する.
        \item[極限過程はBrown運動である]
        任意に$0\le t<s\in[0,1]$を取ると,$\D$の列$(t_n),(s_n)$が存在して$t,s$にそれぞれ収束する.
        極限過程$B$の連続性より,$B_s=\lim_{n\to\infty}B_{s_n}$が成り立つ.
        よって,$B_{t_n}\sim N(0,t_n)$より,
        $E[B_t]=\lim_{n\to\infty}E[B_{t_n}]=0$.
        また,
        \[\Cov[B_s,B_t]=\lim_{n\to\infty}\Cov[B_{s_n},B_{t_n}]=\lim_{n\to\infty}s_n=s.\]
        以上より,$(B_t)_{t\in I}$はBrown運動である.
    \end{description}
\end{construction}

\begin{proposition}[jointly measurable]
    このように構成された$(B_t)_{t\in\R_+}$は,$\Om\times\R_+$上可測である.
    ただし,$(\Om,\A,P)$とは,独立同分布確率変数列$(Z_n)$の定義域となる確率空間とする.
\end{proposition}
\begin{Proof}
    $B_t=\sum^\infty_{i=0}F_i(t)$と定めたから,$\forall_{n\in\N}\;F_n(t)$は$\Om\times[0,1]$上可測であることを示せば良い.
    $\R_+$上のBrown運動は,これら$(B_t)_{t\in[0,1]}$の和として表せるため.
    任意の$c\in\R$について,
    \[\Brace{(\om,t)\in\Om\times[0,1]\mid F_n(t)>c}\]
    が可測であることを示せばよいが,この集合は,
    \begin{enumerate}
        \item $c\ge0$のとき$\bigcup_{k=1}^{2^{n-1}}\Brace{Z_{\frac{2k-1}{2^n}}\ge c}\times\paren{\frac{2k-1}{2^n}-\frac{Z_{\frac{2k-1}{2^n}}-c}{Z_{\frac{2k-1}{2^n}}}\frac{1}{2^n},\frac{2k-1}{2^n}+\frac{Z_{\frac{2k-1}{2^n}}-c}{Z_{\frac{2k-1}{2^n}}}\frac{1}{2^n}}$.
        \item $c<0$のとき\[\bigcup_{k=1}^{2^{n-1}}\Brace{Z_{\frac{2k-1}{2^n}}>c}\times\Square{\frac{2k-2}{2^n},\frac{2k}{2^n}}\cup\Brace{Z_{\frac{2k-1}{2^n}}\le c}\times\paren{\left[\frac{2k-2}{2^n},\frac{2k-1}{2^n}-\frac{Z_{\frac{2k-1}{2^n}}-c}{Z_{\frac{2k-1}{2^n}}}\frac{1}{2}\right)\cup\left(\frac{2k-1}{2^n}+\frac{Z_{\frac{2k-1}{2^n}}-c}{Z_{\frac{2k-1}{2^n}}}\frac{1}{2^n},\frac{2k}{2^n}\right]}.\]
    \end{enumerate}
    と等しいから,たしかに$\Om\times[0,1]$の積$\sigma$-加法族の元である.
\end{Proof}

\subsection{酔歩の極限としての構成}

\begin{tcolorbox}[colframe=ForestGreen, colback=ForestGreen!10!white,breakable,colbacktitle=ForestGreen!40!white,coltitle=black,fonttitle=\bfseries\sffamily,
title=]
    酔歩のスケールとして得られるこの方法は,Brown運動が基本的な対象であることの直感的な説明となる(これを「不変性」という表現で捉えている).
    また,$C(\R_+)$における極限であるから,連続性は自然に従う.
    酔歩と同様にBrown運動は1,2次元においては再帰的であるが,3次元以上では過渡的である.
    もはや酔歩とは異なる点に,スケール不変性が挙げられる.

    さらにここでDonskerの定理が登場する.
    中心極限定理を連続化したものがDonskerの定理(functional central limit theorem)であるが,i.i.d.の誤差の分布は,$-\infty,\infty$では$y=0$になるBrown運動を固定端で走らせれば良い.
\end{tcolorbox}

\begin{notation}[酔歩]
    任意の$T>0$に対して,これを$n\in\N$等分する考え方で,独立同分布に従う$n$個の確率変数$\xi_1,\cdots,\xi_n\sim(0,T/n)$,または$X_i:=\sqrt{n}\xi_i\sim(0,T)$を考える.
    この和$R_k:=\sum_{i=1}^k\xi_i=\frac{1}{\sqrt{n}}\sum^k_{i=1}X_i\;(k\in[n])$を考えると,過程$(R_k)_{k\in[n]}$は$n$ステップの酔歩で,分散が$T/n$である.
    $n$をscale parameterという.
    この過程について,$n\to\infty$の極限を取ることを考える.すると,ステップ数は増え,ステップの幅は小さくなる.
    このとき,中心極限定理より,列$(R_k)_{k\in\N}$は正規分布$N(0,T)$に分布収束する.
\end{notation}

\begin{discussion}[酔歩の線形補間]
    $R_n(t):=\frac{1}{\sqrt{n}}\sum_{i=1}^{\floor{tn/T}}X_i$はcadlagではあるが連続ではないため,より簡単な対象にしたい.そこで,
    $\forall_{k=0,\cdots,n}\;S_n\paren{\frac{kT}{n}}=R_k$を満たす$S_n:[0,T]\to\R$を,線型補間による連続延長とする.
    すると過程$(S_n)_{n\in\N}:\N\times\Om'\to C([0,T])$が定まる.
    2つの過程$(R_n),(S_n)$について,差は$\ep_n(T):=\max(\xi_1,\cdots,\xi_{\floor{nT}+1})$
    なる過程を超えない.
    %$\R^d$上の下側区間の定義関数全体からなる集合$\F$は任意の分布$P$に関してDonskerクラスになるというのがDonskerによる古典的な結果である.
\end{discussion}

\begin{proposition}
    $E[X_1^2]<\infty$ならば,任意の$T>0$に対して次が成り立つ:$\forall_{\delta>0}\;P\Square{\sup_{t\in[0,T]}\abs{R_n(t)-S_n(t)}>\delta}=P(\ep_n(T)>\delta)\to0\;(n\to\infty)$.
\end{proposition}

\begin{theorem}[Donsker's invariance principle/ functional central limit theorem]
    $(X_m)$を平均$0$分散$T$の独立同分布とする.
    これが定める線形補間$(S_n)_{n\in\N}:\N\times\Om'\to C([0,T])$は,Brown運動$(B_t:=\sqrt{T}W_t)_{t\in\R_+}:\Om\to C([0,T])$に分布収束する.
\end{theorem}
\begin{remark}[Donsker's theorem]
    Donskerの元々の論文に載っている定理は,この形であった.
    その後,セミパラの研究者が,外積分に基づいて,独自の理論を作った.
    なお,「不変性」とは,増分過程$(\xi_i)$または$(X_i)$の分布に依らないことを指す.
\end{remark}
\begin{remarks}
    この証明は,一般の酔歩(部分和過程)を,停止時の列におけるBrown運動の値として表現できるというSkorokhodの結果を用いて,Skorokhod埋め込み表現から証明する.
\end{remarks}

\subsection{その他の構成}

\begin{tcolorbox}[colframe=ForestGreen, colback=ForestGreen!10!white,breakable,colbacktitle=ForestGreen!40!white,coltitle=black,fonttitle=\bfseries\sffamily,
title=]
    \begin{enumerate}
        \item Markov性に注目した測度の構成:熱核を用いて実際に確率測度を$\R^D$上に構成する.$D\subset[0,1]$は二進有理数の全体とし,Kolmogorovの拡張定理を用いる極めて具体的な構成である\cite{舟木}.
        \item martingale性に注目した正則化:Gauss過程を得たあと,残るは条件(B4)の連続性であるが,これを示すのにKolmogorovの正規化定理に依らず,martingale性を利用する方法がある\cite{Bass}(Thm 6.2).
        \item $C(\R_+)$上の測度を直接構成して,$\id:C(\R_+)\to C(\R_+)$としてBrown運動を得ることも出来る\ref{prop-construction-of-Wiener-measure}.
        \item また,白色雑音の双対としても構成できる:\ref{sec-Wiener-integral}節へ.
    \end{enumerate}
\end{tcolorbox}

\section{Brown運動の変種}


\subsection{Brown運動の逆}

\begin{tcolorbox}[colframe=ForestGreen, colback=ForestGreen!10!white,breakable,colbacktitle=ForestGreen!40!white,coltitle=black,fonttitle=\bfseries\sffamily,
title=]
    確率過程$M_t:=\frac{1}{\abs{B_t}}$は$L^2(\Om)$-有界な(特に一様可積分な!)局所マルチンゲールであるが,マルチンゲールではない.
\end{tcolorbox}

\begin{lemma}
    $B=(B^1,B^2,B^3)$を3次元Brown運動で$B_0\ne0$を満たすもの,
    $\tau_n:=\inf\Brace{t\in\R_+\mid\abs{B_t}=1/n}$を停止時とする.
    \begin{enumerate}
        \item $\lim_{n\to\infty}\tau_n=\infty\;\as$
        \item $\abs{B_0}>1/n$ならば,次が成り立つ:
        \[\frac{1}{B_{t\land\tau_n}}=\frac{1}{\abs{B_0}}-\sum_{j=1}^3\int^{t\land\tau_n}_0\frac{B^j_s}{\abs{B_s}^3}dB^j_s.\]
        \item $(\abs{B_t}^{-1})_{t\in\R_+}$は局所マルチンゲールである.
        \item $\lim_{t\to\infty}E[\abs{B_t}^{-1}]=0$.
        特に,マルチンゲールではない.
    \end{enumerate}
\end{lemma}

\subsection{固定端Brown運動}

\begin{definition}[Brownian bridge / pinned Brownian motion]
    共分散を$\Gamma(s,t)=\min(s,t)-\frac{st}{T}$とする中心化されたGauss過程$(B_t)_{t\in[0,T]}\;(T>0)$を,時刻$T$で$0$に至る\textbf{Brown橋}という.
\end{definition}
\begin{remarks}
    共分散の形から,両端$\partial[0,T]$で固定され,真ん中$T/2$で最も不確実性が大きくなる過程であると分かる.
    これはBrown運動を条件づけた過程$B_t:=[W_t|W_T=0]\;(t\in[0,T])$と理解できる.
\end{remarks}
\begin{theorem}
    時刻$T$で$a\in\R$に至るBrown橋は
    \begin{enumerate}
        \item 次の微分方程式の解としても実現される:
        \[dX_t=dB_t+\frac{a-X_t}{T-t}dt,\;t\in[0,T),\quad X_0=0.\]
        \item その解は次のように表せる:
        \[X_t=at+(T-t)\int^t_0\frac{dB_s}{T-s}\quad 0\le t<T.\]
    \end{enumerate}
\end{theorem}

\begin{construction}
    1次元Brown運動$(W_t)$に対して
    $\paren{W_t-\frac{t}{T}W_T}_{t\le T}$とすれば,これはBrown橋である.
\end{construction}

\subsection{自由端Brown運動}

\begin{tcolorbox}[colframe=ForestGreen, colback=ForestGreen!10!white,breakable,colbacktitle=ForestGreen!40!white,coltitle=black,fonttitle=\bfseries\sffamily,
title=]
    反射壁Brown運動は拡散過程で,熱方程式のNeumann境界値+Cauchy初期値問題を解く\ref{thm-solution-to-Neumann-problem}.
\end{tcolorbox}

\begin{definition}[reflecting Brownian motion]
    $x\in\R^+$について,$B^+:=\abs{B+x}$と定めた確率過程を\textbf{反射壁Brown運動}という.
\end{definition}

\begin{lemma}[自由端Brown運動の遷移確率]
    任意の$0=t_0<t_1<\cdots<t_n$と$A_i\in\B(\R_+)$について,
    \[P[B^+_{t_1}\in A_1,\cdots,B^+_{t_n}\in A_n]=\int_{A_1}dx_1\cdots\int_{A_n}dx_n\prod^n_{j=1}g_+(t_j-t_{j-1},x_{j-1},x_j),\quad x_0=x,g_+(t,x,y)=\frac{1}{\sqrt{2\pi t}}\paren{e^{-\frac{(y-x)^2}{2t}}+e^{-\frac{(y+x)^2}{2t}}}.\]
\end{lemma}

\begin{lemma}
    $\phi\in C(\R_+)$を$\phi(0)=0$を満たす関数,$x\ge0$とする.
    連続関数$k\in C(\R_+)$が一意的に存在して,次を満たす:
    \begin{enumerate}
        \item $\forall_{t\in\R_+}\;x(t):=x+\phi(t)+k(t)\ge0$
        \item $k(0)=0$かつ$k$は単調増加.
        \item $\forall_{t\in\R_+}\;k(t)=\int^t_01_{\Brace{0}}(x(s))dk(s)$.すなわち,$k$は$x(s)=0$のときのみ増加する.
    \end{enumerate}
\end{lemma}

\begin{theorem}[Skorohod equation]
    過程$(X_t)_{t\in\R_+}$に対して,
    次を満たす$x\ge0$と連続過程$l$とが存在するならば,それは自由端Brown運動である:
    \begin{enumerate}
        \item $\forall_{t\in\R_+}\;X_t:=x+B_t+l_t\ge0$.
        \item $l(0)=0$かつ$l$は単調増加.
        \item $\forall_{t\in\R_+}\;l(t)=\int^t_01_{\Brace{0}}(X_s)dl_s$.
    \end{enumerate}
\end{theorem}

\subsection{Bessel過程}

\begin{definition}
    $n$次元の\textbf{Bessel過程}とは,$X_t:=\norm{B_t}$をいう.
\end{definition}
\begin{proposition}
    Bessel過程は次を満たす:
    \[dX_t=dB_t+\frac{n-1}{2}\frac{dt}{X_t}.\]
    $X_t$の推移確率は,変形Bessel関数を用いて表せる.
\end{proposition}

\subsection{多様体上のBrown運動}

\begin{tcolorbox}[colframe=ForestGreen, colback=ForestGreen!10!white,breakable,colbacktitle=ForestGreen!40!white,coltitle=black,fonttitle=\bfseries\sffamily,
title=]
    通常の正規形の偏微分方程式同様,係数関数が各接空間に値を取るとすれば,多様体上の確率微分方程式を得る.
\end{tcolorbox}

\begin{definition}
    Riemann多様体$(M,g)$上のLaplace作用素$\Laplace_M/2$が生成する拡散過程$(X_t,P_x)_{x\in M}$を$M$上のBrown運動と呼ぶ.ただし,$P_x$は熱方程式の基本解を用いて表示できる.
\end{definition}

\begin{proposition}
    任意の$x\in\R$に対して,$B(0)=x$を満たす\textbf{両側Brown運動}$(B_t)_{t\in\R}$が存在する.
\end{proposition}

\begin{proposition}
    任意の多様体上のBrown運動に対して,Euclid空間へ多様体を埋め込むこと,または直交枠束を用いることで,あるベクトル場が定める確率微分方程式の解として実現される.
\end{proposition}

\subsection{幾何Brown運動}

\begin{definition}[geometric Brownian motion]\label{def-GBM}
    次で表される確率過程を,拡散係数(volatility) $\theta$を持つ\textbf{幾何Brown運動}という:
    \[X_t=\exp\paren{\theta B_t-\paren{\frac{\theta^2t}{2}}}.\]
    これは次の確率微分方程式の解である:
    \[dX_t=\theta X_tdB_t,\quad \theta\in\R.\]
    $\theta\in L^2(\P)$と拡張できる.
\end{definition}

\begin{definition}[stochastic exponential (Doleans-Dade)]
    半マルチンゲール$X$に対して,
    \[dY_t=Y_{t-}dX_t,\quad Y_0=1\]
    の解$Y$を\textbf{指数過程}といい,$\E(X)$と表す.
\end{definition}

\begin{proposition}\mbox{}
    \begin{enumerate}
        \item $X$が連続ならば,
        \[\E(X)=\exp\paren{X-X_0-\frac{1}{2}\brac{X}}.\]
        \item さらに$X$が有界変動ならば,
        \[\E(X)=\exp(X-X_0).\]
        \item $X$が連続な局所マルチンゲール
        \[X_t=\int^t_0\theta_sdB_s-\frac{1}{2}\int^t_0\theta^2_sds\]
        であるとき,$\E(X)$がマルチンゲールであるための十分条件の一つにNovikovの条件\ref{lemma-Novikov}
        \[E\Square{\exp\paren{\frac{1}{2}\int^\infty_0\abs{\theta}^2_sds}}<\infty\]
        がある.
    \end{enumerate}
\end{proposition}

\subsection{CIR過程}

\begin{tcolorbox}[colframe=ForestGreen, colback=ForestGreen!10!white,breakable,colbacktitle=ForestGreen!40!white,coltitle=black,fonttitle=\bfseries\sffamily,
title=]
    Cox, Ingersoll, Rossによる,金利の変動のモデルとして開発された過程.
    CIRモデルは1つのマーケットリスクのみを考慮した,短期利子率を扱う単因子(one-factor)モデルであり,
    $dB_t$の係数に$\sqrt{X_t}$を追加することで$X_t$が負の値を取ることを回避していて(実際$\forall_{t\in\R_+}\;X_t>0\;\as$),
    Vasicekモデルの拡張とみれる.
\end{tcolorbox}

\begin{definition}
    利子率の過程$(X_t)$として提案されたものには,次のようなものがある:
    \begin{enumerate}
        \item $a,b,\sigma>0$が定める確率微分方程式$dX_t=a(b-X_t)dt+\sigma dB_t$の解を\textbf{Vasicek過程}という.
        \item $a,b,\sigma>0$が定める確率微分方程式$dX_t=a(b-X_t)dt+\sigma\sqrt{X_t}dB_t$の解を\textbf{CIR過程}という.
    \end{enumerate}
\end{definition}

\begin{lemma}
    CIR過程は,$2ab>\sigma^2$が成り立つとき,一意に存在する.
\end{lemma}

\subsection{Ornstein-Uhlenbeck拡散}

\begin{tcolorbox}[colframe=ForestGreen, colback=ForestGreen!10!white,breakable,colbacktitle=ForestGreen!40!white,coltitle=black,fonttitle=\bfseries\sffamily,
title=]
    摩擦の存在下での質量の大きいBrown粒子の速度のモデルとして最初の応用が見つかった,定常過程かつMarkov過程でもあるGauss過程である.
    なお,この3条件を満たす(時空間変数の線形変換の別を除いて)唯一の非自明な過程である.
    時間的に平均へドリフトする,平均回帰性(mean-reverting)を持つために,拡散過程でもある.
\end{tcolorbox}

\begin{definition}[Ornstein-Uhlenbeck process]
    次で表される確率過程$X_t$を,ドリフト$\gamma$と拡散係数$\sigma$を持つ\textbf{Ornstein-Uhlenbeck過程}という:
    \[X_t=e^{-t\gamma}X_0+\int^t_0\sigma e^{-(t-s)\gamma}dB_s.\]
    これは次の確率微分方程式の解である:
    \[dX_t=\sigma dB_t-\gamma X_tdt\quad\gamma\in\R.\]
    この確率微分方程式はLangevin方程式とも呼ばれ,次のようにも表せる:
    \[\dd{U(t)}{t}=-\lambda U(t)+\dot{B}(t)\]
    ただし$\dot{B}$は白色雑音である.
\end{definition}

\begin{lemma}
    $(\gamma,\sigma)$に関するOrnstein-Uhlenbeck過程$X$は次を満たす:
    \begin{enumerate}
        \item Brown運動による表示:$X_t=\frac{\sigma}{\sqrt{2\gamma}}e^{-\gamma t}B_{e^{2\gamma t}}$.
        \item $E[X_t]=0,\Cov[X_t,X_s]=\frac{\sigma^2}{2\theta}e^{-\gamma\abs{t-s}}$.
        \item Kolmogorovの連続変形定理を満たし,連続な修正を持つ.また,共分散が$\abs{t-s}$の関数であるため,定常過程であることも解る.
        \item 初期値で条件付けると,$m_t=E[X_t|X_0=x_0]=x_0e^{-\gamma t}$,$\Cov[X_s,X_t|X_0=x_0]=\frac{\sigma^2}{2\gamma}\paren{e^{-\gamma(\abs{t-s})}-e^{-\gamma(t+s)}}$.
        \item (Mehlerの公式) $X_t$の遷移確率は次のように表せる:
        \[P[X_t\in dx]=\frac{1}{(2\pi)^{d/2}(\det\Var[X_t])^{1/2}}\exp\paren{-\frac{1}{2}(x-m_t)^\top\Var[X_t]^{-1}(x-m_t)}dx.\]
    \end{enumerate}
    パラメータ$c:=\frac{\sigma^2}{2\gamma}$を\textbf{サイズ}ともいう.
\end{lemma}

\begin{remarks}\mbox{}
    \begin{enumerate}
        \item 一次元周辺分布は$\forall_{t\in\R}\;X_t\sim N(0,1)$である.
        \item 離散時間の過程に$AR(1)$というものがあり,この連続化とみなせる.
        \item Brown運動と違って,時間反転可能であり,$(X_t)_{t\in\R_+},(X_{-t})_{t\in\R_+}$は分布が等しい.
        \item $-\infty$近傍の振る舞いはBrown運動の$0$近傍の振る舞いに関係がある.
    \end{enumerate}
\end{remarks}

\begin{theorem}[Gauss-Markov過程の性質]
    過程$X$はGauss過程であり,Markov過程でもあるとする.
    \begin{enumerate}
        \item 零でない関数$h\in L(\R_+)$に対して,$Z:=hX$は再びGauss-Markov過程である.
        \item 単調増加関数$f\in L(\R_+)$に対して,$Z:=X\circ f$は再びGauss-Markov過程である.
        \item $X$は非退化で$L^2(\Om)$-連続ならば,ある零でない関数$h\in L(\R_+)$と狭義単調増加関数$f\in L(\R_+)$が存在して,$X_t=h(t)W_{f(t)}$と表せる.
    \end{enumerate}
\end{theorem}

\begin{proposition}[定常Gauss-Markov過程の分類]
    $V$は平均$m$と分散$\sigma^2$を持つ定常なGauss-Markov過程とする.このとき,次のいずれかである:
    \begin{enumerate}
        \item 任意の$0\le t_1<\cdots<t_n$に対して,$V_{t_1},\cdots,V_{t_n}$は独立な$N(m,\sigma^2)$-確率変数である.
        \item ある$\al>0$が存在して,任意の$0\le t_1<\cdots<t_n$に対して,$V_{t_1},\cdots,V_{t_n}$は共分散$\Cov[V_{t+\tau},V_t]=\sigma^2 e^{-\al\abs{\tau}}$を持つ$N(m,\sigma^2)$-確率変数である.
    \end{enumerate}
\end{proposition}

\subsection{Brown運動の回転}

\begin{proposition}[Brown運動の回転不変性]\label{prop-sum-of-two-independent-Brownian-motion}
    $B^1,B^2$を独立なBrown運動とする.
    \begin{enumerate}
        \item $W:=\frac{B^1+B^2}{\sqrt{2}}$はBrown運動である.
        \item 一般に,$m$-次元Brown運動$B$と直交行列$A\in O_m(\R)$に対して,$AB$は再び$m$-次元Brown運動である.
    \end{enumerate}
\end{proposition}
\begin{Proof}
    中心化された連続な過程であることは明らかだから,後は有限な周辺分布を調べれば良い.これは
    確率ベクトル
    \[\begin{pmatrix}W_{t_n}-W_{t_{n-1}}\\\vdots\\W_{t_2}-W_{t_1}\end{pmatrix}=\frac{1}{\sqrt{2}}\begin{pmatrix}B^1_{t_n}-B^1_{t_{n-1}}\\\vdots\\B^1_{t_2}-B^1_{t_1}\end{pmatrix}+\frac{1}{\sqrt{2}}\begin{pmatrix}B^2_{t_n}-B^2_{t_{n-1}}\\\vdots\\B^2_{t_2}-B^2_{t_1}\end{pmatrix}\]
    が$N_n(0,D),D:=\diag(t_n-t_{n-1},\cdots,t_2-t_1)$に従うことから明らか.
\end{Proof}

\subsection{Brown確率場}

\begin{tcolorbox}[colframe=ForestGreen, colback=ForestGreen!10!white,breakable,colbacktitle=ForestGreen!40!white,coltitle=black,fonttitle=\bfseries\sffamily,
title=]
    2次元空間上のBrown確率場$B:\Om\times\R_+^2\to\R$は,Wiener空間上の確率解析において基本的な役割を果たす.
    さらに一般のBrown確率場も,白色雑音から定義できる.
\end{tcolorbox}

\begin{definition}[Brownian random field / Brownian sheet]
    $\R_+^N$をパラメータにもつ中心化されたGauss確率変数の族$(X_a)_{a\in\R_+^N}$が次の2条件を満たすとき,\textbf{$N$次元Brown確率場}という:
    \begin{enumerate}
        \item $E[X_aX_b]=a\land b:=\prod_{i=1}^N(a_i\land b_i)$.
        \item $P[X_0=0]=1$.
    \end{enumerate}
\end{definition}

\begin{proposition}[Brown葉の構成]
    $N=2$のとき,独立な標準Brown運動の列$(B_t^{(n)})_{n\in\N}$を用いて,
    \[B_{s.t}:=sB_t^{(0)}+\sum_{n=1}^\infty B_t^{(n)}\frac{\sqrt{2}}{n\pi}\sin(n\pi s)\quad(s,t)\in[0,1]\times\R_+\]
    で構成される.
\end{proposition}

\section{Brown運動の性質}

\subsection{見本道の幾何的性質}

\begin{tcolorbox}[colframe=ForestGreen, colback=ForestGreen!10!white,breakable,colbacktitle=ForestGreen!40!white,coltitle=black,fonttitle=\bfseries\sffamily,
title=フラクタル曲線の典型例である]
    スケール不変であり,時間反転が可能である=関数$1/t$と$t$によって時間と空間のパラメータを変換すると再びBrown運動である.
    大数の法則によるとBrown運動が$y=\pm x$が囲む空間から無限回出ることはありえないが,$y=\pm\sqrt{n}$が囲む空間からは殆ど確実に無限回出ていく.
\end{tcolorbox}

\begin{proposition}\mbox{}\label{prop-character-of-Brownian-motion-1}
    \begin{enumerate}
        \item 自己相似性・スケール不変性:$\forall_{a>0}\;(X_t:=a^{-1/2}B_{at})_{t\ge0}$はBrown運動である.
        \item (原点)対称性:$(-B)_{t\in\R_+}$はBrown運動である.
        \item 時間一様性:$\forall_{h>0}\;(B_{t+h}-B_h)_{t\in\R_+}$はBrown運動である.
        \item 時間の巻き戻し:$(X_t:=B_1-B_{1-t})_{t\in[0,1]}$と$(B_t)_{t\in[0,1]}$とは分布が等しい.
        \item 時間反転不変性:次もBrown運動である:
        \[X_t=\begin{cases}
            tB_{1/t}&t>0,\\0&t=0
        \end{cases}\]
        \item 大数の法則:$\lim_{t\to\infty}\frac{B_t}{t}=0\;\as$\footnote{時間反転により,$\infty$での性質を$0$に引き戻して考えることが出来る,という議論の順番がきれいだと考えて入れ替えた.}        
        \item $1/2$-大数の法則\footnote{大数の法則と併せてみると,長い目で見て,Brown運動は線型関数よりは遅く増加するが,$\sqrt{t}$よりはlimsupが速く増加する.}:
        \[\limsup_{n\to\infty}\frac{B_n}{\sqrt{n}}=+\infty\;\as,\qquad\liminf_{n\to\infty}\frac{B_n}{\sqrt{n}}=-\infty\;\as\]
    \end{enumerate}
\end{proposition}
\begin{Proof}\mbox{}
    \begin{enumerate}
        \item 
        \begin{enumerate}[({B}1)]
            \item $X_0=a^{-1/2}B_0=0\;\as$
            \item $a^{-1/2}(B_{at_n}-B_{at_{n-1}}),\cdots,a^{-1/2}(B_{2}-B_{1})$は,独立な確率変数の可測関数$\R\to\R;x\mapsto a^{-1/2}x$との合成であるから,再び独立である.
            \item $B_{at}-B_{as}\sim N(0,a(t-s))$だから,$X_t-X_s=a^{-1/2}(B_{at}-B_{as})\sim N(0,a(t-s))$.
            \item $t\mapsto B_t(\om)$が連続になる$\om\in\Om$について,$t\mapsto a^{-1/2}B_{at}$は連続関数との合成からなるため連続.
        \end{enumerate}
        \item (1)と同様の議論による.
        \item (1)と同様の議論による.
        \item (1)と同様の議論とWiener測度の一意性による?
        \item $(X_t)$が平均$0$共分散$\min$のGauss過程であることを示してから,(B4)を示す.
        \begin{description}
            \item[Gauss過程であることの証明] 
            有限次元周辺分布$(X_{t_1},\cdots,X_{t_n})$は,正規確率ベクトル$(B_{1/t_1},\cdots,B_{1/t_n})$の,行列$\diag(t_1,\cdots,t_n)$が定める変換による値であるから,再び多変量正規分布に従う.
            よって,$(X_t)$はGauss過程である.
            平均は$E[X_t]=E[tB_{1/t}]=0$.共分散は
            \[\Cov[X_s,X_t]=E[X_sX_t]=stE[B_{1/s}B_{1/t}]=st\min\paren{\frac{1}{s},\frac{1}{t}}=\min(s,t).\]
            \item[見本道の連続性の証明]
            $t\mapsto X(t)$は$t>0$については殆ど確実に連続であるから,$t=0$での連続性を示せば良い.
            \begin{enumerate}[(a)]
                \item まず,$(X_t)_{t\in\R_+\cap\Q}$の分布は,標準Brown運動の制限$(B_t)_{t\in\R_+\cap\Q}$が$\R^\infty$上に誘導するものと等しい(Kolmogorovの拡張定理の一意性より).よって,$\lim_{t\searrow0,t\in\Q}X_t=\lim_{t\searrow0,t\in\Q}B_t=0\;\as$
                \item $(X_t)$は$\R_{>0}$上で殆ど確実に連続であることと,$\R_{>0}\cap\Q$のその上での稠密性より,極限は一致する:$\lim_{t\searrow0}X_t=\lim_{t\searrow,t\in\Q}X_t=0\;\as$
                
                実際,あるfull set $F\in\F$が存在して$\forall_{\om\in F}\;X_t(\om)$は$t\in\R_{>0}$上連続であるが,
                このとき\[\Abs{\lim_{t\searrow0}X_t(\om)-\lim_{t\searrow 0,t\in\Q}X_t(\om)}=2\ep>0\]とすると,
                \[\exists_{\delta_1>0}\;0<s<\delta_1\Rightarrow\Abs{X_s(\om)-\lim_{t\searrow0}X_t(\om)}<\ep\]
                \[\exists_{\delta_2>0}\;0<s<\delta_2\Rightarrow\Abs{X_s(\om)-\lim_{t\searrow 0,t\in\Q}X_t(\om)}<\ep\]
                が成り立つが,このとき$(0,\min(\delta_1,\delta))\cap\Q\ne\emptyset$上での$X_t$の値について矛盾が生じている.
            \end{enumerate}
        \end{description}
        \item (5)より,
        \[\lim_{t\to\infty}\frac{B(t)}{t}=\lim_{t\to\infty}X(1/t)=X(0)=0\;\as\]
        \item 任意の$c\in\Z_+$について,Brown運動のスケール不変性(1)より,$\{B_n>c\sqrt{n}\}\Leftrightarrow\{n^{-1/2}B_n>c\}\Leftrightarrow\{B_1>c\}$である.
        まず,集合に関するFatouの補題から,$P[\Brace{B_n>c\sqrt{n}\;\io}]\ge\limsup_{n\to\infty}P[\Brace{B_n>c\sqrt{n}}]=P[\Brace{B_1>c}]>0$.
        次に,$X_n:=B_n-B_{n-1}$とすると,$\Brace{B_n>c\sqrt{n}\;\io}=\Brace{\sum^n_{j=1}X_j>c\sqrt{n}\;\io}$は$(X_n)$について可換な事象であるから(手前の有限個の$X_1,\cdots,X_m$に対する置換に対して不変)であるから,Hewitt-Savageの定理より,$P[\Brace{B_n>c\sqrt{n}\;\io}]=1$.
        最後に,任意の$c\in\Z_+$について合併事象を取ると,$P\Square{\limsup_{n\to\infty}\frac{B_n}{\sqrt{n}}=\infty}=1$を得る.
    \end{enumerate}
\end{Proof}

\begin{proposition}[ユニタリ変換不変性 / conformal invariance property]
    多次元正規分布の性質が一般化される.
    $B$を$d$次元Brown運動,$U\in\GL_d(\R)$を直交行列とする.このとき,$(X_t:=UB_t)_{t\in\R_+}$はBrown運動である.
\end{proposition}

\subsection{連続性概念の補足}

\begin{tcolorbox}[colframe=ForestGreen, colback=ForestGreen!10!white,breakable,colbacktitle=ForestGreen!40!white,coltitle=black,fonttitle=\bfseries\sffamily,
title=]
    一様連続性の強さを定量化する手法が連続度である.
    $\ep$-$\delta$論法において,$\delta$から$\ep$を与える関数関係を分類することで,連続性を分類する.
\end{tcolorbox}

\begin{definition}[modulus of continuity]
    $(X,\norm{-}_X),(Y,\norm{-}_Y)$をノルム空間とし,$f:X\to Y$を関数とする.
    $X,Y$のそれ以外の位相的性質は不問とした方が良い.
    \begin{enumerate}
        \item $\om(\delta;f):=\sup_{\abs{x-y}\le\delta}\abs{f(x)-f(y)}$とおくと,$\om:[0,\infty]\to[0,\infty]$が定まる.
        このとき,$\forall_{f\in\Map(X,Y)}\;\om(0;f)=0$に注意.
    \end{enumerate}
\end{definition}

\subsubsection{大域的性質}

\begin{lemma}[uniform continuity]
    次の2条件は同値.
    \begin{enumerate}
        \item $\om(-;f)$は$\delta=0$にて連続:$\lim_{\delta\to0}\om(\delta;f)=0$.
        \item $f$は一様連続.
    \end{enumerate}
\end{lemma}

\begin{definition}[Lipschitz continuity, Holder continuity]
    $f:X\to Y$を関数とする.
    \begin{enumerate}
        \item $\om(\delta;f)=O(\delta)$を満たすとき,\footnote{$O(\delta)\;(\delta\to0)$ではなく,$\delta$の全域で$\delta$の定数倍で抑えられることをいう.}$f$はLipschitz定数$\sup_{\delta\in(0,\infty]}\frac{\om(\delta;f)}{\delta}$のLipschitz連続関数という.
        \item $\al\in[0,1]$について$\om(\delta;f)=O(\delta^\al)$を満たすとき,$f$は$\al$-次Holder連続であるという.なお,$\al$は大きいほど$\delta^\al\;(\delta\in(0,1))$は小さいから,条件は強く,$\al=0$のときは有界性に同値.
        \item $\abs{f}_{C^{0,\al}}:=\sup_{x\ne y}\frac{\abs{f(x)-f(y)}}{\abs{x-y}}$をHolder係数といい,これはHolder空間$C^{0,\al}$に半ノルムを定める.
    \end{enumerate}
    これらの条件は,$\om(\delta;f)$の$\delta\to0$のときの$0$への収束の速さによる分類であるから,これらを満たす関数は特に一様連続であることは明らか.
\end{definition}
\begin{remarks}
    Lipschitz連続性は,$X$上で任意の2点$x\ne y\in X$を取っても,その間の$f$の増分の傾き$\abs{f(x)-f(y)}/\abs{x-y}$はある定数を超えないことを意味する.
    特に有界な1次導関数を持つ場合,その上限$\sup_{x\in X}f'(x)$はLipschitz定数以下になるが,一致するとは限らない.
\end{remarks}

\subsubsection{局所的性質}

\begin{lemma}[continuous, locally uniform continuous]
    $f:X\to Y$と$x_0\in X$について,
    次の2条件は同値.
    \begin{enumerate}
        \item $\lim_{\delta\to0}\om(\delta;f|_{U_\delta(x_0)})=0$.
        \item $f$は$x_0$において連続.
    \end{enumerate}
    また次の2条件も同値.
    \begin{enumerate}
        \item ある近傍$x_0\in U\osub X$が存在して,$f|_U$の連続度$\om$は$\delta=0$にて連続.
        \item $f$は$x_0$において局所一様連続.
    \end{enumerate}
\end{lemma}
\begin{remark}
    この2つは$X$が局所コンパクトならば同値になる.したがって,たとえば無限次元Banach空間では2つの概念は区別する必要がある.
\end{remark}
\begin{Proof}\mbox{}
    \begin{description}
        \item[(1)$\Rightarrow$(2)] 
        任意の$\ep>0$を取ると,仮定よりある$\delta>0$が存在して,$\sup_{\abs{x-y}\le\delta,x,y\in U_{\delta}(x_0)}\abs{f(x)-f(y)}<\ep$.
        よって特に$\abs{x_0-y}<\delta$ならば,$\abs{f(x_0)-f(y)}\le \sup_{\abs{x-y}\le\delta,x,y\in U_{\delta}(x_0)}\abs{f(x)-f(y)}<\ep$.
        \item[(2)$\Rightarrow$(1)]
        任意の$\ep>0$を取ると,ある$\delta>0$について$\forall_{y\in X}\;\abs{x_0-y}<\delta\Rightarrow\abs{f(x_0)-f(y)}<\ep/2$が成り立つ.
        このとき,
        \[\sup_{\abs{x-y}\le\delta,x,y\in U_\delta(x_0)}\abs{f(x)-f(y)}\le\sup_{\abs{x-y}\le\delta,x,y\in U_\delta(x_0)}\paren{\abs{f(x)-f(x_0)}+\abs{f(x_0)-f(y)}}<\ep.\]
    \end{description}
\end{Proof}

\begin{definition}
    Holder係数$\abs{f}_{C^{0,\al}}:=\sup_{x\ne y}\frac{\abs{f(x)-f(y)}}{\abs{x-y}}\in[0,\infty]$が$X$のあるコンパクト集合上で有界ならば,局所Holder連続であるという.
\end{definition}

\subsubsection{関数の分類}

\begin{tcolorbox}[colframe=ForestGreen, colback=ForestGreen!10!white,breakable,colbacktitle=ForestGreen!40!white,coltitle=black,fonttitle=\bfseries\sffamily,
title=]
    ある$\om\in\Om$に対して,これを連続度として持つ関数として,連続性に基づいた類別を作れる.
    似た概念に一様可積分性がある.
\end{tcolorbox}

\begin{definition}
    $x=0$において連続で消える関数全体を
    \[\Om:=\Brace{\om\in\Map([0,\infty],[0,\infty])\mid\lim_{\delta\to0}\om(\delta)=\om(0)=0}\]
    と表す.
    $f:X\to Y,x_0\in X$について,
    \begin{enumerate}
        \item $\om_{x_0}(\delta;f):=\om(\delta;f|_{U_{\delta}(x_0)})$を,$x_0$における局所連続度としよう.
        \item 局所連続度の上方集合を$\Om_{x_0}(f):=\Brace{\om_{x_0}\in\Om\mid\forall_{y\in X}\;\abs{f(x_0)-f(y)}\le\om_{x_0}(\abs{x_0-y})}$と表すと,大域連続度の上方集合は$\Om(f):=\bigcap_{x\in X}\Om_x(f)$と表せる.
    \end{enumerate}
\end{definition}

\begin{definition}[equicontinuity, uniform equicontinuity]
    関数族$\{f_\lambda\}\subset\Map(X,Y)$について,
    \begin{enumerate}
        \item 同程度連続であるとは,任意の$x\in X$において,ある$\om_x(\delta;f)\in\Om_x(f)$が存在して,任意の$f_\lambda$の局所連続度(の上界)となっていることをいう.
        \item 一様に同程度連続であるとは,ある$\om(\delta;f)\in\Om(f)$が存在して,任意の$f_\lambda$の大域連続度(の上界)となっていることをいう.
    \end{enumerate}
    これらの概念も,$X$がコンパクトな場合は一致する.
\end{definition}

\begin{lemma}
    同程度連続な関数列$(f_n)$がある関数$f$に各点収束するとき,$f$は連続である.
\end{lemma}

\begin{theorem}
    有界閉区間$I:=[a,b]$上の関数列$\{f_n\}_{n\in\N}\subset l^\infty(I)$について,次の2条件を満たすならば一様ノルムについて相対コンパクトである,すなわち,一様収束する部分列を持つ.
    \begin{enumerate}
        \item 一様有界である:$\exists_{C\in\R}\;\forall_{n\in\N}\;\norm{f_n}_\infty\le M$.
        \item 一様に同程度連続である.
    \end{enumerate}
\end{theorem}

\subsubsection{確率化}

\begin{definition}[modulus of continuity]
    Brown運動のコンパクト集合への制限$B:[0,1]\times\Om\to\R$の\textbf{連続度}$\om_B$とは,
    \[\sup_{s\ne t\in[0,1]}\frac{\abs{B_t-B_s}}{\om_B(\abs{s-t})}\le 1\]
    かつ$\lim_{\delta\searrow0}\om_B(\delta)=0$を満たす関数$\om_B(\delta):[0,1]\times\Om\to\R_+$をいう.
\end{definition}
\begin{remark}
    Brown運動の見本道は連続だから,
    \[\limsup_{h\searrow 0}\sup_{t\in[0,1-h]}\frac{\abs{B_{t+h}-B_t}}{\varphi(h)}\le 1\]
    を考えても問題ない.
    これは,特に偏差$h$が小さい場合に関する評価になる.
    通常の意味での連続度$\om_B(f)$を得るためには,$[0,1]$上で$h$以上離れている2点についての$\sup_{h\le\abs{s-t}\le\delta}\abs{B_s-B_t}$と$\varphi(\delta)$との大きい方を$\om_B(\delta)$として取れば良い.

    一方で,supを外して,計測関数$\varphi:\R_+\to\R_+$であって,$\limsup_{t\in\R_+}\frac{B(t)}{\varphi(t)}\in\R_+$を満たすものを探すことは,連続度と対になる.
\end{remark}

\subsection{見本道の連続性}

\begin{tcolorbox}[colframe=ForestGreen, colback=ForestGreen!10!white,breakable,colbacktitle=ForestGreen!40!white,coltitle=black,fonttitle=\bfseries\sffamily,
    title=]
    見本道はコンパクト集合上で一様連続である.それがどのくらい強いかを測る尺度がHolder連続性であるが,
    見本道は任意の$0\le\gamma<\frac{1}{2}$について,$\gamma$-Holder連続であることは,Kolmogorovの連続変形定理\ref{thm-Kolmogorov-continuity-criterion}から従う.

    連続度の全体は最小元をもつが,ことに$\delta$が十分$0$に近いとき,$\om_B(\delta)=\sqrt{2\delta\log(\delta)}$が最小となる.これは,$\delta^{1/2}$より少し悪いことを意味する.
\end{tcolorbox}

\begin{proposition}\mbox{}
    \begin{enumerate}
        \item $1/2$-Holder連続ではない:$P\Square{\sup_{s\ne t\in[0,1]}\frac{\abs{B_t-B_s}}{\sqrt{\abs{t-s}}}=+\infty}=1$.\footnote{実は,局所$1/2$-Holder連続な点も存在するが,高々零集合である.また,$\al>1/2$については,ほとんど確実に,任意の点で$\al$-Holder連続でない.これは微分可能性についてのPaleyの結果よりも強い主張である.}
        \item 連続度の例:ある定数$C>0$について,十分小さい任意の$h>0$と任意の$h\in[0,1-h]$について,\[\abs{B_{t+h}-B_t}\le C\sqrt{h\log(1/h)}\;\as\]
        \item $\forall c<\sqrt{2}$について,$\forall_{\ep>0}\;\exists_{h\in(0,\ep)}\;\exists_{t\in[0,1-h]}\;\abs{B_{t+h}-B_t}\ge c\sqrt{h\log(1/h)}$.
        \item 最適評価(Levy 1937):
        $C=\sqrt{2}$のとき最適な連続度となる:\footnote{$\limsup$を$\lim$に置き換えても成り立つ.}
        \[\limsup_{\delta\searrow 0}\sup_{s,t\in[0,1],\abs{t-s}<\delta}\frac{\abs{B_t-B_s}}{\sqrt{2\abs{t-s}\log\abs{t-s}}}=1\;\as\]
        \item 任意の$\al<1/2$について,殆ど確実に見本道は任意の点で局所$\al$-Holder連続である.
        \item 任意の$\al>1/2$について,殆ど確実に見本道は任意の点で局所$\al$-Holder連続でない.これはPaley-Wiener-Zygmund 1933よりも強いことに注意.
        \item ほとんど確実に,見本道のどこかには局所$1/2$-Holder連続であるような点が存在する.これをslow pointという.
    \end{enumerate}
\end{proposition}
\begin{Proof}\mbox{}
    \begin{enumerate}
        \item 時間反転により$1/2$-大数の法則\ref{prop-character-of-Brownian-motion-1}(7)に帰着する.
        $\sup_{s\ne t\in[0,1]}\frac{\abs{B_t-B_s}}{\sqrt{\abs{t-s}}}>\sup_{t\in(0,1]}\frac{\abs{B_t-B_0}}{\sqrt{\abs{t-0}}}$であるから,
        \[P\Square{\sup_{t\in(0,1]}\frac{\abs{B_t}}{\sqrt{t}}=\infty}=1\]
        を示せば良いが,実は
        \[P\Square{\limsup_{t\searrow0}\frac{\abs{B_t}}{\sqrt{t}}=\infty}=1\]
        が成り立つ.実際,$X_t$を(5)の時間反転とすると,
        \[\limsup_{t\searrow0}\frac{\abs{B_t}}{\sqrt{t}}=\limsup_{t\searrow0}\sqrt{t}\abs{X_{1/t}}=\limsup_{s\to\infty}\frac{\abs{X_s}}{\sqrt{s}}=\infty\;\as\]
        \item 保留.
        \item a
        \item (7)と(8)の関係同様,時間反転による.
        \item e
        \item $X$を,$B$の時間反転とし(5),この$t=0$における微分可能性を考える.
        このとき,(7)より,
        \begin{align*}
            D^*X(0)&=\limsup_{h\searrow0}\frac{X_h-X_0}{h}\\
            &\ge\limsup_{n\to\infty}\frac{X_{1/n}-X_0}{1/n}\\
            &\ge\limsup_{n\to\infty}\sqrt{n}X_{1/n}=\limsup_{n\to\infty}\frac{B_n}{\sqrt{n}}=\infty.
        \end{align*}
        すなわち,$D^*X(0)=\infty$で,$D_*X(0)=-\infty$も同様にして得る.特に,$X_t$は$t=0$にて微分可能でない.
        
        ここで,任意の$t>0$に対して,$Y_s:=B_{t+s}-B_t$とすると,$Y_s$も標準Brown運動で(3),$Y_0$にて微分可能でない.
        よって,$B$も$B_t$において微分可能でない.
    \end{enumerate}
\end{Proof}

\begin{proposition}\mbox{}
    \begin{enumerate}
        \item 最小包絡関数:$\limsup_{t\to\infty}\frac{B_t}{\sqrt{2t\log\log(t)}}=1\;\as$\footnote{$1/2$-Holder連続性が$1/2$-大数の法則の系であったように,繰り返し対数の法則はこの系になる.}
        \item 一点における連続性の結果・繰り返し対数の法則(Khinchin 1933):\[\forall_{s\in\R_+}\;\limsup_{t\searrow s}\frac{\abs{B_t-B_s}}{\sqrt{2\abs{t-s}\log\log\abs{t-s}}}=1\;\as\]
    \end{enumerate}
\end{proposition}

\subsection{見本道の可微分性}


\begin{proposition}
    関数$f$の右・左微分係数を$D^*f(t):=\limsup_{h\searrow0}\frac{f(t+h)-f(t)}{h},D_*f(t)$で表すとする
    \begin{enumerate}
        \item 任意の$t\in\R_+$について,殆ど確実に$B_t$は$t$において微分可能でない上に,殆ど確実に$D^*B(t)=+\infty\land D_*B(t)=-\infty$.
        \item (Paley-Wiener-Zygmund 1933) $B$の見本道は殆ど確実に至る所微分不可能性である.\footnote{Paley et al. 1933, Dvoretzky et al. 1961.}
        さらに,ほとんど確実に$\forall_{t\in\R_+}\;D^*B(t)=+\infty\lor D_*B(t)=-\infty$.
        \item ほとんど確実に$\forall_{t\in[0,1)}\;D^*B(t)\in\{\pm\infty\}$という主張は正しくない.
    \end{enumerate}
\end{proposition}

\subsection{2次変分の$L^2$-収束}

\begin{tcolorbox}[colframe=ForestGreen, colback=ForestGreen!10!white,breakable,colbacktitle=ForestGreen!40!white,coltitle=black,fonttitle=\bfseries\sffamily,
title=]
    変分とは,分割の全体からなる有向集合からのネットの極限である.
    確率過程の2次変分過程は,高頻度取引の分野においてrealized volatilityといい,重要な統計量となっている.
    その漸近分散を求めるのに,混合型中心極限定理が必要となる.
    この定理は非エルゴード的な確率システムにおける「中心極限定理」となる.
\end{tcolorbox}

\begin{proposition}[erratic]\mbox{}
    \begin{enumerate}
        \item 殆ど確実に,任意の$0<a<b<\infty$に対して,Brown運動の見本道は$[a,b]$上単調でない.
        \item ほとんど確実に,局所的な増加点$t\in(0,\infty)\;\st\;\exists_{t\in(a,b)\osub\R_+}\;[\forall_{s\in(a,t)}\;f(s)\le f(t)]\land[\forall_{s\in(t,b)}\;f(t)\le f(s)]$を持たない.
    \end{enumerate}
\end{proposition}

\begin{notation}
    $\Sigma(0,T)$で,$[0,T]$上の分割$\sigma$全体の集合とする.すると,$\sigma_1\le\sigma_2:\Leftrightarrow\abs{\sigma_1}\le\abs{\sigma_2}$によって分割の集合には順序が定まる.
    \[J_\sigma:=\sum_{k=1}^n\abs{B_{t_k}-B_{t_{k-1}}}^2\]
    によって,写像$J:\Sigma(0,T)\to\R_+$が定まる.
\end{notation}

\begin{proposition}[2次変動の$L^2$-収束]\label{prop-quadratic-variation-of-Brownian-motion}
    $[0,t]$の部分分割$\pi:=(0=t_0<t_1<\cdots<t_n=t)$の系列$(\pi_n),\pi_n\subset\pi_{n+1}$について,$\abs{\pi}:=\max_{j\in n}(t_{j+1}-t_j)$とする.
    このとき,次の$L^2(\Om)$-収束が成り立つ:
    \[\lim_{\abs{\pi}\to0}J_\pi=\lim_{\abs{\pi}\to0}\sum^{n-1}_{j=0}(B_{t_{j+1}}-B_{t_j})^2=t\quad\in L^2(\Om).\]
\end{proposition}
\begin{Proof}
    確率変数列を$\xi_j:=(B_{t_{j+1}}-B_{t_j})^2-(t_{j+1}-t_j)\;(j\in n)$とおくと,これらは中心化された独立な確率変数列になる.
    実際,
    \[E[\xi_j]=E[(B_{t_{j+1}}-B_{t_j})^2]-E[t_{j+1}-t_j]=t_{j+1}-t_j-(t_{j+1}-t_j)=0.\]
    また,各$\xi_j$は独立な確率変数列$(B_{t_{j+1}}-B_{t_j})$の可測関数による像であるから,やはり独立である.
    
    正規分布の偶数次の中心積率は$\mu_{2r}=\frac{(2r)!}{2^rr!}\sigma^{2r}$と表せるため,$E[(B_{t_{j+1}}-B_{t_j})^4]=3(t_{j+1}-t_j)^2$であることに注意すると,
    \begin{align*}
        E\Square{\paren{\sum^{n-1}_{j=0}(B_{t_{j+1}}-B_{t_j})^2-t}^2}&=E\Square{\paren{\sum^{n-1}_{j=0}\xi_j}^2}=\sum^{n-1}_{j=0}E[\xi^2_j]\\
        &=\sum^{n-1}_{j=0}\paren{3(t_{j+1}-t_j)^2-2(t_{j+1}-t_j)^2+(t_{j+1}-t_j)^2}\\
        &=2\sum^{n-1}_{j=0}(t_{j+1}-t_j)^2\le 2t\abs{\pi}\xrightarrow{\abs{\pi}\to0}0.
    \end{align*}
\end{Proof}
\begin{remark}[2次変動の概収束]
    分割の列を具体的に指定すれば,概収束もする.
    なお,$(\pi_i)$が部分分割の系列でない場合,反例が殆ど確実に存在する.
    というのも,殆ど確実に分割の列で$\abs{\pi_n}\to 0$を満たすものが存在して,
    \[\limsup_{n\to\infty}\sum^{k(n)}_{j=1}(B(t_j^{(n)})-B(t^{(n)}_{j-1}))^2=\infty\]
    が成り立つ.
    他にこういうものを排除する十分条件として,$\sum_{n\in\N}\abs{\pi_n}<\infty$がある.
\end{remark}

\subsection{全変動が非有界}

\begin{tcolorbox}[colframe=ForestGreen, colback=ForestGreen!10!white,breakable,colbacktitle=ForestGreen!40!white,coltitle=black,fonttitle=\bfseries\sffamily,
title=]
    $L^2(\Om)$-収束する事実からも導けるが,ここでは2次変動が概収束する場合について考察しておくことで,
    より簡単な証明も与える.
\end{tcolorbox}

\begin{lemma}
    $X,Z\in L^2(\Om)$は互いに独立で,対称な確率変数とする:$X\overset{d}{=}-X,Z\overset{d}{=}-Z$.
    このとき,
    \[E[(X+Z)^2|X^2+Z^2]=X^2+Z^2\]
    が成り立つ.
\end{lemma}
\begin{Proof}
    $X+Z,X-Z$は同じ分布を持つから,
    \[E[(X+Z)^2|X^2+Z^2]=E[(X-Z)^2|X^2+Z^2]<\infty\]
    よって,両辺の差を取って,$E[XZ|X^2+Z^2]=0$を得る.
\end{Proof}

\begin{proposition}[2次変動の概収束部分列]
    $[0,t]$の分割の列$(\pi^n)_{n\in\N}$は,細分関係についての増大列になっているとする.
    このとき,
    \[\lim_{n\to\infty}\sum^{k(n)}_{j=1}(B_{t_j}^{(n)}-B_{t_j}^{(n)})=t\;\as.\]
\end{proposition}

\begin{proposition}[2次変動が概収束するための十分条件]
    $[0,t]$の分割の列$(\pi^n)_{n\ge 1},\pi^n:=\Brace{0=t^n_0<\cdots<t^n_{k_n}=t}$は$\sum_{n\in\N}\abs{\pi^n}<\infty$を満たしながら$\abs{\pi}\to0$と収束するとする.
    このとき,
    \[\sum^{k_n-1}_{j=0}\paren{B_{t^n_{j+1}}-B_{t^n_j}}^2\xrightarrow{\as}t.\]
\end{proposition}

\begin{corollary}[全変動の発散]
    区間$[0,t]$における全変動
    \[V:=\sup_{\pi}\sum^{n-1}_{j=0}\abs{B_{t_{j+1}}-B_{t_j}}\]
    は殆ど確実に$\infty$である.
\end{corollary}
\begin{Proof}
    Brown運動の見本道は殆ど確実に連続であるから,$\sup_{j\in n}\abs{B_{t_{j+1}}-B_{t_j}}\xrightarrow{\abs{\pi}\to0}0$である.
    よって,もし$V<\infty$ならば,2乗和
    \begin{align*}
        V_2:=\sum^{n-1}_{j=1}(B_{t_{j+1}}-B_{t_j})^2&\le\sup_{j\in n}\abs{B_{t_{j+1}}-B_{t_j}}\paren{\sum^{n-1}_{j=0}\abs{B_{t_{j+1}}-B_{t_j}}}\\
        &\le V\sup_{j\in n}\abs{B_{t_{j+1}}-B_{t_j}}\xrightarrow{\abs{\pi}\to0}0.
    \end{align*}
    よって,2次変動の$L^2$-極限は,
    \[\lim_{\abs{\pi}\to0}\sum^{n-1}_{j=0}(B_{t_{j+1}}-B_{t_j})^2=P[V=\infty]\paren{\lim_{\abs{\pi}\to0}\sum^{n-1}_{j=0}(B_{t_{j+1}}-B_{t_j})^2}=t\]
    と表せる.
    $P[V<\infty]=0$でない限り,2次変動が$t$に$L^2$-収束することに矛盾する.
\end{Proof}

\section{Wiener積分と白色雑音}\label{sec-Wiener-integral}

\begin{tcolorbox}[colframe=ForestGreen, colback=ForestGreen!10!white,breakable,colbacktitle=ForestGreen!40!white,coltitle=black,fonttitle=\bfseries\sffamily,
    title=]
    $B:\Om\to C_0(\R_+)$はWiener測度を押し出す.
    この測度は,$C_0(\R_+)$に限らず,$L^2(\R_+)$上に積分$L^2(\R_+)\to L^2(\Om)$を定める.
    するとこれは,非確率的関数に対する確率積分になる(確率積分はWiener積分の延長になる).

    しかし確率積分では思っても見なかった見方が登場する.そう,写像$L^2(\R_+)\to L^2(\Om)$自体が「クソでかい確率過程」なのだ(しかもGauss系).
    Wiener積分の適切な部分集合$H\subset L^2(\R_+)$への制限として得られる過程$H\to L^2(\Om)$を白色雑音という.
    これは$H^*$上に測度を押し出しかねない.
    実際,白色雑音は一般の核型可分Hilbert空間上にGauss確率測度として定義できる.
\end{tcolorbox}

\subsection{定義と像の性質}

\begin{notation}
    \[\E_0:=\Brace{\varphi_t=\sum^{n-1}_{j=0}a_j1_{(t_j,t_{j+1}]}(t)\in L^2(\R_+)\;\middle|\;n\ge 1,a_0,\cdots,a_{j-1}\in\R,0=t_0<\cdots<t_n}\]
    は$L^2(\R_+)$の稠密部分空間である.
\end{notation}

\begin{definition}[Paley-Wiener integral]\label{def-Wiener-integral}\mbox{}
    \begin{enumerate}
        \item 単関数$\varphi\in\E_0$上の有界線型作用素$\E_0\to L^2(\Om)$
        \[\int_{\R_+}\varphi_tdB_t:=\sum^n_{j=0}a_j(B_{t_{j+1}}-B_{t_j})\]
        は等長写像を定める.
        \item この$L^2(\R_+)$への連続延長$B:L^2(\R_+)\mono L^2(\Om)$を\textbf{Wiener積分}という.
        \item Wiener積分の像は$B(\varphi)=\int^\infty_0\varphi_tdB_t\sim N(0,\norm{\varphi}^2_{L^2(\R_+)})$を満たし,$\Im B<L^2(\Om)$は$L^2(\R_+)$と等長同型な,Brown運動が生成するGauss部分空間となる.
    \end{enumerate}
\end{definition}

\begin{proposition}
    Wiener積分の像
    \[\H_1:=\Brace{\int^\infty_0f(t)dB_t\;\middle|\;f\in L^2(\R_+)}\]
    は,共分散を$\Cov[f,g]=(f|g)_{L^2(\R_+)}$とする中心化されたGauss系である.
\end{proposition}

\begin{definition}[innovation process]
    Gauss過程$X$が\textbf{標準的な表現を持つ}とは,関数$F:\R_+\to L^2(\R_+)$が存在して,
    \[X_t=\int^t_0F(u,t)dB(u)\]
    と表せ,$X_t$が生成する情報系がBrown運動の情報系に一致することをいう.このときBrown運動$B$を,\textbf{$X$の新生過程}という.
\end{definition}

\subsection{白色雑音のWiener積分による表示}

\begin{tcolorbox}[colframe=ForestGreen, colback=ForestGreen!10!white,breakable,colbacktitle=ForestGreen!40!white,coltitle=black,fonttitle=\bfseries\sffamily,
title=]
    ホワイトノイズは,体積確定集合$\M$で添字付けられた中心化されたGauss系で,これら集合の共通部分の測度を共分散とするような構造を持ったものをいう.
\end{tcolorbox}

\begin{definition}[white noise]
    $D\subset\R^m$をBorel集合,$l$を$\R^m$上のLebesgue測度とする.
    \begin{enumerate}
        \item $D$内の測度確定な集合の全体を$\M(D):=\Brace{A\in\B(\R^m)\cap P(D)\mid l(A)<\infty}$と表す.
        \item $\M(D)$上の中心化されたGauss系$(W(A))_{A\in\M(D)}$であって,$\Gamma(A,B)=E[W(A)W(B)]=l(A\cap B)$を満たすものを\textbf{ホワイトノイズ}いう.
    \end{enumerate}
\end{definition}

\begin{example}[ホワイトノイズとBrown運動の関係]
    $D=\R_+$とすると,ホワイトノイズ$W:\M(\R_+)\to L^2(\Om)$を得る.これの,区間の集合$\I=\Brace{[0,t]}_{t\in\R_+}$への制限$W|_{\I}=B$はBrown運動である.
    逆にBrown運動を用いて,ホワイトノイズはWiener積分としての表示
    \[W(A)=\int^\infty_01_A(t)dB_t,\quad A\in\M(\D)\]
    を持つ.特に,定義関数に関するWiener積分はBrown運動である:
    \[B_t=\int^\infty_01_{[0,t]}dB_s.\]
\end{example}

\begin{proposition}
    $f\in L^2(\R_+)$のWiener積分はホワイトノイズである:$W_f=\int^\infty_0f(s)dB(s)$.
\end{proposition}
\begin{proposition}
    $H:=L^2([0,T])$とし,$\mu=N(0,Q)$を非退化なGauss測度とする.
    $W:H\to L^2(H,\mu)$を$\R_+$上のホワイトノイズとする.
    \begin{enumerate}
        \item $B_t:=W([0,t])=W_{1_{[0,t]}}$と定めるとこれはBrown運動の修正である.
        \item この過程の同値類は,$B\in C(\R_+;L^{2m}(H,\mu))\;(\forall_{m\in\N})$を満たす連続過程である.
    \end{enumerate}
\end{proposition}
\begin{Proof}\mbox{}
    \begin{enumerate}
        \item \begin{enumerate}[({B}1)]
            \item $B_t\sim N(0,t)\;(t\ge0)$より,$B_0=0$.
            \item 任意の$0\le s<t$について,$B_t-B_s=W_{1_{(s,t]}}\sim N(0,t-s)$.
            \item 任意の$0\le t_1<\cdots<t_n$に対して,$,1_{(t_1,t_2]},\cdots,1_{(t_{n-1},t_n]}$は直交系である.よって\ref{prop-property-of-white-noise}より,その像$B_{t_2}-B_{t_1},\cdots,B_{t_n}-B_{t_{n-1}}$も独立である.
            \item Gamma関数の反転公式より,
            \[\int^1_0(1-r)^{\al-1}r^{-\al}dr=B(\al,1-\al)=\frac{\pi}{\sin\pi\al}\]
            から,恒等式
            \[\int^t_s(t-\sigma)^{\al-1}(\sigma-s)^{-\al}d\sigma=\frac{\pi}{\sin\pi\al}\quad(0\le s\le\sigma\le t,\al\in(0,1))\]
            を得る.
            これを用いれば,
            \[1_{[0,t]}(s)=\frac{\sin\pi\al}{\pi}\int^t_0(t-\sigma)^{\al-1}\underbrace{1_{[0,\sigma]}(x)(\sigma-s)^{-\al}}_{=:g_\sigma(s)}d\sigma\qquad(s\ge0)\]
            を得る.このとき,$g_\sigma\in L^2([0,T])$かつ$\norm{g_\sigma}^2=\int^T_0g_\sigma^2(s)ds=\frac{\sigma^{1-\al}}{1-\al}$より,$\al\in\paren{0,\frac{1}{2}}$とリスケールすると,$\norm{g_\sigma}^2=\frac{\sigma^{1-2\al}}{1-2\al}$.
            よって,$W:H\to L^2(H,\mu)$は有界線型であったから特に一様連続で,
            \[B_t=W_{1_{[0,t]}}=\frac{\sin\pi\al}{\pi}\int^t_0(t-\sigma)^{\al-1}W_{g_\sigma}d\sigma\]
            と表示でき,$W_{g_\sigma}\sim N\paren{0,\frac{\sigma^{1-2\al}}{1-2\al}}$.
            補題より,あとは$\sigma\mapsto W_{g_\sigma}(x)\in L^{2m}([0,T])\;\mu\dae x\in H$を示せば良い.
            これは正規分布の$2m$次の積率とFubiniの定理から分かる.
        \end{enumerate}
        \item 任意の$t>s$について,$B_t-B_s\sim N_{t-s}$の$2m$次の積率から分かる.
    \end{enumerate}
\end{Proof}
\begin{lemma}
    $m>1,\al\in\paren{\frac{1}{2m},1},T>0$とし,$f\in L^{2m}([0,T];H)$とする.このとき,
    \[F(t):=\int^t_0(t-\sigma)^{\al-1}f(\sigma)d\sigma\quad t\in[0,T]\]
    と定めると,$F\in C([0,T];H)$.
\end{lemma}

\begin{remarks}
    Gauss過程$W:\M(D)\to L^2(\Om)$は内積を保つことに注意すると,これはHilbの等長同型$L^2(D)\iso L^2(\Om)$に延長する.
    この構造を用いて,ホワイトノイズは一般の可分Hilbert空間上に定義出来る.
    ホワイトノイズを$W:H\to L^2(\Om)$とすると,$W:\Om\times H\to \R$と同一視でき,
    これはもしかしたら見本道値確率変数としては$\Om\to H^*$とみなせて,$H^*$上に測度を押し出しているのかもしれない.
\end{remarks}

\subsection{Brown運動の超関数微分としての白色雑音}

\begin{tcolorbox}[colframe=ForestGreen, colback=ForestGreen!10!white,breakable,colbacktitle=ForestGreen!40!white,coltitle=black,fonttitle=\bfseries\sffamily,
title=]
    白色雑音を,もっと一般の可分Hilbert空間上のGauss測度として定める.
    さらに,
    Brown運動を定める線型作用素(の仮想的積分核)として,白色雑音を定める.この関係を「白色雑音は,ブラウン運動の弱微分である」という.
    白色雑音は$\S'$上のGauss測度として一意に定まる.
\end{tcolorbox}

\begin{discussion}[白色雑音は,Brown運動の超関数の意味での時間微分である]
    Wiener測度を緩増加超関数の空間に実現することで,仮想的な時間微分$dB_t$は自然に捉えられる.
    $\S$をSchwartzの急減少関数の空間とし,$\S'$をその双対空間とし,$\brac{-,-}$をこの間のペアリングとする.
    このとき,次の定理が成り立つ.
    \begin{theorem}[白色雑音の存在定理]\label{thm-existence-of-white-noise}
        $(\S',\B(\S'))$上には,標準Gauss測度$N(0,\id_{\S})$がただ一つ存在する.すなわち,
        次を満たす$(\S',\B(\S'))$上の確率測度$\nu$がただ一つ存在する:
        \[\forall_{\varphi\in\S}\quad\int_{\S'}\exp{i\brac{u,\varphi}}\nu(du)=\exp\paren{-\frac{1}{2}\int_\R\varphi^2(t)dt}=\exp\paren{\frac{1}{2}\brac{\id\varphi,\varphi}}\]
    \end{theorem}
    このときの確率空間$(\S',\B(\S'),\nu)$を\textbf{白色雑音}という.
    これを引き起こす過程$\dot{B}$を,$\R$上のBrown運動$(B_t)_{t\in\R}$を用いて構成する.
    \[\dot{B}(\varphi):=-\int_\R B_t\varphi'(t)dt\quad(\varphi\in\S)\]
    とすると,これは$(\S,\B(\S))$上の$\S'$-値確率変数であり,白色雑音$\nu$に従う.
\end{discussion}

\subsection{経路積分}

\begin{tcolorbox}[colframe=ForestGreen, colback=ForestGreen!10!white,breakable,colbacktitle=ForestGreen!40!white,coltitle=black,fonttitle=\bfseries\sffamily,
title=]
    量子力学,量子場理論で生まれた考え方である.
\end{tcolorbox}

\begin{discussion}
    伝播関数(propagator)は,粒子が移動する際の確率振幅を与える.
    これはある積分核$K$が与える積分変換作用素$U$で与えられるが,この積分核が曲者である.
    場の付値が関数$\varphi$で与えられ,関数$\varphi$の空間上の,作用汎関数$S$の積分
    \[K(x,y):=\int\exp(iS(\varphi))D\varphi\]
    で与えられる.しかし,$D\varphi$なる測度が不明瞭どころか,対応する測度が存在しないこともある.
    \textbf{経路積分(path integral)}の語源は,多様体$X$上の粒子を記述するシグマ模型においては,$\varphi\in C([0,1],X)$は厳密に「道」になるため.
    ここで(古典的)Wiener空間が登場する.
\end{discussion}

\subsection{Feynman-Kacの公式}

\begin{tcolorbox}[colframe=ForestGreen, colback=ForestGreen!10!white,breakable,colbacktitle=ForestGreen!40!white,coltitle=black,fonttitle=\bfseries\sffamily,
title=]
    Mark KacはFeynman経路積分とWiener積分の類似性にはじめて注目し,Wiener積分を通じて微分作用素を研究した.
    特に\textit{Can One Hear the Shape of a Drum?} (1966)の論説は有名である.
\end{tcolorbox}

\section{Wiener空間}

\begin{tcolorbox}[colframe=ForestGreen, colback=ForestGreen!10!white,breakable,colbacktitle=ForestGreen!40!white,coltitle=black,fonttitle=\bfseries\sffamily,
title=]
    Wiener(1923)が,初期の無限次元空間とその上の微積分,すなわち,見本道の空間上の変分法の例を作った.
    まずはこの見本道の空間の構造を調べる.
\end{tcolorbox}

\subsection{古典的Wiener空間}

\begin{tcolorbox}[colframe=ForestGreen, colback=ForestGreen!10!white,breakable,colbacktitle=ForestGreen!40!white,coltitle=black,fonttitle=\bfseries\sffamily,
title=]
    Brown運動$\Om\to C(\R_+)$は$C$-過程であり,$C$-空間にWiener測度を押し出す.
    これを古典的Wiener空間という.
\end{tcolorbox}

\begin{definition}[Wiener space (23)\footnote{本人は"Differential space"と呼んだ}]\mbox{}
    \begin{enumerate}
        \item $\Om:=\Brace{\om\in C(\R_+;\R)\mid\om(0)=0}$.
        \item $\F\subset P(\Om)$を,$C(\R_+)$上の広義一様収束位相が生成するBorel $\sigma$-代数とする.このコンパクト開位相について,$\Om$はFrechet空間となる.
    \end{enumerate}
\end{definition}

\begin{proposition}[Wiener測度の構成]\label{prop-construction-of-Wiener-measure}
    $(\Om,\F,P)$をWiener空間とする.
    \begin{enumerate}
        \item $\R_+$上の広義一様収束位相が生成するBorel $\sigma$-代数$\F$は次の円筒集合の集合$\cC\subset P(\Om)$によって生成される:$\B(\Om)=\sigma[\cC]$
        \[\cC=\Brace{C=\Brace{\om\in\Om\mid\forall_{i\in[k]}\;\om(t_i)\in A_i}\in P(\Om)\;\middle|\;k\in\N_{\ge1},A_1,\cdots,A_k\in\B(\R),0\le t_1<\cdots<t_k}\]
        \item $p_t(x)=(2\pi t)^{-1/2}e^{-x^2/(2t)}$を正規分布$N(0,t)$の確率密度関数とする.
        \[P(C)=\int_{A_1\times\cdots\times A_k}p_{t_1}(x_1)p_{t_2-t_1}(x_2-x_1)\cdots p_{t_k-t_{k-1}}(x_k-x_{k-1})dx_1\cdots dx_k.\]
        が成り立ち,これの$\F$上への延長はたしかに一意的である.
        \item 任意の$\om\in\Om$に対して,$B_t(\om):=\om(t)$と見本道を定めると,この対応$\Om\to\Om$はBrown運動である.
    \end{enumerate}
\end{proposition}
\begin{Proof}\mbox{}
    \begin{enumerate}
        \item \begin{description}
            \item[$\sigma(\cC)\subset\F$] 任意の$k\in\N,0\le t_1<\cdots<t_k,A_1,\cdots,A_k\in\B^1(\R)$に対して,$C(t_1\in A_1,\cdots,t_k\in A_k)=\bigcap_{i\in[k]}\pr_{t_i}^{-1}(A_i)$と表せる.$A_i$が全て開集合であるとき,$C$も開集合となる.$A_i$が一般のとき,写像の逆像の集合演算に対する関手性より,$C$は開集合または閉集合の可算和・積で表せる.
            \item[$\F\subset\sigma(\cC)$] コンパクト開位相は半ノルムの族$(\rho_n(\om):=\sup_{t\in K_n}\abs{\om(t)})_{n\in\N}$が定める始位相と一致する:任意の$n\in\N,A\in\B^1(\R)$に関して,$\rho_n^{-1}(A)=W(K_n,A)\in\sigma[\cC]$である.
        \end{description}
        \item 
        $(\Om,\F,P)$は$\sigma$-有限である:時刻$t\in\R_+$を固定し,$\R$の増大コンパクト集合$(A_n)$に対して$C(t\in A_n)$を考えると,$P(C(t\in A_n))<1$である.
        あとは$\cC$が有限加法的であることと,$P$がその上で完全加法的であることを示せば良い.
        \begin{description}
            \item[$\cC$の有限加法性] 任意の$C(\om_1\in A_1,\cdots,\om_k\in A_k)\cup C(\om'_1\in A'_1,\cdots,\om'_l\in A'_l)\in\cC$,$C(\om_1\in A_1,\cdots,\om_k\in A_k)^\comp=C(\om_1\in A_1^\comp,\cdots,\om_k\in A_k^\comp)\in\cC$.
            \item[$P$の完全加法性] $\{C_n\}\subset\cC$を互いに素な集合列で,$C:=\sum_{n\in\N}C_n\in\cC$を満たすとする.これについて$P(C)=\sum_{n\in\N}P(C_n)$を示せば良い.
            このとき,必要ならば添字を並び替えることにより,ある$k\le l\in\N$を用いて,
            \begin{align*}
                C&=\Brace{\om\in\Om\mid\om(t_1)\in A_1,\cdots,\om(t_k)\in A_k,\om(t_{k+1})\in\R,\cdots,\om(t_l)\in\R},&
                C_n&=\Brace{\om\in\Om\mid\om(t_1)\in A_1^n,\cdots,\om(t_l)\in A_l^n}
            \end{align*}
            と表せる.ただし,
            \[A_1=\sum_{n\in\N}A_1^n,\cdots,A_k=\sum_{n\in\N}A_k^n,\R=\sum_{n\in\N}A_i^n\;(k<i\le l).\]
            このとき,$P(C)=\sum_{n\in\N}P(C_n)$は明らか.
        \end{description}
        \item $(B_t)_{t\in\R_+}$の任意の見本道$B_t(\om)=\om(t)$は連続である.
        Brown運動の存在は認めたから,あとは$P$がBrown運動の法則であることを示せば良い.
        \begin{enumerate}[({B}1)]
            \item $\forall_{\om\in\Om}\;B_0(\om)=\om(0)=0$.
            \item 任意の$0\le t_1<\cdots,t_n$について,$B_{t_n}-B_{t_{n-1}},\cdots,B_{t_2}-B_{t_1}$が独立であることを示すために,確率ベクトル$(B_{t_2}-B_{t_1},\cdots,B_{t_n}-B_{t_{n-1}})^\top$の特性関数を調べる.
            Brown運動$(W_t)$の独立増分性より,
            \begin{align*}
                \int_\Om e^{iu\cdot \begin{pmatrix}B_{t_2}-B_{t_1}\\\vdots\\B_{t_n}-B_{t_{n-1}}\end{pmatrix}}&=\int_{\Om'}e^{iu\cdot \begin{pmatrix}W_{t_2}-W_{t_1}\\\vdots\\W_{t_n}-W_{t_{n-1}}\end{pmatrix}}dP'\\
                &=\int_{\Om'}e^{iu(W_{t_2}-W_{t_1})}dP'\times\cdots\times\int_{\Om'}e^{iu(W_{t_n}-W_{t_{n-1}})}dP'\\
                &=\int_\Om e^{iu(B_{t_2}-B_{t_1})}dP\times\cdots\times\int_\Om e^{iu(B_{t_n}-B_{t_{n-1}})}dP.
            \end{align*}
            \item 任意の$0\le s<t$について,$B_t-B_s\sim N(0,t-s)$を示す.補題の確率変数$W:\Om'\to\Om$を用いて,
            \begin{align*}
                \varphi(u)&=\int_\Om e^{iuB_t(\om)-B_s(\om)}dP(\om)\\
                &=\int_\Om e^{iu(\om(t)-\om(s))}P(d\om)\\
                &=\int_\Om e^{iu W(t,\om')-W(s,\om')}P'(d\om')=\exp\paren{-\frac{1}{2}u^2(t-s)}\quad(u\in\R)
            \end{align*}
        \end{enumerate}
    \end{enumerate}
\end{Proof}

\begin{lemma}[コンパクト開位相の性質]
    $K\compsub\R_*,U\osub\R$について,
    \[W(K,U):=\Brace{f\in\Om| f(K)\subset U}\]
    を準基\footnote{一般にコンパクト集合の有限共通部分はコンパクトとは限らないことに注意.}として生成される$\Om$上の位相を\textbf{コンパクト開位相}といい,次が成り立つ.
    \begin{enumerate}
        \item $e:\Om\times\R_+\to\R$は連続.特に$\pr_t:\Om\to\R\;(t\in\R_+)$は連続(逆は言えないことに注意).
        \item $K_n:=[0,n]$として,半ノルムの族$(\rho_n(\om):=\sup_{t\in K_n}\abs{\om(t)})_{n\in\N}$が定める始位相と一致する.
    \end{enumerate}
\end{lemma}

\begin{lemma}[Wiener測度の特徴付け]
    $(B_t)$を$(\Om',\F',P')$上のBrown運動とする.
    このとき,$B:\Om'\to\Om;\om'\mapsto B(-,\om')$とするとこれは可測で,
    命題\ref{prop-construction-of-Wiener-measure}(2)で定まる$\Om$上の測度$P$に対して$P'^B=P\;\on\F$を満たす.
    特に,次の変数変換の公式を得る:
    \[\forall_{\varphi\in\L^1(\Om,\F,P)}\quad\int_\Om\varphi(\om)P(d\om)=\int_{\Om'}\varphi(B(-,\om'))P'(d\om').\]
\end{lemma}
\begin{Proof}
    可測性は次の議論より明らか.

    任意の$k\in\N,0\le t_1<\cdots<t_k,A_1,\cdots,A_k\in\B^1(\R)$について,$C(t_1,\cdots,t_k;A_1,\cdots,A_k)$上での値が一致すればよいが,Brown運動の独立増分性と,増分が正規分布に従うことより,
    \begin{align*}
        P'^B(C(t_1,\cdots,t_k;A_1,\cdots,A_k))&=P'[B_{t_1}\in A_1,\cdots,B_{t_k}\in A_k]\\
        &=\int_{A_1\times\cdots\times A_k}p_{t_1}(x_1)p_{t_2-t_1}(x_2-x_1)\cdots p_{t_k-t_{k-1}}(x_k-x_{k-1})dx_1\cdots dx_k\\
        &=P[C(t_1,\cdots,t_k;A_1,\cdots,A_k)].
    \end{align*}
\end{Proof}

\subsection{Brown運動の情報系}

\begin{tcolorbox}[colframe=ForestGreen, colback=ForestGreen!10!white,breakable,colbacktitle=ForestGreen!40!white,coltitle=black,fonttitle=\bfseries\sffamily,
title=]
    一般の確率空間$(\Om,\F,P)$上のBrown運動が生成する増大情報系をBrownian filtrationという.
    古典的Wiener空間では,これはKolmogorovの$\sigma$-加法族と一致するのであったが,これは一般の定義域$(\Om,\F,P)$でも同様である.
\end{tcolorbox}

\begin{lemma}[自然な情報系の特徴付け]\label{lemma-Blumenthal}
    $(B_t)$をある確率空間$(\Om,\F,P)$上のBrown運動とする.
    このBrown運動の自然な情報系$(\F_t)$について,
    ここでは$\F_t=\sigma[B_s|s\le t]$として必ずしも閉$\sigma$-代数ではないとする.
    \begin{enumerate}
        \item (左連続性) $\F_t$は柱状集合$C(t_1,\cdots,t_n;I_1,\cdots,I_n)\;(0\le t_1<\cdots<t_n\le t)$の全体によって生成される.また,$t_n<t$としても変わらない.
        \item ($0$における右連続性) また$\F$はBrown運動の族$(B_{t+h}-B_h)_{t\in\R_+,h>0}$の柱状集合$D(t_1,\cdots,t_n;h;I_1,\cdots,I_n)\;(h>0)$の全体によっても生成される:
        \[D(t_1,\cdots,t_n;h;I_1,\cdots,I_n):=\Brace{\om\in\Om\mid B_{t_1+h}(\om)-B_{h}(\om)\in I_1,\cdots,B_{t_n+h}(\om)-B_h(\om)\in I_n}\]
        \item ($B_s\indep B_t-B_s$の一般化) 任意の$0\le s<t$について,$\F_s$は$B_t-B_s$に対して独立である.したがって特に,任意の$\F_s$-可測関数$X:\Om\to\R$は$B_t-B_s$と独立である.
        \item (Blumenthal one-zero law) $A\in\F_0^+:=\cap_{\ep>0}\F_\ep$ならば,$P[A]\in\{0,1\}$.
    \end{enumerate}
\end{lemma}
\begin{Proof}\mbox{}
    \begin{enumerate}
        \item $\F_t$は任意の有限列$I_1,\cdots,I_n\in\B^1(\R)$と任意の有限列$0\le s_1<\cdots<s_n\le t$を用いて,$C(s_1,\cdots,s_n;I_1,\cdots,I_n)$によって生成されるということである.
        また,$t_n<t$を満たす柱状集合の全体を用いても,見本道の連続性より,任意の開区間$I\osub\R$について,$\Brace{B_t\in I}=\limsup_{n\to\infty}\Brace{B_{t-1/n}\in I}$と表せる.
        \item $n\in\N,0\le t_1<\cdots<t_n,h>0,I_i\in\B^1(\R)$に対して,
        \[D(t_1,\cdots,t_n;h;I_1,\cdots,I_n):=\Brace{\om\in\Om\mid B_{t_1+h}(\om)-B_{h}(\om)\in I_1,\cdots,B_{t_n+h}(\om)-B_h(\om)\in I_n}\]
        なる形の集合全体が生成する$\sigma$-代数を$\cG$とすると,実は$\cG=\F$である.
        \begin{enumerate}[(a)]
            \item $\F\subset\cG$は,任意の柱状集合$C(s_1,\cdots,s_n;A_1,\cdots,A_n)$は,$\cap_{i\in[n]}C(s_i;A_i)$と表せるから,
            $C(s;A)\;(s\ge0,A\in\B^1(\R))$が$\cG$に入っていることを示せば良い.また$A$として区間$(\infty,a)\;(a\in\R)$を取ると,
            \[\Brace{B_s<a}\overset{\as}{=}\Brace{\lim_{n\to\infty}(B_{s+1/n}-B_{1/n})<a}=\limsup_{n\to\infty}\Brace{B_{s+1/n}-B_{1/n}<a}\in\cG\]
            \item $\cG\subset\F$は,$(B'_s:=B_{s+h}-B_h)_{s\in\R_+}$は再びBrown運動であることより,$D(t_1;h;I_i)=\Brace{B'_{t_1}\in I_1}\in\sigma[B'_{t_1}]\subset\F$.
        \end{enumerate}
        \item $\F_s$は,$C(s_1,\cdots,s_n;J_1,\cdots,J_n)\;(0\le s_1<\cdots<s_n\le s,J_i\in\B^1(\R))$の形の柱状集合がなす集合体によって生成されるから,Dynkin族定理より,
        \[P[(B_t-B_s\in I),C(s_1,\cdots,s_n;J_1,\cdots,J_n)]=P[B_t-B_s\in I]P[C(s_1,\cdots,s_n;J_1,\cdots,J_n)]\]
        を示せば,残りの等式は単調収束定理より従う.

        これは,%2つの確率ベクトル$(B_{s_1},\cdots,B_{s_n}),B_t-B_s$の独立性から従う.
        %任意の$i\in[n]$に対して,$\Cov[B_{s_i},B_t-B_s]=\Cov[B_{s_i},B_t]-\Cov[B_{s_i},B_s]=0$を示せば良いが,これは明らか.
        %というのも,
        左辺は
        \[\int_{J_1\times\cdots\times J_n\times I}p_{s_1}(x_1)p_{s_2}(x_2-x_1)\cdots p_{s_n}(x_n-x_{n-1})p_{t-s}(x)dx_1\cdots dx_ndx\]
        と表せるが,これはFubiniの定理より右辺に等しい.
        \item 
        $A\in\F^+_0$を任意に取る.
        \begin{description}
            \item[$A\indep\cG$] $A$は$\F^+_0$の元であるから,特に$A\in\F_h$である.
            よって,任意の有限部分集合$\{0\le t_1<\cdots<t_n\}\subset\R_+$について,
            各$B_{t_i+h}-B_{h}$と独立であり,
            したがって$D(t_1,\cdots,t_n;h;I_1,\cdots,I_n)$とも独立である.
            $t_1,\cdots,t_n,h,I_1,\cdots,I_n$は任意に取ったから,$A\indep\cG$(Dynkin族定理と単調収束定理による).
            すなわち,$\forall_{G\in\cG}\;P[A\cap G]=P[A]P[G]$.
            \item[結論] 
            $\F^+_0\subset\F\subset\cG$より,$G=A$と取れる.よって,$P[A]=P[A]P[A]$を得る.
        \end{description}
    \end{enumerate}
\end{Proof}

\begin{proposition}[自然な情報系の右連続性]
    Brown運動の自然な情報系$(\F_t:=\F[B_s|s\le t])$について,$\F_t$が閉$\sigma$-代数であるとする(usual conditionの1つ).
    次の2条件が成り立つ.
    \begin{enumerate}
        \item $(\F_t)$-適合的過程の任意のバージョンは適合的である.
        \item $(\F_t)$は右連続である:$\forall_{t\in\R_+}\cap_{s>t}\F_s=\F_t$.
    \end{enumerate}
\end{proposition}
\begin{Proof}\mbox{}
    \begin{enumerate}
        \item $(X_t)_{t\in\R_+}$を$(\Om,\F,P)$上の$(\F_t)$-適合的な確率過程とする:$\forall_{t\in\R_+}\;X_t\in\L_{\F_t}(\Om)$.
        すなわち,$\forall_{t\in\R_+}\;\forall_{A\in\B^1(\R)}\;\Brace{X_t\in A}\in\F_t$.
        $Y$を$X$の同値な確率過程とすると,$\forall_{t\in\R_+}\;P^{X_t}=P^{Y_t}$より,
        特に$P[X_t\in A]=P[Y_t\in A]$が成り立つ.
        よって$Y$
        も$(\F_t)$-適合的になる.
        \item 
        任意の$t\in\R_+$を取る.$t=0$のときは補題(2)より成り立つ.

        $t>0$のときは,$(B_{h+t}-B_t)_{h\in\R_+}$は再びBrown運動であるから,このBrown運動についての自然な情報系を$(\cG_h)$とすると,
        $\cG_0$の元はすべてfull setかnull setであるから,$\cG_0\subset\F_t$,特に
        $\sigma[\F_t,\cG_0]=\F_t$であることより,
        \[\F_t=\sigma[\F_t,\cG_0]=\cap_{\ep\ge0}\sigma[\F_t,\cG_\ep]=\cap_{\ep>0}\sigma[\F_t,\cG_\ep].\]
    \end{enumerate}
\end{Proof}


\subsection{Cameron-Matrin定理}

\begin{tcolorbox}[colframe=ForestGreen, colback=ForestGreen!10!white,breakable,colbacktitle=ForestGreen!40!white,coltitle=black,fonttitle=\bfseries\sffamily,
title=]
    ドリフト$F:\R_+\to\R$付きのBrown運動を調べることは,標準Brown運動を調べることに等しい.
    これはちょうど$F$-Brown橋と標準Brown橋の関係と同じである.
    これは,Banach空間$\Om$上のGauss測度$\mu$の,平行移動に対する挙動を調べていることになる.

    この消息を支える$\Om$の稠密部分空間が,普遍的な役割を果たす.
\end{tcolorbox}

\subsubsection{Cameron-Martin部分空間}

\begin{tcolorbox}[colframe=ForestGreen, colback=ForestGreen!10!white,breakable,colbacktitle=ForestGreen!40!white,coltitle=black,fonttitle=\bfseries\sffamily,
title=]
    $F(0)=0$を満たす
    見本道
    $F\in C([0,1])$のうち,
    ある$F'\in L^2([0,1])$の積分として得られるものの全体をCameron-Martin部分空間という.
\end{tcolorbox}

\begin{notation}[skelton / Cameron-Martin subspace]\mbox{}
    \begin{enumerate}
        \item 標準Brown運動の法則を$L_0$,ドリフト$F$付きのBrown運動の法則を$L_F$で表す.
        $L_F\ll L_0\Leftrightarrow L_0(A)=0\Rightarrow L_F(A)=0$がいかなるときか?
        まず,$F$が連続で,$F(0)=0$が必要なのは明らかである.そうでなければ,$L_F$は連続でなかったり,$0$からスタートしない見本道に正の確率を許してしまうため.
        \item $D[0,1]:=\Brace{F\in C[0,1]\mid\exists_{f\in L^2[0,1]}\;\forall_{t\in[0,1]}\;F(t)=\int^t_0f(s)ds}$をDirichlet空間という.より一般にDirichlet空間は,Dirichlet積分なる半ノルムが有限となるようなHardy空間の部分空間で,正則関数の再生核Hilbert空間ともなる.
        このDirichlet空間には特別な名前がついており,skeltonまたはCameron-Martin部分空間と呼ばれる.
    \end{enumerate}
\end{notation}

\begin{definition}[Dirichlet space]
    $\Om\subset\C$上のDirichlet空間とは,正則関数の再生核Hilbert空間がなす$H^2(\Om)$の部分空間で,
    \[\D(\Om):=\Brace{f\in H^2(\Om)\mid \D(f)<\infty}\quad\D(f):=\frac{1}{\pi}\iint_\Om\abs{f'(z)}^2dA=\frac{1}{4\pi}\iint_\Om\abs{\partial_xf}^2+\abs{\partial_yf}^2dxdy.\]
    この右辺が変分原理であるDirichlet原理を定めるDirichlet積分になっていることから名前がついた.
\end{definition}
\begin{definition}[RKHS: reproductive kernel Hilbert space]
    $X$を集合,$H\subset\Map(X;\R)$をHilbert空間とする.
    評価関数$\ev_x:H\to\R$が任意の$x\in X$について有界線型汎関数である$\forall_{x\in X}\;\ev_x\in B(H)$とき,$H$を\textbf{再生核Hilbert空間}という.
    すなわち,Rieszの表現定理よりある$K_x\in H$が存在して$\ev_x(-)=(-|K_x)$と表せる.この対応$X\to H;x\mapsto K_x$が導く双線型形式$K:X\times X\to\R;(x,y)\mapsto K(x,y):=(K_x|K_y)$を\textbf{再生核}という.
    再生核は対称で半正定値である.
\end{definition}
\begin{theorem}
    関数$K:X\times X\to\R$は対称かつ半正定値であるとする.このとき,ただ一つのHilbert空間$H$が$\Map(X;\R)$内に存在して,$K$を再生核として持つ.
\end{theorem}

\subsubsection{Cameron-Martin定理}

\begin{tcolorbox}[colframe=ForestGreen, colback=ForestGreen!10!white,breakable,colbacktitle=ForestGreen!40!white,coltitle=black,fonttitle=\bfseries\sffamily,
title=]
    $L_F\ll L_0$ならば,標準Brown運動$B$のa.s.性質は$B+F$に引き継がれる.
    これはちょうどCameron-Martin部分空間について$F\in D[0,1]$に同値.
    したがって微分を見てみると良い.
\end{tcolorbox}

\begin{definition}[equivalence measure, absolute continuous, singular]
    可測空間$(\Om,\B)$上の測度を考える.
    \begin{enumerate}
        \item 2つの測度の零集合の全体が一致するとき,すなわち互いに絶対連続であるとき$\mu\ll\nu\land\nu\ll\mu$,これらは同値であるという.
        \item 絶対連続性は,測度の同値類の間に順序を定める.
        \item 2つの測度の「台」が分解出来るとき,すなわち,ある分割$\Om=A\sqcup A^\comp$が存在して$P(A)\cap\B\subset\cN(\mu)$かつ$P(A^\comp)\cap\B\subset\cN(\nu)$が成り立つとき,特異であるといい$\mu\perp\nu$と表す.
        \item Lebesgue分解により,任意の2つの$\sigma$-有限測度$\mu,\nu$について,一方をもう一方に対して絶対連続部分と特異部分との和に分解できる.
    \end{enumerate}
\end{definition}

\begin{theorem}[Cameron-Martin]
    任意の連続関数$F\in C[0,1],F(0)=0$について,次が成り立つ:
    \begin{enumerate}
        \item $F\notin D[0,1]$ならば,$L_F\perp L_0$である.
        \item $F\in D[0,1]$ならば,$L_F$と$L_0$は同値である.
    \end{enumerate}
\end{theorem}
\begin{remarks}
    可微分なドリフト$F$については,Brown運動の法則は同値になる.
    すなわち,ほとんど確実な事象(見本道の連続性や可微分性)が等しい.
\end{remarks}

\subsubsection{証明}

\begin{tcolorbox}[colframe=ForestGreen, colback=ForestGreen!10!white,breakable,colbacktitle=ForestGreen!40!white,coltitle=black,fonttitle=\bfseries\sffamily,
title=]
    証明にはmartingaleを用いる.
\end{tcolorbox}

\begin{notation}
    $F\in C[0,1],n>0$について,2進小数点$\D_n$で分割して考えた,
    $F$の二次変分
    \[Q(F):=\lim_{\abs{\Delta}\to0}\sum_{k=1}^n\paren{F(t_k)-F(t_{k-1})}^2\]
    に至る列を
    \[Q_n(F):=2^n\sum^{2^n}_{j=1}\Square{F\paren{\frac{j}{2^n}}-F\paren{\frac{j-1}{2^n}}}^2\]
    と表す.
\end{notation}

\begin{lemma}
    $F\in C[0,1],F(0)=0$について,
    \begin{enumerate}
        \item $(Q_n(F))_{n\in\N}$は増加列である.
        \item $F\in D[0,1]\Leftrightarrow\sup_{n\in\N}Q_n(F)<\infty$.
        \item $F\in D[0,1]\Rightarrow Q_n(F)\xrightarrow{n\to\infty}\norm{F'}^2_2$.
    \end{enumerate}
\end{lemma}
\begin{Proof}\mbox{}
    \begin{enumerate}
        \item $(a+b)^2=a^2+2ab+b^2\le 2a^2+2b^2$より,任意の$j\in[2^n]$について,
        \begin{align*}
            \Square{F\paren{\frac{j}{2^n}}-F\paren{\frac{j-1}{2^n}}}^2&=\Square{\paren{F\paren{\frac{2j-1}{2^{n+1}}}-F\paren{\frac{j-1}{2^n}}}+\paren{F\paren{\frac{j}{2^n}}-F\paren{\frac{2j-1}{2^{n+1}}}}}^2\\
            &\le2\Square{F\paren{\frac{2j-1}{2^{n+1}}}-F\paren{\frac{j-1}{2^n}}}^2+2\Square{F\paren{\frac{j}{2^n}}-F\paren{\frac{2j-1}{2^{n+1}}}}^2
        \end{align*}
        これを$j\in[2^n]$について和を取ると,
        \[\sum^{2^n}_{j=1}\Square{F\paren{\frac{j}{2^n}}-F\paren{\frac{j-1}{2^n}}}^2\le 2\sum^{2^{n+1}}_{j=1}\Square{F\paren{\frac{j}{2^{n+1}}}-F\paren{\frac{j-1}{2^{n+1}}}}^2\]
        より,$(Q_n(F))_{n\in\N}$の単調増加性を得る.
        \item \begin{description}
            \item[$\Rightarrow$] $F\in D[0,1]$のとき,ある$f:=F'\in L^2[0,1]$が存在して,Cauchy-Schwarzより任意の$j\in[2^n]$について
            \[2^n\paren{\int^{j2^{-n}}_{(j-1)2^{-n}}fdt}^2\le 2^n\int^{j2^{-n}}_{(j-1)2^{-n}}1^2dt\int^{j2^{-n}}_{(j-1)2^{-n}}f^2dt=\int^{j2^{-n}}_{(j-1)2^{-n}}f^2dt\]
            が成り立つから,
            \[Q_n(F)=2^n\sum^{2^n}_{j=1}\paren{\int^{j2^{-n}}_{(j-1)2^{-n}}fdt}^2\le \sum^{2^n}_{j=1}\int^{j2^{-n}}_{(j-1)2^{-n}}f^2dt=\norm{f}_2^2<\infty.\]
            \item[$\Leftarrow$]

        \end{description}
    \end{enumerate}
\end{Proof}

\begin{lemma}[Paley-Wiener stochastic integral]
    $(B_t)_{t\in\R_+}$を標準Brown運動とする.$F\in D[0,1]$について,
    \[\xi_n:=2^n\sum^{2^n}_{j=1}\Square{F\paren{\frac{j}{2^n}}-F\paren{\frac{j-1}{2}}}\Square{B\paren{\frac{j}{2^n}}-B\paren{\frac{j-1}{2^n}}}\]
    はほとんど確実に$L^2$-収束する.
    この極限を$\int^1_0F'dB$で表す.
\end{lemma}

\subsection{抽象的Wiener空間}

\begin{tcolorbox}[colframe=ForestGreen, colback=ForestGreen!10!white,breakable,colbacktitle=ForestGreen!40!white,coltitle=black,fonttitle=\bfseries\sffamily,
title=]
    Leonard Gross 31- により,一般のGauss測度を調べるために開発され,見本道の空間$\Om$よりもCameron-Martin空間を主役に置く.
    可分Hilbert空間$H$上の経路積分など,うまくGauss測度が定義できないことがあるが,これは$H$を稠密部分空間として含むBanach空間$B$を得て,その上での積分として理解できる.
    この組$(H,B)$を抽象Wiener空間という.
    $B$上にGauss測度が存在するかやその性質など,多くの部分は$H$で決まる.

    Gauss測度の構造定理によると,任意のGauss測度はある抽象Wiener空間上に実現される.
\end{tcolorbox}

\begin{definition}[Gross (67)]
    $B$を可分Banach空間,$i:H\mono B$を可分Hilbert空間$H$の連続かつ稠密な埋め込み,$\mu$を$B$上のGauss測度で
    \[\forall_{\varphi\in B^*\subset H^*}\;\int_Be^{i\brac{w|\varphi}}d\mu(w)=e^{-\frac{1}{2}\abs{\varphi}^2_{H^2}}.\]
    を満たすものとする.この3-組$(B,H,\mu)$を\textbf{抽象Wiener空間}という.
    $H$をそのCameron-Martin空間という.
\end{definition}
\begin{remarks}
    実は上は可分Banach空間上のGauss測度が必ず満たす性質である.
    また$B$は完備可分だから$\mu$は緊密で,あるコンパクトなBanach空間$B_1\mono B$について$\mu(B_1)=1$を満たすことをGrossは示した.
    このような$B_1$はある程度の任意性がある.
\end{remarks}

\begin{example}[古典的Wiener空間]
    $C^0([0,T];\R^d)$はWiener測度$\mu$についてWiener空間である.このとき,Cameron-Martin空間は
    \[H:=\Brace{w\in C^0([0,T];\R^d)\mid w\ll\mu\land dw/d\mu\in L^2(\mu)}.\]
    逆に,
    $H$をホワイトノイズ全体の集合,すなわち,$b'(0)=0$を満たす$L^2$-導関数$b'$を持つ実関数$b:[0,T]\to\R$にノルム$\norm{b}:=\int_{[0,T]}b'^2(t)dt$を入れて得るHilbert空間とする:
    \[H:=L^{2,1}_0(T;\R^n):=\Brace{b:[0,T]\to\R\mid 0\text{から始まる見本道}b\text{は絶対連続で微分}b'\text{は}L^2\text{-可積分である}}\]
    このとき,$B:=\Brace{W\in C(T)\mid W(0)=0}$を見本道の空間とすれば,$(H,B)$は古典的Wiener空間である.
    $B$上のGauss測度はBrown運動の法則に一致し,Cameron-Matrin部分空間$H$を零集合に持つ.
    実際,$H$の元であるような見本道(特に微分可能である)は,Brown運動の見本道としては出現しない.
    逆に言えば,$H$上にうまく積分を定義できれば,$B$上に連続延長するかもしれない.
    これは測度論・函数解析と確率論の奇妙な融合である!
\end{example}

\begin{discussion}
    $H$を可分Hilbert空間とする.$H$上の柱状集合を,有限個の有界線型汎函数$\varphi_1,\cdots,\varphi_n\in H^*$を用いて,
    \[\Brace{h\in H\mid\varphi_1(h)\in I_1,\cdots,\varphi_n(h)\in I_n}\;(I_1,\cdots,I_n\in\B^1(\R))\]
    と表せる集合の全体$\cC$とする.これはある$\sigma$-代数$\sigma[\cC]$を生成する.
    $(H,\cC)$上にGaussな有限加法的測度は上述の方法で定まるが(これを\textbf{柱状集合Gauss測度}という.),これは$\sigma[\cC]$上に完全加法的な延長を持たない.

    ここで,可分Banach空間$B$で,連続線型単射$i:H\mono B$で像が$B$上稠密であるようなものが存在するとする.
    $B$の柱状集合$C$を,$i^*B\subset H$が柱状集合であることとして定義する.
    Gross (67)は,$B$の柱状集合体$\cC$上のGaussな有限加法的測度が完全加法的な延長を持つための必要十分条件を導いた.
    これは$B$が$H$に比べて「十分に大きい」ことと理解できる.
    このとき,$H$は零集合になる($H$にGauss測度が存在しないことと整合的).
\end{discussion}

\begin{theorem}[Structure theorem for Gaussian measures (Kallianpur-Sato-Stefan 1969 and Dudley-Feldman-le Cam 1971)]\label{thm-Structure-theorem-for-Gaussian-measures}
    任意の可分Banach空間$B$上の非退化なGauss測度$\mu$とする.このとき,抽象Wiener空間$i:H\mono B$が存在して,$\mu$は$H$上の柱状集合Gauss測度の押し出しによって得られる.
\end{theorem}
\begin{remarks}
    このときの$B$上のGauss測度$\mu$は,任意の汎関数$f\in B^*$に対して,$f_*\mu$は1次元正規分布である.
\end{remarks}



\begin{tcolorbox}[colframe=ForestGreen, colback=ForestGreen!10!white,breakable,colbacktitle=ForestGreen!40!white,coltitle=black,fonttitle=\bfseries\sffamily,
title=]
    こうして$(B,H,\mu)$上の積分法は古典論の延長に思える.
    問題は微分法である.
    Malliavinのアイデアの斬新さは,このWiener空間上の解析にOrnstein-Uhlenbeck作用素を利用したことにある.
\end{tcolorbox}

\subsection{Paley-Wiener積分}

\begin{theorem}[Fernique]
    $\mu$を可分Banach空間上のGauss測度とする.このとき,ある定数$\al\in\R$が存在して,
    \[\int_Be^{\al\norm{x}^2}\mu(dx)<\infty.\]
\end{theorem}

\begin{corollary}\mbox{}
    \begin{enumerate}
        \item 任意の$\varphi\in\B^*$は二乗可積分である.この対応を$I_1:B^*\to L^2(B,\mu)$とする.
        \item $I_1$の$H^*$上への延長$H^*\to L^2(B,\mu)$は再び等長写像である.これを\textbf{1次のWiener積分}という.
    \end{enumerate}
\end{corollary}
\begin{remarks}
    多次元のWiener積分をWiener chaosという.
\end{remarks}

\begin{theorem}[Gauss測度の平行移動]
    $h\in H$について,次は同値:
    \begin{enumerate}
        \item $\mu(\bullet-h)$と$\mu$は互いに絶対連続.
        \item $h\in H$.
    \end{enumerate}
    このとき,次が成り立つ:
    \[\dd{\mu(\bullet-h)}{\mu}(x)=\exp\paren{-\frac{1}{2}\abs{h}^2_H+I_1(\wh{h})(x)},\quad x\in B.\]
    ただし,$\wh{h}\in H^*$は$h\in H$のRiesz表現とした.
    また,$h\notin H$ならば,互いに特異になる.
\end{theorem}


\section{Brown運動のMarkov性}

\subsection{Markov過程の定義}

\begin{definition}[Markov process]
    $(\Om,\F,\bF,P)$を確率基底とする.
    適合過程$X\in\bF$が\textbf{Markov過程}であるとは,次を満たすことをいう:
    \[\forall_{s\in\R_+,t\in\R^+}\;\forall_{f\in L^\infty(\R)}\;E[f(X_{s+t})|\F_s]=E[f(X_{s+t})|X_s]\;\as\]
    なお,$L^\infty(\R)$は$L(\R)_+$としても良い.
\end{definition}
\begin{remarks}[Markov過程の特徴付け]
    Markov過程の有限周辺分布は,起点$X_0$と遷移確率$(p_{s,t})$によって定まる.
\end{remarks}

\begin{definition}[transition operator]
    $X$を時間的に一様なMarkov過程とする.
    \begin{enumerate}
        \item 時刻$0$からの\textbf{遷移作用素}
        $\{P_t\}_{t\in\R_+}\subset B(L^\infty(\R))$とは,遷移確率との積分を取る対応
        \[P_t[f](x):=\int_\R p(0,x,t;dy)f(y)\]
        をいう.遷移確率の定義から,
        遷移作用素は半群をなす.
        \item 生成作用素$A$とは,一様ノルムに関する極限によって定まる対応
        \[Af:=\lim_{h\searrow0}\frac{P_{h}[f]-f}{h}\]
        をいう.時間的に一様でない場合は,これは$(A_s)_{s\in\R_+}$となる.
    \end{enumerate}
\end{definition}

\begin{remarks}
    遷移作用素$(P_{s,t})$の言葉を使えば,Markov性は
    \[\forall_{f\in L^\infty(\R)}\quad E[f(X_t)|\F_s]=P_{s,t}f(X_s)\quad Q\das\]
    と表せる.
    さらに$P_{s,t}$が$P_{\abs{t-s}}$で表せるとき,時間的に一様であるという.
\end{remarks}

\subsection{Brown運動のMarkov性}

\begin{theorem}[Markov property]
    Brown運動$(B_t)_{t\in\R_+}$は自然な情報系$(\F_t)$についてMarkov過程となる.
    すなわち,
    \[(P_tf)(x)=\int_\R f(y)p_t(x-y)dy\;(t>0)\]
    で,時刻$s$に$x$に居た場合の$f(B_{s+t})$の値の平均を返す関数$P_tf:\R\to\R$への対応$P_t:L_b(\R)\to\L(\R)$
    のなす1-パラメータ連続半群$(P_t)_{t\in\R_+}$を用いて,
    \[\forall_{f\in\L(\R)\cap l^\infty(\R)}\;\forall_{s\ge0,t>0}\;E[f(B_{s+t})|\F_s]=(P_tf)(B_s).\]
    また,$f\in\L(\R)_+$としてもよい.
\end{theorem}
\begin{Proof}
    補題\ref{lemma-Blumenthal}(1)より
    $\F_s\indep B_{s+t}-B_s$だから,$\forall_{x\in\R}\;f(B_{s+t}-B_s+x)\indep\F_s$だから,
    \begin{align*}
        E[f(B_{s+t})|\F_s]&=E[f(B_{s+t}-B_s+B_s)|\F_s]\\
        &=E[f(B_{s+t}-B_s+x)|\F_s]|_{x=B_s}\\
        &=E[f(B_{s+t}-B_s+x)]|_{x=B_s}\\
        &=\int_\R f(y+x)\frac{1}{\sqrt{2\pi t}}e^{-\frac{y^2}{2t}}dy|_{x=B_s}\\
        &=\int_\R f(y+B_s)\frac{1}{\sqrt{2\pi t}}e^{-\frac{y^2}{2t}}dy\\
        &=\int_\R f(y)\frac{1}{\sqrt{2\pi t}}e^{-\frac{(B_s-y)^2}{2t}}dy=(P_tf)(B_s).
    \end{align*}
\end{Proof}

\subsection{Brown運動の遷移確率}

\begin{tcolorbox}[colframe=ForestGreen, colback=ForestGreen!10!white,breakable,colbacktitle=ForestGreen!40!white,coltitle=black,fonttitle=\bfseries\sffamily,
title=]
    Brown運動の遷移確率は熱核である.
\end{tcolorbox}


\begin{proposition}
    $d$-次元Brown運動の遷移確率のなす
    連続半群$(P_t)_{t\in\R_+}$を定める転移確率
    \[p_t(x-y)=\frac{1}{(2\pi t)^{d/2}}\exp\paren{-\frac{\abs{x-y}^2}{2t}}\]
    について,
    \begin{enumerate}
        \item 熱方程式の初期値問題
        \[\pp{p}{t}=\frac{1}{2}\Laplace p\;(t>0)\qquad p_0(x-y)=\delta_0(x-y)\]
        の解である.
        \item 任意の$f\in C^{1,2}(\R^+\times\R^d)\cap C(\R_+\times\R^d)$であって
        \[\abs{f(t,x)}+\Abs{\pp{f(t,x)}{t}}+\sum_{j\in[d]}\Abs{\pp{f(t,x)}{x_j}}+\sum_{j,k\in[d]}\Abs{\pp{^2f(t,x)}{x_j\partial x_k}}\le c(t)e^{C\abs{x}},\quad t>0,x\in\R^d,C>0,c:\R^+\to\R_+\text{は局所有界}\]
        を満たすものに対して,
        \[\int_{\R^d}p(t,x)\frac{1}{2}\Lap_xf(t,x)dx=\int_{\R^d}f(t,x)\frac{1}{2}\Lap_xp(t,x)dx=\int_{\R^d}f(t,x)\pp{}{t}p(t,x)dx.\]
    \end{enumerate}
\end{proposition}

\begin{corollary}[Brown運動の遷移確率]
    定理に登場した
    $P_t\varphi(x):=E[\varphi(B_t+x)]\;(\varphi\in \L_b(\R))$
    で定まる転移作用素$\{P_t\}_{t\in\R_+}\subset\Map(\L_b(\R),\L(\R))$は半群をなす:
    \begin{enumerate}
        \item $P_0=\id$.
        \item $P_t\circ P_s=P_{t+s}$,すなわち,
        \[\forall_{f\in L_b(\R)}\;\iint_{\R\times\R} f(y)p_s(x-y)p_t(z-x)dydx=\int_\R f(y)p_{s+t}(z-y)dy.\]
    \end{enumerate}
    また,$P_t$は正作用素である:$\varphi\ge0\Rightarrow P_t\varphi\ge0$.
\end{corollary}

\section{Brown運動に付随するマルチンゲール}

\subsection{代表的なマルチンゲール}

\begin{theorem}
    次の過程は$(\F_t)$-マルチンゲールである.
    \begin{enumerate}
        \item Brown運動:$(B_t)_{t\in\R_+}$.
        \item 中心化された分散の過程:$(B^2_t-t)_{t\in\R_+}$.
        \item 幾何Brown運動:$(\exp(aB_t-a^2t/2))_{t\in\R_+}\;(a\in\R)$.
        \item 複素幾何Brown運動:$(\exp(iaB_t+a^2t/2))_{t\in\R_+}\;(a\in\R)$.
    \end{enumerate}
\end{theorem}
\begin{Proof}\mbox{}
    \begin{enumerate}
        \item $\forall_{0\le s<t}\;B_t-B_s\indep\F_s$補題\ref{lemma-Blumenthal}(1)より,$E[B_t-B_s|\F_s]=E[B_t-B_s]=0$.
        \item 任意の$0\le s<t$について,
        \begin{align*}
            E[B^2_t|\F_t]&=E[(B_t-B_s+B_s)^2|\F_s]\\
            &=E[(B_t-B_s)^2|\F_s]+2E[(B_t-B_s)B_s|\F_s]+E[B^2_s|\F_s]\\
            &=E[(B_t-B_s)^2]+2B_sE[B_t-B_s|\F_s]+B^2_s\\
            &=t-s+B_2^s.
        \end{align*}
        \item 任意の$0\le s<t$について,$B_t-B_s\sim N(0,t-s)$なので,
        \begin{align*}
            E\Square{\exp\paren{aB_t-\frac{a^2t}{2}}\middle|\F_s}&=\exp(aB_s)E\Square{\exp\paren{a(B_t-B_s)-\frac{a^2t}{2}}}\\
            &=\exp\paren{aB_s-\frac{a^2t}{2}}\frac{1}{\sqrt{2\pi(t-s)}}\int_\R e^{ax}e^{-\frac{x^2}{2(t-s)}}dx\\
            &=\exp\paren{aB_s-\frac{a^2t}{2}}e^{\frac{(t-s)^2a^2}{2(t-s)}}\frac{1}{\sqrt{2\pi(t-s)}}\int_\R e^{-\frac{(x-(t-s)a)^2}{2(t-s)}}dx\\
            &=\exp\paren{aB_s-\frac{sa^2}{2}}.
        \end{align*}
        \item 任意の$0\le s<t$について,$B_t-B_s\sim N(0,t-s)$の特性関数より,
        \begin{align*}
            E\Square{\exp\paren{iaB_t+\frac{a^2t}{2}}\middle|\F_s}&=\exp\paren{iaB_s+\frac{a^2t}{2}}E\Square{\exp\paren{ia(B_t-B_s)}}\\
            &=\exp\paren{iaB_s+\frac{a^2t}{2}}\exp\paren{-\frac{1}{2}a^2(t-s)}=\exp\paren{iaB_s+\frac{a^2s}{2}}.
        \end{align*}
    \end{enumerate}
\end{Proof}
\begin{remarks}
    (3)は正規分布の積率母関数とも,対数正規分布の確率密度関数の一部とも見れる.
\end{remarks}

\subsection{到達時刻}

\begin{tcolorbox}[colframe=ForestGreen, colback=ForestGreen!10!white,breakable,colbacktitle=ForestGreen!40!white,coltitle=black,fonttitle=\bfseries\sffamily,
title=Markov過程の到達時刻はLaplace変換の有名な応用先である]
    Brown運動とそれに付随するマルチンゲールを用いて,到達時刻$\tau_a$のLaplace変換を得ることが出来る.
    これは積率母関数を得たことになるから,Laplace逆変換によって$\tau_a$の分布が分かる.
\end{tcolorbox}

\begin{lemma}[到達時刻]
    $a>0$に対して
    $\tau_a:=\inf\Brace{t\in\R_+\mid B_t=a}$とすると,これは$(\F_t)$-停止時である.
    \footnote{ただし,$\Brace{t\in\R_+\mid B_t=a}=\emptyset$のとき,$\tau_a=\infty$とするのはinfの定義である.}
\end{lemma}
\begin{Proof}
    \begin{align*}
        \Brace{\tau_a>t}&=\cap_{s\in[0,t]}\Brace{B_s<a}\\
        &=\cup_{k\in\N}\cap_{s\in[0,t]}\Brace{B_s\le a-\frac{1}{k}}\overset{\as}{=}\cup_{k\in\N}\cap_{s\in[0,t]\cap\Q}\Brace{B_s\le a-\frac{1}{k}}\in\F_t.
    \end{align*}
    最後の等号は,$\subset$は明らかであるが,$\supset$は,殆ど至る所の$\om\in\Om$に対して$s\mapsto B_s(\om)$が連続であるため,$\exists_{k\in\N}\;\forall_{s\in[0,t]\cap\Q}\;B_s\le a-\frac{1}{k}$は$[0,t]$上でも同様の条件が成り立つことを含意する.
    よって,$\Brace{\tau_a\le t}\in\F_t$もわかる.
\end{Proof}
\begin{remarks}
    $\Brace{\tau_a<t}$を示しても十分であるが,これは情報系$(\F_t)$が右連続であることに依る.
\end{remarks}

\begin{proposition}[Laplace transformation of Brownian hitting time]\label{prop-hitting-time}
    任意の$a>0$について,
    \[\forall_{\al>0}\;E[e^{-\al\tau_a}]=e^{-\sqrt{2\al}a}.\]
\end{proposition}
\begin{Proof}
    幾何Brown運動のmartingale性に注目することで,これだけから出てくる.
    \begin{description}
        \item[martingaleの用意] \mbox{}\\
        任意の$\lambda>0$について,$M_t:=e^{\lambda B_t-\lambda^2t/2}$はmartingaleである.\footnote{$\lambda=0$のときは$M_t=1$となり,この場合は不適.考えない.}
        よって特に$\forall_{t\in\R_+}\;E[M_t]=E[M_0]=1$.
        \item[この幾何Brown運動についての任意抽出] \mbox{}\\
        Doobの任意抽出定理より,任意の$N\ge 1$について,$E[M_{\tau_a\land N}|\F_0]=M_0$.両辺の期待値を取って,$E[M_{\tau_a\land N}]=1$.
        \begin{enumerate}[(a)]
            \item (関数列の優関数)  $\forall_{N\in\N}\;M_{\tau_a\land N}=\exp\paren{\lambda B_{\tau_a\land N}-\lambda^2(\tau_a\land N)/2}\le e^{a\lambda}$より,$N\to\infty$の極限について優収束定理が使える.
            \item (関数列の極限) $e^{t\paren{\lambda\frac{B_t}{t}-\frac{\lambda^2}{2}}}\xrightarrow{t\to\infty}e^{-\infty}=0$を示す.
            いま,大数の法則より\[\lambda\frac{B_t}{t}-\frac{\lambda^2}{2}\xrightarrow{t\to\infty}\frac{-\lambda^2}{2}<0\;(\lambda>0\text{のとき})\]だから,
            十分大きな$T>0$について,$\forall_{t\ge T}\;\lambda\frac{B_t}{t}-\frac{\lambda^2}{2}=:-\ep<0$より,$0\le M_t=e^{t\paren{\lambda\frac{B_t}{t}-\frac{\lambda^2}{2}}}\le e^{-\ep T}$.
            
            $T\to\infty$の極限を考えると,$M_t\xrightarrow{t\to\infty}0$.
            よって,
            \[\lim_{N\to\infty}M_{\tau_a\land N}=\begin{cases}
                M_{\tau_a}&\tau_a<\infty,\\
                0&\tau_a=\infty.
            \end{cases}\]
        \end{enumerate}
        よって優収束定理から,結局$E[M_{\tau_a}]=1$とわかる.特に,
        $E[1_{\Brace{\tau_a<\infty}}M_{t_a}]=1-E[1_{\Brace{\tau_a=\infty}}M_{t_a}]=1$.
        すなわち,
        \[E\Square{1_{\Brace{\tau_a<\infty}}\exp\paren{-\frac{\lambda^2\tau_a}{2}}}=e^{-\lambda a}.\]
        あとは$1_{\Brace{\tau_a<\infty}}$の部分を外せば良い.
        \item[到達時刻はほとんど確実に有限] \mbox{}
        $\lambda>0$は任意とした.$\forall_{\lambda>0}\;1_{\Brace{\tau_a<\infty}}\exp\paren{-\frac{\lambda^2\tau_a}{2}}\le1_{\Brace{\tau_a<\infty}}$より可積分な上界が存在する.
        $\lambda\to0$に極限について,単調収束定理より,$E[1_{\Brace{\tau_a<\infty}}]=P[\tau_a<\infty]=1$.
        よって,
        \[E\Square{\exp\paren{-\frac{\lambda^2\tau_a}{2}}}=e^{-\lambda a}.\]
        $\al=\lambda^2/2$の場合を考えれば良い.
    \end{description}
\end{Proof}
\begin{remarks}
    なお,$\tau_a$の確率密度関数を$f:\R_+\to[0,1]$とすると,
    \[E[e^{-\al\tau_a}]=\int_{\R_+}e^{-\al s}f(s)ds=\L[f](\al)\]
    とみれる.
    また$M_t\xrightarrow{t\to\infty}0$はマルチンゲール収束定理からも導ける.
\end{remarks}

\begin{corollary}[到達時刻の分布]\mbox{}\label{cor-density-of-Brownian-hitting-time}
    \begin{enumerate}
        \item Laplace逆変換より,Brown運動の到達時刻の分布関数は,
        \[P[\tau_a\le t]=\int^t_0\frac{ae^{-a^2/(2s)}}{\sqrt{2\pi s^3}}ds.\]
        \item $E[\tau_ae^{-\al\tau_a}]=\frac{ae^{-\sqrt{2\al}a}}{\sqrt{2\al}}$.
        \item $E[\tau_a]=+\infty$.
    \end{enumerate}
\end{corollary}
\begin{Proof}\mbox{}
    \begin{enumerate}
        \item Laplace逆変換による.
        \item 両辺の$\al$に関する微分による.$e^{-\al\tau_a}$は$\om$に関して可積分であるのはもちろん,$\al$についても可微分で,またその導関数$-\tau_ae^{-\al\tau_a}$も$\om$に関して可積分だから,微分と積分は交換して良い.
        \item $\al\to0$の極限を考えることによる(単調収束定理).
    \end{enumerate}
\end{Proof}
\begin{remarks}
    $\tau_a$の平均は存在しない(発散する)が,殆ど確実に有限になるという不可解な消息がある.
    また,(1)は解析的な方法に依らず,鏡像の原理からも導ける.
\end{remarks}

\subsection{脱出時刻}

\begin{tcolorbox}[colframe=ForestGreen, colback=ForestGreen!10!white,breakable,colbacktitle=ForestGreen!40!white,coltitle=black,fonttitle=\bfseries\sffamily,
title=]
    脱出時刻は到達時刻が定める.
\end{tcolorbox}

\begin{proposition}[どっちの端から脱出するかの確率]
    $a<0<b$のとき,
    \[P[\tau_a<\tau_b]=\frac{b}{b-a}.\]
\end{proposition}
\begin{Proof}\mbox{}
    \begin{description}
        \item[任意抽出定理] Doobの任意抽出定理より,$\forall_{t\in\R_+}\;E[B_{t\land\tau_a\land\tau_b}]=E[B_0]=0\;\as$
        $\forall_{t\in\R_+}\;B_{t\land\tau_a\land\tau_b}\in[a,b]$より,優収束定理から$E[B_{\tau_a\land\tau_b}]=0$を得る.

        ここで到達時刻は$\tau_a,\tau_b<\infty\;\as$を満たすから,$\Om=\Brace{\tau_a<\tau_b}\sqcup\Brace{\tau_b<\tau_a}\sqcup\Brace{\tau_b=\tau_a<\infty}\sqcup\Brace{\tau_a\land\tau_b=\infty}$の直和分解のうち最後の2項は零集合だから,
        \[0=P[\tau_a<\tau_b]a+(1-P[\tau_a<\tau_b])b.\]
    \end{description}
\end{Proof}

\begin{proposition}[脱出時刻の期待値]\label{prop-mean-of-Brownian-exit-time}
    $a<0<b$のとき,脱出時刻を
    $T:=\inf\Brace{t\in\R_+\mid B_t\notin(a,b)}$とすると$E[T]=-ab$.
\end{proposition}
\begin{Proof}
    $B_t^2-t$はmartingaleであるから,任意抽出定理より$\forall_{t\in\R_+}\;E[B^2_{T\land t}]=E[T\land t]$.
    $t\to\infty$の極限を考えて,
    \begin{align*}
        E[T]&=E[B_T^2]=a^2\frac{b}{b-a}+b^2\frac{a}{a-b}\\
        &=\frac{ab(a-b)}{b-a}=-ab.
    \end{align*}
\end{Proof}

\section{Brown運動の強Markov性}

\begin{tcolorbox}[colframe=ForestGreen, colback=ForestGreen!10!white,breakable,colbacktitle=ForestGreen!40!white,coltitle=black,fonttitle=\bfseries\sffamily,
title=]
    一般のランダムな時刻$T\in\T^{<\infty}$に対しても,その後の運動$(B_{T+t}-B_T)_{t\in\R_+}$はそれまでの歴史$\F_T$に依らない:$E[f(B_{T+t})|\F_T]=(P_tf)(B_T)$.
\end{tcolorbox}

\subsection{強Markov性の証明}

\begin{theorem}[Hunt (1956) and Dynkin and Yushkevich (1956)]\mbox{}\\
    $T\in\bT^{<\infty}(\Om,(\F_t))$を停止時とする.このとき,$(\wt{B}_t:=B_{T+t}-B_T)_{t\in\R_+}$は再びBrown運動で,$\F_T$と独立である.
\end{theorem}
\begin{Proof}\mbox{}
    \begin{description}
        \item[$T$が有界なとき]
        任意の$\lambda\in\R$について,$M_t:=\exp\paren{i\lambda B_t+\frac{\lambda^2t}{2}}$はmartingaleである.
        よってDoobの任意抽出定理より,
        $\forall_{0\le s\le t}\;E[M_{T+t}|\F_{T+s}]=E[M_{T+s}]$から,$\frac{\lambda^2(T+t)}{2},i\lambda B_{T+s}$が$\F_{T+s}$-可測であることに注意すると,
        \begin{align*}
            E\Square{\exp\paren{i\lambda B_{T+t}+\frac{\lambda^2(T+t)}{2}}\middle|\F_{T+s}}&=E\Square{\exp\paren{i\lambda B_{T+s}+\frac{\lambda^2(T+s)}{2}}}\\
            E\Square{\exp\paren{i\lambda(\underbrace{B_{T+t}-B_{T+s}}_{=\wt{B}_t-\wt{B}_s})}\middle|\F_{T+s}}&=E\Square{\exp\paren{-\frac{\lambda^2}{2}(t-s)}}=\exp\paren{-\frac{\lambda^2}{2}(t-s)}.
        \end{align*}
        これは,
        まず任意の$0\le s\le t$について$\wt{B_t}-\wt{B_s}\sim N(0,t-s)$であることを表しており,さらに任意の$0\le t_1\le t_2\le t_3$について,$B_{t_3}-B_{t_2}\indep B_{t_2}-B_{t_1}$も含意している.
        $B_{T+0}-B_T=0$と,$(B_t)_{t\in\R_+}$から受け継いだ連続性とを併せるとBrown運動であることが分かる.
        \item[$T$が一般のとき] \ref{prop-mean-of-Brownian-exit-time}と同様に処理すれば良い.
        
        任意の$N\in\N$に対して,$T\land N$は有界な停止時だから,$\forall_{0\le s\le t}\;T\land N+s\le T\land N+t\le N+t$も有界.
        よって,Doobの任意抽出定理より,
        \[\forall_{0\le s\le t}\quad E\Square{M_{T\land N+t}|\F_{T\land N+s}}=M_{T\land N+s}.\]
        すなわち,
        \[\forall_{N\in\N}\;\forall_{0\le s\le t}\quad E\Square{\exp\paren{i\lambda\paren{B_{T\land N+t}-B_{T\land N+s}}}\;\middle|\;\F_{T\land N+s}}=\exp\paren{-\frac{\lambda^2}{2}(t-s)}.\]

        ここで,
        \[E\Square{\exp\paren{i\lambda(B_{T+t-B_{T+s}})}|\F_{T+s}}=\exp\paren{-\frac{\lambda^2}{2}(t-s)}\]
        を示したい.右辺は明らかに$\F_{T+s}$-可測で可積分であるから,任意の$A\in\F_{T+s}$に対して
        \[E\Square{\exp\paren{-\frac{\lambda^2}{2}(t-s)}1_A}=E\Square{E\Square{\exp\paren{i\lambda(B_{T+t}-B_{T+s})}|\F_{T+s}}1_{A}}\]
        を示せば良い.いま,$A\in\F_{T+s}$より,$A\cap\Brace{T\le N}\in\F_{N+s}$である.特に,$A\cap\Brace{T\le N}\in\F_{T\land N+s}$.
        よって,右辺は
        \begin{align*}
            &E\Square{E\Square{\exp\paren{i\lambda(B_{T+t}-B_{T+s})}|\F_{T+s}}1_{A}}\\
            &=E\Square{E\Square{\exp\paren{i\lambda(B_{T+t}-B_{T+s})}|\F_{T+s}}1_{A\cap\Brace{T\le N}}}+E\Square{E\Square{\exp\paren{i\lambda(B_{T+t}-B_{T+s})}|\F_{T+s}}1_{A\cap\Brace{T>N}}}\\
            &=E\Square{\exp\paren{-\frac{\lambda^2}{2}(t-s)}1_{A\cap\Brace{T\le N}}}+E\Square{E\Square{\exp\paren{i\lambda(B_{T+t}-B_{T+s})}|\F_{T+s}}1_{A\cap\Brace{T>N}}}.
        \end{align*}
        ここで$N\to\infty$の極限を考えると,第1項は単調収束定理,第2項は優収束定理により,総じて右辺は
        \[\exp\paren{-\frac{\lambda^2}{2}(t-s)}\]
        に収束する.
    \end{description}
\end{Proof}

\subsection{反射原理}

\begin{tcolorbox}[colframe=ForestGreen, colback=ForestGreen!10!white,breakable,colbacktitle=ForestGreen!40!white,coltitle=black,fonttitle=\bfseries\sffamily,
title=]
    確率過程が定めるsupの過程の尾部確率の評価は,
    Kolmogorovの不等式から始まる不等式原理である.
    Brown運動については,元のBrown運動のことばで完全に特徴付けられる.

    この結果は,到達時刻がもたらす強Markov性から従う.
\end{tcolorbox}

\begin{lemma}[胸像の原理]
    時刻$\tau_a<\infty$で折り返した確率過程を
    \[\wh{B}_t:=B_t1_{\Brace{t\le\tau_a}}+(2a-B_t)1_{\Brace{t>\tau_a}}\]
    と定める.
    これは再びBrown運動である.
\end{lemma}
\begin{Proof}
    いま,到達時刻$\tau_a$は有限な停止時だから,$(B_{t+\tau_a}-a)_{t\in\R_+},(-B_{t+\tau_a}+a)_{t\in\R_+}$はいずれもBrown運動であり,$B_{\tau_a}$と独立.
        
        したがって,$(B_t)_{t\in[0,\tau_a]}$の右端に接続した過程は,いずれも同じ過程を定める.1つ目はBrown運動$(B_t)$で,2つ目が$(\wt{B}_t)$である.
        したがって,$(\wh{B}_t)$もBrown運動である.
\end{Proof}

\begin{theorem}[Levy's reflection princple (1939)]
    $M_t:=\sup_{s\in[0,t]}B_s$とする.
    \[\forall_{a>0}\;P[M_t\ge a]=2P[B_t>a].\]
\end{theorem}
\begin{Proof}
    $\{M_t\ge a\}=\Brace{B_t>a}\sqcup\Brace{M_t\ge a,B_t\le a}$と場合分けすると,2つ目の場合は$\Brace{\wt{B}_t\ge a}$と同値.
        よって,
        \[P[M_t\ge a]=P[B_t>a]+P[\wt{B}_t\ge a]=2P[B_t>a].\]
\end{Proof}

\begin{corollary}\label{cor-density-of-sup-process-of-Brownian-motion}
    任意の$a>0$に対して,$M_a:=\sup_{t\in[0,a]}B_t$は次の密度関数を持つ:
    \[p(x)=\frac{2}{\sqrt{2\pi a}}e^{-x^2/(2a)}1_{[0,\infty)}(x).\]
\end{corollary}

\subsection{Brown運動の最大値}

\begin{tcolorbox}[colframe=ForestGreen, colback=ForestGreen!10!white,breakable,colbacktitle=ForestGreen!40!white,coltitle=black,fonttitle=\bfseries\sffamily,
title=]
    広義の$C$-過程であるから,殆ど確実に$[0,1]$上で最大値を取る.
    実はさらに,その点は一意的である!
\end{tcolorbox}

\begin{lemma}
    殆ど確実に,Brown運動$(B_t)_{t\in[0,1]}$ただ一つの点にて最大値を取る.
\end{lemma}
\begin{Proof}\mbox{}
    \begin{description}
        \item[方針] $[0,1]$の分割
        \[\I_n:=\Brace{\Square{\frac{j-1}{2^n},\frac{j}{2^n}}\in P([0,1])\;\middle|\;1\le j\le 2^n}\]
        について,
        \[G:=\Brace{\om\in\Om\mid\exists_{t_1\ne t_2\in[0,1]}\;\sup_{t\in[0,1]}B_t=B_{t_1}=B_{t_2}}\subset\cup_{n\ge 1}\cup_{I_1,I_2\in\I_n,I_1\cap I_2=\emptyset}\Brace{\sup_{t\in I_1}B_t=\sup_{t\in I_2}B_t}\]
        が成り立つので,$\forall_{n\in\N}\;\forall_{I_1,I_2\in\I_n}\;I_1\cap I_2=\emptyset\Rightarrow P\Square{\sup_{t\in I_1}B_t=\sup_{t\in I_2}B_t}=0$を示せば良い.
        \item[証明] 任意の閉区間$[a,b]\subset[0,1]$について,確率変数$\sup_{t\in[a,b]}B_t:\Om\to\R$の確率分布$P[\sup_{t\in[a,b]}B_t|\F_a]:\Om\to[0,1]$は絶対連続であることを示せば十分である.
        
        $\sup_{t\in[a,b]}B_t=\sup_{t\in[a,b]}(B_t-B_a)+B_a$とみると,
        $\sup_{t\in[a,b]}(B_t-B_a)$を$\F_a$で条件づけたものは,$\sup_{t\in[0,b-a]}B_t$と同じ確率分布を持つ.
        系\ref{cor-density-of-sup-process-of-Brownian-motion}より,これは絶対連続である.
    \end{description}
\end{Proof}


\section{Brown運動の生成作用素と偏微分方程式}

\begin{tcolorbox}[colframe=ForestGreen, colback=ForestGreen!10!white,breakable,colbacktitle=ForestGreen!40!white,coltitle=black,fonttitle=\bfseries\sffamily,
title=]
    $P_t[f(x)]=E[f(x+B_t)]$をおくことにより,$\{P_t\}_{t\in\R_+}$は$C_0(\R)$上でHille-吉田の意味での強連続半群となり,
    その生成作用素$\lim_{t\to0}\frac{P_t-I}{t}=\frac{\nabla}{2}$がLaplace作用素である.$I$は恒等作用素とした.
\end{tcolorbox}

\subsection{Brown運動の生成作用素}

\begin{tcolorbox}[colframe=ForestGreen, colback=ForestGreen!10!white,breakable,colbacktitle=ForestGreen!40!white,coltitle=black,fonttitle=\bfseries\sffamily,
    title=]
    $d$-次元Brown運動の生成作用素はLaplace作用素の$1/2$倍になる.
\end{tcolorbox}

\begin{proposition}
    一次元Brown運動の生成作用素は
    $A=\frac{1}{2}\dd{^2}{x^2}$である.
\end{proposition}

\subsection{確率論が与える解作用素}

\begin{proposition}[到達時刻による調和関数の構成]
    $D\osub\R^d$を有界領域,$\sigma$を$x\in D$から出発する$d$次元Brown運動の$\partial D$への到達時刻とすると,$P[\sigma<\infty]=1$.
    このとき,任意の$f\in C_b(\partial D)$に対して,$u(x):=E_x[f(B_\sigma)]$は$D$上の調和関数である.
\end{proposition}
\begin{Proof}\mbox{}
    \begin{description}
        \item[方針] 任意の開球$B(x,\delta)\subset D$を取り,これに対して
        \[u(x)=\dint_{\partial B(x,\delta)}u(y)dy\]
        を示せば良い.
        \item[設定] $x$から出発するBrown運動の$\partial B(x,\delta)$への到達時刻を$\sigma:\Om\to\R_+$とする.
        さらに$\theta_\tau:\Om\to\Om$で,$\om$を$B_\tau(\om)$を原点に据えた場合の見本道を実現する$\om'$へ対応させる対応$\om\mapsto\om'$とする.すなわち,$B_t(\theta_\tau(\om))=B_{t+\tau}(\om)$.
        するとこのとき,
        \[\forall_{\om\in\Om}\quad B_{\sigma(\theta_\tau(\om))}(\theta_\tau(\om))=B_{\sigma(\om)}(\om)\]
        よって,強Markov性より,
        \begin{align*}
            u(x)&=E_x[f(B_\sigma)]=E_x[f(B_\sigma\circ\theta_\tau)]\\
            &=E_x[E_x[f(B_\sigma\circ\theta_\tau)|\F_\tau]]=E_x[E_x[f(B_\sigma\circ\theta_\tau)|B_\tau]]\\
            &=E_x[E_{B_\tau}[f(B_\sigma\circ\theta_\tau)]]=E_x[E_{B_\tau}[f(B_\sigma)]]\\
            &=E_x[u(B_\tau)].
        \end{align*}
    \end{description}
\end{Proof}
\begin{remark}
    この$u$が$\partial D$上でも連続であるためには,$\partial D$の正則性,例えば$C^1$-級などの条件が必要である.
\end{remark}

\begin{observation}
    $D\osub\R^d$を有界領域,$u\in C^2(\o{D})$を調和関数とする.
    $B$を$x\in D$から出発するBrown運動,$\sigma$をその$\partial D$への到達時刻とすれば,
    \[\forall_{t\in\R_+}\quad u(B_{t\land\sigma})=u(x)+\sum_{i\in[d]}\int^{t\land\sigma}_0\pp{u}{x^i}(B_s)dB_s^i+\frac{1}{2}\int^{t\land\sigma}_0\Lap u(B_s)ds.\]
    ここで,$\forall_{s<\sigma}\;\Lap u(B_s)=0$より,$u(B_{t\land\sigma})$は有界なマルチンゲールとなる.
\end{observation}

\chapter{Brown運動に基づく確率解析}

\begin{quotation}
    任意のLevy過程はBrown運動とPoisson点過程とに分解出来るとの観点から,Brown運動に駆動される確率過程のみを扱う.
    すると,
    伊藤解析とは,Hilbert空間の等長同型$I:L^2(\Om\times\R_+,\P)\mono L^2_0(\Om,(\F_t))$とその代数法則をいう.
    \begin{enumerate}
        \item まず発展的可測で大域的に自乗可積分な過程$u\in L^2_\P(\Om\times\R_+)$について,確率積分$I:L^2(\P)\to L^2_0(\Om)$を定義する.これはHilbert空間の等長同型となる.
        \item 射影$L^2_\P(\Om\times\R_+)\epi L^2_\P(\Om\times[0,t])$に沿って系列$(M_t:=I(1_{[0,t]}u))_{t\in\R_+}$が定まり,局所可積分なマルチンゲールとなる.これが一様可積分でもあるとき,ある$I(u)=M_\infty\in L^1(\Om)$の条件付き期待値の系列とみれる!
        \item この不定積分の過程に注目することで,$M_\infty$が存在しない場合も$(M_t)_{t\in\R_+}$は定まることがある.
        $\Om\times\R_+$上で局所自乗可積分なものを$L^2_\infty(\P)$で表す.一方で,$\R_+\to L(\Om)$で殆ど確実に$L^2(\Om)$に入るものを$L^2_\loc(\P)$で表す.
        前者は$1_{[0,t]}(\om)$を当てれば$L^2(\P)$に入る.後者は$1_{[0,\tau(\om)]}(\om)$を当てれば$L^2(\P)$に入る.
        \item こうして$I$は$L^2_\loc(\P)$上に延長出来る.するとこの像は局所マルチンゲールで,$I$は始域と終域の確率収束の測度について連続である.
    \end{enumerate}
\end{quotation}

\section{大域的確率積分}

\begin{tcolorbox}[colframe=ForestGreen, colback=ForestGreen!10!white,breakable,colbacktitle=ForestGreen!40!white,coltitle=black,fonttitle=\bfseries\sffamily,
title=]
    博士論文\cite{Ito42}でLévy過程のLévy-伊藤分解定理を定め,\cite{Ito44}で確率積分を定義した.
    そのアイデアは,見本道毎に見るのではなく,見本道の束$\Om\to C(\R_+)$の平均的な性質に注目することである.
    被積分関数は発展的可測とすることで,得られる確率変数は適合的になる.
\end{tcolorbox}

\subsection{発展的可測過程の空間}

\begin{tcolorbox}[colframe=ForestGreen, colback=ForestGreen!10!white,breakable,colbacktitle=ForestGreen!40!white,coltitle=black,fonttitle=\bfseries\sffamily,
title=発展的可測性とは積空間上における$\P/\B(\R)$-可測性である]
    過程が可測であることより強い条件として,発展的可測性を定義する.
    $L(\P)$の元は可測な適合的過程でもある:$L(\P)\subset L(\F\otimes\B(\R_+))\cap\bF$.
\end{tcolorbox}

\begin{definition}[progressive measurable]
    過程$(u_t)_{t\in\R_+}$が\textbf{発展的可測}であるとは,各時点$t$までの過程$u|_{\Om\times[0,t]}$が積写像として可測であることをいう:
    \[\forall_{t\in\R_+}\;u|_{\Om\times[0,t]}\in\L(\Om\times[0,t],\F_t\times\B([0,t])).\]
\end{definition}

\begin{lemma}[発展的可測性の十分条件]\mbox{}
    \begin{enumerate}
        \item 適合的な可測過程$u:\Om\times\R_+\to\R$には,発展的可測なバージョンが存在する(Meyer 1984, Th'm 4.6).
        \item 適合的な可測過程$u:\Om\times\R_+\to\R$で,
        \[\forall_{T>0}\;E\Square{\int^T_0u^2_sds}<\infty\]
        を満たすものには,$p$-可測(特に$w$-可測,発展的可測)なバージョンが存在する.
        \item 適合的な$D$-過程は発展的可測である.
    \end{enumerate}
\end{lemma}

\begin{theorem}[発展的可測性の特徴付け]
    \[\P:=\Brace{A\in P(\Om\times\R_+)\mid\forall_{t\in\R_+} A\cap(\Om\times[0,t])\in\F_t\times\B([0,t])}\]
    はたしかに$\F\times\B(\R_+)$の部分$\sigma$-代数で,
    可測過程$u:\Om\times\R_+\to\R$について,次の2条件は同値.
    \begin{enumerate}
        \item $\P/\B(\R)$-可測:$u\in\L(\P)$.
        \item $u$は発展的可測.
    \end{enumerate}
\end{theorem}
\begin{Proof}\mbox{}
    \begin{description}
        \item[$\P$のwell-definedness] 明らかに,$\forall_{t\in\R_+} A\cap(\Om\times[0,t])\in\F_t\times\B([0,t])$という条件は,$A=\emptyset,\Om\times\R_+$はこれを満たし,これを満たす$(A_n)$が存在したとき,合併も満たす.
        また,補集合については,$\F_t\times\B([0,t])$が$\Om\times[0,t]$上の$\sigma$-代数をなすことに注意すると,
        $A\cap(\Om\times[0,t])\in\F_t\times\B([0,t])$のとき,
        $\o{A}\cap(\Om\times[0,t])=(\Om\times[0,t])\setminus (A\cap(\Om\times[0,t]))\in\F_t\times\B([0,t])$による.
        \item[$\P\subset\F\times\B(\R_+)$] 
        任意の$A\in\P$を取ると,特に$n\in\N$について,$A_n:=A\cap(\Om\times[0,n])\in\F_n\times\B([0,t])\subset\F\times\B(\R_+)$.
        $A=\cup_{n\in\N}A_n\in\F\times\B(\R_+)$.
        \item[(1)$\Leftrightarrow$(2)] $\P/\B(\R)$-可測過程$u:\Om\times\R_+\to\R$を取る.任意の$t\in\R_+$について,$u|_{\Om\times[0,t]}$によるBorel可測集合$B\in\B(\R)$の逆像は$u^{-1}(B)\cap(\Om\times[0,t])$であるが,$u^{-1}(B)\in\P$より,これは$\F_t\times\B([0,t])$の元である.
    \end{description}
\end{Proof}

\subsection{発展的可測な単過程の確率積分}

\begin{tcolorbox}[colframe=ForestGreen, colback=ForestGreen!10!white,breakable,colbacktitle=ForestGreen!40!white,coltitle=black,fonttitle=\bfseries\sffamily,
title=]
    「発展的可測な2乗可積分過程のなすHilbert空間」を$L^2(\P)(\subset L^2(\F\times\B^1(\R)))$で表し,
    この上での$\R_+$上での確率積分を定義することを考える.
    単過程とは,時間軸を有限に分解して,それぞれの区間上で特定の2乗可積分確率変数$\phi_j\in\L^2(\Om)$(ただし$(t_j,\infty)$の情報とは独立)とみなせる挙動をする過程をいう.
\end{tcolorbox}

\begin{notation}[確率積分を定義する過程のクラス]\mbox{}
    \begin{enumerate}
        \item 空間$(\Om\times\R_+,\P,P\times l)$上の
        2乗可積分関数のHilbert空間を$L^2(\P):=L^2(\Om\times\R_+,\P,P\times l)$と表す,ただし$l$はLebesgue測度とした.
        また,$L^2_T(\P):=L^2(\Om\times[0,T],\P|_{\Om\times[0,T]},P\times l)$なる略記も用いる.
        このような関数を発展的可測な確率過程といい,このクラスに対してまずは$\R_+$上での確率積分を定める.
        \item ノルムを
        \[\norm{u}^2_{L^2(\P)}:=E\Square{\int^\infty_{0}u_s^2ds}=\int^\infty_0E[u^2_s]ds\]
        で表す.最後の等式はFubini-Tonelliの定理による.
    \end{enumerate}
\end{notation}

\begin{definition}[simple process, stochastic integral]\mbox{}
    \begin{enumerate}
        \item 
        $u=(u_t)_{t\in\R_+}\in L^2(\P)$が\textbf{単過程}であるとは,
        \[u_t=\sum_{j=0}^{n-1}\phi_j1_{(t_j,t_{j+1}]}(t),\qquad 0\le t_0<t_1<\cdots<t_n,\phi_j\in\L^2_{\F_{t_j}}(\Om)\]
        を満たすことをいう.単関数全体の集合を$\E\subset L^2(\P)$で表す.
        \item 単過程に対する確率積分$I:\E\to\L^2(\Om)$を
        \[I(u)=\int^\infty_0 u_tdB_t:=\sum^{n-1}_{j=0}\phi_j(B_{t_{j+1}}-B_{t_j})\]
        と定める.
    \end{enumerate}
\end{definition}
\begin{remarks}\label{remarks-tricks-in-the-definition-of-Ito-integral}
    すでにこの時点で,$\phi_j$を$1_{(t_j,t_{j+1}]}$に対して$\F_{t_j}$-可測としているため,$\phi_j\indep (B_{t_{j+1}}-B_{t_j})$を含意することがトリックになる.
    また,$I(u)=:u\cdot B$を\textbf{セミマルチンゲール記法}ともいう.
\end{remarks}

\begin{lemma}[単関数上の積分の性質]\label{lemma-property-of-SI-on-E}
    積分$I:\E\to\L^2(\Om)$は,$u,v\in\E,a,b\in\R$について次を満たす.
    \begin{enumerate}
        \item 線形性:$\int^\infty_0(au_t+bv_t)dB_t=a\int^\infty_0u_tdB_t+b\int^\infty_0v_tdB_t$.
        \item 中心化:$E[I(u)]=E\Square{\int^\infty_0u_tdB_t}=0$.
        \item 等長性:$\norm{I(u)}^2_{\L^2(\Om)}=E\Square{\paren{\int^\infty_0u_tdB_t}^2}=E\Square{\int^\infty_0u^2_tdt}=\norm{u}^2_{\L^2(\P)}$.
    \end{enumerate}
\end{lemma}
\begin{Proof}\mbox{}
    \begin{enumerate}
        \item $u_t,v_t$が定める$\R_+$の分割の細分を取って考えると良い.
        \item $\forall_{j\in n}\;\phi_j\indep (B_{t_{j+1}}-B_{t_j})$より,
        \[E\Square{\int^\infty_0u_tdB_t}=\sum^{n-1}_{j=0}E[\phi_j(B_{t_{j+1}}-B_{t_j})]=\sum^{n-1}_{j=0}E[\phi_j]E[B_{t_{j+1}}-B_{t_j}]=0.\]
        \item $u_t=\sum^{n-1}_{j=0}\phi_j1_{(t_j,t_{j+t}]}(t)\;(\phi_j\in L^2_{\F_{t_j}}(\Om))$,$\Delta B_j:=B_{t_{j+1}}-B_{t_j}\in L^2(\Om)$とすると,
        $(I(u))^2=\sum^{n-1}_{i,j=0}\phi_i\phi_j\Delta B_i\Delta B_j$.

        まず,
        \[E[\phi_i\phi_j\Delta B_i\Delta B_j]=\begin{cases}
            0,&i\ne j,\\
            E[\phi_j^2](t_{j+1}-t_j),&i=j.
        \end{cases}\]
        これは,$i<j$のとき,$\phi_i\indep\Delta B_j,\phi_j\indep\Delta B_j,\Delta B_i\indep\Delta B_j$であること\ref{lemma-Blumenthal}による.
        $i=j$のとき,$\phi^2_i\indep(\Delta B_i)^2$.

        これより,
        \begin{align*}
            \norm{I(u)}^2_{\L^2(\Om)}&=E[I(u)^2]\\
            &=\sum^{n-1}_{j=0}E[\phi_j^2](t_{j+1}-t_j)\\
            &=E[I(u^2)]=\norm{u}^2_{\L^2(\P)}.
        \end{align*}
    \end{enumerate}
\end{Proof}
\begin{remarks}
    証明中の大事な示唆として,$L^2(\P)$内の$C$-過程(Brown運動など)は,2.の証明中のように収束する単過程列を取れる.
    しかし,一般の$u\in L^2(\P)$にこの方法が通用するわけではない.
\end{remarks}

\subsection{一般の発展的可測過程への延長}

\begin{tcolorbox}[colframe=ForestGreen, colback=ForestGreen!10!white,breakable,colbacktitle=ForestGreen!40!white,coltitle=black,fonttitle=\bfseries\sffamily,
title=]
    被積分過程$u\in L^2(\P)$は,発展的可測な連続過程で,そしてそれは発展的可測な単過程で近似できる.
    なお,Brown運動は発展的可測な連続過程に当てはまる.
\end{tcolorbox}

\begin{proposition}
    $\E\subset L^2(\P)$は稠密.
\end{proposition}
\begin{Proof}
    $\E\subset L^2(\P)$の間に,$C$-過程のクラス
    \[C'(\R_+;L^2(\Om)):=\Brace{u\in L^2(\P)\mid u_t:\R_+\to L^2(\Om)\text{は連続}}\]
    を考える.$C'(\R_+;L^2(\Om))=C(\R_+;L^2(\Om))$の如何は
    発展的可測性についての更なる考察が必要であるから,記号を使い分けた.
    同一視して,
    $\E\subset C'(\R_+;L^2(\Om))\subset L^2(\P)$の順に稠密性を示す.
    \begin{description}
        \item[1. $C'(\R_+;L^2(\Om))$の$L^2(\P)$上での稠密性] 任意の$u\in L^2(\P)$を取る.
        \[u^{(n)}_t:=n\int^t_{\paren{t-\frac{1}{n}}\lor0}u_sds\]
        と定めると,$u^{(n)}:\R_+\to L^2(\Om)$は連続である:$u^{(n)}\in C(\R_+;L^2(\Om))$.
        実際,Lebesgueの収束定理より,
        \[\lim_{h\to0}u_{t+h}^{(n)}=\lim_{h\to0}n\int_{\R_+}1_{[(t+h-1/n\lor0),t+h]}(s)u_sds=u_t^{(n)}.\]
        (したがって$\F\times\B^1(\R)$-可測).
        また,Cauchy-Schwarzの不等式とFubiniの定理より,
        \begin{align*}
            \int_{\R_+}\abs{u_t^{(n)}(\om)}^2dt&=\int_{\R_+}n^2\paren{\int_{\R_+}1_{[(t-1/n)\lor0,t]}(s)u_s(\om)ds}^2dt\\
            &\le\int_{\R_+}n^2\int_{\R_+}1_{[(t-1/n)\lor0,t]}(s)ds\int_{\R_+}1_{[(t-1/n)\lor0,t]}(s)u_s^2(\om)dsdt\\
            &\le n\iint_{\R_+\times\R_+}1_{[(t-1/n)\lor0,t]}(s)u_s^2(\om)dsdt\\
            &\le n\iint_{\R_+\times\R_+}1_{[s,s+1/n]}(t)u_s^2(\om)dtds=\int_{\R_+}u_s^2(\om)ds<\infty
        \end{align*}
        であるから,確かに$u_t^{(n)}\in L^2(\P)$でもある.
        よって,$u_t^{(n)}\in C'(\R_+;L^2(\Om))$.
        
        さらに,$u_t^{(n)}\to u_t$が成り立つ.
        実際,$u_t\in L^2(\Om\times\R_+,\P,P\times l)$より,
        \begin{align*}
            u_t^{(n)}&=\frac{\int^t_0u_sds-\int^{(t-1/n)\lor0}_0u_sds}{1/n}\xrightarrow{n\to\infty}u_t.
        \end{align*}
        が,まず任意の$\om\in\Om$について成り立ち,Lebesgueの優収束定理より,
        \[\int^\infty_0\abs{u_t(\om)-u^{(n)}_t(\om)}^2dt\xrightarrow{n\to\infty}0\]
        でもある.
        
        $u_t,u_t^{(n)}\in L^2(\P)$より,再びLebesgueの優収束定理から,
        \[E\Square{\int_{\R_+}\abs{u_t-u_t^{(n)}}^2dt}\xrightarrow{n\to\infty}0.\]
        よって,$L^2(\P)$上の任意の点には,$C'(\R_+;L^2(\Om))$の点列でそれに収束するものが存在する.
        いずれも距離化可能だから,$\o{C'(\R_+;L^2(\Om))}=L^2(\P)$.
        \item[2. $\E$の$L^2(\P)$の位相についての$ C'(\R_+;L^2(\Om))$上での稠密性]
        $u\in  C'(\R_+;L^2(\Om))$を任意に取る.
        \[u^{(n,N)}_t:=\sum^{n-1}_{j=0}u_{t_j}1_{(t_j,t_{j+1}]}(t),\qquad\paren{t_j:=N\frac{j}{n},j\in n}\]
        とすると,$u_{t_j}\in L^2_{t_j}(\P)$は発展的可測より特に
        $u_{t_j}\in L_{\F_{t_j}}^2(\Om)$だから,これはたしかに単過程で,$\E$のネットである.
        $u^{(n,N)}_t$は$t> N$については$0$であることに注意.
        これに対して,
        \[E\Square{\int^\infty_0\Abs{u_t-u_t^{(n,N)}}^2}\le E\Square{\int^\infty_Nu^2_tdt}+N\sup_{\abs{t-s}\le N/n}E[\abs{u_t-u_s}^2].\]
        $n\to\infty,N\to\infty$の極限を考えると,右辺は$0$に収束する.
        よって,$L^2(\P)$の相対位相について,$C'(\R_+;L^2(\Om))\subset\o{\E}$.
    \end{description}
\end{Proof}
\begin{remarks}\mbox{}
    \begin{enumerate}
        \item Wiener空間$C(\R_+)$などの無限次元位相線型空間は局所コンパクトになり得ない.\footnote{位相線型空間が局所コンパクトであることは,有限次元であることに同値.}
        また,$C_c(\R_+;L^2(\Om))$は$L^2(\R_+;L^2(\Om))$上稠密である,という結果は得られそうであるが,高度である.
        そこで,$f\in L^1([a,b],\o{\R})$について$F(x):=\int^x_af(t)dt$は$(a,b)$上殆ど至る所微分可能であり,$F'(x)=f(x)$が成り立つというLebesgueの定理を用いた.
        \item また,$C'(\R_+;L^2(\Om))$は,明らかな方法で単過程近似が出来ることを用いた.病的な関数については,Lebesgueの言葉で処理したのである.
    \end{enumerate}
\end{remarks}

\begin{corollary}[確率積分の等距離延長]\label{cor-extention-of-SI}
    確率積分$I:\E\to L^2(\Om)$は,線型な等長同型$I:L^2(\P)\to L^2(\Om)$に延長できる.
\end{corollary}
\begin{Proof}
    確率積分$I$は$\E$上の有界線型作用素であったから(等長なので特に有界),$L^2(\P)$上に作用素ノルムを変えずに一意に連続延長する.
    連続延長であることより,等距離性は保たれる:ノルムの連続性より,
    \[\norm{I(u)}=\norm{I(\lim_{n\to\infty}u^{(n)})}=\lim_{n\to\infty}\norm{I(u^{(n)})}=\lim_{n\to\infty}\norm{u^{(n)}}=\norm{u}.\]

    これも含めて証明すると,次のようになる:
    任意の$u\in L^2(\P)$に対して,$\E$上の$L^2(\P)$-収束列$(u^{(n)})$が取れ,また$I:\E\to L^2(\Om)$の等長性より像はCauchy列を定めるから,極限
    \[I(u):=\lim_{n\to\infty}\int^\infty_0u_t^{(n)}dB_t\]
    が存在する.
    実際,$2xy\le x^2+y^2$であることより,
    \begin{align*}
        E\Square{\paren{\int^\infty_0u_t^{(n)}dB_t-\int^\infty_0u_t^{(m)}dB_t}^2}\\
        &\le E\Square{\int^\infty_0(u_t^{(n)}-u_t)^2dt}+2E\Square{\int^\infty_0(u_n^{(t)}-u_t)(u_t-t_t^{(m)})dt}+E\Square{\int^\infty_0(u_t-u_t^{(m)})^2dt}\\
        &\le 1\paren{E\Square{\int^\infty_0(u_t^{(n)}-u_t)^2dt}+E\Square{\int^\infty_0(u_t^{(m)}-u_t)^2dt}}\xrightarrow{n,m\to\infty}0.
    \end{align*}
    あとは,収束列$(u^{(n)})$の取り方に依らないことを示せば良い.
    これは,2つの異なる収束列が同じ$u$に依存するとき,2つの収束列の差が定める列は$0$に収束するから,$I$の$0\in\E$での連続性から従う.
\end{Proof}

\subsection{確率積分の性質}

\begin{tcolorbox}[colframe=ForestGreen, colback=ForestGreen!10!white,breakable,colbacktitle=ForestGreen!40!white,coltitle=black,fonttitle=\bfseries\sffamily,
title=]
    $I:L^2(\Om\times\R_+,\P,P\times l)\to L^2(\Om,\F,P)$は,$t\in\R_+$の成分が消えて,純粋な確率変数のみが残ることとなる.
    このとき,$\Im I\subset L^2_0(\Om,\F,P)$を満たす,Hilbの埋め込みとなっている.
\end{tcolorbox}

\begin{proposition}[$I$はHilbの射である]\label{prop-orthonormality-of-stochastic-integral}
    任意の$u,v\in\L^2(\P)$について,
    \begin{enumerate}
        \item 中心化されている:$E[I(u)]=0$.
        \item 内積を保つ:$\Cov[I(u),I(v)]=E[I(u)I(v)]=E\Square{\int_0^\infty u_tdB_t\int_0^\infty v_tdB_t}=E\Square{\int^\infty_0u_sv_sds}=(u|v)_{L^2(\P)}$.
    \end{enumerate}
\end{proposition}
\begin{Proof}
    $\{u^{(n)}\}\subset\E$を$u$に$L^2(\P)$-収束する単過程列
    とすると,$I$の連続性より$\{I(u_n)\}$も$I(u)$に$L^2(\Om)$-収束する.
    \begin{enumerate}
        \item よって,
        \begin{align*}
            \Abs{E[I(u^n)]-E[I(u)]}&\le E[\abs{I(u^n)-I(u)}]=\norm{I(u^n)-I(u)}_{L^1(\Om)}\\
            &\le\norm{I(u^n)-I(u)}_{L^2(\Om)}\xrightarrow{n\to\infty}0.
        \end{align*}
        または,$L^2$-収束する列$(I(u_n))\to I(u)$の期待値の像も収束するためには,一様可積分性を示しても十分である.
        \item $I$の線形性と等長性\ref{cor-extention-of-SI}より,$(I(u+v))^2$の確率積分は
        \begin{align*}
            E[I(u^2)]+E[I(v^2)]+2E[I(uv)]=E[(I(u+v))^2]= E[(I(u)+I(v))^2]=E[I(u^2)]+E[I(v^2)]+2E[I(u)I(v)].
        \end{align*}
        と2通りに表せる.
        ゆえに,$E[I(uv)]=E[I(u)I(v)]$.
    \end{enumerate}
\end{Proof}
\begin{remarks}
    (2)は極化恒等式による証明である.
\end{remarks}

\subsection{不定積分の定義と例}

\begin{tcolorbox}[colframe=ForestGreen, colback=ForestGreen!10!white,breakable,colbacktitle=ForestGreen!40!white,coltitle=black,fonttitle=\bfseries\sffamily,
title=]
    $[0,T]$上の不定積分は,$\R_+$上の確率積分を通じて定める.
    任意の$u\in L^2_T(\P)$に対して,$(T,\infty)$上では$0$として延長すれば,
    不定積分は定められる.

    $\chi^2$-分布の普遍性は,Brown運動を確率積分したものであるから,という観点からも理解できる.
\end{tcolorbox}

\begin{definition}[有限区間上の不定積分]
    任意の$[0,T]\;(T>0)$上の発展的可測な2乗可積分関数$u\in L^2_T(\P)$,ただし,
    \[L^2_T(\P):=L^2(\Om\times[0,T],\P|_{\Om\times[0,T]},P\times l).\]
    の積分を
    \[\int^T_0u_sdB_s:=\int^\infty_0\wt{u}_s1_{[0,T]}(s)dB_s\]
    と定める.ただし,$\wt{u}$は$u$を$(T,\infty)$上で零として延長したものとした.
\end{definition}

\begin{example}[Brown運動の不定積分は$\chi^2$-確率変数]\label{exp-integral-of-Brownian-motion}
    任意の$T>0$に関して,Brown運動$(B_t)_{t\in[0,T]}$は$\F_t\times\B([0,T])$-可測過程である($[0,T]\to\L(\Om,\F_t,P)$が$C$-過程であるため).よって,
    Brown運動自体は発展的可測である.
    その上,連続過程でもある:$B\in C'(\R_+;L^2(\Om))$.
    確かに,
    \[u^{(n)}_t:=\sum^{n-1}_{j=0}B_{t_j}1_{(t_j,t_{j+1}]}(t),\quad t_j:=T\frac{j}{n}\]
    と定めると,これは$B$にノルム収束する単過程の列である.よって,
    \begin{align*}
        I(1_{[0,T]}B)&=\lim_{n\to\infty}I(u^{(n)})\\
        &=\lim_{n\to\infty}\sum^{n-1}_{j=0}B_{t_j}(B_{t_{j+1}}-B_{t_j})\\
        &=\frac{1}{2}\lim_{n\to\infty}\paren{\sum^{n-1}_{j=0}(B_{t_{j+1}}^2-B^2_{t_j})-\sum^{n-1}_{j=0}(B_{t_{j+1}}-B_{t_j})^2}=\frac{1}{2}B_T^2-\frac{1}{2}T.
    \end{align*}
    すなわち,
    \[\int^T_0B_tdB_t=\frac{1}{2}B^2_T-\frac{1}{2}T\]
    が成り立つ.
    従って,Brown運動$B$は,$\R_+$上積分可能ではないが,任意のコンパクト集合上では積分可能で,中心化された$\chi^2$-分布に従う確率変数になる.
\end{example}
\begin{remarks}
    これは,特殊な近づけ方をすると概収束もする.
    これのマルチンゲールに関する一般論がLevyのdownward定理である\cite{Morters-and-Peres}[Th'm 1.35].
\end{remarks}

\section{確率積分のマルチンゲール}

\begin{tcolorbox}[colframe=ForestGreen, colback=ForestGreen!10!white,breakable,colbacktitle=ForestGreen!40!white,coltitle=black,fonttitle=\bfseries\sffamily,
title=]
    $L^2(\P)$上の確率積分は一様可積分なマルチンゲールを定めた.
    これを一般化して,局所マルチンゲールまで許したい.
    ここで,
    $L^2(\P)\subset L^2_\infty(\P)\subset L^2_\loc(\P)$の関係がある.
    まず,$L^2_\infty(\P)$の元の不定確率積分に対する性質を見て,$L^2_\loc(\P)$への拡張は次節で議論する.
\end{tcolorbox}

\begin{notation}
    $\Om\times\R_+$上で局所自乗可積分なものを$L^2_\infty(\P)$で表す.一方で,$\R_+\to L(\Om)$で殆ど確実に$L^2(\Om)$に入るものを$L^2_\loc(\P)$で表す.
    前者は$1_{[0,t]}(\om)$を当てれば$L^2(\P)$に入る.後者は$1_{[0,\tau(\om)]}(\om)$を当てれば$L^2(\P)$に入る.
    \begin{enumerate}
        \item 
        発展的可測で,任意の有界閉集合$[0,T]$上で2乗可積分な過程のなす空間を
        \[L^2_\infty(\P):=\Brace{u\in\L(\P)\mid\forall_{t\in\R_+}\;E_{\Om\times\R_+}[1_{[0,t]}u^2]<\infty}\]
        とすると,これは$L^2(\P)$よりも大きい.
        標準的な埋め込み$L^2_T(\P)\mono L^2(\P)$と射影$L^2(\P)\epi L^2_T(\P)$とはいくつか考えられて,
        $L^2_T(\P)\subset L^2(\P)$の関係はいささか商空間に近いが,$L^2_\infty(\P)$はおそらく正しく定式化すれば「帰納極限」である.
        \item 本来は,発展的可測で,任意の有界閉集合$[0,T]$で,見本過程$u_s(\om)$が殆ど確実に2乗可積分になる過程のなす空間
        \[L^2_\loc(\P):=\Brace{u\in\L(\P)\mid\forall_{t\in\R_+}\;E_{\R_+}[1_{[0,t]}u^2]<\infty\;\as}\]
        で定義できる.
        積空間上で可積分ならば,その切り口は殆ど至る所有限であるから,明らかに
        $L^2_\infty(\P)\subset L^2_\loc(\P)$であるが,$L^2_\loc(\P)$の元が積空間上で可積分とは限らない.
        
        従って,全て併せると$L^2(\P)\subset L^2_\infty(\P)\subset L^2_\loc(\P)$の関係がある.
    \end{enumerate}
\end{notation}

\subsection{確率積分のマルチンゲール性}

\begin{definition}[indefinite integral]
    $u\in L^2_\infty(\P)$に対して,区間$[a,b]\subset\R_+$上での\textbf{不定積分}とは,
    $1_{[a,b]}(s)u_s\in L^2(\P)$であるから,$I(1_{[a,b]}u_s)=:\int^b_au_sdB_s$とすれば良い.
\end{definition}

\begin{proposition}
    不定積分の過程$\paren{M_t:=\int^t_0 u_sdB_s}_{t\in\R_+}\;(u\in L^2_\infty(\P))$は
    martingaleである.すなわち,次の3条件を満たす:
    \begin{enumerate}
        \item $(\F_t)$-適合的である.
        \item 可積分である.
        \item $\forall_{s\in[0,t]}\;E[M_t|\F_s]=M_s\;\as$
    \end{enumerate}
\end{proposition}
\begin{Proof}
    任意の$t\in\R_+$について,$1_{[0,t]}u\in L^2(\P)$より,ある単過程列$\{u^n\}\subset\E$が存在して,$u^n\to 1_{[0,t]}u\;\on L^2(\P)$.このとき,$I[1_{[0,t]}u^n]\to I[1_{[0,t]}u]\;\on L^2(\Om)$.
    \begin{enumerate}
        \item いま,各$I[1_{[0,t]}u^n]$が$\F_t$-可測であることに注意すると,$I[1_{[0,t]}u]$はある概収束する部分列$\{X_j\}\subset\{I[1_{[0,t]}u]\}$に関して$\lim_{j\to\infty}X_j=I[1_{[0,t]}u^n]\;\as$と表せるから,$\F_t$が任意の零集合を含むことより,やはり$\F_t$-可測である.
        %$M_t=I(1_{[0,t]}u)$が$\F_t$-可測であることを示すには,任意の$n\in\N$について$I(1_{[0,t]}u_s^{(n)})$が$\F_t$-可測であることを示せば良いが,これは
        %\[I(1_{[0,t]}u)=\sum_{j=0}^{n-1}u_{t_j}(B_{t_{j+1}}-B_{t_j})\]
        %というように,$\F_t$-可測な確率変数の連続な合成であることから明らか.
        \item $I$の像が$L^2(\Om)$であることより.
        \item 任意の$s\le t$に対して,
        \begin{align*}
            E[I(1_{[0,t]}u^{(n)})|\F_s]&=\sum^{n-1}_{j=0}E[u_{t_j}(B_{t_{j+1}}-B_{t_j})|\F_s]\\
            &=\sum^{n-1}_{j=0}E[E[u_{t_j}(B_{t_{j+1}\land t}-B_{t_j})|\F_{t_j\lor s}]|\F_s]\\
            &=\sum^{n-1}_{j=0}E[u_{t_{j}}E[B_{t_{j+1}\land t}-B_{t_j}|\F_{t_j\lor s}]|\F_s]\\
            &=\sum_{j=0}^{n-1}u_{t_j}(B_{t_{j+1}\land s}-B_{t_j\land s})=I(1_{[0,s]}u^{(n)}).
        \end{align*}
        というように,時刻$s$以降は$\F_s$と独立になるために$B_{t_{j+1}}-B_{t_j}$の項が零因子になり,ちょうど$s$までの和のみが残ることとなる.
        さらに,$I$の連続性より,$I(1_{[0,t]}u^{(n)})\to I(1_{[0,t]}u)\;\on\;L^2(\Om)$であるから,両辺$n\to\infty$の極限を考えると,条件付き期待値が$L^2(Om)$-距離に関する非拡大写像であることより,
        \[E[I(1_{[0,t]}u)|\F_s]=I(1_{[0,s]}u).\]
    \end{enumerate}
\end{Proof}

\subsection{確率積分の代数的性質}

\begin{tcolorbox}[colframe=ForestGreen, colback=ForestGreen!10!white,breakable,colbacktitle=ForestGreen!40!white,coltitle=black,fonttitle=\bfseries\sffamily,
title=]
    $L^\infty(\F_a)$-斉次性は,条件付き期待値の性質$\forall_{Z\in L^\infty(\cG)}\;E[ZX|\cG]=ZE[X|\cG]$.
\end{tcolorbox}

\begin{proposition}[代数法則]\label{prop-algebraic-property-of-indefinite-integral-process}
    不定積分の$(\F_t)$-マルチンゲール$\Brace{M_t:=\int^t_0 u_sdB_s}_{t\in\R_+}\subset L^1(\Om)\;(u\in L^2_\infty(\P))$について,
    \begin{enumerate}
        \item 加法性:$\forall_{a\le b\le c\in\R_+}\;\int^b_au_sdB_s+\int^c_bu_sdB_s=\int^c_au_sdB_s$.
        \item $L^\infty(\F_a)$-斉次性:任意の区間$(a,b)\subset\R_+$と,$\F_a$-可測な有界関数$F:\Om\to\R$について,$\int^b_aFu_sdB_s=F\int^b_au_sdB_s$.
    \end{enumerate}
\end{proposition}
\begin{Proof}\mbox{}
    \begin{enumerate}
        \item 確率積分の線形性より,$I[1_{[a,b]}u_s]+I[1_{[b,c]}u_s]=I[1_{[a,c]}u_s]$.
        \item $1_{[a,b]}Fu_s\in L^2(\P)$であるから,確率積分$\int^b_aFu_sdB_s$は確かに定まる.実際,
        \[E\Square{\int^b_au_s^2(\om)ds}<\infty,\quad F(\om)\in L^2(\Om)\]
        に注意すると,Cauchy-Schwarzの不等式より,
        \begin{align*}
            \int_\Om\int_{\R_+}1_{[a,b]}(s)F(\om)u_s(\om)dsd\om&=\int_\Om F(\om)\int^b_au_s(\om)dsd\om=\paren{F(\om)\;\middle|\;\int^b_au_s(\om)ds}_{L^2(\Om)}\\
            &\le\norm{F}_{L^2(\Om)}\norm{u_s}_{L^2(\Om\times[0,T])}<\infty.
        \end{align*}

        $F$を$\F_a$-可測な有界関数として,$I[1_{[a,b]}Fu]=F\cdot I[1_{[a,b]}u]$を示せば良い.
        すると,$1_{[a,b]}u\in L^2(\P)$に対して,
        単過程列$(u^{(n)})$が取れて,$1_{[a,b]}u$に$L^2(\P)$-収束する.
        この任意の$u^{(m)}$について,$I[1_{[a,b]}Fu^{(m)}]=F\cdot I[1_{[a,b]}u^{(m)}]$が示せれば,あとは$I$のノルム連続性から従う.
        単過程$u^{(m)}$は
        \[u^{(m)}:=\sum^{n^{(m)}-1}_{j=0}\phi^{(m)}_{j}1_{(t_j^{(m)},t_{j+1}^{(m)}]},\qquad\paren{\phi^{(m)}_j\in L^2_{\F_{t_j}}(\Om),0\le t_1^{(m)}<\cdots<t_{n^{(m)}}^{(m)}}\]
        と表せる.
        このとき,
        \[1_{[a,b]}Fu^{(m)}:=\sum^{n^{(m)}-1}_{j=0}F\phi^{(m)}_{j}1_{(t_j^{(m)}\lor a,t_{j+1}^{(m)}\land b]}\]
        も単過程であることを示せば,結果$I[1_{[a,b]}Fu^{(m)}]=F\cdot I[1_{[a,b]}u^{(m)}]$は単過程に対する確率積分$I$の定義からすぐに従う.
        
        $\phi_j^{(m)}$は$\F_{t_j^{(m)}}$-可測としたから,特に$\F_{t_j^{(m)}\lor a}$-可測.
        $F$は$\F_a$-可測としたから特に$\F_{t_j^{(m)}\lor a}$-可測.よって,$1_{[a,b]}Fu^{(m)}$も単過程である.
        %$Fu_s$も,各係数$Fu_{t_j}$は2乗可積分で$\F_{t_j}$-可測な確率変数となるから,$Fu$も単過程である.
        %実際,$F$は仮定から$\F_a\subset\F_{t_j}$-可測で,$u_{t_j}$は$u$が発展的可測であることより$u|_{[0,t_j]}$が$\F_{t_j}\times\B([0,t_j])$-可測であり,$\forall_{I\in\B(\R)}\;u_{t_j}^{-1}(I)=(u|_{[0,t_j]}^{-1}(I))_{t_j}\in\F_{t_j}$という切り口で表されるから,やはり$\F_{t_j}$-可測(Fubiniの定理).
        %したがって,単過程に対する確率積分の定義から,
        %\[I[1_{[a,b]}Fu_s]=\sum^{n-1}_{j=0}Fu_{t_j}(B_{t_{j+1}}-B_{t_j})=F\sum^{n-1}_{j=0}u_{t_j}(B_{t_{j+1}}-B_{t_j})=FI[1_{[a,b]}u_s].\]
        %を得る.
    \end{enumerate}
\end{Proof}
\begin{remarks}
    おそらく,$I(1_{[a,b]}u)$は$\F_b$-可測である.
\end{remarks}

\subsection{連続な修正の存在}

\begin{tcolorbox}[colframe=ForestGreen, colback=ForestGreen!10!white,breakable,colbacktitle=ForestGreen!40!white,coltitle=black,fonttitle=\bfseries\sffamily,
title=]
    一般のマルチンゲールは$D$-変形が存在するが,不定積分のマルチンゲールは$C$-変形が取れる.
\end{tcolorbox}

\begin{proposition}[$C$-過程と同等]
    任意の$u\in L^2_\infty(\P)$の不定積分のマルチンゲール$(M_t)_{t\in\R_+}$には連続な修正が存在する.
\end{proposition}
\begin{Proof}
    任意の$u\in L^2_\infty(\P)$と任意の$T>0$を取ると,$1_{[0,T]}u\in L^2(\P)$より,単過程列$\{u^{(n)}\}\subset\E$が存在して,$u^{(n)}\to 1_{[0,T]}u\in L^2(\P)$.
    ここで,各単過程$u^{(n)}$の定める不定積分のマルチンゲール$I[1_{[0,t]}u^{(n)}]=:M_t^{(n)}$は,
    \[M^{(n)}_t=\sum_{j=0}^{m^{(n)}-1}\phi_j^{(n)}(B_{t_{j+1}^{(n)}\land t}-B_{t_j^{(n)}})\qquad\paren{\phi_j^{(n)}\in L^2_{\F_{t_j^{(n)}}},0\le t_0^{(n)}<t_1^{(n)}<\cdots<t_{m^{(n)}}^{(n)}\le T}\]
    と表せるから,$C$-過程$\Om\to C([0,T];\R)$であり,いま$M_t$に各点$t\in[0,T]$について$L^2(\Om)$-収束することはわかっている.
    これが,殆ど確実に$[0,T]$上一様収束することを示す.すると,極限過程$(J_t)_{t\in[0,T]}$は連続で,$(M_t)_{t\in[0,T]}$の修正である.
    \begin{enumerate}[(a)]
        \item $(M_t^{(n)}-M_t^{(m)})$は連続なマルチンゲールだから,これについてのDoobの不等式より,任意の$\lambda>0$について,
        \begin{align*}
            P\Square{\sup_{t\in[0,T]}\abs{M_t^{(n)}-M_t^{(m)}}>\lambda}&\le\frac{1}{\lambda^2}E[\abs{M^{(n)}_T-M^{(m)}_T}^2]=\frac{1}{\lambda^2}\norm{I(u^{(n)}_T-u^{(m)}_T)}_{L^2(\Om)}\\
            &=\frac{1}{\lambda^2}\norm{u^{(n)}_t-u^{(m)}_t}_{L^2(\P)}\xrightarrow{n,m\to\infty}0.
        \end{align*}
        \item したがって,任意の$k\in\N$に対して,十分大きく$n_k,n_k+1\in\N$を取ると,
        \[P\Square{\sup_{t\in[0,T]}\abs{M_t^{(n_{k+1})}-M_t^{(n_k)}}>\frac{1}{2^k}}\le\frac{1}{2^k}\]
        が成り立つ.よって,$A_k:=\Brace{\sup_{t\in[0,T]}\abs{M_t^{(n_{k+1})}-M_t^{(n_k)}}>\frac{1}{2^n}}\subset\Om$とおくと,$\sum^\infty_{k=1}P[A_k]<\infty$.
        よって,Borel-Cantelliの補題より,$N:=\limsup_{k\to\infty}A_k$とおくとこれは零集合であり,
        \[\forall_{\om\in\Om\setminus N}\;\exists_{k_1\in\N}\;\forall_{k\ge k_1}\quad\sup_{t\in[0,T]}\abs{M_t^{(n_{k+1})}(\om)-M_t^{(n_k)}(\om)}\le\frac{1}{2^k}\]
        が成り立っている.これは,$(M_t^{(n_k)})$が殆ど確実に,$[0,T]$上一様収束することを表している.この極限となる連続関数を$J_t(\om):\Om\times[0,T]\to\R$とおく.
        \item ここで,$(M_t^{(n_k)})$は$\int^t_0u_sdB_s$にも$L^2(\Om)$-収束するから,$\forall_{t\in[0,T]}\;J_t=\int^t_0u_sdB_s\;\as$.
        $T>0$は任意にとったから,結局任意の$t\in\R_+$について成り立つ.
    \end{enumerate}
\end{Proof}

\begin{corollary}[最大不等式]
    $u\in L_\infty^2(\P)$について,
    \begin{enumerate}
        \item $\forall_{T,\lambda>0}\;P\Square{\sup_{t\in[0,T]}\abs{M_t}>\lambda}\le\frac{1}{\lambda^2}E\Square{\int^T_0u^2_tdt}$.
        \item $E\Square{\sup_{t\in[0,T]}\abs{M_t}^2}\le 4E\Square{\int^T_0u^2_tdt}$.
        \item (1),(2)は特に$u\in L^2(\P)$の場合,$T=\infty$としても成立する.
    \end{enumerate}
\end{corollary}
\begin{Proof}
    (1),(2)は,不定積分のmartingale $(M_t)$の連続な修正$(\wt{M}_t)$について(これもマルチンゲール),$\forall_{T>0}\;\wt{M}_T\in L^2(\Om)$であるから,Doobの不等式より,
    \[\forall_{\lambda>0}\quad P\Square{\sup_{0\le t\le T}\abs{\wt{M}_t}>\lambda}\le\frac{1}{\lambda^2}E[\abs{\wt{M}_T}^2],\]
    \[E\Square{\sup_{t\in[0,T]}\abs{\wt{M}_t}^2}\le 4E[\abs{\wt{M}_T}^2].\]
    \begin{enumerate}
        \item $E[\abs{\wt{M}_t}^2]=E[\abs{M_t}^2]$は明らか.
        また,$\Brace{\sup_{0\le t\le T}\abs{\wt{M} _t}\ne\sup_{0\le t\le T}\abs{M_t}}$も零集合であることを示す.
        
        よって,
        1式目の左辺は$M_t$としてもよく,同様に右辺は
        \[=\frac{1}{\lambda^2}\norm{I(1_{[0,T]}u)}_{L^2(\Om)}^2=\frac{1}{\lambda^2}E\Square{\int^T_0u_t^2dt}\]
        に等しい.
        \item 同様にして,2式目の右辺は
        \[=4E\Square{\int^T_0u_t^2dt}\]
        に等しい.
        \item $u\in L^2(\P)$の場合,単調収束定理により,$T\to\infty$としても(1)は成り立つ.
        (2)の左辺が収束するのは,マルチンゲールの2次変分過程が可積分であるから,Lebesgueの優収束定理が使える.
    \end{enumerate}
\end{Proof}

\subsection{確率積分のマルチンゲールの二次変分}

\begin{tcolorbox}[colframe=ForestGreen, colback=ForestGreen!10!white,breakable,colbacktitle=ForestGreen!40!white,coltitle=black,fonttitle=\bfseries\sffamily,
title=]
    $C$-変形が取れたために,二次変分過程を考えられる.
    不定積分のマルチンゲール$M_t$の二次変分過程は,単調増加で連続な見本道を持つ適合的過程$\brac{M}_t=\int^t_0u_s^2ds$となる.
    こうしてある意味「見本道毎の見方」が返ってきた.
\end{tcolorbox}

\begin{proposition}[二次変分]\label{prop-quadratic-covariation-of-the-martingale-of-SI}
    $u\in L^2_\infty(\P)$とする.
    任意の分割$\pi:=\{0=t_0<t_1<\cdots<t_n=t\}\subset[0,t]\;(n\in\N)$に関する二次の変動は,任意の$\abs{\pi_n}\to0$について同一の$L^1(\Om)$-極限を持つ:
    \[S^2_\pi(u):=\sum^{n-1}_{j=0}\paren{\int^{t_{j+1}}_{t_j}u_sdB_s}^2\xrightarrow{L^1(\Om)}\int^t_0u^2_sds\]
    を持つ.
\end{proposition}
\begin{Proof}\mbox{}
    \begin{description}
        \item[$u\in\E$のとき] 
        $u$を単過程とすると,
        \[u=\sum^{m-1}_{i=0}\phi^i1_{(t^i,t^{i+1}]}\quad\phi^i\in L^2_{\F_{t^i}}(\Om),0\le t^0<\cdots<t^m\le t.\]
        と表せる.このとき,
        任意の分割$\pi_n=\Brace{0=t_0<\cdots<t_n=t}$の番号の振り方を
        \[0=t_{00}<\cdots<t_{0n_0}\le t^0<t_{11}<\cdots<t_{1n_1}\le t^1<\cdots<t_{mn_m}=t\]
        を満たすように定めれば,$\abs{\pi_n}\to0$のとき,$\forall_{i\in[m]}\;n_i\to\infty$となる.
        以降,$\phi^{i+1}=0$とする.
        このとき,次のように評価できる:
        \begin{align*}
            S_\pi^2(u)&=\sum_{j\in n}\paren{\int^{t_{j+1}}_{t_j}u_sdB_s}^2\\
            &=\sum_{i\in m}\sum_{j\in n_i}\paren{\int^{t_{(i+1)(j+1)}}_{t_{(i+1)j}}u_sdB_s}^2+\sum_{i\in m}\paren{\int^{t_{(i+2)1}}_{t_{(i+1)n_i}}u_sdB_s}^2\\
            &=\sum_{i\in m}(\phi^i)^2\sum_{j\in n_i}(B_{t_{j+1}}-B_{t_j})^2+\sum_{i\in m}\paren{\phi^i(B_{t^i}-B_{t_{in_i}})+\phi^{i+1}(B_{t_{(i+1)0}}-B_{t^i})}^2\\
            &\xrightarrow[\abs{\pi_n}\to\infty]{L^2(\Om)}\sum_{i\in m}(\phi^i)^2(t_{j+1}-t_j).
        \end{align*}
        なお,最後の収束はBrown運動の二次変分\ref{prop-quadratic-variation-of-Brownian-motion}の収束による.
        \item[一般の$u\in L^2_\infty(\P)$のとき] 
        このとき,$1_{[0,t]}u\in L^2(\P)$より,任意の$\ep>0$に対して,ある単過程$v\in\E$が存在して,
        \[\norm{1_{[0,t]}(u-v)}^2_{L^2(\P)}=E\Square{\int^t_0(u_s-v_s)^2ds}<\ep\]
        を満たす.いま,
        \[E\Square{\Abs{S^2_\pi(u)-\int^t_0u^2_sds}}\le E[\abs{S^2_\pi(u)-S^2_\pi(v)}]+E\Square{\Abs{S^2_\pi(v)-\int^t_0v^2_sds}}+E\Square{\Abs{\int^t_0(u_s^2-v_s^2)ds}}\]
        と評価出来る.
        \begin{enumerate}[(a)]
            \item $v\in\E$であるから,第2項は$0$に収束する.
            \item 第3項は,$u_s^2-v_s^2=(u_s+v_s)(u_s-v_s)$とみてCauchy-Schwarzの不等式を使い,次に三角不等式より,
            \begin{align*}
                E\Square{\Abs{\int^t_0(u_s^2-v_s^2)ds}}&\le E\Square{\int^t_0\abs{u-v}\abs{u+v}ds}\\
                &\le\paren{E\Square{\int^t_0(u-v)^2ds}}^{1/2}\paren{E\Square{\int^t_0(u+v)^2ds}}^{1/2}
                <\sqrt{\ep}\norm{u+v}_{L^2_t(\P)}\\
                &\le\ep^{1/2}(2\norm{u}_{L^2_t(\P)}+\norm{u-v}_{L^2_t(\P)})=\sqrt{\ep}(2\norm{u}_{L^2_t(\P)}+\sqrt{\ep}).
            \end{align*}
            \item 第1項も,確率積分$I:L^2(\P)\to L^2(\Om)$がHilbの同型であり,内積を保存するために
            \begin{align*}
                E[\abs{S^2_\pi(u)-S^2_\pi(v)}]&=E\Square{\Abs{\sum_{j\in N}\paren{\int^{t_{j+1}}_{t_j}udB}^2-\sum_{j\in N}\paren{\int^{t_{j+1}}_{t_j}vdB}^2}}\\
                &\le\sum_{j=0}^{n-1}E\Square{\int^{t_{j+1}}_{t_j}\abs{u+v}dB\int^{t_{j+1}}_{t_j}\abs{u-v}dB}=\sum_{j\in N}E\Square{\int^{t_{j+1}}_{t_j}\abs{u+v}\abs{u-v}ds}
            \end{align*}
            と変形出来るが,これは第3項と全く同じ形に帰着している.
        \end{enumerate}
        以上より,
        \[\limsup_{\abs{\pi}\to0}E\Square{\Abs{S^2_\pi(u)-\int^t_0u^2_sds}}\le 2\ep^{1/2}(2\norm{u}+\ep^{1/2}).\]
    \end{description}
\end{Proof}
\begin{remark}
    $u\in\E$の場合は$S^2_\pi(u)$が陽に計算できたので問題がなかったが,一般の
    $u\in L^2_\infty(\P)$では抽象的に$L^2$-ノルムを計算する必要があり,これを実行すると$B_s^4$の項が出現してしまうが,
    $L^1$-収束ならば一般の場合について示せる.
    さらに,$u\in L^2_\loc(\P)$が定める確率積分の局所マルチンゲールの二次変分も同じだが,確率収束しか示せない\ref{prop-2nd-order-variance-of-indefinite-processes}.
\end{remark}

\subsection{確率積分のマルチンゲールの積率の存在}

\begin{tcolorbox}[colframe=ForestGreen, colback=ForestGreen!10!white,breakable,colbacktitle=ForestGreen!40!white,coltitle=black,fonttitle=\bfseries\sffamily,
title=]
    二次変分の存在が示せたら,マルチンゲールの一般論から,任意次数の積率が存在するという強力な結果が得られる.
\end{tcolorbox}

\begin{corollary}[Burkholder-David-Gundy inequality]\label{cor-Burkholder-for-the-martingale-of-SI}
    $u\in L^2_\infty(\P)$について,
    \[\forall_{p>0,T>0}\quad c_pE\Square{\Abs{\int^T_0u^2_sds}^{p/2}}\le E\Square{\sup_{t\in[0,T]}\Abs{\int^t_0u_sdB_s}^p}\le C_pE\Square{\Abs{\int^T_0u^2_sds}^{p/2}}.\]
\end{corollary}
\begin{remarks}
    これは最大不等式の精緻化で,特に二次変動過程の値$\brac{M}_T$を用いているもの,と見れる.
\end{remarks}

\subsection{停止時までの積分}

\begin{definition}[停止時までの積分]
    $\tau\in\T(\F_t)$を停止時とする.時刻$\tau$までの確率積分$\int^\tau_0u_tdB_t$とは,確率変数
    \[\om\mapsto\paren{\int^{\tau(\om)}_0u_tdB_t}(\om)\]
    とする.
\end{definition}

\begin{proposition}[停止時による剪断過程の積分は,積分区間への停止時の代入に等しい]\label{prop-integral-until-stopping}
    $u\in L^2(\P)$で,$\tau:\Om\to\R_+$を停止時とする.このとき,
    $u1_{[0,\tau]}\in L^2(\P)$で,
    \[I[u_t1_{[0,\tau]}]=\int^\infty_0u_t1_{[0,\tau]}(t)dB_t=\int^\tau_0u_tdB_t.\]
\end{proposition}
\begin{Proof}\mbox{}
    \begin{description}
        \item[$u$が単過程で$\tau$が単関数であるとき] 
        まず,
        \[u_t=F1_{(a,b]}(t)\;(0\le a<b,F\in L^2_{\F_a}(\Om)),\quad\Im\tau\subset\Brace{0=t_0<t_1<\cdots<t_n}\]
        と表せる場合を考える.このとき,$1_{(0,\tau]}$は
        \[1_{(0,\tau]}(t)=\sum^{n-1}_{j=0}1_{\Brace{\tau\ge t_{j+1}}}1_{(t_j,t_{j+1}]}(t)\]
        と表せるが,いま$\tau$は離散値だから$\Brace{\tau\ge t_{j+1}}=\Brace{\tau\le t_j}\in\F_{t_j}$が成り立ち,$1_{(0,\tau]}$は単過程である.
        よって,$u1_{[0,\tau]}=u(1_{\{0\}}+1_{[0,\tau]})\in L^2(\P)$.
        そしてこのとき,
        \begin{align*}
            I(u1_{[0,\tau]})&=F\sum^{n-1}_{j=0}1_{\Brace{\tau\ge t_{j+1}}}I(1_{(t_j,t_{j+1}]\cap(a,b]})\\
            &=F\sum^n_{i=1}1_{t=t_i}I(1_{(a,b]\cap[0,t_i]})\\
            &=F(B_{b\land\tau}-B_{a\land\tau})=F(B^\tau_b-B^\tau_a)=\int^\tau_0u_tdB_t.
        \end{align*}
        \item[$u$が単過程で$\tau$が一般であるとき] 
        $\tau_n\searrow\tau$を満たす離散値停止時の列$(\tau_n)$が取れる\ref{prop-approximation-of-stopping-time-from-upwards}.
        このとき,右辺は$t\mapsto B_t$の連続性より,左辺は$\tau_n\searrow\tau$は$L^2(\Om)$-収束もするので,$I$の連続性より従う.
        \item[$u$も一般であるとき] 
        一般の$u\in L^2(\P)$には,単過程列$\{u^{(n)}\}\subset\E$が存在して,$u^{(n)}\xrightarrow{L^2(\P)}u$.
        このとき,$T=\infty$の場合の最大不等式より,
        \begin{align*}
            E\Square{\Abs{\int^\tau_0u_t^{(n)}dB_t-\int^\tau_0u_tdB_t}^2}&\le E\Square{\sup_{t\in\R_+}\Abs{\int^t_0(u_s^{(n)}-u_s)dB_s}^2}\\
            &\le 4E\Square{\int^\infty_0\abs{u_s^{(n)}-u_s}^2ds}=4\norm{u_s^{(n)}-u_s}_{L^2(\P)}.
        \end{align*}
        よって,
        たしかに
        \[\int^\tau_0u_t^{(n)}dB_t\xrightarrow{L^2(\Om)}\int^\tau_0u_tdB_t\]
        が成り立つ.
    \end{description}
\end{Proof}
\begin{remarks}
    $\tau$で剪断した過程$u_t1_{[0,\tau]}$の確率積分$I(u_t1_{[0,\tau]})$は,任意の$\om\in\Om$に対して,積分区間$[0,\tau(\om)]$とそこでの確率積分の値$I(u1_{[0,\tau(\om)]})(\om)$を同時に考慮する確率変数に等しい,というある種の可換性の成立を示唆する消息である.
\end{remarks}

\section{局所マルチンゲールへの延長}

\begin{tcolorbox}[colframe=ForestGreen, colback=ForestGreen!10!white,breakable,colbacktitle=ForestGreen!40!white,coltitle=black,fonttitle=\bfseries\sffamily,
title=マルチンゲールに軸を取り替えて更なる延長をする]
    定積分$I(u)\in L^2(\Om)$ではなく,不定積分のマルチンゲールに注目すれば,より一般的な非積分関数$u\in L^2_\loc(\P)$に関して
    確率積分$I:L^2_\loc(\P)\to\M^{0,c}_\loc$が定まる.
    いままでは,マルチンゲール$\M^0$には必ず極限が存在したために,自然な$L^2(\P)\to L^2(\Om)$が存在していたのである.
\end{tcolorbox}

\subsection{二次変動の到達時刻の列}

\begin{definition}
    局所2乗可積分な過程$u\in L^2_\loc(\P)$に対して,
    確率時刻$T_n:\Om\to\o{\R_+}\;(n\in\N)$を,$u$の二次変動過程$\paren{\int^t_0u_s^2ds}_{t\in\R_+}$が$n$に到達する時刻
    \[T_n:=\inf\Brace{t\in\R_+\;\middle|\;\int^t_0u_s^2ds=n}\]
    とする.
\end{definition}

\begin{lemma}
    $T_n$は停止時で,剪断過程$u_t^{(n)}:=u_t1_{[0,T_n]}(t)\in L^2(\P)$は確率可積分である.
\end{lemma}

\subsection{定義と局所マルチンゲール性}


\begin{proposition}\label{prop-stochastic-integral-for-locally-integrable-processes}
    局所2乗可積分な過程$u\in L^2_\loc(\P)$に対して,適合的な連続過程$\paren{\int^t_0u_sdB_s}\in C\cap\bF$が存在して,
    \[\forall_{n\in\N}\;\forall_{t\in[0,T_n]}\;\quad\int^t_0u_s^{(n)}dB_s=\int^t_0u_sdB_s.\]
\end{proposition}
\begin{Proof}
    任意に$m\in\N$を取ると,$n\le m$に対して,$L^2(\P)$の元の停止時までの剪断過程の確率積分は,元の$L^2(\P)$の元の確率積分に停止時を代入したものに等しい\ref{prop-integral-until-stopping}から,
    \[\forall_{t\in[0,T_n]}\quad\int^t_0u_s^{(n)}dB_s=\int^{t\land T_n}_0u^{(m)}_sdB_s=\int^t_0u_s^{(m)}dB_s\]
    そこで,
    \[\int^t_0u_sdB_s:=\int^t_0u_s^{(n)}dB_s\quad\on\;\Brace{t\le T_n}\]
    と定める.これは張り合わせの条件を満たすから,たしかに$t\in\R_+$上に一つの関数を定めている.
    こうして,$M_t:\Om=\cup_{n\in\N}\Brace{t\le T_n}\to C(\R;\R)$が定まった.

    これが適合的な連続過程であることは,
    任意の$t\in\R_+$について,ある$n\in\N$が存在して$t\le T_n$であり,
    \[\int^t_0u_sdB_s=\int^t_0u_s^{(n)}dB_s\]
    は$\F_t$-可測であり,右辺は連続な修正を持つことから判る.
\end{Proof}
\begin{remarks}[駆動過程の剪断]
    非積分関数が違うとき,確率積分の枠組みでは駆動過程である$B_t$に帰着させる必要がある(命題\ref{prop-integral-until-stopping}).
    この,「駆動過程の剪断」を用いてこそ,確率積分の定義を一般化出来たのである.
\end{remarks}

\begin{corollary}[一般化された確率積分の局所マルチンゲール性]
    $u\in L^2_\loc(\P)$の定める不定積分の過程$(M_t)$は連続な局所マルチンゲールである:$M\in{}^0\!M_\loc\cap C$.
    すなわち,ある$T_n\nearrow\infty\;\ae$を満たす停止時の列が存在して,
    各$n\in\N$について,停止過程$M^{T_n}=(M_{t\land T_n})_{t\in\R_+}$は(一様可積分な)マルチンゲールである.
\end{corollary}

\subsection{確率積分の確率連続性}

\begin{tcolorbox}[colframe=ForestGreen, colback=ForestGreen!10!white,breakable,colbacktitle=ForestGreen!40!white,coltitle=black,fonttitle=\bfseries\sffamily,
title=]
    $I:L^2(\P)\to L^2(\Om)$とは違って,もはや等長性は持たないが,
    対応$I:L^2_\loc(\P)\to\M^{0,c}_\loc\overset{\ev_T}{\epi} L^1(\Om)$は確率連続である.
\end{tcolorbox}

\begin{proposition}
    $u\in L^2_\loc(\P)$とする.このとき,
    \[\forall_{K,\delta,T>0}\quad P\Square{\Abs{\int^T_0u_sdB_s}\ge K}\le P\Square{\int^T_0u_s^2ds\ge\delta}+\frac{\delta}{K^2}.\]
\end{proposition}

\begin{corollary}[確率積分の連続性]
    列$\{u^{(n)}\}\subset L^2_\loc(\P)$は$u\in L^2_\loc(\P)$に(各点$T\in\R_+$で)確率収束するとする:
    \[\forall_{\ep>0}\;\forall_{T>0}\;\lim_{n\to\infty}P\Square{\Abs{\int^T_0(u^{(n)}_s-u_s)^2dx}>\ep}=0.\]
    このとき,任意の$T>0$について,
    \[\int^T_0u^{(n)}_sdB_s\xrightarrow{p}\int^T_0u_sdB_s.\]
\end{corollary}

\subsection{局所マルチンゲールの二次変分}

\begin{tcolorbox}[colframe=ForestGreen, colback=ForestGreen!10!white,breakable,colbacktitle=ForestGreen!40!white,coltitle=black,fonttitle=\bfseries\sffamily,
title=]
    一般化された確率積分$I:L^2_\loc(\P)\to\M^{0,c}_\loc$についても,二次変分過程が存在し,
    \[\paren{\int_0^tu^2_sdB_s}_{t\in\R_+}\]
    である.これは,まず$L^2_\infty(\P)$にある場合に計算でき,次に$L^2_\loc(\P)$であるときはカットオフ$(-n)\lor f\land n\in L^2_\infty(\P)$を考えることで同様に示せる.
\end{tcolorbox}

\begin{proposition}\label{prop-2nd-order-variance-of-indefinite-processes}
    $u\in L^2_\loc(\P)$の不定積分の過程$(M_t)$について,
    \[\forall_{t\in\R_+}\;\forall_{\pi_n:=\Brace{0=t_0<t_1<\cdots<t_n=t}}\quad\sum^{n-1}_{j=0}\paren{\int^{t_{j+1}}_{t_j}u_sdB_s}^2\xrightarrow[\abs{\pi_n}\to0]{P}\int^t_0u_s^2ds.\]
\end{proposition}

\section{伊藤の公式}

\begin{notation}\mbox{}
    \begin{enumerate}
        \item $L^2_\loc(\P):=\Brace{u\in\L(\P)\;\middle|\;\forall_{t\in\R_+}\;\int^t_0u^2_sds<\infty\;\as}$.
        \item $L^1_\loc(\P):=\Brace{\nu\in\L(\P)\;\middle|\;\forall_{t>0}\int^t_0\abs{\nu_s}ds<\infty\;\as}$.
        \item $C^{1,2}(\R_+\times\R)$で,$t\in R_+$について1階,$x\in \R$について2階連続微分可能な関数全体のなす空間とする.
    \end{enumerate}
\end{notation}

\subsection{伊藤過程の定義と例}

\begin{definition}
    適合的で連続な実過程$X\in C\cap\bF$が\textbf{伊藤過程}であるとは,
    \[\exists_{u\in L^2_\loc(\P)}\;\exists_{\nu\in L^1_\loc(\P)}\;\exists_{X_0\in\L_{\F_0}(\Om)}\quad X_t=X_0+\int^t_0u_sdB_s+\int^t_0v_sds\]
    が成り立つことをいう.
    ただし,$(\F_t)$は右連続な情報系としたから,$\F_0=2$より,$\exists_{c\in\R}\;X_0=c\;\as$に注意\ref{lemma-Blumenthal}.
\end{definition}

\begin{example}
    $B^2_t$は伊藤過程である\ref{exp-integral-of-Brownian-motion}:
    \[B^2_t=\int^t_02B_sdB_s+t.\]
    一般に,可微分関数$f\in C^2(\R)$について,$f(B_t)$と表せる過程は伊藤過程であり,確率不定積分($t$について局所マルチンゲールな成分)と可微分な見本道を持つ過程との和に分解出来る.
\end{example}

\subsection{伊藤過程の二次共変分}

\begin{tcolorbox}[colframe=ForestGreen, colback=ForestGreen!10!white,breakable,colbacktitle=ForestGreen!40!white,coltitle=black,fonttitle=\bfseries\sffamily,
title=]
    $dB_s$と$ds$で絶妙に次元が違うことの消息が宿っており,後に出る計算規則
    \begin{center}
    \begin{tabular}{c|cc}\hline
    $\times$&$dB_t$&$dt$\\\hline
    $dB_t$&$dt$&$0$\\
    $dt$&$0$&$0$\\\hline
    \end{tabular}
    \end{center}
    の基となる.
\end{tcolorbox}

\begin{notation}[共変分]
    伊藤過程
    \[X_t=X_0+\int^t_0u^X_sdB_s+\int^t_0v_s^Xds\]
    に対して,
    \[M^X_t:=\int^t_0u^X_sdB_s,\quad A^X_t:=\int^t_0v^X_sds.\]
    とおき,
    \[\brac{X,Y}_t:=\lim_{\abs{\pi_n}\to0}\sum_{j\in n}(X_{t_{j+1}}-X_{t_j})(Y_{t_{j+1}}-Y_{t_j})\]
    と定めると,これは極限として定めたので,明らかに($\om\in\Om$を止めるごとに)双線型写像である.
\end{notation}

\begin{proposition}[伊藤過程の二次共変分]\label{prop-quadratic-covariation-of-ito-process}
    $X,Y$を伊藤過程とする:
    \[X_t=X_0+\int^t_0u^X_sdB_s+\int^t_0v_s^Xds,\qquad Y_t=Y_0+\int^t_0u_s^YdB_s+\int^t_0v_s^Yds.\]
    このとき,$X,Y$の二次共変分は,局所マルチンゲール$M_t^X,M_t^Y$の二次共変分に一致する:
    \[\brac{X,Y}_t=\int^t_0u_s^Xu_s^Yds=\Brac{\int^t_0u_s^XdB_s,\int^t_0u_s^YdB_s}_t\]
    すなわち,次が成り立つ:
    \[\forall_{t\in\R_+}\;\forall_{\pi_n=\Brace{0=t_0<t_1<\cdots<t_n=t}}\quad\sum_{j\in n}(Y_{t_{j+1}}-Y_{t_j})(X_{t_{j+1}}-X_{t_j})\xrightarrow[\abs{\pi_n}\to0]{P}\int^t_0u^X_su^Y_sds.\]
\end{proposition}
\begin{Proof}
    この極限$\brac{X,Y}$が存在するならば,その線形性より,
    \[\brac{X|Y}=\brac{M^X|M^Y}+\brac{M^X|A^Y}+\brac{A^X|M^Y}+\brac{A^X|A^Y}\]
    である.後ろの3項が零であることを示す.
    \begin{enumerate}
        \item $A^X,A^Y$は$t\in\R_+$について絶対連続であるから特に$[0,t]$上一様連続であり,それと三角不等式より,
        \begin{align*}
            \sum_{j\in n}(A^X_{t_{j+1}}-A^X_{t_j})(A^Y_{t_{j+1}}-A^Y_{t_j})&\le\sup_{\abs{s-u}\le\abs{\pi_n},s,u\in[0,t]}\abs{A^X_u-A^X_s}\sum_{j\in n}\abs{A^Y_{t_{j+1}}-A^Y_{t_j}}\\
            &\le\sup_{\abs{s-u}\le\abs{\pi_n},s,u\in[0,t]}\abs{A^X_u-A^X_s}\int^t_0\abs{v_s^Y}ds.
        \end{align*}
        $v_s^Y\in L^1_\loc(\P)$より,右辺は任意の$t\in\R_+$について有限.よって,$\abs{\pi_n}\to0$の極限で,
        \[\sum_{j\in n}(A^X_{t_{j+1}}-A^X_{t_j})(A^Y_{t_{j+1}}-A^Y_{t_j})\xrightarrow{P}0.\]
        よって$\brac{A^X|A^Y}$は存在し,$=0$.
        \item $M^X$は$t\in\R_+$について連続であるから,特に$[0,t]$上一様連続であるから,同様にして
        \begin{align*}
            \sum_{j\in n}(M^X_{t_{j+1}}-M^X_{t_j})(A^Y_{t_{j+1}}-A^Y_{t_j})&\le\sup_{\abs{s-u}\le\abs{\pi_n},s,u\in[0,t]}\abs{M^X_u-M^X_s}\sum_{j\in n}\abs{A^Y_{t_{j+1}}-A^Y_{t_j}}\\
            &\le\sup_{\abs{s-u}\le\abs{\pi_n},s,u\in[0,t]}\abs{M^X_u-M^X_s}\int^t_0\abs{v_s^Y}ds.
        \end{align*}
        同様にして,$\brac{M^X|A^Y}=\brac{A^X|M^Y}=0$.
        \item 最後に,第1項は,$u^X+u^Y\in L^2_\loc(\P)$であるから,これについての二次変分過程\ref{prop-2nd-order-variance-of-indefinite-processes}は
        \begin{align*}
            \sum_{j\in n}\paren{\int^{t_{j+1}}_{t_j}u_s^XdB_s}^2&+\sum_{j\in n}\paren{\int^{t_{j+1}}_{t_j}u_s^YdB_s}^2+2\sum_{j\in n}\paren{\int^{t_{j+1}}_{t_j}u_s^XdB_s}\paren{\int^{t_{j+1}}_{t_j}u_s^YdB_s}=\sum_{j\in n}\paren{\int^{t_{j+1}}_{t_j}(u_s^X-u_s^Y)dB_s}^2\\
            &\xrightarrow{P}\int^t_0(u^X_s+u^Y_s)^2ds=\int^t_0((u_s^X)^2+(u^Y_s)^2+2u_s^Xu_s^Y)ds.
        \end{align*}
        よって,連続写像定理より,
        \[\sum_{j\in n}\paren{\int^{t_{j+1}}_{t_j}u_s^XdB_s}\paren{\int^{t_{j+1}}_{t_j}u_s^YdB_s}\xrightarrow{P}\int^t_0u_s^Xu_s^Yds.\]
    \end{enumerate}
    以上より,
    \[\forall_{t\in\R_+}\;\forall_{\pi_n=\Brace{0=t_0<t_1<\cdots<t_n=t}}\quad\sum_{j\in n}(Y_{t_{j+1}}-Y_{t_j})(X_{t_{j+1}}-X_{t_j})\xrightarrow[\abs{\pi_n}\to0]{P}\int^t_0u^X_su^Y_sds.\]
\end{Proof}

\subsection{伊藤過程の確率微分形式}

\begin{definition}[stochastic differential]
    \[X_t=X_0+\int^t_0u_sdB_s+\int^t_0v_sds\]
    を伊藤過程とする.
    \begin{enumerate}
        \item これを$dX_t=u_sdB_s+v_sds$と略記し,これを\textbf{確率微分形式による表示}という.
        \item 2つの確率微分形式$dX,dY$の\textbf{積}とは,互いの二次共変分過程を表すとする:
        \[dX\cdot dY:=d\brac{X,Y}.\]
    \end{enumerate}
\end{definition}

\subsection{確率積分の定めるマルチンゲール差分列}

\begin{tcolorbox}[colframe=ForestGreen, colback=ForestGreen!10!white,breakable,colbacktitle=ForestGreen!40!white,coltitle=black,fonttitle=\bfseries\sffamily,
title=]
    $[0,t]$の任意の分割$\{t_i\}\subset[0,t]$に対して,
    \[\ep_i:=\int^{t_{i+1}}_{t_i}u_s^2ds-\paren{\int^{t_{i+1}}_{t_i}u_sdB_s}^2\]
    はマルチンゲール差分列を定める:$E[\ep_i|\F_{t_i}]=0$.
    そもそも$\lim_{\abs{\pi_n}\to0}\sum_{i\in n}\paren{\int^{t_{i+1}}_{t_i}u_sdB_s}^2=\int^t_0u_s^2ds$であるが,この収束が起こるに当たって,小さな$\ep_i$はmartingale差分になっているので,足し合わせると零になる,ということが起こっているのである.
\end{tcolorbox}

\begin{proposition}\label{prop-martingale-difference-produced-from-quadratic-variation}
    任意の$u\in L^2(\P),[a,b]\subset\R_+$について,
    \[\ep:=\int^b_au_s^2ds-\paren{\int^b_au_sdB_s}^2\]
    は$E[\ep|\F_a]=0$を満たす.
\end{proposition}
\begin{Proof}\mbox{}
    \begin{enumerate}
        \item $u\in\E$を単過程とする:
        \[u=\sum_{j=0}^{k-1}\phi_j1_{(t_j,t_{j+1}]}\quad\phi_j\in L^2(\F_{t_j}),a=t_0<\cdots<t_k=b.\]
        とすると,
        \[\ep=\sum_{j=0}^{k-1}\phi_j^2(t_{j+1}-t_j)-\paren{\sum_{j=0}^{k-1}\phi_j(B_{t_{j+1}}-B_{t_j})}^2.\]
        この第2項の条件付き期待値は,
        \begin{align*}
            E\Square{\paren{\sum_{j=0}^{k-1}\phi_j(B_{t_{j+1}}-B_{t_j})}^2\;\middle|\;\F_a}&=E\Square{\sum_{j\in k}\phi_j^2(B_{t_{j+1}}-B_{t_j})^2\;\middle|\;\F_a}+2E\Square{\sum_{j>i}\phi_i\phi_j(B_{t_{i+1}}-B_{t_i})(B_{t_{j+1}}-B_{t_j})\;\middle|\;\F_a}\\
            &=\sum_{j\in k}E[\phi_j^2E[(B_{t_{j+1}}-B_{t_j})^2|\F_{t_j}]|\F_a]+2\sum_{j>i\in k}E[\phi_i\phi_j(B_{t_{i+1}}-B_{t_i})E[B_{t_{j+1}}-B_{t_j}|\F_{t_j}]|\F_a]\\
            &=\sum_{j\in k}E[\phi_j^2(t_{j+1}-t_j)|\F_a].
        \end{align*}
        と整理出来る.
        これより,$E[\ep|\F_a]=0$が従う.
        \item 一般の$u\in L^2(\P)$を取ると,単過程列$\{u^{(n)}\}\subset\E$で$u^{(n)}\xrightarrow{L^2(\P)}u$を満たすものが取れる.
        これについての
        \[\ep^{(n)}:=\sum_{j\in k^{(n)}}(\phi_j^{(n)})^2(t_{j+1}-t_j)-\paren{\sum_{j\in k^{(n)}}\phi_j^{(n)}(B_{t_{j+1}}-B_{t_j})}^2\]
        も$\ep^{(n)}\xrightarrow{L^1(\Om)}\ep$が示せれば,条件付き期待値も収束する$0=E[\ep^{(n)}|\F_a]\xrightarrow{L^1(\Om)}E[\ep|\F_a]=0$ことが判る.
        これはCauchy-Schwarzの不等式から
        \begin{align*}
            \norm{\ep-\ep^{(n)}}_1&\le\Norm{\int^b_au_s^2ds-\int^b_a(u_s^{(n)})^2ds}_1+\Norm{\paren{\int^b_au_sdB_s}^2-\paren{\int^b_su^{(n)}dB_s}^2}_1\\
            &\le \norm{u-u^{(n)}}_{2}\norm{u+u^{(n)}}_2+\norm{u-u^{(n)}}_2\norm{u+u^{(n)}}_2\xrightarrow{n\to\infty}0.
        \end{align*}
        より判る.
    \end{enumerate}
\end{Proof}

\begin{corollary}
    任意の$u\in L^2_\loc(\P),[a,b]\subset\R_+$について,$E[\ep|\F_a]=0$.
\end{corollary}
\begin{Proof}
    $T_n:=\inf\Brace{t\in\R_+\;\middle|\;\int^t_0u_s^2ds=n}$による
    剪断過程$\{u_t^{(n)}\}\subset L^2(\P)$の列について,命題\ref{prop-stochastic-integral-for-locally-integrable-processes}より,
    \[\forall_{n\in\N}\;\forall_{t\in[0,T_n]}\;\quad\int^t_0u_s^{(n)}dB_s=\int^t_0u_sdB_s.\]
    よって,任意の$[a,b]\subset\R_+$に対して十分大きな$n\in\N$を取れば,成り立つことが判る.
\end{Proof}

\subsection{伊藤の公式}

\begin{tcolorbox}[colframe=ForestGreen, colback=ForestGreen!10!white,breakable,colbacktitle=ForestGreen!40!white,coltitle=black,fonttitle=\bfseries\sffamily,
title=]
    Brown運動$B$の汎関数$f\in C^{1,2}(\R_+\times\R)$について,
    \[f(t,B_t)=f(0,0)+\int^t_0\pp{f}{x}(s,B_s)dB_s+\int^t_0\paren{\pp{f}{t}(s,B_s)+\frac{1}{2}\pp{^2f}{x^2}(s,B_s)}ds.\]
    一般の伊藤過程$X$については,
    \[df(t,X_t)=\pp{f}{t}(t,X_t)dt+\pp{f}{x}(t,X_t)\underbrace{dX_t}_{=u_tdB_t+v_tdt}+\frac{1}{2}\pp{^2f}{x^2}(t,X_t)\underbrace{(dX_t)^2}_{=u_t^2dt}\]
    に従うから,
    \[f(t,X_t)=f(0,X_0)+\int^t_0\pp{f}{t}(t,X_s)ds+\int^t_0\pp{f}{x}(s,X_s)(u_sdB_s+v_sds)+\frac{1}{2}\int^t_0\pp{^2f}{x^2}(s,X_s)u^2_sds.\]
\end{tcolorbox}

\begin{theorem}[Ito's formula]
    $f\in C^{1,2}(\R_+\times\R)$,$X$を伊藤過程とする.
    このとき,$Y_t:=f(t,X_t)$は再び伊藤過程で,次のような修正を持つ:
    \[\forall_{t\in\R_+}\;Y_t\overset{\as}{=}f(0,X_0)+\int^t_0\pp{f}{t}(s,X_s)ds+\int^t_0\pp{f}{x}(s,X_s)u_sdB_s+\int^t_0\pp{f}{x}(s,X_s)v_sds+\frac{1}{2}\int^t_0\pp{^2f}{x^2}(s,X_s)u_s^2ds.\]
\end{theorem}
\begin{remarks}\mbox{}
    \begin{enumerate}
        \item $X$による整理:前の命題から,共変分の$\brac{X}_t=\int^t_0u^2_sds=\Brac{\int^t_0u_sdB_s}_t$の記法を採用すると,
        \[Y_t=f(0,X_0)+\int^t_0\pp{f}{t}(s,X_s)ds+\int^t_0\pp{f}{x}(s,X_s)dX_s+\frac{1}{2}\int^t_0\pp{^2f}{x^2}(s,X_s)d\brac{X}_s\]
        と表記できる.微分形式の記法$dX_t:=u_tdB_t+v_tdt,d\brac{X}_t:=u^2_tdt$に注意.特に非自明なのは,$\dd{}{t}\int^t_0u_sdB_s=u_t\dd{B_t}{t}$.
        \item 二次変分$d\brac{X}$の$(dX)^2$への書き換え:すると,形式的計算規則$(dB_t)^2=dt,dB_tdt=dtdB_t=dtdt=0$に従うことで,\textbf{伊藤の公式の微分形}
        \[df(t,X_t)=\pp{f}{t}(t,X_t)dt+\pp{f}{x}(t,X_t)dX_t+\frac{1}{2}\pp{^2f}{x^2}(t,X_t)(dX_t)^2\]
        とも表せる.なお,計算規則より,$(dX_t)^2=u^2_tdt=d\brac{X}_t$である.
    \end{enumerate}
\end{remarks}

\begin{lemma}[伊藤の公式の証明(簡易版)]
    \[X_t=X_0+\int^t_0u^X_sdB_s+\int^t_0v_s^Xds\]
    について,
    \begin{enumerate}
        \item それぞれの過程はwell-definedである:
        \[\pp{f}{t}(s,X_s),\pp{f}{x}(s,X_s)v_s,\pp{^2f}{x^2}(s,X_s)u_s^2\in L^1_\loc(\P),\quad\pp{f}{x}(s,X_s)u_s\in L^2_\loc(\P).\]
        \item $v=0,\pp{f}{t}=0$とする.このとき,$X_t=X_0+\int^t_0u_sdB_s$で,$f$は第一引数$t$について定数で,
        \[f(X_t)=f(X_0)+\int^t_0f'(X_s)u_sdB_s+\frac{1}{2}\int^t_0f''(X_s)u_s^2ds.\]
    \end{enumerate}
\end{lemma}
\begin{Proof}\mbox{}
    \begin{enumerate}
        \item $\pp{f}{t}(s,X_s)$は$s\in[0,t]$に関する連続関数であるから,$[0,t]$上有界,特に$[0,t]$上可積分である.
        $\pp{f}{x}(s,X_s)v_s,\pp{^2f}{x^2}(s,X_s)u_s^2$はいずれも$[0,t]$上の有界関数と可積分関数の積であるから,Holderの不等式より可積分である.
        $\pp{^2f}{x^2}(s,X_s)u_s$はこの2乗がその形になっている.
        \item \mbox{}
        \begin{description}
            \item[方針] $f,u_s$に関する条件
            \begin{enumerate}
                \item $f\in C^2_b(\R)$と有界で,
                \item $\abs{X_0}\le N$を満たす任意に取った$N\in\N^+$について,
                $\int^\infty_0u_s^2ds<N,\sup_{t\in\R_+}\abs{X_t}\le N$を満たす.特に,$u_s\in L^2(\P)$.
            \end{enumerate}
            の下で示す.すると,一般の$u_s\in L^2_\loc(\P)$と$f\in C^2(\R)$について,
            \[T_N:=\inf\Brace{t\in\R_+\;\middle|\;\int^t_0u^2_sds\ge N\lor\abs{X_t}\ge N}\]
            を停止時の列とすると$T_N\nearrow\infty$であり,$u^{(N)}_s:=1_{[0,T_N]}u_s$は(b)を満たす.さらに,
            $\{f_N\}\subset C_c^2(\R)$を$\forall_{x\in[-N,N]}\;f(x)=f_N(x)$を満たす関数の列とすると,広義一様に$f_N\to f$であり,$f_N$は(a)を満たすから,伊藤の公式より,
            \[\forall_{t\in\R_+}\;f_N(X_t^{(N)})\overset{\as}{=}f_N(X_0^{(N)})+\int^t_0f_N'(X_s^{(N)})u_sdB_s+\frac{1}{2}\int^t_0f''_N(X_s^{(N)})u^2_sds,\quad X_t^{(N)}:=X_0+\int^t_0u_s^{(N)}dB_s.\]

            さて,ここで発展的可測な過程-$L^2$:$u_s\in L^2_\loc(\P)$に関する確率積分の定義\ref{prop-stochastic-integral-for-locally-integrable-processes}から,
            \[\forall_{N\in\N}\;\forall_{t\in [0,T_N']}\;\int^t_0u_s1_{[0,T_N']}dB_s=\int^t_0u_sdB_s,\quad T'_N:=\inf\Brace{t\in\R_+\;\middle|\;\int^t_0u_s^2ds\ge N}.\]
            ここで,$T_N\le T_N'$より,$[0,T_N]$上では$1_{[0,T_N]}=1_{[0,T_N']}$であることに注意すれば,命題\ref{prop-integral-until-stopping}より,
            \[\int^t_0u_s1_{[0,T_N']}dB_s=\int^{t\land T_N'}_0u_s1_{[0,T_N]}dB_s=\int^{t\land T_N}_0u_s1_{[0,T_N]}dB_s=\int^t_0u_s1_{[0,T_N]}dB_s.\]
            2つを併せて
            \[\forall_{N\ge 1}\;\forall_{t\in[0,T_N]}\;\int^t_0u_s1_{[0,T_N]}dB_s=\int^t_0u_sdB_s\]
            が成り立つが,これに$t=t\land T_N\in[0,T_N]$を代入しても問題ない.これより,
            \[X_t^{(N)}=X_0+\int^{T_N\land t}_0u_sdB_s=X_{T_N\land t}\]
            を得る.したがってこれを上の式に代入し,注意深く変形すると
            \[\forall_{N\in\N}\;\forall_{t\in\R_+}\quad f(X_{T_N\land t})=f(X_0)+\int^{T_N\land t}_0f'(X_s)u_sdB_s+\frac{1}{2}\int^{T_N\land t}_0f''(X_s)u_s^2ds.\]
            $N\to\infty$としても,この等式は$\as$に成り立ち続ける.
            \item[Taylorの定理による展開] $[0,t]$の等分割$t_i:=\frac{it}{n}$を考えると,各区間$[t_{i},t_{i+1}]$における2次のTaylorの定理より,確率変数の列$\wt{X}_i\in\paren{\inf_{u\in[t_i,t_{i+1}]}X_{u},\sup_{u\in[t_i,t_{i+1}]} X_{u}}$が存在して,
            \[f(X_t)-f(X_0)=\sum_{i\in n}f'(X_{t_i})(X_{t_{i+1}}-X_{t_i})+\frac{1}{2}\sum_{i\in n}f''(\wt{X}_i)(X_{t_{i+1}}-X_{t_i})^2.\]
            右辺の2項が公式の右辺の形に$n\to\infty$の極限で確率収束することが示せれば,概収束する部分列を取ることが出来ることを意味するため,伊藤の公式の成立が確認できる.
            \begin{enumerate}
                \item まず第一項について,$\sum_{i\in n}f'(X_{t_i})(X_{t_{i+1}}-X_{t_i})\xrightarrow{L^2(\Om)}\int^t_0f'(X_s)u_sdB_s$である.
                
                実際,確率積分は線型であることと,$f'(X_{t_i})$は確率積分の内外に自由に移動できること\ref{prop-algebraic-property-of-indefinite-integral-process}に注意すれば,
                \begin{align*}
                    \Norm{\sum_{i\in n}\paren{f'(X_{t_i})\int^{t_{i+1}}_{t_i}u_sdB_s-\int^{t_{i+1}}_{t_i}f'(X_s)u_sdB_S}}^2_2&=\Norm{\sum_{i\in n}\int^{t_{i+1}}_{t_i}(f'(X_{t_i})-f'(X_s))u_sdB_s}^2_2\\
                    &\le 2\sum_{i\in n}\Norm{\int^{t_{i+1}}_{t_i}(f'(X_{t_i})-f'(X_s))u_sdB_s}^2_2\\
                    &=2\sum_{i\in n}E\Square{\int^{t_{i+1}}_{t_i}(f'(X_{t_i})-f'(X_s))^2u_s^2ds}\\
                    &\le2\sum_{i\in n}E\Square{\sup_{\abs{s-u}\le t/n}(f'(X_s)-f'(X_u))^2\int^{t_{i+1}}_{t_i}u_s^2ds}\xrightarrow{n\to\infty}0.
                \end{align*}
                最後の収束は,$f'(X_s)$の$s\in[0,t]$上の一様連続性と,$\int^{t_{i+1}}_{t_i}u_s^2ds\le N<\infty$の仮定による.
                \item 次に第二項について,$\sum_{i\in n}f''(\wt{X}_i)(X_{t_{i+1}}-X_{t_i})^2\xrightarrow{P}\int^t_0f''(X_s)u^2_sds$である.
                実際,両辺を引き,うまく適切な項を加えると,次の3項の和とみなせる:
                \begin{align*}
                    \int^t_0f''(X_s)&u^2_sds-\sum_{i\in n}f''(\wt{X}_i)(X_{t_{i+1}}-X_{t_i})^2\\
                    &=\sum_{i\in n}\int^{t_{i+1}}_{t_i}(f''(X_s)-f''(X_{t_i}))u^2_sds\\
                    &\qquad+\sum_{i\in n}f''(X_{t_i})\paren{\int^{t_{i+1}}_{t_i}u^2_sds-\paren{\int^{t_{i+1}}_{t_i}u_sdB_s}^2}\\
                    &\qquad+\sum_{i\in n}(f''(X_{t_i})-f''(\wt{X}_i))\paren{\int^{t_{i+1}}_{t_i}u_sdB_s}^2\\
                    &=:A_1^{(n)}+A_2^{(n)}+A_3^{(n)}.
                \end{align*}
                \begin{enumerate}[({第}1{項})]
                    \item 次のように評価できる:
                    \[\abs{A_1^{(n)}}=\Abs{\sum_{i\in n}\int^{t_{i+1}}_{t_i}(f''(X_s)-f''(X_{t_i}))u_s^2ds}\le\sup_{\abs{s-r}\le t/n}\abs{f''(X_s)-f''(X_r)}\int^t_0u_s^2ds\xrightarrow[\as]{n\to\infty}0.\]
                    最後の収束は$f''(X_s)$の$s\in[0,t]$上での一様連続性による.
                    \item 第2項は$\xi_i:=\int^{t_{i+1}}_{t_i}u^2_sds-\paren{\int^{t_{i+1}}_{t_i}u_sdB_s}^2$とおくと,
                    \[A_2^{(n)}:=\sum_{i\in n}f''(X_{t_i})\xi_i\]
                    と表せる.このとき,$E[\xi_i|\F_{t_i}]=0$であり\ref{prop-martingale-difference-produced-from-quadratic-variation},
                    またBurkholderの不等式\ref{cor-Burkholder-for-the-martingale-of-SI}から確率変数$\int_0^tu_sdB_s$の任意次の積率は存在し,特に$L^2$-有界であるから,
                    \[\forall_{i<j\in n}\; E[f''(X_{t_i})\ep_if''(X_{t_j})\ep_j]=E[E[f''(X_{t_i})\ep_if''(X_{t_j})\ep_j|\F_{t_j}]]=E[f''(X_{t_i})\ep_if''(X_{t_j})E[\ep_j|\F_{t_j}]]=0.\]
                    と$f''\in C_b(\R)$にに注意して,
                    \begin{align*}
                        \norm{A_2^{(n)}}_2^2&=\sum_{i\in n}E[f''(X_{t_i})^2\xi_i^2]+2\sum_{i<j\in n}E[f''(X_{t_i})\ep_if''(X_{t_j})\ep_j]\\
                        &=\sum_{i\in n}E[f''(X_{t_i})^2\xi_i^2]\le\norm{f''}^2_\infty\sum_{i\in n}E[\xi_i^2]\\
                        &\le\norm{f''}^2_\infty\sum_{i\in n}E\Square{\paren{\int^{t_{i+1}}_{t_i}u_s^2ds}^2+\paren{\int^{t_{i+1}}_{t_i}u_sdB_s}^4}\\
                        &\le\norm{f''}^2_\infty E\Square{N\sup_{i\in n}\int^{t_{i+1}}_{t_i}u_s^2ds+\sup_{i\in n}\abs{X_{t_{i+1}}-X_{t_i}}^2\sum_{i\in n}\paren{\int^{t_{i+1}}_{t_i}u_sdB_s}^2}\xrightarrow{n\to\infty}0.
                    \end{align*}
                    最後の収束は,$X_s$の$s\in[0,t]$上の一様連続性と,$\int^{t_{i+1}}_{t_i}u_s^2ds$の絶対連続性と,
                    $\sum_{i=0}^{n-1}\paren{\int^{t_{i+1}}_{t_i}u_sdB_s}^2$が有界であるため:
                    \[\sum_{i=0}^{n-1}\paren{\int^{t_{i+1}}_{t_i}u_sdB_s}^2\xrightarrow{P}\int^t_0u_s^2ds.\]
                    \item 次のように評価できる:
                    \[\abs{A_3^{(n)}}=\Abs{\sum_{i\in n}(f''(X_{t_i})-f''(\wt{X}_i))\paren{\int^{t_{i+1}}_{t_i}u_sdB_s}^2}\le\sup_{i\in n}\abs{f''(X_{t_i})-f''(\wt{X}_i)}\sum_{i\in n}\paren{\int^{t_{i+1}}_{t_i}u_sdB_s}^2\xrightarrow{P}0.\]
                    最後の収束は,有界項$\sum_{i=0}^{n-1}\paren{\int^{t_{i+1}}_{t_i}u_sdB_s}^2$が確率収束についてしか有界であることを確認していないためである.
                \end{enumerate}
            \end{enumerate}
            \item[結論] 以上の収束は,ある部分列を取ることで全て$\as$の意味で成立させることが出来ることから,伊藤の公式を得る.
        \end{description}
    \end{enumerate}
\end{Proof}
\begin{remarks}\mbox{}
    \begin{enumerate}
        \item 2つの有界性条件$f\in C^2_b(\R)$と$\forall_{\om\in\Om}\;\int^\infty_0u_s^2(\om)ds<\infty$を課して良いのは,この状況下で結果が得られれば,前者は$C^2(\R)$で$C_c^2(\R)$が(広義一様位相について)稠密であること,後者の$u_s\in L^2_\loc(\P)$の確率積分はそもそも$\infty$に発散する停止過程の言葉で定義したので,$u\in L^2(\P)$の範囲で示せば十分なのである.
        \item この条件の下で,伊藤の公式の本質はTaylorの定理であり,1次項が$f'(X_s)dX_s$を産み,2次項が$f''(X_s)dX_s^2$を産む.
        いずれも,$t\in\R_+$毎に示せば良いから,$f'(X_s),f''(X_s)$の$s\in[0,t]$に関する一様連続性の過程により,緻密に評価することとなる.
        二階微分の項について,$A_1^{(n)}$は代表点$t_k$の取り方によるズレであり,一階微分の場合と全く同様の議論になる.
        $A_3^{(n)}$はTaylorの定理の代表点$\wt{X}_i$の取り方によるズレであり,こちらの場合も似た議論になる.
        しかしこの2つのズレを繋ぐ$A_2^{(n)}$はmartingale differenceとしての構造を見る必要があるが,確率積分は「等長」というか連続なので,結局は収束する.
    \end{enumerate}
\end{remarks}

\begin{lemma}[伊藤の公式の証明(完全版)]
    一般の
    \[X_t=X_0+\int^t_0u^X_sdB_s+\int^t_0v_s^Xds\]
    と$v\in L^1_\loc(\P),f\in C^{1,2}(\R_+\times\R)$についても成り立つ:
    \[\forall_{t\in\R_+}\;f(t,X_t)\overset{\as}{=}f(0,X_0)+\int^t_0\paren{\pp{f}{t}+\pp{f}{x}v_s+\frac{1}{2}\pp{^2f}{x^2}u^2_s}ds+\int^t_0\pp{f}{x}u_sdB_s.\]
\end{lemma}

\subsection{系として得られる公式}

\begin{example}[Brown運動の$n$乗公式]\mbox{}\label{exp-n-exponential-of-Brownian-motion}
    \begin{enumerate}
        \item $f(x)=x^2,X_t=B_t=\int^t_0dB_t$とすると,
        \[f(X_t)=B_t^2=2\int^t_0B_tdB_t+t\]
        を得る.$f(x)=x^3$とすると,
        \[B_t^3=3\int^t_0B^2_tdB_t+3\int^t_0B_tdt\]
        を得る.一般に,
        \[\forall_{n\ge2}\quad B^n_t=n\int^t_0B_s^{n-1}dB_s+\frac{n(n-1)}{2}\int^t_0B_s^{n-2}ds.\]
    \end{enumerate}
\end{example}

\begin{example}[局所マルチンゲールの構成公式]\mbox{}\label{exp-local-martingale-from-Ito-formula}
    \begin{enumerate}
        \item \textbf{指数局所マルチンゲール}:$f(t,x):=\exp\paren{ax-\frac{a^2t}{2}}$とすると,
        \begin{align*}
            f(t,B_t)&=f(0,B_0)+\int^t_0a\exp\paren{aB_s-\frac{a^2s}{2}}dB_s+\int^t_0\paren{-\frac{a^2}{2}}\exp\paren{aB_t-\frac{a^2t}{2}}dt+\frac{1}{2}\int^t_0a^2\exp\paren{aB_t-\frac{a^2t}{2}}dt\\
            &=1+a\int^t_0Y_sdB_s.
        \end{align*}
        特に,(ドリフトのない)幾何Brown運動$Y_t=e^{aB_t-a^2t/2}$は確率微分方程式
        \[dY_t=aY_tdB_t,\quad Y_0=1\]
        の解である.$Y'_t=e^{aB_t}$は解ではない!
        これは,$f$が次の関係を満たすためである.
        \[\pp{f}{t}+\frac{1}{2}\pp{^2f}{x^2}=0.\]
        \item 一般に,$f\in C^{1,2}(\R_+\times\R)$は
        上の関係式を満たすとすると,
        \[f(t,B_t)=f(0,0)+\int^t_0\pp{f}{x}(s,B_s)dB_s.\]
        これは再び伊藤過程であるだけでなく,連続な局所マルチンゲールである.
        これが自乗可積分でもあるための条件は$\pp{f}{x}\in L^2_\infty(\P)$を満たすことである.
    \end{enumerate}
\end{example}

\section{局所時間に関する田中の公式}

\begin{tcolorbox}[colframe=ForestGreen, colback=ForestGreen!10!white,breakable,colbacktitle=ForestGreen!40!white,coltitle=black,fonttitle=\bfseries\sffamily,
title=]
    Brown運動の汎関数の例に,Brown運動の局所時間がある.
    これについて,伊藤の公式により,局所時間に関する田中の公式を得る.
\end{tcolorbox}

\begin{definition}[Brownian local time]
    点$x\in\R$におけるBrown運動の\textbf{局所時間}とは,次によって定まる確率場$(L^x_t)_{(x,t)\in\R\times\R_+}$をいう:
    \[L^x_t(\om):=\lim_{\ep\searrow0}\frac{1}{2\ep}l\paren{\Brace{s\in[0,t]\mid\abs{B_s-x}\le\ep}}.\]
\end{definition}

\begin{lemma}\mbox{}
    \begin{enumerate}
        \item (Levy 48) 任意の$(x,t)\in\R\times\R_+$に対して極限$L^x_t\in L(\Om)$は存在する.
        \item $(L_t^x)$には$\R\times\R_+$上連続な修正が存在する.
    \end{enumerate}
\end{lemma}
\begin{remarks}
    次の証明(Trotter 58)は,田中の公式の右辺が連続な修正を持つことを示すことによって,局所時間の存在を示す.
    すなわち,田中の公式とは,存在した場合の必要条件を先に導き,そこから連続な修正を持つことを示すことで存在を導く補題としての立場のものが,
    「公式」として独立したものである.
\end{remarks}
\begin{Proof}\mbox{}
    \begin{description}
        \item[方針] 
    \end{description}
\end{Proof}

\begin{proposition}[局所時間の性質]
    $\R$上の
    測度$\mu_t:\B(\R)\to[0,\infty]\;(t\in\R_+)$を
    \[\mu_t(A):=\int^t_01_{\Brace{B_s\in A}}ds,\quad A\in\B(\R).\]
    で定める.
    \begin{enumerate}
        \item $\mu_t\ll l$.
        \item $\dd{\mu_t}{l}(x)=L^x_t\;\as$
    \end{enumerate}
    すなわち,次が成り立つ:
    \[\forall_{f\in L^\infty(\R)\cup L(\R)_+}\quad\int^t_0f(B_s)ds=\int_\R f(x)L^x_tdx.\]
\end{proposition}
\begin{remarks}
    この観察から,$L^x_t$とは$(B_s)_{s\in[0,t]}$が点$x\in\R$で過ごす時間の割合,形式的には
    \[L^x_t=\int^t_0\delta_x(B_s)ds\]
    と理解出来ることがわかる.
    この性質を持つ連続関数$L:\R\times\R_+\to\R_+$を局所時間として定義し,極限としての定義を性質とすることもできる.
\end{remarks}

\begin{theorem}[Tanaka (63)]\mbox{}
    \begin{enumerate}
        \item $\forall_{(x,t)\in\R\times\R^+}\;\frac{1}{2}L^x_t=(B_t-x)_+-(-x)_+-\int^t_01_{\Brace{B_s>x}}dB_s\;\as$
        \item $\forall_{(x,t)\in\R\times\R^+}\;\frac{1}{2}L^x_t=(B_t-x)_--(-x)_-+\int^t_01_{\Brace{B_s<x}}dB_s\;\as$
    \end{enumerate}
\end{theorem}

\begin{corollary}
    \[\forall_{(x,t)\in\R\times\R^+}\quad L^x_t=\abs{B_t-x}-\abs{x}-\int^t_0\sign(B_s-x)dB_s.\]
\end{corollary}

\section{伊藤の公式の多次元化}

\begin{definition}[Ito process]
    $n$次元の連続な適合過程
    $(X_t)_{t\in\R_+}\in\bF\cap C$が$m$次元Brown運動に駆動される\textbf{$n$次元伊藤過程}であるとは,
    \[X_t=X_0+\int^t_0u_sdB_s+\int^t_0v_sds,\quad v\in M_{n1}(L^1_\loc(\P)),u\in M_{nm}(L^2_\loc(\P)).\]
    という表示を持つことをいう.
\end{definition}

\begin{notation}
    ここで,
    \[X=\begin{pmatrix}X^1\\\vdots\\X^n\end{pmatrix},u_s= \begin{pmatrix}(u_s)^1\\\vdots\\(u_s)^n\end{pmatrix} =\begin{pmatrix}(u_s)^1_1&\cdots&(u_s)^1_m\\\vdots&\ddots&\vdots\\(u_s)^n_1&\cdots&(u_s)^n_m\end{pmatrix},v=\begin{pmatrix}v^1\\\vdots\\v^n\end{pmatrix}\]
    の記法を導入しよう.
\end{notation}

\subsection{多次元版の伊藤の公式}

\begin{tcolorbox}[colframe=ForestGreen, colback=ForestGreen!10!white,breakable,colbacktitle=ForestGreen!40!white,coltitle=black,fonttitle=\bfseries\sffamily,
title=]
    微分系に注目すれば,可微分関数$f\in C^{1,2}(\R_+\times\R^n)$によって次のような変換を受ける,というのが伊藤の公式である:
    \[df(t,X_t)=\pp{f}{t}(t,X_t)dt+\nabla_xf(t,X_t)dX_t+\frac{1}{2}\sum_{i,j\in[n]}\pp{^2f}{x_i\partial x_j}(t,X_t)dX_t^idX_t^j.\]
\end{tcolorbox}

\begin{theorem}[multidimensional version of Ito formula (1942)]
    $X$を$m$次元Brown運動を駆動過程とする$n$次元伊藤過程,
    $f\in C^{1,2}(\R_+\times\R^n)$を関数とする.
    このとき,$Y_t:=f(t,X_t)$も$n$次元伊藤過程で,次のように表示される:
    \begin{align*}
        \forall_{t\in\R_+}\quad Y_t&\overset{\as}{=}f(0,X_0)+\int^t_0\pp{f}{t}(s,X_s)ds+\sum_{i\in[n]}\int^t_0\pp{f}{x_i}(s,X_s)dX_s^i\\
        &\qquad\qquad+\frac{1}{2}\sum_{i,j\in[n]}\int^t_0\pp{^2f}{x_i\partial x_j}(s,X_s)(u_su_s^\top)_{ij}ds.
    \end{align*}
\end{theorem}
\begin{remarks}\mbox{}
    \begin{enumerate}
        \item (二次変動の項の変形) $m$次元伊藤過程$X$の成分$X^i,X^j$の二次共変動を,
        局所マルチンゲール成分$u_sdB_s$の第$i,j$横ベクトルの内積を通じて
        \[\brac{X^i,X^j}_t:=\int^t_0((u_s)^i|(u_s)^j)ds=\int^t_0(u_su_s^\top)_{ij}ds=\int^t_0\sum_{k\in[m]}(u_s)_{ik}(u_s)_{jk}ds=\Brac{\sum_{k\in[m]}\int^t_0(u_s)_{ik}dB_s^k,\sum_{k\in[m]}\int^t_0(u_s)_{jk}dB_s^k}.\]
        と定義出来る.すると,
        \[d\brac{X^i,X^j}_t=(u_su_s^\top)_{ij}ds\]
        と表せるから,伊藤の公式は
        \begin{align*}
            Y_t&=f(0,X_0)+\int^t_0\pp{f}{t}(s,X_s)ds+\sum_{i\in[n]}\int^t_0\pp{f}{x_i}(s,X_s)dX^i_s\\
            &\qquad\qquad+\frac{1}{2}\sum_{i,j\in[n]}\int^t_0\pp{^2f}{x_i\partial x_j}(s,X_s)d\brac{X^i,X^j}_s.
        \end{align*}
        と表せる.
        \item こうして,伊藤の公式の微分形は
        \[df(t,X_t)=\pp{f}{t}(s,X_s)ds+\sum_{i\in[n]}\pp{f}{x^i}(s,X_s)dX_s^i+\frac{1}{2}\sum_{i,j\in[n]}\pp{^2f}{x^i\partial x^j}(s,X_s)d\brac{X^i,X^j}_s.\]
        さらにベクトル解析の記号$\nabla_x=\paren{\pp{}{x^1},\cdots,\pp{}{x^n}}$と,そもそも$dX^i_t=(u^i)_tdB_t=\sum_{k\in[m]}(u^i_k)_tdB^k_t$に気をつけて,計算規則
        \[dX^i_tdX^j_t=((u^i)_tdB_t)((u^j)_tdB_t)=((u^i)_t|(u^j)_t)dt=d\brac{X^i,X^j}_t\]
        を用いて,次のように表される:
        \[dY_t=\pp{f}{t}(t,X_t)dt+\nabla_xf(t,X_t)dX_t+\frac{1}{2}\sum_{i,j\in[n]}\pp{^2f}{x_i\partial x_j}(t,X_t)dX_t^idX_t^j.\]
    \end{enumerate}
\end{remarks}

\subsection{一般次元の伊藤の公式の証明}

\begin{notation}
    次に注意:
    \[\forall_{i\in[n]}\quad dX_t^i=(u_s)^idB_s=\sum_{j\in[m]}(u_s)^i_jdB^j_s.\]
\end{notation}

\begin{lemma}[Taylor展開から得る一階微分の項の評価]\label{lemma-Taylor-1-for-Ito-lemma}
    $X$を$m$次元Brown運動に駆動される$n$次元伊藤過程
    \[X_t=X_0+\int^t_0u_sdB_s\]
    とし,$\pi_N=\Brace{0=t_0<\cdots<t_N=t}$を$[0,t]$の等分割$t_k=\frac{kt}{N}$とする.このとき,任意の$f\in C(\R)$について,
    \[\sum_{k=0}^{N-1}f(X_{t_k})(X^i_{t_{k+1}}-X^i_{t_k})\xrightarrow[N\to\infty]{L^2(\Om)}\int^t_0f(X_s)dX^i_s.\]
    ただし,$dX_s^i=(u^i)_sdB_s=\sum_{l=1}^m(u^i_l)_sdB_s^l$に注意.
\end{lemma}
\begin{Proof}
    \[X_{t_{k+1}}^i-X_{t_k}^i=\int^{t_{k+1}}_{t_k}u_s^idB_s=\int^{t_{k+1}}_{t_k}dX^i_s\]
    に注意すれば,命題より,$L^2$-ノルムは次のように評価できる:
    \begin{align*}
        \Norm{\int^t_0f(X_s)dX_s^i-\sum_{k=0}^{N-1}f(X_{t_k})(X_{t_{k+1}}^i-X_{t_j}^i)}^2_{L^2(\Om)}&=\Norm{\sum_{k=0}^{N-1}\int^{t_{k+1}}_{t_k}(f(X_s)-f(X_{t_k}))dX_s^i}^2_{L^2(\Om)}\\
        &=\sum_{k=0}^{N-1}E\Square{\paren{\int^{t_{k+1}}_{t_k}(f(X_s)-f(X_{t_k}))dX_s^i}^2}\\
        &\qquad+2\sum_{l>k}E\Square{\paren{\int^{t_{k+1}}_{t_k}(f(X_s)-f(X_{t_k}))dX^i_s}\paren{\int^{t_{l+1}}_{t_l}(f(X_s)-f(X_{t_l}))dX^i_s}}\\
        &=\sum_{k=0}^{N-1}E\Square{\paren{\sum_{l=1}^m\int^{t_{k+1}}_{t_k}(f(X_s)-f(X_{t_k}))u^i_ldB^l}^2}\\
        &\le\sum_{k=0}^{N-1}\sum_{l=1}^m\Norm{\int^{t_{k+1}}_{t_k}(f(X_s)-f(X_{t_k}))(u^i_l)_sdB_s^l}^2_{L^2(\Om)}\\
        &=\sum_{k=0}^{N-1}\sum_{l=1}^mE\Square{\int^{t_{k+1}}_{t_k}(f(X_s)-f(X_{t_k}))^2(u^i_l)_s^2ds}\\
        &\le m\sum_{k=1}^{N-1}E\Square{\sup_{\abs{s-u}\le t/N}(f(X_s)-f(X_u))^2\int^{t_{k+1}}_{t_k}(u^i_l)_s^2ds}.
    \end{align*}
    $f$は連続としたから,$f(X_s)$の$s\in[0,t]$上の一様連続性より,これは$0$に収束する.
    なお,途中の変形において,$\int^t_0(f(X_s)-f(X_{t_l}))dX^i_s$のマルチンゲール性による
    \begin{align*}\forall_{k>l}\quad
        &E\Square{\paren{\int^{t_{k+1}}_{t_k}(f(X_s)-f(X_{t_k}))dX^i_s}\paren{\int^{t_{l+1}}_{t_l}(f(X_s)-f(X_{t_l}))dX^i_s}}\\
        &=E\Square{\paren{\int^{t_{k+1}}_{t_k}(f(X_s)-f(X_{t_k}))dX^i_s}E\Square{\int^{t_{l+1}}_{t_l}(f(X_s)-f(X_{t_l}))dX^i_s\middle|\F_{t_{k+1}}}}=0
    \end{align*}
    を用いた.
\end{Proof}

\begin{lemma}[Taylor展開から得る二階微分の交差項の評価]\label{lemma-Taylor-2-1-for-Ito-lemma}
    $(B^1,\cdots,B^m)$を$m$次元Brown運動,$\pi_N=\Brace{0=t_0<\cdots<t_N=t}$を$[0,t]$の等分割$t_k=\frac{kt}{N}$,$u^i_l,u^j_h\in L^2(\P)$とする.このとき,
    \begin{enumerate}
        \item 独立なBrown運動の二次共変分は零である:
        \[\forall_{i\ne j\in[m]}\quad\sum_{k\in N}(B^i_{t_{k+1}}-B^i_{t_k})(B^j_{t_{k+1}}-B^j_{t_k})\xrightarrow[\abs{\pi_N}\to0]{L^2(\Om)}0.\]
        \item 独立なBrown運動に駆動される確率積分のマルチンゲールの二次共変分は零である:
        \[\forall_{l\ne h\in[m]}\quad\sum_{k\in N}\int^{t_{k+1}}_{t_k}u^i_ldB^l\int^{t_{k+1}}_{t_j}u^j_hdB^h\xrightarrow[\abs{\pi_N}\to0]{L^1(\Om)}0.\]
    \end{enumerate}
\end{lemma}
\begin{Proof}\mbox{}
    \begin{enumerate}
        \item $L^2(\Om)$-ノルムは次のように評価できる:
        \begin{align*}
            \Norm{\sum_{k\in N}(B^i_{t_{k+1}}-B^i_{t_k})(B^j_{t_{k+1}}-B^j_{t_k})}_2^2&=E\Square{\paren{\sum_{k\in N}(B^i_{t_{k+1}}-B^i_{t_k})(B^j_{t_{k+1}}-B^j_{t_k})}^2}\\
            &\le 2E\Square{\sum_{k=0}^{N-1}(B_{t_{k+1}}^i-B_{t_k}^i)^2(B_{t_{k+1}}^j-B_{t_k}^j)^2}\\
            &=2\sum_{k\in N}E\Square{(B_{t_{k+1}}^i-B_{t_k}^i)^2(B_{t_{k+1}}^j-B_{t_k}^j)^2}\\
            &=2\sum_{k\in N}(t_{k+1}-t_k)^2=2\sum_{k\in N}\paren{\frac{t}{N}}^2=\frac{2t^2}{N}\xrightarrow{N\to\infty}0.
        \end{align*}
        \item \begin{description}
            \item[単過程の$u\in\E$のとき] $u^i_l$と$u^j_h$の不連続点を合併することで,同じ添字$\Brace{0=t^0<\cdots<t^M=t}$を用いて,
            \[u^i_l=\sum_{m\in M}\phi^m1_{(t^m,t^{m+1}]},\quad u^j_h=\sum_{m\in M}\psi^m1_{(t^m,t^{m+1}]},\qquad\phi^m,\psi^m\in L^2_{\F_{t^m}}(\Om).\]
            と表せる.
            また,$0=t^0,t^M=t$と取ったので,$\phi^0,\phi^{M-1}=0$の可能性があることに注意.
            ここで,$\pi_N$の添字を辞書式順序で$\Brace{0=t_{00}<\cdots<t_{0(N_0-1)}<t_{11}<\cdots<t_{1(N_1-1)}<t_{21}<\cdots<t_{(M-1)1}<\cdots<t_{(M-1)(N_{M-1}-1)}<t_{(M-1)(N_{M-1})}=t}$と振り直し,さらに
            $t^m=t_{(m-1)N_{m-1}}=t_{m0}$とも定めると,$\abs{\pi_N}\to0$のとき$N_m\to\infty(m=0,\cdots,M-1)$で,
            \begin{align*}
                \sum_{k\in N}\int^{t_{k+1}}_{t_k}u^i_ldB^l\int^{t_{k+1}}_{t_k}u^j_hdB^h&=\sum_{m\in M}\sum_{k=0}^{N_m}\phi^m(B^l_{t_{m(k+1)}}-B^l_{t_{mk}})\psi^m(B^h_{m(k+1)}-B^h_{mk})\\
                &=\sum_{m\in M}\phi^m\psi^m\sum_{k=0}^{N_m}(B^l_{t_{m(k+1)}}-B^l_{t_{mk}})(B^h_{t_{m(k+1)}}-B^h_{t_{mk}}).
            \end{align*}
            と表せる.(1)の議論より,これは$0$に$L^2(\Om)$-収束する.
            \item[一般の$u\in L^\infty(\Om)$のとき] 任意の$\ep>0$に対して,
            $\norm{1_{[0,t]}(u^i_l-v^i_l)}_{L^2(\P)},\norm{1_{[0,t]}(u^j_h-v^j_h)}_{L^2(\P)}<\ep$を満たす$v^i_l,v^j_h\in\E$を取る.
            これに対して,$0$への$L^1(\Om)$-収束を示したい項を
            \[\sum_{k\in N}\int^{t_{k+1}}_{t_k}u^i_ldB^l\int^{t_{k+1}}_{t_j}(u^j_h-v^j_h)dB^h+\sum_{k\in N}\int^{t_{k+1}}_{t_k}(u^i_l-v^i_l)dB^l\int^{t_{k+1}}_{t_j}v^j_hdB^h+\sum_{k\in N}\int^{t_{k+1}}_{t_k}v^i_ldB^l\int^{t_{k+1}}_{t_j}v^j_hdB^h\]
            と3つに分解すると,第三項が$0$に$L^2(\Om)$-収束する場合は,上の議論による.
            \begin{enumerate}
                \item 第一項の$L^1(\Om)$-ノルムは,Cauchy-Schwarzの不等式を2回使うことで,
                \begin{align*}
                    E\Square{\Abs{\sum_{k\in N}\int^{t_{k+1}}_{t_k}u^i_ldB^l\int^{t_{k+1}}_{t_j}(u^j_h-v^j_h)dB^h}}&\le \sum_{k\in N}E\Square{\int^{t_{k+1}}_{t_k}u^i_ldB^l\int^{t_{k+1}}_{t_j}(u^j_h-v^j_h)dB^h}\\
                    &\le\sum_{k\in N}\paren{E\Square{\paren{\int^{t_{k+1}}_{t_k}u^i_ldB^l}^2}E\Square{\paren{\int^{t_{k+1}}_{t_j}(u^j_h-v^j_h)dB^h}^2}}^{1/2}\\
                    &\le\paren{\sum_{k\in N}E\Square{\paren{\int^{t_{k+1}}_{t_k}u^i_ldB^l}^2}}^{1/2}\paren{\sum_{k\in N}E\Square{\paren{\int^{t_{k+1}}_{t_j}(u^j_h-v^j_h)dB^h}^2}}^{1/2}\\
                    &\le\paren{E\Square{\int^t_0(u^i_l)^2ds}}^{1/2}\paren{E\Square{\int^t_0(u^j_h-v^j_h)^2ds}}^{1/2}\\
                    &<\sqrt{\ep}\paren{E\Square{\int^t_0(u^i_l)^2ds}}^{1/2}.
                \end{align*}
                と評価できる.
                \item 第二項の$L^1(\Om)$-ノルムも全く同様にして,
                \begin{align*}
                    E\Square{\Abs{\sum_{k\in N}\int^{t_{k+1}}_{t_k}(u^i_l-v^i_l)dB^l\int^{t_{k+1}}_{t_j}v^j_hdB^h}}
                    <\sqrt{\ep}\paren{E\Square{\int^t_0(v^j_h)^2ds}}^{1/2}\le
                    \sqrt{\ep}(\norm{u}_{L^2_t(\P)}+\sqrt{\ep}).
                \end{align*}
                と評価できる.
            \end{enumerate}
        \end{description}
    \end{enumerate}
\end{Proof}
\begin{remarks}[11/16の講究の焦点]
    $u$の4次の積率の存在が不明であるため,$L^2(\Om)$-ノルムの評価ができないことは,確率積分のマルチンゲールの二次変分の証明\ref{prop-quadratic-covariation-of-the-martingale-of-SI}と同様である.
    この問題は,上のように$L^1(\Om)$-ノルムで議論するか,Burkholderの不等式\ref{cor-Burkholder-for-the-martingale-of-SI}を用いることで打開できる.
\end{remarks}

\begin{lemma}[Taylor展開から得る二階微分の項の評価(駆動過程が同じ場合)]\label{lemma-Taylor-2-2-for-Ito-lemma}
    $(B^1,\cdots,B^m)$を$m$次元Brown運動とし,$\pi_N=\Brace{0=t_0<\cdots<t_N=t}$を$[0,t]$の等分割$t_k=\frac{kt}{N}$とする.このとき,
    任意の$i,j\in[n]$と$l\in[m]$について,
    \[\sum_{k=0}^{N-1}\pp{^2f}{x^i\partial x^j}(\wt{X}_k)\paren{\int^{t_{k+1}}_{t_k}(u^i_l)_sdB_s^l}\paren{\int^{t_{k+1}}_{t_k}(u^j_l)_sdB_s^l}\xrightarrow[\abs{\pi_N}\to0]{P}\int^t_0\pp{^2f}{x^i\partial x^j}(X_s)(u^i_l)_s(u^j_l)_sds.\]
\end{lemma}
\begin{Proof}
    確率収束を示したい差は,次のように3つの項に分解できる:
    \begin{align*}
        &\int^t_0\pp{^2f}{x^i\partial x^j}(X_s)(u^i_l)_s(u^j_l)_sds-\sum_{k=0}^{N-1}\pp{^2f}{x^i\partial x^j}(\wt{X}_k)\paren{\int^{t_{k+1}}_{t_k}(u^i_l)_sdB_s^l}\paren{\int^{t_{k+1}}_{t_k}(u^j_l)_sdB_s^l}\\
        &\qquad=\sum_{k=0}^{N-1}\int^{t_{k+1}}_{t_k}\paren{\pp{^2f}{x^i\partial x^j}(X_s)-\pp{^2f}{x^i\partial x^j}(X_{t_k})}(u^i_l)_s(u^j_l)_sds\\
        &\qquad\qquad+\sum_{k=0}^{N-1}\pp{^2f}{x^i\partial x^j}(X_{t_k})\paren{\int^{t_{k+1}}_{t_k}(u^i_l)_s(u^j_l)_sds-\int^{t_{k+1}}_{t_k}(u^i_l)_sdB_s^l\int^{t_{k+1}}_{t_k}(u^j_l)_sdB_s^l}\\
        &\qquad\qquad+\sum_{k=0}^{N-1}\paren{\pp{^2f}{x^i\partial x^j}(X_{t_k})-\pp{^2f}{x^i\partial x^j}(\wt{X}_k)}\int^{t_{k+1}}_{t_k}(u^i_l)_sdB_s^l\int^{t_{k+1}}_{t_k}(u^j_l)_sdB_s^l\\
        &\qquad=:A_1^{(N)}+A_2^{(N)}+A_3^{(N)}.
    \end{align*}
    まず,$(dX^i_l)_s:=(u^i_l)_sdB_s^l,(dX_l^j)_s:=(u^j_l)_sdB_s^l$とおくとこれは伊藤過程で,この二次共変分について次が成り立つ(命題\ref{prop-quadratic-covariation-of-ito-process}):
    \[\sum_{k=0}^{N-1}((X_l^i)_{t_{k+1}}-(X_l^i)_{t_k})((X_l^j)_{t_{k+1}}-(X_l^j)_{t_k})\xrightarrow[\abs{\pi_N}\to0]{P}\int^t_0(u_l^i)_s(u_l^j)_sds.\]
    \begin{description}
        \item[第一項] この項は
        \[\abs{A_1^{(N)}}\le\sup_{\abs{s-u}\le t/N}\int^t_0(u^i_l)(u^j_h)ds\]
        と評価でき,$A_1^{(N)}\to0\;\as$が$X_s$の$s\in[0,t]$に関する一様連続性から判る.
        \item[第三項] 上述の事実に注意すれば,同様に$A_3^{(N)}\pto0$が判る.
        \item[第二項] マルチンゲール差分列の構造に注目すれば,$A_2^{(N)}\to 0\in L^2(\Om)$が判る.まず,
        \[\ep_k:=\int^{t_{k+1}}_{t_k}(u^i_l)_s(u^j_l)_sds-\int^{t_{k+1}}_{t_k}(u^i_l)_sdB_s^l\int^{t_{k+1}}_{t_k}(u^j_l)_sdB_s^l\]
        はマルチンゲール差分列である:$E[\ep_k|\F_{t_k}]=0$.これに注目すれば,
        \begin{align*}
            \norm{A_2^{(N)}}_2^2&=\Norm{\sum_{k=0}^{N-1}\pp{^2f}{x^i\partial x^j}(X_{t_k})\ep_k}^2_2\\
            &=\sum_{k=1}^{N-1}E\Square{\paren{\pp{^2f}{x^i\partial x^j}(X_{t_k})}^2\ep_k^2}+2\sum_{k_1<k_2\in N}\underbrace{E\Square{\pp{^2f}{x^i\partial x^j}(X_{t_{k_1}})\ep_{k_1}\pp{^2f}{x^i\partial x^j}(X_{t_{k_2}})\ep_{k_2}}}_{=0}\\
            &\le\Norm{\pp{^2f}{x^i\partial x^j}}^2_\infty\sum_{k=1}^{N-1}E[\ep_k^2]\\
            &=\Norm{\pp{^2f}{x^i\partial x^j}}^2_\infty E\Square{\sum_{k=0}^{N-1}\paren{\int^{t_{k+1}}_{t_k}(u^i_l)(u^j_l)ds}^2+\sum_{k=0}^{N-1}\paren{\int^{t_{k+1}}_{t_k}(u^i_l)dB_s^l\int^{t_{k+1}}_{t_k}(u^j_l)dB_s^l}^2\right.\\
            &\hspace{5cm}\left.-2\sum_{k=0}^{N-1}\int^{t_{k+1}}_{t_k}(u^i_l)(u^j_l)ds\int^{t_{k+1}}_{t_k}(u^i_l)dB_s^l\int^{t_{k+1}}_{t_k}(u^j_l)dB_s^l}\\
            &\le\Norm{\pp{^2f}{x^i\partial x^j}}^2_\infty E\Square{N\sup_{k\in N}\int^{t_{k+1}}_{t_k}(u_l^i)(u_l^j)ds\right.\\
            &\qquad\left.+\sup_{k\in N}\abs{(X_l^i)_{t_{k+1}}-(X_l^i)_{t_k}}\abs{(X_l^j)_{t_{k+1}}-(X_l^j)_{t_k}}\sum_{k=0}^{N-1}\int^{t_{k+1}}_{t_k}(u^i_l)dB_s^l\int^{t_{k+1}}_{t_k}(u^j_l)dB_s^l\right.\\
            &\qquad\left. -2\sup_{k\in N}\int^{t_{k+1}}_{t_k}(u_l^i)(u_l^j)ds\sum_{k=0}^{N-1}\int^{t_{k+1}}_{t_k}(u^i_l)dB_s^l\int^{t_{k+1}}_{t_k}(u^j_l)dB_s^l }
        \end{align*}
        と評価できる.最右辺は,まず第一項と第三項がLebesgue積分の積分区間に対する絶対連続性より,第二項が$(X^i_l)_s,(X^j_l)_s$の$s\in[0,t]$上の一様連続性により,全て$0$に収束する.
        もちろん,第二項と第三項のその他の因数は,有界であることは上述した伊藤過程の二次の共変分に確率収束することによる.

        また,$E[\ep_k^2]$を展開するに当たって,恒等式$(A-B)^2\le 2(A^2+B^2)$を用いても良い.
    \end{description}
\end{Proof}

\begin{lemma}[一般次元の伊藤の公式(簡略版)]
    $X$を$m$次元Brown運動に駆動される$n$次元伊藤過程,$f\in C^2(\R^n)$とすると,
    \[\forall_{t\in\R_+}\;f(X_t)\overset{\as}{=}f(X_0)+\sum_{i\in[n]}\int^t_0\pp{f}{x^i}(s,X_s)dX_s^i+\frac{1}{2}\sum_{i,j\in[n]}\pp{^2f}{x^i\partial x^j}(s,X_s)(u_su_s^\top)_{ij}ds.\]
\end{lemma}
\begin{Proof}\mbox{}
    \begin{description}
        \item[問題の所在] $f\in C_b^2(\R^n),\int^\infty_0u_s^2ds<N,\sup_{s\in\R_+}\abs{X_s}<N$の仮定の下で示せば良い.
        すると,一般の$f\in C^2(\R^n)$について$f_N\nearrow f$を$f_N:=f|_{[-N,N]^n}$で定め,一般の$u_s\in L^2_\loc(\P;\R^{nm})$について
        剪断過程$u^{(N)}_s:=1_{[0,T_N]}u_s$を$T_N:=\inf\Brace{t\in\R_+\;\middle|\;\int^t_0u_s^2ds\ge N\lor\abs{X_t}\ge N}$について考えれば,
        この各$N\in\N$についての伊藤の公式は
        \[\forall_{t\in\R_+}\;f(X_{t\land T_N})\overset{\as}{=}f(X_0)+\sum_{i\in[n]}\int^{t\land T_N}_0\pp{f}{x^i}(s,X_s)dX_s^i+\frac{1}{2}\sum_{i,j\in[n]}\int^{t\land T_N}_0\pp{^2f}{x^i\partial x^j}(s,X_s)(u_su_s^\top)_{ij}ds\]
        と表され,$N\to\infty$の極限を考えることで,一般の場合の公式も得る.
        \item[証明の方針] $\pi_N:=\Brace{0=t_0<\cdots<t_N=t}$を$[0,t]$の$N$等分割$t_k:=\frac{kt}{N}$として,Taylorの定理より,任意の$k\in N$について
        $\inf_{t_k\le s\le t_{k+1}}X_s\le\wt{X}_k\le\sup_{t_k\le s\le t_{k+1}}X_s$を満たす$\wt{X}_k:\Om\to\R^n$が存在して(可測とは限らない),
        \begin{align*}
            f(X_t)-f(X_0)&=\sum_{k=0}^{N-1}(f(X_{t_{k+1}})-f(X_{t_k}))\\
            &=\sum_{k=0}^{N-1}\sum_{i=1}^n\pp{f}{x^i}(X_{t_k})(X_{t_{k+1}}^i-X_{t_k}^i)+\frac{1}{2}\sum_{k=0}^{N-1}\sum_{i,j=1}^n\pp{^2f(\wt{X}_k)}{x^i\partial x^j}(X_{t_{k+1}}^i-X_{t_k}^i)(X_{t_{k+1}}^j-X_{t_k}^j).
        \end{align*}
        よって,次の2点を示せば良いことが判る:
        \begin{enumerate}
            \item $i\in[n]$について,\[A^{(N)}_i:=\sum_{k=0}^{N-1}\pp{f}{x^i}(X_{t_k})(X^i_{t_{k+1}}-X_{t_k}^i)\xrightarrow[N\to\infty]{P}\int^t_0\pp{f}{x^i}(s,X_s)dX^i_s.\]
            \item $i,j\in[n]$について,\[A^{(N)}_{i,j}:=\sum_{k=0}^{N-1}\pp{^2f(\wt{X}_k)}{x^i\partial x^j}(X_{t_{k+1}}^i-X_{t_k}^i)(X_{t_{k+1}}^j-X_{t_k}^j)\xrightarrow[N\to\infty]{P}\int^t_0\pp{^2f(s,X_s)}{x^i\partial x^j}(u_su_s^\top)_{ij}ds.\]
        \end{enumerate}
        \item[(1)の証明]
        (1)の$A^{(N)}_i$の$L^2(\Om)$-収束が上の補題\ref{lemma-Taylor-1-for-Ito-lemma}から従う.
        $dX_s^i$は$dB^l$の線型和であり,交差項を処理する段階が1次元の場合と異なる.
        \item[(2)の証明]
        次に
        (2)の$A^{(N)}_{ij}$は,補題に従うとさらに
        \begin{align*}
            &\sum_{k=0}^{N-1}\pp{^2f(\wt{X}_k)}{x^i\partial x^j}(X_{t_{k+1}}^i-X_{t_k}^i)(X_{t_{k+1}}^j-X_{t_k}^j)\\
            &=\sum_{k=0}^{N-1}\pp{^2f(\wt{X}_k)}{x^i\partial x^j}\paren{\int^{t_{k+1}}_{t_k}(u^i)_sdB_s\int^{t_{k+1}}_{t_k}(u^j)_sdB_s}
            =\sum_{k=0}^{N-1}\pp{^2f(\wt{X}_k)}{x^i\partial x^j}\paren{\sum_{l,h=1}^m\int^{t_{k+1}}_{t_k}(u^i_l)_sdB_s^l\int^{t_{k+1}}_{t_k}(u_h^j)_sdB_s^h}\\
            &=\sum_{l=1}^m\sum_{k=0}^{N-1}\pp{^2f(\wt{X}_k)}{x^i\partial x^j}\paren{\int^{t_{k+1}}_{t_k}(u_l^i)dB^l\int^{t_{k+1}}_{t_k}(u_l^j)dB^l}+2\sum_{l>h\in[m]}\sum_{k=0}^{N-1}\pp{^2f(\wt{X}_k)}{x^i\partial x^j}\paren{\int^{t_{k+1}}_{t_k}(u_l^i)dB^l\int^{t_{k+1}}_{t_k}(u_m^j)dB^m}\\
            &=B^{ij}_{l}+B^{ij}_{lh}.
        \end{align*}
        と分解でき,この第二項が$B^{ij}_{lh}\xrightarrow{L^2(\Om)}0$と消えて,第一項が
        \[B^{ij}_l\pto\int^t_0\pp{^2f}{x^i\partial x^j}(X_s)(u_su_s^\top)_{ij}ds=\int^t_0\pp{^2f}{x^i\partial x^j}(X_s)d\brac{X^i,X^j}_s=\int^t_0\pp{^2f}{x^i\partial x^j}(X_s)\sum_{l=1}^m(u^i_l)_s(u^j_l)_sds.\]
        に確率収束する.
        実際,
        \begin{enumerate}[(a)]
            \item 第二項$B^{ij}_{lh}$は$L^2$-ノルムが
            \[\Norm{\sum_{k\in N}f_{x_jx_i}(\wt{X}_k)\paren{\int^{t_{k+1}}_{t_k}(u_l^i)dB^l\int^{t_{k+1}}_{t_k}(u_m^j)dB^m}}_{L^2(\Om)}\le\norm{f_{x_jx_i}}_{L^\infty_t(\Om)}\Norm{\int^{t_{k+1}}_{t_k}(u_l^i)dB^l\int^{t_{k+1}}_{t_k}(u_m^j)dB^m}_{L^2(\Om)}\]
            と抑えられ,補題\ref{lemma-Taylor-2-1-for-Ito-lemma}より右辺は$0$に収束する.
            \item 第一項$B^{ij}_{l}$の確率収束は,各$l\in [m]$について駆動過程が同一であるから,補題\ref{lemma-Taylor-2-2-for-Ito-lemma}のように,一次元の場合と平行に議論出来る.
        \end{enumerate}
    \end{description}
\end{Proof}

\subsection{系として得られる公式}

\begin{tcolorbox}[colframe=ForestGreen, colback=ForestGreen!10!white,breakable,colbacktitle=ForestGreen!40!white,coltitle=black,fonttitle=\bfseries\sffamily,
title=]
    まず部分積分公式
    \[d(XY)_s=X_sdY_s+Y_sdX_s+dX_sdY_s.\]
    を得る.Leibniz則も破れていることに注意.
\end{tcolorbox}

\begin{corollary}[部分積分公式]\label{cor-integration-by-parts}
    $X,Y$を$m$次元Brown運動に駆動される$n$次元伊藤過程とする.
    \[X_tY_t=X_0Y_0+\int^t_0X_sdY_s+\int^t_0Y_sdX_s+\int^t_0d\brac{X,Y}_s.\]
\end{corollary}
\begin{Proof}
    $f(x,y)=xy$とすると,$f_{xx}=f_{yy}=0$に注意して,
    \begin{align*}
        f(X_t,Y_t)&=f(X_0,Y_0)+\int^t_0\paren{\pp{f}{x}(X_s,Y_s)dX_s+\pp{f}{y}(X_s,Y_s)dY_s}+\frac{1}{2}\int^t_02\pp{^2f}{x\partial y}(X_s,Y_s)dX_sdY_s\\
        &=X_0Y_0+\int^t_0\paren{Y_sdX_s+X_sdY_s}+\int^t_0dX_sdY_s.
    \end{align*}
    あとは$dX_sdY_s=d\brac{X,Y}_s$による.
\end{Proof}
\begin{remark}
    ただし,
    \begin{align*}
        dX_sdY_s&=u^X_sdB_su^Y_sdB_s=((u^X)^1_sdB_s^1+\cdots+(u^X)^m_sdB_s^m)((u^Y)^1_sdB_s^1+\cdots+(u^Y)^m_sdB_s^m)\\
        &=(u^X)^1_s(u^Y)^1_sds+\cdots+(u^X)^m_s(u^Y)^m_sds=u^X_s(u^Y_s)^\top ds=d\brac{X,Y}_s
    \end{align*}
    に注意.
\end{remark}

\begin{proposition}[Brown運動の汎関数としてのマルチンゲールの構成]
    $f\in C^{1,2}(\R_+\times\R^n)$,$B$を$m$-次元Brown運動とする.
    このとき,$n$次元確率過程
    \[X_t:=f(t,B_t)-\int^t_0\paren{\frac{1}{2}\Laplace f(s,B_s)+\pp{f}{t}(s,B_s)}ds\]
    を考える.
    \begin{enumerate}
        \item $X_t$は局所マルチンゲールである:$X_t\in\M_\loc$.
        \item さらに次を満たすならば,$X$はマルチンゲールでもある:
        \[\exists_{K_1\ge0}\;\exists_{\beta\in[0,2)}\;\exists_{K_2\in L^1_\loc(\R_+)_+}\;\quad\sum_{i\in[m]}\paren{\pp{f}{x_i}(t,x)}^2\le K_2(t)\exp(K_1\abs{x}^\beta).\]
    \end{enumerate}
    特に,調和関数$f:\R^n\to\R$に対して(あるいは熱方程式の解$f:\R_+\times\R^n\to\R$に対して),$f(B_t)$は局所マルチンゲールである.
\end{proposition}
\begin{Proof}\mbox{}
    \begin{enumerate}
        \item $m$次元Brown運動$B$は,$I_m$を単位行列として
        \[B_t=\int^t_0I_mdB_s\]
        と表されるから,伊藤の公式より
        \[f(t,B_t)=f(0,B_t)+\int^t_0\paren{\pp{f}{t}(s,B_s)+\frac{1}{2}\sum_{i,j\in[n]}\pp{^2f}{x^i\partial x^j}(s,B_s)\delta_{ij}}ds+\int^t_0\nabla_xf(s,B_s)\cdot dB_s.\]
        から,
        \begin{align*}
            X_t&=f(t,B_t)-\int^t_0\paren{\frac{1}{2}\Lap f(s,B_s)+\pp{f}{t}(s,B_s)}ds\\
            &=f(0,B_0)+\int^t_0\nabla f(s,B_s)\cdot dB_s=f(0,B_0)+\int^t_0\sum_{i\in[m]}f_{x_i}(s,B_s)dB_s^i.
        \end{align*}
        と表せる.
        よって,$X_t$はドリフト項を持たない伊藤過程であり,特に連続な局所マルチンゲールである.
        \item $X_t$がマルチンゲールであるためには,任意の$i\in[m]$に対して
        \[\pp{f}{x^i}(s,B_s)\in L^2_\infty(\P)\]
        であれば十分である.マルチンゲールの和は再びマルチンゲールであることに注意.
        よって,任意の$t\in\R_+$について,Fubini-Tonelliの定理より
        \begin{align*}
            E\Square{\int^t_0\paren{f_{x^i}(s,B_s)}^2ds}&\le E\Square{\int^t_0K_2(s)\exp(K_1\abs{B_s}^\beta)ds}\\
            &=\int^t_0K_2(s)E[e^{K_1\abs{B_s}^\beta}]ds
        \end{align*}
        となるが,
        $B_s\sim N(0,s)$と$0\le\beta<2$より,
        $L>(2tK_1)^{-(2-\beta)}$と取れば,任意の$s\in[0,t]$に対して
        $(0\le)2sK_1\abs{x}^{\beta-2}<1$.すなわち,$0<(1-2sK_1\abs{x}^{\beta-2})=:K\le 1$.
        これより,任意の$s\in[0,t]$について,
        \begin{align*}
            E[e^{K_1\abs{B_s}^\beta}]&=\int_\R e^{K_1\abs{x}^\beta}\frac{1}{\sqrt{2\pi s}}e^{-\frac{x^2}{2s}}dx\\
            &=\frac{1}{\sqrt{2\pi s}}\int_\R e^{-\frac{x^2}{2s}\paren{1-2sK_1\abs{x}^{-(2-\beta)}}}dx\\
            &\le\frac{1}{\sqrt{2\pi s}}\int^L_{-L}e^{-\frac{x^2}{2s}\paren{1-2sK_1\abs{x}^{-(2-\beta)}}}dx+\frac{1}{\sqrt{2\pi s}}\int_{\abs{x}\ge L}e^{-\frac{x^2}{2s}K}dx<\infty.
        \end{align*}
        である.
    \end{enumerate}
\end{Proof}
\begin{remarks}[熱方程式の解について局所マルチンゲール性は保存する]
    $n=1$の場合,$f\in C^{1,2}(\R_+\times\R)$が熱方程式
    \[\pp{f}{t}+\frac{1}{2}\pp{^2f}{x^2}=0\]
    を満足するならば,
    \[f(t,B_t)=f(0,0)+\int^t_0\pp{f}{x}(s,B_s)dB_s\]
    は再び伊藤過程であるだけでなく,ドリフト項を持たないため連続な局所マルチンゲールになる.
    さらに自乗可積分マルチンゲールでもあるためには,$\pp{f}{x}\in L^2_\infty(\P)$であれば良い.
\end{remarks}

\section{伊藤の公式の局所マルチンゲールへの応用}

\begin{tcolorbox}[colframe=ForestGreen, colback=ForestGreen!10!white,breakable,colbacktitle=ForestGreen!40!white,coltitle=black,fonttitle=\bfseries\sffamily,
title=]
    伊藤の公式はBrown運動について多くのことを教えてくれる.
\end{tcolorbox}

\subsection{Brown運動の特徴付け}

\begin{tcolorbox}[colframe=ForestGreen, colback=ForestGreen!10!white,breakable,colbacktitle=ForestGreen!40!white,coltitle=black,fonttitle=\bfseries\sffamily,
title=]
    拡散過程の中で,$A=\frac{1}{2}\Lap$を生成作用素とするものとしてBrown運動は特徴付けられた.
    何度も用いていた二次変分過程の特徴によってBrown運動が特徴付けられる.
\end{tcolorbox}

\begin{theorem}[Levy]
    $X\in C\cap {}^0\!M_\loc$を$d$-次元過程とする.次は同値:
    \begin{enumerate}
        \item $X$は$d$次元Brown運動である.
        \item 二次変分過程は$\forall_{i,j\in[m]}\;\brac{X^i,X^j}_t=\delta_{ij}t$である.
        \item 任意の$f_1,\cdots,f_d\in L^2(\R_+)$に対して,指数過程
        \[\E^{if}_t:=\exp\paren{i\sum_{j\in[d]}\int^t_0f_k(s)dX^k_s+\frac{1}{2}\sum_{k\in[d]}\int^t_0f^2_k(s)ds}\]
        は複素値マルチンゲールである.
    \end{enumerate}
\end{theorem}
\begin{remarks}
    1次元局所マルチンゲール$X\in M_\loc$が$X^2_t-t\in M_\loc$を満たすとする.$X$が$C$-過程でもあるならBrown運動に限る.
    一般の過程だと,強度$c=1$のPoisson過程は$N_t-t$も$(N_t-t)^2-t$もマルチンゲールである.
\end{remarks}

\subsection{局所マルチンゲールの構成}

\begin{definition}
    次を満たす多項式$h_n$を\textbf{Hermite多項式}という:
    \[\sum_{n\in\N}\frac{u^n}{n!}h_n(x)=\exp\paren{ux-\frac{u^2}{2}},\quad u,x\in\R.\]
\end{definition}

\begin{lemma}\mbox{}
    \begin{enumerate}
        \item 次のようにも表せる:
        \[h_n(x)=e^{\frac{x^2}{2}}(-1)^n\dd{^n}{x^n}(e^{-\frac{x^2}{2}}).\]
        \item 任意の$a>0$について,
        \[\exp\paren{ux-\frac{au^2}{2}}=\exp\paren{u\sqrt{a}\frac{x}{\sqrt{a}}-\frac{(u\sqrt{a})^2}{2}}=\sum_{n\in\N}\frac{u^n}{n!}a^{n/2}h_n\paren{\frac{x}{\sqrt{a}}}=:\sum_{n\in\N}\frac{u^n}{n!}H_n(x,a).\]
        \item 次の微分法則を満たす:
        \[\pp{H_n}{x}=nH_{n-1},\quad\paren{\frac{1}{2}\pp{^2}{x^2}+\pp{}{a}}H_n(x,a)=0.\]
    \end{enumerate}
    ただし,
    \[H_n(x,a):=a^{n/2}h_n\paren{\frac{x}{\sqrt{a}}},\quad H_n(x,0):=x^n.\]
\end{lemma}

\begin{proposition}
    $M\in {}^0\!M_\loc$について,
    \[L^{(n)}_t:=H_n(M_t,[M]_t)\]
    を考える.
    \begin{enumerate}
        \item $L^{(n)}\in M_\loc$.
        \item 次のように確率積分によって表示できる:
        \[L^{(n)}_t=n!\int^t_0dM_{s_1}\int^{s_1}_0dM_{s_2}\cdots\int^{s_{n-1}}_0dM_{s_n}.\]
    \end{enumerate}
\end{proposition}

\subsection{連続マルチンゲールの時間変換を受けたBrown運動としての表現}

\begin{tcolorbox}[colframe=ForestGreen, colback=ForestGreen!10!white,breakable,colbacktitle=ForestGreen!40!white,coltitle=black,fonttitle=\bfseries\sffamily,
title=]
    連続な局所マルチンゲールは,時間軸変換を受けたBrown運動である.
\end{tcolorbox}

\begin{theorem}[Dambis-Dubins-Schwarz]
    $M$を連続な局所マルチンゲールであって$\lim_{t\to\infty}\brac{M,M}_t=\infty\;\as$を満たすとする.
    このとき,$\tau(s):=\Brace{t>0\mid\brac{M,M}_t>s}$について,
    \begin{enumerate}
        \item $B_s:=M_{\tau(s)}$は$(\F_{\tau(s)})$-Brown運動である.
        \item $M_t=B_{\brac{M}_t}\;\as$が成り立つ.
    \end{enumerate}
\end{theorem}
\begin{remark}
    Brown運動の空間$(\Om,\F,P,\bF)$を拡張することで,条件$\brac{M,M}_\infty=\infty$を取り去ることが出来る.
    また,$\beta_s$も$(\F_t)$-適合とは限らないが,同様に時間軸を調整した情報系については適合的である.
\end{remark}

\subsection{局所マルチンゲールの積率不等式}

\begin{theorem}[Burkholder-Davis-Gundy]
    任意の$p>0$に対して,$c_p,C_p>0$が存在して,任意の連続な局所マルチンゲール$M$で$M_0=0$を満たすものに対して,次が成り立つ:
    \[\forall_{t\in\R_+}\quad c_pE[(M^*_t)^{2p}]\le E[\brac{M}_t^p]\le C_pE[(M^*_t)^{2p}],\quad M^*:=\sup_{0\le s\le t}\abs{M_s}.\]
\end{theorem}

\section{Stratonovich積分}

\begin{tcolorbox}[colframe=ForestGreen, colback=ForestGreen!10!white,breakable,colbacktitle=ForestGreen!40!white,coltitle=black,fonttitle=\bfseries\sffamily,
title=]
    伊藤積分での$I(u)\;(u\in\E)$は前方Riemann和とした\ref{remarks-tricks-in-the-definition-of-Ito-integral}.
    ここで対称Riemann和を取ったならば,また別の積分が定義出来る.
\end{tcolorbox}

\subsection{伊藤過程に対する定義}

\begin{tcolorbox}[colframe=ForestGreen, colback=ForestGreen!10!white,breakable,colbacktitle=ForestGreen!40!white,coltitle=black,fonttitle=\bfseries\sffamily,
title=]
    ここでは駆動過程も一般の伊藤過程としよう.
\end{tcolorbox}

\begin{proposition}[(Fisk-)Stratonovich integral]
    $X,Y$を伊藤過程,$\pi_n:=\Brace{0=t_0<t_1<\cdots<t_n=t}\subset[0,t]$を分割とする.
    次の収束が成り立つ:
    \[\sum_{j\in\pi}\frac{Y_{t_j}+Y_{t_{j+1}}}{2}(X_{t_{j+1}}-X_{t_j})\xrightarrow[\abs{\pi_n}\to0]{P}\int^t_0Y_sdX_s+\frac{1}{2}\brac{X,Y}_t=:\int^t_0Y_s\circ dX_s.\]
\end{proposition}
\begin{Proof}
    \[\sum_{j\in n}\frac{Y_{t_j}+Y_{t_{j+1}}}{2}(X_{t_{j+1}}-X_{t_j})=\sum_{j\in  n}Y_{t_j}(X_{t_{j+1}}-X_{t_j})+\sum_{j\in n}\frac{1}{2}(Y_{t_{j+1}}-Y_{t_j})(X_{t_{j+1}}-X_{t_j})\]
    と分解すると,第二項は伊藤過程の二次共変分に確率収束する\ref{prop-quadratic-covariation-of-ito-process}.
    続いて第一項は
    \[\sum_{j\in n}Y_{t_j}\int^{t_{j+1}}_{t_j}u_s^XdB_s+\sum_{j\in n}Y_{t_j}\int^{t_{j+1}}_{t_j}v_s^Xdx\xrightarrow{P}\int^t_0Y_su_sdB_s+\int^t_0Y_sv_s^Xds\]
    すなわち,
    \[\sum_{j\in n}\int^{t_{j+1}}_{t_j}(Y_{t_j}-Y_s)u_s^XdB_s\xrightarrow{P}0,\quad\sum_{j\in n}\int^{t_{j+1}}_{t_j}(Y_{t_j}-Y_s)v_s^Xds\xrightarrow{P}0\]
    を示せば良い.
    第二項は,
    \begin{align*}
        \sum_{j\in n}\int^{t_{j+1}}_{t_j}(Y_{t_j}-Y_s)v_sds&\le\sup_{\abs{u-s}\le\abs{\pi_n}}\int^{t}_0v_sds\xrightarrow{n\to\infty}0\;\as
    \end{align*}
    が$Y_s$の$s\in[0,t]$に関する一様連続性から判る.
    第一項は$u_s^X\in L^2(\P)$の場合,
    \begin{align*}
        E\Square{\paren{\sum_{j\in n}\int^{t_{j+1}}_{t_j}(Y_{t_j}-Y_s)u_s^XdB_s}^2}&=\sum_{j\in n}E\Square{\paren{\int^{t_{j+1}}_{t_j}(Y_{t_j}-Y_s)u_s^XdB_s}^2}\\
        &=E\Square{\int^t_0(Y_{t_j}-Y_s)^2(u_s^X)^2ds}\to0
    \end{align*}
    が示せる.一般の場合の$u_s^X\in L^2_\loc(\P)$は停止時の議論による.
\end{Proof}

\subsection{変数変換公式}

\begin{tcolorbox}[colframe=ForestGreen, colback=ForestGreen!10!white,breakable,colbacktitle=ForestGreen!40!white,coltitle=black,fonttitle=\bfseries\sffamily,
title=]
    伊藤の公式と違い,形式的には$\R^n$上の微積分と全く同様の規則に従う.
\end{tcolorbox}

\begin{proposition}
    $f\in C^3(\R)$ならば,
    \[f(X_t)=f(X_0)+\int^t_0f'(X_s)\circ dX_s.\]
\end{proposition}
\begin{Proof}
    伊藤の公式と命題より,
    \begin{align*}
        f(X_t)&=f(X_0)+\int^t_0f'(X_s)dX_s+\frac{1}{2}\int^t_0f''(X_s)d\brac{X}_s\\
        &=f(X_0)+\int^t_0f'(X_s)\circ dX_s-\frac{1}{2}\brac{f'(X),X}_t+\frac{1}{2}\int^t_0f''(X_s)d\brac{X}_s.
    \end{align*}
    後は次の補題による.
\end{Proof}

\begin{lemma}
    \[\brac{f'(X),X}_t=\Brac{\int^t_0f''(X_s)u_sdB_s,\int^t_0u_sdB_s}=\int^t_0f''(X_s)d\brac{X}_s.\]
\end{lemma}
\begin{Proof}
    伊藤の公式より,過程$f'(X_s)$の局所マルチンゲール成分は$f''(X_s)u_sdB_s$であるためであり,この$u_sdB_s$との二次共変分は$f''(X_s)u_s^2ds=f''(X_s)d\brac{X}_s$である.
\end{Proof}

\section{後退確率積分}

\begin{tcolorbox}[colframe=ForestGreen, colback=ForestGreen!10!white,breakable,colbacktitle=ForestGreen!40!white,coltitle=black,fonttitle=\bfseries\sffamily,
title=]
    後退確率積分は,通常の確率積分の言葉で特徴づけることが出来る.
\end{tcolorbox}

\subsection{後ろ向き確率積分の定義}

\begin{tcolorbox}[colframe=ForestGreen, colback=ForestGreen!10!white,breakable,colbacktitle=ForestGreen!40!white,coltitle=black,fonttitle=\bfseries\sffamily,
title=]
    非常に不自然で危険な香りのする定義をするが,これが後ろ向きのBrown運動$(B_T-B_t)_{t\in[0,T]}$に関する後ろ向き過程$u_{T-t}$の確率積分と全く等価になっている点が肝要なのである.
\end{tcolorbox}

\begin{definition}[backward filtration / future filtration, backward predictable]
    期限$T>0$を定める.
    \begin{enumerate}
        \item $\wh{\F}_t:=\sigma[\Brace{B_T-B_s}_{t\le s\le T}\cup\cN]$とすると,$(\wh{\F}_t)_{t\in[0,T]}$は閉$\sigma$-代数の減少列である.これを\textbf{減退情報系}という.
        \item 過程$u$が\textbf{後退可予測}とは,$\forall_{t\in[0,T]}\;u|_{\Om\times[t,T]}$が$\wh{\F}_t\otimes\B([t,T])$-可測であることをいう.
        \item $[0,T]$上の二乗可積分な後退可予測関数全体の空間を
        \[L^2_T(\wh{\P}):=\Brace{u\in\L_T(\wh{\P})\;\middle|\;E\Square{\int^T_0u_s^2ds}<\infty}\]
        で表す.
    \end{enumerate}
\end{definition}
\begin{remarks}
    $B_T$というランダムな到着点を逆に出発点とした「巻き戻し運動の世界」のフィルトレーションの巻き戻しを減退乗法系,
    「巻き戻し運動の世界」に関する発展的可測性を「後退可予測」という.
\end{remarks}
\begin{memo}
    $\wh{\P}$は発展的可測の場合と同様に,$\Om\times\R_+$上の$\sigma$-代数として表現出来るか?
\end{memo}

\begin{definition}[backward Ito stochastic integral]
    減退情報系付き確率空間$(\Om,\F,P,(\wh{\F}_t))$上の二乗可積分過程
    $u\in L^2_T(\wh{\P})$の\textbf{後退伊藤積分}とは,次の$L^2(\Om)$-極限をいう:
    \[\int^T_0u_t\wh{d}B_t:=\lim_{n\to\infty}\sum_{j=0}^{n-2}\frac{1}{\abs{\pi_n}}\paren{\int^{t_{j+2}}_{t_{j+1}}u_sds}(B_{t_{j+1}}-B_{t_j}),\quad t_j=\frac{T}{n}j.\]
    %\[\int^T_0u_t\wh{d}B_t:=\lim_{n\to\infty}\sum^{n-2}_{j=0}\paren{\frac{n}{T}\int^{(j+2)T/n}_{(j+1)T/n}u_sds}(B_{(j+1)T/n}-B_{jT/n})\quad\in L^2(\Om).\]
\end{definition}

\begin{lemma}[後退Riemann和の極限としての特徴付け]\mbox{}
    \begin{enumerate}
        \item 任意の$t_2\ge t_1\in[0,T]$について,確率変数$B_{t_2}-B_{t_1}$は任意の$\wh{\F}_s\;(s\notin (t_1,t_2))$と独立である.
        \[\int^{(j+2)\frac{T}{n}}_{(j+1)\frac{T}{n}}u_sds,\quad B_{(j+1)\frac{T}{n}}-B_{j\frac{T}{n}}.\]
        \item 上述の$L^2(\Om)$-極限
        \[\lim_{n\to\infty}\sum_{j=0}^{n-2}\paren{\frac{n}{T}\int^{(j+2)\frac{T}{n}}_{(j+1)\frac{T}{n}}u_sds}(B_{(j+1)\frac{T}{n}}-B_{j\frac{T}{n}})\]
        は存在する.
        \item 過程$u\in L^2_T(\wh{\P})$はさらに$L^2(\Om)$-連続で$(\wh{\F}_t)$-適合的であるとする.このとき,次の等式が$L^2(\Om)$-で成り立つ:
        \[\int^T_0u_t\wh{d}B_t=\lim_{n\to\infty}\sum^{n-1}_{j=0}u_{(j+1)T/n}(B_{(j+1)T/n}-B_{jT/n})\quad\in L^2(\Om).\]
    \end{enumerate}
\end{lemma}
\begin{Proof}\mbox{}
    \begin{enumerate}
        \item 明らか.
        \item \begin{description}
            \item[方針] いま,過程$(u_{T-t})_{t\in[0,T]}$は,増大情報系$(\wh{\F}_{T-t})_{t\in[0,T]}$に対して,通常の意味で発展的可測であるから,$L^2_T(\P)$の中で単過程の集合$\E$はやはり稠密である.
            そこで,対応$\E\to L^2(\Om)$が線型なのは明らかだから,あとは有界であることを示せば良い.
            \item[問題の所在] まず,
            \[u_t=\sum_{i=0}^{m-1}\phi_i1_{(t_i,t_{i+1}]}(t),\quad\phi_i\in L^2_{\wh{\F}_{t_i}}(\Om),0=t_0<\cdots<t_m=T.\]
            と表せるとすると,
            \begin{align*}
                \norm{u}_{L^2_T(\wh{\P})}^2&=\sum_{i\in m}E[\phi_i^2](t_{i+1}-t_i).
            \end{align*}
            よって特に$\sup_{i\in[m]}E[\phi_i^2]\le\frac{1}{\inf_{i\in m}\abs{t_{i+1}-t_i}}\norm{u}_{L^2_T(\wh{\P})}^2$であるから,
            あとは任意の$n\in\N$に対して,
            \[\Norm{\sum_{j=0}^{n-2}\frac{1}{\abs{\pi_n}}\paren{\int^{t_{j+2}}_{t_{j+1}}u_sdB_s}(B_{t_{j+1}}-B_{t_j})}_{L^2(\Om)}^2\]
            が$\sup_{i\in[m]}E[\phi^2_i]$の定数倍で抑えられれば良い.
            \item[証明] まず,$[0,T]$の$n$等分割$\Brace{\frac{T}{n},2\frac{T}{n},\cdots,n\frac{T}{n}}$に対して,$0=t_0<t_1<\cdots<t_m=T$のどの間にあるかで,$t_m=\tau_{0}<\tau_1<\cdots<\tau_{n_m-1}\le\tau_{n_m}=t_{m+1}$と番号をつけ直す.
            この中で,$\phi_i$が$\wh{\F}_{t_i}$可測であることに注意すると,
            \begin{align*}
                \frac{n^2}{T^2}\sum_{j=0}^{n-2}E\Square{\paren{\int^{t_{j+2}}_{t_{j+1}}u_sdB_s}^2(B_{t_{j+1}}-B_{t_j})^2}&=\frac{n^2}{T^2}\sum_{i=0}^{m-1}\sum_{k=0}^{n_m-1}E\Square{\paren{\int^{\tau_{k+1}}_{\tau_k}\phi_ids}^2(B_{\tau_k}-B_{\tau_{k-1}})^2}\\
                &\le\frac{n^2}{T^2}\sum_{i=0}^{m-1}\sum_{k=0}^{n_m-1}E\Square{(\phi_i)^2\frac{T^n}{n^2}}(\tau_{t_{k+1}}-\tau_{t_k})\\
                &=\frac{n^2}{T^2}\sum_{i=0}^{m-1}\sum_{k=0}^{n_m-1}E[\phi^2_i]\frac{T^2}{n^2}n_m\le T\sup_{i\in[m]}E[\phi^2_i].
            \end{align*}
        \end{description}
        \item 
        \begin{align*}
            &\sum_{j=0}^{n-2}\paren{\frac{n}{T}\int^{(j+2)\frac{T}{n}}_{(j+1)\frac{T}{n}}u_sds}(B_{(j+1)\frac{T}{n}}-B_{j\frac{T}{n}})-\sum_{j=0}^{n-1}u_{(j+1)\frac{T}{n}}(B_{(j+1)\frac{T}{n}}-B_{j\frac{T}{n}})\\
            &=\sum_{j=0}^{n-2}(B_{(j+1)\frac{T}{n}}-B_{j\frac{T}{n}})\paren{\frac{n}{T}\int^{(j+2)\frac{T}{n}}_{(j+1)\frac{T}{n}}u_sds-u_{(j+1)\frac{T}{n}}}+u_T(B_T-B_{\frac{n-1}{n}T})
        \end{align*}
        の$L^2(\Om)$-ノルムが$0$に収束することを示せば良い.第2項は明らかに$n\to0$で$0$に$L^2(\Om)$-収束する.
        第1項は,いま因数$\paren{\frac{n}{T}\int^{(j+2)\frac{T}{n}}_{(j+1)\frac{T}{n}}u_sds-u_{(j+1)\frac{T}{n}}}$が$\wh{\F}_{(j+1)\frac{T}{n}}$可測であるから,
        \begin{align*}
            E\Square{\paren{\sum_{j=0}^{n-2}(B_{(j+1)\frac{T}{n}}-B_{j\frac{T}{n}})\paren{\frac{n}{T}\int^{(j+2)\frac{T}{n}}_{(j+1)\frac{T}{n}}u_sds-u_{(j+1)\frac{T}{n}}}}^2}&=\sum_{j=0}^{n-2}\frac{T}{n}E\Square{\paren{\frac{n}{T}\int^{(j+2)\frac{T}{n}}_{(j+1)\frac{T}{n}}u_sds-u_{(j+1)\frac{T}{n}}}^2}
        \end{align*}
        と変形出来るが,$u_s$が連続であることより,これは明らかに$0$に収束する.
    \end{enumerate}
\end{Proof}
\begin{remarks}[11/30の講究の焦点]
    
\end{remarks}

\subsection{伊藤積分との関係}

\begin{proposition}[伊藤積分との関係]
    $\wh{B}_t:=B_T-B_{T-t}$とすると,これは$\wh{\F}_{T-t}$-Brown運動である.このとき,
    任意の後退可予測な過程$u\in L^2_T(\wh{\P})$について,次が成り立つ:
    \[\int^T_0u_t\wh{d}B_t=\int^T_0u_{T-t}d\wh{B}_t.\]
\end{proposition}
\begin{Proof}
    上述の後方Riemann和の極限としての特徴付けから明らか.
\end{Proof}

\section{確率積分の一般化}

\subsection{Riemann和としての方法の総覧}

\begin{tcolorbox}[colframe=ForestGreen, colback=ForestGreen!10!white,breakable,colbacktitle=ForestGreen!40!white,coltitle=black,fonttitle=\bfseries\sffamily,
title=マルチンゲールになるのは伊藤積分のみ]
    Brown運動の見本道が有界変動でないことより,$\ep\in[0,1]$をいじることで$L^2(\Om)$-極限が種々の様相を示す.
    マルチンゲールを基本言語に据えれば,古典的な微積分規則は棄却されて自然であることが判る.
\end{tcolorbox}

\begin{proposition}
    任意の$\ep\in[0,1]$について,
    \[\sum_{i\in[n]}((1-\ep)B_{t_i}+\ep B_{t_{i+1}})(B_{t_{i+1}}-B_{t_i})\xrightarrow[\abs{\pi_n}\to0]{L^2(\Om)}\frac{1}{2}B_t^2+\paren{\ep-\frac{1}{2}}t.\]
    が成り立つ.さらに,右辺がマルチンゲールであることと$\ep=0$とは同値.
\end{proposition}

\subsection{Young積分とラフパス}

\begin{tcolorbox}[colframe=ForestGreen, colback=ForestGreen!10!white,breakable,colbacktitle=ForestGreen!40!white,coltitle=black,fonttitle=\bfseries\sffamily,
title=]
    Terry Lyonsの発想は,Levyの確率面積の概念の助けを借りて,「逆にパスごとの定義に戻る」ということである.
    Riemann-Stieltjes積分の方法の一般化によって,$X_t,Y_t\in\R^d$がそれぞれ$\al,\beta$-Holder連続で$\al+\beta>1$を満たす場合に$\int^T_0X_tdY_t$が定義出来る.
\end{tcolorbox}


\begin{history}
    では一般の$f\in\Map(\R^d,\R)$について,伊藤積分と同様の方法で積分が定義出来るための十分条件を考えたい.
    これは
    \[B^{i,j}_{0,t}(\om):=\int^t_0B^i_u(\om)dB^j_u(\om)\quad(i,j\in[d])\]
    が定義できて(Young積分では出来ない),かつ,ある一定の代数的条件(Chen's identity)と解析的条件(Holder連続性)を満たす,
    という形のものが,Terry Lyonsによって発見された.
    そしてYoung積分の場合の議論と並行に,逐次積分にあたる積分が定義されれば,方程式の解はラフパスの連続な汎関数となる.
    この性質がラフパス解析に強力さを与えている.
\end{history}

\begin{context}
    Gauss過程,非整数Brown運動はセミマルチンゲールではなく,このような確率過程をBrown運動などのマルチンゲールから考察するにはラフパスの理論が必要になる.
\end{context}

\subsection{伊藤過程の一般化}



\begin{history}
    Ito, K. (1974). Stochastic Differentials. \textit{Applied Math. and Optimization.} 1:374-381. にて,
    数年前からquasi-martingaleと呼ばれていた半マルチンゲールの全体の空間$Q$を代数とみて,
    伊藤積分,Stratonovich積分,Kunita-Watanabeの二次変分などの操作が,Banach代数として理解することで"stochastic calculus"を打ち立てようという試みが見られる.
\end{history}

\begin{definition}[semi-martingale]
    $X\in C$が\textbf{連続な$(\F_t,P)$-セミマルチンゲール}であるとは,ある$M\in M_\loc\cap C$と$A\in C\cap\bF\cap\A$を用いて$X=M+A$と表せることをいう.
\end{definition}

\begin{proposition}
    $X$を連続なセミマルチンゲールとする.
    \begin{enumerate}
        \item $X$は有限な二次変動を持つ.
        \item $(X|X)=(M|M)$.
    \end{enumerate}
    特に,分解$X=M+A$は一意である.
\end{proposition}

\begin{definition}
    $X,Y$を連続な半マルチンゲールとする.
    \[(X|Y):=(M|N)=\frac{1}{4}((X+Y|X+Y)-(X-Y|X-Y))\]
    と定める.
\end{definition}

\subsection{伊藤積分の一般化}

\begin{tcolorbox}[colframe=ForestGreen, colback=ForestGreen!10!white,breakable,colbacktitle=ForestGreen!40!white,coltitle=black,fonttitle=\bfseries\sffamily,
title=]
    駆動過程は連続な半マルチンゲールにまで一般化され,この場合被積分過程は発展的可測な局所有界過程(狭くなっている).
\end{tcolorbox}

\begin{definition}
    $M\in H^2$を$L^2(\Om)$-有界な連続マルチンゲールとする.
    \begin{enumerate}
        \item 被積分関数は,有界変動過程$[M]_s$について自乗可積分な発展的可測過程の同値類$L^2(M)$とする.これはノルム\[\norm{K}_M^2:=E\Square{\int^\infty_0K_s^2d(M|M)_s}<\infty\]についてHilbert空間をなす.
        \item ある$K\cdot M\in{}^0\!H^2$が唯一つ存在して$\forall_{N\in H^2}\;(K\cdot M|N)=K\cdot(M|N)$を満たす.
        この対応$L^2(M)\mono {}^0\!H^2$は等長である.
    \end{enumerate}
\end{definition}

\begin{definition}\mbox{}
    \begin{enumerate}
        \item ここから,$M\in M_\loc\cap C$とし,任意の$d[M]_s$について局所自乗可積分な発展的可測過程$K\in L^2_\loc(M)$に対して,
        ある$K\cdot M\in {}^0\!M_\loc\cap C$が唯一つ存在して$\forall_{N\in M_\loc\cap C}\;(K\cdot M|N)=K\cdot(M|N)$.
        \item 最後に,$K\in L^\infty_\loc(\R_+\times\R^n)$を局所有界,$X=M+A$を連続な半マルチンゲールとすると,確率積分は
        \[K\cdot X:=K\cdot M+K\cdot A\]
        で定める.$K\cdot X$は再び連続な半マルチンゲールである.
    \end{enumerate}
\end{definition}

\begin{theorem}[優収束定理]
    $X$を連続な半マルチンゲール,$(K^n)$を局所有界な過程で$0$に各点収束する列で,ある局所有界な優過程$K$を持つものとする:$\forall_{n\in\N}\;\abs{K^n}\le K$.
    このとき,広義一様に$(K^n\cdot X)\xrightarrow{P}0$.
\end{theorem}

\begin{proposition}[Riemann和の極限としての特徴付け]
    $K$を左連続で局所有界とする.任意の分割$\pi_n\subset[0,t],\abs{\pi_n}\to0$について,
    \[\sum_{i\in[n]}K_{t_i}(X_{t_{i+1}}-X_{t_i})\xrightarrow{P}\int^t_0K_sdX_s.\]
\end{proposition}

\subsection{伊藤の公式の一般化}

\begin{history}
    Krylov, N. V. (1971). On an inequality in the theory of stochastic integrals. \textit{Th. of Prob. and its Appl.} 16: 438-448.
    にて,凸多面体の理論を用いて,$f\in C^2$から$f\in W^2_{d,\loc}$へ一般化した.これはBellman方程式と拡散過程のコントロールの議論で肝要になる.
\end{history}

\section{マルチンゲールの積分表現}

\begin{tcolorbox}[colframe=ForestGreen, colback=ForestGreen!10!white,breakable,colbacktitle=ForestGreen!40!white,coltitle=black,fonttitle=\bfseries\sffamily,
title=]
    不定積分の対応$L^2_\infty(\P)\ni u\mapsto(M_t=I(1_{[0,t]}u))\in\M^2$
    \[\xymatrix@R-2pc{
        L^2_\infty(\P)\ar[r]&\M^2(\bF)\\
        \rotatebox[origin=c]{90}{$\in$}&\rotatebox[origin=c]{90}{$\in$}\\
        u\ar@{|->}[r]&M_t=\int^t_0u_sdB_s.
    }\]
    は,
    Brown運動の情報系$\bF=(\F_t)$に関するマルチンゲールで二乗可積分なものの全体の空間$\M^2(\bF)$上に全射を定める.

    具体的な表示の実際の構成において,Malliavin解析が活躍する.
\end{tcolorbox}

\subsection{$L^2$-確率変数の表現}

\begin{tcolorbox}[colframe=ForestGreen, colback=ForestGreen!10!white,breakable,colbacktitle=ForestGreen!40!white,coltitle=black,fonttitle=\bfseries\sffamily,
title=]
    まず$T>0$を止めて考える.
    すると,確率積分は,$\F_T$-可測な自乗可積分確率変数の全体$L^2_{\F_T}(\Om)$上に全射$L^2_T(\P)\epi L^2_{\F_T}(\Om)$を定める.
    すなわち,自乗可積分確率変数が確率積分であることは,Brown運動の情報系に対して可測であることに同値.
\end{tcolorbox}

\begin{theorem}[integral representation theorem for $L^2$ random variable (Ito)]
    \label{thm-integral-representation-for-L2-rv}
    任意の$T>0$,$F\in L^2_{\F_T}(\Om)$に対して,唯一つの過程$u\in L^2_T(\P)$が存在して,
    \[F=E[F]+\int^T_0u_sdB_s.\]
\end{theorem}
\begin{Proof}\mbox{}
    \begin{description}
        \item[モデルとなる議論] まず,$F\in L^2_{\F_T}(\Om)$が,決定論的な関数$h\in L^2([0,T])$を用いた次のような表式を持つ場合を考える:
        \[F=\exp\paren{\int^T_0h_sdB_s-\frac{1}{2}\int^T_0h^2_sds}.\]
        この$F$に至る過程
        \[Y_t=\exp\paren{\underbrace{\int^t_0h_sdB_s-\frac{1}{2}\int^t_0h^2_sds}_{=:X_t}}\quad t\in[0,T]\]
        は,$f(x)=e^x$に関する伊藤の公式より,
        \begin{align*}
            Y_t&=f(X_t)=f(X_0)+\int^t_0f'(X_s)h_sdB_s-\int^t_0\frac{1}{2}f'(X_s)h_s^2ds+\frac{1}{2}\int^t_0f''(X_s)h^2_sds\\
            &=1+\int^t_0Y_sh_sdB_s.
        \end{align*}
        $t=T$を代入することで,
        \[F=1+\int^T_0Y_sh_sdB_s\]
        を得る.これはたしかに,確率積分は中心化されていることより$E[F]=1$,$Yh\in L^2_\infty(\P)$を満たすことに注意.

        実際これは,$(Y_t)_{t\in[0,T]}$が連続な局所マルチンゲールであり,$E[\abs{Y_T}^2]<\infty$より,Doobの不等式から
        \[E\Square{\sup_{t\in[0,T]}\abs{Y_t}^2}\le 4E[\abs{F}^2]\]
        を得るため,Cauchy-Schwarzの不等式から
        \[E\Square{\int^T_0(Y_sh_s)^2ds}\le E\Square{\sup_{t\in[0,T]}\abs{Y_t}^2\int^T_0 h_s^2ds}\le 4\norm{F}_{L^2(\Om)}\norm{h}_{L^2([0,T])}.\]
        \item[問題の所在] $F$が上述の「決定論的関数が定める$0$から始まる伊藤過程の指数関数」の形の線型和で表せる場合も同様.
        あとは,一般の$F\in L^2_{\F_T}(\Om)$が,上述の「決定論的関数が定める$0$から始まる伊藤過程の指数関数」の形の確率変数の線型和の列$(F_n)$で$L^2(\Om)$-近似できることを用いれば良い.
        \begin{description}
            \item[存在] 上の議論より,$u^{(n)}\in L^2_T(\P)$が存在して,
            \[F_n=E[F_n]+\int^T_0u_s^{(n)}dB_s\]
            と表せる.確率積分の等長性より,
            \begin{align*}
                E[(F_n-F_m)^2]&\ge\Var[F_n-F_m]=E\Square{\paren{\int^T_0(u^{(n)}_s-u^{(m)}_s)dB_s}^2}\\
                &=E\Square{\int^T_0(u^{(n)}_s-u^{(m)}_s)^2ds}.
            \end{align*}
            であるから,$u^{(n)}$も$L^2_T(\P)$のCauchy列である.
            よって,極限$u\in L^2_T(\P)$が存在し,Lebesgueの優収束定理と再び確率積分の等長性(特に$L^2$-連続性)より,
            \begin{align*}
                F&=\lim_{n\to\infty}F_n=\lim_{n\to\infty}\paren{E[F_n]+\int^T_0u_s^{(n)}dB_s}\\
                &=E[F]+\int^T_0u_sdB_s.
            \end{align*}
            \item[一意性] 仮に$u^{(1)},u^{(2)}\in L^2_T(\P)$について
            \[F=E[F]+\int^T_0u^{(1)}_sdB_s=E[F]+\int^T_0u^{(2)}dB_s\]
            と表せるならば,確率積分の等長性より,
            \[0=E\Square{\paren{\int^T_0(u^{(1)}_s-u^{(2)}_s)dB_s}^2}=E\Square{\int^T_0(u^{(1)}_s-u^{(2)}_s)^2ds}\]
            から,$u_s^{(1)}(\om)=u_s^{(2)}(\om)\;\ae(\om,s)\in\Om\times[0,T]$.
        \end{description}
    \end{description}
\end{Proof}

\begin{lemma}[指数型マルチンゲールの全体性]
    \[\E:=\Brace{F(h):=\exp\paren{\int^T_0h_sdB_s-\frac{1}{2}\int^T_0h^2_sds}\in L^2_{\F_T}(\Om)\;\middle|\;h\in L^2([0,T])}\]
    が生成する線型部分空間は$L^2_{\F_T}(\Om)$上稠密である.
    この条件を,$L^2_{\F_T}(\Om)$上\textbf{total}であるという.
\end{lemma}
\begin{Proof}
    実は$[0,T]$上の単関数の全体を$\S\subset L^2([0,T])$とすれば,
    $\{F(h)\}_{h\in\S}$がすでにtotalである.
    \begin{description}
        \item[方針] 一般に局所凸空間の凸部分集合の,弱位相に関する稠密集合と元来の位相についての稠密集合とは一致する.
        さらに$L^2_{\F_T}(\Om)$はHilbert空間で,弱位相を$w^*$-位相とは一致するから,
        従って,$\E$がtotalであることを示すには,$\E$が$L^2_{\F_T}(\Om)$上の分離族であることを示せば良い(Hahn-Banachの延長定理の帰結).
        すなわち,任意の$Y\in L^2(\Om,\F_T)$が$\forall_{F\in\E}\;(Y|F)=0$を満たすならば,$Y=0$であることを示せば良い.
        \item[主張] 任意の$\Brace{0=t_0<t_1<\cdots<t_n}\subset[0,T]$に対して,$\R^n$上の測度を
        \[\mu(A):=E[Y1_A(B_{t_1}-B_{t_0},\cdots,B_{t_n}-B_{t_{n-1}})]\qquad A\in\B(\R^n)\]
        で定める.すると,
        関数
        \[\varphi(u_1,\cdots,u_n)=E\Square{Y\exp\paren{\sum_{i=1}^nu_i(B_{t_i}-B_{t_{i-1}})}}\]
        は明らかに$\C^n$上の正則関数に延長するから,$Y$についての仮定より$\C^n$上で$0$.
        特に$\mu$のFourier変換が
        \[E\Square{Y\exp\paren{i\sum_{i=1}^nu_i(B_{t_i}-B_{t_{i-1}})}}=0\]
        であることから,Fourier変換の単射性より$\mu=0$が解る.
        $\sigma[B_{t_1}-B_{t_0},\cdots,B_{t_n}-B_{t_{n-1}}]=\sigma[B_{t_1},\cdots,B_{t_n}]$より,
        特に,$\forall_{A\in\sigma[B_{t_1},\cdots,B_{t_n}]}\;E[Y1_A]=0$.
        \item[証明] 
        ここで,$\sigma[B_{t_1},\cdots,B_{t_n}]$の形の部分$\sigma$-代数は$\sigma[B_t;t\ge0]$の乗法系をなすから,任意の$A\in\F_T$についても$E[Y1_A]=0$.これは$Y=0$を意味する.
    \end{description}
\end{Proof}
\begin{remarks}[指数関数の論理的重要性]
    これが「指数型のマルチンゲール」に注目する所以の一つであろうが,このようなことが成り立つための大事な理由の一つに,証明の途中の$\varphi$の解析接続が成功しているということがある.
\end{remarks}

\subsection{$M^2$-確率過程の表現}

\begin{tcolorbox}[colframe=ForestGreen, colback=ForestGreen!10!white,breakable,colbacktitle=ForestGreen!40!white,coltitle=black,fonttitle=\bfseries\sffamily,
title=]
    Brown運動の情報系$\bF$に関しては,$\M^2(\bF)$の元が$\M^2(\bF)\cap C$に属する修正によって表現出来ることが系として得られる.
\end{tcolorbox}

\begin{corollary}[martingale representation theorem]
    $M\in\M_2$は二乗可積分な$(\F_t)$-マルチンゲールとする.このとき,唯一の$u\in L^2_\infty(\P)$が存在して,
    \[\forall_{t\in\R_+}\quad M_t=E[M_0]+\int^t_0u_sdB_s.\]
    と表せる.特に,$M$は連続な修正を持つ.
\end{corollary}
\begin{Proof}
    任意にマルチンゲール$\{M_t\}\subset L^2(\Om)$を取る.
    このとき,任意の$t\in\R_+$に対して,ある$u^{(t)}\in L^2_t(\P)$が存在して,
    \[M_t=E[M_t]+\int^{t}_0u^{(t)}dB=E[M_0]+\int^t_0u^{(t)}dB\]
    と表せる.
    このとき,任意の$0\le s<t$に対して$u^{(s)}=u^{(t)}$であることを示せば良い.
    いま,マルチンゲール性$E[M_t|\F_s]=M_s$より,
    \[E[M_0]+E\Square{\int^t_0u^{(t)}dB\middle|\F_s}=\int^s_0u^{(t)}dB=\int^s_0u^{(s)}dB.\]
    $u^{(s)},u^{(t)}\in L^2_t(\P)$上で確率積分は全単射だから,$u^{(s)}=u^{(t)}$が従う.
\end{Proof}
\begin{remarks}\mbox{}
    \begin{enumerate}
        \item 任意の連続な局所マルチンゲール$M$に関しても同様の結論が成り立つ\cite{LeGall}.
        \item Brown運動の情報系に関する任意の停止時は可予測である\cite{Revuz-Yor} (V.3.3).
    \end{enumerate}
\end{remarks}

\begin{example}[Brown運動の3乗のマルチンゲール表現]
    伊藤の公式より
    \[B^3_T=\int^T_03B^2_tdB_t+3\int^T_0B_tdt\]
    であった\ref{exp-n-exponential-of-Brownian-motion}.
    しかしこの有界変動部分は,$X_t=B_t,Y_t=t$に関する部分積分公式\ref{cor-integration-by-parts}から,
    \[TB_T=\int^T_0B_tdt+\int^T_0tdB_t+\int^T_0dB_tdt\]
    より,
    \[\int^T_0B_tdt=TB_T-\int^T_0tdB_t=\int^T_0(T-t)dB_t.\]
    以上を併せて,
    \[\tcboxmath{B^3_T=\int^T_03(B_t^2+(T-t))dB_t.}\]
\end{example}

\section{Girsanovの定理}

\begin{tcolorbox}[colframe=ForestGreen, colback=ForestGreen!10!white,breakable,colbacktitle=ForestGreen!40!white,coltitle=black,fonttitle=\bfseries\sffamily,
title=]
    (線型な)ドリフトを持ったBrown運動が通常のBrown運動に戻るような$C_0(\R_+)$上の測度変換を与える.
    Cameron and Martin (1944)が最初に示したが,Girsanov (1960)が一般化した.
\end{tcolorbox}

\begin{definition}[probability density function]
    $(\Om,\F,P)$を確率空間,
    $L\in L^1(\Om)_+$を平均$1$の非負確率変数とする.
    \begin{enumerate}
        \item $Q[A]:=E[1_AL]$とおくと,これは新しい確率測度を定める.
        \item $L$は$Q$の$P$に対する\textbf{(確率)密度(関数)}といい,$L=\dd{Q}{P}$と表す.
        \item 当然$Q\ll P$である:$P[A]=0\Rightarrow Q[A]=0$.
        \item $L>0\;P\das$とき,$P\ll Q$でもあり,2つの確率測度は\textbf{同値}または\textbf{互いに絶対連続}であるという.これは,零集合の全体が一致することに同値:$\cN(P)=\cN(Q)$.
    \end{enumerate}
\end{definition}

\subsection{Novikovの条件}

\begin{tcolorbox}[colframe=ForestGreen, colback=ForestGreen!10!white,breakable,colbacktitle=ForestGreen!40!white,coltitle=black,fonttitle=\bfseries\sffamily,
title=]
    指数マルチンゲール$(L_t)$が,本当にマルチンゲール(一般には一様可積分なマルチンゲール)になるための条件を与えるのがNovikovである.
    この関門がGirsanovの定理の本質である.
    なお,$L_t\ge0$で,$L_0\in L^1(\Om)$であるため,優マルチンゲールであることはわかっている.
\end{tcolorbox}

\begin{lemma}[確率変動するドリフト$\theta t$を持ったBrown運動]\label{lemma-Novikov}
    $(\Om,\F,P)$を確率空間,$(B_t)_{t\in\R_+}$をその上の$(\F_t)$-Brown運動とする.
    拡散係数の過程
    $\theta\in L^2_T(\P)$に対して,これが定める指数マルチンゲールを
    \[L_t:=\exp\paren{\int^t_0\theta_sdB_s-\frac{1}{2}\int^t_0\theta^2_sds}=:e^{M_t-A_t}\quad(t\in[0,T])\]
    と定める.
    \begin{enumerate}
        \item $L$は局所マルチンゲールであり,
        次の線型な確率微分方程式を満たす:
        \[L_t=1+\int^t_0\theta_sL_sdB_s.\]
        \item (Novikov 72) さらに次も満たすならば,$\forall_{t\in[0,T]}\;E[L_t]=1$が成り立つ.
        特に,$(L_t)_{t\in[0,T]}$はマルチンゲールである:
        \[E\Square{\exp\paren{\frac{1}{2}\int^T_0\theta^2_sds}}=E[e^{A_T}]<\infty.\]
    \end{enumerate}
\end{lemma}
\begin{Proof}\mbox{}
    \begin{enumerate}
        \item 伊藤過程
        \[X_t:=\int^t_0\theta_sdB_s-\frac{1}{2}\int^t_0\theta^2_sds=:M_t-A_t\]
        の可微分関数$f(x)=e^x$に関する変換であるから,伊藤の公式より解る.
        \item 
        \begin{description}
            \item[局所マルチンゲール部分$M_t$がマルチンゲールである] 与えられた条件は
            \[E\Square{\exp\paren{\frac{1}{2}\brac{M}_T}}<\infty\]
            と同値であるから,特に時刻$T$の二次変分の確率変数$\brac{M}_T$は任意階数$n=1,2,\cdots$の積率を持つ.
            すると,$M_\loc\cap C$に関するBurkholderの不等式より,$M^*_T:=\sup_{t\in[0,T]}\abs{M_t}$も任意階数の積率を持つ.
            よって,$M$はマルチンゲールである($T=\infty$のときもこの議論で$M$が一様可積分なマルチンゲールであることが解る).
            \item[$e^{M_t/2}$は劣マルチンゲールである] 
            \[e^{\frac{1}{2}M_T}=e^{\frac{1}{2}M_T-\frac{1}{4}\brac{M}_T}e^{\frac{1}{4}\brac{M}_T}\]
            に関する
            Cauchy-Schwarzの不等式より,
            \begin{align*}
                E\Square{e^{\frac{1}{2}M_T}}&\le\paren{E\Square{e^{M_T-\frac{1}{2}\brac{M}_T}}E\Square{e^{\frac{1}{2}\brac{M}_T}}}^{1/2}\\
                &=E[L_T]^{1/2}E\Square{e^{\frac{1}{2}\brac{M}_T}}^{1/2}<\infty
            \end{align*}
            を得る.よって,$(e^{\frac{1}{2}M_t})_{t\in[0,T]}$は劣マルチンゲールである.
            \item[問題の所在] 任意の$\eta<1$について,
            \[\exp\paren{\eta M_t-\frac{\eta^2}{2}\brac{M}_t}=\paren{\exp\paren{M_t-\frac{1}{2}\brac{M}_t}}^{\eta^2}\paren{\exp\paren{\frac{\eta M_t}{1+\eta}}}^{1-\eta^2}.\]
            なる分解に対するHolderの不等式より,
            \begin{align*}
                E\Square{\exp\paren{\eta M_t-\frac{\eta^2}{2}\brac{M}_t}}&\le E\Square{\exp\paren{M_t-\frac{1}{2}\brac{M}_t}}^{\eta^2}E\Square{\exp\paren{\frac{\eta M_t}{1+\eta}}}^{1-\eta^2}\\
                &\le E\Square{\exp\paren{M_t-\frac{1}{2}\brac{M}_t}}^{\eta^2}E\Square{\exp\paren{\frac{M_t}{2}}}^{2\eta(1-\eta)}\\
                &\le E\Square{\exp\paren{M_t-\frac{1}{2}\brac{M}_t}}^{\eta^2}E\Square{\exp\paren{\frac{M_T}{2}}}^{2\eta(1-\eta)}.
            \end{align*}
            なる評価が得られる.
            ここで,
            \[\forall_{t\in\R_+}\;\forall_{\eta<1}\;E\Square{\exp\paren{\eta M_t-\frac{\eta^2}{2}\brac{M}_t}}=1\]
            が成り立つから,$\eta\to1$とすることで,
            \[E\Square{\exp\paren{M_t-\frac{1}{2}\brac{M}_t}}\ge1\]
            を得る.$L_t$の優マルチンゲール性より,
            \[\forall_{t\in\R_+}\quad E\Square{\exp\paren{M_t-\frac{1}{2}\brac{M}_t}}=1\]
            が結論付けられる.
            \item[証明] $p>1$を
            \[\frac{\eta\sqrt{p}}{\sqrt{p}-1}\le1,\quad r:=\frac{\sqrt{p}+1}{\sqrt{p}-1},s:=\frac{\sqrt{p}+1}{2}\]
            を満たすように取ると,$r,s$は共役.これについての分解
            \[\exp\paren{\eta M_t-\frac{\eta^2}{2}\brac{M}_t}^p=\exp\paren{\sqrt{\frac{p}{r}}\eta M_t-\frac{p}{2}\eta^2\brac{M}_t}\exp\paren{\paren{p\eta-\sqrt{\frac{p}{r}}}M_t}\]
            に対してHolderの不等式を用いると,$M$のみに依存する定数$C>0$が存在して,
            \[\forall_{T\in\bT^{<\infty}}\quad E\Square{\paren{\exp\paren{\eta M_T-\frac{\eta^2}{2}\brac{M}_T}}^p}\le C.\]
            Doobの最大不等式より,局所マルチンゲール$\exp\paren{\eta M_t-\frac{\eta^2}{2}\brac{M}_t}$は実は真のマルチンゲールであり,よって,
            \[E\Square{\exp\paren{\eta M_t-\frac{\eta^2}{2}\brac{M}_t}}=1\]
            を得る.
        \end{description}
    \end{enumerate}
\end{Proof}
\begin{remarks}\mbox{}
    \begin{enumerate}
        \item $L$がマルチンゲールになることにより,$E[L_T]=1$で,これを密度に持つ確率分布$Q$を得る:$L_T=\dd{Q}{P}$.
        \item 残りの$L_t$は,$\forall_{t\in[0,T]}\;L_t=E[L_T|\F_t]$として得られるから,$Q$の$\F_t$上での密度関数である.実際,
        \begin{align*}
            \forall_{A\in\F_t}\quad Q[A]&=E[1_AL_T]=E[E[1_AL_T|\F_t]]\\
            &=E[1_AE[L_T|\F_t]]=E[1_AL_t].
        \end{align*}
    \end{enumerate}
\end{remarks}

\begin{lemma}[一般の十分条件]
    拡散係数の過程
    $\theta\in L^2(\P)$に対して,これが定める指数マルチンゲールを
    \[L_t:=\exp\paren{\int^t_0\theta_sdB_s-\frac{1}{2}\int^t_0\theta^2_sds}\quad(t\in\R_+)\]
    と定める.
    $\forall_{t\in\R_+}\;E[L_t]=1$を満たすならば,$L$は一様可積分なマルチンゲールである.
\end{lemma}
\begin{Proof}
    局所マルチンゲール$L$に$\forall_{t\in\R_+}\;E[L_t]=1$を課すと,さらにDirichlet類でもあるから,一様可積分なマルチンゲールになる.
    なお,そもそも$L$は非負な局所マルチンゲールであるから,優マルチンゲールである.
    優マルチンゲールが$\forall_{t\in\R_+}\;E[L_t]=E[L_0]$を満たすならばマルチンゲールであることは一般論である.
\end{Proof}
\begin{remarks}
    次の観察と同様のことが成功するためである.
\end{remarks}

\subsection{Girsanovの定理}

\begin{tcolorbox}[colframe=ForestGreen, colback=ForestGreen!10!white,breakable,colbacktitle=ForestGreen!40!white,coltitle=black,fonttitle=\bfseries\sffamily,
title=]
    伊藤の公式による局所マルチンゲールの構成\ref{exp-local-martingale-from-Ito-formula}から,
    マルチンゲールを構成出来る場合を考える.
    そしてこれを元のBrown運動に戻すための測度変換を与える.
\end{tcolorbox}

\begin{observation}[正規確率変数を中心化する測度変換の与え方]
    $(\Om,\F,P)$上の正規確率変数
    $X\sim N(m,\sigma^2)$に対して,
    \[L:=\exp\paren{-\frac{m}{\sigma^2}X+\frac{m^2}{2\sigma^2}}\]
    と定めると,
    \[E[L]=\int_\R e^{\frac{m^2-2mx}{2\sigma^2}}\frac{1}{\sqrt{2\pi\sigma^2}}e^{-\frac{(x-m)^2}{2\sigma^2}}dx=\frac{1}{\sqrt{2\pi\sigma^2}}\int_\R e^{-\frac{x^2}{2\sigma^2}}dx=1.\]
    したがって$L$はある確率分布$Q$の密度関数と見れる:$\dd{Q}{P}=L$.
    この新たな確率分布を備えた確率空間$(\Om,\F,Q)$上で確率変数$X$を見直して見ると,中心化されている.
    実際,その特性関数は,
    \begin{align*}
        E_Q[e^{itX}]&=E[e^{itX}L]=\frac{1}{\sqrt{2\pi\sigma^2}}\int_\R e^{-\frac{x^2}{2\sigma^2}+itx}dx=\exp\paren{-\frac{\sigma^2t^2}{2}}.
    \end{align*}
    であるから,$X\sim N(0,\sigma^2)$.
    総じて,次のように密度を変換して,正規確率変数$X$を中心化した:
    \[\frac{1}{\sqrt{2\pi\sigma^2}}e^{-\frac{(x-m)^2}{2\sigma^2}}dx\mapsto L\frac{1}{\sqrt{2\pi\sigma^2}}e^{-\frac{(x-m)^2}{2\sigma^2}}dx=\frac{1}{\sqrt{2\pi\sigma^2}}e^{-\frac{x^2}{2\sigma^2}}.\]
\end{observation}

\begin{theorem}[Girsanov (1960)]\label{thm-Girsanov}
    $\theta\in L^2_T(\P)$がNovikovの条件
    \[E\Square{\exp\paren{\frac{1}{2}\int^T_0\theta^2_sds}}<\infty\]
    を満たし,$B$を$(\Om,\F,P,(\F_t)_{t\in\R_+})$上のBrown運動とする.このとき,
    \[W_t:=B_t-\int^t_0\theta_sds\]
    は確率空間$(\Om,\F_T,Q,(\F_t)_{t\in[0,T]})$上でみれば,$[0,T]$上のBrown運動である.
\end{theorem}
\begin{Proof}\mbox{}
    \begin{description}
        \item[問題の所在] $W$は連続過程であり,$W_0=0\;\as$であるから,あとは任意の$s<t\in[0,T]$について,$W_t-W_s$は$\F_s$と独立で$N(0,t-s)$に従うことを示せば良い.
        すなわち,任意の$s<t\in[0,T],A\in\F_s$について,
        \[\forall_{\lambda\in\R}\quad E_Q[1_Ae^{i\lambda(W_t-W_s)}]=Q[A]e^{-\frac{\lambda^2}{2}(t-s)}\]
        を示せば良い(Kacの定理による独立性の特徴付けと併せて).
        \item[証明] $Q$の$(\F_s\subset)\F_t$上での$P$に対する密度は$L_t$であるから,
        \begin{align*}
            E_Q[1_Ae^{i\lambda(W_t-W_s)}]&=E[1_Ae^{i\lambda(W_t-W_s)}L_t]\\
            &=E\Square{1_A\exp\paren{i\lambda\paren{B_t-B_s-\int^t_s\theta_\nu d\nu}}\exp\paren{\int^t_0\theta_\nu dB_\nu-\frac{1}{2}\int^t_0\theta^2_\nu d\nu}}\\
            &=E\Square{1_A\exp\paren{\int^s_0\theta_\nu dB_\nu-\frac{1}{2}\int^t_0\theta^2_\nu d\nu}\exp\paren{\int^t_s(i\lambda+\theta_\nu)dB_\nu-\frac{1}{2}\int^t_s(i\lambda+\theta_\nu)^2d\nu}}\\
            &=E\Square{1_A\exp\paren{\int^s_0\theta_\nu dB_\nu-\frac{1}{2}\int^s_0\theta^2_\nu d\nu}\exp\paren{\int^t_s(i\lambda+\theta_\nu)dB_\nu-\frac{1}{2}\int^t_s((i\lambda+\theta_\nu)^2+\lambda^2)d\nu}}\\
            &=E[1_AL_s\Psi_{s,t}]e^{-\frac{\lambda^2}{2}(t-s)}\qquad \Psi_{s,t}:=\exp\paren{\int^t_s(i\lambda+\theta_\nu)dB_\nu-\frac{1}{2}\int^t_s(i\lambda+\theta_\nu)^2d\nu}.
        \end{align*}
        と計算できる.ここで,補題より$E[\Psi_{s,t}|\F_s]=1$であるから,
        \begin{align*}
            E[1_AL_s\Psi_{s,t}]&=E[E[1_AL_s\Psi_{s,t}|\F_s]]\\
            &=E[1_AL_sE[\Psi_{s,t}|\F_s]]\\
            &=E[1_AL_s]=E_Q[1_A]=Q[A].
        \end{align*}
    \end{description}
\end{Proof}

\subsection{ドリフトを持ったBrown運動の到達時刻}

\begin{tcolorbox}[colframe=ForestGreen, colback=ForestGreen!10!white,breakable,colbacktitle=ForestGreen!40!white,coltitle=black,fonttitle=\bfseries\sffamily,
title=]
    定常的なドリフトを持った場合,
    指数過程$L_t$は幾何Brown運動\ref{def-GBM}になる.
    線型なドリフトを持つBrown運動の到達時刻が,幾何Brown運動を通じて,Brown運動への測度変換を考えることで研究できる.
\end{tcolorbox}

\begin{notation}
    $\theta\in\R$を時間に依らないとすると,
    \[\left.\frac{dQ}{dP}\right|_{\F_t}=L_t=\exp\paren{\theta B_t-\frac{\theta^2}{2}t}\]
    という退化した場合を得る.
    このとき,Girsanovの定理より,$P$の下でのBrown運動$B$は,$Q$の下ではドリフト$\theta t$を持ったBrown運動である.
\end{notation}

\begin{proposition}
    $a\ne0$への到達時刻を
    $\tau_a:=\inf\Brace{t\in\R_+\mid B_t=a}$とすると,この$Q$に関する確率密度関数は
    \[\dd{\tau_a}{Q}=f(s)=\frac{\abs{a}}{\sqrt{2\pi s^3}}\exp\paren{-\frac{(a-\theta s)^2}{2s}}\quad(s>0)\]
    となる.
\end{proposition}
\begin{Proof}\mbox{}
    \begin{description}
        \item[可測性] $\forall_{t\in\R_+}\;\Brace{\tau_a\le t}\in\F_{\tau_a\land t}$が成り立つ.
        実際,任意の$s\in\R_+$に対して,
        \begin{align*}
            \Brace{\tau_a\le t}\cap\Brace{\tau_a\land t\le s}&=\Brace{\tau_a\le t}\land\Brace{\tau_a\le s}=\Brace{\tau_a\le t\land s}\in\F_{s\land t}\subset\F_s.
        \end{align*}
        が確認できる.
        \item[任意停止定理] 実際,$(\F_t)$マルチンゲール$L_t$に対する任意停止定理より,
        \begin{align*}
            Q[\tau_a\le t]&=E[1_{\Brace{\tau_a\le t}}L_t]=E[1_{\Brace{\tau_a\le t}}E[L_t|\F_{\tau_a\land t}]]\\
            &=E[1_{\Brace{\tau_a\le t}}L_{\tau_a\land t}]=E[1_{\Brace{\tau_a\le t}}L_{\tau_a}]\\
            &=E\Square{1_{\Brace{\tau_a\le t}}\exp\paren{\theta a-\frac{1}{2}\theta^2\tau_a}}\\
            &=\int^t_0\exp\paren{\theta a-\frac{1}{2}\theta^2s}g(s)ds.
        \end{align*}
        と,$\tau_a$の$P$に関する密度\ref{cor-density-of-Brownian-hitting-time}
        \[g(s)=\frac{\abs{a}}{\sqrt{2\pi s^3}}\exp\paren{-\frac{a^2}{2s}}\]
        を用いてかける.よって,
        \[\dd{\tau_a}{Q}=\exp\paren{\theta a-\frac{1}{2}\theta^2s}g(s)=\frac{\abs{a}}{\sqrt{2\pi s^3}}\exp\paren{-\frac{(a-\theta s)^2}{2s}}.\]
    \end{description}
\end{Proof}

\begin{corollary}
    有限時間内に$a\in\R$に到達する確率は,
    \[Q[\tau_a<\infty]=e^{\theta a-\abs{\theta a}}.\]
\end{corollary}
\begin{Proof}
    命題の証明中で用いた$t\in\R_+$を$t\nearrow\infty$と考えることで,
    \begin{align*}
        Q[\tau_a<\infty]&=e^{\theta a}\int^\infty_0e^{-\frac{1}{2}\theta^2s}g(s)ds\\
        &=e^{\theta a}E[e^{-\frac{1}{2}\theta^2\tau_a}]=e^{\theta a-\abs{\theta a}}.
    \end{align*}
    ただし,到達時刻のLaplace変換\ref{prop-hitting-time}は,
    $E[e^{-\frac{\theta^2}{2}\tau_a}]=e^{-\sqrt{\theta}\abs{a}}=e^{-\abs{\theta a}}$に注意.
\end{Proof}
\begin{remarks}
    $\theta a\ge0$のとき,すなわち,位置$a$とドリフトの方向が一致するか,ドリフトがない場合であるが,このとき確率1で到達する.
    一方で$\theta a<0$だと,確率は$\exp(-2\theta a)$となる.ドリフトが強くて$a$が遠いほど,確率は指数関数的に減少する.
    なお,$\theta=0$の通常のBrown運動においても,$P[\tau_a<\infty]=1$であるが,$E[\tau_a]=\infty$である\ref{cor-density-of-Brownian-hitting-time}.
\end{remarks}

\chapter{Brown運動に基づくMalliavin解析}

\begin{quotation}
    Malliavin (1976) はWiener空間またはGauss空間上での無限次元解析を創始した.
    これは当初Malliavinが「確率変分法」と呼んだ通り,見本道の空間上の変分法を与える理論で,
    拡散過程の遷移確率密度に対する真に確率論的な扱いを可能にする.

    Brown運動に限らず,一般の等直交なGauss過程$H\to L^2(\Om)$について定義出来る.
\end{quotation}

\begin{notation}
    滑らかな関数であって,
    それ自身とその任意解の偏導関数が多項式増大である関数の全体を
    $C_p^\infty(\R^n)$で表す.
\end{notation}
\begin{remark}
    通常,微分により多項式増大関数のクラスから飛び出ることがある.例えば$f(x):=\sin(e^x)$の導関数は$f'(x)=e^x\cos(e^x)$であるが,緩増加分布$f\in\S'(\R)$ではある.
\end{remark}

\section{有限次元での議論}

\begin{tcolorbox}[colframe=ForestGreen, colback=ForestGreen!10!white,breakable,colbacktitle=ForestGreen!40!white,coltitle=black,fonttitle=\bfseries\sffamily,
title=]
    \begin{enumerate}
        \item $\R^n$上の関数の微分$D:C^\infty_p(\R^n)\to\X^1(\R^n)$の,
        $L^2(\rN_n(0,I_n))$での随伴は
        発散作用素$\delta:\X^1_p(\R^n)\to C^1(\R^n)$である.
        \item 自然な定義域$D^p:D^{p,2}\to L^2(\rN)$上にこの関係を保ったまま延長出来る.
        \item これは結局,$(\R^n,\rN_n(0,I_n))$,1次元で言えば$(\R,e^{-\frac{x^2}{2}}dx)$上の関数解析・調和解析である.
    \end{enumerate}
\end{tcolorbox}

\subsection{微分作用素の定義と随伴関係}

\begin{definition}[derivative operator, divergence operator]
    確率空間を
    $(\Om,\F,P):=(\R^n,\B(\R^n),\rN_n(0,I_n))$とする.$L(\Om)$を$L(\rN)$と表す.
    \begin{enumerate}
        \item 勾配作用素$D:=\nabla=\paren{\pp{}{x_1},\cdots,\pp{}{x_n}}$を\textbf{微分作用素}と改名する.
        \item 可微分ベクトル場をスカラーに写す作用素$\delta:C^1(\R^n;\R^n)\to C(\R^n)$
        \[\delta(u):=\sum_{i\in[n]}\paren{u_ix_i-\pp{u_i}{x_i}}=(u|x)-\div\, u,\qquad (u:\R^n\to\R^n)\]
        を\textbf{発散作用素}という.
    \end{enumerate}
\end{definition}

\begin{proposition}[随伴関係]\label{prop-adjoint-of-Malliavin-Derivative-on-finite-dimensional-Gaussian-space}
    測度空間$(\R^n,\B(\R^n))$上に標準Gauss測度$\rN_n(0,I_n)$を考え,
    $F\in C_p^1(\R^n),u\in C_p^1(\R^n;\R^n)$をそれぞれ偏導関数も含め多項式増大とする.このとき,次が成り立つ:
    \[E[(u|DF)]=E[F\delta(u)].\]
\end{proposition}
\begin{Proof}\mbox{}
    \begin{enumerate}[{Step}1]
        \item $\rN_n(0,I_n)$の密度関数を
        \[p_n(x):=\frac{1}{(2\pi)^{\frac{n}{2}}}e^{-\frac{\abs{x}^2}{2}},\qquad(x\in\R^n).\]
        $p_n:\R^n\to[0,1]$で表すと,
        \[\pp{p_n}{x_i}(x)=-x_ip_n(x),\qquad(x\in\R^n).\]
        が成り立つ.
        \item $F,u_i$も多項式増大であるから,$F(x)u_i(x)p_n(x)\xrightarrow{\abs{x}\to\infty}0$.
        \item 部分積分により,次のように計算出来る:
        \begin{align*}
            \int_{\R^n}(u|DF)p_n\,dx&=\sum_{i\in[n]}\int_{\R^n}\pp{F}{x_i}u_ip_n\,dx\\
            &=\sum_{i\in[n]}\paren{\SQuare{Fu_ip_n}_{\R^n}-\int_{\R^n}F\pp{(u_ip_n)}{x_i}\,dx}\\
            &=\sum_{i\in[n]}\paren{-\int_{\R^n}F\pp{u_i}{x_i}p_n\,dx+\int_{\R^n}Fu_ix_ip_n\,dx}\\
            &=\int_{\R^n}Fp_n\paren{\sum_{i\in[n]}u_ix_i-\pp{u_i}{x_i}}dx=\int_{\R^n}F\delta(u)p\,dx.
        \end{align*}
    \end{enumerate}
\end{Proof}

\subsection{多項式増大関数の空間について}

\begin{tcolorbox}[colframe=ForestGreen, colback=ForestGreen!10!white,breakable,colbacktitle=ForestGreen!40!white,coltitle=black,fonttitle=\bfseries\sffamily,
title=]
    以降$n=1$の場合を考える.この1次元理論では関数解析の様相を多分に呈する.
    続いて$N\sim\rN(0,1)$を無限次元化することを考える\cite{Nourdin-Peccati12-NormalApproximation}.
\end{tcolorbox}

\begin{proposition}[\cite{Nourdin-Peccati12-NormalApproximation} Prop. 1.1.5]\mbox{}\label{prop-denseness-of-monomial-in-Gaussian-space}
    \begin{enumerate}
        \item 単項式の全体$\{x^n\}_{n\in\N}$は$L^q(\rN(0,1))\;(q\in[1,\infty))$上で全体的である,すなわち,稠密な線型部分空間を生成する.
        \item $C_p^\infty(\R)$は$L^q(\rN(0,1))\;(q\in[1,\infty))$上で稠密である.
    \end{enumerate}
\end{proposition}
\begin{Proof}\mbox{}
    \begin{enumerate}
        \item 単項式の全体$\{x^n\}_{n\in\N}$が$L^q(\rN(0,1))$上の分離族であることを示せば良い.すなわち,
        任意の$\eta\in{(1,\infty]}$と,任意の$g\in L^\eta(\rN(0,1))$に対して,
        \[\forall_{k\in\N}\;\int_\R g(x)x^k\,d\rN(x)=0\quad\Rightarrow\quad g=0\;\ae\]
        を示せば良い.

        実際,任意の$x\in\R$について評価
        \[\Abs{g(x)e^{-\frac{x^2}{2}}\sum_{k=0}^n\frac{(itx)^k}{k!}}\le\abs{g(x)}e^{\abs{tx}-\frac{x^2}{2}}.\]
        より,Lebesgueの優収束定理から,
        \[\int_\R g(x)e^{itx}d\rN(x)=\lim_{n\to\infty}\sum_{k=0}^n\frac{(it)^k}{k!}\int_\R g(x)x^kd\rN(0,1)(x)=0.\]
        Fourier変換の単射性から,$g=0\;\ae$である.
    \end{enumerate}
\end{Proof}
\begin{remarks}
    しかし,$C_p^\infty(\R)$上の微分作用素の自然な定義域を得たい場合は,$L^q(\rN(0,1))$上で完備化する訳では無い.
\end{remarks}

\subsection{微分作用素の自然な定義域}

\begin{proposition}[\cite{Nourdin-Peccati12-NormalApproximation} Lemma 1.1.6]
    微分作用素$D^p:C_p^\infty(\R)\to C_p^\infty(\R)$は,任意の$q\in\cointerval{1,\infty}$と$p\in\N^+$について$L^q(\rN(0,1))$上可閉である.
\end{proposition}
\begin{Proof}
    ここでは$q>1$とする.
    \begin{description}
        \item[方針] Hilbert空間内の作用素の可閉性は,定義域内の$0$に収束する列について,像も唯一の集積点を$0$に持つことで特徴付けられる.
        したがって,
        $\{f_n\}\subset C_p^\infty(\R)$は$0$に$L^1(\rN(0,1))$-収束し,$f_n^{(p)}$もある$\eta\in L^q(\rN(0,1))$に$L^q$-収束するとして,
        $\eta=0$を示せば良い.
        \item[証明] $D,\delta$の随伴性より,任意の$g\in C^\infty_p(\R)$について,
        \begin{align*}
            \int_\R\eta(x)g(x)d\rN(x)&=\lim_{n\to\infty}\int_\R f_n^{(p)}(x)g(x)d\rN(x)\\
            &=\lim_{n\to\infty}\int_\R f_n(x)\delta^pg(x)d\rN(x).
        \end{align*}
        $\delta^pg\in C_p^\infty(\R)\subset L^{q^*}(\rN)$より,Holderの不等式から,
        \[\int_\R\eta(x)g(x)d\rN(x)=0.\]
        $C_p^\infty(\R)$の$L^q(\rN)$上の稠密性から,$\eta=0\;\ae$
    \end{description}
\end{Proof}

\begin{definition}[Sobolev-Malliavin空間]
    $q\in\cointerval{1,\infty},p\ge1$について,
    \begin{enumerate}
        \item ノルム
        \[\norm{f}{D^{p,q}}:=\paren{\sum_{k=0}^p\int_\R\abs{f^{(k)}(x)}^qd\rN(x)}^{1/q}.\]
        を考え,これに関する$C_p^\infty(\R)$の閉包を$D^{p,q}$で表す.
        \item $f\in D^{p,q}$の微分は$L^q(\rN)$-極限$f^{(j)}:=\lim_{n\to\infty}f_n^{(j)}$で定める.
        \item $D^{\infty,q}:=\cap_{q\ge1}D^{p,q}$とする.
        \item 有界線型作用素$D^p:D^{p,q}\to L^q(\rN)$を\textbf{$L^q$-ノルムに関する微分作用素}という.
    \end{enumerate}
\end{definition}
\begin{remarks}
    この空間は,$L^q(\rN)$の元であって$L^q$-の範囲で$p$階超関数微分可能な関数の全体のなすBanach空間に一致する.
    $\{f_n\}\subset C_p^\infty(\R)$について,$f_n\to f\;\In D^{p,q}$とは次の2条件に同値:
    \begin{enumerate}
        \item $f_n\to f\;\In L^q(\rN)$.
        \item $j\in[p]$階導関数$f_n^{(j)}$も$L^q(\rN)$のCauchy列である.
    \end{enumerate}
\end{remarks}

\begin{lemma}[相互関係]\mbox{}
    \begin{enumerate}
        \item 微分の階数が深いほどノルムが大きくなるように作ったから,任意の$m\in\N,\ep\ge0$について,$D^{p,q+\ep}\subset D^{p+m,q}$が成り立つ.
        \item $D^p:D^{p,q}\to L^q(\rN)$と$D^p:D^{p,q'}\to L^{q'}(\rN)$は$D^{p,q}\cap D^{p,q'}$上で一致する.
    \end{enumerate}
\end{lemma}

\subsection{Ornstein-Uhlenbeck作用素の定義}

\begin{definition}
    \textbf{Ornstein-Uhlenbeck半群}$\{T_t\}_{t\in\R_+}\subset B(L^q(\rN))\;(q\ge1)$とは,
    \[P_tf(x):=\int_\R f(e^{-t}x+\sqrt{1-e^{-2t}}y)d\rN(y),\qquad x\in\R,f\in C_p^\infty(\R).\]
    の延長として定まる有界線型作用素の族をいう.
\end{definition}

\begin{lemma}\mbox{}
    \begin{enumerate}
        \item $P_0f=f$.
        \item $P_\infty f:=\lim_{t\to\infty}P_tf\equiv\int_\R f(y)d\rN(y)$.
    \end{enumerate}
\end{lemma}

\begin{proposition}[Ornstein-Uhlenbeck作用素の延長とノルム減少性]
    任意の$t\in\R_+$と$q\in\cointerval{1,\infty}$について,$P_t$は$L^q(\rN)$上に延長し,ノルム減少的な有界線型作用素を定める.
\end{proposition}
\begin{Proof}
    Jensenの不等式より,
    \begin{align*}
        \int_\R\abs{P_tf(x)}^qd\rN(x)&=\int_\R\Abs{\int_\R f(e^{-t}x+\sqrt{1-e^{-2t}}y)d\rN(x)}^qd\rN(x)\\
        &\le\int_{\R^2}\abs{f(e^{-t}x+\sqrt{1-e^{-2t}}y)}^qd\rN(x)\rN(y)\\
        &=\int_\R\abs{f(x)}^qd\rN(x).
    \end{align*}
    最後の等式は,ベクトル
    \[\vctr{e^{-t}}{\sqrt{1-e^{-2t}}}\]
    が長さ$1$であるため,$X\sim\rN_2(\b{1}_2,I_2)$ならば$a^\top X\sim\rN(a^\top\b{1}_2,a^\top I_2 a)$であるためである.
\end{Proof}

\subsection{Ornstein-Uhlenbeck作用素の性質}

\begin{proposition}[Ornstein-Uhlenbeck作用素の半群性]
    任意の$s,t\in\R_+$について,$P_tP_s=P_{t+s}\in B(L^1(\rN))$.
\end{proposition}
\begin{Proof}
    任意の$f\in L^1(\rN)$を取る.このとき,
    \begin{align*}
        P_tP_sf(x)&=\int_{\R^2}f(e^{-s-t}x+e^{-s}\sqrt{1-e^{-2t}}y+\sqrt{1-e^{-2s}}z)d\rN(y)d\rN(z)\\
        &=\int_\R f(e^{-s-t}x+\sqrt{1-e^{-2s-2t}}y)d\rN(y)=P_{t+s}f(x).
    \end{align*}
    ただし,最後の等式は$N,N'\sim\rN(0,1)$を独立としたとき,
    \[e^{-s}\sqrt{1-e^{-2t}}N+\sqrt{1-e^{-2s}}N'\overset{d}{=}\sqrt{1-e^{-2(s+t)}}N.\]
    による.
\end{Proof}

\begin{proposition}
    任意の$t\in\R_+$について,
    \begin{enumerate}
        \item $P_t:D^{1,2}\to D^{1,2}$と定まる.
        \item $DP_t=e^{-t}P_tD$.
    \end{enumerate}
\end{proposition}
\begin{Proof}\mbox{}
    \begin{enumerate}
        \item $P_t$の$L^2$-ノルム減少性による.
        \item $f\in C_p^\infty(\R)$について,
        \begin{align*}
            DP_tf(x)&=e^{-t}\int_\R f'(e^{-t}x+\sqrt{1-e^{-2t}}y)d\rN(y)\\
            &=e^{-t}P_tf'(x)=e^{-t}P_tDf(x).
        \end{align*}
        この結論は$D^{1,2}$上に延長する.
    \end{enumerate}
\end{Proof}

\subsection{Ornstein-Uhlembeck生成作用素}

\begin{definition}
    $\{P_t\}\subset L^2(\rN)$とみて,
    \[L:=\dd{}{t}\biggr|_{t=0}P_t.\]
    と表す.定義域は,
    \[\Dom L=\Brace{f\in L^2(\rN)\;\middle|\;\lim_{h\to0}\frac{P_hf-f}{h}\text{は}L^2(\rN)\text{で収束する}}.\]
\end{definition}

\begin{proposition}
    任意の$f\in C^\infty_p(\R)$に対して,$Lf=-\delta Df$.
\end{proposition}
\begin{Proof}
    一般に,任意の$t\in\R_+$に対して,
    \begin{align*}
        \dd{}{t}P_t&=\lim_{h\to0}\frac{P_{t+h}-P_t}{h}=\lim_{h\to0}P_t\frac{P_h-\id_{L^2(\rN)}}{h}=P_tL.
    \end{align*}
    同様にして,$\dd{}{t}P_t=LP_t$でもある.
    一方で,部分積分により,
    \begin{align*}
        \dd{}{t}P_tf(x)&=-xe^{-t}\int_\R f'(e^{-t}x+\sqrt{1-e^{-2t}}y)d\rN(y)+\frac{e^{-2t}}{\sqrt{1-e^{-2t}}}\int_\R f'(e^{-t}x+\sqrt{1-e^{-2t}}y)y\rN(y)\\
        &=-xe^{-t}\int_\R f'(e^{-t}x+\sqrt{1-e^{-2t}}y)d\rN(y)+e^{-2t}\int_\R f''(e^{-t}x+\sqrt{1-e^{-2t}}y)d\rN(y).
    \end{align*}
    でもある.
    $t=0$について議論を特殊化すると,
    \[Lf(x)=-xf'(x)+f''(x).\]
\end{Proof}

\begin{proposition}[Heisenbergの関係]
    任意の$f\in C_p^\infty(\R)$について,
    \begin{enumerate}
        \item $(D\delta-\delta D)f=f$.
        \item 任意の$p\in\N^+$について,$(D\delta^p-\delta^pD)f=p\delta^{p-1}f$.
    \end{enumerate}
\end{proposition}

\subsection{生成作用素の応用}

\begin{proposition}[Poincare inequality]
    $N\sim\rN(0,1),f\in D^{1,2}$のとき,
    \[\Var[f(N)]\le E[f'^2(N)].\]
\end{proposition}

\section{Malliavin微分}

\begin{tcolorbox}[colframe=ForestGreen, colback=ForestGreen!10!white,breakable,colbacktitle=ForestGreen!40!white,coltitle=black,fonttitle=\bfseries\sffamily,
title=]
    Malliavin微分$D:L^2(\Om)\to L^2(\Om;H)$は,確率変数を取って,$L^2$な見本道を持つ確率過程$\Om\times\R_+\to\R$を返す.
\end{tcolorbox}

\subsection{定義域について}

\begin{notation}
    $B$を$(\Om,\F,P)$上のBrown運動で,$\F$をその自然な情報系$\F=\sigma[B]$とする.
    \begin{enumerate}
        \item $\R^n$の一般化として,
        $H:=L^2(\R_+)$を決定論的過程のなすHilbert空間とする.内積を$(-|-)$で表す.
        \item 決定論的過程$h\in L^2(\R_+)$の確率積分$H\to L^2(\Om)$を
        \[B(h):=\int^\infty_0h(t)dB_t,\quad h\in H.\]
        と表す.
        \item 滑らかな柱状確率変数の全体がなす$L^2(\Om)$の部分集合$\S$を
        \[\S:=\Brace{f(B(h_1),\cdots,B(h_n))\in L^2(\Om)\mid f\in C^\infty_p(\R^n),h_i\in H}.\]
        で定める.
        \item $L^2(\Om;H)$の部分集合$\S_H$を
        \[\S_H:=\Brace{u_t=\sum_{j\in[n]}F_jh_j(t)\in L^2(\Om;H)\;\middle|\;F_j\in\S,h_j\in H}.\]
        で定める.
    \end{enumerate}
\end{notation}
\begin{remarks}
    確率過程$\{X_t\}\subset L^2(\P)$を積分して$L^2(\Om)$の元を得るのが(大域的)確率積分であった.
    一方で,
    Brown運動の情報系$\F$について可測な確率変数$F\in L(\Om)$に対して(すなわち$F=f(N)\;(N\sim\rN(0,1))$と想定出来る),
    その微分$D_tF\in L(\Om,H)$として,
    発展的可測?な
    確率過程$\Om\times\R_+\to\R$を得たい.
\end{remarks}

\begin{lemma}\mbox{}
    \begin{enumerate}
        \item $\S$の元の2通りの表し方に対して,$DF$はwell-definedである.
        \item $\S$は$L^q(\Om)\;(q\ge1)$上稠密である.
    \end{enumerate}
\end{lemma}
\begin{Proof}\mbox{}
    \begin{enumerate}
        \item a
        \item $\S$は$\{H_n(X(h))\}_{h\in\partial B,n\in\N}$を含むため,定理\ref{thm-Wiener-Ito-chaotic-decomposition}から従う.
        そもそも多項式$p\in\R[X_1,\cdots,X_n]$に対して$p(B(h_1),\cdots,B(h_n))\;(h_1,\cdots,h_n\in H)$という形の確率変数は$L^q(\Om)\;(1\le q<\infty)$上稠密になる.
        1次元との対応\ref{prop-denseness-of-monomial-in-Gaussian-space}がある.
    \end{enumerate}
\end{Proof}

\subsection{多項式増大関数の空間上での定義と例}

\begin{tcolorbox}[colframe=ForestGreen, colback=ForestGreen!10!white,breakable,colbacktitle=ForestGreen!40!white,coltitle=black,fonttitle=\bfseries\sffamily,
title=]
    Malliavin微分$D:L^2(\Om)\to L^2(\Om;H)$を,まず稠密部分集合$\S\subset L^2(\Om)$上で定義する.
\end{tcolorbox}

\begin{definition}[Malliavin derivative, divergence / Skorokhod integral]\mbox{}
    \begin{enumerate}
        \item $L^2(\Om)$-確率変数
        \[F=f(B(h_1),\cdots,B(h_n))\quad\in\S\]
        について,その\textbf{微分}$DF:\Om\to H$を,
        \[D_tF:=\sum_{i\in[n]}\pp{f}{x_i}(B(h_1),\cdots,B(h_n))h_i(t).\]
        で定める.
        \item $L^2$な見本道を持つ確率過程
        \[u_t=\sum_{j=1}^nF_jh_j(t)\quad\in\S_H.\]
        について,その\textbf{微分}$Du:\Om\to H\otimes H\simeq B^2(H)$を,
        \[(Du)_t:=\sum_{j=1}^n(D_tF_j)\otimes h_j(t)=\sum_{i,j=1}^n\pp{f_j}{x_i}(h_i\otimes h_j)(t).\]
        で定める.
        \item $L^2$な見本道を持つ確率過程
        \[u_t=\sum_{j=1}^nF_jh_j(t)\quad\in\S_H.\]
        について,その\textbf{発散}$\delta(u):\Om\to\R$を,
        \[\delta(u):=\sum_{j\in[n]}F_jB(h_j)-\sum_{j\in[n]}(DF_j|h_j).\]
        で定める.
    \end{enumerate}
\end{definition}
\begin{remarks}\mbox{}
    \begin{enumerate}
        \item $F$は決定的過程$h_i$の確率積分$B(h_i)$の汎関数である.その微分は,「汎関数の微分の汎関数」を係数とした決定論的過程の和である.
        \item $L^2(\Om)$-確率変数の微分の一般化として,これを係数とする確率過程$u$の微分は,係数たる$L^2(\Om)$-確率変数の微分を新たに係数とした過程とする.
        \item $\delta$はこのような形の過程に対して,$h_j$の代わりに$B(h_j)$を当てた過程の和から,$DF_j$の$h_j$成分$(DF_j|h_j)$を減じたものとする.
    \end{enumerate}
\end{remarks}

\begin{example}\mbox{}
    \begin{enumerate}
        \item Wiener積分の逆:$\forall_{h\in H}\;D(B(h))=h$.
        決定論的過程$h\in H$に対しては,Wiener積分の逆になっている.
        実際,$f(x)=x,n=1$であるから,$D_t(B(h))=h(t)$.
        \item 正規確率変数の微分は区間の定義関数:
        $\forall_{t_1\in\R_+}\;D(B_{t_1})=1_{[0,t_1]}$.このとき,$h=1_{[0,t_1]}$かつ$f(x)=x,n=1$であるから,$D_t(B_{t_1})=h(t)=1_{[0,t_1]}$.
        \item Wiener積分としての発散:$\forall_{h\in H}\;\delta(h)=B(h)$.このとき,$F=1,n=1$であるから,
        \[\delta(h)=B(h)-(D1|h)=B(h).\]
    \end{enumerate}
\end{example}

\subsection{定義の確認}

\begin{lemma}\mbox{}
    \begin{enumerate}
        \item $\S\subset L^2(\Om)$は線型部分空間である.
        \item たしかに$DF\in L^2(\Om;H)$である.
        \item 対応$D:S\to L^2(\Om;H)$は線型である.
        \item 対応$D:S\to L^2(\Om;H)$は非有界である.
    \end{enumerate}
\end{lemma}
\begin{Proof}\mbox{}
    \begin{enumerate}
        \item まず,$C_p^\infty(\R)$は線型空間をなす.すると,任意の$F,G\in\S$について,
        \[F=f(B(h_1),\cdots,B(h_n)),\quad G=g(B(k_1),\cdots,B(k_m))\]
        に対して,$F+G=(f+g)(B(h_1),\cdots,B(h_n),B(k_1),\cdots,B(k_m)),aF=(af)(B(h_1),\cdots,B(h_n))$と表せるため,$\S$の元であることが解る.
        \item \[D_tF=\sum_{i\in[n]}\pp{f}{x_i}(B(h_1),\cdots,B(h_n))h_i(t)\]
        は,$h_i\in H$の元の線型結合であることに注意すれば,
        \[E\Square{\paren{\sum_{i\in[n]}f_{x_i}(B(h_1),\cdots,B(h_n))h_i(t)}^2}\]
        が有限であることは,Burkholderの不等式より,$B(h_i)\in L^2(\Om)$が任意階数の積率を持つことによる.
        \item スカラー倍について可換であることはよいだろう.任意の$F_1,F_2\in\S$に対して,
        \[F_1=f_1(B(h_1),\cdots,B(h_n)),\quad F_2=f_2(B(h_1),\cdots,B(h_n))\]
        と表せると仮定しても一般性を失わない.
    \end{enumerate}
\end{Proof}

\begin{lemma}\mbox{}
    \begin{enumerate}
        \item (Leibnitz則) 任意の$F_1,F_2\in\S$について,
        \[D(F_1F_2)=(DF_1)F_2+F_1(DF_2).\]
    \end{enumerate}
\end{lemma}
\begin{Proof}\mbox{}
    \begin{enumerate}
        \item ある$h_1,\cdots,h_n\in H$について,
        \[F_1=f_1(B(h_1),\cdots,B(h_n)),\quad F_2=f_2(B(h_1),\cdots,B(h_n)).\]
        と表せると仮定して良い.このとき,
        \begin{align*}
            D(F_1F_2)&=\sum_{i\in[n]}\pp{(f_1f_2)}{x_i}(B(h_1),\cdots,B(h_n))h_i\\
            &=f_2(B(h_1),\cdots,B(h_n))\sum_{i\in[n]}\pp{f_1}{x_1}(B(h_1),\cdots,B(h_n))h_i+f_1(B(h_1),\cdots,B(h_n))\sum_{i\in[n]}\pp{f_2}{x_i}(B(h_1),\cdots,B(h_n))h_i\\
            &=F_2(DF_1)+F_1(DF_2).
        \end{align*}
    \end{enumerate}
\end{Proof}

\subsection{随伴関係}

\begin{tcolorbox}[colframe=ForestGreen, colback=ForestGreen!10!white,breakable,colbacktitle=ForestGreen!40!white,coltitle=black,fonttitle=\bfseries\sffamily,
title=]
    この随伴関係は,Malliavin微分に対する部分積分公式と歴史的には呼ばれる.
\end{tcolorbox}

\begin{proposition}
    任意の$F\in\S$と$u\in\S_H$について,
    \[E[F\delta(u)]=E[(DF|u)].\]
\end{proposition}
\begin{Proof}\mbox{}
    \begin{enumerate}[{Step}1]
        \item 任意の$F\in\S,u\in\S_H$を取る.
        すると,ある$n\in\N$とある$f,g_j\in C_p^\infty(\R^n)$と,
        $H$の正規直交系$h_1,\cdots,h_n$について
        \[F:=f(B(h_1),\cdots,B(h_n)),\quad u=\sum_{j=1}^ng_j(B(h_1),\cdots,B(h_n))h_j\]
        と表せると仮定して良い($F$または$u$のどちらかでは使わない$h_i$があって良い).
        \item すると,有限次元の場合に帰着する.
        まず,確率積分のユニタリ性\ref{prop-orthonormality-of-stochastic-integral}
        \[E[B(h_i)B(h_j)]=(h_i|h_j)\]
        より,$B(h_1),\cdots,B(h_n)$は互いに独立で,$(B(h_1),\cdots,B(h_n))\sim\rN_n(0,I_n)$.
        いま,$DF,\delta(u)$を計算すると
        \begin{align*}
            D_tF&=\sum_{i\in[n]}\pp{f}{x_i}(B(h_1),\cdots,B(h_n))h_i(t)\\
            \delta(u)&=\sum_{j\in[n]}g_j(B(h_1),\cdots,B(h_n))B(h_j)-\sum_{j\in[n]}(Dg_j(B(h_1),\cdots,B(h_n))|h_j)\\
            &=\sum_{j\in[n]}g_j(B(h_1),\cdots,B(h_n))B(h_j)-\sum_{j\in[n]}\pp{g_j}{x_j}(B(h_1),\cdots,B(h_n)).
        \end{align*}
        となるから,有限次元上での随伴関係\ref{prop-adjoint-of-Malliavin-Derivative-on-finite-dimensional-Gaussian-space}より,
        \begin{align*}
            E[(DF|u)]&=E\Square{\sum_{i\in[n]}\pp{f}{x_i}(B(h_1),\cdots,B(h_n))g_i(B(h_1),\cdots,B(h_n))}\\
            &=\sum_{i\in[n]}E'\Square{\pp{F}{x_i}g_i}\\
            &=\sum_{i\in[n]}E'\Square{F\delta(u)}\\
            &=E\Square{f(B(h_1,\cdots,h_n))\sum_{j\in[n]}\paren{g_j(B(h_1),\cdots,B(h_n))B(h_j)-\pp{g_j}{x_j}(B(h_1),\cdots,B(h_n))}}\\
            &=E[F\delta(u)].
        \end{align*}
        を得る.ただし,$E'$は確率空間$(\R^n,\B(\R^n),\rN_n(0,I_n))$上での期待値とした.
    \end{enumerate}
\end{Proof}

\subsection{微分の性質}

\begin{notation}[方向微分]
    以降,微分$F$をして得る過程の$h\in H$成分を
    $D_hF:=(DF|h)\in L^2(\Om)$と表す.
    $D_h$を作用素と見ると,線型作用素の合成であるからやはり線型であることに注意.
    また,$u\in\S_H$については,
    \begin{align*}
        D_hu&:=\sum_{i=1}^n(DF_i|h)h_i=\sum_{j=1}^n\paren{\sum_{i=1}^n\pp{f_i}{x_j}h_j\middle|h}h_i\\
        &=\sum_{i,j=1}^n\pp{f_i}{x_j}(h_j|h)h_i.
    \end{align*}
    となる.これは,Hilbert-Schmidt作用素$Du\in B^2(H,H)\simeq H\otimes H$の$h$での値とも捉えられ,その意味では$Du\cdot h$と表す\cite{Kunze13-Malliavin} p.16.
\end{notation}

\begin{proposition}
    $u,v\in\S_H,F\in\S,h\in H$と,正規直交基底$\{e_i\}_{i\in\N^+}\subset H$について,
    \begin{enumerate}
        \item 発散と方向微分についてのHeisenbergの交換関係:$D_h(\delta(u))=\delta(D_hu)+(h|u)$.
        \item 発散同士の内積の表示:$E[\delta(u)\delta(v)]=E[(u|v)]+E\Square{\sum_{i,j\in\N^+}D_{e_i}(u|e_j)D_{e_j}(v|e_i)}$.
        \item 発散の積の微分則:$\delta(Fu)=F\delta(u)-(DF|u)$.
    \end{enumerate}
\end{proposition}
\begin{Proof}\mbox{}
    \begin{enumerate}
        \item 任意に
        \[u=\sum_{j\in[n]}F_jh_j\quad\in\S_H\]
        を取る.するとこのとき,
        \begin{align*}
            \delta(D_hu)&=\sum_{i=1}^n(DF_i|h)B(h_i)-\sum_{i=1}^n\Paren{D(DF_i|h)\bigg|h_i}\\
            &=\sum_{i=1}^n\Paren{(D_hF_i)B(h_i)-D_{h_i}(D_hF_i)}.
        \end{align*}

        一方で,
        \begin{align*}
            D_h(\delta(u))&=D_h\paren{\sum_{j\in[n]}F_jB(h_j)-\sum_{j\in[n]}(DF_j|h_j)}\\
            &=\sum_{j\in[n]}D_h(F_jB(h_j))-\sum_{j\in[n]}D_h(DF_j|h_j)\\
            &=\sum_{j\in[n]}\Paren{(D_hF_j) B(h_j)+F_j (h_j|h)}-\sum_{j\in[n]}D_h(DF_j|h_j)\\
            &=\sum_{j\in[n]}\paren{(D_hF_j)B(h_j)-D_h(D_{h_j}F_j)}+(h|u).
        \end{align*}
        であるから,あとは$D_{h_i}(D_hF_i)=D_h(D_{h_i}F_i)$を確認すればよいが,これは表示
        \[D_{h_i}D_hF_i=\sum_{j,k=1}^n\pp{^2f_i}{x_k\partial x_j}(h_j|h)(h_k|h).\]
        より明らかに可換である.
        \item 随伴関係より,
        \begin{align*}
            E[\delta(u)\delta(v)]&=E[(D\delta(u)|v)]\\
            &=E\Square{\paren{D\delta(u)\middle|\sum_{i=1}^\infty(v|e_i)e_i}}=E\Square{\sum_{i=1}^\infty(v|e_i)(D\delta(u)|e_i)}.
        \end{align*}
        (1)から$D_{e_i}(\delta(u))=\delta(D_{e_i}u)+(e_i|u)$であるから,
        \begin{align*}
            E[\delta(u)\delta(v)]&=E\Square{\sum_{i=1}^\infty(v|e_i)(e_i|u)+\sum_{i=1}^\infty(v|e_i)\delta(D_{e_i}u)}\\
            &=E[(v|u)]+E\Square{\sum_{i=1}^n\paren{D(v|e_i)\middle|D_{e_i}u}}\\
            &=E[(v|u)]+E\Square{\sum_{i=1}^n\paren{D(v|e_i)\middle|\sum_{j=1}^\infty(u|e_j)e_j}}\\
            &=E[(v|u)]+E\Square{\sum_{i,j=1}^n\paren{D(v|e_i)|De_i(u|e_j)e_j}}=E[(v|u)]+E\Square{\sum_{i,j=1}^n\paren{D_{e_j}(v|e_i)|D_{e_i}(u|e_j)}}.
        \end{align*}
        \item 任意の$G\in\S$について,
        \begin{align*}
            E[\delta(Fu)G]&=E[(Fu|DG)]=E[(u|F(DG))]\\
            &=E[(u|D(FG)-G(DF))]=E[\delta(u)FG]-E[(u|G(DF))]\\
            &=E[G(\delta(u)F-(u|DF))].
        \end{align*}
        が成り立つ.$\S\subset L^2(\Om)$の稠密性より,一般の関係
        \[\delta(Fu)=\delta(u)F-(u|DF)\]
        を得る.
    \end{enumerate}
\end{Proof}
\begin{remarks}[Hilbert-Schmidt作用素としての同一視]
    \[F_j=f_j(B(h_1),\cdots,B(h_n)),\quad u=\sum_{i=1}^nF_ih_i\]
    とすると,
    \[Du=\sum_{i,j=1}^n\pp{f_j}{x_i}h_i\otimes h_j\quad\leftrightarrow\quad \sum_{i,j=1}^n\pp{f_j}{x_i}(-|h_i)h_j.\]
    と同一視出来て,
    \[D_hu=Du\cdot h=\sum_{i,j1}^n\pp{f_j}{x_i}(h_i|h)h_j,\quad (Du)^*\cdot h=\sum_{i,j=1}^n\pp{f_j}{x_i}(h_j|h)h_i.\]
    であるが,
    \[D(u|h)=\sum_{i,j=1}^n\pp{f_j}{x_i}(h_j|h)h_i=(Du)^*\cdot h\]
    となっている.
    これと作用素の跡の定義
    \[\Tr(T):=\sum_{i=1}^\infty(Te_i|e_i),\qquad T\in B^1(H)\]
    用いると,(2)の結果は
    \[E[\delta(u)\delta(v)]=E[(u|v)]+E[\tr(DuDv)].\]
    とまとめられる.
\end{remarks}
\begin{Proof}
    \begin{align*}
        E[\delta(u)\delta(v)]&=E[(u|v)]+E\Square{\sum_{i=1}^n\Paren{D(v|e_i)\bigg|De_iu}}\\
        &=E[(u|v)]+E[((Du)^*\cdot e_i|Dv\cdot e_i)]=E[(u|v)]+E[\tr(Du|Dv)].
    \end{align*}
    を得る.
\end{Proof}

\subsection{Hilbert空間のテンソル積}

\begin{definition}[tensor product of Hilbert spaces]
    $H,V$をHilbert空間,$(h_\al)_{\al\in A},(v_\beta)_{\beta\in B}$をそれぞれの正規直交基底とする.
    \begin{enumerate}
        \item \textbf{テンソル積}$H\otimes V$とは,
        \[\Brace{\sum_{\al\in A,\beta\in B}a_{\al,\beta}h_\al\otimes v_\beta\;\middle|\;a_{\al,\beta}\in l^2(A\times B)}.\]
        の全体からなる集合に,$l^2(A\times B)$と同型になるような内積を入れたHilbert空間である.
        よってこのとき,$(h_\al\otimes v_\beta)_{(\al,\beta)\in A\times B}$は正規直交基底である.
        \item \textbf{普遍双線型写像}$\rho:H\times V\to H\otimes V$を,
        \[\rho\paren{\sum_{\al\in A}a_\al h_\al,\sum_{\beta\in B}b_\beta v_\beta}:=\sum_{(\al,\beta)\in A\times B}a_\al b_\beta h_\al\otimes v_\beta.\]
        で定める.$(a_\al)\in l^2(A),(b_\beta)\in l^2(B)$のとき,$(a_\al b_\beta)\in l^2(A\times B)$が成り立つので,右辺はwell-definedである.
        これが連続な双線型作用素であることは,
        \[\Norm{\sum a_\al b_\beta h_\al\otimes v_\beta}^2_{H\otimes V}=\sum a^2_\al b^2_\beta=\sum_{a\in A}a_\al^2\sum_{b\in B}b_\beta^2=\Norm{\sum a_\al h_\al}^2_H\Norm{\sum b_\beta v_\beta}^2_V.\]
        による.
    \end{enumerate}
\end{definition}

\begin{proposition}[universality of tensor product \cite{Kunze13-Malliavin} Prop1.2.10]
    $H,V,U$をHilbert空間とし,$\eta:H\times V\to U$を連続な双線型作用素とする.
    このとき,ある有界線型作用素$T_\eta:H\otimes V\to U$が存在して,次が可換になる:
    \[\xymatrix{
        H\times V\ar[r]^-\eta\ar[d]_-{\rho}&U\\
        H\otimes V\ar@{-->}[ur]_-{T_\eta}
    }\]
\end{proposition}

\begin{corollary}[$L^2$-空間のテンソル積]
    $(\Om_i,\F_i,\mu_i)\;(i=1,2)$を$\sigma$-有限な測度空間とする.
    このとき,
    \[L^2(\Om_1,\F_1,\mu_1)\otimes L^2(\Om_2,\F_2,\mu_2)\simeq_\Hilb L^2(\Om_1\times\Om_2,\F_1\otimes\F_2,\mu_1\otimes\mu_2).\]
\end{corollary}

\begin{definition}
    $(\Om,\F,\mu)$を非原子的な測度空間,
    $H:=L^2(\Om,\F,\mu)$をHilbert空間とする.
    \begin{enumerate}
        \item $H^{\otimes k}\simeq_\Hilb L^2(\Om^k,\F^{\otimes k},\mu^{\otimes k})$を同一視する.
        \item $H^{\odot k}$で,$L^2(\Om^k,\F^{\otimes k},\mu^{\otimes k})$のうち殆ど至る所対称であるもののなす部分空間と同一視する:
        \[f(a_1,\cdots,a_k)\overset{\ae}{=}\frac{1}{k!}\sum_{\sigma\in S_k}f(a_{\sigma(1)},\cdots,a_{\sigma(k)}).\]
    \end{enumerate}
\end{definition}

\subsection{Hilbert-Schmidt作用素}

\begin{tcolorbox}[colframe=ForestGreen, colback=ForestGreen!10!white,breakable,colbacktitle=ForestGreen!40!white,coltitle=black,fonttitle=\bfseries\sffamily,
title=]
    テンソル積の埋め込み$H\otimes V\mono B(H,V)$の像をHilbert-Schmidt作用素という.
\end{tcolorbox}

\begin{definition}\mbox{}
    \begin{enumerate}
        \item 次の双線型作用素$\eta$を$h\otimes v$と表す:
        \[\xymatrix@R-2pc{
            \eta:H\times V\ar[r]&B(H,V)\\
            \rotatebox[origin=c]{90}{$\in$}&\rotatebox[origin=c]{90}{$\in$}\\
            (h,v)\ar@{|->}[r]&(-|h)v.
        }\]
        すると,$\Im(h\otimes v)=\R v$を満たす.
        \item $H,V$を可分Hilbert空間とする.$T\in B(H,V)$が\textbf{Hilbert-Schmidt作用素}であるとは,ある正規直交基底$\{e_n\}\subset H$について,
        \[\sum_{n\in\N}\norm{Te_n}^2_V<\infty.\]
        が成り立つことをいう.この値は正規直交基底$\{e_n\}\subset H$の選び方には依らず,この値をノルム$\norm{T}_{\tr}^2$とする.
        \item Hilbert-Schmidt作用素の全体は線型空間をなす.これを
        \[B^2(H,V):=\Brace{T\in B_0(H)\mid\norm{T}_\Tr<\infty}.\]
        で表す.
    \end{enumerate}
\end{definition}

\begin{theorem}[Hilbert-Schmidt作用素]
    $H,V$を可分Hilbert空間とする.このとき,
    \[H\otimes V\simeq_\Hilb B^2(H,V).\]
    また,内積は次が定める:
    \[(T|S)_\Tr:=\tr(S^*T)=\sum_{n,m\in\N}a_{nm}b_{nm},\qquad T=\sum_{n,m\in\N}a_{nm}h_n\otimes v_m,S=\sum_{n,m\in\N}b_{nm}h_n\otimes v_m.\]
\end{theorem}
\begin{remarks}
    Hilbert-Schmidt作用素の空間$B^2(H,V)$に対して,$S^*T\in B^1(H)$は跡類作用素になるために,これを利用して内積を定めている.
\end{remarks}

\section{Sobolev空間}

\subsection{微分の可閉性}

\begin{tcolorbox}[colframe=ForestGreen, colback=ForestGreen!10!white,breakable,colbacktitle=ForestGreen!40!white,coltitle=black,fonttitle=\bfseries\sffamily,
title=]
    次の命題によって,$D:\S\to L^2(\Om;H)$を延長する.
\end{tcolorbox}

\begin{theorem}[Malliavin微分の可閉性]
    任意の$p\ge1$について,
    非有界作用素
    $D:L^p(\Om)\to L^p(\Om;H)$は可閉である.
\end{theorem}
\begin{Proof}\mbox{}
    \begin{description}
        \item[方針] 作用素$D:L^p(\Om)\to L^p(\Om;H)$が可閉であることと,$L^p(\Om)$の任意の$0$への収束列$\{F_N\}_{N\in\N}\subset L^p(\Om)$に対して,$\{DF_N\}\subset L^p(\Om;H)$がただ一つの集積点$0$を持つことが同値である\ref{lemma-characterization-of-closability}.
        したがって,$DF_N\to\eta\;\In L^p(\Om;H)$を仮定して,$\eta=0$であることを示せばよい.
        \item[証明] $L^p(\Om;H)$の稠密部分集合\ref{lemma-dense-subset-of-S_H}
        \[(\S_H)_b:=\Brace{u=\sum_{j=1}^NG_jh_j\in\S_H\;\middle|\;G_jB(h_j)\in L^\infty(\Om),DG_j\in L^\infty(\Om;H)}.\]
        の任意の元$u$について,随伴関係より,
        \[E[(\eta|u)]=\lim_{N\to\infty}E[(DF_N|u)]=\lim_{N\to\infty}E[F_N\delta(u)]=0.\]
        最後の収束は,Holderの不等式
        \[\norm{F_N\delta(u)}_{L^1(\Om)}\le\norm{F_N}_{L^p(\Om)}\norm{\delta(u)}_{L^{p^*}(\Om)}\le\norm{F_N}_{L^p(\Om)}\norm{\delta(u)}_{L^{\infty}(\Om)}.\]
        による.
    \end{description}
\end{Proof}

\begin{lemma}[グラフの特徴付け]\label{lemma-characterization-of-graph}
    部分空間$\cG\subset X\otimes Y$について,
    \begin{enumerate}
        \item $\cG$はある線型作用素のグラフである.
        \item 任意の$(0,y)\in\cG$について,$y=0$である.
    \end{enumerate}
    特に,$\cG$がグラフならば,任意の$\cG$の部分空間はグラフである.
\end{lemma}
\begin{Proof}\cite{Conway} Def. X.1.3.
    (2)$\Rightarrow$(1)を示せば良い.
    部分空間$\cG$に対して定義域を
    \[\D:=\Brace{x\in X\mid \exists_{y\in Y}\;(x,y)\in\cG}.\]
    と定める.このとき,任意の$x\in\D$に対して,$(x,y)\in\cG$を満たす$y\in Y$は一意的である.
    仮に$y_1,y_2$はいずれも$(x,y_1),(x,y_2)\in\G$を満たすとすると,$\cG$は線型空間であるから$(0,y_1-y_2)\in\cG$を満たす.
    すると仮定より$y_1=y_2$が従う.
    よって,これについて$T(x)=y$として対応$T:\D\to Y$を定めると,これは線型作用素を定め,グラフは$\G$である.
\end{Proof}

\begin{lemma}[可閉性の特徴付け]\label{lemma-characterization-of-closability}
    $H$内の作用素$T$について,次の条件は同値.
    \begin{enumerate}
        \item $T$は可閉である.
        \item $\D(T)$内の$0$に収束する列$(x_n)$について,$(Tx_n)$のただ一つの集積点は$0$である.
        \item ある閉作用素$S$が存在して,$T\subset S$を満たす.
    \end{enumerate}
\end{lemma}
\begin{Proof}\mbox{}
    \begin{description}
        \item[(1)$\Leftrightarrow$(3)] (1)$\Rightarrow$(3)は明らかだから逆を示す.
        これは作用素のグラフの特徴付け\ref{lemma-characterization-of-graph}から特に,グラフの部分空間はやはりある線型作用素のグラフであることから従う.
        \item[(1)$\Rightarrow$(2)] $\oo{\Graph T}$がある作用素のグラフであるとき,任意の$(0,v)\in\oo{\Graph T}$に対して$v=0$が成り立つ.$\D(T)$内の任意の$0$に収束する列$\{x_n\}\subset\D(T)$に対して,$Tx_n$が$0$以外の集積点を持つならば,$Tx_{n_k}$が$0$以外に収束する部分列$\{x_{n_k}\}\subset\D(T)$が取れてしまうので(これは一般の第1可算な位相空間で成り立つ.距離化可能な位相線型空間は第1可算である),$\oo{\Graph T}$がグラフであることに矛盾.$\{Tx_n\}$が$0$に集積しないならば,$\oo{\Graph T}$がある線型作用素のグラフであることに矛盾する.
        \item[(2)$\Rightarrow$(1)] 逆に,(2)の条件が満たされるとき,
        任意の$(0,v)\in \oo{\Graph T}$に対して,これに収束する点列が取れるから,(2)の仮定より$v=0$が必要.
        $\oo{\Graph T}$がある作用素のグラフになることが解る.
    \end{description}
\end{Proof}

\begin{lemma}\label{lemma-dense-subset-of-S_H}
    任意の$p\ge1$について,次の集合
    \[(\S_H)_b:=\Brace{u=\sum_{j=1}^NG_jh_j\in\S_H\;\middle|\;G_jB(h_j)\in L^\infty(\Om),DG_j\in L^\infty(\Om;H)}.\]
    は$L^p(\Om;H)$上稠密である.
\end{lemma}

\subsection{$L^p$上の定義域の定義}

\begin{tcolorbox}[colframe=ForestGreen, colback=ForestGreen!10!white,breakable,colbacktitle=ForestGreen!40!white,coltitle=black,fonttitle=\bfseries\sffamily,
title=]
    $\S\subset L^p(\Om)$は稠密であるが,これをそのまま延長すると有界ではない.
    $D^{1,p}\subsetneq L^p(\Om)$までの延長であれば,そのノルム($L^p(\Om)$と$L^p(\Om;H)$の間の$l^p$-直和のノルム)について有界線型作用素となる.
\end{tcolorbox}

\begin{definition}\mbox{}
    \begin{enumerate}
        \item 次のセミノルム$\norm{-}_{D^{1,p}}$に関する$\S\subset L^p(\Om)$の閉包を$D^{1,p}$で表す.
        \[\norm{F}_{D^{1,p}}:=\Paren{E[\abs{F}^p]+E[\norm{DF}_{L^2(\Om;H)}^p]}^{p/1}=\paren{E[\abs{F}^p]+E\Square{\Abs{\int^\infty_0(D_tF)^2dt}^{p/2}}}^{1/p}.\]
        この空間を\textbf{$L^p(\Om)$上の$D^1$の定義域}という\cite{Nourdin-Peccati12-NormalApproximation}.
        \item 同様にして,次のセミノルム$\norm{-}_{D^{1,p}(H)}$に関する$\S_H\subset L^p(\Om;H)$の閉包を$D^{1,p}(H)$で表す:\cite{Kunze13-Malliavin} Prop.1.3.4
        \[\norm{u}_{D^{1,p}(H)}:=\Paren{E[\norm{u}_{H}^p]+E[\norm{Du}_{H\otimes H}^p]}^{1/p}.\]
    \end{enumerate}
\end{definition}
\begin{remarks}[ノルムの定め方について]\mbox{}
    \begin{enumerate}
        \item ノルム$\norm{F}_{D^{1,p}}$とは,2つのBanach空間$L^p(\Om)$と$L^p(\Om;H)$との$l^p$-直和のノルムである.
        したがって,等長な埋め込み$D^{1,p}\mono L^p(\Om)\oplus_{l^p} L^p(\Om;H)\simeq L^p(\Om;\R\oplus_{l^p} H)$が存在し,この閉部分空間である.
        \item なお,有限個の直和について,$l^p$-直和はすべての$1\le p\le \infty$について等価なノルムを定める.
        (特に,グラフノルムと等価である).よって,次は同値:
        \begin{enumerate}
            \item $F_N\to F\;\In D^{1,p}$.
            \item $F_N\to F\;\In L^p(\Om)$かつ$DF_N\to DF\;\In L^p(\Om;H)$.
        \end{enumerate}
        \item 閉グラフ定理より,$D:D^{1,p}\to L^p(\Om;H)$は有界線型作用素である.
        \item また,$D^{1,2}$は$L^2(\Om)$と$L^2(\Om;H)$の内積の和
        \[(F|G)=E[(F|G)]+E\Square{\int^\infty_0D_tFD_tGdt}.\]
        について再びHilbert空間をなす.
        \item $D^{1,2}(H)$もHilbert空間をなすが,こちらはより複雑で,
        \[(u|v)=E[(u|v)]+E\Square{\int^\infty_0\int^\infty_0D_tu_sD_sv_tdtst}=E[(u|v)]+E[\tr(DuDv)]=E[\abs{\delta(u)}^2].\]
        となっており,$\delta:D^{1,2}(H)\to L^2(\Om)$がHilbの同型を与えるようになっている.
    \end{enumerate}
\end{remarks}
\begin{remark}[包含関係について]
    $l^p$-直和と定めたから,$\norm{-}_{D^{1,p}}$の大きさは$p$に関して単調減少で,
    $D^{1,p+\ep}subset D^{1,p}$となる.
\end{remark}

\begin{lemma}[Hilbert空間の有限直和 \cite{Conway} I.6.1]
    Hilbert空間$H,K$について,内積
    \[(h_1\otimes k_1|h_2\otimes k_2):=(h_1|h_2)+(k_1|k_2),\qquad h_1,h_2\in H_1,k_1,k_2\in K.\]
    について$H\oplus K$は再びHilbert空間となる.
\end{lemma}

\subsection{Malliavin微分の連鎖律}

\begin{tcolorbox}[colframe=ForestGreen, colback=ForestGreen!10!white,breakable,colbacktitle=ForestGreen!40!white,coltitle=black,fonttitle=\bfseries\sffamily,
title=]
    一般の$\varphi\in\Lip(\R)$について成り立つ.
\end{tcolorbox}

\begin{proposition}[Malliavin微分の連鎖律]
    過程$F$とこれに対する変換$\varphi$とを次のように取る.
    \begin{enumerate}[{[A}1{]}]
        \item 多項式増大関数$\varphi\in C(\R)$は可微分で,その導関数はある$\al\ge0$について$\abs{\varphi'(x)}\le C(1+\abs{x}^\al)$を満たすとする.
        \item $p\ge\al+1$について,$F\in D^{1,p}$とする.
    \end{enumerate}
    このとき,次が成り立つ:
    \begin{enumerate}
        \item $q:=\frac{p}{\al+1}(\le p)$について$\varphi(F)\in D^{1,q}$.特に$L^q(\Om)$の意味でMalliavin微分可能である.
        \item $\varphi(F)$のMalliavin微分は,
        \[D(\varphi(F))=\varphi'(F)DF,\qquad\In\; D^{1,q}(H).\]
    \end{enumerate}
\end{proposition}
\begin{Proof}\mbox{}
    \begin{enumerate}
        \item \begin{enumerate}[{Step}1]
            \item まず,$\varphi(F)\in L^q(\Om)$である.
            \begin{Proof}[【証】]\renewcommand{\qedsymbol}{$\Box$}
                $F\in D^{1,p}$より特に$F\in L^p(\Om)$,$\abs{F}^{1+\al}\in L^q(\Om)$である.
                いま,積分により$\abs{\varphi(x)}\le C'(1+\abs{x}^{\al+1})$であるが,$\abs{\varphi(F)}\le C'(1+\abs{F}^{\al+1})$の右辺は$L^q(\Om)$の元の線型結合であるから,特に$\varphi(F)\in L^q(\Om)$.
            \end{Proof}
            \item 次に,$\varphi'(F)DF\in L^q(\Om;H)$でもある.
            \begin{Proof}[【証】]\renewcommand{\qedsymbol}{$\Box$}
                まず
                \[\norm{\varphi'(F)DF}_{L^q(\Om;H)}\le C\norm{DF}_{L^q(\Om;H)}+C\norm{\abs{F}^\al DF}_{L^q(\Om;H)}.\]
                と評価できるが,第1項は$C\norm{DF}_{L^p(\Om;H)}$で抑えられ,第2項はHolderの不等式より,$1+\al$と$\frac{1+\al}{\al}$が共役指数であることに注意すれば,
                \begin{align*}
                    \norm{\abs{F}^\al DF}_{L^q(\Om;H)}^q&=E\Square{\abs{F}^{\frac{\al}{1+\al}p}\abs{DF}^{\frac{p}{1+\al}}}\\
                    &\le (E[\abs{F}^p])^{\frac{\al}{\al+1}}(E[\abs{DF}^p])^{\frac{1}{1+\al}}<\infty.
                \end{align*}
            \end{Proof}
            \item あとは,$\varphi(F)$に$D^{1,q}$の意味で収束する$\S$の列$\{\varphi_m(F_n)\}\subset\S$が存在することを示せばよい.
            これは次のStep3において示される.
        \end{enumerate}
        \item \begin{enumerate}[{Step}1]
            \item まず,主張を$\varphi\in C_p^\infty(\R)$と$F\in\S$が成り立つ場合について証明する.
            そのために,
            $F$と$\varphi$の近似列$\{F_n\}\subset\S$と$\{\varphi_m\}\subset C_p^\infty(\R)$が取れることを示す.
            \begin{Proof}[【証】]\renewcommand{\qedsymbol}{$\Box$}
                $F$に$D^{1,p}$-収束する列$\{F_n\}\subset\S$を取り,
                \[\al_1(x):=Me^{-\frac{1}{1-\abs{x}^2}}1_{\Brace{\abs{x}<1}},\qquad\int_\R\al_1dx=1.\]
                に対して定まる
                $\{\al_m(x):=m\al_1(mx)\}\subset C^\infty_c(\R)$を軟化子とする.
                これについて$\varphi_m:=\varphi*\al_m\in C^\infty_p(\R)$を満たすことを示せばよい.
                
                実際,軟化子$\al_m$の性質$\supp\al_m\subset[-1/m,1/m]$と,不等式
                $\abs{x-y}^\al\le2^\al(\abs{x}^\al\lor\abs{y}^\al)\le2^\al(\abs{x}^\al+\abs{y}^\al)$に注意すれば,評価
                \begin{align*}
                    \abs{\varphi_m(x)}&=\int_\R\al_m(y)\abs{\varphi(x-y)}dy\\
                    &\le C'\int_\R\al_m(y)dy+C'\int_\R\al_m(y)\abs{x-y}^{\al+1}dy\\
                    %&\le C'+2^{\al+1}C'\int_\R\al_m(\abs{x}^{\al+1}+\abs{y}^{\al+1})dy\\
                    &\le C'+C'2^{\al+1}\int_\R\al_m(y)\abs{y}^{\al+1}dy+C'2^{\al+1}\abs{x}^{\al+1}\\
                    &\le C'+C'2^{\al+1}\int^1_{-1}\al_1(x)\paren{\frac{\abs{x}}{m}}^{\al+1}dx+C'2^{\al+1}\abs{x}^{\al+1}\\
                    &\le \wt{C'}(1+\abs{x}^{\al+1}).
                \end{align*}
                と$m$に依らずに評価できる.特に$\varphi_m \in C_p(\R)$.
                またこの導関数も,
                \[\abs{\varphi_m'(x)}\le \wt{C}(1+\abs{x}^\al),\qquad\exists_{C_m>0}.\]
                %\[(\varphi*\al_m)^{(k)}(x)=\int_\R\al_m^{(k)}(x-y)\varphi(y)dy.\]
                と表せ,同様の議論を続けることが出来る.よって$\varphi_m\in C_p^\infty(\R)$.
            \end{Proof}
            \item 次に,近似列$\{F_n\}\subset\S$と$\{\varphi_m\}\subset C_p^\infty(\R)$の間には命題の主張
                \[D(\varphi_m(F_n))=(\varphi_m)'(F_n)DF_n\]
            が成り立つことを示す.
            \begin{Proof}[【証】]\renewcommand{\qedsymbol}{$\Box$}
                これは任意の$F=f(B(h_1),\cdots,B(h_n))\in\S$と$\varphi_m\in C^\infty_p(\R)$とについて,$\varphi_m(F)\in\S$であることから
                \[D(\varphi_m(F))=\sum_{i=1}^n\varphi_m'(F)\pp{f}{x_i}(B(h_1),\cdots,B(h_n))h_i=\varphi_m'(F)DF.\]
                より解る.
            \end{Proof}
            \item 最後に$\{\varphi_m(F_n)\}\subset\S$の$\varphi(F)$への$D^{1,q}$-収束を示す:いま,$F_n\to F\;\In D^{1,p}$すなわち$F_n\to F\;\In L^p(\Om)$かつ$DF_n\to DF\in L^p(\Om;H)$であるが,
            このとき$\varphi_m(F_n)\xrightarrow{n,m\to\infty}\varphi(F)\;\In D^{1,q}$が成り立つ.すなわち,
            \[\varphi_m(F_n)\xrightarrow{n,m\to\infty}\varphi(F)\;\In L^q(\Om),\quad\land\quad(\varphi_m)'(F_n)DF_n\xrightarrow{n,m\to\infty}\varphi'(F)DF\;\In L^q(\Om;H)\]
            である.これにより,(1)の証明と(2)の証明とが同時に完了する.
            \begin{Proof}[【証】]\renewcommand{\qedsymbol}{$\Box$}
                まず$\varphi_m(F_n)\xrightarrow{n,m\to\infty}\varphi(F)\;\In L^q(\Om)$について示す.
                \[\norm{\varphi_m(F_n)-\varphi_m(F)}_{L^q(\Om)}\le\norm{\varphi_m(F_n)-\varphi_m(F)}_{L^q(\Om)}+\norm{\varphi_m(F)-\varphi(F)}_{L^q(\Om)}.\]
                \begin{enumerate}[{第}1{項}]
                    \item 次の評価が成り立つ:
                    \begin{align*}
                        \abs{\varphi_m(F_n)-\varphi_m(F)}&=\abs{\varphi'_m(\wt{F}_n)}\abs{F_n-F}\\
                        &\le C_m'(1+\abs{\wt{F}_n}^\al)\abs{F_n-F}\\
                        &\le C_m'(1+(\abs{F}+\abs{F_n})^\al)\abs{F_n-F}.
                    \end{align*}
                    よって,Holderの不等式から,
                    \begin{align*}
                        \norm{\varphi_m(F_n)-\varphi_m(F)}_{L^q(\Om)}&\le \wt{C}\norm{F_n-F}_{L^q(\Om)}+\wt{C}\norm{(\abs{F}+\abs{F_n})^\al\abs{F_n-F}}_{L^q(\Om)}\\
                        &\le \wt{C}\norm{F_n-F}_{L^q(\Om)}+\wt{C}\Paren{E[(\abs{F}+\abs{F_n})^p]^{\frac{\al}{1+\al}}E[\abs{F_n-F}^p]^{\frac{1}{1+\al}}}^{1/q}.
                    \end{align*}
                    より,$n\to\infty$のとき$0$に収束する.
                    \item $\varphi_m\to\varphi$は広義一様収束するから,任意の$K\csub\R$について$N_1\in\N$が存在して,
                    \[\abs{\varphi_m(x)-\varphi(x)}<\ep,\qquad(x\in K,m\ge N_1).\]
                    これに基いて
                    \[\abs{\varphi_m(F)-\varphi(F)}=1_K(F)\abs{\varphi_m(F)-\varphi(F)}+1_{K^\comp}(F)\abs{\varphi_m(F)-\varphi(F)}<1_K(F)\ep+1_{K^\comp}(F)\abs{\varphi_m(F)-\varphi(F)}.\]
                    と評価出来るが,ある$\o{C}>0$が存在して
                    \[\abs{\varphi_m(F)-\varphi(F)}\le\abs{\varphi_m(F)}+\abs{\varphi(F)}\le\o{C}(1+\abs{F}^{\al+1}).\]
                    と表せることより,この項は$L^q(\Om)$-有界である.
                    よって,$E[1_{K^\comp}(F)]$
                \end{enumerate}
                次に,$(\varphi_n)'(F_n)DF_n\to\varphi'(F)DF\;\In L^q(\Om;H)$について,$\al=0$の場合は$\varphi'$が有界である場合であり,次の命題と同様に解決できる(以下の議論で$p/\al=\infty$と見ても良い).
                $\al>0$の場合は,$1+\al$と$\frac{1+\al}{\al}$が共役指数であることに注意して,
                \begin{align*}
                    &\norm{\varphi'_n(X_n)-\varphi'(X)DX}_{q}\\
                    &\le\norm{\varphi'_n(X_n)DX_n-\varphi'(X_n)DX_n}_q+\norm{\varphi'(X_n)DX_n-\varphi'(X)DX_n}_q+\norm{\varphi'(X)DX_n-\varphi'(X)DX}_q\\
                    &\le E[\abs{DX_n}^p]^{\frac{1}{1+\al}}E[\abs{\varphi'_n(X_n)-\varphi'(X_n)}^{p/\al}]^{\frac{\al}{1+\al}}\\
                    &\qquad+E[\abs{DX_n}^p]^{\frac{1}{1+\al}}E[\abs{\varphi'(X_n)-\varphi'(X)}^{p/\al}]^{\frac{\al}{1+\al}}+E[\abs{DX_n-DX}^p]^{\frac{1}{1+\al}}E[\abs{\varphi'(X)}^{\frac{p}{\al}}]^{\frac{\al}{1+\al}}
                \end{align*}
                と評価できる.いずれの項も$0$に収束するのであるが,第1項は$\varphi'_n(X_n)\to\varphi'(X_n)$が任意の$\om\in\Om$について成り立つことによる.
                第2項は$\abs{\varphi'(X_n)-\varphi'(X)}\le C''\abs{X_n-X}^\al$と評価できることによる.
                第3項は$\abs{\varphi'(X)}^{p/\al}$が有界であることによる.
            \end{Proof}
        \end{enumerate}
    \end{enumerate}
\end{Proof}

\begin{proposition}[$\al=0$の場合は議論が簡単になる \cite{Kunze13-Malliavin}]
    $\varphi\in C^1(\R)$は有界な導関数を持つとする.$X\in D^{1,p}$について,$\varphi(X)\in D^{1,p}$かつ
    \[D\varphi(X)=\varphi'(X)DX.\]
\end{proposition}
\begin{Proof}\mbox{}
    \begin{description}
        \item[$\S$と$C_b^\infty(\R)$上での証明] $X\in\S$かつ$\varphi\in C_b^\infty(\R)$の場合は,$\varphi(X)\in\S$より,簡単に示せる.
        \item[一般の場合] まず$\{X_n\}\subset\S$であって$X_n\to X\;\In D^{1,p}$を満たすもの,すなわち,$X_n\to X\;\In L^p(\Om),DX_n\to DX\;\In L^p(\Om;H)$を満たすものが存在する.
        続いて,$\{\al_n\}\subset C_c^\infty(\R)$を軟化子として,$\varphi_n:=\varphi*\al_n$とすると,$\varphi_n\in C^\infty_b(\R)$であり,$\varphi_n\to\varphi,\varphi'_n\to\varphi'$はいずれもコンパクト一様収束であり,評価$\abs{\varphi_n}\le C(1+\abs{x})$が成り立つ.
        \begin{enumerate}
            \item 導関数の広義一様収束$\varphi'_n\to\varphi'$については,$\varphi$の連続性から,任意の$\ep>0$に対して,$\delta>0$が存在して,$y\in(-\delta,\delta)\Rightarrow\abs{f(x-y)-f(x)}<\ep\;(x\in K)$.
            $\supp\al'_n\subset\supp\al_n\subset[-1/n,1/n]$が成り立つから,
            \[\varphi'_n(x)-\varphi'(x)=\int_{B(0,1/n)}\Paren{f(x-y)-f(x)}\al_n'(y)dy.\]
            の評価が任意の$x\in\R$について成り立ち,特に$x\in (-\delta,\delta)$については,$n>1/\delta$に取ることによって,
            \[\abs{\varphi'_n(x)-\varphi'(x)}\le\ep\int\al_n=\ep.\]
            の評価が得られる.これは明らかに任意階の導関数について成り立つ.
            \item 評価$\abs{\varphi_n}\le C(1+\abs{x})$については,まず$\varphi\in C_b^1(\R)$であるために,$\abs{\varphi'(x)}\le C$より,$\abs{\varphi(x)}\le C(1+\abs{x})$の評価が成り立つ.
            このとき,引き続き$\abs{\varphi*\al_n(x)}\le C'(1+\abs{x})$が成り立つ.
        \end{enumerate}
        \item[収束の議論] 
        \begin{enumerate}
            \item $\varphi_n(X_n)\to\varphi(X)\;\In L^p(\Om)$について,平均値の定理とHolderの不等式から
            \begin{align*}
                \norm{\varphi_n(X_n)-\varphi(X)}_{L^p(\Om)}&\le\norm{\varphi_n(X_n)-\varphi(X_n)}_{L^p(\Om)}+\norm{\varphi(X_n)-\varphi(X)}_{L^p(\Om)}\\
                &\le\norm{\varphi_n(X_n)-\varphi(X_n)}_{L^p(\Om)}+\norm{\varphi'(\wt{X}_n)(X_n-X)}_{L^p(\Om)}\\
                &\le\norm{\varphi_n(X_n)-\varphi(X_n)}_{L^p(\Om)}+\norm{\varphi'^p}_\infty\norm{X_n-X}_{L^p(\Om)}.
            \end{align*}
            右辺第2項は明らかに$0$に収束する.右辺第1項も任意の$\om\in\Om$について$\varphi_n(X_n(\om))\to\varphi(X_n(\om))$であるから,Lebesgueの優収束定理より,$0$に収束する.
            \item $\varphi_n'(X_n)DX_n\to\varphi'(X)DX\;\In L^p(\Om;H)$について,Holderの不等式から
            \begin{align*}
                &\norm{\varphi_n'(X_n)DX_n-\varphi'(X)DX}_{L^p(\Om;H)}\\
                &\le\norm{\varphi_n'(X_n)DX_n-\varphi'(X_n)DX_n}_{L^p(\Om;H)}+\norm{\varphi'(X_n)DX_n-\varphi'(X)DX_n}_{L^p(\Om;H)}+\norm{\varphi'(X)DX_n-\varphi'(X)DX}_{L^p(\Om;H)}\\
                &\le\norm{DX_n}_{L^p(\Om;H)}\norm{\abs{\varphi_n'(X_n)-\varphi'(X_n)}^p}_{L^\infty(\Om)}\\
                &\qquad +\norm{DX_n}_{L^p(\Om;H)}\norm{\abs{\varphi'(X_n)-\varphi'(X)}^p}_{L^\infty(\Om)}+\norm{\abs{\varphi'(X)}^p}_{\infty}\norm{DX_n-DX}_{L^p(\Om;H)}.
            \end{align*}
            と評価できる.
        \end{enumerate}
    \end{description}
\end{Proof}

\begin{lemma}[\cite{Kunze13-Malliavin} Lemma 1.2.8]
    $1<p<\infty$,$\{X_n\}\subset D^{1,p}$について,次を仮定する:
    \begin{enumerate}[{[A}1{]}]
        \item $X_n\to X\;\In L^p(\Om)$.
        \item $\{DX_n\}\subset L^p(\Om;H)$は有界:
        \[\sup_{n\in\N}E[\norm{DX_n}^p_H]<\infty.\]
    \end{enumerate}
    このとき,$X\in D^{1,p}$かつ$DX_n\wto DX\;\In L^{p}(\Om;H)$は弱収束する.
\end{lemma}
\begin{Proof}
    $1<p<\infty$としているから,$L^p(\Om;K)$は任意のHilbert空間$K$について回帰的Banach空間である.
    等長な埋め込み$D^{1,p}\mono L^p(\Om)\oplus_{l^p} L^p(\Om;H)\simeq L^p(\Om;\R\oplus_{l^p} H)$
    により閉部分空間とみなせるから,$D^{1,p}$も回帰的である.
    \begin{enumerate}
        \item $\{X_n\}\subset L^p(\Om)$は有界である.すると$\{DX_n\}\subset L^p(\Om;H)$も有界であるという仮定から,
        $\{X_n\}$は$D^{1,p}$の点列としても有界である.
        ノルム空間について,回帰的であることと単位閉球が弱コンパクトであることとは同値であるから,
        $D^{1,p}$の有界列は弱位相について相対コンパクトであり,弱収束する部分列が取れる:$X_{n_k}\wto X\;\In L^p$.
        $D^{1,p}$は弱閉であるから,$X\in D^{1,p}$である.
        \item $X_n\wto X\;\In D^{1,p}$を得る.
        $D$の有界性より,$DX_n\wto DX\;\In D^{1,p}(H)$も従う.
    \end{enumerate}
\end{Proof}

\begin{theorem}[\cite{Kunze13-Malliavin} Prop1.2.9]
    $p\in(1,\infty)$,$X_j\in D^{1,p}\;(j\in[m])$とし,$\varphi:\R^m\to\R$をLipschitz連続関数とする.このとき,$\varphi(X)\in D^{1,p}$であり,$\abs{Y}\le \norm{\varphi}_{\Lip(\R^m)}\;\as$を満たす$Y\in L(\Om;\R^m)$が存在して,
    \[D\varphi(X)=\sum_{j=1}^mY_j(DX_j).\]
    さらに,$X$の法則が絶対連続であるとき,$Y_j=\partial_j\varphi(X)$が成り立つ\cite{Nualart-Introduction} Exercise 3.3.3.
\end{theorem}
\begin{Proof}
    Lipschitz定数を
    $L:=\norm{\varphi}_{\Lip(\R^m)}$とおく.
    \begin{enumerate}
        \item $\varphi_n:=\varphi*\al_n$と定めると,$\varphi_n\in C^\infty(\R^m)$であり,$\varphi_n\to\varphi$は広義一様収束し,$D\varphi_n$は有界である.
        また実は$\abs{D\varphi_n}\le L$が成り立つ.
        よって前の命題から,$\varphi_n(X)\in D^{1,p}$かつ
        \[D\varphi_n(X)=\sum_{j=1}^n\partial_j\varphi_n(X)DX_j.\]
        \item いま,$\varphi_n(X)\to\varphi(X)\;\In L^p(\Om)$,$\{D\varphi_n(X)\}\subset L^p(\Om;H)$は有界であるから,補題から$\varphi(X)\in D^{1,p}$かつ$D\varphi_n(X)\wto D\varphi(X)\;\In L^p(\Om;H)$.
        さらに,$\{D\varphi_n(X)\}\subset L^p(\Om;\R^m)$も有界であるから,ある部分列が存在して,ある$Y\in L^p(\Om;\R^m)$に弱収束する.
        \item Lipschitz連続関数も絶対連続であるから,
    \end{enumerate}
\end{Proof}


\subsection{発散の延長}

\begin{definition}
    \[\Dom\delta:=\Brace{u\in L^2(\Om;H)\;\middle|\;\begin{array}{l}\text{ある}\delta(u)\in L^2(\Om)\text{が存在して}\\\text{任意の}F\in D^{1,2}\text{について}\\E[(DF|u)]=E[\delta(u)F]\end{array}}.\]
    と定めると,$\delta(u)\in L^2(\Om)$は一意に定まり,対応$\delta:\Dom(\delta)\to L^2(\Om)$が定まる.
    これは$\S\subset D^{1,2}$は$L^2(\Om)$上稠密であるため.
\end{definition}

\begin{lemma}\mbox{}
    \begin{enumerate}
        \item 対応$\delta:\Dom\delta\to L^2(\Om)$は線型で,値は中心化されている:$E[\delta(u)]=0$.
        \item $\delta$は閉作用素である.すなわち,ある$\{u_n\}\subset\S_H$が
        \[u_n\xrightarrow{L^2(\Om;H)}u,\quad\land\quad \delta(u_n)\xrightarrow{L^2(\Om)}G\]
        を満たすならば,$u\in\Dom\delta$かつ$\delta(u)=G$.
        \item $\delta:\dom(\delta)\to L^2(\Om)$は有界線型作用素である.
    \end{enumerate}
\end{lemma}
\begin{Proof}\mbox{}
    \begin{enumerate}
        \item 任意の$u,v\in\Dom(\delta)$をとる.すると,任意の$F\in D^{1,2}$に対して,
        \begin{align*}
            E[(DF|u+v)]=E[(DF|u)]+E[(DF|v)]=E[F\delta(u)]+E[F\delta(v)]=E[F(\delta(u)+\delta(v))].
        \end{align*}
        が成り立ち,一意性より$\delta(u+v)=\delta(u)+\delta(v)$が成り立つ.
        よって,$\Dom(\delta)$は線型空間であり,$\delta$はその上の線型作用素である.
        また,$F=1$ととると,
        \[E[\delta(u)]=E[1\delta(u)]=E[(D1|u)]=0.\]
        を得る.
        \item 任意の$F\in D^{1,2}$について,内積の連続性より,
        \[E[\delta(u_n)F]=E[(DF|u_n)]\]
        の両辺はそれぞれ$E[GF],E[(DF|u)]$に収束するため.
        \item 次の節の命題による.
    \end{enumerate}
\end{Proof}

\begin{lemma}[発散の定義域の特徴付け]
    $u\in L^2(\Om;H)$について,次は同値:
    \begin{enumerate}
        \item ある$c\ge0$が存在して,任意の$Y\in D^{1,2}$について,$\abs{E[(DY|u)]}\le c\norm{Y}_{L^2(\Om)}$.
        \item $u\in \Dom(\delta)$.
    \end{enumerate}
\end{lemma}
\begin{Proof}
    $u\in H$に対して,
    (1)の条件は,
    $\varphi(Y):=E[(DY|u)]$によって定まる対応$D^{1,2}\to\R$が,$L^2(\Om)$のノルムについて有界な線型汎関数になるための条件である.
    よって$L^2(\Om)$上への一意な連続延長$L^2\to\R$を持つが,これはある$X\in L^2(\Om)$に関するRiesz表現を持つ:$E[(DY|h)]=\varphi(Y)=E[XY]$.
    よって$X=\delta(h)$.
    逆も辿れる.
\end{Proof}

\subsection{発散の性質}

\begin{proposition}
    次の性質は,$\S_H$より一般の$u,v\in\Dom\delta$について成り立つ:
    $h\in H$,$\{e_i\}_{i\in\N^+}\subset H$を正規直交基底として,
    \begin{enumerate}
        \item $D^{1,2}(H)\subset\Dom(\delta)$であり,任意の$u,v\in D^{1,2}(H)$について,
        \[E[\delta(u)\delta(v)]=E[(u|v)]+E\Square{\int^\infty_0\int^\infty_0D_su_tD_tv_sdsdt}=E[(u|v)]+E[\tr(DuDv)].\]
        \item 任意の$u,v\in D^{1,2}(H)$について,
        \[E[\delta(u)^2]\le E\Square{\int^\infty_0(u_t)^2dt}+E\Square{\int^\infty_0\int^\infty_0(D_su_t)^2dsdt}=\norm{u}^2_{1,2,H}.\]
        \item 任意の$F\in D^{1,2}$と$u\in\dom(\delta)$について,
        \[\delta(Fu)=F\delta(u)-(DF|u)\]
        が$Fu\in L^2(\Om;H)$かつ右辺も$L^2(\Om)$に属する限り成り立つ.
        \item 任意の$u\in D^{2,2}(H)$と$h\in H$について,$D_hu\in\dom(\delta),\delta(u)\in D^{1,2}$であり,
        \[D_h(\delta(u))=\delta(D_hu)+(h|u).\]
    \end{enumerate}
\end{proposition}
\begin{Proof}\mbox{}
    \begin{enumerate}
        \item \begin{description}
            \item[ノルムの定義] $u,v\in\S(H)$の場合はすでに示してある.特に,$u=v\in\S(H)$と取ると,
            \[E[\delta(u)^2]=E[\norm{u}^2_H]+E[\norm{Du}^2_{H\otimes H}]=\norm{u}^2_{D^{1,2}(H)}\]
            の成立を表す.すなわち,$\delta:D^{1,2}(H)\to L^2(\Om)$は等長である.
            よって,ただ一つの延長が$D^{1,2}(H)$上に存在するから,$D^{1,2}(H)\subset\Dom(\delta)$は確かである.
            \item[収束列] したがって,任意の$u\in D^{1,2}(H)$に対して,これに収束する$\{u_n\}\subset\S(H)$が存在する.すなわち,\[u_n\to u\;\In L^2(\Om;H),\quad\land\quad Du_n\to Du\;\In L^2(\Om;H\otimes H).\]
            するとこのとき,$\delta(u_n)$は収束列であるから,ある$X\in L^2(\Om)$が存在して,
            $\delta(u_n)\to X\;\In L^2(\Om)$.
            これについて,内積の連続性より,任意の$Y\in D^{1,2}$に対して
            \[E[(DY|u)]=\lim_{n\to\infty}E[(DY|u_n)]=\lim_{n\to\infty}E[Y\delta(u_n)]=E[YX].\]
            より,一意性から$X=\delta(u)$が必要.
            以上より,$\delta:D^{1,2}(H)\to L^2(\Om)$は随伴関係を満たしながら延長する.
            \item[結論] 以上より,任意の$u=v\in D^{1,2}(H)$の場合について結論を得た.
            極化恒等式の議論により,一般の$u\ne v\in D^{1,2}(H)$の場合も解る.
        \end{description}
        \item なぜ不等号なんだ??
        \item $F\in\S,u\in\S_H$の場合は,任意の$F,G\in\S$と$u\in\S_H$について,
        \[E[G\delta(Fu)]=E[G(F\delta(u)-(DF|u))].\]
        が成り立ち,$\S\subset L^2(\Om)$が稠密であることを用いて示した.
        まず,$F\in\S$は$D^{1,2}$上稠密であるから,内積の連続性から一般の$F\in D^{1,2}$について成り立つ.
        さらに,$Fu\in L^2(\Om;H)$かつ$F\delta(u)-(DF|u)\in L^2(\Om)$のとき,任意の$G\in D^{1,2}$についても成り立つから,結論が従う.
    \end{enumerate}
\end{Proof}

\subsection{高階の微分}

\begin{definition}[$k$-th Malliavin derivative]
    確率変数を$k$-径数付けられた過程へ写す対応
    $D^k:\S\to L^2(\Om;H^{\otimes k})$を
    \[D^k_{t_1,\cdots,t_k}F:=\sum_{i_1,\cdots,i_k\in[n]}\pp{^kf}{x_{i_1}\cdots\partial x_{i_k}}(B(h_1),\cdots,B(h_n))h_{i_1}(t_1)\cdots h_{i_k}(t_k).\]
    で定める.
    この過程を,$t_1,\cdots,t_k\in\R_+$を省けば,
    \[D^kF:=\sum_{i_1,\cdots,i_k\in[n]}\pp{^kf}{x_{i_1}\cdots\partial x_{i_k}}(B(h_1),\cdots,B(h_n))h_{i_1}\otimes\cdots\otimes h_{i_k}.\]
    と表す.
\end{definition}

\begin{proposition}
    任意の$1\le p<\infty$について,作用素$D^k:L^p(\Om)\to L^p(\Om;H^{\otimes k})$は可閉である.
\end{proposition}

\begin{definition}[Sobolev spaces]\mbox{}
    \begin{enumerate}
        \item $D^{k,p}$を次のセミノルム$\norm{F}_{k,p}$に関する$\S$の閉包とする:
        \[\norm{F}_{D^{k,p}}:=\Paren{E[\abs{F}^p]+E[\norm{DF}^p_H]+\cdots+E[\norm{D^kF}^p_{H^{\otimes k}}]}^{1/p}=\paren{E[\abs{F}^p]+E\Square{\sum_{j\in[k]}\Abs{\int_{\R^j_+}(D_{t_1,\cdots,t_j}^jF)^2dt_1\cdots dt_j}^{p/2}}}^{1/p}.\]
        \item $D^{k,\infty}:=\bigcap_{p\ge2}D^{k,p},D^{\infty,2}:=\bigcap_{k\ge1}D^{k,2},D^\infty:=\bigcap_{k\ge1}D^{k,\infty}$とする.
        \item $D^{k,p}(H)$も同様に定める.
    \end{enumerate}
\end{definition}
\begin{remarks}
    $q\ge p\ge 2,l\ge k\Rightarrow D^{l,q}\subset D^{k,p}$.
\end{remarks}

\begin{proposition}\mbox{}
    \begin{enumerate}
        \item (Leibnitz則) 任意の$F,G\in\S$と$k\ge2$について,
        \[D^k(FG)=\sum_{i=0}^k\comb{k}{i}(D^iF)(D^{k-i}G).\]
        \item (Holderの不等式) $p,q,r\ge2$は$1/p+1/q=1/r$を満たし,$F\in D^{k,p},G\in D^{k,p}$とする.このとき,$FG\in D^{k,r}$で,
        \[\norm{FG}_{k,r}\le c_{p,q,r}\norm{F}_{k,p}\norm{G}_{k,q}.\]
    \end{enumerate}
\end{proposition}

\section{確率積分としての発散}

\subsection{微分の局所性}

\begin{tcolorbox}[colframe=ForestGreen, colback=ForestGreen!10!white,breakable,colbacktitle=ForestGreen!40!white,coltitle=black,fonttitle=\bfseries\sffamily,
title=]
    局所化によってさらに定義域を広めることが出来る.
    $L^1_\loc(\P)$への確率積分の延長もその例であった.
\end{tcolorbox}

\begin{definition}
    $H_1,H_2$をBanach空間とする.
    \begin{enumerate}
        \item 作用素$T:L(\Om:H_1)\to L(\Om;H_2)$が\textbf{局所的}であるとは,$X(\om)=0\;\ae \om\in A$ならば$TX=0\;\ae\om\in A$が成り立つことをいう.
        \item $\F_{[a,b]}:=\sigma[B_s-B_a|s\in[a,b]]$とする.
    \end{enumerate}
\end{definition}

\begin{lemma}[Malliavin微分の局所性]
    $F\in D^{1,2}\cap L^2_{\F_{[a,b]}}(\Om)$について,$D_tF=0\;(l,P)\dae(t,\om)\in[a,b]^\comp\times\Om$.
\end{lemma}
\begin{Proof}\mbox{}
    \begin{description}
        \item[$F\in\S\cap L^2(\Om,\F_{[a,b]},P)$の場合] 任意の$F\in\S\cap L^2(\Om,\F_{[a,b]},P)$は,ある$h_1,\cdots,h_n\in L^2(\R_+)$であって$h_i(t)=0\;\ae t\in[a,b]^\comp$を満たすものを用いて
        \[F=f(B(h_1),\cdots,B(h_n))\]
        と表せるということを示せばよい.
        \item[一般の場合の近似] 
    \end{description}
\end{Proof}

\begin{lemma}[発散作用素の局所性 \cite{Kunze13-Malliavin} Prop.1.3.6]
    
\end{lemma}

\subsection{発散は確率積分の延長である}

\begin{tcolorbox}[colframe=ForestGreen, colback=ForestGreen!10!white,breakable,colbacktitle=ForestGreen!40!white,coltitle=black,fonttitle=\bfseries\sffamily,
title=]
    $u$が適合的でないとき,$u\notin L^2(\P)$であるが,この場合の一般のGauss過程に対する確率積分の(真の)延長を\cite{Skorokhod75-GeneralizationOfStochasticIntegral}が定義した.
    この定義は微分の随伴であることを\cite{Gaveau-Trauber82}が指摘した.
    Nualart and Pardoux (88)はMalliavin解析の手法を用いて,Skorohod積分に関する確率解析を構築した.
\end{tcolorbox}

\begin{theorem}[\cite{Gaveau-Trauber82}]\mbox{}
    \begin{enumerate}
        \item $L^2(\P)\subset\Dom(\delta)$.
        \item 任意の$u\in L^2(\P)$について,
        \[\delta(u)=\int^\infty_0u_tdB_t.\]
    \end{enumerate}
\end{theorem}
\begin{Proof}\mbox{}
    \begin{description}
        \item[$u\in\E$が単過程の場合] 
        \item[一般の$u\in L^2(\P)$の近似] 
    \end{description}
\end{Proof}

\begin{proposition}
    $u,v$を適合過程とする.
    \begin{enumerate}
        \item $s<t\Rightarrow D_tv_s=0$かつ$s>t\Rightarrow D_su_t=0$.
        \item 任意の$u,v\in D^{1,2}(H)$について,$E[\delta(u)\delta(v)]=E\Square{\int^\infty_0u_tv_tst}$.
        \item 任意の${u\in D^{1,2}(H)}$について,$D_t\paren{\int^\infty_0u_sdB_s}=u_t+\int^\infty_tD_tu_sdB_s$.
    \end{enumerate}
\end{proposition}

\subsection{過程のLebesgue積分の微分}

\begin{proposition}[内積と微分の可換性]
    $u\in D^{1,2}(H),h\in H$について,$(u|h)\in D^{1,2}$が成り立ち,
    \[D_t(u|h)=(D_tu|h).\]
\end{proposition}

\begin{corollary}\mbox{}
    \begin{enumerate}
        \item $D_t\int^T_0u_sds=\int^T_0D_tu_sds$.
        \item $D_t\int^T_0B_sds=\int^T_0D_tB_sds=T-t$.
    \end{enumerate}
\end{corollary}

\section{等直交Gauss過程}


\begin{tcolorbox}[colframe=ForestGreen, colback=ForestGreen!10!white,breakable,colbacktitle=ForestGreen!40!white,coltitle=black,fonttitle=\bfseries\sffamily,
title=]
    Brown運動が定める積分$B:H\to {}^0\!L^2(\Om)$は中心化されたGauss系
    であった.内積を保存するGauss系ならば,一般のもので良い.これを等直交Gauss過程という.
\end{tcolorbox}

\subsection{定義と存在}

\begin{definition}[isonormal Gaussian process (Segal 54)]
    $H$を可分Hilbert空間とする.$H$上の\textbf{等直交Gauss過程}とは,
    $H$に添え字付けられたGauss系$\H_1:=\Brace{W(h)}_{h\in H}\subset L^2(\Om)$であって,
    次を満たすものをいう:
    \begin{enumerate}
        \item 中心化されている.
        \item $E[W(h)W(g)]=(h|g)$を満たす.
    \end{enumerate}
    このとき,$\H_1\subset{}^0\!L^2(\Om)$はGauss部分空間である.
\end{definition}

\begin{proposition}[\cite{Nourdin-Peccati12-NormalApproximation} Prop. 2.1.1]
    任意の可分な実Hilbert空間$H$に対して,$H$上の等直交Gauss過程が存在する.
\end{proposition}
\begin{Proof}
    任意に可分実Hilbert空間$H$を取り,$\{e_i\}_{i\ge1}$をその正規直交基底とする.
    $Z_i\sim\rN(0,1)$をある$(\Om,\F,P)$上の独立確率変数列とする.
    このとき,任意の$h\in H$に対して,級数
    \[\sum_{i\in\N^+}(h|e_i)Z_i\]
    は$L^2(\Om)$の意味でも概収束の意味でも収束する.この極限を$X(h)$とおけば良い.
    \begin{enumerate}
        \item 構成から,任意の$H$の有限部分集合について,対応する$X(h)$の族は多変量正規分布に従う.
        \item $Z_i\sim\rN(0,1)$であったから,Parsevalの等式より,
        \[E[X(h)X(h')]=\sum_{i\in\N^+}(h|e_i)(h'|e_i)=(h|h').\]
    \end{enumerate}
\end{Proof}

\subsection{Hilbert空間の埋め込みとしての特徴付け}

\begin{proposition}[等直交Gauss過程は線型空間の埋め込み]
    $X=(X_h)_{h\in H}$を等直交Gauss過程とする.
    \begin{enumerate}
        \item 対応$h\mapsto X_h$は線型である.
        \item 対応$h\mapsto X_h$は$L^2(\Om)$の部分空間への同型である.
    \end{enumerate}
\end{proposition}
\begin{Proof}\mbox{}
    \begin{enumerate}
        \item 任意に$\lambda,\mu\in\R,h,g\in H$を取ると,
        \[E\Square{\Paren{X(\lambda h+\mu g)-\lambda X(h)-\mu X(g)}^2}=0\]
        より,$X(\lambda h+\mu g)=\lambda X(h)+\mu X(g)$である.

        実際,
        \begin{align*}
            &E\Square{\Paren{X(\lambda h+\mu g)-\lambda X(h)-\mu X(g)}^2}\\
            &=E[X(\lambda h+\mu g)^2]-2E[X(\lambda h+\mu g)(\lambda X(h)+\mu X(g))]+E[(\lambda X(h)+\mu X(g))^2]
        \end{align*}
        となるが,第1項と第3項は等直交性からいずれも$\norm{\lambda h+\mu g}^2$に等しい.
        第2項はその2倍の符号を変えたものに等しい.中心化されている事実は使わなかった.
    \end{enumerate}
\end{Proof}
\begin{remarks}
    等直交Gauss過程とは,単にある実可分Hibert空間$H$からの,$L^2(\Om)$の中心化されたGauss部分空間(これは閉である)への閉埋め込みである.
\end{remarks}

\subsection{等直交Gauss過程の例}

\begin{example}[多変量正規分布の全体]
    $H:=\R^d$とし,任意の$h=(c_1,\cdots,c_d)^\top\in H$に対して,
    \[X(h):=\sum_{j=1}^dc_jZ_j\]
    と定めればこれは等直交であるが,これは
    要するに多変量正規分布$X(h)\sim\rN_d(0,c_j^2)$への埋め込み$\R^d\mono L^2(\Om)$である.
\end{example}

\begin{example}[Gauss測度 / 白色雑音]
    $A$を完備可分距離空間とし,その上の
    $\sigma$-有限で非原子的な測度空間$(A,\B(A),\mu)$を考えると,
    この上の体積確定集合の全体$\M^1\subset L^1(A)$は可分である.
    \begin{enumerate}
        \item コントロール$\mu$による\textbf{$A$上のGauss測度}または\textbf{白色雑音}とは,$A$上の体積確定集合$\M^1$に添字付けられた$L^2(\Om)$の部分空間$\{G(B)\}_{B\in\M^1}$であり,
        $E[G(B)G(C)]=\mu(B\cap C)$を満たすもの
        をいう.
        \item $\M^1$の代わりに$H:=L^2(A)$を考え,確率積分によって
        \[X(h):=\int_Ah(a)G(da).\]
        と定めると,これも等直交Gauss過程である.
        \item 特別な場合として,$A:=\R_+$を取って得るGauss測度$\{G(B)\}_{B\in\M^1(\R_+)}$の$[0,t)$という形の集合への制限$W_t:=G([0,t))$はBrown運動(の連続とは限らないバージョン)であり,
        $X$はBrown運動が生成するGauss部分空間に一致する.
    \end{enumerate}
\end{example}

\begin{example}[Gauss過程の延長として得る等直交Gauss過程]
    $Y=(Y_t)_{t\in\R_+}$を中心化された連続なGauss過程とし,
    共分散を$R(s,t)$とする.
    $Y$が生成するGauss部分空間は当然等直交Gauss過程となろうが,
    共分散関数のみに注目して,次のようにしても構成出来る:
    \begin{enumerate}
        \item $[0,T]$上の単関数からなる集合を
        \[\E:=\Brace{\sum_{i=1}^na_i1_{[0,t_i]}\in L^1(\R_+)\;\middle|\;n\in\N,t_i>0}.\]
        とする.
        \item \[(f|h):=\sum_{i,j\in[n]}a_ic_jR(s_i,t_j),\qquad f=\sum_{i\in[n]}a_i1_{[0,s_i]},h=\sum_{j\in[n]}c_j1_{[0,t_j]}.\]
        は$\E$上に内積を定めるが,これに関する閉包を$H$とすると,可分なHilbert空間ではあるが,$L^1(\R_+)$を飛び出して超関数や複素数値関数を含み得る.
        \item 任意の$h\in\E$に対して,
        \[X(h):=\sum_{j\in[n]}c_jY_{t_j}\]
        と定め,$h\in H$に関してはその$L^2(\Om)$-極限を取る.
    \end{enumerate}
\end{example}

\begin{example}[分数Brown運動]
    分数Brown運動が等直交Gauss過程を定めるが,添字の空間$H$が実Hilbert空間になるか
    どうかはHurst指数に依存する.
    \begin{enumerate}
        \item $H\in(0,1)$を\textbf{Hurst指数}とする.$R_H(t,s)=E[B^H_tB^H_s]:=2^{-1}(t^{2H}+s^{2H}-\abs{t-s}^{2H})$によって定まる連続なGauss過程$B^H=(B^H_t)_{t\in\R_+}$を\textbf{分数Brown運動}という.
        \item Kolmogorov (40)が自己相似過程に関する理論の一環で導入し,MandelbrotとVan Ness (68)が両側Brown運動に関する確率積分によって表示し,Brown運動に関連した現在の名前を付けた.
        \item Harold Hurst (51)がナイル川流域の貯水量のモデルに使用した.$H=1/2$の場合が標準Brown運動である.一般の$H$について定常増分を持つが,独立ではなく,$H>1/2$で負の相関,$H>1/2$で正の相関を持つ.
    \end{enumerate}
\end{example}

\begin{example}[Gaussian free field \cite{Nourdin-Peccati12-NormalApproximation} Ex. 2.1.7]
    
\end{example}

\subsection{非整数Brown運動の性質}

\begin{proposition}
    $B^H$について,
    \begin{enumerate}
        \item $B^H_t-B^H_s\sim N(0,\abs{t-s}^{2H})$.特に,独立増分を持つ.
        \item $B^H$の見本道は,$\gamma<H$について,$[0,T]$上$\gamma$-Holder連続である.
        \item 自己相似性:任意の$a>0$について,$B^H$と$(a^{-H}B^H_{at})_{t\in\R_+}$は分布同等である.
    \end{enumerate}
\end{proposition}

\begin{proposition}
    $B^H\;(H>1/2)$について,
    \begin{enumerate}
        \item 積分核$K_H\in L^2(\R_+^2)$を
        \[K_H(t,s):=c_Hs^{1/2-H}\int^t_sr^{H-3/2}(r-s)^{H-1/2}dr,\quad s<t,c_H:=\paren{\frac{H(2H-1)}{\beta(2-2H,H-1/2)}}^{1/2}.\]
        で定めると,次が成り立つ:
        \[R_H(t,s)=\int^{t\land s}_0K_H(t,u)K_H(s,u)du.\]
        \item $\abs{H}\subset L([0,T])$で,次のノルムに関するBanach空間を表すとする:
        \[\norm{\varphi}^2_{\abs{H}}:=H(2H-1)\int^T_0\int^T_0\abs{r-u}^{2H-2}\abs{\varphi(r)}\abs{\varphi(u)}dudr<\infty.\]
        このとき,次の包含関係が成り立つ:$L^2([0,T])\subset L^{1/H}([0,T])\subset\abs{H}\subset H$.
    \end{enumerate}
\end{proposition}

\begin{proposition}
    $B^H\;(H<1/2)$について,
    \begin{enumerate}
        \item 積分核$K_H\in L^2(\R_+^2)$を
        \[K_H(t,s):=c_H\paren{\paren{\frac{t}{s}}^{H-1/2}(t-s)^{H-1/2}-\paren{\frac{1}{2}-H}s^{1/2-H}\int^t_sr^{H-3/2}(r-s)^{H-1/2}dr},\quad s<t,c_H:=\paren{\frac{2H}{(1-2H)\beta(1-2H,H+1/2)}}^{1/2}.\]
        で定めると,次が成り立つ:
        \[R_H(t,s)=\int^{t\land s}_0K_H(t,u)K_H(s,u)du.\]
        \item $\forall_{\al>1/2-H}\;C^\al([0,T])\subset H\subset L^2([0,T])$.
    \end{enumerate}
\end{proposition}

\subsection{非整数Brown運動のWiener積分による表示}

\begin{tcolorbox}[colframe=ForestGreen, colback=ForestGreen!10!white,breakable,colbacktitle=ForestGreen!40!white,coltitle=black,fonttitle=\bfseries\sffamily,
title=]
    $H>1/2$のとき,$K^*_H$は非整数積分で表せ,$H<1/2$のとき非整数微分で表せる.
\end{tcolorbox}

\begin{proposition}
    $K^*_H:\E\to L^2([0,T])$を
    \[K^*_H1_{[0,t]}(s):=K_H(t,s)1_{[0,t]}(s)\]
    で定める.
    \begin{enumerate}
        \item 線型な等長写像であり,あるHilbert空間$H$上への延長$H\mono L^2([0,T])$を持つ.
        \item この延長は全射である:$\Im(K^*_H)=L^2([0,T])$.
    \end{enumerate}
\end{proposition}

\begin{definition}
    $W=\{W(\varphi)\}_{\varphi\in H}$を$W(\varphi):=B^H((K^*_H)^{-1}\varphi)$で定める.
\end{definition}

\begin{theorem}\mbox{}
    \begin{enumerate}
        \item $W$は中心化されたGauss系で,その共分散は
        \[E[W(\varphi)W(\psi)]=(\varphi|\psi)_{L^2([0,T])}\]
        で与えられる.
        \item $W$の$\E$への制限$W_t=B^H((K^*_H)^{-1}1_{[0,t]})$はBrown運動である.
        \item 任意の$\varphi\in H$について,
        \[B^H(\varphi)=\int^T_0K^*_H\varphi(s)dW_s.\]
        特に,
        \[B^H_t=\int^t_0K_H(t,s)dW_s.\]
    \end{enumerate}
\end{theorem}

\section{一般のMalliavin微分}

\subsection{正規Hilbert空間上での定義}

\begin{notation}
    正規な可分Hilbert空間$(H,\mu:=N_Q)$を考える.可算な正規直交基底$(e_n)$が取れる.
    \begin{enumerate}
        \item $\brac{e_1,\cdots,e_n}$が張る部分空間への直交射影を$P_n$で表す.
        \item ${}^uC_b(H)$で$H$上の一様連続な有界汎関数全体の集合を表す.これは,一様ノルムについてBanach空間をなす.
        \item ${}^uC_b^k(H)$で$k$階一様連続にFrechet微分可能な汎関数全体のなす空間に,各階の導関数に関する1-ノルムを入れて得るBanach空間とする.
        \item 成分を$x_k:=\brac{x,e_k}\;(x\in H)$,$D_k\varphi:=\brac{D\varphi,e_k}\;(\varphi\in {}^uC_b^1(H))$.このとき,$D\varphi\in H^*$に注意.
        \item \textbf{指数関数の空間}を,部分空間
        \[\E(H):=\Brac{\Re(\ep_h)\in L^2(H,\mu)\mid \ep_h(x):=e^{i\brac{x,h}},x,h\in H}\]
        とする.
    \end{enumerate}
\end{notation}

\begin{definition}
    $M:=Q^{1/2}D:\E(H)\to L^2(H,\mu;H)$とする.これは,
    \begin{enumerate}
        \item $\varphi\in\E(H)$はFrechet導関数$D\varphi:H\to H^*$を持つ.これに$Q^{1/2}:H\to\Im Q^{1/2}<H$を合成する.
        \item $DW_f$は,$W_f$が$f\in Q^{1/2}(H)$に関してしか定義されていない通り,稠密部分空間$\Im Q^{1/2}$上にしか定まっていない.が,$Q^{1/2}DW_f$は$f\in H$上で定まる.
    \end{enumerate}
\end{definition}

\subsection{指数関数による近似}

\begin{proposition}
    任意の$\varphi\in{}^uC_b^1(H)$について,ある二重列$\{\varphi_{n,k}\}\subset\E(H)$が存在して,
    \begin{enumerate}
        \item $\forall_{x\in H}\;\lim_{n\to\infty}\lim_{k\to\infty}\varphi_{n,k}(x)=\varphi(x)$.
        \item $\forall_{x\in H}\;\lim_{n\to\infty}\lim_{k\to\infty}D\varphi_{n,k}(x)=D\varphi(x)$.
        \item $\forall_{n,k\in\N}\;\norm{\varphi_{n,k}}_0+\norm{D\varphi_{n,k}}_0\le\norm{\varphi}_0+\norm{D\varphi}_0$.
    \end{enumerate}
\end{proposition}

\begin{corollary}\mbox{}
    \begin{enumerate}
        \item ${}^uC_b(H)$は$L^2(H,\mu)$上稠密である.
        \item $\E(H)$は$L^2(H,\mu)$上稠密である.
    \end{enumerate}
\end{corollary}

\subsection{連続延長}

\begin{proposition}
    $M:\E(H)\to L^2(H,\mu;H)$は可閉である.
    この定義域を\textbf{Malliavin-Sobolev空間}といい,$D^{1,2}(H,\mu)$で表す.
    1階導関数との1-ノルムについて,これはHilbert空間をなす.
\end{proposition}

\begin{proposition}[$C^1$-級緩増加関数はMalliavin微分可能である]
    $C_1^p(H)\subset D^{1,2}(H,\mu)$.
\end{proposition}

\begin{proposition}[Lipschitz関数はMalliavin微分可能である]
    $\Lip^1(H)\subset D^{1,2}(H,\mu)$.
\end{proposition}

\begin{proposition}[微分積分学の基本定理?]
    任意の$f\in H$について,$W_f\in D^{1,2}(H,\mu)$で,$MW_f=f$.
\end{proposition}

\chapter{Wiener Chaos}

\begin{quotation}
    確率重積分による$L^2(\Om)$の正規直交基底を構成する.
    これについての表示をWiener chaos展開という.
    これによって,2つの微分作用素をより詳しく調べることが出来る.
\end{quotation}

\section{確率重積分}

\begin{tcolorbox}[colframe=ForestGreen, colback=ForestGreen!10!white,breakable,colbacktitle=ForestGreen!40!white,coltitle=black,fonttitle=\bfseries\sffamily,
title=]
    等直交Gauss過程として,$H:=L^2(\R,e^{-x^2}dx)$を取ると,この上の直交系にはHermite多項式がある.
    この埋め込みが定める$L^2(\Om)$のGauss部分空間を\textbf{Wiener chaos}という.
\end{tcolorbox}

\subsection{Hermite多項式の定義}

\begin{definition}
    $H_0=1$とし,
    \[H_n(x):=\frac{(-1)^n}{n!}e^{\frac{x^2}{2}}\dd{^n}{x^n}\paren{e^{-\frac{x^2}{2}}},\qquad n\ge1.\]
    と定める.
\end{definition}

\begin{lemma}
    特性関数$F(u,x):=\exp\paren{ux-\frac{u^2}{2}}$の整級数展開は
    \[F(u,x)=\sum_{n=0}^\infty H_n(x)u^n.\]
    で与えられる.
\end{lemma}

\begin{lemma}
    $n\ge1$について,
    \begin{enumerate}
        \item $H'_n(x)=H_{n-1}(x)$.
        \item $(n+1)H_{n+1}(x)=xH_n(x)-H_{n-1}(x)$.
        \item $H_n(-x)=(-1)^nH_n(x)$.
    \end{enumerate}
\end{lemma}

\subsection{Hermite多項式の性質}

\begin{proposition}
    $(Z,Y)\sim\paren{\b{0}_2,\paren{\smtrx{1}{\rho}{\rho}{1}}}$を2次元正規分布とする.
    このとき,任意の$n,m\in\N$について,
    \[E[H_n(Z)H_m(Y)]=\begin{cases}
        n!\rho^n&n=m,\\
        0&\otherwise.
    \end{cases}\]
\end{proposition}
\begin{Proof}\mbox{}
    \begin{description}
        \item[$\rho>0$のとき] $N,\wt{N}\sim\rN(0,1)$を独立とすると,
        \[\vctr{Z}{Y}\overset{d}{=}\vctr{N}{\rho N+\sqrt{1-\rho^2}\wt{N}}.\]
        これについて,次のように計算出来る:
        \begin{align*}
            E[H_n(Z)H_m(Y)]&=E[H_n(N)T_{\log(1/\rho)}H_m(N)]\\
            &=\rho^mE[H_n(N)H_m(N)]\\
            &=\begin{cases}
                n!\rho^n&n=m,\\
                0&\otherwise.
            \end{cases}
        \end{align*}
        \item[$\rho=0$のとき] 次のように計算出来る:
        \[E[H_n(Z)H_m(Y)]=E[H_n(Z)]E[H_m(Y)]=\begin{cases}
            1&n=m=0,\\
            0&\otherwise.
        \end{cases}\]
        \item[$\rho<0$のとき] $H_m(-N)=(-1)^mH_m(N)$に気をつければ同様.
    \end{description}
\end{Proof}

\begin{definition}[$n$-th Wiener chaos]
    $X$を$H$-値の等直交Gauss過程とする.
    これが定める
    \textbf{$n$次Wiener混沌}とは,$H$の単位球面$\partial B:=\Brace{h\in H\mid\norm{h}=1}$に添字付けられた$n$次までのHermite多項式による像
    $\{H_n(X(h))\}_{h\in \partial B}$が生成する線型閉部分空間$\H_n\subset L^2(\Om)$をいう.
\end{definition}

\begin{example}
    $\H_0$は定数関数全体からなる空間である.$H_1(x)=x$より,$\H_1=\Im X=\{X(h)\}_{h\in H}$である.
\end{example}

\begin{corollary}[Wiener chaosの直交性]\label{cor-normality-of-Wiener-chaos}
    $X$を$H$-値の等直交Gauss過程とする.
    任意の$n\ne m\in\N$について,$\H_n\perp\H_m$である.
\end{corollary}

\subsection{Wiener-Itoの混沌分解}

\begin{theorem}[Wiener-Ito chaotic decomposition\footnote{\cite{Nourdin-Peccati12-NormalApproximation} Th'm 2.2.4}]\mbox{}\label{thm-Wiener-Ito-chaotic-decomposition}
    \begin{enumerate}
        \item $\{H_n(X(h))\}_{h\in\partial B,n\in\N}$が生成する線型部分空間は$L^q(\Om)\;(q\in[1,\infty))$上で稠密.
        \item 次が成り立つ:
        \[L^2(\Om)=\bigoplus_{n\in\N}\H_n.\]
        特に,任意の$F\in L^2(\Om)$に対して,$F_n\in\H_n$の列が一意的に存在して,$F=E[F]+\sum_{n\in\N^+}F_n$が$L^2(\Om)$の意味で収束して成り立つ.
    \end{enumerate}
\end{theorem}
\begin{Proof}
    Hilbert空間の直和は完備化を取ることに注意すれば,
    系\ref{cor-normality-of-Wiener-chaos}より,(1)が示せれば(2)が従う.
    完備性は,任意の$X\in L^2(\Om)$について,
\end{Proof}

\begin{corollary}[1次元の場合の結論]
    $\Brace{(n!)^{\frac{1}{2}}H_n\in L^2(\R,\B(\R),\phi(x;0,1))\mid n\in\N}$は$L^2(\R,\B(\R),\phi(x;0,1))$の正規直交基底である.
\end{corollary}
\begin{Proof}
    $H=\R$と取った場合に当たる.
\end{Proof}

\subsection{混沌の表示}

\begin{definition}
    $X$を$H$-値等直交Gauss過程とする.
    \[\P^0_n:=\Brace{p(X(h_1),\cdots,X(h_n))\in L^2(\Om,\F,P)\mid p\in\R[X_1,\cdots,X_k],\deg p\le n,h_1,\cdots,h_n\in H}.\]
    とし,この閉包を$\P_n$で表す.
\end{definition}

\begin{proposition}[\cite{Kunze13-Malliavin} Prop 1.1.14]
    \[\P_n=\H_0\oplus\cdots\oplus\H_n.\]
\end{proposition}

\chapter{Ornstein-Uhlenbeck半群}

\begin{quotation}
    2つの微分作用素の最も肝要と言える特徴付けは,Ornstein-Uhlenbeck半群の生成作用素によって与えられる.
    これにより,部分積分公式を得る.
\end{quotation}

\section{Ornstein-Uhlenbeck作用素}

\begin{definition}
    Ornstein-Uhlenbeck半群$\{T_t\}_{t\in\R_+}\subset \End(L^2(\Om))$とは,
    \[T_t(F)=\sum_{n=0}^\infty e^{-nt}I_n(f_n).\]
    をいう.
\end{definition}
\begin{remarks}
    これは,$f\in C_p^\infty(\R)$については,
    \[T_f(f)(x)=\int_\R f(e^{-t}x+\sqrt{1-e^{-2t}}y)d\rN(y),\qquad(x\in\R).\]
    と定義出来る.
\end{remarks}

\chapter{確率積分の逆問題}

\begin{quotation}
    $F\in L^2(\Om)$について,$F=\delta(u)$を満たす$u\in\Dom u$を見つける.これには2つの方法がある:
    \begin{enumerate}
        \item Clark-Ocone公式(84):$u$が適合過程の場合に適用可能.
        \item Ornstein-Uhlenbeck半群の生成作用素の逆を用いる.
    \end{enumerate}
\end{quotation}

\chapter{確率微分方程式}

\section{強解の存在と一意性}

\begin{tcolorbox}[colframe=ForestGreen, colback=ForestGreen!10!white,breakable,colbacktitle=ForestGreen!40!white,coltitle=black,fonttitle=\bfseries\sffamily,
title=]
    さらに係数が$C^1$で偏導関数が$C_b$ならば,Malliavinの意味で可微分な解を持つ\ref{prop-Malliavin-differentiability-of-diffusion-processes}.
\end{tcolorbox}

\begin{definition}[strong solution]
    $B$を$d$-次元Brown運動とし,$m$-次元の(時間的に一様な)確率微分方程式
    \[dX_t=\sum_{j\in[d]}\sigma_j(X_t)dB_t^j+b(X_t)dt,\quad X_0=x_0\in\R^d,\sigma_j,b\in L(\R^m;\R^m).\]
    の\textbf{強解}とは,次の2つを満たす適合過程$X\in\bF$をいう:
    \begin{enumerate}
        \item $\forall_{T>0}\;\forall_{p\ge2}\;E\Square{\sup_{t\in[0,T]}\abs{X_t}^p}<\infty$.
        \item $X_t=x_0+\sum_{j\in[d]}\int^t_0\sigma_j(X_s)dB_s^j+\int^t_0b(X_s)ds$.
    \end{enumerate}
\end{definition}

\begin{theorem}
    係数が$\sigma_j,b\in\Lip(\R^m;\R^m)$,すなわち,
    \[\forall_{x,y\in\R^m}\;\max_{j\in[d]}\paren{\abs{\sigma_j(x)-\sigma_j(y)},\abs{b(x)-b(y)}}\le K\abs{x-y}\]
    を満たすならば,一意な強解が存在する.
\end{theorem}

\section{弱解とmartingale問題}

\begin{definition}\mbox{}
    \begin{enumerate}
        \item 確率基底とその上の過程$X$との組を\textbf{弱解}という.弱解が等しいとは,Wiener空間${}^0\!C(\R_+)$上に押し出された分布が等しいことをいう.
        \item $\A:C^2_b(\R^d)\to C(\R^d)$を微分作用素とする.確率測度$P\in P(C(\R_+))$が$x\in\R^d$を出発点とする\textbf{$\A$-マルチンゲール問題}の解であるとは,次の2条件を満たすことをいう:
        \begin{enumerate}
            \item $P[\om_0=x]=1$.
            \item 次の$M$は$P$について$(\F_t)$-マルチンゲールである.
            \[\forall_{\varphi\in C^2_b(\R^d)}\quad M_t(\varphi)=M_t(\om,\varphi):=\varphi(\om_t)-\varphi(\om_0)-\int^t_0\A\varphi(\om_s)ds.\]
        \end{enumerate}
        \item 任意の$x\in\R^d$に対してマルチンゲール問題の解が一意に存在するとき,well-posedであるという.
    \end{enumerate}
\end{definition}

\begin{theorem}[マルチンゲール問題との等価性]
    次の2条件は同値:
    \begin{enumerate}
        \item 出発点$x$を持つ確率微分方程式の弱解$X$が存在して,その分布は$P_x$である.
        \item $P_x$は次の$\A$-マルチンゲール問題の解であり,これを分布に持つ弱解が存在する:
        \[\A:=\frac{1}{2}\sum_{i,j\in[d]}a^{ij}(x)\pp{^2}{x^i\partial x^j}+\sum_{i\in[d]}b^i(x)\pp{}{x^i}.\]
    \end{enumerate}
    また,解の一意性も,確率微分方程式の弱解と,マルチンゲール問題の解とで同値になる.
\end{theorem}

\section{解の強Markov性}

\begin{theorem}[強Markov性]
    適切な$\A$-マルチンゲール過程の一意な解を$(P_x)_{x\in\R^d}$とすると,これは強Markov性を持つ.
\end{theorem}
\begin{remark}
    解が一意ではなくとも,その中から強Markov性を持つものを選び出せる.これをKrylovのMarkov選択という.
\end{remark}

\begin{proposition}
    適切な$\A$-マルチンゲール問題について,
    $T_t\varphi(x):=E_x[\varphi(X_t)],\varphi\in L^\infty(\R^d)$とおくと,次が成り立つ:
    \begin{enumerate}
        \item 半群性:$\forall_{t,s\ge0}\;T_tT_s=T_{t+s}$.
        \item $\pp{T_t}{t}=T_t\A$.
    \end{enumerate}
    この$\A$をMarkov過程$X$の\textbf{生成作用素}という.
\end{proposition}

\section{Feynman-Kacの公式}

\begin{problem}
    $B$を$d$-次元Brown運動とし,$d$-次元の,時間的に一様で連続な係数を持つ確率微分方程式
    \[dX_t=\sum_{j\in[d]}\sigma_j(X_t)dB_t^j+b(X_t)dt,\quad X_0=x_0\in\R^d,\sigma_j,b\in C(\R^d;\R^d).\]
    を考える.
    \[\A\varphi(x):=\frac{1}{2}\sum_{i,j\in[d]}a^{ij}(x)\pp{^2\varphi}{x^i\partial x^j}(x)+\sum_{i\in[d]}b^i(x)\pp{\varphi}{x^i}(x).\]
    これは$\sigma,b$が例えば有界なとき,弱解$X$が存在する.
    ここではさらに一意性も仮定し,その境界$\partial D$への到達時刻を$\sigma$とする.
\end{problem}

\subsection{Dirichlet問題}

\begin{problem}
    $D\osub\R^d$を有界領域,$f\in C(\partial D)$を境界値,$V,g\in C(\o{D})$とする.
    \[\mathrm{D}\quad\begin{cases}
        \A u+Vu=-g\quad\mathrm{in}\;D\\
        u=f\quad\on\partial D.
    \end{cases}\]
    を考える.
\end{problem}

\begin{theorem}
    $\forall_{x\in D}\;P_x[\sigma<\infty]=1$とし,(D)は解を持つとする.このとき,次のように表示できる.特に,(D)の解は一意である:
    \[u(x)=E_x\Square{f(X_\sigma)\exp\paren{\int^\sigma_0V(X_s)dx}+\int^\sigma_0g(X_s)\exp\paren{\int^s_0v(X_r)dr}ds}\quad x\in\o{D}.\]
\end{theorem}

\begin{proposition}
    $\A$が一様に楕円形(すなわち係数行列$(a^{ij}(x))$は一様に正定値)ならば,$\forall_{x\in D}\;P_x[\sigma<\infty]=1$.
\end{proposition}

\subsection{放物型のCauchy問題}

\begin{tcolorbox}[colframe=ForestGreen, colback=ForestGreen!10!white,breakable,colbacktitle=ForestGreen!40!white,coltitle=black,fonttitle=\bfseries\sffamily,
title=]
    放物型のPDEに,Schrödinger方程式$iu_t+\Lap u=0$と熱・拡散方程式$u_t-\Lap u=0$とがある.
    KacはFeynmanの経路積分による量子化の発見に触発されて,その複素化されていないバージョンでもある熱拡散方程式に対して,
    確率過程の言葉で厳密な定式化を行った.
    一方で,Schrödinger方程式に対しては,測度論の方法に頼ることはできない.
\end{tcolorbox}

\begin{problem}[Kolmogorovの後退方程式に関するCauchy問題]
    初期値$f\in C_b(\R^d),V,g\in C_b(\R^d)$を持ち,$\A$の係数は1次の増大条件
    \[\norm{\al(t,x)}+\abs{b(t,x)}\le K(1+\abs{x})\quad(\abs{x}\to\infty)\]
    を満たすとする.次のKolmogorovの後退方程式のCauchy問題の解$u\in C^{1,2}(\R^+\times\R^d)\cap C(\R_+\times\R^d)$を考える:
    \[\mathrm{(C)}\quad\begin{cases}
        \pp{u}{t}=\A u+Vu+g\quad t>0,x\in\R^d,\\
        u(0,x)=f(x),\quad x\in\R^d.
    \end{cases}\]
\end{problem}


\begin{theorem}[Feynman-Kac formula]
    (C)の解$u$が存在して緩増加$\forall_{T>0}\;\exists_{C,p>0}\;\forall_{t\in[0,T]}\;\forall_{x\in\R^d}\;\abs{u(t,x)}\le C(1+\abs{x}^p)$ならば,解は一意で,次のように表示できる:
    \[u(t,x)=E_x\Square{f(X_t)\exp\paren{\int^t_0V(X_s)ds}+\int^t_0g(X_s)\exp\paren{\int^s_0V(X_r)dr}ds}.\]
\end{theorem}

\subsection{Neumann問題}

\begin{problem}
    $f\in C_b(\R_+)$が定めるNeumann問題
    \[\mathrm{(N)}\quad\begin{cases}
        \pp{u}{t}(t,x)=\frac{1}{2}\pp{^2u}{x^2}(t,x),\quad t>0,x\in\R_+\\
        u(0,x)=f(x),\quad x\in\R_+,\\
        \partial^+u(t,0)=0,\quad t>0.
    \end{cases}\]
    の解$u\in C^\infty(\R^+\times\R_+)\cap C(\R_+\times\R_+)$を考える.
\end{problem}

\begin{theorem}\label{thm-solution-to-Neumann-problem}
    \[u(t,x):=E_x[f(\abs{B_t})],\quad t\ge0,x\in\R_+\]
    は解である.
\end{theorem}

\chapter{密度推定}

\begin{quotation}
    Wiener空間値確率変数$\Om\to C_0(\R_+)$の密度を与える公式と,その正則性を判定する基準を与える.
    これはHormanderの超楕円性定理に対する確率論的アプローチでもある.
\end{quotation}

\subsection{拡散過程のMalliavin可微分性}

\begin{proposition}\label{prop-Malliavin-differentiability-of-diffusion-processes}
    $\sigma_j,b\in C^1(\R^m;\R^m)$は有界な偏導関数を持つとする.
    このとき,
    \begin{enumerate}
        \item $\forall_{t\in\R_+}\;\forall_{i\in[m]}\;X^i_t\in\bD^{1,\infty}$.
        \item $\forall_{t\le t}\;\forall_{j\in[d]}\;D^j_rX_t=\sigma_j(X_r)+\sum_{k\in[m]}\sum_{l\in[d]}\int^t_r\partial_k\sigma_l(X_s)D^j_rX^k_sdB_s^l+\sum_{k\in[m]}\int^t_r\partial_kb(X_s)D^j_rX^k_sds$.
    \end{enumerate}
\end{proposition}

\chapter{正規近似} 

\begin{quotation}
    Malliavin解析とSteinの手法を併せることで,正規近似の問題に取り組める.
\end{quotation}

\section{Steinの補題}

\begin{tcolorbox}[colframe=ForestGreen, colback=ForestGreen!10!white,breakable,colbacktitle=ForestGreen!40!white,coltitle=black,fonttitle=\bfseries\sffamily,
title=]
    Gauss核の定数倍が満たす微分方程式
    \[\phi'(x)-x\phi(x)\]
    に注目する.これの一般化である非斉次方程式
    \[f'(w)-wf(w)=1_{\ocinterval{-\infty,z}}(w)-\Phi(z),\qquad z\in\R\]
    をSteinの方程式という.
\end{tcolorbox}

\subsection{発想の根幹}

\begin{lemma}[Steinの補題]
    $X\in L^1(\Om)$について,次は同値:
    \begin{enumerate}
        \item $X\sim\rN(0,1)$.
        \item 任意の$f\in C^1_b(\R)$について,$E[f'(X)-f(X)X]=0$.
    \end{enumerate}
\end{lemma}
\begin{Proof}\mbox{}
    \begin{description}
        \item[(1)$\Rightarrow$(2)] $f,f'$はいずれも有界としたから,$E[f'(X)],E[f(X)]<\infty$に注意.
        部分積分により,
        \begin{align*}
            E[f'(X)]&=-\int_\R f(x)\phi'(x)dx\\
            &=\int_\R f(x)x\phi(x)dx=E[f(X)X].
        \end{align*}
        \item[(2)$\Rightarrow$(1)] $X$は可積分としたから,特性関数$\varphi(uy):=E[e^{iuX}]$の微分は
        \[\varphi'(u)=iE[Xe^{iuX}]=E[-ue^{iuX}]=-u\varphi(u).\]
        と計算できる.ただし,$f(X):=ie^{iuX}$とみて$f'(X)=-ue^{iuX}$であることを用いた.
        すると,この微分方程式を規格化条件$\varphi(0)=1$の下で解くと,$\varphi(u)=e^{-\frac{u^2}{2}}$.
    \end{description}
\end{Proof}

\subsection{一般化}

\begin{tcolorbox}[colframe=ForestGreen, colback=ForestGreen!10!white,breakable,colbacktitle=ForestGreen!40!white,coltitle=black,fonttitle=\bfseries\sffamily,
title=]
    $f$の有界性を仮定せずとも,Fubiniの定理を用いてより精緻な議論ができる.
    $W$の可積分性を仮定せずとも,$\phi'(x)=-x\phi(x)$の一般化であるSteinの方程式の解を用いることで,
    $W\sim\rN(0,1)$の十分条件が与えられる.
    これが\cite{Stein72-BoundForErrorOfNormalApproximation}の内容である.
\end{tcolorbox}

\begin{lemma}
    確率変数$W\in L(\Om)$について,(1)$\Rightarrow$(2)$\Rightarrow$(3)($\Rightarrow$(1))である.特に3条件は同値:
    \begin{enumerate}
        \item 任意の$C^1$-級の有界連続関数$f$について,$f'(Z)\in L^1(\Om)\;(Z\sim\rN(0,1))$ならば$E[f'(W)]=E[Wf(W)]$.
        \item $W\sim\rN(0,1)$.
        \item 任意の絶対連続関数$f:\R\to\R$について,$f'(Z)\in L^1(\Om)\;(Z\sim\rN(0,1))$ならば$E[f'(W)]=E[Wf(W)]$.
    \end{enumerate}
\end{lemma}
\begin{Proof}\mbox{}
    \begin{description}
        \item[(2)$\Rightarrow$(3)] $W\sim\rN(0,1)$とし,$f:\R\to\R$は絶対連続で$E[f'(W)]<\infty$とする.
        いま,Gauss核の微分方程式$\phi'(x)=-x\phi(x)$の両辺を積分することで
        \[\int^w_{-\infty}-xe^{-\frac{x^2}{2}}dx=e^{-\frac{w^2}{2}}.\]
        また,$w>0$のとき
        \[\int^w_{-\infty}-xe^{-\frac{x^2}{2}}dx=\int^{-w}_{-\infty}-xe^{-\frac{x^2}{2}}dx=\int^\infty_wxe^{-\frac{x^2}{2}}dx.\]
        が成り立つ.
        さらに積分範囲が$\Brace{(x,w)\in\R^2\mid x\le w\le0}$であることに注意すれば,Fubiniの定理より,
        Fubiniの定理より,
        \begin{align*}
            E[f'(W)]&=\frac{1}{\sqrt{2\pi}}\int^\infty_{-\infty}f'(w)e^{-\frac{w^2}{2}}dw\\
            &=\frac{1}{\sqrt{2\pi}}\int^0_{-\infty}f'(w)\paren{\int^w_{-\infty}-xe^{-\frac{x^2}{2}}dx}dw\\
            &\qquad +\frac{1}{\sqrt{2\pi}}\int^\infty_0f'(w)\paren{\int^\infty_wxe^{-\frac{x^2}{2}}dx}dw\\
            &=\frac{1}{\sqrt{2\pi}}\int^0_{-\infty}\paren{\int^0_xf'(w)dw}(-x)e^{-\frac{x^2}{2}}dx\\
            &\qquad +\frac{1}{\sqrt{2\pi}}\int^\infty_0\paren{\int^x_0f'(w)dw}xe^{-\frac{x^2}{2}}dx\\
            &=\frac{1}{\sqrt{2\pi}}\int_\R\SQuare{f(x)-f(0)}xe^{-\frac{x^2}{2}}dx=E[Wf(W)].
        \end{align*}
        \item[(1)$\Rightarrow$(2)] $z\in\R$の定めるStein方程式の有界な解$f_z$は可微分でもある.
        仮定より,
        \[0=E[f'_z(W)-Wf_z(W)]=E[1_{\ocinterval{-\infty,z}}(W)-\Phi(z)]=P[W\le z]-\Phi(z).\]
        であるから,$W\sim\rN(0,1)$.
    \end{description}
\end{Proof}

\begin{lemma}[Stein方程式の解]
    $z\in\R$に依存して定まる非斉次な線型1階方程式
    \[f'(w)-wf(w)=1_{\ocinterval{-\infty,z}}(w)-\Phi(z),\qquad z\in\R\]
    を満たす有界関数$f_z$は,次のただ一つである:
    \[f_z(w)=\begin{cases}
        \sqrt{2\pi}e^{\frac{w^2}{2}}\Phi(w)(1-\Phi(z))&w\le z,\\
        \sqrt{2\pi}e^{\frac{w^2}{2}}\Phi(z)(1-\Phi(w))&w>z.
    \end{cases}\]
\end{lemma}
\begin{Proof}
    この微分方程式には積分因子$e^{-\frac{w^2}{2}}$が見つかり,これを乗じると
    \[\paren{e^{-\frac{w^2}{2}}f(w)}'=e^{-\frac{w^2}{2}}\paren{1_{\ocinterval{-\infty,z}}(w)-\Phi(z)}\]
    を得る.この両辺を積分して,
    \begin{align*}
        f_z(w)&=e^{\frac{w^2}{2}}\int^w_{-\infty}\Paren{1_{\ocinterval{-\infty,z}}(x)-\Phi(z)}e^{-\frac{x^2}{2}}dx\\
        &=-e^{\frac{w^2}{2}}\int^\infty_w\Paren{1_{\ocinterval{-\infty,z}}(x)-\Phi(z)}e^{-\frac{x^2}{2}}dx.
    \end{align*}
    最後の等号は,等式
    \[\int^\infty_{-\infty}\Paren{1_{\ocinterval{-\infty,z}}(x)-\Phi(z)}e^{-\frac{x^2}{2}}dx=0\]
    による.
    よって,$w\le z$のときは第一行を見ると,被積分関数$(1_{\ocinterval{-\infty,z}}(x)-\Phi(z))$は$1-\Phi(z)$の形で積分の外に出て,積分は$\sqrt{2\pi}\Phi(w)$に等しい.
    $w<z$のときは第二行を見て被積分関数$(1_{\ocinterval{-\infty,z}}(x)-\Phi(z))$は$-\Phi(z)$の形で外に出て,積分は$1-\Phi(w)$に等しい.

    これの有界性は次の補題による.

    一般解はこの$f_z$に$Ce^{\frac{w^2}{2}}$を加えたものとして表せるが,$C=0$のときが唯一有界な時である.
\end{Proof}

\begin{lemma}
    $z\in\R$の定めるStein方程式の有界な解を$f_z$とする.このとき,
    \begin{enumerate}
        \item $w\mapsto wf_z(w)$は単調増加である.
        \item 任意の$w,u\in\R$について,\[\abs{wf_z(w)},\abs{wf_z(w)-uf_z(u)},\abs{f'_z(w)},\abs{f'_z(w)-f'_z(u)}\le1.\]
        \item 任意の$w\in\R$について,$0<f_z(w)\le\min\paren{\frac{\sqrt{2\pi}}{4},\frac{1}{\abs{z}}}$.
        \item 任意の$w,u,v\in\R$について,
        \[\abs{(w+u)f_z(w+u)-(w+v)f_z(w+v)}\le\paren{\abs{w}+\frac{\sqrt{2\pi}}{4}}(\abs{u}+\abs{v}).\]
    \end{enumerate}
\end{lemma}

\subsection{Stein関数}

\begin{definition}
    $h:\R\to\R$を可測関数,$h(Z)\in L^1(\Om)\;(Z\sim\rN(0,1))$とする.
    $Nh:=E[h(Z)]$について,
    \[f'(w)-wf(w)=h(w)-Nh.\]
    を\textbf{$h$に関するSteinの方程式},これを満たす$f$を\textbf{Stein関数}という.
\end{definition}

\begin{lemma}\mbox{}
    \begin{enumerate}
        \item $h$が有界ならば,
        \[\norm{f_h}\le\sqrt{\frac{\pi}{2}}\norm{h(\bullet)-Nh},\quad\norm{f'_h}\le2\norm{h(\bullet)-Nh}.\]
        \item $h$が絶対連続ならば,
        \[\norm{f_h}\le 2\norm{h'},\quad\norm{f'_h}\le\sqrt{\frac{2}{\pi}}\norm{h'},\quad\norm{f''_h}\le2\norm{h'}.\]
    \end{enumerate}
\end{lemma}

\subsection{多変数のStein関数}

\begin{notation}
    $h:\R^p\to\R$と$u\ge0$について,
    \[(T_uh)(\x):=E\Square{h\paren{\x e^{-u}+\sqrt{1-e^{-2u}}Z}},\qquad Z\sim\rN_p(0,I_p),\x\in\R^p.\]
\end{notation}

\begin{lemma}[\cite{Chen-Goldstein-Shao10-Normal-Approximation-with-Stein-Method} Lemma 2.6]
    $h:\R^p\to\R$を3階微分可能で,3階までの導関数は有界であるとする.このとき,
    \[g(x):=-\int^\infty_0\Paren{T_uh(x)-Nh}du,\qquad(x\in\R^p)\]
    はStein関数である.
\end{lemma}

\chapter{跳躍過程}

\begin{quotation}
    Poisson配置に関する確率積分を定義する.
\end{quotation}

\chapter{参考文献}

\bibliography{../StatisticalSciences.bib,../SocialSciences.bib,../mathematics.bib,../statistics.bib}
\begin{thebibliography}{99}
    %%% 確率解析の中心的話題は以下の通り.
    \item{Nualart}%スペインBarcelona大学で教えてからKansas大学へ渡米
    David Nualart and Eulalia Nualart (2018). \textit{Introduction to Malliavin Calculus}. Cambridge University Press.
    \item{Prato}%方針がすごくNualartに近い.ローマ大学.
    Giuseppe Prato. \textit{Introduction to Stochastic Analysis and Malliavin Calculus}.
    \item{LeGall}%独学に極めて向いている.Paris 6の新世代騎手.
    Jean-Francois Le Gall. (2013). \textit{Brownian Motion, Martingales, and Stochastic Calculus.} Springer.
    \item{McKean}%Fellerに教わったのち,Strook, Varadhan, DonskerらのいるNYマフィアへ.
    McKean, H. P. Jr. (1969). \textit{Stochastic Integrals}. Academic Press.
    \item{Tandem}
    松本裕行, 谷口説男 (2016). \textit{Stochastic Analysis: Itô and Malliavin Calculus in Tandem}. (Cambridge Studies in Advanced Mathematics).
    \item{楠岡成雄}
    楠岡成雄 (2018). 『確率解析』.(知泉書館,数理経済学叢書).

    確率微分方程式
    \item{舟木}
    舟木直久『確率微分方程式』
    \item{谷口}
    谷口説男 (2016) 『確率微分方程式』(数学の輝き,共立出版).
    \item{IkedaWatanabe}
    Ikeda, N., and Watanabe, S. \textit{Stochastic Differential Equations and Diffusion Processes}.
    \item{渡辺}
    渡辺信三 (1975). 『確率微分方程式』(産業図書).

    確率過程
    \item{LipsterShiryayev}
    Lipster, R. S. and Shiryayev, A. N. (1986). \textit{Theory of Martingale}. Kluwer Academic Publishers.
    \item{StrookVaradhan}
    Strook, D. W., and, Varadhan, S. R. S. (1979). \textit{Multidimensional Diffusion Processes}. Springer.
    \item{Strook}
    Strook, D. W. (2010). \textit{Probability Theory: An Analytic View}. Cambridge Univ. Press.
    \item{RogersWilliams}
    Rogers, L.C.G. and Williams, D. (1987) \textit{Diffusions, Markov processes and martingales}. John Wiley \& Sons.
    \item{Borodin}
    Borodin, A. N. (2013). \textit{Stochastic Processes}. Birkhauser.
    \item{Scheutzow}
    Michael Scheutzo. (2018). \textit{Stochastic Processes}. Lecture Notes, Technische Universitat Berlin.
    \item{Bremaud}
    Pierre Brémaud. (2020). \textit{Probability Theory And Stochastic Processes}. Springer.
    \item{JacodShiryayev}
    Jean Jacod, and Shiryayev, A. N. (2003). \textit{Limit Theorems For Stochastic Processes}. Springer.

    その他
    \item
    Daniel Revuz, and Marc Yor. (1999). \textit{Continuous Martingales and Brownian Motion}, 3rd. Springer.
    \item{Bass}
    Richard Bass - Stochastic Processes
    \item{厚地}
    厚地淳『確率論と関数論』
    \item{Morters-and-Peres}
    Morters and Peres - Brownian Motion

    歴史的文献
    \item{Kolmogorov31}
    Kolmogorov, A. N. (1933). Analytical methods in probability theory.
    「私はKolmogorovのこの論文(「解析的方法」)の序文にあるアイデアからヒントを得て,マルコフ過程の軌道を表す確率微分方程式を導入したが,これが私のその後の研究の方向を決めることになった.」
    \item{Ito42}
    Kiyosi Ito (1942). "Differential equations determining a Markoff process" (PDF). Zenkoku Sizyo Sugaku Danwakai-si (J. Pan-Japan Math. Coll.) (1077): 1352–1400.
    \item{Ito44}
    Kiyosi Itô (1944). "Stochastic integral". Proceedings of the Imperial Academy. 20 (8): 519–524.
    \item{KunitaWatanabe}
    Kunita, H., and Watanabe, S. (1967). On Square Integrable Martingales. \textit{Nagoya Mathematics Journal}. 30: 209-245.
    \item{Meyer}
    Meyer, P. A. (1967). Intégrales Stochastiques. \textit{Séminaire de Probabilités I}. Lecture Notes in Math., 39: 72-162. 
    劣martingaleのDoob-Meyer分解を用いて,確率積分が一般の半マルチンゲールについて定義された.
    こうして確率解析の復権が起こった.
\end{thebibliography}

\end{document}