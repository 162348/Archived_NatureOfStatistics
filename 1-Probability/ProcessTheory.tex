\documentclass[uplatex,dvipdfmx]{jsreport}
\title{確率過程論}
\author{司馬博文}
\date{\today}
\pagestyle{headings} \setcounter{secnumdepth}{4}
\input{/Users/Hirofumi Shiba/NatureOfStatistics/preamble_no_fonts.tex}
%\input{/Users/hirofumi.shiba48/NatureOfStatistics/preamble_no_fonts.tex}
%\input{/Users/hirof/NatureOfStatistics/preamble_no_fonts.tex}
\usepackage[math]{anttor}
\begin{document}
\tableofcontents

\chapter{確率過程の一般理論}

\begin{quotation}
    確率的な方法を使って数学的対象を調べることも,現実的対象を調べることも出来る.
    統計推測への応用も,調和解析への応用も考えたい.

    値の空間が等しい確率変数の族を確率過程といい,このときの値域である位相空間を状態空間という.\footnote{最も一般的にはBanach空間を取ることが流行らしい.}
    確率変数族には独立性の概念が拡張できたが,これは応用上自然ではない.
    遥かに緩いクラスとして,マルチンゲールを定義する.
    1930年代に,独立確率変数の和の理論を整備する過程で豊かに育ったKolmogorovのアイデアを一般化する試みの中で,Levyがマルチンゲールの概念を発明し,Doobが理論を立てた.
    Brown運動も確率積分もマルチンゲールになる.

    解析学に可測関数,連続関数,解析関数というようなクラスがあるように,確率論にもマルチンゲール,加法過程,Markov過程,定常過程などのクラスがある.
    解析学に指数関数,Bessel関数などの特殊関数があるように,確率論にもWeiner過程,Poisson過程というような特殊過程がある.
    ただし,分類の指導方針が全く違う.確率論の指導原理は独立性であってきた.
\end{quotation}

\begin{quote}
    A stochastic  process is the mathematical abstraction of an empirical processes whose development is governed by probabilistic laws. \cite{Doob} Preface, 一文目.
\end{quote}

\begin{notation}[空間の名前に関する記法]\mbox{}
    \begin{enumerate}
        \item 確率変数$X\in L(\Om)$に対して,$\F[X]:=\sigma[X]\lor 2$で,$X$が生成する閉$\sigma$-代数を表す.
        \item 確率過程$X=(X_t)$に対して,$\F[X]:=(\F_t[X])_{t\in T}$で,確率過程$X=(X_t)$が生成する自然な情報系を表す.
        \item Markov時刻の全体を$\T(\Om,\F,(\F_t),P)=\T((\F_t))$で表す.
        \begin{enumerate}
            \item そのうち可予測なものの全体を$\T_p$で表す.
            \item $\T^{<\infty}$で停止時の全体を表す.
        \end{enumerate}
        \item 増大情報系$\bF=(\F_t)_{t\in T}$の記法をそのまま$(\F_t)$-適合的な過程の全体$\bF=(\F_t)_{t\in T}\subset\Map(T;L(\Om))$にも用いる.$\bF(\F_t)$とも表す.
        \item $C$-過程,$D$-過程の全体を$C,D\subset\Map(T;L(\Om))$で表す.
        \item $\Om\times\R_+$上の関数の空間として,$L(\fP)\subset L(\D)\subset L(\P)\subset L(\F\otimes\B(\R_+))$.それぞれ,適合$C$-過程,適合$D$-過程,発展的可測過程である.
        \item 過程の空間$X$に対して,
        \begin{enumerate}
            \item $X^2:=X\cap L^2(T;L^2(\Om))$で自乗可積分なものを表す.
            \item ${}^0\!X$で$0$から始まるもののなす閉部分空間を表す.
            \item 特に情報系$\bF=(\F_t)$に適合していることを強調するとき,$X\cap\bF=X(\F_t)$と表す.
        \end{enumerate}
    \end{enumerate}
\end{notation}

\begin{notation}[過程の構成に関する記法]
    確率過程$X\in D$について,
    \begin{enumerate}
        \item 左連続化を$X_-:=(X_{t-})_{t\in\R_+}$と表す.
        \item ジャンプの過程を$\Delta X_t:=X_t-X_{t-}$と表す.
        \item 時刻$t$までの上限の過程を$X^*_t:=\sup_{s\le t}\abs{X_s}$と表す.$(\Delta X)_t^*=\sup_{s\le t}\abs{\Delta X_s}$である.
        \item 時刻$t$までの二次変分の過程を$\sum(\Delta X_s)^2:=\paren{\sum_{0\le s\le t}(\Delta X_s)^2}_{t\in\R_+}$で表す.
        \item 確率時刻$\tau$に対しては$\Delta X_\tau:=X_\tau1_{\tau<\infty}-X_{\tau-}1_{\tau<\infty}$と表す.
        \item 確率時刻$\tau$に対する停止過程を$X^\tau:=X_{t\land\tau}$で表す.
    \end{enumerate}
\end{notation}

\section{無限次元測度空間}

\begin{tcolorbox}[colframe=ForestGreen, colback=ForestGreen!10!white,breakable,colbacktitle=ForestGreen!40!white,coltitle=black,fonttitle=\bfseries\sffamily,
title=]
    関数解析の知識に,測度論を加える必要がある.
\end{tcolorbox}

\subsection{$C,D$-空間}

\begin{tcolorbox}[colframe=ForestGreen, colback=ForestGreen!10!white,breakable,colbacktitle=ForestGreen!40!white,coltitle=black,fonttitle=\bfseries\sffamily,
title=]
    $C(T)\subset D(T)\subset\L(T)$に注意.
\end{tcolorbox}

\begin{theorem}[完備可分距離空間のBorel集合族の特徴付け]
    $(S,\tau)$を完備可分距離空間とする.$\tau_1\subset\tau$も位相で,$(S,\tau_1)$がHausdorffならば,2つが生成するBorel $\sigma$-代数は等しい:$\B(S,\tau)=\B(S,\tau_1)$.
\end{theorem}

\begin{theorem}[$C,D$上のBorel集合族の特徴付け]
    $\F\subset \L(T)$上のKolmogorov $\sigma$-代数$\B_K(\F)$とは,評価写像の族$(\pi_t:\F\to\R)_{t\in T}$を可測にする$\F$上の最小の$\sigma$-代数なのであった.
    \begin{enumerate}
        \item $C(T)$に広義一様収束位相を入れて考える.
        すると,これについてのBorel集合族$\B(C(T))$は,Kolmogorovの$\sigma$-代数$\B_K(C(T))=\lor_{t\in T}\pi_t^*(\B(\R))$に等しい.
        \item $D(T)$にSkorohod位相を入れると,これは完備可分距離化可能空間である.すると,これについてのBorel集合族$\B(D(T))$は,Kolmogorovの$\sigma$-代数$\B_K(D(T))=\lor_{t\in T}\pi_t^*(\B(\R))$に等しい.
    \end{enumerate}
\end{theorem}
\begin{remarks}
    特に,(少なくとも$T=[0,1]$のときは)$D(T)$のSkorohod位相の$C(T)$への相対化が広義一様収束位相である.
    要は2つは$\L(T)$からみれば同じ位相である.では$\L(T)$上の位相としては何に当たるのだろうか?
\end{remarks}

\begin{history}
    Skorokhod空間は1956年に導入された.
\end{history}

\subsection{写像空間上の測度の存在と一意性}

\begin{tcolorbox}[colframe=ForestGreen, colback=ForestGreen!10!white,breakable,colbacktitle=ForestGreen!40!white,coltitle=black,fonttitle=\bfseries\sffamily,
    title=]
    Kolmogorovの拡張定理は,Hopfの拡張定理の一般化である.
    そこから確率過程論についても,種々の対象が構成できる.
    $\R_+$上の確率過程も,有限次元周辺分布を指定することで構成できるが,$\R^\infty$の場合と違って一意性は担保されない.
\end{tcolorbox}

\begin{theorem}[帰納極限の構成]
    確率空間列$(\R^n,\B(\R^n))$上の確率測度列$(\mu_n)$が次の一貫性条件をみたすとき,$(\R^\N,\B(\R^\N))$上の確率測度$\mu$であって$\forall_{A\in\B(\R^n)}\;\mu(A\times\R^\N)=\mu_n(\A)$を満たすものが一意的に存在する.ただし,
    $A\times\R^\N=\Brace{(\om_n)\in\R^\N\mid (\om_1,\cdots,\om_n)\in A}$とした.
    なお,$\R^\N$には直積位相を考える.
    \begin{quotation}
        (consistency) $\forall_{n\in\N}\;\forall_{A\in\B(\R^n)}\;\mu_{n+1}(A\times\R)=\mu_n(A)$.
    \end{quotation}
    特に,この一貫性条件は$\R^\N$上の測度に延長できるための必要十分条件である.
    これは$\R$を一般の完備可分空間としても成り立つ.
\end{theorem}

\subsection{$C$-空間上の測度の存在とAscoli-Arzelàの定理}

\begin{theorem}
    $C(\R_+;\R^d)$-過程$(X_n)_{n\in\N}$が次の2条件を満たすならば,ある部分列$\{n_k\}$と確率空間$(\wh{\Om},\wh{\F},\wh{P})$とその上の$(X_{n_k})_{k\in\N}$と分布同等な確率変数列$(\wh{X}_{n_k})_{k\in\N}$が存在して,$\wh{X}$に概広義一様収束する.
    \begin{enumerate}
        \item 一様(本質的)有界:$\lim_{N\to\infty}(\sup_{n\in\N}P[\abs{X_n(0)}>N])=0$.
        \item 同程度一様連続:$\forall_{\ep>0}\;\forall_{T>0}\;\lim_{h\searrow0}(\sup_{n\in\N}P\Square{\max_{\abs{t-s}\le h}\abs{X_n(t)-X_n(s)}>\ep})=0$.
    \end{enumerate}
    さらに,$(X_n)$の任意の有限次元周辺分布が収束するならば,$\{X_n\}=\{X_{n_k}\}$と取れる.すなわち,部分列を取る必要はない.
\end{theorem}

\begin{proposition}[Kolmogorovの連続性定理]
    $C(\R_+;\R^d)$-過程$(X_n)_{n\in\N}$が次の2条件を満たすならば,上の定理の2つの要件を満たす.
    \begin{enumerate}
        \item $\exists_{M,\gamma>0}\;\forall_{n\in\N}E[\abs{X_n(0)}^\gamma]\le M$.
        \item $\exists_{\al,\beta>0}\;\exists_{\{M_k\}\subset\R^+}\;\forall_{n,k\in\N}\;\forall_{t,s\in[0,k]}\;E[\abs{X_n(t)-X_n(s)}^\al]\le M_k\abs{t-s}^{1+\beta}$.
    \end{enumerate}
\end{proposition}

\begin{corollary}[さらに一般の構成]
    確率分布族$\{P_{t_1,\cdots,t_n}\}_{n\in\N^+,t_1<\cdots<t_n\in\R_+}$
    が次の条件を満たすとき,$\Map(\R_+;\R^d)$上の確率測度$P$が存在して,その有限次元周辺分布になる:
    \begin{enumerate}[({C}1)]
        \item $P_{t_1,\cdots,t_n}\in P(\R^{n\times d})$である.
        \item 任意の部分集合$\{t_{k_1}<\cdots<t_{k_m}\}\subset\{t_1<\cdots<t_n\}$について,$P_{t_{k_1}},\cdots,P_{t_{k_m}}$は$P_{t_1,\cdots,t_n}$の対応する周辺分布である.
    \end{enumerate}
    さらに次を満たすとき,確率測度$P$は$C(\R_+;\R^d)$上の確率測度とも見れる:
    \[\forall_{\al,\beta,T>0}\;\exists_{M_T>0}\;\forall_{0\le t_1<t_2<T}\quad\int_{\R^d\times\R^d}\abs{x_{t_1}-x_{t_2}}^\al P_{t_1,t_2}(dx_{t_1}dx_{t_2})\le M_T(t_2-t_1)^{1+\beta}.\]
\end{corollary}

\section{過程という枠組み}

\subsection{過程の定義}

\begin{tcolorbox}[colframe=ForestGreen, colback=ForestGreen!10!white,breakable,colbacktitle=ForestGreen!40!white,coltitle=black,fonttitle=\bfseries\sffamily,
title=]
    写像$T\to \L(\Om)$を過程という.
    $C(\Om)\subset D(\Om)\subset \L(\Om)$に値が収まるとき,
    $\Om\times T\to\R$も可測で,$\Om\to D(T)$もそのSkorohod位相について可測になる.
\end{tcolorbox}

\begin{definition}[stochastic process]
    $T\in\Meas$を可測空間,$E\in\Top$を位相空間とする.
    \begin{enumerate}
        \item 写像$X_-:T\to \L(\Om;E)$を\textbf{確率過程}という.
        その値域としての確率変数の族$\{X_t\}_{t\in T}\subset \L(\Om;E)$も確率過程という.
        \item $L(T)=\L(T)/\cN$は確率収束の位相について完備可分距離空間(特にFrechet空間)となる.
        この位相について$X$が連続であるとき,これを\textbf{確率連続}であるという:
        \[\forall_{s\in T}\;\forall_{\ep>0}\;\lim_{t\to s}P[\abs{X_t-X_s}>\ep]=0.\]
        \item 付随する2変数写像$\Om\times T\to\R;(\om,t)\mapsto X_t(\om)$が積可測であるとき,過程$X$は\textbf{可測過程}であるという.
        \item 逆に,可測写像$\Om\times T\to\R$は可測過程を定める,すなわち,$\forall_{t\in T}\;X_t\in \L(\Om)$が成り立つことはFubiniの定理に含意されている.従って逆向きの転置に関して,$X_-(\om)\in\L(T)$も成り立つ.
    \end{enumerate}
\end{definition}

\begin{definition}[continuous process, right-continuous process]
    過程$(X_t)$について,
    \begin{enumerate}
        \item 付随する写像$X_\bullet:\Om\to \Map(T;E)$を\textbf{転置}といい,その値$(X_t(\om))_{t\in T}:T\to E$を\textbf{見本過程}という.
        \item その値域$\Im X_\bullet$が$D\subset\Map(T;E)$に収まるとき\textbf{$D$-過程}といい,さらに$C$にも収まるとき\textbf{$C$-過程}という.
        \item 確率1で収まるとき,これらを広義の$D$-過程,$C$-過程という.
    \end{enumerate}
\end{definition}
\begin{remarks}
    この見本道の空間への写像は,可測性についての問題が大きいために過程の定義としては採用し得ない.
\end{remarks}

\begin{theorem}[$D$-過程について,2つの見方は等価である]\mbox{}
    \begin{enumerate}
        \item 過程$(X_t)$が$C$-過程,または$D$-過程であるとき,対応$X_\bullet:\Om\to D(T;E)$はたしかに可測で,たしかにこれは確率変数である.
        \item 逆に,確率変数$X:\Om\to C(T)$が与えられたとき,各$X_t:\Om\to E$はたしかに可測であり,したがってこれは$C$-過程である.
        \item $C$-過程と$D$-過程は可測過程である.
    \end{enumerate}
\end{theorem}

\begin{remark}[確率過程とは何か]
    \begin{enumerate}
        \item 添字集合$T$を時間と見るとき,$T\subset\o{\R}$とする.
        統計力学では$T$は冪集合のなす有向集合,Gauss過程の見本道の正則性理論では単に距離空間とすることもある.
    \end{enumerate}
\end{remark}

\subsection{過程の同値性}

\begin{tcolorbox}[colframe=ForestGreen, colback=ForestGreen!10!white,breakable,colbacktitle=ForestGreen!40!white,coltitle=black,fonttitle=\bfseries\sffamily,
title=]
    模型としての同値性は,法則同等によって定めるのが良いであろう.
    より強い同等性の概念は,
\end{tcolorbox}

\begin{definition}[indistinguishable, modification, equivalent]
    $(X_t),(Y_t)$を過程とする.
    \begin{enumerate}
        \item $P[\forall_{t\in T}\;X_t=Y_t]=1$のとき,\textbf{強同等}または\textbf{識別不可能}であるという.これは$\Brace{(\om,t)\in\Om\times\R_+\mid X\ne Y}$が消えゆく運命であることに同値.
        \item $\forall_{t\in T}\;P[X_t=Y_t]=1$のとき,\textbf{同等}であるという.また,$Y_t$を$X_t$の\textbf{修正}または\textbf{変形}という.
        \item $\forall_{n\in\N^+}\;\forall_{t_1,\cdots,t_n\in T}\;P^{(X_{t_1},\cdots,X_{t_n})}=P^{(Y_{t_1},\cdots,Y_{t_n})}$が成り立つとき,\textbf{法則同等}または\textbf{同値}であるという.
    \end{enumerate}
\end{definition}
\begin{remarks}\mbox{}
    \begin{enumerate}
        \item 2つが可測過程であるとき,左転置$\Om\to L(T)$が全く同じ写像を定めることが強同等である.
        $C(T)\subset\L(T)$に注意すると,$C$-過程を$\Im X\cap C(T)$が充満集合であるという広義の意味で解釈すれば,$C$-過程と強同等な過程は$C$-過程である.
        \item 2つが可測過程であるとき,右転置$T\to L(\Om)$が全く同じ写像を定めることをいう.
        \item 2つが可測過程であるとき,左転置$\Om\to L(T)$が,$L(T)$上に同じ有限周辺分布を持つ分布を押し出すことをいう.
    \end{enumerate}
\end{remarks}

\begin{example}\mbox{}
    \begin{enumerate}
        \item 標準確率空間$([0,1],\B([0,1]),l)$上の過程
        \[X_t(\om)=\begin{cases}
            0&B_t(\om)\ne0,\\
            1&B_t(\om)=0.
        \end{cases}\]
        は,$Y=0$と比べると,任意の$t\in\R_+$に対して$P[X_t=Y_t]=1$である.
        分かりにくかったら$B$の代わりに$\delta$としても良い.
        つまり同等になってしまう.これは強同等の概念を要請する所以である.
        \item $D$-過程$X,Y$が,任意の有理点$t\in\Q_+$において$X_t=Y_t\;\as$であるならば,
        強同等である.
        実際,$\cup_{t\in\R_+}\Brace{X_t(\om)\ne Y_t(\om)}=\cup_{t\in\Q_+}\Brace{X_t(\om)\ne Y_t(\om)}$は零集合である.
    \end{enumerate}
\end{example}

\begin{theorem}[$D$-過程の有限周辺分布が等しいなら分布全体は等しい]
    $D$過程$(X_t),(Y_t)$について,
    \begin{enumerate}
        \item 2つは法則同等であるとする.このとき,左転置$X_\bullet:\Om\to D(T),Y_\bullet:\Om\to D(T)$が押し出す$D(T)$上の確率分布$P^{X_\bullet},P^{Y_\bullet}$は等しい.
        \item 2つは同等であるとする.このとき,識別不可能である\cite{Nualart-Introduction}.
    \end{enumerate}
\end{theorem}
\begin{Proof}
    Dynkin族定理から.
\end{Proof}

\subsection{過程の可分修正}

\begin{tcolorbox}[colframe=ForestGreen, colback=ForestGreen!10!white,breakable,colbacktitle=ForestGreen!40!white,coltitle=black,fonttitle=\bfseries\sffamily,
title=]
    $T\subset\R$のとき,本来ならば
    \[A:=\Brace{\om\in\Om\mid X_\bullet(\om)\in C(T)}\]
    という集合は可測であるかどうかもわからない.
    この問題は,過程$(X_t)$は適宜可分に取り直せることによって無視できる.
    この点を克服したのがDoobであった.
    今後,暗黙のうちに過程は可分な修正を取る.
\end{tcolorbox}

\begin{definition}[separability]
    過程$(X_t)$が可分であるとは,ある加算部分集合$S\subset T$が存在して,次が成り立つことをいう:
    \[P[\forall_{t\in T}\;\liminf_{S\ni s\to t}X_s\le X_t\le\limsup_{S\ni s\to t}X_s]=1.\]
\end{definition}

\begin{lemma}
    広義の$C$-過程は可分である.
\end{lemma}

\begin{theorem}[Th'm 8.2 (Doob,\cite{Doob})]\label{thm-Doob-separable-theorem}
    任意の過程$(X_t)$に対して,可分な修正が存在する.
    すなわち,$T$を非可算集合,$\{X_t\}_{t\in T}\subset\L(\Om)$とする.
    \begin{enumerate}
        \item 任意の$A\in\sigma[X_t,t\in T]$に対して,ある可算部分集合$S\subset T$が存在して,$A\in\sigma[X_t,t\in S]$が成り立つ.
        \item 任意の$Y\in L^1(\Om)$について,可算部分集合$S\subset T$が存在して,$E[Y|X_t,t\in T]=E[Y|X_t,t\in S]\;\ae$
    \end{enumerate}
\end{theorem}

\subsection{$L^2$-過程と半正定値核}

\begin{tcolorbox}[colframe=ForestGreen, colback=ForestGreen!10!white,breakable,colbacktitle=ForestGreen!40!white,coltitle=black,fonttitle=\bfseries\sffamily,
title=]
    一様可積分なマルチンゲールは本質的には条件付き期待値だけである.
    同様にして,$L^2$-過程は半正定値核と対応する.
    $L^2(\Om)\subset L(\Om)$の部分集合$\xi:T\to L^2(\Om)$を$L^2$-過程という.
    このクラスの過程は共分散関数を持つ.
\end{tcolorbox}

\begin{proposition}[半正定値核と一対一対応する]\mbox{}
    \begin{enumerate}
        \item 2次の中心積率$K_\xi(t,s):=E[(\xi(t)-E[\xi(t)])\o{(\xi(s)-E[\xi(s)])}]$は正の定符号核である.
        \item 逆に,正の定符号核は,ある2次過程の中心積率である.
    \end{enumerate}
    これを\textbf{共分散関数}という.
\end{proposition}

\begin{lemma}
    区間$T\subset\R$上の
    2次過程$\xi:T\to L^2(\Om)$について,
    \begin{enumerate}
        \item $L^2$-連続であることは,$K_\xi:T^2\to\R$が$\Delta\subset T^2$上連続であることに同値.
        \item $L^2$-導関数
        \[\xi'(t):=\lim_{n\to\infty}\frac{\xi(t+h_n)-\xi(t)}{h_n}\quad((h_n)\text{は0に収束する任意の数列})\]
        が存在することは,$K_\xi$が$\Delta$上で2階の偏導関数を持つことに同値.
        \item $(a,b)\subset T$上で2次変分を持つことは,$K_\xi\in L^1((a,b)^2)$に同値.
    \end{enumerate}
\end{lemma}

\subsection{$L^2$-過程の可分変形}

\begin{tcolorbox}[colframe=ForestGreen, colback=ForestGreen!10!white,breakable,colbacktitle=ForestGreen!40!white,coltitle=black,fonttitle=\bfseries\sffamily,
title=]
    マルチンゲールの可分変形は,第1種の不連続性しか持たない.
    同様にして$L^2$-過程の可分変形は$C$-過程になるための条件が核の言葉で表せる.
\end{tcolorbox}

\begin{theorem}[2次過程が$C$-過程である条件]
    2次過程$\xi:T\to L^2(\Om)$について,
    \[\frac{K_\xi(t+h,t+h)-K_\ep(t,t+h)-K_\ep(t+h,t)+K_\ep(t,t)}{h^2}=O(1)\]
    が成り立つならば,過程$\xi$の可分な変形は$C$-過程である.
\end{theorem}

\subsection{マルチンゲールの可分変形}

\begin{tcolorbox}[colframe=ForestGreen, colback=ForestGreen!10!white,breakable,colbacktitle=ForestGreen!40!white,coltitle=black,fonttitle=\bfseries\sffamily,
    title=]
    Markov性やmartingale性などは独立性の一般化であると了解しやすい.
    マルチンゲールの可分変形は第1種の不連続性しか持たないため,この不連続点での値を右連続に修正すれば$D$-過程になる.
    この事実は上渡回数に関する不等式による.
    さらに,平均$s\mapsto E[X_s]$が連続ならば,この$D$-過程は修正でもある(不連続点の全体は零集合).
\end{tcolorbox}

\begin{definition}[Markov chain, martingale]
    確率変数列$(X_n)$が定める試行の列$(\fA^n)$について,
    \begin{enumerate}
        \item $\forall_{k\in[n]}\;\forall_{i\in[r_k]}\;P[A_i^k|\A^1\cdots\A^{k-1}]=P[A^k_i|\fA^{k-1}]$が成り立つとき,これを\textbf{Markov連鎖}という.
        \item $\forall_{k\in[n]}\;\forall_{i\in[r_k]}\;E[X_{n+1}|\A^1\cdots\A^n]=X_n\;\as$が成り立つとき,これを\textbf{martingale}という.
    \end{enumerate}
\end{definition}

\begin{remarks}[確率解析の精神]
    複雑な相互作用のある系を,独立な確率変数の系で等価な表現をすることを
    reductionという.
    そのときに因果性(時間的前後関係)を保存する$\forall_{t\in T}\;\F_t=\F'_t$とき,新たな過程を新生過程(innovation)という.
    加法過程は線型演算だけで新生過程が求められる.
    i.i.d.をそのまま連続化しようとし,可分性の仮定も満たすものは,加法過程の時間微分を持っていれば良い.それがGauss型でもあるとき,これを白色雑音という.
\end{remarks}

\begin{theorem}[Doob]\mbox{}
    \begin{enumerate}
        \item 任意の可分な$(\F_t)$-劣マルチンゲール$X$$\wt{X}$は,第一種の不連続点しか持たない:
        \[P[\forall_{t\in\R_+}\;\wt{X}_{t+0},\wt{X}_{t-0}\text{が存在する}]=1.\]
        なお,任意の劣マルチンゲールに可分変形\ref{thm-Doob-separable-theorem}が取れることに注意.
        \item $(\F_t)$-劣マルチンゲール$X$が確率連続ならば,すなわち,
        \[\forall_{s\in\R_+}\;\lim_{t\searrow s}E[\abs{X_t-X_s}\land 1]=0\]
        が成り立つならば($X$がマルチンゲールならばこれを満たす),$Y$は修正でもある:$\forall_{t\in\R_+}\;X_t=Y_t\;\as$
    \end{enumerate}
\end{theorem}
\begin{Proof}
    与えられた劣マルチンゲールに対する可分変形\ref{thm-Doob-separable-theorem}$\wt{X}$は,
    \[P[\forall_{t\in\R_+}\;\wt{X}_{t+0},\wt{X}_{t-0}\text{が存在する}]=1.\]
    を満たすから,$\wh{X}_t:=\wt{X}_{t+0}$と定めれば良い.
\end{Proof}
\begin{remarks}
    この操作によって何個の点で値が変更されるかは不明であるから,これが修正であるとは限らない.
    が,確率連続性の過程を置くと,修正点は可算個で済み,修正となる.
\end{remarks}

\begin{definition}[$D$-modification]
    同等な$D$-過程は識別不可能であるから,
    このような$(Y_t)$は識別不可能な違いを除いて一意に定まり,\textbf{$D$-変形}という.
\end{definition}

\subsection{もう一つのマルチンゲール正則性定理}

\begin{theorem}[regularity theorem]
    $(X_t)_{t\in\R_+}$を$(\F_t)$-劣マルチンゲールとする.このとき,
    \[P\Square{\forall_{t\in\R^+}\;\lim_{\Q\ni r\nearrow t}X_r<\infty\land\forall_{t\in\R_+}\;\lim_{\Q\ni r\searrow t}X_r<\infty}=1.\]
\end{theorem}

\begin{definition}
    $(X_t)_{t\in\R_+}$を$(\F_t)$-劣マルチンゲールとする.
    \begin{enumerate}
        \item $t\in\R_+$について,$X_{t+}:=\limsup_{\Q\ni r\searrow t}X_r$.
        \item $t\in\R^+$について,$X_{t-}:=\liminf_{\Q\ni r\nearrow t}X_r$.
    \end{enumerate}
\end{definition}

\begin{proposition}
    劣マルチンゲール$(X_t)$について,
    \begin{enumerate}
        \item $\forall_{t\in\R_+}\;E[\abs{X_{t+}}]<\infty$.
        \item $\forall_{t\in\R_+}\;X_t\le E[X_{t+}|\F_t]\;\as$
        \item $t\mapsto E[X_t]$が右連続ならば,特に$X$がマルチンゲールならば,(2)は等式が成り立つ.
        \item $(X_{t+})$は$(\F_{t+})$-劣マルチンゲールである.
    \end{enumerate}
    右極限についても全く同様の事実が成り立つ.
\end{proposition}

\begin{theorem}
    $X$を右連続な$(\F_t)$-劣マルチンゲールとする.このとき,
    \begin{enumerate}
        \item $(\F_{t+})$-劣マルチンゲールでもある.
        \item $X$は殆ど確実に$D$-過程である.
    \end{enumerate}
\end{theorem}

\begin{theorem}
    $X$を$(\F_t)$-劣マルチンゲール,$(\F_t)$は完備で右連続とする.
    さらに$t\mapsto E[X_t]$が右連続ならば($X$がマルチンゲールならばこれを満たす),$(\F_t)$-劣マルチンゲールな$D$-変形を持つ.
\end{theorem}

\section{条件付き期待値}

\begin{tcolorbox}[colframe=ForestGreen, colback=ForestGreen!10!white,breakable,colbacktitle=ForestGreen!40!white,coltitle=black,fonttitle=\bfseries\sffamily,
title=]
    $E_\cG:L^1(\Om)\to L^1_\cG(\Om)$が定まる.
    $L^2(\Om)$上に制限して見ると,任意の$X\in L^2(\Om)$に対して,$\cG<\F$-可測関数のなす部分空間$L_\cG^2(\Om)\subset L_\cG^2(\Om)$への直交射影の値(のバージョン)として得られる$L_\cG^1(\Om)$の元を,条件付き期待値という.
    これは最小二乗の意味での最適推定値であるとも言える.
\end{tcolorbox}
\begin{tcolorbox}[colframe=ForestGreen, colback=ForestGreen!10!white,breakable,colbacktitle=ForestGreen!40!white,coltitle=black,fonttitle=\bfseries\sffamily,
    title=]
    $\cG$の元$B\in\cG$に対して,その立場の上での$X$の期待値$E[X1_B]$を返す符号付き測度を$Q:\cG\to\R$と表そう.
    するとその$P|_\cG$に関する密度関数が$E[X|\cG]$である.
    $E[-|\cG]:L^1(\Om)\to\R$は正な線型汎関数となっている.
\end{tcolorbox}

\subsection{定義と構成}

\begin{definition}[conditional expectation, conditional probability, regular]
    $(\Om,\F,P)$を確率空間とし,
    $\cG$を$\F$の部分$\sigma$-代数とする.可積分確率変数$X\in L^1(\Om)$について,
    \begin{enumerate}
        \item 次の2条件を満たす,$P$-零集合を除いて一意な確率変数を\textbf{条件付き期待値}といい,$E[X|\cG]$で表す.
        \begin{enumerate}[(a)]
            \item $\cG$-可測でもある$P$-可積分確率変数である.
            \item 任意の$\cG$-可測集合$B\in\cG$上では$X$と期待値が同じ確率変数になる:
            $\forall_{B\in\cG}\;E[X1_B]=E[E[X|\cG]1_B]$.\footnote{これは2段階に分けて積分していると見れる.}
        \end{enumerate}
        \item $P[A|\cG]:=E[1_A|\cG]\;(A\in\F)$を\textbf{条件付き確率}というが,確率測度を定めるとは限らない.これが確率測度を定めるとき,\textbf{正則}条件付き確率という.\footnote{完備で可分な距離空間上のBorel確率空間上では存在と一意性が成り立つ.}
    \end{enumerate}
\end{definition}
\begin{remark}[条件付き確率の正則性]\mbox{}
    \begin{enumerate}
        \item $E[X|\cG]$が可積分であるという条件は余分で,$L^1$-ノルム減少性から従う.
        \item 条件付き確率が存在しない場合は,次のような場合に起きる:任意の互いに素な可測集合列$\{F_n\}\subset\F$について,条件付き期待値の線形性と単調収束定理より,
        \[P\Square{\sum F_n\middle|\cG}=E\Square{\sum 1_{F_n}\middle|\cG}=\sum E[1_{F_n}|\cG]=\sum P[F_n|\cG]\;\as\]
        が成り立つが,このときの零集合
        \[\cN:=\Brace{\om\in\Om\;\middle|\;P\Square{\sum F_n\middle|\cG}\ne \sum P[F_n|\cG]}\]
        が,任意の(おそらく非可算無限個ある)互いに素な可測集合列$\{F_n\}\subset\F$について,一様に零集合を取れるとは限らないが,
        「標準確率空間」については気にしなくてよい.
        特に,$\R^\infty$における条件付き確率は必ず正則になるという結果はDoobによる.
    \end{enumerate}
\end{remark}
\begin{remarks}[「相互情報量」としての条件付き期待値]
    $X$が$\cG$と独立なとき,$E[X|\cG]=E[X]\;\ae$である.すなわち,情報$\cG$を得ても,最良の推定は定数$E[X]$でしかない.
    $X$が$\cG$-可測であるとき,$E[X|\cG]=X\;\ae$である.すなわち,既に知っている施策$X$によって完全に模倣・予測できる.
\end{remarks}


\begin{corollary}[存在と一意性]
    任意の可積分確率変数$X\in L^1(\Om)$に対して,条件付き期待値$E[X|\cG]$は存在し,零集合での差を除いて一意である.
\end{corollary}
\begin{Proof}
    条件付き期待値は,$\cG$の元に対して,その立場の上での$X$の期待値を返す測度$Q:\cG\to\R$の,$(\Om,\cG)$上の確率密度関数であると見れば,
    Radon-Nikodymの定理の簡単な系である.
    任意の事象$B\in\cG$に対して,そのときの$X$の条件付き期待値を返す対応
    \[Q(B):=E[1_BX]=\int_BxdP^X.\quad(B\in\cG)\]
    は$(\Om,\cG)$上の測度である.
    これが$P|_{\cG}$に対して絶対連続であることに注意すれば良い:$P[B]=0\Rightarrow Q[B]=0$.
\end{Proof}

\subsection{条件付き確率の存在と特徴付け}

\begin{definition}
    $(X,\B)$を可測空間,$\zeta$がその分割とする.
    \begin{enumerate}
        \item $\S=\{S_i\}_{i\in I}\subset P(X)$が$\zeta$の\textbf{基底}であるとは,$\tau(x)(i)=1_{S_i}(x)$と定めたときの写像$\tau:X\to 2^I$が$X$に定める同値類が$\zeta$に等しいことをいう.
        \item 可算個の可測集合からなる基底$\{S_n\}_{n\in\N}\subset\B$を持つとき,$\zeta$を\textbf{可測分割}という.
    \end{enumerate}
\end{definition}

\begin{definition}[(regular) conditional distribution / probability kernel]
    $(X,\B,m)$をLebesgue空間,$\zeta$をその分割,$(B_n)$をその基底,$(B_n)$の生成する$\sigma$-加法族を$\M$とする.
    確率測度の族$\{m(-|C)\}_{C\in X/\zeta}\subset P(X)$が次の条件を満たすとき,\textbf{$\zeta$が定める条件付き測度}または\textbf{確率核}という.
    \begin{enumerate}
        \item $\forall_{C\in X/\zeta}\;m(\pi^{-1}(C)|C)=1$.
        \item 任意の$C\in X/\zeta$について,$\M\cap\pi^{-1}(C)$の$m(-|C)$に関する完備化を$\M_C$とすると,$(\pi^{-1}(C),\M_C,m(-|C))$はLebesgue空間となる.
        \item 任意の$B\in\B$について,次の3条件が成り立つ:
        \begin{enumerate}[(a)]
            \item $m_\zeta$-a.e.の$C$について,$B\cap\pi^{-1}(C)\in\M_C$.
            \item $m(B|-):X/\zeta\to[0,1]$は$\B_\zeta$-可測である.
            \item $\forall_{Z\in\B_\zeta}\;\int_Zm(B|C)dm_\zeta(C)=m(B\cap\pi^{-1}(Z))$.
        \end{enumerate}
    \end{enumerate}
\end{definition}

\begin{proposition}[一意性]
    Lebesgue空間の任意の分割に関する条件付き測度は,$m_\zeta$-a.e.の違いを除いて一意的である.
\end{proposition}

\begin{proposition}[存在]
    $X$をLebesgue空間とする.その分割$\zeta$について,次の2条件は同値.
    \begin{enumerate}
        \item 条件付き測度が存在する.
        \item $\zeta$は可測分割である.
    \end{enumerate}
\end{proposition}

\begin{proposition}[特徴付け]
    $P[A|\cG]\in L^1_\cG(\Om)$はa.s.の違いを除いて,次の条件を満たす唯一の元である.
    \[\forall_{B\in\cG}\;E[P[A|\cG]1_B]=P[A\cap B].\]
    特に,
    \begin{enumerate}
        \item $P[A|\cG]=P[A]\;\as$は$A\perp\cG$に同値.
        \item $P[A|\cG]=1_A\;\as$は$A\in\o{\cG}$に同値.
    \end{enumerate}
\end{proposition}

\subsection{条件付き期待値の特徴付け}

\begin{tcolorbox}[colframe=ForestGreen, colback=ForestGreen!10!white,breakable,colbacktitle=ForestGreen!40!white,coltitle=black,fonttitle=\bfseries\sffamily,
title=]
    $L^2(\Om)$上の射影としての性質を$L^1(\Om)$上に連続延長したものと見れる.
\end{tcolorbox}

\begin{corollary}[Jensenの不等式の系]
    条件付き期待値$E[-|\B]:L^p\to L^p$は任意の$p\in[1,\infty]$についてノルム減少的で,像は$E[L^p|\B]= L^p_\B(\Om)$とあらわせる.
\end{corollary}
\begin{Proof}
    凸関数$f(x)=\abs{x}^p$について,$\abs{E_\B[X]}^p\le E_\B[\abs{X}^p]$より,
    \[E[\abs{E_\B[X]}^p]\le E[E_\B[\abs{X}^p]]=E[\abs{X}^p]<\infty\]
    だから,たしかに像は$L^p$に入り,$\norm{E_\B[X]}_p\le\norm{X}_p$である.
    全射性は,任意の$X\in\L_\B^p$について,$E_\B[X]=X$より.
\end{Proof}

\begin{theorem}[射影としての特徴付け]
    任意の部分$\sigma$-代数$\cG$について,
    \begin{enumerate}
        \item $E^\cG:L^2\to L^2$はHilbert空間$L^2$からその閉部分空間$L_\cG^2$への射影作用素と一致する.すなわち,$E_\cG[X]$は$\norm{X-\wt{X}}_2$を最小にする$\wt{X}\in L_\cG^2$として特徴付けられる.
        \item 唯一の線型作用素$E_\cG:L^1(\Om)\to L^1_\cG(\Om)$が存在して,$\forall_{X\in L^1}\;\forall_{A\in\cG}\;E[E_\cG[X]1_A]=E[X1_A]$を満たす.
    \end{enumerate}
\end{theorem}

\begin{theorem}[射影としての性質:自己共役性]\mbox{}
    \begin{enumerate}
        \item $E[-|\cG]:L^2(\Om)\to L^2_\cG(\Om)$は自己共役である:$E[XE[Y|\cG]]=E[E[X|\cG]Y]$.
        \item この性質は$L^1(\Om)$への連続延長でも保たれる.
    \end{enumerate}
\end{theorem}

\subsection{作用素としての性質}

\begin{proposition}
    $(\Om,\F,P)$を確率空間,$X\in L^1(\Om)$を可積分確率変数,$\cG,\H$を$\F$の$\sigma$-部分代数とする.
    \begin{enumerate}
        \item (一意性) $Y$も条件付き期待値の定義を満たすとする.このとき,$E[Y]=E[X]$.
        \item ($L_\cG$上定値) $X$が$\cG$-可測であったならば,$E[X|\cG]=X\;\as$
        \item (三角不等式) $\abs{E[X|\cG]}\le E[\abs{X}|\cG]\;\as$
        \item (線型性) $E[-|\cG]$は$L^1(\Om)$上の線型汎関数である:$E[a_1X_1+a_2X_2|\cG]=a_1E[X_1|\cG]+a_2E[X_2|\cG]\;\as$
        \item (正性) $X\ge0\Rightarrow E[X|\cG]\ge0$.
        \item (単調収束定理) $0\le X_n\nearrow X\Rightarrow E[X_n|\cG]\nearrow E[X|\cG];\as$
        \item (Fatouの補題) $X_n\ge0\Rightarrow E[\liminf X_n|\cG]\le\liminf E[X_n|\cG]\;\as$
        \item (強い優収束定理) $\forall_{n\in\N}\;\abs{X_n}\in L^1(\Om)$かつ$X_n\xrightarrow{\as}X$ならば,$E[X_n|\cG]\xrightarrow{\as}E[X|\cG]$かつ$L^1$でも収束する.
        \item (Jensen) 凸関数$c:\R\to\R$に対して,$c(E[X|\cG])\le E[c(X)|\cG]\;\as$ 特に,$\norm{-}_p\;(p\ge1)$は凸関数であるから$\norm{E[X|\cG]}_p\le\norm{X}_p$.
        \item (Tower property) $\H<\cG\Rightarrow E[E[X|\cG]|\H]=E[E[X|\H]|\cG]=E[X|\H]\;\as$
        \item (可測関数) $Z\in L(\Om,\cG),ZX\in L^1(\Om)$のとき,$E[ZX|\cG]=ZE[X|\cG]\;\as$
    \end{enumerate}
\end{proposition}
\begin{Proof}\mbox{}
    \begin{enumerate}
        \item 条件付き期待値の一意性より,$Y=E[X|\cG]\;\as$.任意の$G\in\cG$について,条件付き期待値$E[X|\cG]$は$G$上では$X$と平均が等しいから,
        $E[Y1_G]=E[E[X|\cG]1_G]=E[X1_G]$が成り立つ.$G=\Om$と取れば良い.
        \item $X$は自身の条件付き期待値としての要件を満たすから,一意性より.
        \item $-\abs{x}\le x\le\abs{x}$と正性より,$-E[\abs{X}|\cG]\le E[X|\cG]\le E[\abs{X}|\cG]$.
        \item 右辺の$a_1E[X_1|\cG]+a_2E[X_2|\cG]$も,$a_1X_1+a_2X_2$の$B\in\cG$上での期待値を与える測度$Q$の確率密度関数となっている.
    \end{enumerate}
\end{Proof}

\begin{theorem}[Th'm 8.4 \cite{Doob}]
    $Y\in L^1(\Om),\delta>0$とすると,
    \[X_\delta:=\sum_{j\in\Z}(j+1)\delta P[j\delta<Y\le(j+1)\delta|\cG]\]
    は殆ど確実に絶対収束し,$(\delta_k)$が$0$に収束するならば,$\lim_{k\to\infty}X_{\delta_k}=E[Y|\cG]\;\as$
\end{theorem}


\subsection{独立性との関係}

\begin{proposition}
    $(\Om,\F,P)$を確率空間,$X\in L^1(\Om)$を可積分確率変数,$\cG,\H$を$\F$の$\sigma$-部分代数とする.
    \begin{enumerate}
        \item $\H\indep(\sigma[X]\lor\cG)$ならば,$E[X|\cG\lor\H]=E[X|\cG]\;\as$ 特に,$X\indep\H\Rightarrow E[X|\H]=E[X]\;\as$
        \item $X\indep(\sigma[Y]\lor\cG)$ならば,$E[XY|\cG]=E[X]E[Y|\cG]\;\ae$
        \item $X\indep\cG$かつ$X\indep Y$ならば,上は必ずしも成り立たない.
    \end{enumerate}
\end{proposition}

\begin{lemma}
    $\cG,\H$を$\F$の部分$\sigma$-代数,$X,Y\in L^1(\Om),A\in\cG\cap\H$とする.このとき,
    \[A\cap\cG=A\cap\F\land X=Y\;\as\;\on A\quad\Rightarrow\quad E[X|\cG]=E[Y|\H]\;\as\;\on A.\]
\end{lemma}

\begin{proposition}[Doob]
    $X\indep Y|Z$とする.このとき,$P[X\in B|Y,Z]=P[X\in B|Z]$.
\end{proposition}

\begin{theorem}[条件付き独立]
    任意の部分$\sigma$-代数$\F,\cG,\H$について,次の2条件は同値:
    \begin{enumerate}
        \item $\F\underset{\cG}{\indep}\H$.
        \item $P^{\F\lor\cG}=P^\cG\;\as\;\on\H$.
    \end{enumerate}
    ただし,$P^\F[A]:=E^\F[1_A]$とした.
\end{theorem}
\begin{remarks}
    
\end{remarks}

\subsection{条件付き独立性}

\begin{tcolorbox}[colframe=ForestGreen, colback=ForestGreen!10!white,breakable,colbacktitle=ForestGreen!40!white,coltitle=black,fonttitle=\bfseries\sffamily,
title=]
    条件付き独立性が難しい理由は,射影作用素$E_\cG:L^1(\Om)\to L^1_\cG(\Om)$が,$X$の定める$\sigma$-代数$\sigma[X]$をどのように変えて$\sigma[E[X|\cG]]$とするかが見えにくいことに起因する.
\end{tcolorbox}

\begin{definition}
    $\B_1\indep\B_2|\A$とは,
    \[\forall_{B_1\in\B_1,B_2\in\B_2}\;P[B_1\cap B_2|\A]=P[B_1|\A]P[B_2|\A]\]
    と定める.
\end{definition}

\begin{theorem}
    次の2条件は同値.
    \begin{enumerate}
        \item $\B_1\indep\B_2|\A$.
        \item $\forall_{B_1\in\B_1}\;P[B_1|\A\lor\B_2]=P[B_1|\A]$.
    \end{enumerate}
\end{theorem}

\subsection{束上の関数としての性質}

\begin{tcolorbox}[colframe=ForestGreen, colback=ForestGreen!10!white,breakable,colbacktitle=ForestGreen!40!white,coltitle=black,fonttitle=\bfseries\sffamily,
title=]
    第二引数についてマルチンゲールをなす.
\end{tcolorbox}

\begin{lemma}
    部分$\sigma$-代数$\cG,\H$について,次の2条件は同値:
    \begin{enumerate}
        \item $\forall_{X\in L^1(\Om)}\;E[X|\cG]=E[X|\H]$.
        \item $\o{\cG}=\o{\H}$.
    \end{enumerate}
\end{lemma}

\begin{theorem}[条件付き期待値の束の一様可積分性 (Doob)]
    $\Lambda$を$\F$の部分$\sigma$-加法族の全体とする.任意の$X\in L^1(\Om)$に対して,$\{E[X|\cG]\}_{\cG\in\Lambda}\subset L^1(\Om)$は一様可積分である.
\end{theorem}
\begin{Proof}
    一様可積分性の特徴付けより,ある単調増加凸関数$g:\R_+\to\R^+$であって$\lim_{t\to\infty}\frac{G(t)}{t}=\infty$を満たすものについて,$\sup_{\cG}E[g\circ E[Y|\cG]]<\infty$を示せば良い.
    $X$は非負としても一般性を失わず,Jensenの不等式から$g\circ E[Y|\cG]\le E[g\circ Y|\cG]$が従う.
\end{Proof}
\begin{remarks}
    この事実により,一様可積分なマルチンゲール$M$が$L^1(\Om)$に同型であることが,時間$T$の順序構造に依らないことが判る(Meyer\cite{Meyer}).
    一方で,概収束$\lim_{n\to\infty}X_n=X_\infty\;\as$は$T$が全順序でないならば失敗し得る.
\end{remarks}

\begin{corollary}[連続性]
    部分$\sigma$-代数の列$\{\cG_n\}_{n\in\N}$が,単調増大で$\cG$に至るか,単調減少で$\cG$に至るならば,$\forall_{f\in L^1(\Om)}\;E[f|\cG_n]\to E[f|\cG]$がa.s.の意味と$L^1$の意味で成り立つ.
    特に,$P[A|\cG_n]\to P[A|\cG]$.
\end{corollary}

\begin{theorem}[最大不等式]
    $X\in L^1(\Om),\{\cG_n\}$を増大列とする.
    \[\forall_{\lambda>0}\;P\Square{\sup_{n\ge 1}E[\abs{X}|\cG_n]>\lambda}\le\frac{E[\abs{X}]}{\lambda}.\]
\end{theorem}


\section{情報系と確率過程}

\begin{tcolorbox}[colframe=ForestGreen, colback=ForestGreen!10!white,breakable,colbacktitle=ForestGreen!40!white,coltitle=black,fonttitle=\bfseries\sffamily,
title=]
    情報は$\sigma$-部分代数で,データは確率変数で表すとしたら,2つの構造が何らかの意味で整合して居る必要がある.これを適合的という.
    さらに空間$\Om$だけでなく時空間$\Om\times\R_+$に目を向け,この空間の部分集合を運命という.
\end{tcolorbox}

\subsection{閉$\sigma$-代数と情報理論}

\begin{tcolorbox}[colframe=ForestGreen, colback=ForestGreen!10!white,breakable,colbacktitle=ForestGreen!40!white,coltitle=black,fonttitle=\bfseries\sffamily,
title=]
    任意の閉$\sigma$-代数$\cG\subset\F$はある実確率変数が生成する.
    すなわち,観測の結果得られる情報を表すのが閉$\sigma$-代数の概念だと考えられる.
\end{tcolorbox}

\begin{definition}\mbox{}
    \begin{enumerate}
        \item $(\Om,\F,P)$上の部分$\sigma$-代数であって,すべての$P$-零集合を含むものを\textbf{閉$\sigma$-代数}または\textbf{情報}という.
        \item その全体を$\Phi=\Phi(\Om,\F,P)=\Brace{\B\lor2\subset\F\mid\B<\F}\subset\F$で表す.
        \item 確率変数$X$に対して,これが$\Om$上に定める分割が生成する最小の閉$\sigma$-代数を\textbf{$X$で生成される閉$\sigma$-代数}といい,$\sigma[X]\lor 2=:\F[X]<\F$で表すこととしよう.
        \item 確率変数の族$\{X_\lambda\}_{\lambda\in\Lambda}$については,$\F[X_\lambda,\lambda\in\Lambda]:=\bigvee_{\lambda\in\Lambda}\F[X_\lambda]$と表す.
    \end{enumerate}
\end{definition}

\begin{theorem}
    $(\Om,\F,P)$を確率空間とする.
    任意の閉$\sigma$-代数$\cG\in\Phi$は,ある実確率変数$X\in\L(\Om)$によって生成される:$\cG=\F[X]$.
\end{theorem}

\begin{proposition}
    $X,Y\in\L(\Om)$について,$X\prec Y\;\as:\Leftrightarrow [\exists_{\varphi\in\Map(\Om,\Om)}\;X=\varphi\circ Y\;\as]$と表すと,これは同値類$\sim$とその上の順序を定め,
    \begin{enumerate}
        \item $Y\prec X\;\as\Leftrightarrow\F[Y]\subset\F[X]$.
        \item $Y\sim X\;\as\Leftrightarrow\F[Y]=\F[X]$.
    \end{enumerate}
\end{proposition}
\begin{remarks}
    $Y\prec X$とは,$Y$の与える情報が$X$よりも少ないこと,$X$の値がわかれば確率1で$Y$の値がわかることを意味する.
\end{remarks}

\subsection{増大情報系}

\begin{tcolorbox}[colframe=ForestGreen, colback=ForestGreen!10!white,breakable,colbacktitle=ForestGreen!40!white,coltitle=black,fonttitle=\bfseries\sffamily,
title=]
    情報系$(\F[X_t])_{t\in T}$について,過去の記憶を取り$(\F[X_s;s\le t])_{t\in T}$とすれば単調増大になり,さらに$\paren{\F_t:=\bigcap_{s>t}\F[X_u;u\le s]}_{t\in T}$とすれば右連続にもなるから,特に意識せず情報系(filtration)と呼ぶこととする.
    また,任意の情報系は,ある実過程が生成することに注意.
\end{tcolorbox}

\begin{definition}[filtration]\mbox{}
    \begin{enumerate}
        \item $T$に関する\textbf{($\sigma$-代数の)増大系}と言ったときは,単なる部分$\sigma$-代数の増大列とする:$\forall_{s,t\in T}\;s\le t\Rightarrow\F_s\subset\F_t$.
        \item $T$に関する\textbf{増大情報系}$\{\F_t\}_{t\in T}\subset\Phi(\Om,\F,P)$とは,
        \begin{enumerate}[(a)]
            \item 各$\F_t$は閉じている:$\N\subset\F_t$.
            \item 右連続性:$\F_t=\F_{t+}:=\bigvee_{s>t}\F_s$.
        \end{enumerate}
        を満たす閉$\sigma$-代数の族をいう.
        \item 任意の単調増大性を満たす系$\{\F_t\}\subset\Phi$に対して,$(\F_{t+})_{t\in T}$は情報系である.これを\textbf{右連続化}という.
        \item 確率過程$(X_t)$に対して,$\paren{\F_t:=\bigcap_{s>t}\F[X_u;u\le s]}$を\textbf{$X$が生成する情報系}または\textbf{自然な情報系}といい,$\F[X]:=(\F_t[X])_{t\in T}$で表す.
        \item 任意の過程はその自然な情報系に適合している.
    \end{enumerate}
\end{definition}

\begin{definition}[stochastic basis]\mbox{}
    \begin{enumerate}
        \item 完備な確率空間$(\Om,\F,P)$と増大情報系$(\F_t)_{t\in\R_+}$の組を\textbf{確率基底}という\cite{Liptser-Shiryaev-Statistics}.
        \item 確率基底$(\Om,\F,P,(\F_t))$が\textbf{完備}とは,$\F,\F_t$が全て$\cN$を含むこと(条件(a))をいう.
        情報系の完備性(a)と右連続性(b)と確率空間の完備性とを併せて\textbf{usual condition}といい,確率論による過程論の基礎となる.
        が,usual conditionを満たさない状況も肝要になる場面もある.
    \end{enumerate}
\end{definition}

\begin{proposition}[$D$-過程が生成する情報系の特徴付け]
    $D$-過程$(X_t)_{t\in T}$について,
    これが確率変数の族として生成する情報系$\F[X_t,t\in T]=\bigvee_{t\in T}\F[X_t]$と,
    $D$-値確率変数$X_\bullet:\Om\to D(T)$として生成する情報系とは等しい:
    $\F[X_t,t\in T]=\F[X_\bullet]=:\F_\infty$.
\end{proposition}

\begin{remark}[増大情報系の哲学]
    情報系を定値に取ると,19世紀の決定論的世界観になる(Dellacherie\cite{Dellacherie-Meyer}は決定論的情報系と呼んでいる).
    微分系を積分することで,遥か未来の情報も現在の情報に含まれているのだ.
\end{remark}

\subsection{適合過程}

\begin{proposition}
    $(\F_t)$は完備とする.
    \begin{enumerate}
        \item 適合過程の任意の意味での極限は適合的である.
        \item 適合過程と識別不可能な過程は適合的である.
    \end{enumerate}
\end{proposition}

\subsection{発展的可測性}

\begin{tcolorbox}[colframe=ForestGreen, colback=ForestGreen!10!white,breakable,colbacktitle=ForestGreen!40!white,coltitle=black,fonttitle=\bfseries\sffamily,
title=]
    可測過程の概念を適合過程の概念を参考にして強めると,
    発展的可測性$L(\P)\subset L(\Om\times\R_+)$に辿り着く.
    この概念は,可測過程$X$がその可測性を達成する過程についても口出ししている点で強い概念であることがわかる.
    この上に確率積分が定義できる.また,ほとんど$L(\F\otimes\B(\R_+))\cap\bF\subset L(\P)$である.
    より詳しく見ると,$w^*$-可測過程と$p^*$-可測過程がある.
\end{tcolorbox}

\begin{definition}[adapted, predictable, progressively measurable]\mbox{}
    \begin{enumerate}
        \item $\forall_{t\in T}\;X_t\in\L(\F_t)$のとき,$X$は$F$に\textbf{適合}するという.これは$\forall_{t\in T}\;\F[X_t]\subset\F_t$に同値.
        \item $\forall_{t\in T}\;X_t\in\L(\F_{t-1})$のとき,$X$は$F$で\textbf{可予測}であるという.$X_t=E[X|\F_{t-1}]$より,右辺から計算可能になる.
        \item 適合過程$X\in(\F_t)$が\textbf{発展的可測}であるとは,$\forall_{t\in\R_+}\;X|_{\Om\times[0,t]}\in L(\Om\times\R_+,\F_t\otimes\B([0,t]))$が成り立つことをいう.
    \end{enumerate}
\end{definition}

\begin{proposition}[発展的可測性の必要条件]
    発展的可測過程は適合的である.
\end{proposition}
\begin{Proof}
    任意の$t\in\R_+$について,$X|_{\Om\times[0,t]}:\Om\times\R_+\to\R$は$\F_t\otimes\B([0,t])$-可測である.
    Fubiniの定理から,二変数の可測関数の片方を止めた関数は可測であるから,$X_t(-)\in L_{\F_t}(\Om)$.
\end{Proof}

\begin{lemma}[発展的可測性の十分条件]\mbox{}
    \begin{enumerate}
        \item 適合過程$X\in\bF$が任意の$\ep>0$に対して,情報系$(\F_{t+\ep})_{t\in\R_+}$について発展的可測ならば,元の$\bF$についても発展的可測である.
        \item 一般の過程$X$が任意の$\ep>0$に対して,情報系$(\F_{t+\ep})_{t\in\R_+}$について発展的可測ならば,元の$\bF$の右連続化$(\F_{t+})_{t\in\R_+}$についても発展的可測である.
    \end{enumerate}
\end{lemma}
\begin{Proof}
    \[X_s=\lim_{\ep\to0}X_s1_{[0,t-\ep)}(s)+X_t1_{\Brace{t}}(s)\quad(s\le t)\]
    と見ると,$X_s1_{[0,t-\ep)}$は明らかに$\F_t\times\B([0,t])$-可測で,$X$の適合性より$X_t1_{\Brace{t}}(s)$も$\F_t\times\B([0,t])$-可測.
    $X$が適合でないとしても,上式の右辺は$\F_{t+}$について同様のことが言える.
\end{Proof}

\begin{corollary}[発展的可測過程の例]
    状態空間$E$は距離化可能とする.
    \begin{enumerate}
        \item $D\cap\bF$に属する過程は発展的可測である.\footnote{Liptser and Shiryayev\cite{Liptser-Shiryaev-Statistics} Problem1}
        \item 全く同様にして,左連続な過程は発展的可測である.
        \item 適合的な可測過程には,発展的可測な修正が存在する.\footnote{Meyer 1984, Th'm 4.6}
    \end{enumerate}
\end{corollary}
\begin{Proof}\mbox{}
    \begin{enumerate}
        \item 任意の$n\in\N$について
        \[X^{(n)}_t:=X_{\frac{k+1}{2^n}}\quad t\COinterval{\frac{k}{2^n},\frac{k+1}{2^n}}\]
        と定めると,任意の$\ep>2^{-n}$について$(\F_{t+\ep})$-発展的可測である.
        右連続性の仮定より,$X=\lim_{n\to\infty}X^{(n)}$は任意の$\ep>0$について$(\F_{t+\ep})$-発展的可測.よって,$X$は$\bF$-発展的可測である.
        \item 同様.
    \end{enumerate}
\end{Proof}

\begin{theorem}[発展的可測性の$\sigma$-代数による特徴付け]
    \[\P:=\Brace{A\in P(\Om\times\R_+)\mid\forall_{t\in\R_+} A\cap(\Om\times[0,t])\in\F_t\times\B([0,t])}\]
    はたしかに$\F\times\B(\R_+)$の部分$\sigma$-代数で,
    可測過程$u:\Om\times\R_+\to\R$について,次の2条件は同値.
    \begin{enumerate}
        \item $\P/\B(\R)$-可測:$u\in\L(\P)$.
        \item $u$は発展的可測.
    \end{enumerate}
\end{theorem}
\begin{Proof}\mbox{}
    \begin{description}
        \item[$\P$のwell-definedness] 明らかに,$\forall_{t\in\R_+} A\cap(\Om\times[0,t])\in\F_t\times\B([0,t])$という条件は,$A=\emptyset,\Om\times\R_+$はこれを満たし,これを満たす$(A_n)$が存在したとき,合併も満たす.
        また,補集合については,$\F_t\times\B([0,t])$が$\Om\times[0,t]$上の$\sigma$-代数をなすことに注意すると,
        $A\cap(\Om\times[0,t])\in\F_t\times\B([0,t])$のとき,
        $\o{A}\cap(\Om\times[0,t])=(\Om\times[0,t])\setminus (A\cap(\Om\times[0,t]))\in\F_t\times\B([0,t])$による.
        \item[$\P\subset\F\times\B(\R_+)$] 
        任意の$A\in\P$を取ると,特に$n\in\N$について,$A_n:=A\cap(\Om\times[0,n])\in\F_n\times\B([0,t])\subset\F\times\B(\R_+)$.
        $A=\cup_{n\in\N}A_n\in\F\times\B(\R_+)$.
        \item[(1)$\Leftrightarrow$(2)] $\P/\B(\R)$-可測過程$u:\Om\times\R_+\to\R$を取る.任意の$t\in\R_+$について,$u|_{\Om\times[0,t]}$によるBorel可測集合$B\in\B(\R)$の逆像は$u^{-1}(B)\cap(\Om\times[0,t])$であるが,$u^{-1}(B)\in\P$より,これは$\F_t\times\B([0,t])$の元である.
    \end{description}
\end{Proof}

\subsection{$w^*$-可測過程}

\begin{tcolorbox}[colframe=ForestGreen, colback=ForestGreen!10!white,breakable,colbacktitle=ForestGreen!40!white,coltitle=black,fonttitle=\bfseries\sffamily,
title=]
    運命の空間$\Om\times\R_+$の中で,特殊な$\sigma$-代数に注目する必要がある,それは適合的な$C,D$-過程が生成するものである.
    このとき,\[\P\subset\W\subset\P\subset\F_\infty\otimes\B(\R_+)\]が成り立つ.
\end{tcolorbox}

\begin{definition}[optional / well measurable process, predictable process]
    $(\Om,\F,P,\bF)$を確率基底とする.
    特に,$\bF$は右連続で完備とする.
    \begin{enumerate}
        \item $\W:=\sigma[[0,\tau),\tau\in\T]$を\textbf{随意$\sigma$-代数},その元を\textbf{随意集合}という.
        \item $\P:=\sigma[(0,\tau],\tau\in\T]$を\textbf{可予測$\sigma$-代数},その元を\textbf{可予測集合}という.
        \item $\W$-可測な過程$X:\Om\times\R_+\to\R$を\textbf{随意過程}または\textbf{任意停止過程},$\P$-可測な過程を\textbf{可予測過程}という.
    \end{enumerate}
    明らかに,$w$-可測過程も$p$-可測過程も,適合的な可測過程である.
\end{definition}

\begin{theorem}[$\sigma$-代数の特徴付け]
    $(\F_t)$を増大情報系とする.
    \begin{enumerate}
        \item $\W$は,$\bF\cap D$が生成する$\sigma$-代数に等しい.
        特に,任意の$D\cap\bF$過程は随意過程である.
        \item また,$\cointerval{a,b}\times C\;(a<b,C\in\F_a)$という形の集合が生成する$\sigma$-代数も$\W$に一致する.
        \item $\P$は,$\bF\cap C$が生成する$\sigma$-代数に等しい.
        \item また,左連続な適合過程の全体が生成する$\sigma$-代数も$\P$であり,単過程の全体が生成する$\sigma$-代数も$\P$であり,$\cointerval{b,c}\times C\;(a<b<c,C\in\F_a)$と表せる集合の生成する$\sigma$-代数も$\P$である.
    \end{enumerate}
\end{theorem}

\begin{corollary}
    可予測過程について,
    \begin{enumerate}
        \item $\P\subset\W$が成りたち,可予測ならば$w$-可測である.
        \item 可予測ならば$(\F_{t-})$-適合的である.特に適合過程である.
        \item 可予測過程は$X_0$の値を変えても可予測である.すなわち,$(0,\infty)$上で定まっているとみなせる.
    \end{enumerate}
\end{corollary}
\begin{Proof}
    任意の適合的な左連続過程$X$を取る.
    これに対して,
    \[X_t^{(n)}:=X_{\frac{k}{2^n}},\quad t\in\COinterval{\frac{k}{2^n},\frac{k+1}{2^n}}\]
    と取ると,$(X^{(n)})$は右連続な適合的過程の列で,$\forall_{t\in\R_+}\;X_t^{(n)}\to X_t$を満たす.
    よって,$X$は$\W$-可測である.
\end{Proof}

\begin{corollary}
    $w^*$-可測過程について,
    \begin{enumerate}
        \item $w$-可測過程$X\in L(\W)$は発展的可測である.
        \item $X\in D\cap(\F_t)$のとき,$X_-$は可予測過程である.
        \item $\forall_{\tau\in\T}\;[[\tau]]\in\D$であるが,$\fP$ではそうとは限らない.
        \item 任意のMarkov時刻$T\in\bT$について,$X_T1_{\Brace{T<\infty}}$は$\F_T$-可測である.
        \item 停止過程$X^T$は再び随意過程である.
    \end{enumerate}
\end{corollary}

\section{停止時}

\begin{tcolorbox}[colframe=ForestGreen, colback=ForestGreen!10!white,breakable,colbacktitle=ForestGreen!40!white,coltitle=black,fonttitle=\bfseries\sffamily,
title=]
「微分」の定義に解析の芽が詰まっており,この定義に辿り着くのに多くの天才を要したのと同様,停止時の概念も確率論の芽が詰まっている(Dellacherie\cite{Dellacherie-Meyer-A}).
\end{tcolorbox}

\subsection{定義と特徴付け}

\begin{tcolorbox}[colframe=ForestGreen, colback=ForestGreen!10!white,breakable,colbacktitle=ForestGreen!40!white,coltitle=black,fonttitle=\bfseries\sffamily,
title=]
    情報系は右連続としているために,停止時に数々の特徴付けが存在する.
\end{tcolorbox}

\begin{definition}[random time, Markov time, stopping time]
    $(\Om,\F,P,\bF)$を確率基底とする.
    \begin{enumerate}
        \item 確率変数$\tau\in L(\Om;\o{\R}_+)$を\textbf{確率時刻}という.
        \item $\forall_{t\in\R_+}\;\Brace{\tau\le t}\in\F_t$を満たす確率時刻を,\textbf{$\bF$-Markov時刻}という.Markov時刻の全体を$\T(\Om,\F,(\F_t),P)=\T((\F_t))$で表す.
        \item さらに$P[\tau<\infty]=1$を満たすとき,\textbf{$\bF$-停止時}という.
    \end{enumerate}
\end{definition}

\begin{lemma}[停止時の特徴付け]
    情報系$\bF$について,$\bF$-
    Markov時刻$\tau\in\bT(\bF)$について,次は同値:
    \begin{enumerate}
        \item $\tau$はMarkov時刻である.
        \item $\forall_{t\ge0}\;\Brace{\tau<t}\in\F_t$.
    \end{enumerate}
    さらに,$\tau$が離散である場合,次も同値になる.
    \begin{enumerate}\setcounter{enumi}{6}
        \item $\forall_{n\in\N}\;\{\tau=n\}\in\F_n$である.
    \end{enumerate}
\end{lemma}
\begin{remark}\mbox{}
    \begin{enumerate}
        \item 一般の増大系$(\F_t^0)$に関して,$\Brace{T<t}\in\F_t^0$は$\Brace{T\le t}\in \F_{t+}^0$に同値であり,
        これがMarkov時刻の特徴付けになるのは増大系$(\F_t^0)$が右連続な場合で,その場合に限る.
        \item 離散の場合と連続の場合で異なる点は,
        $\forall_{t\ge0}\;\Brace{\tau=t}\in\F_t$だけでは,
        $\Brace{\tau\ge t}$という形の事象を加算和で表現できるとは限らない点である.
    \end{enumerate}
\end{remark}

\begin{lemma}
    $(\Om,\F,P,\bF)$を確率基底とする.次は同値:
    \begin{enumerate}
        \item $\tau\in\bT$.
        \item $1_{(0,\tau]}$は可予測過程である.
        \item $1_{[0,\tau)}$は$w$-可測過程である.
    \end{enumerate}
\end{lemma}

\subsection{停止時の例とデビューによる特徴付け}

\begin{tcolorbox}[colframe=ForestGreen, colback=ForestGreen!10!white,breakable,colbacktitle=ForestGreen!40!white,coltitle=black,fonttitle=\bfseries\sffamily,
title=]
    Markov時刻$\Om\to[0,\infty]$は$[0,t]$の逆像が$\F_t$-可測,という発展的条件で定義された.
    これは$\Om\times\R_+$上の「発展的可測な集合のデビュー」として特徴付けられる.
    すなわち,あらゆるMarkov時刻は「運命との初邂逅」と理解できる.
\end{tcolorbox}

\begin{definition}[random set, evanescent / $P$-negligible, thin set, exhausting, debut]\mbox{}
    \begin{enumerate}
        \item 部分集合$A\subset\Om\times\R_+$であって,
        任意の切片$A_t=\Brace{\om\in\Om\mid(\om,t)\in A}\;(t\in\R_+)$が$\F$-可測であるものを\textbf{確率集合}または\textbf{運命過程}という.
        \item すなわち,ある部分集合$A\subset\Om\times\R_+$を用いて$(1_A(t,\om))_{t\in\R_+}$と表せるような,$2$を状態空間とする過程を確率集合という.
        \item 確率集合$A\subset\Om\times\R_+$の$\Om$への射影$\pi_1(A)$(第一成分の値域)が零集合ならば,\textbf{消えゆく運命}という.
        \item すなわち,過程$(1_A(t,\om))_{t\in\R_+}$が$0$と識別不可能であるとき,消えゆく運命であるという.
        \item ある確率時刻の列$\{\tau_n\}\subset L(\Om;[0,\infty])$によって$A=\cup_{n\in\N}[[\tau_n]]$と与えられる運命を\textbf{薄い集合}という.このとき$\forall_{i\ne j}\;[[\tau_i]]\cap[[\tau_j]]=\emptyset$ならば,$A$を\textbf{埋め尽くす}という.
        \item ランダム集合$A\subset\Om\times\R_+$に対して$D_A:=\inf_{(\om,t)\in A}\pr_2((\om,t))$を\textbf{$A$のデビュー}という.
    \end{enumerate}
\end{definition}

\begin{theorem}[デビューによるMarkov時刻の特徴付け]
    $(\Om,\F,P,\bF)$を確率基底とする.すなわち,$\bF$は右連続で完備であるとする.
    \begin{enumerate}
        \item 集合$A\subset\Om\times\R_+$を発展的可測とする.$A$のデビュー$D_A$はMarkov時刻である.
        \item 任意のMarkov時刻$\tau\in\T$について,$A:=\cointerval{\tau,\infty}$とすれば,$\tau=D_A$である.$A$としてグラフをとってもよい:$\tau=D_{[\tau]}$でもある.
    \end{enumerate}
\end{theorem}
\begin{Proof}
    Dellacherie and Meyer 1978 \cite{Dellacherie-Meyer}.
\end{Proof}

\begin{corollary}[entry time, hitting time]
    $S$を可分距離空間,
    $X$を$S$-値の$w$-可測過程,$B\in\B(S)$とする.このとき,次は停止時である:
    \begin{enumerate}
        \item $B$への侵入時刻$U_B:=\inf\Brace{t\in\R_+\mid X_t\in B}$は停止時である.
        \item $B$への到達時刻$T_B:=\inf\Brace{t\in\R^+\mid X_t\in B}$は停止時である.
    \end{enumerate}
\end{corollary}

\begin{example}[停止時の例]\label{exp-discrete-Markov-time}\mbox{}
    \begin{enumerate}
        \item $X\in\bF\cap C$の集合$A\subset\R^d$への\textbf{到達時刻}は
        \[\tau_A(\om):=\inf\Brace{t>0\mid X_t(\om)\in A}\]
        で定める.$A$が開または閉であるとき,$\tau_A$は$\bF$-Markov時になる.
        \item 一方で,$X\in\bF\cap D$の場合,$A$が開集合でも,到達時刻$\tau_A$は$(\F_{t+})$についてしかMarkov時にならない.
        \item \textbf{最終脱出時刻}(last exit time)とは,$(\F_n)$-適合確率過程$(X_n)$対して,任意の事象$A\in\B(\R)$に対し,
        \[\sigma_A(\om):=\max\Brace{n\in\o{\N}\mid X_n(\om)\in A}+1\]
        とすると,これは確率変数ではあるが,Markov時刻にはならない.「これが最後か?」を判定するには,さらに先の情報が必要だからである.
    \end{enumerate}
\end{example}

\subsection{停止時の構成}

\begin{lemma}[Markov時刻の任意停止]
    $\tau\in\T,A\in\F_\tau$について,
    $\tau_A:=1_{A}\tau+1_{A^\comp}\infty$は再びMarkov時刻である.
\end{lemma}

\begin{proposition}[停止時の構成]
    $(\tau_n)_{n\in\N}$を$\bF$-Markov時の可算列とする.
    \begin{enumerate}
        \item $\sup_{n\in\N}\tau_n$もMarkov時である.
        \item $\bF$が右連続ならば,$\inf_{n\in\N}\tau_n,\limsup_{n\in\N}\tau_n,\liminf_{n\in\N}\tau_n$もMarkov時である.
    \end{enumerate}
\end{proposition}

\begin{theorem}[停止時の離散近似]
    任意の$\bF$-Markov時に対して,有限個の値しか取らない$\bF$-Markov時の減少列が存在して,その極限に等しい.
\end{theorem}
\begin{Proof}
    任意のMarkov時刻$T\in\bT$に対して,
    \[T_k=\begin{cases}
        \infty&T\ge k,\\
        q2^{-k}&(q-1)2^{-k}\le T<q2^{-k},q<2^kk,
    \end{cases}\]
    とすれば良い.
\end{Proof}

\subsection{停止過程と情報量}

\begin{tcolorbox}[colframe=ForestGreen, colback=ForestGreen!10!white,breakable,colbacktitle=ForestGreen!40!white,coltitle=black,fonttitle=\bfseries\sffamily,
title=]
    過程を事前に決めた規則でランダムに止める過程も,再びたしかに確率過程となる.
    この過程が生成する$\sigma$-代数$\F_\infty$を,時点$\tau$までの情報量という.
    これはフィルトレーション$\F_0\subset\F_1\subset\F_2\subset\cdots$のランダム化の意味で一般化である.
\end{tcolorbox}

\begin{definition}[stopped process / optional stopping, information]
    $\tau$を$(\F_t)$-停止時,$(X_t)_{t\in T}$を$(\F_t)$-適合過程とする.
    \begin{enumerate}
        \item $X^\tau:=(X_{t\land\tau(\om)}(\om))_{t\in T}$はを,\textbf{$\tau$-停止過程}という.これはたしかに確率過程になる.
        \item $F^\tau:=\{\F_n^\tau\}$を$F$の\textbf{$\tau$-停止情報系}という.
        \item 次によって定まる閉$\sigma$-代数を,\textbf{時点$\tau$までの情報}という.
        \[\F_\tau:=\Brace{A\in\F\mid\forall_{t\in T}\;A\cap\Brace{\tau\le t}\in\F_t}=\F[X_{t\land\tau(\om)}(\om);t\in T]\]
    \end{enumerate}
\end{definition}

\begin{proposition}[停止時までの情報の性質]
    $\tau,\tau_n,\sigma\in\T$とする.
    \begin{enumerate}
        \item $\F_\tau$は確かに完備な$\sigma$-代数となる.
        \item $\tau$は$\F_\tau$-可測である.
    \end{enumerate}
\end{proposition}

\begin{proposition}[発展的可測性の保存]
    $X$を発展的可測,$T$をMarkov時とする.
    \begin{enumerate}
        \item 確率変数$1_{\Brace{T<\infty}}X_T$は$\F_T$-可測である.
        \item 停止過程$X^T$は$(\F_{t\land T})$-発展的可測である.
    \end{enumerate}
\end{proposition}

\section{停止時と運命}

\begin{tcolorbox}[colframe=ForestGreen, colback=ForestGreen!10!white,breakable,colbacktitle=ForestGreen!40!white,coltitle=black,fonttitle=\bfseries\sffamily,
title=]
    停止時は「ランダムな時刻」として成功を収めた.
    さらに,時区間もランダム化することを考える.
\end{tcolorbox}

\subsection{停止時のグラフ}

\begin{example}[停止時のグラフ]
    停止時$\sigma,\tau$について,
    \[[[\sigma,\tau]]:=\Brace{(\om,\tau)\in\Om\times\R_+\mid\sigma(\om)\le t\le\tau(\om)}.\]
    と表す.$[[\tau]]$をグラフという.
\end{example}

\subsection{$D$-過程の定めるジャンプ過程}

\begin{example}[$D$-過程の局所化]
    $X\in D$について,
    \begin{enumerate}
        \item $X_{t-}:=\lim_{s\to t}X_s\;(t>0),X_{0-}=X_0$として,$X_-=(X_{t-})_{t\in\R_+}$が定まる.
        \item $\Delta X_t:=X_t-X_{t-}$として,$\Delta X=(\Delta X_t)_{t\in\R_+}$が定まる.
        \item $X^*_t:=\sup_{s\le t}\abs{X_s}$として$X^*=(X^*_t)_{t\ge0}$が定まる.
        \item $(\Delta X)_t^*:=\sup_{s\le t}\abs{\Delta X_s}$として$(\Delta X)^*:=((\Delta X)_t)_{t\in\R_+}^*$が定まる.
        \item $\sum_s(\Delta X_s)^2:=\paren{\sum_{s\le t}(\Delta X_s)^2}_{t\in\R_+}$が定まる.
    \end{enumerate}
    最後に,確率時刻$\tau:\Om\to[0,\infty]$に対して,
    \[\Delta X_\tau:=X_\tau 1_{\Brace{\tau<\infty}}-X_{\tau-}1_{\Brace{\tau<\infty}}\]
    と定める.ただし,$\Brace{\tau<\infty}\subset\Om\times\R_+$の指示関数とした.
\end{example}

\begin{theorem}[$D$-過程のジャンプにはMarkov時刻を当てることが出来る]
    $X\in D\cap(\F_t)$について,運命$\Brace{\Delta X\ne0}$は薄い.
    すなわち,あるMarkov時刻の列$\{\tau_k\}\subset\T$について,$\Brace{\Delta X\ne0}=\cup_{k\in\N}[[\tau_k]]$.
\end{theorem}

\subsection{過程の剪断}

\begin{theorem}[発展的可測過程のランダムな停止]
    $X$を発展的可測,$\tau\in\T$をMarkov時刻とする.
    ランダムな瞬間$1_{\Brace{\tau<\infty}}X_\tau$は$\F_\tau$可測な確率変数である.
\end{theorem}

\begin{theorem}
    $\tau\in\T$,$X$を発展的可測(で随意)な過程とする.
    このとき,剪断過程$X^\tau$も発展的可測(で随意な)である.
\end{theorem}

\subsection{停止時による過程の局所化}

\begin{tcolorbox}[colframe=ForestGreen, colback=ForestGreen!10!white,breakable,colbacktitle=ForestGreen!40!white,coltitle=black,fonttitle=\bfseries\sffamily,
title=]
    $\tau$で止める確率過程を$X^\tau_t:=X_{\min(t,\tau)}$と表すと,
    これは過程$X$をランダムに裁断したもののようであり,
    これを用いて種々の性質を局所化出来る.
\end{tcolorbox}

\begin{definition}[locally martingale, locally integrable]
    $\cK$を過程のクラスとする.このとき,$X\in\cK_\loc$とは,Markov時刻の$\infty$に発散する増大列$\{\tau_n\}\subset\T$が存在して,$\forall_{n\in\N}\;X^{\tau_n}\in\cK$を満たすことをいう.
    \begin{enumerate}
        \item 随意過程$X\in D\cap(\F_t)$が\textbf{局所martingale}であるとは,$\infty$に収束するMarkov時の増大列$(\tau_n)$が存在して,任意の$n\in\N$について$1_{\tau_n>0}X^{\tau_n}$が一様可積分なmartingaleになることをいう.
        なお,この一様可積分性は取り払っても同じクラスを定める.
        \item 非負な増大過程が\textbf{局所可積分}であるとは,$\infty$に収束する停止時の増大列$(\tau_n)$が存在して,$\forall_{n\in\N}\;1_{\tau_n>0}X^{\tau_n}\in L^1(\Om)$を満たすことをいう.
    \end{enumerate}
\end{definition}

\section{Markov時刻の定める分解}

\begin{tcolorbox}[colframe=ForestGreen, colback=ForestGreen!10!white,breakable,colbacktitle=ForestGreen!40!white,coltitle=black,fonttitle=\bfseries\sffamily,
title=]
    $\T\subset L(\Om;[0,\infty])$は,直交分解$\T=\T_p\oplus \T_p^\perp$を許す.
\end{tcolorbox}

\begin{definition}[predictable / announcable, accessible, totally inaccessible]
    確率時刻$\tau:\Om\to[0,\infty]$について,
    \begin{enumerate}
        \item \textbf{可予測}であるとは,そのグラフが可予測であることをいう:$[\tau]\in\fP$.
        可予測なMarkov時刻を$\T_p$で表す.
        \item \textbf{事前通告可能}であるとは,ある$\forall_{\tau>0}\;\tau_n<\tau$を満たすMarkov時の増大列$(\tau_n)$の極限であることをいう.
        この列$\{\tau_n\}\subset\T$を\textbf{事前通告}という.
        \item \textbf{到達可能}であるとは,ある可予測な時刻の列$\{\tau_n\}\subset\T_p$が存在して,$[\tau]\subset\cup_{n\in\N}[\tau_n]$が成り立つことをいう.すなわち,
        ある停止時の列$(\tau_n)$が存在して,殆ど確実に$\exists_{n\in\N}\;\tau_n=\tau$が成り立つことをいう.
        \item \textbf{到達不可能}であるとは,任意の可予測な時刻$\sigma$に対して,$P[\tau=\sigma<\infty]=0$が成り立つことをいう.
    \end{enumerate}
\end{definition}
\begin{example}
    到達可能かつ到達不可能ならば,$\tau=\infty\;(P\dae)$である.
\end{example}

\subsection{停止時の分解}

\begin{theorem}[停止時の分解]
    
\end{theorem}

\subsection{可予測時刻の性質}

\begin{tcolorbox}[colframe=ForestGreen, colback=ForestGreen!10!white,breakable,colbacktitle=ForestGreen!40!white,coltitle=black,fonttitle=\bfseries\sffamily,
title=]
    確率時刻$\tau$が可予測であることと,事前通告可能であることは同値.
\end{tcolorbox}

\begin{proposition}[可予測時刻の特徴付け]\mbox{}
    \begin{enumerate}
        \item 可予測時刻はMarkov時刻である:$\T_p\subset\T$.
        \item 確率時刻$\tau:\Om\to[0,\infty]$について,次は同値:
        \begin{enumerate}
            \item $\tau\in\T_p$.
            \item 確率区間$(0,\tau),\cointerval{0,\tau},\cointerval{\tau,\infty}$のうち,可予測なものが存在する.
        \end{enumerate}
    \end{enumerate}
\end{proposition}

\begin{theorem}[可予測時刻の性質]
    $\tau,\tau_n,\sigma\in\T_p$とする.
    \begin{enumerate}
        \item $\sigma\land\tau,\sigma\lor\tau\in\T_p$.
        \item $\sup_{n\in\N}\tau_n\in\T_p$.
        \item $\cup_{n\in\N}\Brace{\inf_{n\in\N}\tau_n=\tau_n}=\Om$ならば,$\inf_{n\in\N}\tau_n\in\T_p$.
        \item 任意のMarkov時$\tau\in\T$は,可予測時刻の減少列$(\tau_n)$の極限である:$\tau_n\searrow\tau$.
        \item $A\in\F_{t-}$ならば,$\tau_A\in\T_p$.
        \item $A\in\fP$ならば,$D_A\in\T_p$は$[D_A]\cup A\in\fP$に同値.
        \item $A\in\F_{\sigma-},t\in\T$ならば,$A\cap\Brace{\sigma\le\tau}\in\F_{t-}$.
    \end{enumerate}
\end{theorem}

\begin{theorem}[事前通告可能時刻と可予測時刻は同値]\mbox{}
    \begin{enumerate}
        \item $\tau$を事前通告可能時刻とする.このとき,$\tau$は可予測で,$\F_{\tau-}=\lor_{n\in\N}\F_{\tau_n}$.
        \item $\tau$が可予測ならば,可予測な時刻の列$\{\tau_n\}$による事前通告が存在する:$\tau_n\nearrow\tau$.
    \end{enumerate}
    特に,確率時刻$\tau$が可予測であることと,事前通告可能であることは同値.
\end{theorem}

\subsection{$D$-過程の定めるジャンプ過程は可予測}

\begin{theorem}\mbox{}
    \begin{enumerate}
        \item $A\subset\Om\times\R_+$は薄くて可予測な運命であるとする.このとき,可予測な時刻の列$(\tau_n)$であって,$A$を埋め尽くすものが存在する.
        \item $X\in D\cap(\F_t)$を可予測とする.このとき,$\Delta X$も可予測で,薄い集合$\Brace{\Delta X\ne0}$を埋め尽くす可予測時刻の列が存在する.
    \end{enumerate}
\end{theorem}

\subsection{到達可能時刻の性質}

\begin{proposition}
    $\tau\in\T,A\in\F_\tau$とする.$\tau$が到達可能/到達不可能ならば,$\tau_A$もそうである.
\end{proposition}

\begin{theorem}[Markov時刻の直交分解]
    任意のMarkov時刻$T\in\T$に対して,識別不可能な違いを除いて一意的なMarkov時刻の組$(\tau,\sigma)$が存在して,次を満たす:
    \begin{enumerate}
        \item $\tau$は到達可能時刻である.
        \item $\sigma$は到達不可能である.
        \item $[T]=[\tau]+[\sigma]$と非交和で表せる.
    \end{enumerate}
\end{theorem}

\subsection{随意過程の分解}

\begin{theorem}[随意過程の軌跡]
    任意の随意過程$X\in D\cap(\F_t)$について,次が存在する:
    \begin{enumerate}
        \item ある左連続な過程$Y$,
        \item 到達不可能時刻の列$(\sigma_n)$
        \item 可予測時刻の列$(\tau_m)$
    \end{enumerate}
    \[X=Y+\sum_{n\in\N}\Delta X_{\sigma_n}1_{[\sigma_n]}+\sum_{m\in\N}\Delta X_{\tau_m}1_{[\tau_m]}.\]
    さらに,$X$が可予測ならば,$\Delta X_{\sigma_n}$は$0$と識別不可能である.
\end{theorem}

\subsection{随意過程の可予測性}

\begin{definition}[left quasicontinuous]
    $X\in D\cap(\F_t)$が$\forall_{\tau\in\T_p}\;\Delta X_\tau 1_{\Brace{\tau<\infty}}\overset{P}{=}0$を満たすとき,\textbf{左擬連続}であるという.
\end{definition}

\begin{theorem}
    $X\in D\cap(\F_t)$について,次は同値:
    \begin{enumerate}
        \item $X$は左擬連続である.
        \item $X$のジャンプ時刻は,到達不可能なMarkov時刻によって埋め尽くされている.
        \item $\tau$に収束するMarkov時刻の増大列$\{\tau_n\}\subset\T$について,$\lim_{n\to\infty}X_{\tau_n}\overset{P}{=}X_\tau\;\on\Brace{\tau<\infty}$.
    \end{enumerate}
\end{theorem}

\begin{theorem}
    随意過程$X\in D\cap(\F_t)$について,次は同値:
    \begin{enumerate}
        \item $X$は可予測である.
        \item 次の2条件を満たす:
        \begin{enumerate}
            \item $\forall_{\tau\in\T_p}\;X_\tau1_{\Brace{\tau<\infty}}$は可測過程である.
            \item 任意の到達不可能時刻$\sigma$について,$\Brace{\Delta X\ne0}\cap[\sigma]$は$P$-零集合である.
        \end{enumerate}
    \end{enumerate}
\end{theorem}

\subsection{随意過程の識別不可能性}

\begin{theorem}[section theorem]
    $A\in\D$を随意集合とする.任意の$\ep>0$に対してMarkov時刻$\tau_\ep\in\T$が存在して,$[\tau_\ep]\subset A$かつ
    \[P[\tau_\ep<\infty]\ge P[\pi(A)]-\ep\]
    が成り立つ.
    同様の結論が,$\D$を$\fP$に,$\T$を$\T_p$に変えても成り立つ.
\end{theorem}

\begin{corollary}[2つの随意過程が識別不可能であるための条件]
    $X,Y$を随意過程とする.任意のMarkov時刻$\tau$について,$X_\tau=Y_\tau\;\on\Brace{\tau<\infty}$ならば,$X,Y$は識別不可能である:$X=Y$.
\end{corollary}

\subsection{一般の過程の分解}

\begin{theorem}
    $X$を可測過程で,$X\ge0\lor\abs{X}<c$を満たすとする.
    このとき,識別不可能な違いを除いて一意的に随意過程と可予測過程の組$({}^0X,{}^pX)$が存在して,次を満たす:
    \begin{enumerate}
        \item $\forall_{\tau\in\T}\;E[X_\tau1_{\Brace{\tau<\infty}}|\F_\tau]={}^0X_\tau1_{\Brace{\tau<\infty}}$.
        \item $\forall_{\tau\in\T_p}\;E[X_\tau1_{\Brace{\tau<\infty}}|\F_{\tau-}]={}^pX_\tau1_{\Brace{\tau<\infty}}$.
    \end{enumerate}
    それぞれを随意・可予測な射影という.
\end{theorem}
\begin{remarks}
    上の条件(1),(2)は次と同値である:
    \begin{enumerate}
        \item $\forall_{\tau\in\T}\;E[X_\tau1_{\Brace{\tau<\infty}}]=E[{}^0X_\tau1_{\Brace{\tau<\infty}}]$.
        \item $\forall_{\tau\in\T_p}\;E[X_\tau1_{\Brace{\tau<\infty}}]=E[{}^pX_\tau1_{\Brace{\tau<\infty}}]$.
    \end{enumerate}
\end{remarks}

\begin{example}
    $W$をWiener過程とする.${}^pW=W$である.
\end{example}

\begin{theorem}
    $X$を可測過程とし,$X\in D\land {}^0X\in D$を満たすとする.このとき,${}^p(X_-)={}^0(X)_-$.
\end{theorem}

\chapter{マルチンゲール過程}

\begin{quotation}
    Markov過程の多くの主定理は既にマルチンゲールの構造に含まれている.
    また,ポテンシャル論への多くの応用も持つ(Meyer, \cite{Meyer}).
    
    劣マルチンゲールは「単調増加関数」,マルチンゲールは「定数関数」に当たる概念である.
    任意の再標本化について保たれる性質であり,また,「有界な単調増加列は収束する」に当たる結果が得られる.
\end{quotation}

\begin{notation}
    次のような記法体系を提案する.
    \begin{enumerate}
        \item martingale過程の全体を$\M\subset\bF$で表す.
        \begin{enumerate}
            \item 右連続なmartingaleの全体を$\M\cap D$で表す.
            \item 一様可積分な右連続martingaleの全体を$M\subsetneq\M\cap D$で表す.$M\simeq_\Vect L^1(\Om,\F_\infty)$が成り立つ.
            \item 局所マルチンゲールの全体を$M_\loc$で表す:
            \[M\subset M_\loc:=\Brace{X\in D\cap\bF\mid \exists_{\Brace{\tau_n}\subset\T}\;P[\tau_n\le\tau_{n+1}]=1\land P[\lim_{n\to\infty}\sigma_n=\infty]=1\land\forall_{n\in\N}X^{\tau_n}\in M}\]
            $\M\cap D\subset M_\loc$であるが,$\M\not\subset M_\loc$である.
            \item $(\D)$でDirichlet類の全体を表す.$M\cap D=(\D)\cap M_\loc$が成立する.
            \item 一般のBanach空間$\bH^p$を
            \[\bH^p:=\Brace{X\in\M\cap D\mid\norm{X}_{\H^p}:=\norm{X^*}_{L^p(\Om,\F_\infty)}<\infty},\quad1\le p\le\infty.\]
            と定める.
            $H^p:=\bH^p\cap C$は閉部分空間になる\ref{cor-the-space-Hp}.
            $\bH^1\subsetneq M$であるが,$p>1$については$L^p$-有界なマルチンゲールは$\bH^p$の元である.
            \item $L^2(\Om)$-有界なマルチンゲールの全体について,$\bH^2\simeq_\Hilb L^2(\Om,\F_\infty)$が成り立つ.
            その部分空間を$H^{2}:=\bH^2\cap C$とし,$H^{2,d}$を純不連続な$\bH^2$の元とすると,$\bH^2={}^0\!H^{2}\oplus H^{2,d}$と直交分解される.
        \end{enumerate}
        \item 見本道が殆ど確実に有限な全変動を持つ$D$-過程の全体を
            \[\A:=\Brace{A\in\bF\cap D\mid\forall_{\om\in\Om\;\as}\;A(\om)|_{[0,t]}\in\BV([0,t])}\]
            で表す.
            \begin{enumerate}
                \item 見本道が殆ど確実に単調増加する$D$-過程の全体を
                \[\A^+:=\Brace{A\in\bF\mid\forall_{\om\in\Om\;\as}\;A_t(\om)\text{は単調増加}}\]
                と表す.
                \item 任意の$A\in\A$は,$A^+,A^-\in\A^+$を用いて$A_t=A^+-A^-$と表せる.
            \end{enumerate}
    \end{enumerate}
\end{notation}

\section{マルチンゲールの定義と構成}

\begin{tcolorbox}[colframe=ForestGreen, colback=ForestGreen!10!white,breakable,colbacktitle=ForestGreen!40!white,coltitle=black,fonttitle=\bfseries\sffamily,
title=]
    この節の内容は$T=\N$の場合について述べるが,$\R$でも可分変形を通じて通用する結果である.
    位相の議論が必要がない点のみが相違点であり,その意味で純粋な理論が見れる.
    マルチンゲールの強力な点は再標本化によって保たれることであり,
    これが豊穣なマルチンゲール不等式を生み,これがマルチンゲール収束定理を導く.
\end{tcolorbox}

\subsection{定義と基本性質}

\begin{definition}[martingale, submartingale]\mbox{}
    \begin{enumerate}
        \item 確率過程$(X_n)$が情報系$(\F_n)$について\textbf{$(\F_n)$-マルチンゲール}であるとは,次の3条件が成り立つことをいう:
        \begin{enumerate}[({M}1)]
            \item $(\F_n)$-適合的である:$\forall_{n\in\N}\;X_n\in\L_{\F_n}(\Om)$.
            \item 可積分列である:$\forall_{n\in\N}\;X_n\in L^1(\Om)$.
            \item martingale性:$\forall_{n\in\N}\;E[X_{n+1}|\F_n]=X_n\;\as$
        \end{enumerate}
        $\bF$-マルチンゲールの全体を$\M(\bF)$で表す.
        \item (3)の代わりに$\forall_{n\in\N}\;E[X_{n+1}|\F_n]\ge X_n\;\as$が成り立つとき,\textbf{$F$-劣マルチンゲール}であるといい,
        $\forall_{n\in\N}\;E[X_{n+1}|\F_n]\le X_n\;\as$が成り立つとき,\textbf{$F$-優マルチンゲール}であるという.
        \item 単に\textbf{マルチンゲール}とは,$F$が過程$(X_n)$が定める自然な増大系である場合をいう.
    \end{enumerate}
\end{definition}
\begin{remarks}[定義の特徴付け]
    次の条件は(M3)に同値.
    \begin{enumerate}
        \item $\forall_{n\in\N}\;\forall_{A\in\F_n}\;E[X_{n+1},A]=E[X_n,A]$.これはつまり,
        \[\forall_{s<t}\;\forall_{A\in\F_s}\quad\int_AX_sdP=\int_AX_tdP.\]
        \item $\forall_{n\in\N}\;\forall_{Y\in L^\infty_{\F_n}(\Om)}\;E[X_{n+1}Y]=E[X_nY]$.
    \end{enumerate}
\end{remarks}

\begin{lemma}[マルチンゲールの基本性質]
    $(X_n)$を$(\F_n)$-マルチンゲールとする.
    \begin{enumerate}
        \item $\forall_{m\ge n\ge1}\;E[X_m|\F_n]=X_n\;\as$
        \item $\forall_{n\in\N}\;E[X_n]=\const$
        \item $(X_n)$は(その自然な情報系について)マルチンゲールである.
    \end{enumerate}
    これより,マルチンゲールとは,「ある情報系$\bF$が存在して$\bF$-マルチンゲールになる」という性質と同等であるため,$\bF$は省略できる.
\end{lemma}
\begin{Proof}\mbox{}
    \begin{enumerate}
        \item 任意の$n\in\N$に対して,$\forall_{k\in\N}\;E[X_{n+k}|\F_n]=X_n$をいう.$k=0$のとき,これは$X_n$の$\F_n$-可測性からわかる.$k>0$のとき,繰り返し期待値の法則と帰納法の仮定から,
        \[E[X_{n+k}|\F_n]=E[E[X_{n+k}|\F_{n+k-1}]|\F_n]=E[X_{n+k-1}|\F_n]=X_n.\]
        \item (1)の両辺の期待値を取れば良い.
        \item $(X_n)$は$(\F_n)$-マルチンゲールだから,$\forall_{n\in\N}\;\sigma[X_1,\cdots,X_n]\subset\F_n$より,繰り返し期待値の法則から,
        \[E[X_{n+1}|\sigma[X_1,\cdots,X_n]]=E[E[X_{n+1}|\F_n]|\sigma[X_1,\cdots,X_n]]=E[X_n|\sigma[X_1,\cdots,X_n]]=X_n.\]
    \end{enumerate}
\end{Proof}

\begin{lemma}[劣マルチンゲールの基本性質]\label{lemma-property-of-submartingale}
    $(X_n)$を$(\F_n)$-劣マルチンゲールとする.
    \begin{enumerate}
        \item $\forall_{m\ge n\ge1}\;E[X_m|\F_n]\ge X_n\;\as$
        \item $(E[X_n])_{n\in\N}$は実数の単調増加列である.
        \item $(X_n)$は(その自然な情報系について)劣マルチンゲールである.
        \item $X$がマルチンゲールでもあることと,$t\mapsto E[X_t]$が定値であることは同値.
    \end{enumerate}
\end{lemma}
\begin{Proof}
    $=$を$\ge$に証明置換すれば良い.
    (4)は,(1)からすぐに判る,任意の$m\ge n\in T$について$E[X_m|\F_n]=X_n$であることは,両辺に$E$を作用させて$E[X_m]=E[X_n]$であることとは同値.
\end{Proof}

\subsection{マルチンゲールの例と構成}

\begin{example}[中心化されたi.i.d.列の部分和の列,条件付き期待値の列]\mbox{}
    \begin{enumerate}
        \item $\{X_n\}\subset L^1(\Om)$を独立同分布列とする.
        このとき,部分和の列$\paren{S_n:=\sum_{i=1}^nX_i}$は明らかに$\forall_{n\in\N}\;S_n\in L^1(\Om)$であるから,
        条件付き期待値のa.s.線形性と,自然な情報系$\F_n:=\F[X_1,\cdots,X_n]$について,$X_{n+1}\indep\F_n$より,
        \[E[S_{n+1}|\F_n]\overset{\as}{=}E[S_n|\F_n]+E[X_{n+1}|\F_n]\overset{\as}{=}S_n+E[X_{n+1}].\]
        よって,$X_n$が中心化されていたならば,これはマルチンゲールを定める.
        \item よって,原点から出発する$\Z$上の対称で単純な酔歩はマルチンゲールである
        \item $(\F_n)$を情報系とし,$X\in\L^1(\Om)$を可積分確率変数とする.
        この情報系が定める条件付き期待値の列を$X_n:=E[X|\F_n]$とおけば,$(X_n)$はマルチンゲールである.
        
        実際,$\forall_{n\in\N}\;X_n\in L^1_{\F_n}(\Om)$は条件付き期待値の定義から明らかであり,
        $E[X_{n+1}|\F_n]=E[E[X|\F_{n+1}]|\F_n]\overset{\as}{=}E[X|\F_n]=X_n$.
    \end{enumerate}
    (1)の状況は「中心化された公平な賭け」などの意味論を持つ.コイントスをして,表なら$+x$円,裏なら$-x$円の賭けで,所持金を$X_n$とすると,これはマルチンゲールである.
\end{example}

\begin{lemma}[マルチンゲールの線型束演算に関する保存]
    $(X_n),(Y_n)$を$(\F_n)$-劣マルチンゲールとする.
    \begin{enumerate}
        \item $(aX_n+bY_n+c)$は$(\F_n)$-劣マルチンゲールである.特に$X,Y$が$(\F_t)$-マルチンゲールでも同様で,$\M(\bF)$は線型空間をなす.
        \item $X_n\lor Y_n$は$(\F_n)$-劣マルチンゲールである.優マルチンゲールは同様に$\land$に関して保存する.
        \item $\{A_n\}\subset L^1(\Om)$を単調増加な適合過程とすると,$X_n+A_n$は再び$(\F_n)$-劣マルチンゲールになる.実は,任意の劣マルチンゲールはこうして得られる.
    \end{enumerate}
    ただし,2つのマルチンゲール$X,Y$がそれぞれ対応する情報系が違うとき,$X_n+Y_n$が適合的とは限らず,線型演算によって保存されるとは限らないことに注意.
\end{lemma}

\begin{lemma}[劣マルチンゲールの構成]
    $\psi:\R\to\R$を凸関数,$(X_n)$を$(\F_n)$-マルチンゲールとし,$\{\psi(X_n)\}\subset L^1(\Om,\F)$は可積分過程になるとする.
    \begin{enumerate}
        \item $(\psi(X_n))_{n\in\N}$は$(\F_n)$-劣マルチンゲールである.
        \item $\psi$が広義単調増加に取れるならば,$(X_n)$が$(\F_n)$-劣マルチンゲールに過ぎなくとも,$(\psi(X_n))_{n\in\N}$は$(\F_n)$-劣マルチンゲールとなる.
    \end{enumerate}
\end{lemma}
\begin{Proof}
    条件$\{\psi(X_n)\}\subset L^1(\Om,\F)$より,任意の$\psi(X_n)$と$\F_m$について条件付き期待値$E[\psi(X_n)|\F_m]$が存在する.
    \begin{enumerate}
        \item 条件付き期待値のJensenの不等式と,$(X_n)$が$(\F_n)$-マルチンゲールであることより,
        \[E[\psi(X_{n+1})|\F_n]\ge\psi(E[X_{n+1}|\F_n])=\psi(X_n)\;\as\]
        \item 条件付き期待値のJensenの不等式と,$X_n\le E[X_{n+1}|\F_n]\Rightarrow\psi(X_n)\le\psi(E[X_{n+1}|\F_n])\;\as$より,
        \[E[\psi(X_{n+1})|\F_n]\ge\psi(E[X_{n+1}|\F_n])\ge\psi(X_n)\;\as\]
    \end{enumerate}
\end{Proof}

\begin{example}[マルチンゲールに付属する劣マルチンゲール]\mbox{}
    \begin{enumerate}
        \item $F$-マルチンゲール$(X_n)$に対して,$(X_n^2),(\abs{X_n})$,一般に$\abs{X_t}^\lambda\;(\lambda\ge1)$は,$\forall_{t\in T}\; E[\abs{X_t}^\lambda]<\infty$さえ満たせば,全て$F$-劣マルチンゲールである.
        \item $F$-劣マルチンゲール$(X_n)$に対して,$(X_n^+:=X_n\lor0)$も$F$-劣マルチンゲールである.
    \end{enumerate}
\end{example}

\subsection{劣マルチンゲールのDoob分解}

\begin{tcolorbox}[colframe=ForestGreen, colback=ForestGreen!10!white,breakable,colbacktitle=ForestGreen!40!white,coltitle=black,fonttitle=\bfseries\sffamily,
title=]
    劣マルチンゲールは,初期状態$X_0$から始まるマルチンゲールと,可予測な単調増大成分とに一意的に分解できる.
\end{tcolorbox}

\begin{theorem}[Doob-Meyer decomposition theorem]
    任意の$(\F_n)$-劣マルチンゲール$(X_n)$は,
    $(\F_n)$-マルチンゲール$(M_n)$と
    $(\F_n)$-可予測な広義増加列$(A_n),A_0=0$とが一意的に存在して,これらの和に分解される:$X_n=M_n+A_n\;\as$.
\end{theorem}
\begin{Proof}\mbox{}
    \begin{description}
        \item[一意性] 任意の$n\in\N$について,
        $(A_n)$の階差列は$\F_n$-可測で可積分$A_{n+1}-A_n\in L^1(\F_n)$であるから,
        \begin{align*}
            A_{n+1}-A_n&=E[A_{n+1}-A_n|\F_n]\\
            &=E[X_{n+1}-E_n|\F_n]-E[M_{n+1}-M_n|\F_n]=E[X_{n+1}-E_n|\F_n]
        \end{align*}
        が必要.すなわち,$A_0=0$と併せると,
        \[A_n:=\sum_{k=0}^{n-1}E[X_{k+1}-X_k|\F_k],\quad M_n:=X_n-A_n\]
        と一意的に表示されることが必要.
        \item[存在] 上の構成について,$(A_n)$は$\F_{n-1}$-可測な確率変数の和だから可予測性で,$(X_n)$の劣マルチンゲール性より増大性も明らかだから,あとは$(M_n)$のマルチンゲール性を示せば良い.
        \begin{align*}
            E[M_{n+1}-M_n|\F_n]&=E[X_{n+1}-X_n|\F_n]-E[A_{n+1}-A_n|\F_n]\\
            &=E[X_{n+1}-X_n|\F_n]-E\Square{E[X_{n+1}-X_n|\F_n]|\F_n}=0.
        \end{align*}
    \end{description}
\end{Proof}

\begin{observation}
    独立な非負確率変数列$\{Y_n\}\subset L^1(\Om)_+$に対して,
    $X_n:=\sum_{k\in[n]}Y_k$と定めると,これは自然な情報系$(\F_n)$について
    \[E[X_{n+1}|\F_n]=X_n+E[Y_{n+1}]\ge X_n\]
    を満たすから劣マルチンゲールである.これは二通り
    \[X_n=0+X_n,\quad X_n=(X_n-E[X_n])+E[X_n]\]
    にマルチンゲール成分と単調増大成分とに分けられるが,
    後者のみがDoob分解である.
\end{observation}

\subsection{マルチンゲール変換}

\begin{tcolorbox}[colframe=ForestGreen, colback=ForestGreen!10!white,breakable,colbacktitle=ForestGreen!40!white,coltitle=black,fonttitle=\bfseries\sffamily,
title=]
    $(X_n)$のマルチンゲール変換$(H\cdot X)_n$とは,確率積分$\int^t_0HdX$に相当する.
\end{tcolorbox}

\begin{theorem}\mbox{}
    \begin{enumerate}
        \item $\{X_n\}\subset L^1(\Om)$を$(\F_n)$-マルチンゲール,$\{Y_n\}\subset L^\infty(\Om)$を$(\F_n)$-適合過程とする.このとき,
        \[S_n:=\sum_{k=0}^{n-1}Y_k(X_{k+1}-X_k),\quad S_0=0,\]
        は再び$(\F_n)$-マルチンゲールである.
        \item $\{X_n\}\subset L^1(\Om)$を$(\F_n)$-劣マルチンゲール,$\{H_n\}\subset L^\infty(\Om)_+$を正な可予測過程とする.このとき,
        \[H\cdot X:=\sum_{n\in\N}H_n(X_n-X_{n-1}),\quad S_0=X_0,\]
        は再び$(\F_n)$-劣マルチンゲールである.
    \end{enumerate}
\end{theorem}

\begin{corollary}
    特に,Markov時刻$T\in\bT$に対して,$X^T$は再び劣マルチンゲールである.
\end{corollary}
\begin{Proof}
    $H_n:=1_{\Brace{n\le T}}=1-1_{\Brace{T\le n-1}}$とすると,$\F_{n-1}$-可測である.
\end{Proof}

\subsection{離散時に関するDoobの任意抽出定理}

\begin{tcolorbox}[colframe=ForestGreen, colback=ForestGreen!10!white,breakable,colbacktitle=ForestGreen!40!white,coltitle=black,fonttitle=\bfseries\sffamily,
title=]
    劣マルチンゲールは任意の$n\le m$に対して$E[X_m|\F_n]\ge X_n\;\as$を満たすのであった\ref{lemma-property-of-submartingale}.
    マルチンゲールから「ランダムな部分列を抽出」しても,取り出した部分列はマルチンゲールである.
\end{tcolorbox}

\subsubsection{任意抽出に対するマルチンゲール性の保存}

\begin{tcolorbox}[colframe=ForestGreen, colback=ForestGreen!10!white,breakable,colbacktitle=ForestGreen!40!white,coltitle=black,fonttitle=\bfseries\sffamily,
title=]
    Doobの任意抽出定理にはいくつかの要件のバージョンがある.
\end{tcolorbox}

\begin{theorem}[有界停止時刻に於ける任意抽出]\label{thm-martingale-property-for-random-stopping-time}
    $\exists_{N\in\N}\;\sigma\le\tau\le N\;\as$を有界な停止時とする.
    \begin{enumerate}
        \item $(X_n)$が$(\F_n)$-マルチンゲールならば,$E[X_\tau|\F_\sigma]=X_\sigma\;\as$
        \item $(X_n)$が$(\F_n)$-劣マルチンゲールならば,$E[X_\tau|\F_\sigma]\ge X_\sigma\;\as$
    \end{enumerate}
\end{theorem}
\begin{Proof}\mbox{}
    \begin{enumerate}
        \item \begin{enumerate}[(a)]
            \item $X_\tau\in L^1(\Om)$であることは次のように確認できる:
            \[E[\abs{X_\tau}]=\sum^N_{k=0}E[\abs{X_\tau},\{\tau=k\}]\le\sum^N_{k=0}E[\abs{X_k}]<\infty.\]
            \item $\forall_{n\in\N}\;\forall_{a\in\R}\;\Brace{X_\sigma<a}\cap\{\sigma=n\}=\{X_n<a\}\cap\{\sigma=n\}\in\F_n$より,$X_\sigma$は$\F_\sigma$-可測.
            \item 任意の$A\in\F_\sigma$をとって,$E[X_\tau,A]=E[X_\sigma,A]$を示せば良い.
        \end{enumerate}
    \end{enumerate}
\end{Proof}
\begin{remark}[有界でない場合は極めて簡単な反例が存在する]
    停止時が有界でない場合は反例が存在する.原点から出発する$\Z$上の対称で単純な酔歩はマルチンゲールであり,
    $-k$への到達時刻$\tau_{-k}:=\min\Brace{n\in\N\mid X_n=-k}$も停止時を定めるが,これは有限ではあっても有界ではなく,$\tau_{-1}<\tau_{-2}$かつ$X_{\tau_{-1}}\equiv-1>X_{\tau_{-2}}\equiv-2\;\as$が成り立つ.
\end{remark}

\begin{corollary}[マルチンゲールの特徴付け]
    $X\in\bF$を可積分な適合過程とする.次は同値:
    \begin{enumerate}
        \item $X$は$\bF$-マルチンゲールである.
        \item $\forall_{S,T\in\bT(\bF)}\;E[X_S]=E[X_T]$.
    \end{enumerate}
\end{corollary}

\subsubsection{任意標本化定理}

\begin{tcolorbox}[colframe=ForestGreen, colback=ForestGreen!10!white,breakable,colbacktitle=ForestGreen!40!white,coltitle=black,fonttitle=\bfseries\sffamily,
title=]
    マルチンゲールから,ある規則に則って,時点$(\tau_n)$を抽出して観察する.
    これは統計的実験のようなもので,停止過程を一般化する.
\end{tcolorbox}

\begin{definition}[optional sampling]
    $\bF$-停止時の増大列$\Sigma:=\{\sigma_i\}_{i\in\N}$に対して,
    \begin{enumerate}
        \item $X^\Sigma:=(X_{\sigma_n})_{n\in\N}$を\textbf{$\Sigma$-標本化過程}という.
        \item $\F^\Sigma:=\{\F_{\sigma_n}\}_{n\in\N}$を\textbf{$\Sigma$-抽出情報系}という.
    \end{enumerate}
\end{definition}
\begin{example}[停止過程は抽出過程である]
    任意の停止時$\sigma$は停止時の増大列
    $(\sigma_n:=\sigma\land n)$を定めるから,これについて$\sigma$-停止過程と$(\sigma_n)$-抽出過程とは同じ.
\end{example}

\begin{corollary}\mbox{}
    \begin{description}
        \item[任意標本化定理] \mbox{}\\$(X_n)$を$(\F_n)$-劣マルチンゲール,$\{\tau_k\}\subset\bT^{<\infty}(\bF)$を有界な停止時の広義単調増加列とする:$\forall_{n\in\N}\;\exists_{N_n\in\N}\;\tau_n\le N_n$.
        このとき,標本化過程$X^\Sigma=(X_{\tau_k})_{k\in\N}$は$(\F_{\tau_k})$-劣マルチンゲールである.
        \item[任意停止定理] \mbox{}\\
        $\bF$-劣マルチンゲールの
        有界な停止時$\sigma\in\bT^{<\infty}(\bF)$に関する停止過程$X^\sigma$は再び$(\F_{\sigma\land n})$-劣マルチンゲールである.
    \end{description}
\end{corollary}

\section{マルチンゲール不等式}

\subsection{Doobの最大不等式}

\begin{tcolorbox}[colframe=ForestGreen, colback=ForestGreen!10!white,breakable,colbacktitle=ForestGreen!40!white,coltitle=black,fonttitle=\bfseries\sffamily,
title=]
    劣マルチンゲールに対しては,$\max_{1\le k\le n}X_K$に関する評価を,$X_n$のみを用いて与えられる.
    一般の確率過程では決して成り立たないが,Kolmogorovの不等式を一般化する形で,マルチンゲールについては成り立つ.
    このときも,「劣マルチンゲールであるから,最後の時点$S_n$にだけ注目すれば良い」という構造が引き起こす不等式関係なのであった.
\end{tcolorbox}

\begin{theorem}[Doob inequality]\label{thm-Doob-inequality}
    $(X_n)$を$(\F_n)$-劣マルチンゲールとする.このとき,$X_n^+:=X\lor0$とすると,
    \begin{enumerate}
        \item $X^*_N:=\max_{1\le k\le N}X_k$とすると,任意の$a>0$と$N\in\N$について,
        \[P\Square{X^*_N\ge a}\le\frac{1}{a}E[X_N1_{\Brace{X^*_N\ge a}}]\le\frac{1}{a}E[X_N^+].\]
        \item 双対的に,任意の$a>0$と$N\in\N$について,
        \[P\Square{\min_{1\le k\le N}X_k\le -a}\le\frac{1}{a}E[X_N^+]-\frac{1}{a}E[X_1].\]
    \end{enumerate}
\end{theorem}
\begin{Proof}
    $(X^+_n)$は再び劣マルチンゲールだから,
    初めから非負な劣マルチンゲール$(X_n)$をとっても,一般性を失わない.
    \begin{enumerate}
        \item 
        \begin{description}
            \item[タイマーの設定] $(X^*_n)$は$(\F_n)$-適合的な過程である.
            これを用いて,$(X_n)$の値を順次監視し,$a$以上の値が出たら停止するためのタイマーを
            \[\sigma:=\begin{cases}
                \inf\Brace{k\in N+1\mid X_k\ge a},&X^*_N\ge a,\\
                N,&X^*_N<a.
            \end{cases}\]
            とすると,これは有界な$(\F_n)$-停止時で,$\{Y_N\ge a\}$上では$X_\sigma\ge a$を満たす.
            実際,任意の$k\in\N$について,$k<N$のときは,$(Y_n)$が$(\F_n)$-適合的であることより
            $\{\sigma\le k\}=\{\exists_{m\in k+1}\;X_m\ge a\}=\{Y_k\ge a\}\in\F_k$であり,
            $k\ge N$のときは$\sigma$は必ず$N$以下であることから$\{\sigma\le k\}=\Om\in\F_k$.
            \item[タイマーに関する任意停止] よって,停止時$\sigma\le N$に関する劣マルチンゲール性
            \ref{thm-martingale-property-for-random-stopping-time}より,
            \[E[X_N]\ge E[X_\sigma]\ge aP[Y_N\ge a].\]
            これとMarkov不等式を併せて,
            \[\forall_{a>0}\;\forall_{N\in\N}\;\quad P\paren{X^*_N\ge a}\le\frac{1}{a}E\Square{X_N,\max_{1\le k\le n}X_k\ge a}\le\frac{1}{a}E[X_N^+].\]
        \end{description}
    \end{enumerate}
\end{Proof}

\begin{corollary}[Kolmogorov]
    実確率変数列$\{X_n\}\subset L^2(\Om,\F,P)$は独立で,$E[X_n]=0,V_n:=\Var[X_n]<\infty$を満たすとする.
    このとき,$S_k:=\sum^k_{i=1}X_k$とおくと,
    \[\forall_{a>0}\quad P[S^*_n\ge a]\le\frac{1}{a^2}\sum^n_{i=1}V_i.\]
\end{corollary}
\begin{Proof}
    列$(X_n)$は中心化されているから,$(S_k)$はマルチンゲールであり,従って$(S_k^2)$は劣マルチンゲールである.
    よってDoobの最大不等式を$(S_k^2)$と$a^2$に対して用いると,任意の$a>0$について,
    \[P[\max_{1\le k\le N}S_k^2\ge a^2]=P[\max_{1\le k\le N}\abs{S_k}\ge a]\le\frac{E[S^2_N]}{a^2}.\]
\end{Proof}

\subsection{Doobの最大不等式の系}

\begin{tcolorbox}[colframe=ForestGreen, colback=ForestGreen!10!white,breakable,colbacktitle=ForestGreen!40!white,coltitle=black,fonttitle=\bfseries\sffamily,
title=]
    Doobの不等式とその系は,$\bF$の右連続性と完備性に依らず成立し,また連続時間についても成り立つ.
\end{tcolorbox}

\begin{corollary}
    $(X_n)$を$(\F_n)$-劣マルチンゲール,$a>0$とする.
    \begin{enumerate}
        \item \[P\Square{X_N^*\ge a}\le\frac{E[\abs{X_N}]}{a},\quad P\Square{\min_{0\le k\le N}X_k\le -a}\le\frac{1}{a}(E[\abs{X_N}]-E[X_0]).\]
        \item よって,\[E\Square{\max_{0\le k\le N}\abs{X_k}\ge a}\le\frac{E[2\abs{X_N}+\abs{X_0}]}{a}.\]
    \end{enumerate}
\end{corollary}

\begin{corollary}
    $\{X_n\}\subset L^1(\Om)$をマルチンゲール,または正な劣マルチンゲールとする.
    このとき,
    \begin{enumerate}
        \item 任意の$a>0$と$p\ge1$に対して,
        \[P\Square{\max_{1\le k\le N}\abs{X_N}\ge a}\le\frac{1}{a^p}E[\abs{X_N}^p].\]
        \item $p>1$でもあるとき,
        \[E[\abs{X_N}^p]\le E\Square{\max_{1\le k\le N}\abs{X_N}^p}\le\paren{\frac{p}{p-1}}^pE[\abs{X_N}^p].\]
    \end{enumerate}
\end{corollary}
\begin{Proof}
    $X_N\in L^p(\Om)$ならば,
    $\psi(x)=x^p$は凸関数になるため,$(\abs{X_n}^p)$は劣マルチンゲールである.
\end{Proof}

\subsection{Doobの$L^p$-不等式}

\begin{tcolorbox}[colframe=ForestGreen, colback=ForestGreen!10!white,breakable,colbacktitle=ForestGreen!40!white,coltitle=black,fonttitle=\bfseries\sffamily,
title=]
    Doobの不等式は区間$I\subset\R$上でも成立する.
    このときは$L^p$-ノルムの言葉で表せる.
\end{tcolorbox}

\begin{theorem}[Doob]
    $I\subset\R_+$を区間,
    $(X_t)_{t\in I}$を非負の劣マルチンゲール,または,$D$-マルチンゲールであるとする.
    このとき,
    \begin{enumerate}
        \item 任意の$p\ge1$について,$\lambda^pP[X^*\ge\lambda]\le\sup_{t\in I}E[\abs{X_t}^p]$.
        \item 任意の$p>1$について,$\norm{X^*}_p\le\frac{p}{p-1}\sup_{t\in I}\norm{X_t}_{p}$.
    \end{enumerate}
\end{theorem}
\begin{Proof}
    まず有理点$D:=\Q\cap I$に注目する.$D$には有限集合の増大列$(D_n)$で$D=\cup_{n\in\N}D_n$を満たすものが存在するため,$D_n$上のDoobの不等式に対して$n\to\infty$とすれば良い.これは,
    劣マルチンゲールの平均$E[\abs{X_t}^p]$は$t\in\R_+$について増大することによる.
    あとは,過程$X^*$が$D$-過程であることによる.
\end{Proof}
\begin{remarks}\mbox{}
    \begin{enumerate}
        \item 区間$I$が右の端点$T$を含む場合,$\sup_{t\in I}\norm{X_t}_p=\norm{X_T}_p$に注意.
        \item 連続な局所マルチンゲールについても,任意の有界区間$I$について成り立つことが解る.
    \end{enumerate}
\end{remarks}

\begin{corollary}\mbox{}
    \begin{enumerate}
        \item $p=2$のとき,$\norm{X^*}_{L^2(\Om)}\le 2\sup_{t\in I}\norm{X_t}_{L^2(\Om)}$.
        \item 一方で$\norm{X_t}_{L^2(\Om)}\le\norm{X^*}_{L^2(\Om)}$でもあるから,$X^*\in L^2(\Om)$と$\sup_{t\in I}\norm{X_t}^2<\infty$,すなわち,$X$が$L^2$-有界であることとは同値.このとき,特に$X$は一様可積分である.
        \item ただし,$L^1(\Om)$-有界なマルチンゲールが一様可積分でない,すなわち,$X^*$が可積分でないことがあり得る.
    \end{enumerate}
\end{corollary}

\begin{corollary}[可積分なマルチンゲールのなすBanach空間]\label{cor-the-space-Hp}
    \[H^p:=\Brace{X\in\M\cap C\mid X^*:=\sup_{t\in\R_+}\abs{X_t}\in L^p(\Om)}\quad(p\ge 1)\]
    はノルム$\norm{X}_{H^p}:=\norm{X^*}_p$について,Banach空間をなす.
    特に,$p>1$のとき,$H^p,\bH^p$は$M$の閉部分空間である.
\end{corollary}

\subsection{上渡回数定理}

\begin{tcolorbox}[colframe=ForestGreen, colback=ForestGreen!10!white,breakable,colbacktitle=ForestGreen!40!white,coltitle=black,fonttitle=\bfseries\sffamily,
    title=]
    ここで「上渡回数」なる特殊な確率変数に興味を集中することにする.
    劣マルチンゲールの挙動は上昇トレンドがあるので,上渡回数が評価でき,これが収束性について深い示唆をもたらす.
    というのも,
    いつまでも区間$[a,b]$付近には留まらず先に行くか,$[a,b]$内で概収束をすることが予想される.
\end{tcolorbox}

\begin{definition}[upcrossing number]\mbox{}
    \begin{enumerate}
        \item 実数$a<b$について,確率変数列$\{\sigma_1,\tau_1,\sigma_2,\tau_2,\cdots\}\subset\L(\Om)$を次のように定めると,停止時の狭義単調増加列となる:
        \begin{align*}
            \sigma_1&:=\min\Brace{n\ge 1\mid X_n\le a},&\tau_1&:=\min\Brace{n>\sigma_1\mid X_n\ge b},\\
            \sigma_{k}&:=\min\Brace{n>\tau_{k-1}\mid X_n\le a},&\tau_{k}&:=\min\Brace{n>\sigma_{k}\mid X_n\ge b}.
        \end{align*}
        \item 停止時の狭義単調増加列$\sigma_1,\tau_1,\sigma_2,\tau_2,\cdots$に対して,$\beta_n:=\max\{k\in\N\mid\tau_{k}\le n\}$と定めると,$\N$-値確率変数の列となる.
        各成分$\beta_n$を,\textbf{時刻$n$までの$[a,b]$間の上向き横断回数}という.
    \end{enumerate}
\end{definition}

\begin{theorem}
    $(X_n)$が$(\F_n)$-劣マルチンゲールならば,
    \[\forall_{N\in\N}\quad E[\beta_N]\le\frac{1}{b-a}E[(X_N-a)^+].\]
\end{theorem}
\begin{Proof}
    $k:=\beta_N$とし,
    確率変数を
    $Z_N:=\sum^k_{i=1}(X_{\sigma_{i+1}\land N}-X_{\tau_i\land N})$とおく.
    これは,$a$を2回目以降に初めて$a$を下回った時の点$X_{\sigma_{i+1}}$と,その前に初めて$b$を上回った点$X_{\tau_i}$との距離を(その後は次に$b$を越すまで計測は休憩),時刻$N$が過ぎるまで足し合わせたものである.
    なお,マルチンゲールは$D$-過程としたことに注意すると,$X_{\sigma_i}\le a,b\le X_{\tau_i}$が成り立つ.
    $Z_N$をこのように定義することで,$X_{\sigma_{i+1}}-X_{\tau_i}\le a-b<0$という形での評価が可能になる.
    \begin{description}
        \item[$Z_N$の評価] $Z_N\le\beta_N(a-b)+(X_N-a)^+$が成り立つことを示す.
        \begin{enumerate}[(a)]
            \item 最後に超えたのが$b$であるとき,
            \begin{align*}
                Z_N&=\sum^{k-1}_{i=1}(X_{\sigma_{i+1}}-X_{\tau_i})+(a-X_{\tau_k})+(X_N-a)\\
                &\le(k-1)(a-b)+(a-b)+(X_N-a)\le k(a-b)+(X_N-a)^+.
            \end{align*}
            \item 最後に超えたのが$a$であるとき,
            \[Z_N=\sum^k_{i=1}(X_{\sigma_{i+1}}-X_{\tau_i})\le k(a-b)\le k(a-b)+(X_N-a)^+.\]
        \end{enumerate}
        \item[証明] 両辺の平均値を取ると,$E[Z_N]\le-(b-a)E[\beta_N]+E[(X_N-a)^+]$となるから,$E[Z_N]\ge0$を言えば良い.
        変な話だが$E[X_{\sigma_{i+1}\land N}]\ge E[X_{\tau_i\land N}]$を示せばこれは従うから,すなわち$\sigma_i,\tau_i$が$F$-停止時であることを示せば結論が従う.

    \end{description}
\end{Proof}

\section{マルチンゲール収束定理}

\begin{tcolorbox}[colframe=ForestGreen, colback=ForestGreen!10!white,breakable,colbacktitle=ForestGreen!40!white,coltitle=black,fonttitle=\bfseries\sffamily,
title=]
    独立確率変数の和に関する収束定理はKolmogorovの不等式などにより示された.
    マルチンゲールに対するDoobの不等式はさらに強力である.
\end{tcolorbox}

\subsection{劣マルチンゲールの収束定理}

\begin{tcolorbox}[colframe=ForestGreen, colback=ForestGreen!10!white,breakable,colbacktitle=ForestGreen!40!white,coltitle=black,fonttitle=\bfseries\sffamily,
title=]
    劣マルチンゲールの正部分の列が$L^1$-有界ならば$X_n$は概収束極限を持つ.
    その証明では,劣マルチンゲールの上向き横断回数の評価が肝要になる.
\end{tcolorbox}

\begin{theorem}[\cite{Meyer} Th'm 17]
    $(\F_n)$-劣マルチンゲール$(X_n)$は$L^1$-有界であるとする:
    $\sup_{n\in\N}E[X_n^+]<\infty$\footnote{一様可積分ならば成り立つ.}.
    このとき,
    \begin{enumerate}
        \item ある可積分確率変数$X_\infty\in L^1(\Om)$に概収束する:$\lim_{n\to\infty}X_n=X_\infty\;\as$
        \item この極限$X_\infty$は$\F_\infty:=\bigvee_{n\in\N}\F_n$について可測である.
        \item さらに$(X_n)$が一様可積分ならば,$X_n\to X$は$L^1$-収束もし,さらに$(X_n)_{n\in\o{\N}}$も劣マルチンゲールになる:$\forall_{n\in\N}\;E[X_\infty|\F_n]\ge X_n$.
    \end{enumerate}
\end{theorem}
\begin{Proof}\mbox{}
    \begin{enumerate}
        \item \begin{description}
            \item[極限の存在] \[A:=\Brace{\liminf_{n\to\infty}X_n<\limsup_{n\to\infty}X_n}=\bigcup_{a<b\in\Q}\Brace{\liminf_{n\to\infty}X_n<a<b<\limsup_{n\to\infty}X_n}=:A_{a,b}.\]
            とおいて,$P[A]=0$を示せば良い.

            まず,
            区間$[a,b]$に対する時点$N$までの$X$の上渡回数$\beta_N$は
            \[E[\beta_N]\le\frac{E[\abs{X_N}]+\abs{a}}{b-a}\]
            を満たし,$(\beta_N)_{N\in\N}$の単調増大性と併せて,単調収束定理より,
            \[E\Square{\lim_{N\to\infty}\beta_N}=\lim_{N\to\infty}E[\beta_N]\le\frac{\sup_{N\in\N}E[\abs{X_N}]+\abs{a}}{b-a}<\infty.\]
            したがって,$P\Square{\lim_{N\to\infty}\beta_N}=0$を得たが,これはどういう意味か.

            いま,$\liminf_{n\to\infty}X_n<a<b<\limsup_{n\to\infty}X_n$ということは,$X_n$が$[a,b]$を無限回横切るということであるから,
            \[P[A_{a,b}]\le P\Square{\lim_{n\to\infty}\beta_N=\infty}=0.\]
            以上より,$P[A]=0$.
            \item[極限の可積分性] $X_\infty=\pm\infty$の可能性を排除する.
            Fatouの定理より,
            \[E[\abs{X_\infty}]=E[\lim_{n\to\infty}\abs{X_n}]\le\liminf_{n\to\infty}E[\abs{X_n}]<\infty.\]
        \end{description}
        \item $\F_\infty$-可測関数$X_n$の極限であるため.
        \item 
        \begin{description}
            \item[$L^1$-収束性] 任意の$\lambda>0$に対して
            \[E[\abs{X_n-X}]=E[\abs{X_n-X}1_{\Brace{\abs{X_n-X}<\lambda}}]+E[\abs{X_n-X}1_{\Brace{X_n-X}\ge\lambda}]\]
            と分解できる.一様可積分性より,第2項は十分小さくとれ,第1項はLebesgueの優収束定理が成り立つ.
            \item[条件付き期待値の列として表せること] 任意の$n\in\N$について,$\forall_{m\ge n}\;E[X_m|\F_n]\ge X_n$である.
            条件付き期待値は$L^1$-ノルム減少的であるから,$E[X_m|\F_n]\to E[X_\infty|\F_n]$が$L^1(\Om)$-収束の意味で成り立つ.
            特に,適当な部分列が存在して,$E[X_{m_p}|\F_n]\to E[X_\infty|\F_n]\;\as$
            よって,$X_n\le E[X_\infty|\F_n]\;\as$である.
        \end{description}
    \end{enumerate}
\end{Proof}
\begin{remarks}\mbox{}
    \begin{enumerate}
        \item 劣マルチンゲールの平均は単調増加であるから,特に下に有界である.従って,
        有界性条件$\sup_{n\in\N}E[X_n^+]<\infty$は,平均の一様有界性$\sup_{n\in\N}E[\abs{X_n}]<\infty$に同値.
        \item この結果は全く同様に連続な場合に拡張されるのは,$(\F_t)$が完備かつ右連続なときのみであり,このとき$D$-変形が存在することによる.
    \end{enumerate}
\end{remarks}

\begin{corollary}
    正な優マルチンゲールは,概収束極限$\lim_{t\to\infty}X_t$を持つ.
\end{corollary}

\subsection{一様可積分なマルチンゲールの特徴付け}

\begin{tcolorbox}[colframe=ForestGreen, colback=ForestGreen!10!white,breakable,colbacktitle=ForestGreen!40!white,coltitle=black,fonttitle=\bfseries\sffamily,
title=]
    Doobの収束定理から明らかに従うが,これはDoobの一般論の前からLevyによって知られていた.
\end{tcolorbox}

\begin{corollary}[Levy]
    $(X_n)\in\bF$を適合過程とする.次は同値:
    \begin{enumerate}
        \item $X_n$は一様可積分なマルチンゲールである.
        \item ある$Y\in L^1(\Om)$が存在して,$X_n=E[Y|\F_n]\;\as$と表せる.
    \end{enumerate}
\end{corollary}

\begin{theorem}
    $(X_t)$をマルチンゲールとする.次の3条件は同値:
    \begin{enumerate}
        \item $L^1(\Om)$-極限$\lim_{t\to\infty}X_t$が存在する.
        \item ある$X_\infty\in L^1(\Om)$が存在して,$(X_t)_{t\in\o{\R_+}}$もマルチンゲールである:$\forall_{t\in\R_+}\;E[X_\infty|\F_t]$を満たす.
        \item $(X_t)_{t\in\R_+}$は一様可積分である.
    \end{enumerate}
    上の条件を満たすとき,$X_\infty$は概収束極限でもある.
\end{theorem}

\subsection{$L^p$-収束について}

\begin{proposition}
    $(\F_n)$-劣マルチンゲール$(X_n)$はさらに$p>1$について$L^p(\Om)$-有界でもあるとする:$\sup_{n\in\N}E[\abs{X_n}^p]<\infty$.
    このとき,
    \begin{enumerate}
        \item ある$X_\infty\in L^p(\Om)$が存在して,$L^p(\Om)$の意味で$\lim_{n\to\infty}X_n=X_\infty$.
        \item $X_\infty\in L^p(\F_\infty)$.
        \item $\forall_{n\in\N}\;X_n\le E[X_\infty|\F_n]$.
    \end{enumerate}
\end{proposition}

\begin{remarks}
    $L^p$-有界なマルチンゲール$X$は,$X^*\in L^p(\Om)$も満たすから,自動的に$X\in\bH^p$である.
    しかし,$L^1$-有界性だけだと,一様可積分性も保証されず,$X\in\bH^1$も保証されない.
\end{remarks}

\subsection{逆向き収束定理}

\begin{definition}
    可積分な過程$(X_n)_{n\in-\N}=(\cdots,X_2,X_1,X_0)$が\textbf{$(\F_{-n})_{n\in\N}$-マルチンゲール}であるとは,
    \[\forall_{n\in\N}\;E[X_{-n}|\F_{-n-1}]= X_{-n-1}\]
    を満たすことをいう.
    これを減少する情報系$(\F_n)$に関する\textbf{$(\F_n)$-逆マルチンゲール}ともいう.
\end{definition}

\begin{theorem}[Levy's downward theorem]
    $(\F_{-n})_{n\in\N}$を部分$\sigma$-代数の単調減少列とし,
    この減少情報系に関する$(\cG_n)$-逆マルチンゲール$\{X_n\}\subset L^1(\Om)$を考える.
    \begin{enumerate}
        \item $(X_n)$は一様可積分である.
        \item ある$X_{-\infty}\in L^1(\Om)$が存在して,概収束と$L^1$-収束の意味で次が成り立つ:$\lim_{n\to\infty}X_n=X_{-\infty}\;\as$
        \item 極限$X_{-\infty}$は$\F_{-\infty}:=\bigwedge_{n\in\N}\F_{-n}$について可測で,$\forall_{n\in\N}\;X_{-\infty}=E[X_{-n}|\F_{-\infty}]$が成り立つ.
    \end{enumerate}
\end{theorem}

\begin{proposition}[劣マルチンゲールに対する逆向き収束定理]
    $(X_n)_{n\in-\N}$が$(\F_{-n})$-劣マルチンゲールで$L^1(\Om)$-有界ならば,次を満たす:
    \begin{enumerate}
        \item $(X_n)$は一様可積分である.
        \item ある$X_{-\infty}\in L^1(\Om)$が存在して,概収束と$L^1$-収束の意味で次が成り立つ:$\lim_{n\to\infty}X_n=X_{-\infty}\;\as$
        \item 極限$X_{-\infty}$は$\F_{-\infty}:=\bigwedge_{n\in\N}\F_{-n}$について可測で,$\forall_{n\in\N}\;X_{-\infty}\le E[X_{-n}|\F_{-\infty}]$が成り立つ.
    \end{enumerate}
\end{proposition}

\section{連続劣マルチンゲールのDoob-Meyer理論}

\begin{tcolorbox}[colframe=ForestGreen, colback=ForestGreen!10!white,breakable,colbacktitle=ForestGreen!40!white,coltitle=black,fonttitle=\bfseries\sffamily,
title=]
    マルチンゲール$\M\subset\bF$は,代表元として$D$-変形を取り直せる.
    こうして,$\M\cap D$は位相線型空間になる(はず).
\end{tcolorbox}

\subsection{連続時間マルチンゲールの性質}

\begin{definition}
    一般の確率空間$(\Om,\F,P)$上の増大系$(\F_t)$について,
    過程$(X_t)$が$(\F_t)$-マルチンゲールであるとは,次の3条件を満たすことをいう.
    \begin{enumerate}[({M}1)]
        \item $(\F_t)$-適合である:$\forall_{t\ge0}\;X_t$は$\F_t$-可測.
        \item 可積分である:$\forall_{t\ge0}\;X_t\in\L^1(\Om,\F)$.
        \item $\forall_{0\le s\le t}\;E[X_t|\F_s]=X_s\;\as$
    \end{enumerate}
    条件(3)の代わりに$\forall_{0\le s\le t}\;E[X_t|\F_s]\ge X_s\;\as$をみたすとき,$(\F_t)$-劣マルチンゲールという.
\end{definition}
\begin{remark}[Jacod-Shiryaev\cite{Jacod-Shiryaev}]
    確率基底は完備とはしなかった.
    これは次の成立による.$X$を完備化$(\Om,\F^P,P,\bF^P)$上の劣マルチンゲールとすると,$X$と識別不可能な$\bF$-適合過程$X'$が存在して,ある$\bF$-停止時$T\in\bT^{<\infty}(\bF)$について,任意の見本道$X_\bullet(\om)$は右連続で,$T(\om)$を除いて左極限を持つ.
\end{remark}

\begin{proposition}
    $(X_t)$を劣マルチンゲールとする.$t\mapsto E[X_t]$は単調増加である.
\end{proposition}

\subsection{連続マルチンゲールと一様可積分性}

\begin{proposition}[非負な劣マルチンゲールの一様可積分性]
    $\{X_t\}\subset L^1(\Om)_+$を(殆ど確実に)非負な$(\F_t)$-劣マルチンゲールとする.
    \begin{enumerate}
        \item 任意の$N\in\N$に対して,$(X_t)_{t\in[0,N]}$は一様可積分である.
        \item $[0,N]$上に値を取る離散な$(\F_t)$-停止時の全体を$\Sigma_N$とすると,$(X_\sigma)_{\sigma\in\Sigma_N}$も一様可積分である.
        \item $(X_t)$は$D$-過程でもあるならば,任意の有界な$(\F_t)$-停止時$\sigma\le\tau\le N$に対して,$E[X_\tau|\F_\sigma]\ge X_\sigma$.
        \item $[0,N]$上に値を取る一般の有界な停止時の全体を$\bT^{<N}$とする.$(X_\sigma)_{\sigma\in\bT^{<N}}$も一様可積分である.
    \end{enumerate}
\end{proposition}

\begin{corollary}
    任意の$(\F_t)$-マルチンゲールは,上の命題の性質を全て満たす.
\end{corollary}
\begin{Proof}
    $\Brace{\abs{X_t}}$が非負な$(\F_t)$-劣マルチンゲールであるため.
\end{Proof}

\subsection{Doobの任意停止定理}

\begin{proposition}[Doob's stopping theorem]
    $X\in\bF\cap D$について,次は同値:
    \begin{enumerate}
        \item $X$は$\bF$-マルチンゲールである.
        \item 任意の有界な停止時$T\in\bT(\bF)$について,$X_T\in L^1(\Om)$かつ$E[X_T]=E[X_0]$.
    \end{enumerate}
\end{proposition}
\begin{remarks}
    $X\in\bF\cap D$は局所有界とすれば,$X_T\in L^1(\Om)$を満たすような有界な停止時$T\in\bT(\bF)$についてのみ$E[X_T]=E[X_0]$を満たすことが特徴付けになる.
\end{remarks}

\begin{corollary}
    $M$をマルチンゲール,$T$を停止時とする.$M^T$は再び$(\F_{T\land t})$-マルチンゲールであり,また$(\F_t)$-マルチンゲールでもある.
\end{corollary}
\begin{Proof}
    $M^T\in\bF\cap D$は明らか.任意の有界な停止時$S$について,$S\land T$も有界だから,任意停止より,
    \[E[M_S^T]=E[M_{S\land T}]=E[M_0]=E[M_0^T].\]
\end{Proof}

\subsection{Doobの任意抽出定理}

\begin{theorem}[任意抽出定理]
    $X$を一様可積分なマルチンゲールとする($D$-変形を取れるので,$D$-過程であると仮定してもよい).
    \begin{enumerate}
        \item $(X_S)_{S\in\bT}$は一様可積分である.
        \item 任意の停止時$S\le T\in\bT$について,$X_S=E[X_T|\F_S]=E[X_\infty|\F_S]\;\as$
    \end{enumerate}
\end{theorem}
\begin{remarks}[有界な停止時の場合]
    上の結果は全てこの定理の系である.有界な停止時$S\le T$の場合は,マルチンゲールも$(X_t)_{t\in[0,N]}$を取ってよく,右に閉じている実区間上のマルチンゲールは一様可積分である.
\end{remarks}


\begin{example}[一様可積分でないマルチンゲールは任意停止に違反する]
    $X_t=e^{B_t-t/2}$は一様可積分でない.
    \[T:=\inf\Brace{t\in\R_+\mid X_t\le\al}\;(\al<1)\]
    とすると,$E[X_T]=\al$であるが,$E[X_0]=1$である.
\end{example}

\begin{corollary}
    $X$を劣マルチンゲール,$S\le T\in\bT^{<\infty}(\bF)$を有界な停止時とする.
    \begin{enumerate}
        \item $E[X_T|\F_S]\ge X_S$.
        \item 停止過程$X^T:=X_{t\land T}$は$(\F_{t\land T})$-劣マルチンゲールでも,$(\F_t)$-劣マルチンゲールでもある.
    \end{enumerate}
\end{corollary}

\subsection{Doob-Meyerの分解}

\begin{tcolorbox}[colframe=ForestGreen, colback=ForestGreen!10!white,breakable,colbacktitle=ForestGreen!40!white,coltitle=black,fonttitle=\bfseries\sffamily,
title=]
    これは確率過程の一般理論から到達される消息である.
\end{tcolorbox}

\begin{theorem}
    $Z\in{}^0\!\M\cap(\D)$を劣マルチンゲールとする.
    一意的に$M\in {}^0\!M\cap D$と可予測な$A\in{}^0\!\A^+$が存在して,一意的に$Z=M+A$と分解できる.
\end{theorem}

\subsection{Doobのマルチンゲール収束定理}

\begin{theorem}
    $X$を優マルチンゲールで,ある$Y\in L^1(\Om)$について$\forall_{t\in\R_+}\;X_t\ge E[Y|\F_t]$を満たすとする.
    このとき,ある実確率変数$X_\infty$が存在して,$X_t\asto X_\infty$.
\end{theorem}

\section{二次変動とDoob-Meyerの分解}

\subsection{全変動と二次変動}

\begin{notation}
    $X:\R_+\to\R$を関数,$t\in\R_+$とする.
    \begin{enumerate}
        \item 分割$\pi_n\subset[0,t]$について,
        \[S^{\pi_n}_t:=\sum_{i\in[n]}\abs{X_{t_{i+1}}-X_{t_i}}.\]
        $S_t^-$は,分割の細分に関する順序を保つ:$\pi'\subset\pi\Rightarrow S^{\pi'}_t\ge S^{\pi}_t$.
        \item 一方で,
        \[T^{\pi_n}_t:=\sum_{i\in[n]}(X_{t_{i+1}}-X_{t_i})^2.\]
        は特に順序を保存するわけではない.
    \end{enumerate}
\end{notation}

\begin{definition}[total variation, quadratic variation]
    実過程$X$について,
    \begin{enumerate}
        \item 任意の$t\in\R_+$に対して$S_t:=\sup_{\pi}S^\pi_t<\infty$が成り立つこと,すなわちネットの極限として関数$S:\R_+\to\R_+$が定まることを,(有限な)\textbf{全変動}を持つという.
        \item $S:\R_+\to\R_+$は単調増加であるが,これがさらに有界関数でもあるとき,関数$X$は\textbf{有界変動}であるという.
        \item 実過程$X$が\textbf{二次変分}を持つとは,ある過程$[X]:=\brac{X,X}$が存在して,任意の$t\in\R_+$と$[0,t]$の分割の列$(\pi_n),\abs{\pi_n}\to0$について,
        \[T_t^{\pi_n}\xrightarrow[n\to\infty]{P}\brac{X,X}_t\]
        を満たすことをいう.
    \end{enumerate}
\end{definition}
\begin{remark}
    ある過程が二次変分を持っても,$\forall_{t>0}\;\sup_{\pi}T^\pi_t=\infty$を満たし得る.
    例えばBrown運動がそうである.
\end{remark}

\begin{proposition}\mbox{}
    \begin{enumerate}
        \item 任意の有限な全変動を持つ関数は,2つの単調増加関数の差で表せる.
        \item 特に,任意の有限変動関数は左極限$A_{t-}$を持つ.
    \end{enumerate}
\end{proposition}
\begin{Proof}
    関数$A$は全変動$S$を持つとすると,
    \[\frac{S+A}{2},\quad\frac{S-A}{2}\]
    はいずれも単調増加関数である.
\end{Proof}

\subsection{有限変動関数の積分の復習}

\begin{theorem}
    Radon測度$\mu\in\RM(\R_+)$と右連続な有限変動関数$A$とは次の形で一対一対応する:$\forall_{t\in\R_+}\;A_t=\mu([0,t])$.
\end{theorem}

\begin{definition}
    $f\in L^\infty_\loc(\R_+)$を局所有界なBorel関数とする.この$A$に対する\textbf{Stieltjes積分}を$\int^t_0f_sdA_s$または$(f\cdot A)_t$で表す.
\end{definition}

\begin{theorem}[絶対連続関数の特徴付け]
    有限な変動を持つ関数$A$について,
    \begin{enumerate}
        \item $A$は殆ど至る所微分可能である.
        \item ある有限な変動を持つ関数$B$であって$B'=0\;\ae$を満たすものが存在して,
        \[A_t=B_t+\int^t_0A'_sds.\]
    \end{enumerate}
    $B=0$であるとき,有限変動関数$A$は\textbf{絶対連続}であるといい,この場合は$\mu\ll l$である.
\end{theorem}

\begin{proposition}
    $A,B$を有限変動関数とすると,
    \[\forall_{t\in\R_+}\quad A_tB_t=A_0B_0+\int^t_0A_sdB_s+\int^t_0B_{s-}dA_s.\]
\end{proposition}

\subsection{増加過程とマルチンゲール}

\begin{tcolorbox}[colframe=ForestGreen, colback=ForestGreen!10!white,breakable,colbacktitle=ForestGreen!40!white,coltitle=black,fonttitle=\bfseries\sffamily,
title=]
    増加過程に関する積分はパス毎にStieltjes積分として定義できる.
    一般のマルチンゲールへの一般化が肝要である.
    その際に,全変動がないので,二次変動が肝要になる.
\end{tcolorbox}

\begin{definition}
    適合過程$A\in\bF$が
    \begin{enumerate}
        \item 増加過程であるとは,見本道が殆ど確実に右連続で単調増加であることをいう.
        \item 有界変動であるとは,見本道が殆ど確実に右連続で有界変動であることをいう.
    \end{enumerate}
\end{definition}

\begin{proposition}
    $(\M\cap C)\cap\A$の元は定数のみである.
\end{proposition}

\subsection{マルチンゲールの二次変動の存在と特徴付け}

\begin{theorem}\mbox{}\label{thm-quadratic-variation-for-bounded-martingale}
    \begin{enumerate}
        \item 連続なマルチンゲール$M\in C\cap\bF$を有界とすると,これは有限な二次変動$\brac{M,M}$を持つ.
        \item 過程$\brac{M,M}\in C\cap{}^0\!\A^+$は,次の性質を持つ唯一の$\brac{M,M}_0=0$から始まる連続な増加過程として特徴付けられる:$M^2-\brac{M,M}\in\M$.
    \end{enumerate}
\end{theorem}

\begin{lemma}
    任意の停止時$T\in\bT(\bF)$について,$\brac{M^T,M^T}=\brac{M,M}^T$.
\end{lemma}

\section{局所マルチンゲールと二次共変動}

\begin{tcolorbox}[colframe=ForestGreen, colback=ForestGreen!10!white,breakable,colbacktitle=ForestGreen!40!white,coltitle=black,fonttitle=\bfseries\sffamily,
title=]
    任意の連続な局所マルチンゲール$M\in M_\loc\cap C$に伴う$M^2$は,$(M|M)_0=0$から始まる単調増加成分$(M|M)\in {}^0\!\A^+\cap C$と局所マルチンゲール成分とに一意に分解できる.
    この消息と極化を通じて,任意の連続な局所マルチンゲール$M,N\in M_\loc\cap C$に対して,一意な有限変動成分$(M|N)\in{}^0\A$が取り出せる:$MN-(M|N)\in M_\loc$.
\end{tcolorbox}

\subsection{局所マルチンゲールの定義と構成}

\begin{definition}[local martingale]
    $X\in\bF\cap D$が\textbf{$(\F_t),P$-局所マルチンゲール}であるとは,次を満たす停止時の増大列$\{T_n\}\subset\bT(\bF)$が存在することをいう:
    \begin{enumerate}
        \item $\lim_{n\to\infty}T_n=\infty\;\as$
        \item $\forall_{n\in\N}\;X^{T_n}1_{\Brace{T_n>0}}\in M(\bF)$.
    \end{enumerate}
\end{definition}
\begin{remarks}\mbox{}
    \begin{enumerate}
        \item (2)の$X^{T_n}1_{\Brace{T_n>0}}$が一様可積分であるという仮定は省ける.
        というのも,$(T_n)$を$(T_n\land n)$に取り替えれば,全ての$X^{T_n\land n}1_{\Brace{T_n\land n>0}}$は有界区間$[0,n]$上で定義されていると見れて,これがマルチンゲールならば自動的に一様可積分である.
        \item さらに言えば,$S_n:=\Brace{t\in\R_+\mid\abs{X_t}=n}$として,$T_n$を$(T_n\land S_n)$に取り替えると,各$X^{T_n\land S_n}1_{\Brace{T_n\land S_n>0}}$は有界でもある.
        \item $\M\cap D\subset M_\loc$であるが,$\M\not\subset M_\loc$である.$T_n=n$と取れば良い.
    \end{enumerate}
\end{remarks}

\begin{proposition}[マルチンゲールに還元する停止時について]
    $X$を局所マルチンゲールとする.
    \begin{enumerate}
        \item $X^T1_{\Brace{T>0}}\in M$.
        \item $X_01_{\Brace{T>0}}$は可積分であり,$Y_t:=X_t-X_0$について$Y^T\in M$.
    \end{enumerate}
\end{proposition}

\begin{corollary}[局所マルチンゲールの構成]
    $X\in M_\loc$を局所マルチンゲールとする.
    \begin{enumerate}
        \item $T$が$X$を還元するならば,任意の$S\le T$も還元する.
        \item 2つの局所マルチンゲールの和は局所マルチンゲールである.
        \item $Z\in L(\F_0)$ならば,$ZX\in M_\loc$.特に,$M_\loc$は線型空間である.
        \item 局所マルチンゲールの停止過程は局所マルチンゲールである.
    \end{enumerate}
\end{corollary}

\begin{proposition}
    非負な局所マルチンゲールは優マルチンゲールである.
\end{proposition}

\subsection{マルチンゲールになるための条件}

\begin{tcolorbox}[colframe=ForestGreen, colback=ForestGreen!10!white,breakable,colbacktitle=ForestGreen!40!white,coltitle=black,fonttitle=\bfseries\sffamily,
title=]
    $(\D)\subset M$であり,
    $M_\loc\cap (\D)=M$.
\end{tcolorbox}

\begin{definition}[Dirichlet class]
    過程$X\in\bF$が
    \begin{enumerate}
        \item \textbf{Dirichletクラス}であるとは,族
        $(X_T)_{T\in\bT^{<\infty}}$が一様可積分であることをいう.
        これを$X\in(\D)$で表す.
        \item \textbf{DLクラス}であるとは,
        任意の$a>0$に対して,族$(X_T)_{T\in\bT^{\le a}}$が一様可積分であるという.
        これを$X\in(\DL)$で表す.
    \end{enumerate}
\end{definition}

\begin{theorem}
    $X\in M_\loc$を局所マルチンゲールとする.
    \begin{enumerate}
        \item $X$は一様可積分なマルチンゲールである:$X\in M$.
        \item $X$はDL類である.
    \end{enumerate}
\end{theorem}

\subsection{二次変動と二次共変動の定義}

\begin{theorem}
    $M\in M_\loc\cap C$を連続な局所マルチンゲールとする.
    \begin{enumerate}
        \item 唯一の$\brac{M,M}_0=0$から始まる連続な増加過程$\brac{M,M}\in{}^0\!\A^+\cap C$が存在して,
        $M^2-\brac{M,M}\in M_\loc\cap C$を満たす.
        \item 任意の$t\in\R_+$と分割の列$\pi_n\subset[0,t],\abs{\pi_n}\to0$について,$\sup_{s\in[0,t]}\abs{T_s^{\pi_n}(M)-\brac{M,M}_s}\xrightarrow{P}0$.
    \end{enumerate}
\end{theorem}

\begin{corollary}
    $M$がマルチンゲールならば,収束$\lim_{n\to\infty}T^{\pi_n}\to(X|X)$は$L^1(\Om)$-収束の意味でも成り立つ.
\end{corollary}

\begin{theorem}
    $M,N\in M_\loc\cap C$を連続な局所マルチンゲールとする.
    \begin{enumerate}
        \item 唯一の有限変動過程$\brac{M,N}\in{}^0\!\A$が存在して,$MN-\brac{M,N}\in M_\loc$を満たす.
        \item 任意の$t\in\R_+$と分割の列$\pi_n\subset[0,t],\abs{\pi_n}\to0$について,$\sup_{s\in[0,t]}\abs{\wt{T}_s^{\pi_n}(M)-\brac{M,N}_s}\xrightarrow{P}0$.
    \end{enumerate}
    ただし,
    \[\wt{T}_s^{\pi_n}:=\sum_{i\in[n]}(M^s_{t_{i+1}}-M^s_{t_i})(N^s_{t_{i+1}}-N^s_{t_i}).\]
\end{theorem}
\begin{Proof}
    極化恒等式から,
    \[\brac{M,N}:=\frac{1}{4}\paren{(M+N|M+N)-(M-N|M-N)}\]
    と定めれば良い.
    一意性は,適当な停止と有界なマルチンゲールに対する二次変動の一意性\ref{thm-quadratic-variation-for-bounded-martingale}による.
\end{Proof}

\subsection{二次共変動の性質}

\begin{proposition}
    $T\in\bT$をMarkov時とすると,$(M^T|N^T)=(M|N^T)=(M|N)^T$.
\end{proposition}

\begin{proposition}[内積を定める]\mbox{}
    \begin{enumerate}
        \item $(-|-):(M_\loc\cap C)^2\to{}^0\A$は対称な双線型で,$(M|M)\ge0$を満たす.
        \item さらに,非退化である.すなわち,$(M|M)=0\Leftrightarrow\forall_{t\in\R_+}\;M_t=M_0\;\as$.
    \end{enumerate}
\end{proposition}

\begin{proposition}[強い非退化性]
    $M\in M_\loc$について,次は同値:
    \begin{enumerate}
        \item $[a,b]$上で$\forall_{t\in[a,b]}\;M_t=M_a\;\as$
        \item $(M|M)_b(\om)=(M|M)_a(\om)$.
    \end{enumerate}
\end{proposition}

\begin{proposition}[独立性と二次共変動]
    $M,N\in M_\loc\cap C$が互いに独立ならば,
    \begin{enumerate}
        \item $(M|N)=0$.
        \item $MN$は局所マルチンゲールである.
    \end{enumerate}
\end{proposition}

\subsection{二次共変動の不等式}

\begin{tcolorbox}[colframe=ForestGreen, colback=ForestGreen!10!white,breakable,colbacktitle=ForestGreen!40!white,coltitle=black,fonttitle=\bfseries\sffamily,
title=]
    Schwartzの不等式も,Holderの不等式も,類似が成り立つ.
\end{tcolorbox}

\begin{proposition}
    $M,N\in M_\loc\cap C$と,可測過程$H,K$について,任意の$t\in[0,\infty]$について,
    \[\int^t_0\abs{H_s}\abs{K_s}\abs{d(M|N)}_s\le\paren{\int^t_0H_s^2d(M|M)_s}^{1/2}\paren{\int^t_0K_s^2d(N|N)_s}^{1/2}\;\as\]
\end{proposition}

\begin{corollary}[Kunita-Watanabe]
    任意の$p\ge1$について,
    \[E\Square{\int^\infty_0\abs{H_s}\abs{K_s}\abs{d(M|N)}_s}\le\Norm{\paren{\int^\infty_0H_s^2d(M|M)_s}^{1/2}}_p\Norm{\paren{\int^\infty_0K_s^2d(N|N)_s}^{1/2}}_{p^*}.\]
\end{corollary}

\subsection{積率不等式}

\begin{tcolorbox}[colframe=ForestGreen, colback=ForestGreen!10!white,breakable,colbacktitle=ForestGreen!40!white,coltitle=black,fonttitle=\bfseries\sffamily,
title=]
    Doobの不等式\ref{thm-Doob-inequality}を,マルチンゲールの$p$次のモーメントに関する評価式に書き直せる.
\end{tcolorbox}

\begin{notation}
    $\{M_n\}\subset\L^2(\Om)$を,$M_0=0$から始まる2乗可積分なマルチンゲールとする.
    このとき,$\forall_{n\in\N}\;E[M_n]=0$に注意.
\end{notation}

\begin{definition}
    $p\ge 1$について,
    マルチンゲール$(M_n)$の\textbf{$p$次変分}または\textbf{$p$次変動}とは,次で定まる実数列$([M]_n)$をいう:
    \[[M]_n:=\sum^n_{k=1}\abs{M_k-M_{k-1}}^p.\]
    特に$p=1$のとき,\textbf{変分}あるいは\textbf{全変動}という.
    この記法は特に$p=2$のときに用いる.
\end{definition}

\begin{proposition}
    2次変分$[M]_n$は,次の2条件をみたす:
    \begin{enumerate}
        \item $(M_n^2-[M]_n)$はマルチンゲールである.
        \item $([M]_n)$は増加過程である:$0=[M]_0\le[M]_1\le\cdots$.
    \end{enumerate}
\end{proposition}

\begin{remark}
    $(M^2_n)$は劣マルチンゲールだから,Doob分解$M_n^2=N_n+A_n$を持つ.このとき,$(M_n^2-A_n)$はマルチンゲールであるが,$(A_n)$も命題の2条件を満たす.
    $(A_n)$も$(M_n)$の2次変分と呼び,$(\brac{M}_n)$で表す.
    $(\brac{M}_n)$は可予測でもあるが,一般に$([M]_n)$はそうではない.明確な区別が必要である.
    一方で,連続マルチンゲールにおいては,2つの概念は1つに退化する.
\end{remark}

\begin{theorem}[Burkholder-Davis-Gundy]
    $(M_n)$を$M_0=0$を満たす$p$乗可積分なマルチンゲールとする.
    このとき,次が成り立つ:
    \[\forall_{p\ge 1}\;\exists_{c_p,C_p>0}\quad c_pE\Square{[M]^{p/2}_n}\le E\Square{\max_{1\le k\le n}\abs{M_k}^p}\le C_pE\Square{[M]^{p/2}_n}.\]
\end{theorem}
\begin{remarks}
    右辺は,Doobの不等式の$E[\abs{M_n}^p]$を$E[[M]_n^{p/2}]$で置き換えたものになっている.$p=2$のとき両者は一致するが,応用上は2次変分の方が計算しやすいことが多い.
\end{remarks}

\section{$L^2$-有界なマルチンゲール}

\subsection{$L^2$-有界なマルチンゲールのなすHilbert空間}

\begin{observation}
    有界なマルチンゲールは$\bH^2$に入る.
    Doobの$L^2$-不等式より,$M\in\bH^2$ならば$M^*_\infty\in L^2(\Om)$である.
    特に,$M$は一様可積分で,ある$M_\infty\in L^2(\F_\infty)$について$M_t=E[M_\infty|\F_t]$と表せる.
\end{observation}

\begin{theorem}
    $\bH^2$はノルム$\norm{M}_{\bH^2}:=E[M^2_\infty]^{1/2}=\lim_{t\to\infty}E[M_t^2]^{1/2}$についてHilbert空間をなし,$H^2$はその閉部分空間をなす.
\end{theorem}

\begin{remark}
    $\norm{M^*_\infty}_2=E\Square{\paren{\sup_{t\in\R_+}\abs{M_t}}^{2}}^{1/2}$は,Doobの不等式より$\norm{M^*_\infty}_2\le2\norm{M}_{H^2}$でもあるから,上のノルムに同値である.
    が,これはHilbert空間を定めない.
\end{remark}

\subsection{二次変動}

\begin{proposition}
    $M\in M_\loc\cap C$について,次は同値:
    \begin{enumerate}
        \item $M\in H^2$でもある.
        \item $M_0\in L^2(\Om)$かつ$(M|M)_\infty\in L^1(\Om)$.
    \end{enumerate}
    このとき,$M^2-(M|M)\in M$は一様可積分なマルチンゲールで,任意の$S\le T\in\bT$について,
    \[E[M^2_T-M^2_S|\F_S]=E[(M_T-M_S)^2|\F_S]=E[(M|M)^T_S|\F_S].\]
\end{proposition}

\begin{corollary}
    $M\in {}^0\!H^2$のとき,
    \[\norm{M}_{\bH^2}=\norm{(M|M)^{1/2}_\infty}_2\equiv E[(M|M)_\infty]^{1/2}.\]
\end{corollary}

\begin{proposition}
    $M\in M_\loc\cap C$と$t\in\R_+$について,次は同値:
    \begin{enumerate}
        \item $(M_s)_{s\in[0,t]}$は$L^2$-有界なマルチンゲールである.
        \item $M_0\in L^2(\Om)$かつ$(M|M)_t\in L^1(\Om)$.
    \end{enumerate}
\end{proposition}

\subsection{収束定理}

\begin{proposition}
    $M\in M_\loc\cap C$について,$1_{(M|M)_\infty<\infty}M$は概収束する.
\end{proposition}

\section{$L^p$-有界なマルチンゲールと積率不等式}

\begin{tcolorbox}[colframe=ForestGreen, colback=ForestGreen!10!white,breakable,colbacktitle=ForestGreen!40!white,coltitle=black,fonttitle=\bfseries\sffamily,
title=]
    $\bH^2$のノルム$\norm{M}_{\bH^2}=E[M^2_\infty]^{1/2}$は,部分空間${}^0\!H^2$では$\norm{M}_{\bH^2}=\norm{(M|M)_\infty^{1/2}}_2=E[(M|M)_\infty]^{1/2}$と特徴付けられる.
    これは$\norm{M^*_\infty}_2$と同値になる.
\end{tcolorbox}

\subsection{BDG-不等式の主張}

\begin{theorem}[Burkholder-Davis-Gundy (72)]
    任意の$p\in(0,\infty)$について,ある定数$c_p,C_p$が存在して,任意の$M\in {}^0\!M_\loc\cap C$に対して,
    \[c_pE[(M|M)_\infty^{p/2}]\le E[(M^*_\infty)^p]\le C_pE[(M|M)^{p/2}_\infty].\]
\end{theorem}
\begin{remarks}
    この定理は,$M^*_\infty\in L^p(\Om)$を満たす連続マルチンゲールのBanach空間$H^p$上のノルム$\norm{M}_{H^p}^p=E[(M^*_\infty)^p]$と同値なノルム$E[(M|M)^{p/2}_\infty]$を与えている,と見れる.
    $p\ge1$について$H^p\subset\M$で,$p>1$のとき$H^p$の元は$L^p$-有界な連続マルチンゲールと解せる.
    $H^1$の元は$L^1$-有界連続マルチンゲールの空間よりも,$M$よりも真に小さい.
\end{remarks}
\begin{history}
    $p\in(1,\infty)$の場合についてBurkholder (1966)が示す.続いてBurkholder and Gundy (1970)で$0<p\le 1$の場合を部分的に解決し,Davis (1970)が$p=1$の場合を全てのマルチンゲールについて示した.
    最後に,Burkholder, Davis, Gundy (1972)が上の形にまとめた.
\end{history}

\begin{corollary}
    任意の停止時$T\in\bT$に対して,
    \[c_pE[(M|M)^{p/2}_T]\le E[(M^*_T)^p]\le C_pE[(M|M)^{p/2}_T].\]
\end{corollary}

\subsection{Hilbert空間値の局所マルチンゲールについて}

\begin{definition}
    $(M_t)_{t\in[0,T]}$を可分Hilbert空間$H$に値を取る連続な局所マルチンゲールとする.
    この二次変分過程は,正規直交基底$(e_n)_{n\in\N}$についての成分毎に
    \[[M]_t:=\sum_{n\in\N}[(M|e_n)]_t,\quad t\in\R_+\]
    と定める.ただし,$(M|e_n)$は$H$の内積である.
\end{definition}

\begin{theorem}
    $(M_t)_{t\in[0,T]}$を可分Hilbert空間$H$に値を取る連続な局所マルチンゲールとする.
    このとき,任意の$p>0$に対して,$c_p>0$が存在して,
    \[E[\norm{M_t}^p_H]\le c_pE[(M|M)_T^{p/2}].\]
\end{theorem}

\section{一様可積分なマルチンゲール:$L^1$空間}

\begin{tcolorbox}[colframe=ForestGreen, colback=ForestGreen!10!white,breakable,colbacktitle=ForestGreen!40!white,coltitle=black,fonttitle=\bfseries\sffamily,
title=]
    $\F$-可測な可積分な確率変数を
    情報系$(\F_t)$で条件づけたものは一様可積分なマルチンゲールを定める.
    実は$\M$は全てこのようにして定まり,$\D$変形を取れば一意性を得る:$\M^D\simeq_\Set L^1(\Om,\F_\infty)$.
\end{tcolorbox}

\subsection{一様可積分なマルチンゲールの構造}

\begin{theorem}[マルチンゲールが一様可積分であることの特徴付け]
    $X\in\oM$について,次は同値:
    \begin{enumerate}
        \item 一様可積分である:$X\in\M$.
        \item ある$X_\infty\in L^1(\Om,\F_\infty)$が存在して,$\lim_{t\to\infty}E[\abs{X_t-X_\infty}]=0$.
        \item ある$X_\infty\in L^1(\Om,\F_\infty)$が存在して,$\forall_{t\in\R_+}\;X_t=E[X_\infty|\F_t]\;\as$
    \end{enumerate}
    なおこのとき,さらに$X\in\M^D$でもあるなら,次も成り立つ:
    \begin{enumerate}[(a)]
        \item $X_t\to X_\infty\;\as$
        \item $\forall_{\sigma,\tau\in\T}\;X_{\tau\land\sigma}=E[X_\sigma|\F_\tau]\;\as$
    \end{enumerate}
\end{theorem}

\begin{theorem}
    $Y\in L^1(\Om)$について,識別不可能な違いを除いて唯一のマルチンゲール$X\in\M^D$が存在して,
    $X_t=E[Y|\F_t]$を満たす.
\end{theorem}

\begin{theorem}
    $\{\tau_n\}\subset\T$をMarkov時刻の増大列とする.このとき,$\lim_{n\to\infty}X_{\tau_n}=E[X_{\lim_{n\to\infty}\tau_n}|\lor_{n\in\N}\F_{\tau_n}]\;\as$
    特に,$\tau$が可予測ならば,$X_{\tau-}=E[X_\tau|\F_{\tau-}]$.
\end{theorem}

\subsection{二次変動過程}

\begin{tcolorbox}[colframe=ForestGreen, colback=ForestGreen!10!white,breakable,colbacktitle=ForestGreen!40!white,coltitle=black,fonttitle=\bfseries\sffamily,
title=]
    中心化された連続な局所マルチンゲールは,連続な増加過程と局所マルチンゲールとに分解できる:$\M^{0,C}_\loc=\A^{+,0,C}+\M_\loc$.
    これを擬似的なノルム(の二乗)として,極化を通じて共分散に当たる量を定義できる.
\end{tcolorbox}

\begin{theorem}[Nualart\cite{Nualart-Introduction}]
    中心化された連続な局所マルチンゲール$M\in\M^{0,C}_\loc$について,次の条件を満たすただ一つの中心化された連続な増加過程$\brac{M}\in\A^{+,0,C}$が存在して,
    \[\forall_{t\in\R_+}\;\forall_{\pi_n:=\Brace{0=t_0<t_1<\cdots<t_n=t}}\;\sum_{j\in n}(M_{t_{j+1}}-M_{t_j})^2\xrightarrow[\abs{\pi_n}\to0]{P}\brac{M}_t.\]
    ただし,$M^2_t-\brac{M}_t\in\M_\loc$.
\end{theorem}

\begin{definition}[quadratic covariation]
    $M,N\in\M^{0,C}_\loc$について,\textbf{二次共変分}とは,次の極化恒等式によって定まる過程
    \[\brac{M,N}_t:=\frac{1}{4}(\brac{M+N}_t-\brac{M-N}_t)\]
    をいう.
\end{definition}

\begin{Proof}
    この極限$\brac{X,Y}$が存在するならば,その線形性より,
    \[\brac{X|Y}=\brac{M^X|M^Y}+\brac{M^X|A^Y}+\brac{A^X|M^Y}+\brac{A^X|A^Y}\]
    である.後ろの3項が零であることを示す.
    \begin{enumerate}
        \item $A^X,A^Y$は$t\in\R_+$について絶対連続であるから特に$[0,t]$上一様連続であり,それと三角不等式より,
        \begin{align*}
            \sum_{j\in n}(A^X_{t_{j+1}}-A^X_{t_j})(A^Y_{t_{j+1}}-A^Y_{t_j})&\le\sup_{\abs{s-u}\le\abs{\pi_n},s,u\in[0,t]}\abs{A^X_u-A^X_s}\sum_{j\in n}\abs{A^Y_{t_{j+1}}-A^Y_{t_j}}\\
            &\le\sup_{\abs{s-u}\le\abs{\pi_n},s,u\in[0,t]}\abs{A^X_u-A^X_s}\int^t_0\abs{v_s^Y}ds.
        \end{align*}
        $v_s^Y\in L^1_\loc(\P)$より,右辺は任意の$t\in\R_+$について有限.よって,$\abs{\pi_n}\to0$の極限で,
        \[\sum_{j\in n}(A^X_{t_{j+1}}-A^X_{t_j})(A^Y_{t_{j+1}}-A^Y_{t_j})\xrightarrow{P}0.\]
        よって$\brac{A^X|A^Y}$は存在し,$=0$.
        \item $M^X$は$t\in\R_+$について連続であるから,特に$[0,t]$上一様連続であるから,同様にして
        \begin{align*}
            \sum_{j\in n}(M^X_{t_{j+1}}-M^X_{t_j})(A^Y_{t_{j+1}}-A^Y_{t_j})&\le\sup_{\abs{s-u}\le\abs{\pi_n},s,u\in[0,t]}\abs{M^X_u-M^X_s}\sum_{j\in n}\abs{A^Y_{t_{j+1}}-A^Y_{t_j}}\\
            &\le\sup_{\abs{s-u}\le\abs{\pi_n},s,u\in[0,t]}\abs{M^X_u-M^X_s}\int^t_0\abs{v_s^Y}ds.
        \end{align*}
        同様にして,$\brac{M^X|A^Y}=\brac{A^X|M^Y}=0$.
    \end{enumerate}
    以上より,
\end{Proof}

\section{連続なマルチンゲール}

\subsection{任意停止定理}

\begin{tcolorbox}[colframe=ForestGreen, colback=ForestGreen!10!white,breakable,colbacktitle=ForestGreen!40!white,coltitle=black,fonttitle=\bfseries\sffamily,
title=]
    次の定理はマルチンゲールにも,列マルチンゲールにも成り立つ.
\end{tcolorbox}

\begin{theorem}[Optional stopping theorem]
    $M\in\M^C$を連続マルチンゲール,$S\le T\in\T$を有界な停止時とする.このとき,$E[M_T|\F_S]=M_S$.
    特に,$E[M_T]=E[M_S]$.
\end{theorem}

\begin{corollary}
    $M\in\M$をマルチンゲール,$T\in\T$を有界な停止時とする.このとき,$M^T$は再びマルチンゲールである.
\end{corollary}

\subsection{最大不等式}

\begin{theorem}[Doob]
    $M\in\M_\loc^C$を$T\in\R_+$において$p\ge1$乗可積分とする:$E[\abs{M_T}^p]<\infty$.
    このとき,
    \begin{enumerate}
        \item $\forall_{\lambda>0}\;P\Square{\sup_{t\in[0,T]}\abs{M_t}>\lambda}\le\frac{1}{\lambda^p}E[\abs{M_T}^p]$.
        \item $p>1$ならば,次も成り立つ:$E\Square{\sup_{t\in[0,T]}\abs{M_t}^p}\le\paren{\frac{p}{p-1}}^pE[\abs{M_T}^p]$.
    \end{enumerate}
\end{theorem}

\subsection{BDG-不等式}

\begin{tcolorbox}[colframe=ForestGreen, colback=ForestGreen!10!white,breakable,colbacktitle=ForestGreen!40!white,coltitle=black,fonttitle=\bfseries\sffamily,
title=Burkholder-David-Gundy不等式]
    最大不等式の(2)を精緻化して,二次過程の値を用いて抑えると,背後のHilbert空間的な構造が見えてくる.
\end{tcolorbox}

\begin{theorem}
    $M\in\M_\loc^C$を$T\in\R_+$において$p>0$乗可積分とする:$E[\abs{M_T}^p]<\infty$.
    このとき,
    \[\exists_{c<C\in\R}\;cE[\brac{M}_T^{p/2}]\le E\Square{\sup_{t\in[0,T]}\abs{M_t}^p}\le CE[\brac{M}_T^{p/2}].\]
\end{theorem}

\begin{theorem}
    この$M\in\M_\loc^C$が一般の可分Hilbert空間$H$に値を取るとする.このとき,二次変分過程は$H$の正規直交基底$\{e_i\}$を用いて
    \[\brac{M}_T=\sum_{i\in\N}\brac{\brac{M,e_i}_H}_T\]
    と表せる.
\end{theorem}

\subsection{連続局所マルチンゲールのBrown運動による表示}

\begin{theorem}[Dambis-Dubins-Schwarz (\cite{Matsumoto-Taniguchi})]
    $M\in\M^C_\loc$は$\lim_{t\to\infty}\brac{M}_t=\infty\;\as$を満たすとする.このとき,
    \begin{enumerate}
        \item $\sigma(s):=\inf\Brace{t>0\mid\brac{M}_t>s}$について,$B_s:=M_{\sigma(s)}$は$(\F_{\sigma(s)})$-Brown運動である.
        \item $M_t=B_{\brac{M}_t}$.
    \end{enumerate}
\end{theorem}

\section{自乗可積分なマルチンゲール}

\begin{definition}\mbox{}
    \begin{enumerate}
        \item 随意過程$X\in(\F_t)\cap D$が\textbf{自乗可積分}であるとは,二次過程$\R_+\to L^2(\Om)$であって,像が$L^2(\Om)$-有界であることをいう.
        \item $\oM^2:=\oM\cap\Map(\R_+,L^2(\Om))$とする:$\forall_{t\in\R_+}E[X_t^2]<\infty$.
    \end{enumerate}
\end{definition}

\subsection{Hilbert空間としての構造}

\begin{tcolorbox}[colframe=ForestGreen, colback=ForestGreen!10!white,breakable,colbacktitle=ForestGreen!40!white,coltitle=black,fonttitle=\bfseries\sffamily,
title=]
    $\M^2\simeq_\Hilb L^2(\Om;\F_\infty)$.
\end{tcolorbox}

\begin{theorem}\mbox{}
    \begin{enumerate}
        \item $\M^2\subset\M$である.
        \item $\forall_{X_\infty\in L^2(\Om,\F_\infty)}\;X_t:=E[X_\infty|\F_t]\in\M^2$.
    \end{enumerate}
\end{theorem}

\begin{definition}\mbox{}
    \begin{enumerate}
        \item $(X|Y):=E[X_\infty Y_\infty]$とすると,$\M^2$はこれを内積としてHilbert空間をなす.
        \item $X,Y\in\M_\loc$が\textbf{強く直交する}$X\indep Y$とは,$XY\in\M_\loc$かつ$X_0Y_0=0$を満たすことをいう.
        \item $p\ge1$について,
        \[\H^p:=\Brace{X\in\M\mid\norm{X}_{\H^p}:=\norm{X^*_\infty}_{L^p(\Om,\F_\infty)}<\infty}\]
        をBanach空間とする.
    \end{enumerate}
\end{definition}

\begin{theorem}
    $X,Y\in\M^2$について,次は同値:
    \begin{enumerate}
        \item $X\indep Y$.
        \item $X_0Y_0=0$かつ$\forall_{\tau\in\T}\;X_\tau\perp Y_\tau$.
    \end{enumerate}
\end{theorem}

\subsection{$\M^2$の直交分解}

\begin{definition}
    $\cK\subset\H^2$が\textbf{安定部分空間}であるとは,次を満たすことをいう:
    \begin{enumerate}
        \item $\cK$はノルム閉部分空間である.
        \item $\forall_{X\in\cK}\;\forall_{\tau\in\T}\;X^\tau\in\cK$.
        \item $\forall_{X\in\cK}\;\forall_{A\in\F_0}\;1_AX\in\cK$.
    \end{enumerate}
    $\cK$を安定部分空間として,$Y\in\cK^\perp:\Leftrightarrow\forall_{X\in\cK}\;E[X_\infty Y_\infty]=0$と定める.
\end{definition}

\begin{theorem}
    $\cK^\perp$も安定部分空間であり,$\forall_{X\in\cK}\;\forall_{Y\in\cK^\perp}\;X\indep Y$.
\end{theorem}

\subsection{自乗可積分マルチンゲールの直交分解}

\begin{tcolorbox}[colframe=ForestGreen, colback=ForestGreen!10!white,breakable,colbacktitle=ForestGreen!40!white,coltitle=black,fonttitle=\bfseries\sffamily,
title=]
    
\end{tcolorbox}

\begin{theorem}
    $\cK=\H^{2,c}$は安定部分空間である.すなわち,
    任意の$M\in\M^2$に対して,一意的な分解$M=M^c+M^d$で,$M^c$は連続な軌道を持ち$M^c_0=0$を満たし,$M^d$は純不連続なマルチンゲールであり,$M^d\indep M^c$を満たす.
\end{theorem}

\section{増大過程}

\begin{definition}
    $A\in(\F_t)\cap D$について,
    \begin{enumerate}
        \item 増大過程であるとは,$A_0=0$かつ$A_t$は単調増加であることをいう.
        その全体を$\A^+$で表す.
        \item $\A$で,$\forall_{\om\in\Om}\;\forall_{t\in\R_+}\;A|_{[0,t]}(\om)\in\BV[0,t]$を表す.
    \end{enumerate}
\end{definition}

\chapter{Gauss過程}

\begin{quotation}
    Gauss測度は一般の可分Hilbert空間上に定義できる.
    その空間の上での変分学をMalliavin解析という.

    まずGauss確率変数の族の$L^2(\Om)$内での振る舞いを見る.
    するとこの転置はHilbert空間値確率変数で,$\Om\to H$の部分空間上にGauss測度を押し出しているものと見れる.
\end{quotation}

\section{$L^2$のGauss部分空間}

\begin{tcolorbox}[colframe=ForestGreen, colback=ForestGreen!10!white,breakable,colbacktitle=ForestGreen!40!white,coltitle=black,fonttitle=\bfseries\sffamily,
title=]
    $L^2(\Om)$の部分集合としてGauss系を定義すると,Gauss部分空間を取る過程がGauss過程ということになる.
\end{tcolorbox}

\subsection{定義と特徴付け}

\begin{definition}[Gaussian subset, Gaussian space]
    部分集合$X=\{X_\lambda\}_{\lambda\in\Lambda}\subset L^2(\Om)$が
    \begin{enumerate}
        \item \textbf{Gauss系}であるとは,任意の有限個の線型結合が正規分布に従うことをいう.
        \item \textbf{Gauss部分空間}であるとは,\textbf{中心化されたGauss確率変数のみ}によって生成された$L^2(\Om)$-部分空間をいう.
    \end{enumerate}
\end{definition}

\begin{theorem}[有限次元周辺分布による特徴付け]
    確率変数族$\{X_\lambda\}\subset\L(\Om,\F,P)$について,
    \begin{enumerate}
        \item Gauss系である.
        \item 有限次元周辺分布が多変量正規分布に従う.
    \end{enumerate}
\end{theorem}

\subsection{Gauss系の構成}

\begin{theorem}[平均と共分散による構成]
    $m:\Lambda\to\R$と半正定値関数$V:\Lambda^2\to\R$について,これらを平均と共分散とするGauss系$X=(X_\lambda)_{\lambda\in\Lambda}$が法則同等を除いて一意的に存在する.
\end{theorem}
\begin{remarks}
    これが,Gauss系が弱定常であるならば強定常でもある由縁である.
\end{remarks}

\begin{theorem}[極限による構成\cite{LeGall}(Prop. 1.1)]
    $X=\{X_n\}_{n\in\N}$をGauss系とする:$X_n\sim N(m_n,\sigma_n^2)$.
    このとき,
    \begin{enumerate}
        \item $(X_n)$が法則収束するならば,その極限は$N(m,\sigma^2)\;(m:=\lim_{n\to\infty}m_n,\sigma^2:=\lim_{n\to\infty}\sigma_n^2)$である.
        \item $X_n$の確率収束と任意の$L^p\;(1\le p<\infty)$-収束とは同等である.
    \end{enumerate},
\end{theorem}

\begin{corollary}[線型演算による構成]
    $\S\subset L^2(\Om,\F,P)$をGauss系とする.
    \begin{enumerate}
        \item $\S$が生成する線型部分空間$\L[\S]$もGauss系である.
        \item $\S$が生成するノルム閉部分空間$\oo{\L}[\S]$もGauss系である.
    \end{enumerate}
\end{corollary}

\subsection{Gauss系の独立性}

\begin{tcolorbox}[colframe=ForestGreen, colback=ForestGreen!10!white,breakable,colbacktitle=ForestGreen!40!white,coltitle=black,fonttitle=\bfseries\sffamily,
title=]
    Gauss系の中では独立性と共分散が$0$であることと(従って対独立であること)は同値になる.
    Gauss系の間でも同様で,さらに中心化されたGauss系に考察を限れば,これは純粋にHilbert空間の性質に翻訳出来る.
\end{tcolorbox}

\begin{theorem}[系が独立であることの共分散による特徴付け]
    $\{X_\lambda\}\subset L^2(\Om)$をGauss系とする.このとき,次の2条件は同値:
    \begin{enumerate}
        \item $X_\lambda$は独立.
        \item 共分散関数が対角集合を除いた集合$\Lambda^2\setminus\Delta$上零である
    \end{enumerate}
\end{theorem}

\begin{theorem}[部分空間が独立であることの直交性による特徴付け]
    $H\subset L^2(\Om)$をGauss部分空間,$\{H_i\}\subset H$をその部分空間の族とする.
    このとき,次は同値;
    \begin{enumerate}
        \item $(H_i)$は独立.
        \item $(H_i)$は互いに直交する.
    \end{enumerate}
\end{theorem}
\begin{remark}
    $H_1,H_2$が共通のGauss部分空間$H$に属さない場合は成り立たない.
    というのも,$(X_1,X_2)$がGaussベクトルにならないならば,互いに直交して$E[X_1X_2]=0$も独立とは限らない.
\end{remark}

\subsection{Gauss系の条件付き期待値}

\begin{tcolorbox}[colframe=ForestGreen, colback=ForestGreen!10!white,breakable,colbacktitle=ForestGreen!40!white,coltitle=black,fonttitle=\bfseries\sffamily,
title=]
    中心化されたGauss系$\cT\subset\S$について,$\S$の$\F[\cT]$による条件付き期待値は,
    $L^2_{\F[\cT]}(\Om)$への射影であるだけでなく,さらに小さい空間$\oo{\L}[\cT]$への正射影となる.
    すなわち,Gauss系のGauss系による条件付けは再びGauss系である.
\end{tcolorbox}

\begin{corollary}
    $H\subset L^2(\Om)$をGauss部分空間,$K\subset H$を閉部分空間,
    $p_K:L^2(\Om)\to K$をその直交射影とする.
    \begin{enumerate}
        \item $\forall_{X\in H}\;E[X|\sigma[K]]=p_K(X)$.
        \item 条件付き確率$P[X\in A|\sigma[K]](\om)$は,$Q(\om,-):=N(p_K(X)(\om),\sigma^2),\sigma^2:=E[(X-p_K(X))^2]$を用いて
        \[P[X\in A|\sigma[K]](\om)=Q(\om,A)\]
        で表せる確率変数である.
    \end{enumerate}
\end{corollary}
\begin{remarks}\mbox{}
    \begin{enumerate}
        \item 一般に$\forall_{X\in L^2(\Om)}\;E[X|\sigma[K]]=p_{L^2_{\sigma[K]}(\Om)}(X)$が成り立つが,Gaussベクトルの場合は「$\sigma[K]$-可測確率変数の空間への直交射影」に限らず,「$K$への直交射影」という極めて小さい部分空間で記述出来ている.
        これは線型回帰問題に対する強い道具である.
        データ$(X_1,X_2)$から$X_3$を予測するとき,
        ベストな予測は線型な予測子$\lambda_1 X_1+\lambda_2X_2$に絞ることが可能で,
        さらに$X_3-(\lambda_1X_1+\lambda_2X_2)$が$\brac{X_1,X_2}$に直交するように選べば良い.
        \item 情報$\sigma[K]$を知った場合の確率分布は,$X\sim N(p_K(X),\sigma^2)$に更新される,と読める.
    \end{enumerate}
\end{remarks}

\subsection{再生核Hilbert空間としての見方}

\begin{definition}[RKHS: reproductive kernel Hilbert space]
    $X$を集合,$H\subset\Map(X;\R)$をHilbert空間とする.
    評価関数$\ev_x:H\to\R$が任意の$x\in X$について有界線型汎関数である$\forall_{x\in X}\;\ev_x\in B(H)$とき,$H$を\textbf{再生核Hilbert空間}という.
    すなわち,Rieszの表現定理よりある$K_x\in H$が存在して$\ev_x(-)=(-|K_x)$と表せる.この対応$X\to H;x\mapsto K_x$が導く双線型形式$K:X\times X\to\R;(x,y)\mapsto K(x,y):=(K_x|K_y)$を\textbf{再生核}という.
    再生核は対称で半正定値である.
\end{definition}
\begin{theorem}
    関数$K:X\times X\to\R$は対称かつ半正定値であるとする.このとき,ただ一つのHilbert空間$H(K)$が$\Map(X;\R)$内に存在して,$K$を再生核として持つ.
\end{theorem}

\begin{theorem}
    $(E,\mu)$を測度空間,$\mu$-体積確定な集合の全体を$\M^1$とする.
    \begin{enumerate}
        \item $v_{\al\beta}:=\mu(\al\cap\beta)\;(\al,\beta\in\M^1)$とすると,対称な半正定値関数である.
        \item これが定めるHilbert空間$H(v)$は$L^2(E,\mu)$と同型である.
    \end{enumerate}
\end{theorem}

\subsection{Gauss系の平均と分散}

\begin{tcolorbox}[colframe=ForestGreen, colback=ForestGreen!10!white,breakable,colbacktitle=ForestGreen!40!white,coltitle=black,fonttitle=\bfseries\sffamily,
title=]
    可分な空間上には,平均と共分散を指定すればGauss系が存在する.
\end{tcolorbox}

\begin{theorem}\mbox{}
    \begin{enumerate}
        \item $\S=(X_\al)_{\al\in A}$をGauss系とし,$m,v$を平均と共分散とする.
        $v$が定める再生核Hilbert空間$H(v)$は可分である.
        \item 任意の写像$m:A\to\R$と対称半正定値関数$v$について,再生核Hilbert空間$H(v)$は可分であるとする.
        このとき,ある確率空間$(\Om,P)$とその上のGauss系$(X_\al)_{\al\in A}$が法則同党を除いて一意的に存在して,$m,v$はそれぞれ平均と共分散である.
    \end{enumerate}
\end{theorem}

\section{Gauss-martingale過程}

\begin{proposition}
    $M\in C\cap\M$を連続なマルチンゲールでGauss過程でもあるとする.このとき,$(M|M)$は決定論的である.
\end{proposition}

\section{多変量Gauss分布}

\begin{tcolorbox}[colframe=ForestGreen, colback=ForestGreen!10!white,breakable,colbacktitle=ForestGreen!40!white,coltitle=black,fonttitle=\bfseries\sffamily,
title=]
    一般の可分Hilbert空間上に定義する前に,$E=\R^d$の場合を考える.
\end{tcolorbox}

\subsection{定義}

\begin{definition}
    $X:\Om\to E$をGauss確率変数とする.
    \begin{enumerate}
        \item $\exists_{m_X\in E}\;\forall_{u\in E}\;E[(u|X)]=(u|m_X)$.この$m_X$を平均という.
        \item ある非負二次形式$q_X:E\to\R$が存在して,$\forall_{u\in E}\;\Var[(u|X)]=q_X(u)$を満たす.この$q_X$を分散という.
        \item $E$の正規直交基底$(e_1,\cdots,e_d)$について,$u$の成分を$u=\sum_{j\in[d]}u_je_j$とすると,次の表示をもつ:
        \[q_X(u)=\sum_{j,k\in[d]}u_ju_k\Cov[X_j,X_k].\]
        \item 平均$q_X$は,対称で半正定値な自己同型$\gamma_X\in B(E)$を,内積を通じて
        \[q_X(u)=(u|\gamma_X(u))\]
        によって定める.$\gamma_X$の$(e_1,\cdots,e_d)$に関する行列表示は$(\Cov[X_j,X_k])_{j,k\in[d]}$が与える.
    \end{enumerate}
\end{definition}

\subsection{独立性}

\begin{proposition}
    Gaussベクトル$X=\sum_{j\in[d]}X_je_j$について,
    次の3条件は同値:
    \begin{enumerate}
        \item $X_1,\cdots,X_d$は独立.
        \item $(\Cov[X_j,X_k])_{j,k\in[d]}$は対角行列である.
        \item $q_X$は基底$(e_1,\cdots,e_d)$について対角表示される.
    \end{enumerate}
\end{proposition}

\begin{corollary}
    $X$を中心化されたGaussベクトル,$\gamma_X$を対角化する基底を$\{\ep_1,\cdots,\ep_d\}\subset E$,対応する固有値を$\lambda_1\ge\cdots\ge\lambda_r>0=\lambda_{r+1}=\cdots=\lambda_d$とする.
    このとき,
    \begin{enumerate}
        \item $Y_j\sim N(0,\lambda_j)$として,
        \[X=\sum_{j\in[r]}Y_j\ep_j\]
        と表せる.
        \item $X$の押し出す分布の台は$\supp(P^X)=\brac{\ep_1,\cdots,\ep_r}$.
        \item $P_X$は$E$上のLebesgue測度に関して絶対連続であることと,$r=d$は同値.
    \end{enumerate}
\end{corollary}

\section{可分Hilbert空間上のGauss測度}

\begin{tcolorbox}[colframe=ForestGreen, colback=ForestGreen!10!white,breakable,colbacktitle=ForestGreen!40!white,coltitle=black,fonttitle=\bfseries\sffamily,
title=]
    これから(可分な)無限次元Hilbert空間上のGauss測度に関する積分論を展開することになるから,
    Gauss測度も無限次元に通用する言葉で定義する.
    $\R^n$上の正規分布の一般化であると同時に,確率過程の平均と共分散との一般化でもある.
    \cite{Prato}による.
    実は,これで一般のBanach空間上のGauss測度を定義したこととなる\ref{thm-Structure-theorem-for-Gaussian-measures}.
\end{tcolorbox}

\begin{notation}
    $H$を可分Hilbert空間,$(H,\B(H))$をBorel可測空間,$P(H)$をその上のBorel確率測度全体の集合とする.
\end{notation}

\subsection{無限次元有限測度の平均と分散}

\begin{tcolorbox}[colframe=ForestGreen, colback=ForestGreen!10!white,breakable,colbacktitle=ForestGreen!40!white,coltitle=black,fonttitle=\bfseries\sffamily,
title=]
    今まで確率空間$(\Om,\F)$に言及して平均と分散を定義することはなかったが,
    可分Hilbert空間$H$を確率空間として,その分布の扱い方を定める.
    $H$上の有限次元分布は有界線型汎関数を通じて定まるが,それはすべて内積で表現できることから,内積を用いた定義になるが,これは成分$X_i$の拡張に過ぎない.
\end{tcolorbox}

\begin{definition}[mean, covariance operator]
    $\mu\in P(H)$について,
    \begin{enumerate}
        \item $E[\abs{x}]<\infty$のとき,任意の$x\in H$に対して,その内積の平均$E[\brac{x|Y}]\;(Y\sim\mu)$を対応させる
        線型形式$F:H\to\R;x\mapsto\int_H\brac{x|y}\mu(dy)$は有界である(Cauchy-Schwarzによる).
        
        このRiesz表現$\exists!_{m\in H}\;F(x)=\brac{x|m}$を定める元$m\in H$を$\mu$の\textbf{平均}という.
        \item $E[\abs{x}]\le E[\abs{x}^2]<\infty$のとき,双線型形式$G:H\times H\to\R;(x,y)\mapsto\int_H\brac{x|z-m}\brac{y|z-m}\mu(dz)$は有界かつ対称(自己共役)である.
        
        このRiesz表現を定める自己有界作用素$\exists!_{Q\in B(H)}\;\brac{Qx|y}=\brac{x|Qy}=G(x,y)$を\textbf{共分散}という.
    \end{enumerate}
\end{definition}
\begin{remarks}\mbox{}
    \begin{enumerate}
        \item 平均は,$E[\brac{x|Y}]=\brac{x|m}$という可換性を満たす元$m$,
        \item 共分散には2つの見方があり,
        対称な(半)双線型形式$G(x,y)=E[\brac{x|Z-m}\brac{y|Z-m}]=E[(\brac{x|Z}-E[\brac{x|Z}])(\brac{y|Z}-E[\brac{y|Z}])]$とも($\brac{x|Z},\brac{y|Z}$を改めて$X,Y:H\to\R$と定めれば見慣れた共分散の定義$\Cov[X,Y]$となる),
        それを表現する有界線型作用素$Q\in B(H)$とも理解できる.
        \item 内積はベクトル$(X_1,\cdots,X_n)$からの成分の抽出の一般化と見ると,$\brac{x|Z}$とは成分$Z_x$と思える.つまり,
        $x,y\in H$は添字集合と見る.
    \end{enumerate}
\end{remarks}

\begin{lemma}[共分散作用素の性質]
    $E[\abs{x}^2]<\infty$とする.
    \begin{enumerate}
        \item $Q$は正作用素であり,かつ自己共役である.
        \item $H$の任意の正規直交基底$(e_k)$について,$\Tr Q:=\sum_{k\in\N}\brac{Qe_k,e_k}=\int_H\abs{x-m}^2\mu(dx)<\infty$.
        \item $Q$はコンパクト作用素である.(2)と併せると特に,$Q$は跡類作用素である:$Q\in B^1(H)$.
        \item 分散共分散公式:$E[\abs{x}^2]=\Tr Q+\abs{m}^2$.
    \end{enumerate}
\end{lemma}
\begin{Proof}\mbox{}
    \begin{enumerate}
        \item $Q$を定める双線型形式$G$は対称であったから,$Q$も対称である:$G(x,y)=G(y,x)\Rightarrow\brac{Qx|y}=\brac{Q^*x|y}$.
        また,$\forall_{x,x\ge0}\;\brac{Qx|x}=G(x,x)\ge0$より正である.
        \item 単調収束定理とParseval恒等式より,
        \begin{align*}
            \Tr Q=\sum_{k\in\N}\brac{Qe_k|e_k}&=\sum_{k\in\N}\int_H\abs{\brac{x-m|e_k}}^2\mu(dx)\\
            &=\int_H\sum_{k\in\N}\abs{\brac{x-m|e_k}}^2\mu(dx)=\int_H\abs{x-m}^2\mu(dx)<\infty.
        \end{align*}
        \item 正作用素$T$が$\exists_{p>0}\;\Tr(\abs{T}^p)<\infty$を満たすならば,コンパクト作用素である.
    \end{enumerate}
\end{Proof}

\begin{notation}
    正作用素の空間を$B^+(H)$とし,有限な跡を持つ作用素の空間を$B^+_1(H)\subset B^+(H)$で表す.
\end{notation}

\subsection{Gauss測度の定義と性質}

\begin{tcolorbox}[colframe=ForestGreen, colback=ForestGreen!10!white,breakable,colbacktitle=ForestGreen!40!white,coltitle=black,fonttitle=\bfseries\sffamily,
title=]
    $H$上のGauss分布を特性関数の言葉を用いて定義する.
    このように定義することで,Wiener測度などの無限次元空間上のGauss測度を扱える.
    そこでの積分が確率解析の主な戦場である.
\end{tcolorbox}

\begin{definition}[Gaussian measure]
    $\mu\in P(H)$が\textbf{Gauss測度$N(m,Q)$}であるとは,ある$m\in H,Q\in L^+_1(H)$を用いて,
    \[\wh{\mu}(h)=e^{i(m|h)}e^{-\frac{1}{2}(Qh|h)}\]
    と表せるものをいう.$Q$が単射$\Ker Q=0$であるとき,$\mu$は非退化であるという.

    なお,自己共役作用素$Q=Q^*$が単射であることは,$\Im Q$がノルム閉ならば$Q=Q^*$が可逆であることに同値であるが,実は$\Im Q$がノルム閉であることは必ずしも成り立たない.
    $\mu=N_{m,Q}$と表す.
\end{definition}
\begin{remarks}
    これは一般のBanach空間とそのペアリングについても成り立つ.
\end{remarks}

\begin{theorem}
    $\supp\mu$は,平均ベクトル$m\in H$を含むあるaffine部分空間になる.
\end{theorem}

\begin{theorem}[Fernique]
    $\nu\sim N(m,Q)$のとき,ある定数$\al\in\R$が存在して,
    \[\int_Be^{\al\norm{x}^2}\mu(dx)<\infty.\]
\end{theorem}
\begin{remark}
    大偏差原理によることで,最善の$\al\in\R$を特徴づけることができる.
\end{remark}

\subsection{ホワイトノイズ}

\begin{tcolorbox}[colframe=ForestGreen, colback=ForestGreen!10!white,breakable,colbacktitle=ForestGreen!40!white,coltitle=black,fonttitle=\bfseries\sffamily,
title=]
    Banach空間上のGauss測度には特徴があり,ペアリングによって位相が移り合うように,測度も移り合う.
    双対空間上のGauss測度をノイズと呼ぶ.
    なお,ペアリングによって微分演算を移すのが超関数微分,積分演算を移すのがPettis積分であった.

    こうして,正規分布に従う独立同分布列(族)の存在定理が,ホワイトノイズ関数という一般的な形で得られる.
    これが$H$を可分にしていた理由である.
    するとこのホワイトノイズを標準的な対象として,任意の測度空間$(A,\A,\mu)$にGauss測度を$(L^2(A))^*$の元という形で構成することも出来る.
\end{tcolorbox}

または次のように定義することも出来る\cite{Revuz-Yor}

\begin{proposition}[一般化された独立同分布列の存在定理:ホワイトノイズ]\label{prop-existence-of-noises}
    $H$を可分実Hilbert空間とする.
    ある確率空間$(\Om,\F,P)$とその上の確率変数の族$X=(X_h)_{h\in H}$が存在して,次の2条件を満たす:
    \begin{enumerate}
        \item 写像$X:H\to\Meas(\Om,\R);h\mapsto X_h$は線型である.
        \item 任意の$h\in H$について,確率変数$X_h$は中心化されたGauss変数で,線型写像$X:H\to L^2(\Om)$は等長である:$E[X_h^2]=\norm{h}^2_H$.
        \item 埋め込み$\{X_h\}_{h\in H}\mono L^2(\Om)$は$L^2(\Om)$のGauss部分空間である.
    \end{enumerate}
\end{proposition}
\begin{remark}
    この同一視により,独立性$X_h\indep X_{h'}$は添字空間$H$における直交性として理解できる.
\end{remark}

\begin{definition}[Gaussian measure / white noise]
    $(A,\A,\mu)$を可分な$\sigma$-有限測度空間とする.$H:=L^2(A,\A,\mu)$として,中心化されたGauss変数の族$X=(X_h)_{h\in H}$を取る.
    この写像$X:L^2(A,\mu)\to\L(\Om)$を$(A,\A)$上の\textbf{強度$\mu$のGauss測度/白色雑音}という.
    測度確定な可測集合$F\in\A,\mu(F)<\infty$のGauss測度$X(F)$は$X(1_F)$とも表す.
\end{definition}

\section{可分Hilbert空間値のGauss確率変数}

\subsection{Hilbert空間上のGauss変数}

\begin{tcolorbox}[colframe=ForestGreen, colback=ForestGreen!10!white,breakable,colbacktitle=ForestGreen!40!white,coltitle=black,fonttitle=\bfseries\sffamily,
title=]
    特にGauss測度の押し出しによって構成されるGauss変数を考える.
    一般の有界線型汎関数,特に回転変換は
    Gauss測度を保つ.
\end{tcolorbox}

\begin{proposition}[有界線型汎関数はGauss測度を保つ]
    $(H,\mu)$を可分な正規Hilbert空間とする:$\mu\sim N(a,Q)$.
    このとき,任意の可分Hilbert空間$K$へのaffine変換$T:H\to K;x\mapsto Ax+b\;(A\in B(H,K),b\in K)$とすると,$T_*\mu\sim N_{Aa+b,AQA^*}$.
\end{proposition}
\begin{Proof}
    特性関数を計算すると,任意の$k\in K$について,
    \begin{align*}
        \wh{T_*\mu}(k)&=\int_K e^{i(k|y)}(T_*\mu)(dy)\\
        &=\int_He^{i(k|T(x))}\mu(dx)\\
        &=\int_He^{i(k|Ax+b)}\mu(dx)\\
        &=e^{i(k|b)}\int_He^{i(A^*k|x)}\mu(dx)=e^{i(k|Aa+b)}e^{-\frac{1}{2}(AQA^*k,k)}.
    \end{align*}
\end{Proof}

\begin{corollary}
    $(H,\mu)$を可分な正規Hilbert空間とする:$\mu\sim N(0,Q)$.
    任意の点$z_1,\cdots,z_n\in H$について,$A:H\to\R^n$を$Ax:=((x|z_1),\cdots,(x|z_n))\;(x\in H)$と定めると,
    $Q_A:=(Qz_i,z_j)_{i,j\in[n]}$について$A\sim N(0,Q_A)$.
\end{corollary}

\begin{corollary}[Gauss測度の回転不変性]
    線型写像$R:H\times H\to H\times H$を
    \[R(x,y):=\begin{pmatrix}\cos\theta&-\sin\theta\\\sin\theta&\cos\theta\end{pmatrix}\begin{pmatrix}x\\y\end{pmatrix}\]
    と定めると,$(H\times H,\mu\times\mu)\;(\mu\sim N(0,Q))$上の2つの測度$\mu\times\mu$と$R_*\mu\times\mu$とは等しい:
    \[\forall_{F\in\L_b(H\times H;H\times H)}\;\int_{H\times H}F(x,y)\mu(dx)\mu(dy)=\int_{H\times H}F(R(x,y))\mu(dx)\mu(dy).\]
\end{corollary}

\subsection{Gauss過程の特徴付け}

\begin{tcolorbox}[colframe=ForestGreen, colback=ForestGreen!10!white,breakable,colbacktitle=ForestGreen!40!white,coltitle=black,fonttitle=\bfseries\sffamily,
title=]
    $C(T,\R)$上のGauss変数を特にGauss過程ともいう.
\end{tcolorbox}

\begin{definition}[Gaussian process, centered]
    $(\Om,\F,P)$上の実確率過程$X=(X_t)_{t\in\R_+}$について,
    \begin{enumerate}
        \item \textbf{Gauss過程}であるとは,任意の有限部分集合$\{t_1,\cdots,t_n\}\subset \R_+$について,確率ベクトル$(X_{t_1},\cdots,X_{t_n}):\Om\to\R^n$は$n$次元正規分布に従うことをいう.
        \item Gauss過程$B$の\textbf{共分散}とは,関数$\Gamma:\R_+\times\R_+\to\R;(s,t)\mapsto\Cov[X_s,X_t]=E[(X_s-E[X_s])(X_t-E[X_t])]$をいう.
        \item \textbf{中心化されている}とは,$\forall_{t\in\R_+}\;E[X_t]=0$を満たすことをいう.
    \end{enumerate}
\end{definition}

\begin{definition}[semi-definite positive function]
    関数$\Gamma:T\times T\to\R$が\textbf{半正定値}であるとは,
    $T$の任意の有限部分集合$\{t_1,\cdots,t_d\}\subset T\;(d\in\N)$に対して,行列$(\Gamma(t_i,t_j))_{i,j\in[d]}$は半正定値であることをいう.
\end{definition}

\begin{proposition}[Gauss過程の共分散の特徴付け]\label{prop-existence-of-Gaussian-process}
    $T$を任意の集合とする.関数$\Gamma:T\times T\to\R$と任意の関数$m:T\to\R$について,次の2条件は同値.
    \begin{enumerate}
        \item ある平均$m$のGauss過程が存在して,その共分散である.
        \item 対称な半正定値関数である.
    \end{enumerate}
\end{proposition}
\begin{Proof}
    Kolmogorovの拡張定理\ref{thm-Kolmogorov-extension-theorem}による.
    ここでは$a=0$として証明する.
    \begin{description}
        \item[(1)$\Rightarrow$(2)] $T$の対称性は積の可換性より明らか.
        任意の有限部分集合$\{t_i\}_{i\in[n]}\subset T$と任意のベクトル$a=(a_i)_{i\in[n]}\in\C^n$を取る.
        これについて,2次形式$a^*(\Sigma(i,j))a$が非負であることを示せば良い.
        \begin{align*}
            \sum_{i,j\in[n]}&=\Gamma(t_i,t_j)a_i\o{a_j}\\
            &=E\Square{\sum_{i\in[n]}a_i(X_{t_i}-E[X_{t_i}])\sum_{j\in[n]}\o{a_j}(X_{t_j}-E[X_{t_j}])}\\
            &=E\Square{\Abs{\sum_{i\in[n]}a_i(X_{t_i}-E[X_{t_i}])}^2}\ge0.
        \end{align*}
        \item[(2)$\Rightarrow$(1)] 
        確率分布族$(P_{t_1,\cdots,t_n})_{n\in\N_{\ge1},\{t_i\}\subset T}$を,$\R^n$上の平均$0$,分散共分散行列を$\Sigma_{t_1,\cdots,t_n}:=(\Gamma(t_i,t_j))_{i,j\in[n]}$とする正規分布とする.
        $\Gamma$は対称は半正定値関数としたので,$\Sigma_{t_1,\cdots,t_n}$も対称な半正定値行列であり,これに対応する正規分布はたしかに存在する.
        また定義より,一貫性条件を満たす.
        よって,$(P_{t_1,\cdots,t_n})_{n\in\N_{\ge1},\{t_i\}\subset T}$を有限次元周辺分布とする
        確率過程$X=(X_t)_{t\in\R_+}$が存在するが,これは平均$0$で分散$\Gamma$のGauss過程である.
    \end{description}
\end{Proof}

\begin{lemma}[共分散公式]
    $X,Y$をGauss確率変数とする.
    $X_t:=tX+Y\;(t\in\R_+)$によって定まる確率過程$X=(X_t)_{t\in\R_+}$はGauss過程であって,
    平均$m_X(t)=tE[X]+E[Y]$と分散
    \[\Gamma_X(s,t)=st\Var[X]+(s+t)\Cov[X,Y]+\Var[Y]\]
    を持つ.
\end{lemma}
\begin{Proof}
    平均は期待値の線形性より明らか.
    分散については,次のように計算が進む.
    ただし,見やすくするため$\mu_X:=E[X],\mu_Y:=E[Y]\in\R$とする.
    \begin{align*}
        \Gamma(s,t)&=E[(X_s-E[X_s])(X_t-E[X_t])]\\
        &=E[X_sX_t]-(s\mu_X+\mu_Y)E[X_t]-(t\mu_X+\mu_Y)E[X_s]+(s\mu_X+\mu_Y)(t\mu_X+\mu_Y)\\
        &=stE[X^2]+(s+t)E[XY]+E[Y^2]+(s\mu_X+\mu_Y)(t\mu_X+\mu_Y)=st\Var[X]+(s+t)\Cov[X,Y]+\Var[Y].
    \end{align*}
\end{Proof}

\subsection{Gauss変数の独立性}

\begin{tcolorbox}[colframe=ForestGreen, colback=ForestGreen!10!white,breakable,colbacktitle=ForestGreen!40!white,coltitle=black,fonttitle=\bfseries\sffamily,
title=]
    あるGauss確率変数の成分(一般化して,有界線型汎関数との合成)の間の独立性は,直交性で捉えられる.
\end{tcolorbox}

\begin{proposition}[一般の確率変数の独立性の特徴付け]
    $X_1,\cdots,X_n:\Om\to H$を確率変数,$X:=(X_1,\cdots,X_n)$を積写像とする.
    \begin{enumerate}
        \item $X_1,\cdots,X_n$は独立である.
        \item 特性関数が分解する:$\forall_{h=(h_1,\cdots,h_n)\in H^n}\;\wh{X}(h)=\prod_{i=1}^n\wh{X_i}(h_j)$.
        \item 任意の期待値が分解する:$\forall_{\varphi_1,\cdots,\varphi_n\in\L_b(\Om;\R)}\;E[\varphi_1(X_1)\cdots\varphi_n(X_n)]=E[\varphi_1(X_1)]\cdots E[\varphi_n(X_n)]$.
    \end{enumerate}
\end{proposition}

\begin{notation}[正規Hilbert空間上の線型な実確率変数]
    $H$上の実確率変数のうち,特に有界線型汎関数によって定まるもの:$F_v(x)=(x,v)\;(v\in H)$を考える.これはGauss確率変数である:$F_v\sim N_{\brac{Qv,v}}$.
\end{notation}

\begin{proposition}[Gauss変数の独立性の特徴付け]
    $v_1,\cdots,v_n\in H$とする.
    \begin{enumerate}
        \item 線型確率変数$F_{v_1},\cdots,F_{v_n}$が独立である.
        \item $(Q_{v_1,\cdots,v_n})$は対角行列である.
    \end{enumerate}
    (2)は,各$F_{v_i},F_{v_j}\;(i\ne j)$が$L^2(H)$上で直交することに同値.
\end{proposition}

\subsection{Gauss空間}

\begin{tcolorbox}[colframe=ForestGreen, colback=ForestGreen!10!white,breakable,colbacktitle=ForestGreen!40!white,coltitle=black,fonttitle=\bfseries\sffamily,
    title=]
    $X$が$d$次元Gauss変数であるとは,任意の線型汎関数$\al:\R^d\to\R$について$\al(X)$がGauss確率変数であることと同値.
    この他にも,Gauss変数の独立性はGauss空間で幾何学的に捉えられる.
\end{tcolorbox}

\begin{notation}
    $X\sim N(0,0)$として,定数関数はGauss確率変数とする.
\end{notation}

\begin{proposition}[Gauss確率変数全体の空間は閉部分空間をなす]
    $(X_n)$を$X\in\Meas(\Om,\R)$に確率収束するGauss確率変数の列とする.このとき,$X$もGaussで,族$\{\abs{X_n}^p\}$は一様可積分で,$X_n\to X$は任意の$p\ge1$について$L^p$-収束もする.
    また,$X$の平均と共分散はそれぞれの各点収束極限である.
\end{proposition}

\begin{definition}[Gaussian subspace]
    Hilbert空間$L^2(\Om,\F,P)$の閉部分空間であって,中心化されたGauss確率変数のみからなるものを\textbf{Gauss空間}という.
\end{definition}

\begin{proposition}[独立性の特徴付け]
    $(G_i)_{i\in I}$をあるGauss空間の閉部分空間の族とする.次の2条件は同値:
    \begin{enumerate}
        \item $\sigma$-代数$\sigma(G_i)$の族は独立である.
        \item 各$G_i$は組ごとに直交する:$\forall_{i,j\in I}\;G_i\perp G_j$.
    \end{enumerate}
\end{proposition}

\subsection{ホワイトノイズ関数}

\begin{tcolorbox}[colframe=ForestGreen, colback=ForestGreen!10!white,breakable,colbacktitle=ForestGreen!40!white,coltitle=black,fonttitle=\bfseries\sffamily,
title=]
    $Q$がコンパクト作用素であること,正作用素であること,すべてが交錯している.
    white noiseとは中心化された有限な分散を持つ独立確率変数の組として得られる確率ベクトルをいう.
\end{tcolorbox}

\begin{notation}
    $\mu=N_Q$を中心化された非退化なGauss分布とする.二乗根作用素$Q^{1/2}$の像$\Im Q^{1/2}$をCameron-Martin部分空間という.
    これは$H$の真の部分空間であるが,実は$H$上稠密である.
    この部分空間上に定まる線型作用素$W_-(-):\Im Q^{1/2}\times H\to\R$を
    \[W_f(x):=\Brac{Q^{-1/2}f,x}\quad(f\in Q^{1/2}(H),x\in H)\]
    とおく.

    なお,$Q$が単射(=像の上で可逆)であることより$Q^{1/2}:H\to\Im Q^{1/2}$も単射で,よって$\Im Q^{1/2}$上で可逆だから,$Q^{-1/2}:\Im Q^{1/2}\to H$はひとまず定まっている.
    次に$W_-$は有界線型であるから,
    一意な延長$\o{W}:H\mono L^2(H,\mu)$が存在する.この$L^2(H,\mu)$の部分空間は有界線型汎関数のみからなり,特にGauss空間である.
    すると,独立性の特徴付けより,$W_{f_1},\cdots,W_{f_n}$が独立であることと,$f_1,\cdots,f_n$が直交であることは同値.
\end{notation}

\begin{definition}
    \[Q^{1/2}x:=\sum^\infty_{k=1}\sqrt{\lambda_k}(x|e_k)e_k\quad(x\in H)\]
    を二乗根作用素という.この像を\textbf{Gauss測度$N(0,Q)$に関するCameron-Martin部分空間}という.
\end{definition}

\begin{lemma}\mbox{}
    \begin{enumerate}
        \item $Q\in B^1(H)$は可逆でない.特に,$Q^{-1}$は非有界作用素である.
        \item $Q(H)\subsetneq H$である.
        \item $Q(H)$は$H$上稠密である.
        \item $Q^{1/2}(H)$も$H$上稠密である.
        \item $W:Q^{1/2}(H)\to L^2(H,\mu)$は$H$上に一意に延長し,Hilbert空間の埋め込みを定める.
        \item $\Im W=\oo{H^*}$.
    \end{enumerate}
\end{lemma}

\begin{definition}[white noise]\label{def-white-noise-1}
    Hilbert空間の埋め込み$W:H\mono L^2(H,\mu);f\mapsto \brac{Q^{-1/2}x,f}=\sum_{h=1}^\infty\lambda_h^{-1/2}\brac{x,e_h}\brac{f,e_h}$
    を\textbf{ホワイトノイズ}という.
\end{definition}

\begin{proposition}[white noiseはGauss系である]\mbox{}\label{prop-property-of-white-noise}
    \begin{enumerate}
        \item $z\in H$に対して,値$W_z$は$N(0,\abs{z}^2)$に従う実Gauss変数である.
        \item $z_1,\cdots,z_n\in H\;(n\in\N)$に対して,$(W_{z_1},\cdots,W_{z_n})$は$N_n(0,(\brac{z_h,z_k})_{h,k\in[n]})$に従うGauss変数である.
        \item $W_{z_1},\cdots,W_{z_n}$が独立であることと$z_1,\cdots,z_n$が直交することは同値.
    \end{enumerate}
\end{proposition}

\begin{proposition}[Cameron-Martin部分空間は零集合]
    $\mu(Q^{1/2}(H))=0$.
\end{proposition}

\begin{tbox}{red}{}
    この記法を使うと,$D$を勾配として,$M\varphi:=Q^{1/2}D\varphi$がMalliavin微分となる.
    $M$は可閉作用素になり,その閉包はMalliavin-Sobolev空間$D^{1,2}(H,\mu)$上の作用素となる.
    随伴$M^*$はSkorohod積分またはGauss発散作用素という.
\end{tbox}

\section{離散時Gauss過程}

\begin{tcolorbox}[colframe=ForestGreen, colback=ForestGreen!10!white,breakable,colbacktitle=ForestGreen!40!white,coltitle=black,fonttitle=\bfseries\sffamily,
title=]
    \cite{Hida-Gauss}が予測の理論に基づいて確立した世界観を,離散の場合で見る.
\end{tcolorbox}

\begin{notation}
    中心化されたGauss系$X:\N^+\to L^2_0(\Om)$を考える.
    Schmidtの直交化より,正規直交系$\{\xi_n\}\subset L^2_0(\Om)$が存在して,$X_n=\sum_{j\le n}a_{n,j}\xi_j$と表せる.
    $\{\xi_n\}$は再び中心化された標準Gauss系で,独立でもある.
\end{notation}

\begin{proposition}
    $\forall_{k<n\in\N^+}\;E[X_n|X_1,\cdots,X_k]=\sum_{j\in[k]}a_{n,j}\xi_j$.
\end{proposition}

\begin{definition}
    $X=(X_n)_{n\in\N^+}$をGauss過程とする.
    \begin{enumerate}
        \item 組$(a_{n,j},\xi_n)$が,$\sum_{j\le n}a_{n,j}\xi_j\overset{d}{=}X_n$を満たすとき,$X$の\textbf{表現}という.
        \item $X_n=\sum_{j\le n}a_{n,j}\xi_j$と命題の条件を満足するとき,\textbf{標準表現}という.このとき,$(\xi_n)$を\textbf{新生変数列},$(a_{n,j})$を\textbf{核}という.
    \end{enumerate}
\end{definition}

\chapter{半マルチンゲール過程}

\begin{quotation}
    確率積分によって表現できる過程として,
    局所マルチンゲールと有界変動過程という2つの過程を手に入れた.
    これらの和として表せる過程は,数学的に閉じている重要な対象と考えられる.

    Levy過程は半マルチンゲールである.
    martingale性の要諦は局所martingaleが十分に表現していて,これを二次過程に限定しては見失うものがある.
    そして,$X$を半マルチンゲールとして,真に確率積分$H\cdot X=\int HdX$が定義されるのである\cite{Liptser-Shiryaev-Statistics}.

    半マルチンゲール過程は,離散過程,点過程,Markov過程と拡散過程を含む実用上も十分な広さを持つクラスである上に,この範囲の過程について確率積分の方法を用いることができるという大きな美点がある\cite{Jacod-Shiryaev}.
\end{quotation}

\section{連続な場合の定義と性質}

\begin{definition}
    $X\in C$が\textbf{連続な$(\F_t,P)$-セミマルチンゲール}であるとは,ある$M\in M_\loc\cap C$と$A\in C\cap\bF\cap\A$を用いて$X=M+A$と表せることをいう.
\end{definition}

\begin{proposition}
    $X$を連続なセミマルチンゲールとする.
    \begin{enumerate}
        \item $X$は有限な二次変動を持つ.
        \item $(X|X)=(M|M)$.
    \end{enumerate}
    特に,分解$X=M+A$は一意である.
\end{proposition}

\begin{definition}
    $X,Y$を連続な半マルチンゲールとする.
    \[(X|Y):=(M|N)=\frac{1}{4}((X+Y|X+Y)-(X-Y|X-Y))\]
    と定める.
\end{definition}

\section{一般の場合の定義}

\begin{tcolorbox}[colframe=ForestGreen, colback=ForestGreen!10!white,breakable,colbacktitle=ForestGreen!40!white,coltitle=black,fonttitle=\bfseries\sffamily,
title=]
    二乗可積分マルチンゲールと,増大過程の差で表せる過程との,和で表せる過程を半マルチンゲールという.
    半マルチンゲールが$C$-過程でもあるとき,前述の定義と合致する.
\end{tcolorbox}

\begin{definition}[semimartingale]
    確率基底$(\Om,\F,(\F_t),P)$において,$X\in(\F_t)\cap D$が次の表示を持つとき,\textbf{$(\F_t)$-半マルチンゲール}という:
    \[\exists_{X_0\in L(\Om,\F_0)}\;\exists_{M_t\in\M^{2,0}_\loc}\;\exists_{V\in\A^0}\quad X_t=X_0+M_t+V_t.\]
    なお,$M$は一意的には定まらないが,その連続部分$M^c$は一意に定まり,これを\textbf{半マルチンゲール$X$の連続マルチンゲール部分}という.
\end{definition}
\begin{remarks}
    $M_t,A_t$を$C\cap\bF$で取ればこれは一意に定まる.
\end{remarks}

\begin{example}
    拡散過程(Markov $C$-過程),伊藤過程,種々のジャンプ過程(特に計数過程)を含む.
    こうして$C$-過程の範囲を脱出するので,可測性の問題がより複雑な形で出現する.
    また,Gauss過程,非整数Brown運動はセミマルチンゲールではなく,このような確率過程をBrown運動などのマルチンゲールから考察するにはラフパスの理論が必要になる.
\end{example}

\section{汎関数解析}

\begin{quotation}
    厚地\cite{厚地}が近い研究をしている.
    \begin{quote}
        種々の多様体とその上の特徴的な関数族を訪ね歩きたいのだが,このように調べたい関数にその土地固有の確率過程を放り込んで展開すれば,何かしら見えてくると思われる.
    \end{quote}
\end{quotation}

\section{確率過程の統計推測への応用}

\begin{history}[filtering problem]
    確率過程としてモデル化できる現象を,雑音が加わって汚染された観測データをもとにして,各時点毎に逐次推定する問題を\textbf{フィルター問題}という.
    40sにKolmogorovとWienerによって始められ,60sのKalmanとBucyにより算譜化されたために工学にも応用される.
\end{history}

\begin{example}
    回帰モデル$X_i=f(X_{i-1},\cdots,X_{i-p})+\ep_i$において,
    サンプリングが均等でないときなど,$\ep_i$は何か連続的な確率過程を積分して定まる,と考えると
    数理モデルとして非常に自然である.連続関数$f$について,
    \[Y_t=Y_0+\int^t_0f(Y_s)ds+W_t\]
    とし,$W_{t_i}-W_{t_{i-1}}\sim\N(0.t_i-t_{i-1})$を標準Weiner過程とする.

    このようなモデルのうち,特に株価の対数を$Y_t$とおいたときに使われるパラメトリックモデルに,
    \textbf{Vasicek過程}
    \[Y_t=Y_0-\int^t_0\al_1(Y_s-\al_2)ds+\beta W_t\]
    などがあり,離散的観測$\{Y{t_0},\cdots,Y_{t_n}\}$に基づいて未知パラメータ$\al_1,\al_2,\beta$の推定を考える.
    このときにマルチンゲール理論が使える.

    その理由は,martingaleというクラスの形式的定義が,自然に統計モデルの「ノイズの直交性」の拡張となっていると考えられるためである.
    これは独立性の仮定による代数規則$E[\ep_i\ep_j]=0$の抽出となっているのである.

    大きな応用分野として生存解析におけるcensored data\footnote{消息不明になる瞬間があること.癌の再発データにおいて,他の原因による死亡など.}の解析がある.
    このとき,$N_t$を死亡数,$Y_t$をcensorされずに残っている観測対象数,癌の再発時刻の分布関数を$F$,密度関数を$f$とすると,
    \[N_t-\int^t_0\al(s)Y_sds\qquad\al(t)=\frac{f(t)}{1-F(t)}\]
    はマルチンゲールになる.$\al$はハザード関数といい,患者が時刻$t$で生存しているという条件の下,その時間に死亡する条件付き確率となる.
    このマルチンゲールの期待値は常に$0$だから,$N_t$の不偏推定量が見つかったことになる.
    なお,
    \[\int^t_0\frac{1}{Y_s}(dM_s-\al(s)Y_sds)\]
    もマルチンゲールとなることがわかる.
\end{example}

\chapter{加法過程}

\begin{quotation}
    Brown運動の独立増分性を抽出して加法過程といい,時間一様性も加えたものをLevy過程という.
    Brown運動は連続であるが,Levy過程・加法過程は$D$-過程とする.
    Poisson過程は高さ1の跳躍でのみ増加する,跳躍のみで増加する過程の代表である.

    Levy過程の特性量は3つの成分からなり,ドリフト,Brown運動,跳躍過程である.
    任意のLevy過程は半マルチンゲールである.
\end{quotation}

\begin{notation}\mbox{}
    \begin{enumerate}
        \item 時間$[s,t]$の間のすべての増分が定める$\sigma$-加法族を$\sigma_{s,t}[dX]:=\sigma[X_v-X_u;u\le v\in[s,t]]$と表す.
        \item $\sigma_{s+,t}[dX]:=\sigma[X_v-X_u;u\le v\in(s,t]]$.
        \item 閉$\sigma$-代数についても,$\F_{s,t}[dX]:=\cap_{\ep>0}\sigma_{s-\ep,t+\ep}[dX]\lor 2$.
    \end{enumerate}
\end{notation}

\section{定義と特徴付け}

\begin{tcolorbox}[colframe=ForestGreen, colback=ForestGreen!10!white,breakable,colbacktitle=ForestGreen!40!white,coltitle=black,fonttitle=\bfseries\sffamily,
    title=]
    見本道が殆ど至る所cadlagである過程を$D$-過程という.
    見本過程$\R_+\to\Meas(\Om,\R);s\mapsto X_s$が確率連続な加法過程で$D$-過程でもあるものを,Levy過程という.
    任意の確率連続な加法過程はLevy過程に同等であり,Levy過程の構造は解明済みである.

    大雑把にいえば,連続な加法過程はGauss型,すなわちブラウン運動に限り,非連続的な加法過程はPoisson型に限る.
    任意のLevy過程は,ドリフト付きのBrown運動とLevyのジャンプ過程との和に分解できる.
\end{tcolorbox}

\subsection{定義}

\begin{definition}[additive process, Levy process]
    $(X_t)_{t\in\R_+}$は,,
    (1),(2)のみを満たすとき\textbf{加法過程}または\textbf{独立増分過程}といい,
    (3),(4)も満たすとき\textbf{Levy過程}といい,(5)も満たすとき\textbf{一様Levy過程}という.
    \begin{enumerate}
        \item 加法性:$\forall_{n=2,3,\cdots}\;\forall_{0\le t_1<\cdots<t_n}\;X_{t_n}-X_{t_{n-1}},\cdots,X_{t_2}-X_{t_1}$は独立.
        \item 中心性:$X_0=0\;\as$
        \item $D$-過程である.
        \item 確率連続である.
        \item 定常性:$\forall_{s\in\R_+}\;X_{t+s}- X_t$は$t$に依存しない.
    \end{enumerate}
\end{definition}

\begin{theorem}[確率連続な加法過程にはLevy過程である修正が存在する]
    (1),(2),(4)を満たすならば,その修正であって(3)も満たす修正が存在し,識別不可能な違いを除いて一意的である.
    すなわち,確率連続な加法過程にはLevy過程である修正が存在する.
\end{theorem}

\subsection{例}

\begin{example}
    離散時間の加法過程は,遷移確率が空間的に一様なMarkov連鎖と見れる.さらに時間的にも一様なものがLevy過程となる.
    \begin{enumerate}
        \item $\Z^d$-酔歩は加法過程である:任意の長さ$k\ge2$の部分列$(n_j)_{j\in[k]}$について,\textbf{増分}の過程$(X_{n_j}-X_{n_{j-1}})_{j\in[k]}$は独立.
        \item 一般に独立な実確率変数列$(Y_n)_{n\in\N}$の部分和の過程${X_t:=\sum_{k\le t}Y_k(\om)}_{t\in\R_+}$は加法過程である.
        \item $\R_+$上のLebesgue測度を平均に持つPoisson配置$\{Y(A,\om)\}_{A\in\B^1\cap\R_+}$に対して$X_t(\om):=Y([0,t],\om)$とおくと,これは加法過程である.これを\textbf{Poisson過程}という.
        これは「Poisson点過程の積分」という意味で,(2)の例の一般化になっている.
        \item Poisson過程は次のようにも定義できる.$(\tau_i)_{i\in\N^+}$を$\Exp(\lambda)\;(\lambda>0)$に従う独立同分布列とし,$T_n:=\sum_{i\in[n]}\tau_i$とする.
        このとき,過程$N_t:=\sum_{n\ge1}1_{\Brace{t\ge T_n}}$を\textbf{強度$\lambda$のPoisson過程}という.
        $N_t\sim\Pois(\lambda t)$が成り立つ.
        Poisson過程は特徴量$(0,0,\nu)\;(\nu=\lambda\delta_1)$を持つLevy過程である.
        \item \textbf{強度$\lambda$サイズ分布$\mu$の複合Poisson過程}とは,$N$をPoisson過程,$(Y_i)_{i\in\N^+}$を$\mu$の独立同分布で$N$と独立なものとして,
        \[X_t:=\sum_{i\in[N_t]}Y_i\]
        で定まる確率過程をいう.すなわち,ジャンプ時刻がPoisson過程で与えられ,ジャンプのサイズは与えられた法則に従う.これは特徴量$(0,0,\nu)\;(\nu=\lambda\mu)$を持つLevy過程である.
        \item ドリフト$\beta$を持つBrown運動$X_t=\al B_t+\beta t$は$(\beta,\al^2,0)$が定めるLevy過程である.逆に,$C$-過程でもあるLevy過程はドリフトを持つBrown運動である.
    \end{enumerate}
\end{example}

\begin{remark}
    このとき,$S$の原点を$0$とするとその転移確率は一様に$p(s,-):=p(s,0,-)$と表わせ,Chapman-Kolmogorovの等式も
    \[\forall_{s,t\in\R_+}\;\forall_{x,z\in S}\;\int_Sp(s,dy-x)p(t,dz-y)=p(s+t,dz-x)\]
    と表せる.
\end{remark}

\subsection{特性関数による特徴付け}

\begin{theorem}[特性関数 Lévy-Khintchine formula]
    Levy過程$X$の特性関数は,$\beta\in\R,\al^2\in\R^+,\nu$を$\B(\R)$上の$\nu(\{0\})=0,\int_\R(1\land\abs{z}^2)\nu(dz)<\infty$を満たす$\sigma$-有限測度として,
    \[\varphi(u)=E[e^{iuX_t}]=e^{t\psi(u)},\quad\psi(u):=i\beta u-\frac{\al^2u^2}{2}+\int_\R(e^{iuz}-1-iuz1_{\Brace{\abs{z}\le1}})\nu(dz).\]
    と表せる.$(\beta,\al^2,\nu)$を\textbf{特性量}といい,$\beta$をドリフト,$\al^2$を拡散係数,$\nu$をLevy測度という.
\end{theorem}

\subsection{Poisson空間}

\section{付属するマルチンゲール}

\begin{theorem}
    $X$が確率連続な加法過程ならば,
    \begin{enumerate}
        \item $\F_{s,t}=\sigma_{s,t}\lor 2$.
        \item $s_0<s_1<\cdots<s_n$ならば,$(\F_{s_{i-1}s_i})_{i\in[n]}$は独立.
    \end{enumerate}
\end{theorem}
\begin{remarks}
    これより,$s<t\Rightarrow X_t-X_s$は$\F_s[X]$と独立と分かる.
\end{remarks}

\begin{proposition}
    \[Y^a_t:=\frac{e^{iaX_i}}{E[e^{iaX_i}]}\]
    は$(\F_t[X])$-マルチンゲールである.
    なお,複素過程がマルチンゲールとは,実部と虚部がいずれもマルチンゲールであることをいう.
\end{proposition}

\section{Gauss型とPoisson型のLevy過程}

\begin{tcolorbox}[colframe=ForestGreen, colback=ForestGreen!10!white,breakable,colbacktitle=ForestGreen!40!white,coltitle=black,fonttitle=\bfseries\sffamily,
title=]
    ドリフトを持ったBrown運動を除いて,他のすべての決定的でないLevy過程は不連続な見本過程を持つことは驚愕の事実である.
\end{tcolorbox}

\begin{definition}
    Levy過程の中で,
    \begin{enumerate}
        \item 見本過程が殆ど確実に連続であるとき,\textbf{Gauss型}であるという.
        \item 見本過程が殆ど確実に飛躍$1$で増加する階段関数となるとき,\textbf{Poisson型}であるという.
    \end{enumerate}
\end{definition}

\begin{theorem}[Gauss型とPoisson型Levy過程]
    $X$をLevy過程とする.
    \begin{enumerate}
        \item $X$がさらに(概)連続過程であれば,増分$X_b-X_a\;(b>a)$はGauss分布に従う.
        \item $X$がさらに殆ど至る所飛躍$1$で増加する階段関数を見本過程に持つならば,増分$X_b-X_a\;(b>a)$はPoisson分布に従う.
    \end{enumerate}
\end{theorem}

\section{ジャンプの描像}

\begin{tcolorbox}[colframe=ForestGreen, colback=ForestGreen!10!white,breakable,colbacktitle=ForestGreen!40!white,coltitle=black,fonttitle=\bfseries\sffamily,
title=]
    任意の正数$a>0$に対してこれより大きいジャンプ$X_\al(\om)-X_{\al-}(\om)>a$は有限個しかないが,$a\to0$を調べるには慎重な議論が居る.
\end{tcolorbox}

\begin{definition}
    $(\Om,\F,P)$について,
    \begin{enumerate}
        \item $\F$の部分集合$\{\B_{s,t}\}_{s<t\in\R_+}$が\textbf{加法系}であるとは,次の2条件が成り立つことをいう:
        \begin{enumerate}[(a)]
            \item $s<t<u\Rightarrow\B_{s,t}=\B_{s,t}\lor\B_{t,u}$.
            \item $\paren{[s_i,t_i)}_{i\in[n]}$が互いに素であるならば,$\{\B_{s_i,t_i}\}_{i\in[n]}$は独立.
        \end{enumerate}
        \item 確率過程$X$が$X_0=0$を満たし,ある加法系$\{\B_{s,t}\}$について$\forall_{s<t\in\R_+}\;\sigma[X_t-X_s]\subset\B_{s,t}$ならば,$X$は$\sigma_{s,t}[dX]\subset\B_{s,t}$を満たす加法過程である.これを\textbf{$\{\B_{s,t}\}$に従属する加法過程}という.
    \end{enumerate}
\end{definition}
\begin{example}
    $\sigma_{s,t}[dX]$は加法系であり,これを\textbf{$X$が生成する加法系}という.
\end{example}

\begin{lemma}[well-definedness]
    $X=(X_t)$は加法系$\{\B_{s,t}\}$に従属するとする.このとき,$E\in\B(\R\setminus\{0\})$に対して,歩幅の条件$X_\al(\om)-X_{\al-}(\om)\in E$を満たす時刻$\al\in(s,t]$の数$N((s,t]\times E,\om)$は$\B_{s,t}$-可測である.
\end{lemma}

\begin{lemma}[独立性の十分条件]
    $X=(X_t)$はLevy過程,$Y=(Y_t)$はPoisson型のLevy過程で,共に加法系$\{\B_{s,t}\}$に従属するとする.
    この2つが,任意の$\om$に対して,互いの見本過程が共通の飛躍時刻を持たないならば,$X,Y$は独立である.
\end{lemma}

\begin{lemma}
    任意の$E\in\B(\R\setminus(-a,a))\;(a>0)$に対して,その歩幅に含まれる跳躍の回数の過程$N_E(t):=N((0,t]\times E,\om)$は,$\{\B_{s,t}\}$に従属するPoisson型Levy過程である.
\end{lemma}

\begin{notation}
    $E\in\B(\R\setminus(-a,a))\;(a>0)$に対して,
    \begin{enumerate}
        \item $S_E(t):=\sum_{s\le t}(X(s)-X(s-))1_E(X(s)-X(s-))$とおくと,歩幅が$E$に属する跳躍のみを加えていくことにより得られる過程である.これは補題より$\{\B_{s,t}\}$に従属する過程である.また確率連続であり,Levy過程である.
        \item $X_E(t):=X(t)-S_E(t)$とおいても同様の性質を満たす.
    \end{enumerate}
\end{notation}

\begin{lemma}
    任意の互いに素な集合族$E_1,\cdots,E_n\in\B(\R\setminus(-a,a))\;(a>0),\;E:=\sum_{k=1}^nE_k$に対して,
    \begin{enumerate}
        \item $N_{E_1},\cdots,N_{E_n},X_E$は独立である.
        \item $S_{E_1},\cdots,S_{E_n},X_E$も独立である.
    \end{enumerate}
\end{lemma}

\begin{notation}\mbox{}
    \begin{enumerate}
        \item 飛躍の全体を
        \[J(\om):=\Brace{(t,X(t,\om)-X(t-,\om))\in\R_+\times\R\mid X(t,\om)-X(t-,\om)\ne0}\]
        とおく.これは$\Gamma:=\R_+\times\R\setminus\{0\}$の可算な部分集合となる.
        これは集合値確率変数であることに注意.
        \item $N(B,\om):=\abs{J(\om)\cap B}\;(B\in\B(\Gamma))$を,$B$の中に入る飛躍の数とする.これは,$\B(\Gamma)$上の$\o{\N}$-値測度に値を取る確率変数となっている.特に,$N:=(N(B,\om))_{B\in\B(\Gamma)}$は固有偶然配置である.この平均(測度)を$n(B):=E[N(B)]$とする.
    \end{enumerate}
\end{notation}

\begin{lemma}
    $N=(N(B,\om))$は強度$n$のPoisson固有配置である.
\end{lemma}

\begin{lemma}\mbox{}
    \begin{enumerate}
        \item $S_E(t,\om)=\int_{(0,t]}\int_E uN(dsdu)\;(E\in\B(\R\setminus(-a,a)))$.
        \item この確率変数の特性関数は$\exp\paren{\int_{[0,t]}\int_E(e^{izu}-1)n(dsdu)}$で表せる.
        \item $\forall_{t\in\R_+}\;\int_{[0,t]}\int_{\R\setminus\{0\}}(u^2\land 1)n(dsdu)<\infty$.
    \end{enumerate}
\end{lemma}

\begin{notation}
    自然数$n\in\N$に対して,歩幅$u$が$[1/n,1]$に入る跳躍を加え合わせると
    \[S_n(t,\om):=\int_{s\le t}\int_{1/n\le\abs{u}<1}uN(dsdu,\om)=S_E(t,\om)\quad E:=\Brace{u\in\R\setminus\{0\}\mid 1/n\le\abs{u}<1}.\]
    となり,これはLevy過程になる.これを用いて,
    \[T_n(t,\om):=S_n(t,\om)-E[S_n(t,\om)]\]
    は平均$0$のLevy過程である.
\end{notation}

\begin{lemma}
    $m>n$について,
    \[P\Square{\sup_{s\le t}\abs{T_m(s,\om)-T_n(s,\om)}>\ep}\le\frac{1}{\ep^2}\int_{1/m\le\abs{u}<1/n}u^2n_t(du).\]
    ただし,$n_t(E):=n((0,t]\times E)$とした.
\end{lemma}
\begin{remarks}
    これは,$D[0,t]$-値確率変数$T_n(\om):=(T_n(s,\om))_{0\le s\le t}$が,$D$の一様ノルムについて確率収束することが分かった.
    $(D[0,t],\norm{-}_\infty)$は$l^\infty([0,t])$の非可分な閉部分空間であることに注意.
\end{remarks}

\begin{lemma}[Banach空間値確率変数の概収束性]
    任意の$t\in\R_+$に関して,殆ど確実に,$(T_n(s,\om))_{s\in[0,t]}$は$n\to\infty$のとき一様ノルムについて収束する.
\end{lemma}

\begin{notation}
    \[\phi(u):=\begin{cases}
        u\land 1,&u>0,\\
        u\lor(-1),&u<0.
    \end{cases}\]
\end{notation}

\begin{lemma}
    \[\int_{s\le t}\int_{\abs{u}\ge 1/n}\phi(u)n(dsdu)=\int_{\abs{u}\ge 1/n}\phi(u)n_t(du)\]
    は$t$について連続である.
\end{lemma}

\section{Levy-Ito分解}

\begin{tcolorbox}[colframe=ForestGreen, colback=ForestGreen!10!white,breakable,colbacktitle=ForestGreen!40!white,coltitle=black,fonttitle=\bfseries\sffamily,
title=]
    Gauss過程が平均と共分散で特徴付けられたように,Levy過程は特性量$(n,m,v)$で特徴付けられる.
    $\beta$をドリフト,$\al^2\ge0$を拡散係数,$v$をLevy測度という.
\end{tcolorbox}

\begin{theorem}[Levy過程の分解定理 (Levy-Ito decomposition)]
    $X$をLevy過程とする.
    $\Gamma=\R_+\times(\R\setminus\{0\})$上の固有なPoisson配置$N$と,これと独立なGauss型Levy過程$G$が存在して,
    \[X(t,\om)=G(t,\om)+\lim_{n\to\infty}\Square{\int_{s\le t}\int_{\abs{u}>1/n}(uN(dsdu,\om)-\phi(u)n(dsdu))}\]
    と表せる.
    ただし,$n$はPoisson配置$N$の平均測度で,
    \[\int_{s\le t}\int_{\abs{u}\ge0}(u^2\land 1)n(dsdu)<\infty\;(t\in\R_+)\quad n(\{t\}\times\R\setminus\{0\})=0\]
    を満たす.
\end{theorem}

\begin{definition}[Levy-Khintchine triplet]\mbox{}
    \begin{enumerate}
        \item Levy過程$X$の連続部分$G$を用いて,
        \textbf{平均}と\textbf{分散}$m(t):=E[G(t)],v(t):=\Var[G(t)]$は有限確定する.
        \item $n$をPoisson配置$N=N_X$の平均測度という.
        \item 組$(n,m,v)$を,Levy過程$X$の\textbf{特性量}または\textbf{Levy-Khintchine組}という.
    \end{enumerate}
\end{definition}

\begin{lemma}\mbox{}
    \begin{enumerate}
        \item $m$は連続関数で$m(0)=0$.
        \item $v$は連続な単調増加関数で$v(0)=0$.
        \item $n$は$\Gamma$上の測度(Poisson点過程)で,次を満たす:
        \[\forall_{t\in\R_+}\quad n(\{t\}\times(\R\setminus\{0\}))=0,\quad\int_{s\le t}\int_{\abs{u}>0}(u^2\land 1)n(dsdu)<\infty.\]
    \end{enumerate}
\end{lemma}

\begin{theorem}
    補題の条件を満たす$(n,m,v)$に対して,これを特性量として持つLevy過程が存在して,法則同等を除いて一意的である.
\end{theorem}

\begin{corollary}[時間的に一様な場合]
    $X_t-X_s\;(t>s)$の確率法則が$t-s$の値のみに関係するようなLevy過程を,\textbf{時間的に一様なLevy過程}という.
    このとき,特性値は次のように表せる.
    \[m_X(t)=mt,\quad v_X(t)=vt,\quad n_X(dtdu)=dt\cdot n(du).\]
\end{corollary}

\begin{corollary}[構成定理]\label{cor-construction-of-Levy-process}
    確率分布族$\{\mu_{s,t}\}_{s\le t\in\R_+}\subset P(\R)$は,一貫性条件$\mu_{s,t}*\mu_{t,u}=\mu_{s,u}$を満たし,$(s,t)\mapsto P(\R)$は弱位相について連続とする.
    このとき,$\forall_{t>s}\;X_t-X_s\sim\mu_{s,t}$を満たすLevy過程$X$が存在し,法則同等を除いて一意的である.
\end{corollary}

\begin{theorem}[特性関数 Lévy-Khintchine formula]
    Levy過程$X$の特性量を$(n,m,v)$とする.
    \[E[\exp(iz(X(t)-X(s)))]=\exp\Brace{i(m(t)-m(s))z-\frac{1}{2}(v(t)-v(s))z^2+\int_{\abs{u}>0}(e^{izu}-1-i\phi(u)z)n((s,t]\times du)}.\]
\end{theorem}

\section{Brown運動}

\begin{tcolorbox}[colframe=ForestGreen, colback=ForestGreen!10!white,breakable,colbacktitle=ForestGreen!40!white,coltitle=black,fonttitle=\bfseries\sffamily,
title=]
    連続な加法過程は,必然的にGauss型である.これをBrown運動という.
    ドリフトはないものをまずは見る.
\end{tcolorbox}

\subsection{定義}

\begin{tcolorbox}[colframe=ForestGreen, colback=ForestGreen!10!white,breakable,colbacktitle=ForestGreen!40!white,coltitle=black,fonttitle=\bfseries\sffamily,
title=]
    熱方程式$u_t(x,t)=u_{xx}(x,t),u(x,0)=0$の基本解
    \[H(x,t)=\frac{1}{\sqrt{4\pi t}}\exp\paren{-\frac{x^2}{4t}}\]
    は熱核または熱方程式の初期値問題のGreen関数と呼ばれ,初期条件$u(x,0)=f(x)$に関する解は
    \[u(x,t)=H*f=\int_\R H(x-y,t)f(y)dy\]
    と表される.
    熱の拡散と確率の拡散,エントロピーの概念は深いどこかでつながっているのであろうか.
\end{tcolorbox}

\begin{definition}
    確率空間$(\Om,\F,P)$上の実数値確率過程$(B_t)_{t\in\R_+}$が\textbf{Brown運動}であるとは,次の3条件をみたすことをいう:
    \begin{enumerate}
        \item $B_0=0\;\as$
        \item 任意の見本道$B_t(\om):\R_+\to\R$は連続:$B_t\in W$.
        \item 任意の長さ$n\in\N_{>0}$の$\R_+$の狭義増加列$(t_j)_{j\in[n]},t_j=0$が定める増分の組$(B_{t_j}-B_{t_{j-1}})_{j\in[n]}$は互いに独立にGauss分布$N(0,t_{j}-t_{j-1})$に従う.
    \end{enumerate}
\end{definition}
\begin{remarks}
    実は(3)のうち増分の正規性はなくても従うことは,Levy過程の理論による.
\end{remarks}

\begin{theorem}[Brown運動の存在]\label{thm-existence-of-Brownian-motion}
    ある確率空間$(\Om,\F,P)$が存在して,その上にBrown運動が存在する.
\end{theorem}

\subsection{Wiener測度}

\begin{tcolorbox}[colframe=ForestGreen, colback=ForestGreen!10!white,breakable,colbacktitle=ForestGreen!40!white,coltitle=black,fonttitle=\bfseries\sffamily,
title=]
    古典的Wiener空間$W_0$は$0$から始まる連続な見本過程$R_+\to\R$全体の空間で,Banach空間となる.
    Brown運動はここに確率測度を押し出し(見本道のばらつき),Brown運動は,関数解析的には$W_0$上の確率測度の1つと同一視出来る.
\end{tcolorbox}

\begin{notation}[classical Wiener space]
    次の言葉を使えば,Brown運動とはWiener空間に値を持つ確率変数$\Om\to W_0$であって,カリー化$\R_+\to\Meas(\Om,\R)$は任意の有限成分について独立なGauss分布を定めるものをいう.
    \begin{enumerate}
        \item $W=W^1:=C(\R_+)$を連続な見本道の空間とする.
        \item $W_0:=\Brace{w\in W\mid w_0=0}$とする.これを\textbf{古典的Wiener空間}という.
    \end{enumerate}
    それぞれの空間には一様ノルムは入れられないので,広義一様収束位相を考え,Borel集合族によって可測空間とみなす.
    $\R_+$なのでBanach空間ではない.
\end{notation}

\begin{definition}[Wiener measure (23)]
    $(W_0,\B(W_0))$上の射影の族$(B_t)_{t\in\R_+},B_t(\om):=\pr_t(\om)=\om_t$がBrown運動になるような確率測度$P$を\textbf{Wiener測度}という.
\end{definition}

\begin{lemma}
    Wiener測度は一意的に存在する.
\end{lemma}
\begin{Proof}\mbox{}
    \begin{description}
        \item[存在] Brown運動の存在\ref{thm-existence-of-Brownian-motion}による.
        ある空間$(\Om,\F,P)$上のBrown運動$B:\Om\to W$を取る.これによる像測度$P^B$は$P^B(W_0)=1$を満たすから,$W_0$への制限を取れば,これがWiener測度である.
        \item[一意性] $W_0$の柱状集合全体$\cC$\ref{remark-cylinder-sets}上では一意的である.$\cC$は$\pi$-系・情報族であり,$\B(W_0)=\sigma(\cC)$を満たすため,一意に延長される.
    \end{description}
\end{Proof}

\subsection{特性値}

\begin{lemma}[積率]
    $B_t$の奇数次の積率は消えてきて,偶数次の積率は
    \[E[B_t^2]=t,\quad E[B_t^4]=3t^2,\quad E[B^6_t]=15t^3,\quad,E[B^{2n+1}_t]=(2n+1)(2n-1)\cdots 3\cdot 1t^{2n+1}.\]
\end{lemma}

\begin{lemma}[共分散]
    $\forall_{t,s\in\R_+}\;E[B_tB_s]=t\land s$.
\end{lemma}

\subsection{独立増分性}

\begin{tcolorbox}[colframe=ForestGreen, colback=ForestGreen!10!white,breakable,colbacktitle=ForestGreen!40!white,coltitle=black,fonttitle=\bfseries\sffamily,
title=]
    加法過程としての独立増分性は,martingale問題に繋がる.
\end{tcolorbox}

\begin{notation}
    ブラウン運動$(B_t)_{t\in\R_+}$の自然な情報系を$\F^B_t:=\sigma[B_s;s\le t]$と表す.
\end{notation}

\begin{proposition}
    $0\le s<t$に関して,$B_t-B_s$は$\F_s^B$と独立.
\end{proposition}

\begin{corollary}
    Brown運動$(B_t)_{t\in\R_+}$は$(\F^B_t)$に関してmartingaleである.
    その2次変分は$\brac{B}_t=t$で与えられる.
\end{corollary}

\subsection{可微分性}

\begin{theorem}[Paley-Wiener-Zygmond]
    $B_t(\om)$は$\om\dae$に対して,$t$について至る所微分不可能である.
\end{theorem}

\begin{theorem}[modulus of continuity]
    $1/2$-Holder連続性よりやや悪い連続度を持つ.
    \[\limsup_{t_2-t_1=\ep\searrow0,0\le t_1<t_2\le1}\frac{\abs{B_{t_2}-B_{t_1}}}{\sqrt{2\ep\log(1/\ep)}}=1\;\as\]
\end{theorem}

\begin{theorem}[重複対数の法則]
    \[\limsup_{t\searrow0}\frac{B_t}{\sqrt{2t\log\log(1/t)}}=1\;\as\]
\end{theorem}

\section{Poisson過程}

\begin{tcolorbox}[colframe=ForestGreen, colback=ForestGreen!10!white,breakable,colbacktitle=ForestGreen!40!white,coltitle=black,fonttitle=\bfseries\sffamily,
title=]
    $D$-過程であるが連続過程ではなく,見本過程が至る所ジャンプしているとき,これはPoisson型Levy過程である.
    ここまでいかずとも,少しドリフト$0$分散$0$のBrown運動を混ぜて,ジャンプを持つ加法過程で特に基本的なPoisson過程を見る.
\end{tcolorbox}

\subsection{定義}

\begin{tcolorbox}[colframe=ForestGreen, colback=ForestGreen!10!white,breakable,colbacktitle=ForestGreen!40!white,coltitle=black,fonttitle=\bfseries\sffamily,
title=]
    $\R_+$上に強さ$\lambda$のPoisson点過程を考える.その総数を整数で切ったものをPoisson過程と呼ぼう.
\end{tcolorbox}

\begin{definition}[Poisson process]
    $(\Om,\F,P)$上の$\Z_+$-値確率過程$(N_t)_{t\in\R_+}$が\textbf{パラメータ$\lambda>0$を持つPoisson過程}であるとは,次の条件を満たすことをいう:
    \begin{enumerate}
        \item $N_0=0\;\as$
        \item 任意の見本道$N_t(\om)$は右連続かつ単調増加である.
        \item 任意の長さ$n\in\N_{>0}$の$\R_+$の狭義増加列$(t_j)_{j\in[n]},t_0=0$が定める増分の組$(N_{t_j}-N_{t_{j-1}})_{j\in[n]}$は独立で,それぞれパラメータ$\lambda(t_j-t_{j-1})$を持つPoisson分布に従う.
    \end{enumerate}
\end{definition}

\begin{discussion}[Poisson過程の構成]
    Kolmogorovの拡張定理により存在は保証されるが,次のように構成できる.
    独立にパラメータ$\lambda$の指数分布\ref{def-exponential-distribution}に従う$\R_+$-値確率変数列$(S_j)$を取る:$P[S_j\ge t]=e^{-\lambda t}$.
    なお,パラメータ$\lambda>0$の指数分布密度関数は
    \[p(x)=\lambda e^{-\lambda x}1_{[0,\infty)}(x)\]
    と表せる.
    このとき
    \[Z_0=0,\quad Z_k=\sum^k_{j=1}S_j\]
    とするとこれは強さ$\lambda$のPoisson点過程であり,
    $t\in\R_+$を超えた$Z_k$の数の過程
    \[N_t:=\max\Brace{k\in\N\mid Z_k\le t}\in\o{\Z}_+\]
    はPoisson過程となり,$P[N_t\in\Z_+]=1$.
\end{discussion}

\begin{theorem}
    $(U_n)$を$\R$上の増加酔歩で,$T_n:=U_n-U_{n-1}\sim\Exp(c)\;(c>0)$とする.
    これに対して,$X_t(\om):=n\;\st U_n(\om)\le t<U_{n+1}(\om)$と定めると,$X_t\sim\Pois(ct)$で,$(X_t)_{t\in\R_+}$はLevy過程である.
    これをパラメータ$c>0$のPoisson過程という.
\end{theorem}
\begin{remarks}
    $X_t$は$t$までに起こった
    跳躍の回数で,$U_n$は$n$回目の跳躍が起こる時刻である.
\end{remarks}

\subsection{複合Poisson過程}

\begin{theorem}[compound Poisson process]
    $\sigma\in P(\R^d)$は$\sigma(\{0\})=0$を満たすとする.
    パラメータ$c>0$のPoisson過程$(N_t)_{t\in\R_+}$と$\R^d$上の酔歩$(S_n)_{n\in\N},S_1\sim\sigma$は独立であるとする.
    このとき,$X_t(\om):=S_{N_t(\om)}(\om)$は$\R^d$上のLevy過程である.これを\textbf{$\sigma,c$が定める複合Poisson過程}という.
\end{theorem}

\begin{definition}
        $m=0,v=0$のときの$\psi(z)=\int_{\abs{u>0}}(e^{izu}-1)n(du)$と表せる$\mu$のクラスを,\textbf{複合Poisson分布}という.これに対応する過程を\textbf{複合Poisson過程}という.
\end{definition}

\subsection{独立増分性}

\begin{proposition}
    $0\le s<t$について,$N_t-N_s$は$\F^N_s=\sigma[N_u;u\le s]$と独立である.
\end{proposition}

\begin{corollary}
    $(N_t-\lambda t)_{t\ge0}$は$(\F_t^N)$に関してマルチンゲールである.
\end{corollary}

\section{無限分解可能分布}

\begin{tcolorbox}[colframe=ForestGreen, colback=ForestGreen!10!white,breakable,colbacktitle=ForestGreen!40!white,coltitle=black,fonttitle=\bfseries\sffamily,
title=]
    Levy過程の特性関数は,一般の無限分解可能分布に敷衍できる.
    これはLevy過程の構成に必要だった一貫性条件\ref{cor-construction-of-Levy-process}からも予見できたかもしれない.

    一様なLevy過程は,ある1-パラメータ連続変換半群$\{\mu_t\}_{t\in\R_+}$で定まる.このとき$\mu_1$は無限可解分布である.
\end{tcolorbox}

\subsection{定義と特徴付け}

\begin{tcolorbox}[colframe=ForestGreen, colback=ForestGreen!10!white,breakable,colbacktitle=ForestGreen!40!white,coltitle=black,fonttitle=\bfseries\sffamily,
title=]
    確率変数$X$が,任意の$n\in\N$に対して,ある分布$\mu_n$が存在してそれに従う独立同分布変数$n$個の和で表せるとき,
    これを無限可解であるという.
    この連続化は半群の構造に注目する,本質的に1-パラメータ化に同じ.
    これは半群$(P(\R),*,\delta_0)$の言葉を用いて定義できる:連続半群の準同型$\R_+\to P(\R)$であって,$\mu_1=\mu$を満たすものが存在すること.
    しかも$\delta_0$は極点なので,ここからループを投げるみたいな,$P(\R)$の幾何的な知識と繋がることを表す!
\end{tcolorbox}

\begin{definition}[infinitely decomposable distribution]
    1次元の分布$\mu\in P(\R)$が\textbf{無限可解}であるとは,分布族$\{\mu_t\}_{t\in\R_+}$が存在して,$\mu$を乗法単位元として$\R_+$と同型な連続半群となることをいう:
    \begin{enumerate}
        \item $\R_+\to P(\R);t\mapsto\mu_t$は弱位相に関して連続.
        \item $\forall_{t,s\in\R_+}\;\mu_t*\mu_s=\mu_{t+s}$.
        \item $\mu_0=\delta_0$.
        \item $\mu_1=\mu$.
    \end{enumerate}
    すなわち,連続半群の準同型$\R_+\to P(\R)$であって,$\mu_1=\mu$を満たすものが存在することをいう.
    なお,パラメータの連続変換により,このとき任意の$\mu_t$も無限可解になることに注意.
\end{definition}

\begin{lemma}[無限可解性の特徴付け]
    次の4条件は同値.
    \begin{enumerate}
        \item $\mu$は無限可解である.
        \item $\forall_{n\in\N}\;\mu=\mu_n*\mu_n*\cdots *\mu_n$と表せる.
        \item 任意の$\ep>0$に対し,分解$\mu=\mu_1*\cdots*\mu_n$であって,Levy距離に関して$d_L(\mu_i,\delta)<\ep$を満たすものが存在する.
        \item $\mu_n=\mu_{n1}*\cdots*\mu_{nm(n)},\max_{k\in[m(n)]}d_L(\mu_{nk},\delta)\to0$を満たす列$(\mu_n)$が存在して,$\mu_n\to\mu$.
    \end{enumerate}
\end{lemma}

\begin{example}[reproducing property]
    畳み込みの演算について閉じている(そして何らかの関手性を持つ)分布族を\textbf{再生性}というのであった.
    この分布族の元は,$\delta_0$を含む場合,その分布族の中で完結して取れる.
    たとえば$\mu=N(m,v)$については$\mu_t:=N(tm,vm)$と取れば良い.
\end{example}

\begin{example}[時間的に一様なLevy過程]
    時間的に一様なLevy過程$X$は,$\{\mu_t\sim X_t-X_0\}_{t\in\R_+}$がちょうど1-パラメータ連続半群となり,
    よっておのずと無限可解である.
    また逆に,連続半群$(\mu_t)$に対して,これが定める一様なLevy過程が存在するのは構成定理\ref{cor-construction-of-Levy-process}による.
\end{example}

\begin{example}
    安定分布(正規分布,Cauchy分布,片側Levy分布),複合Poisson分布,$F$分布,対数正規分布,$t$分布など正規分散平均混合[増田弘毅, 2002].(Steutel and van Harn, 2004).
\end{example}

\subsection{Levy分解}

\begin{tcolorbox}[colframe=ForestGreen, colback=ForestGreen!10!white,breakable,colbacktitle=ForestGreen!40!white,coltitle=black,fonttitle=\bfseries\sffamily,
title=]
    無限可解分布,連続半群$\{\mu_t\}_{t\in\R_+}$,時間的に一様なLevy過程(の法則同値類)$X$の間に,次の全単射対応がある:
    \[\mu\xrightarrow{\mu=\mu_1}\{\mu_t\}\xrightarrow{\mu_t=P^{X(t)}}X\]
\end{tcolorbox}

\begin{theorem}[無限可解分布のLevy分解定理]
    任意の無限可解分布$\mu$は,特性関数$\F\mu(z)$が$\F\mu(z)=e^{\psi(z)}$なる形になる.ただし,
    \[\psi(z)=imz-\frac{v}{2}z^2+\int_{\abs{u}>0}(e^{izu}-1-i\phi(u)z)n(du),\quad m\in\R,v\ge0,\int_{\abs{u}>0}(u^2\land 1)n(du)<\infty.\]
\end{theorem}

\begin{corollary}
    無限可解分布,連続半群$\{\mu_t\}_{t\in\R_+}$,時間的に一様なLevy過程(の法則同値類)$X$の間に,次の全単射対応がある:
    \[\mu\xrightarrow{\mu=\mu_1}\{\mu_t\}\xrightarrow{\mu_t=P^{X(t)}}X\]
\end{corollary}

\begin{example}
    Poisson分布,Cauchy分布もLevy分解を定め,これらに対応する一様Levy過程をPoisson過程,Cauchy過程という.
\end{example}

\subsection{複合Poisson過程}

\begin{discussion}
    $\mu$を無限分解可能分布とする.
    \[\int_{\abs{u}\in(0,1)}\abs{u}n(du)<\infty\]
    が成り立つとき,$\phi(u)=0$と取ってよい.
    このとき$\mu$に対応する特性値$(n,m,v)$がただ一つ存在し,特性関数は
    \[\psi(z)=imz-\frac{v}{2}z^2+\int_{\abs{u}>0}(e^{izu}-1)n(du)\]
    によって定まる.
\end{discussion}

\begin{definition}
    特に$n$が$\R\setminus\{0\}$上で有界で$m=v=0$とする.このとき対応する特性関数は
    \[\psi(z)=\int_{\abs{u}>0}(e^{izu}-1)n(du)\]
    によって定まり,このときの無限可解分布$\mu$を\textbf{複合Poisson分布}といい,対応する時間的に一様なLevy過程を\textbf{複合Poisson過程}という.
\end{definition}

\begin{lemma}[複合Poisson分布の特徴付け]
    $\lambda:=n(\R\setminus\{0\})<\infty$とする.
    複合Poisson分布$\mu$は,$v^{*n}$をPoisson分布$\Pois(\lambda)$によって加重平均を取ったものである:
    \[\mu=\sum^\infty_{n=1}e^{-\lambda}\frac{\lambda^n}{n!}\nu^{*n}.\]
\end{lemma}

\begin{proposition}
    $N\sim\Pois(\lambda),(X_n)_{n\in\N}\overset{\iid}{\sim}\nu$で互いに独立とする.
    \[Y:=X_1+X_2+\cdots+X_N\]
    は複合Poisson分布$p_{\lambda,\nu}$に従う.
\end{proposition}
\begin{remarks}
    $dt$間に$\lambda dt$の確率で事故が起こり,そのときの損害額は分布$\nu$に従うとする.
    独立性の仮定の下で,時刻$t$までの被害総額$X(t,\om)$は複合Poisson分布$p_{\lambda,\nu}$に対応する時間的に一様なLevy過程となる.
\end{remarks}

\chapter{Markov過程}

\begin{quotation}
    独立性は一切の過去の履歴に依らないが,Markov性は,現在の状態のみに依存する性質を指す.
    加法過程はMarkov過程である.

    Brown運動は状態空間,時間パラメータのいずれも連続な場合であり,Poisson過程は状態空間は離散的である例である.
    状態空間が離散的な場合,\textbf{Markov連鎖}ともいう.
    またここで偏微分方程式との関係から,確率過程の一般化も自然に出現する.
    添字集合を多様体$M$とした確率過程$\Om\times M\to\R^n$を\textbf{確率場}という.
    このとき,時間概念が空間に置き換わっている.

    $x_{n+1}$の$x_n$への依存の仕方は経時変化しないという,時間的一様性の仮定をおいて議論する.
    すると,Markov過程は推移作用素を定めることで分布が決まる.
    これは大数の法則を一般化する.
    また,推移作用素になり得る作用素は放物型偏微分方程式によって特徴付けられる.
\end{quotation}

\begin{notation}\mbox{}
    \begin{enumerate}
        \item $I$を可算集合とし,これを離散値の場合の\textbf{状態空間}とする.
        見本過程は列$\N\to I$となり,経時的に$I$上を動き回ることになる.
        \item $\one$はすべての成分が$1$であるような縦ベクトルも表す.
        \item $\delta_i$で,$i$成分のみが$1$でそれ以外が$0$であるようなベクトルを表す.
    \end{enumerate}
\end{notation}

\section{Markov過程の定義と核}

\begin{tcolorbox}[colframe=ForestGreen, colback=ForestGreen!10!white,breakable,colbacktitle=ForestGreen!40!white,coltitle=black,fonttitle=\bfseries\sffamily,
title=]
    Markov過程は初期分布と転移確率(が存在するならば)の2つによって,法則同等を除いて一意に定まる.
    これを連続の場合も含めて議論するには,「転移確率」の定義を考える必要がある.
\end{tcolorbox}

\subsection{Markov連鎖の定義と特徴付け}

\begin{definition}[Markov chain]\label{def-Markov-chain}
    確率過程$(X_n)_{n\in\N}$について,
    \begin{enumerate}
        \item \textbf{Markov過程}であるとは,Markov性を持つことをいう:
        \[\forall_{n\in\N}\;\forall_{E\in\B^1}\;P[X_{n+1}\in E|\sigma[X_1,\cdots,X_n]]=P[X_{n+1}\in E|\sigma[X_n]]\;\as\]
        \item \textbf{$I$上のMarkov連鎖}であるとは,Markov性を持つことをいう.この場合のMarkov性は次に同値:
        \[\forall_{n\in\N}\;\forall_{a_0,\cdots,a_{n+1}\in I}\;P[X_{n+1}=a_{n+1}|X_0=a_0,\cdots,X_n=a_n]=P[X_{n+1}=a_{n+1}|X_n=a_n]\;\as\]
        \item さらに,$I$が可算集合であるとき,\textbf{可算Markov過程}ともいう\cite{Popov21-RandomWalk}.
    \end{enumerate}
\end{definition}

\begin{theorem}[Markov性の特徴付け]
    $(X_n)$について,次は同値:
    \begin{enumerate}
        \item Markov性を持つ.
        \item (連続時への一般化の道) $\forall_{n,m\in\N}\;\forall_{E\in\B^1}\;P[X_{n+m}\in E|\sigma[X_1,\cdots,X_n]]=P[X_{n+m}\in E|\sigma[X_n]]\;\as$
        \item (関数による表現) $\forall_{f\in L^\infty(\Om)}\;\forall_{n\in\N}\;E[f(X_{n+1})|(X_1,\cdots,X_n)]=E[f(X_{n+1})|X_n]\;\as$
    \end{enumerate}
\end{theorem}

\subsection{連続時間の場合の定義}

\begin{definition}[Markov process]\label{def-Markov-process}
    $(X_t)_{t\in\R_+}$が\textbf{Markov過程}であるとは,Markov性を持つことをいう:
    \[\forall_{u>t\in\R_+}\;\forall_{E\in\B(\R)}\;P[X_u\in E|X_s,s\le t]=P[X_u\in E|X_s]\;\as\]
\end{definition}

\begin{theorem}[Markov性の特徴付け]
    $(X_t)$について,次は同値:
    \begin{enumerate}
        \item Markov性を持つ.
        \item (関数による表現) $\forall_{f\in L^\infty(\Om)}\;\forall_{u>t\in\R_+}\;E[f(X_{u})|X_s,s\le t]=E[f(X_{u})|X_t]\;\as$
        \item (稠密部分集合) $\forall_{f\in C_c^\infty(\R)}\;\forall_{u>t\in\R_+}\;E[f(X_{u})|X_s,s\le t]=E[f(X_{u})|X_t]\;\as$
        \item (可算な表現) 任意の$0\le s_1<s_2<\cdots<s_n<t$に対して,
        \[P[X_t\in E\mid X_{s_1},\cdots,X_{s_n}]=P[X_t\in E\mid X_{s_n}].\]
    \end{enumerate}
\end{theorem}

\subsection{Markov連鎖の例}

\begin{example}[Wright-Fisher model]
    $N$個の白黒二種類の球を壺$A$に入れる.
    壺$A$から球を復元抽出して色を確認し,同じ色の球を$B$に追加し,$\abs{B}=N$になったら停止して,$B$の中の黒石の和を$X_1$とする.
    $A,B$の役割を入れ替えて同じことを繰り返す.
    \[X_0=x_0,\quad X_{n+1}|X_n\sim \rB(N,X_n/N),\qquad x_0\in[N]\]
    はMarkov連鎖となり,これを\textbf{Weight-Fisher模型}という.Feller 1951で数学的に扱われた,集団遺伝学の模型である.
    このMarkov連鎖は次を満たす:
    \[\lim_{n\to\infty}P[X_m\in\partial[0,M]]=1.\]
\end{example}

\begin{definition}[branching process]
    Galtonが苗字の消滅のモデルとして開発した,絶滅のモデル.
    $\N$-値確率変数の独立同分布な二重列$\{\xi_j^{(n)}\}_{n,j\in\N}$について,
    \[X_0=1,\quad X_{n+1}=\sum_{j=i}^{X_n}\xi^{(n)}_j.\]
    と定めて得られるMarkov過程を\textbf{分岐過程}という.事実,子孫の数は現在の個体数のみに依存する.
    $\lim_{n\to\infty}P[X_n=0]$を\textbf{絶滅確率}という.
\end{definition}

\begin{example}[autoregressive process]
    \[X_0=x_0,\quad X_{n+1}=\mu+\al(X_n-\mu)+B_{n+1},\qquad\mu,\al,x_0\in\R,\sigma>0,B_n\sim\rN(0,\sigma^2)\;\iid\]
    で定まる過程を\textbf{Gauss型自己回帰過程}という.$\al=1$の場合を酔歩という.このとき,
    \[X_n\sim\rN(\mu_n,\sigma^2_n),\quad\mu_n:=\mu+\al^n(x_0-\mu),\sigma^2_n:=\sum_{i\in n}\al^{2i}\sigma^2.\]
    特に,$(X_n)$は$\rN(\mu,\frac{\sigma^2}{1-\sigma^2})$に分布収束する.
    さらに,次の2条件を満たすものを\textbf{定常な自己回帰過程}という:
    \begin{enumerate}
        \item $\abs{\al}<1$.
        \item 初期値$x_0$は次のように確率的に当たる:$X_0\sim\cN\paren{\mu,\frac{\sigma^2}{1-\al^2}}$.
    \end{enumerate}
\end{example}

\subsection{転移核の定義:可測性の問題}

\begin{tcolorbox}[colframe=ForestGreen, colback=ForestGreen!10!white,breakable,colbacktitle=ForestGreen!40!white,coltitle=black,fonttitle=\bfseries\sffamily,
title=]
    Markov過程は$t,x,u$に依存する条件付き確率
    \[P(t,x,u;A):=P[X_{u}\in A|X_t=x]\]
    と初期分布とが,法則同等を除いて一意に特徴付ける.これは確率測度とは限らず,密度を持つとも限らない.
    この,$(t,x,u)\in\R_+\times E\times\R_+$上の測度値関数を\textbf{Markov核}という.
\end{tcolorbox}

\begin{definition}[translation / Markov kernel, translation probability]
    $X$をMarkov過程とする.
    \begin{enumerate}
        \item $\R$上の条件付き測度の$t\le u\in\R$による族
        \[p(t,x,u;E):=P[X_u\in E|X_t=x],\qquad E\in\B(\R)\]
        を\textbf{広義の転移確率}または\textbf{Markov核}という.
        これは転移作用素からみて積分核になるためである.
        \item 条件付き測度に関する知識から,$p(t,-,u;E)$は$x\in\R$に関する可測関数になるように取れることが解る.
        \item Markov核が例外点なしにChapman-Kolmogorov方程式を満たす(命題\ref{prop-Chapman-Kolmogorov}を$(P^{X(s)})\dae x$の制約なしに満たす)場合,特に\textbf{転移確率}と呼ぶ.
        したがって,
        条件付き確率測度が存在するとは限らないのと同様,一般のMarkov過程に転移確率は存在するとは限らない.
    \end{enumerate}
\end{definition}

\begin{lemma}[Markov核の一意性]
    転移確率$x\mapsto p(t,x,u;E)$は一意とは限らないが,零集合の差を除いて定まる.すなわち,
    任意の2つの転移確率$p,q$は,\[p(t,x,u,E)=q(t,x,u,E)\;P^{X_t}\dae x\]を満たす.
    また,例外集合は$E$に無関係に取れるが,$t,u$には依存する.
\end{lemma}

\begin{observation}[Markov半群のなす半群の連続化]
    離散集合$I$上の遷移行列$\bP$が満たす規則は次のようにかける.
    \begin{enumerate}
        \item $\forall_{i\in I}\;\sum_{j\in I}p_{ij}^{(n)}=1$.
        \item $\forall_{i,j\in I}\;\forall_{n,m\in\N}\;\sum_{k\in I}p_{ik}^{(n)}p_{kj}^{(m)}=p_{ij}^{(n+m)}$.
    \end{enumerate}
    $I$を一般のポーランド空間,$\N$を$\R_+$へ,遷移行列を遷移作用素へ一般化したい.
    \begin{enumerate}
        \item $\forall_{s\in\R_+}\;\forall_{x\in S}\;p(s,x,-)\in P(S)$.時刻$0$に$x$から出発するMarkov過程の,時刻$s$での位置の分布.
        \item $\forall_{s,t\in\R_+}\;\forall_{x,z\in S}\;\int_Sp(s,x,dy)p(t,y,dz)=p(s+t,x,dz)$.
        または,$\forall_{A\in\B(S)}\;\int_Sp(s,x,dy)p(t,y,A)=p(s+t,x,A)$.
    \end{enumerate}
    こうして,行列積は積分に一般化される.(2)を時間一様なChapman-Kolmogorovの等式という.
    これは,時刻$0$に$x$から初めて,$s+t$に$A$に至るまでの時刻$s$での経由地$y\in S$について積分しても等しくなる,という意味を持つ.
\end{observation}

\begin{proposition}[Chapman-Kolmogorov]\label{prop-Chapman-Kolmogorov}
    Markov核$p$は,任意の$s<t<u$について,次を満たす:
    \[p(s,x,u,E)=\int_\R p(s,x,t,dy)p(t,y,u,E)\quad(P^{X(s)})\dae x.\]
    また,例外集合は$E$に無関係に取れるが,$s,t,u$には依存する.
\end{proposition}

\subsection{転移確率の定義}

\begin{tcolorbox}[colframe=ForestGreen, colback=ForestGreen!10!white,breakable,colbacktitle=ForestGreen!40!white,coltitle=black,fonttitle=\bfseries\sffamily,
    title=]
    Markov過程の発展がある密度関数のなす半群によって記述できるための条件を表す放物型偏微分方程式をChapman-Kolmogorov方程式という.
    これを解いて推移確率とし,Kolmogorovの拡張定理に基づけば拡散過程が構成できる.
    Kolmogorovは初期から物理学への応用を見据えて,多様体の言葉で論じていた.
\end{tcolorbox}

\begin{definition}[transition probability]
    $X$をMarkov過程とする.
    \begin{enumerate}
        \item 次の3条件を満たす関数$p:(s,x,t,E):\R_+\times\R\times\R_+\times\B(\R)\to[0,1]\;(s<t)$を(狭義の)\textbf{転移確率}という:
        \begin{enumerate}[(a)]
            \item $p(s,-,t,E):\R\to[0,1]$はBorel可測.
            \item $p(s,x,t,-):\B(\R)\to[0,1]$は確率測度.
            \item $\forall_{0\le s<t<u\in\R_+}\;\forall_{x\in\R}\;\forall_{E\in\B(\R)}\;p(s,x,u,E)=\int_\R p(s,x,t,dy)p(t,y,u,E)$.
        \end{enumerate}
        \item 転移確率$p$がMarkov過程$(X_t)$を定めるとは,
        \[\forall_{s,t,E}\;p(s,X(s),t,E)=P[X_t\in E|X_s]\;P\dae\]
        が成り立つことをいう.
        \item 転移確率$p$が\textbf{時間的に一様}であるとは,$p(s,x,t,E)=p(0,x,t-s,E)$が成り立つことをいう.
    \end{enumerate}
\end{definition}
\begin{remarks}
    条件(c)は$(P_t)_{t\in\R_+}$が半群をなすことに同値.
\end{remarks}

\begin{lemma}[転移確率が定めるMarkov過程]
    $p$を転移確率とするMarkov過程$(X_t)$について,任意の$n\in\N,0\le s_1<\cdots<s_n$と任意の有界Borel可測関数$f:\R^n\to\R$について
    \[E[f(X_{s_1},\cdots,X_{s_n})]=\int_\R\cdots\int_\R p_{s_1}(dx_1)p(s_1,x_1,s_2,dx_2)\cdots p(s_{n-1},x_{n-1},s_n,dx_n)f(x_1,x_2,\cdots,x_n),\quad  p_{s_1}(E):=P[X_{s_1}\in E].\]
\end{lemma}

\begin{corollary}
    Markov過程は,初期分布$p_0(E):=P[X_0\in E]$と転移確率の2つによって,法則同等を除いて一意に定まる.
\end{corollary}
\begin{remark}
    Levy過程のときのような構成定理をいうには,Kolmogorovの拡張定理を$\R^\infty$上から$\R^{\R_+}$上へと一般化する必要がある.
\end{remark}

\subsection{加法過程はMarkov過程である}

\begin{theorem}
    $(X_t)$を加法過程とする.
    \[\mu_{s,t}(E):=P[X_t-X_s\in E]\;(s<t)\]
    とおくと,$X$は$p(s,x,t,E):=\mu_{s,t}(E-x)$を転移確率とするMarkov過程である.
\end{theorem}

\subsection{Markov過程の構成}

\begin{proposition}[初期分布と転移確率による構成]
    適当な確率空間の上に,初期分布$\nu$と遷移行列$\bP$をもち,殆ど至る所$I$値なMarkov連鎖が存在する.
\end{proposition}
\begin{Proof}
    $I$は可算だから単射$I\mono\N$が存在する.以降,$I\mono\N\mono\R$として,$\R$の部分集合と同一視する.
    \begin{description}
        \item[構成] 各$n\in\N$に対して,$(\R^{n+1},\B(\R^{n+1}))$上の測度$P_{n+1}$を
        \[P_{n+1}(A):=\sum_{(i_0,\cdots,i_n)\in I^{n+1}\cap A}\nu_{i_0}p_{i_0i_1}\cdots p_{i_{n-1}i_n}\quad(A\in\B(\R^{n+1}))\]
        とすると,これはたしかに確率測度である.
        \item[一貫性] 任意の$A\in\B(\R^{n+1})$について,
        \[P_{n+2}(A\times\R)=\sum_{(i_0,\cdots,i_n)\in I^{n+1}\cap A}\nu_{i_0}p_{i_0i_1}\cdots p_{i_{n-1}i_n}\paren{\sum_{i_{n+1}\in I}p_{i_ni_{n+1}}}=P_{n+1}(A).\]
        \item[検証] Kolmogorovの拡張定理\ref{thm-Kolmogorov-extension-theorem}より,$(\R^\N,\B(\R^\N))$上の確率測度$P$であって,$P(A\times\R^\N)=P_{n+1}(A)$を満たすものがただ一つ存在する.
        この空間上の実数値確率変数列$(X_n)$を,$X_n(\om)=\om_n\;(\om=(\om_0,\cdots)\in\R^\N)$と定めれば,これは殆ど至る所$I$-値の,求めるMarkov過程である.
    \end{description}
\end{Proof}

\begin{proposition}[単射による保存]
    単射$f:\R\to\R$について,Markov過程$(X_t)$の像$(f(X_t))$はMarkov過程である.
\end{proposition}

\begin{proposition}
    Brown運動$(B_t)$に対して,$(B_t^2)$はMarkov過程である.
\end{proposition}

\section{生成作用素}

\begin{tcolorbox}[colframe=ForestGreen, colback=ForestGreen!10!white,breakable,colbacktitle=ForestGreen!40!white,coltitle=black,fonttitle=\bfseries\sffamily,
title=]
    Banach空間上の線型作用素の1径数半群に関するHille-Yoshida理論は,
    Markov過程の作用素論的な解析を可能にする.
\end{tcolorbox}

\subsection{生成作用素の定義}

\begin{tcolorbox}[colframe=ForestGreen, colback=ForestGreen!10!white,breakable,colbacktitle=ForestGreen!40!white,coltitle=black,fonttitle=\bfseries\sffamily,
title=]
    生成作用素により,偏微分方程式論とマルチンゲール論が交差する.
\end{tcolorbox}

\begin{definition}[transition operator]
    $X$をMarkov過程とする.
    \begin{enumerate}
        \item 時刻$0$からの\textbf{遷移作用素}
        $\{P_{s,t}\}_{s,t\in\R_+}\subset B(L^\infty(\R))$とは,遷移確率との積分を取る対応
        \[P_{s,t}[f](x):=E[f(X_t)|X_s=x]=\int_\R p(s,x,t;dy)f(y)\]
        をいう.遷移確率の定義から,
        遷移作用素は半群をなす.
        \item 生成作用素$(A_s)$とは,$L^\infty(\R)$上の一様ノルムに関する極限によって定まる対応
        \[A_s[f]:=\lim_{h\searrow0}\frac{P_{s+h}[f]-P_s[f]}{h}\]
        をいう.
    \end{enumerate}
\end{definition}

\begin{remarks}
    遷移作用素$(P_{s,t})$の言葉を使えば,Markov性は
    \[\forall_{f\in L^\infty(\R)}\quad E[f(X_t)|\F_s]=P_{s,t}f(X_s)\quad Q\das\]
    と表せる.
    さらに$P_{s,t}$が$P_{\abs{t-s}}$で表せるとき,時間的に一様であるという.
\end{remarks}

\begin{proposition}
    $X$を時間的に一様なMarkov過程,$A$をその生成作用素とする.
    \begin{enumerate}
        \item $\dd{}{t}P_t[f](x)=AP_t[f](x)$.
        \item $u(t,x):=P_t[f](x)$とおくと,次の偏微分方程式を満たす:
        \[\pp{u}{t}(t,x)=Au(t,x),\quad u(0,x)=f(x).\]
        \item 過程$M^f$を
        \[M^f_t:=f(X_t)-f(X_0)-\int^t_0A[f](X_s)dx,\quad t\ge0\]
        と定めると,これはマルチンゲールになる.
    \end{enumerate}
\end{proposition}

\subsection{Kolmogorovの定理}

\begin{theorem}
    転移確率$p(s,x,t;E)$は次を満たすとする:
    \begin{enumerate}
        \item $\int_\R p(s,x,s+h;dy)(y-x)=a(s,x)h+o(h)$.
        \item $\int_\R p(s,x,s+h;dy)(y-x)^2=b(s,x)h+o(h)$.
        \item $\int_\R p(s,x,s+h;dy)(y-x)^3=o(h)$.
    \end{enumerate}
    このとき,生成作用素は次のように表せる:
    \[A_s=a(s,x)\dd{}{x}+\frac{1}{2}b(s,x)\dd{^2}{x^2}.\]
\end{theorem}

\section{可算Markov連鎖}

\begin{tcolorbox}[colframe=ForestGreen, colback=ForestGreen!10!white,breakable,colbacktitle=ForestGreen!40!white,coltitle=black,fonttitle=\bfseries\sffamily,
title=]
    Markov過程には,拡散過程として到達確率を見る解析と,その極限分布を見る解析との2つがある.
    これを状態空間と時間とのいずれも離散の場合に見る.
\end{tcolorbox}

\subsection{定義と存在}

\begin{definition}[time homogeneous countable Markov chain]
    $\{X_n\}\subset L(\Om;\Sigma)$が\textbf{均一発展する可算Markov連鎖}であるとは,次を満たすことをいう:
    \begin{enumerate}[{[A}1{]}]
        \item Markov性:$\forall_{y\in\Sigma}\;\forall_{n\in\N}\;\forall_{m\in\N^+}\;P[X_{n+m}=y|X_0,\cdots,X_n]=P[X_{n+m}=y|X_n]\;\as$
        \item 時間一様性:$P[X_{n+m}=y|\F_n]$は次の意味で$n$に依らない:
        \[P[X_{n+m}=y|\F_n]=:p_m(X_n,y)\;\as\]
        \item 可算性:状態空間$\Sigma$は可算集合である.
    \end{enumerate}
\end{definition}

\begin{theorem}
    任意の可算集合$\Sigma$,その上の確率分布$\nu\in P(\Sigma)$,遷移行列$P=\{p_1(x,y)\}_{x,y\in\Sigma}$について,
    これに対応する可算Markov連鎖が存在する.
\end{theorem}

\subsection{Chapman-Kolmogorov方程式}

\begin{proposition}
    $(X_n)$を可算Markov連鎖とする.
    可算状態空間$\Sigma$上の
    遷移確率$(p_m(X_n,y):=P[X_{n+m}=y|\F_n])_{X_n,y\in\Sigma}$は次の意味で半群性を持つ:
    \[p_{n+m}(x,y)=\sum_{z\in\Sigma}p_n(x,z)p_m(z,y),\qquad(x,y\in\Sigma).\]
\end{proposition}

\subsection{可算Markov連鎖の到達確率}

\begin{tcolorbox}[colframe=ForestGreen, colback=ForestGreen!10!white,breakable,colbacktitle=ForestGreen!40!white,coltitle=black,fonttitle=\bfseries\sffamily,
    title=]
    差分は前進$\Delta f(x):=f(x+1)-f(x)$と後退$\nabla f(x):=f(x)-f(x-1)$の2つが考えられる.
    これが連続になると確率微分方程式となるのだ.
    到達確率は差分作用素に関する偏微分方程式の解として特徴付けられる.
\end{tcolorbox}

\begin{definition}[hitting / absorption probability]
    $(X_n)$を可算Markov連鎖とする.
    \begin{enumerate}
        \item 集合$A\subset I$に対して,$A$への到達時刻とは,$\tau_A:=\min\Brace{n\in\N\mid X_n\in A}$として定まるMarkov時$\Om\to\o{\N}$であった\ref{exp-discrete-Markov-time}.
        \item 到達時刻が有限になる確率を\textbf{到達確率}または\textbf{吸収確率}といい,次のように表す:
        \[a_i:=P_i[\tau_A<\infty]:=P[\tau_A<\infty|X_0=i].\]
        \item \textbf{差分作用素}$\L:L^\infty(\Sigma)\to L(\Sigma)$を,
        \[\L[f](i):=\sum_{j\in I}p_{ij}f(j)-f(i),\qquad(i\in I,f\in L^\infty(\Sigma)).\]
        と定める.
    \end{enumerate}
\end{definition}

\begin{theorem}[到達確率の特徴付け]
    到達確率$(a_i)_{i\in I}$は,方程式系
    \[\begin{cases}
        \L[a](i)=0&i\notin A,\\
        a_i=1&i\in A.
    \end{cases}\]
    の最小の非負解である.後者は前者の\textbf{境界条件}という.
\end{theorem}
\begin{remarks}
    差分方程式を書き直すと,$i\in I\setminus A$に関して$a_i=\sum_{j\in I}p_{ij}a_j$となり,$i$からの遷移確率に関する,到達確率の平均になる.
\end{remarks}

\begin{theorem}[差分作用素の定めるマルチンゲール]
    $(X_n)$を可算Markov連鎖,$f\in L^\infty(I)$を有界関数とする.
    \[Y_n:=f(X_n)-f(X_0)-\sum_{k=0}^{n-1}\L f(X_k)\]
    によって定まる過程$(Y_n)$は$(\F_n)$-マルチンゲールである.
\end{theorem}

\begin{remarks}[martingale problem]
    一般に,線型作用素$\L$に対して,$\L$を差分作用素とするMarkov過程$(X_n)$を見つける問題を\textbf{martingale問題}という\cite{Stroock-Varadhan06-MultidimensionalDiffusionProcesses}.
\end{remarks}

\subsection{可算Markov連鎖のエルゴード性}

\begin{tcolorbox}[colframe=ForestGreen, colback=ForestGreen!10!white,breakable,colbacktitle=ForestGreen!40!white,coltitle=black,fonttitle=\bfseries\sffamily,
title=]
    確率行列$\bP$の,確率分布の空間$P(I)$への作用を考えると,不動点が存在する.
\end{tcolorbox}

\begin{definition}[primitive / ergodic, irreduciable, aperiodic]
    Markov連鎖$((X_n),I,\bP)$あるいはその確率行列$\bP=(p_{ij})$について,$\bP^n=:(p_{ij}^{(m)})$と表す.
    \begin{enumerate}
        \item $\bP$が\textbf{原始的}または\textbf{エルゴード的}であるとは,$\exists_{n_0\in\N}\;\bP^{n_0}>0$が成り立つことをいう.
        \item $\bP$が\textbf{既約}であるとは,$\forall_{i,j\in I}\;\exists_{n_1\in\N}\;p_{ij}^{(n_1)}>0$を満たすことをいう.
        \item 既約行列$\bP$の\textbf{周期}とは,$\gcd\Brace{n\ge1\mid p^{(n)}(x,x)>0}$をいう.この値は$x\in\Sigma$の取り方に依らない.
        \item 既約行列が\textbf{非周期的}であるとは,周期が1であることをいう.
    \end{enumerate}
\end{definition}

\begin{lemma}[エルゴード性の特徴付け]
    Markov連鎖$((X_n),I,\bP)$が$\abs{I}<\infty$を満たすとき,次の3条件は同値.
    \begin{enumerate}
        \item $\bP$は原始的である.
        \item $\bP$は既約で非周期的である.すなわち,すべての状態$i\in I$について,$\exists_{n_2\in\N}\;\forall_{n\ge n_2}\;p_{ii}^{(n)}>0$.
        \item $\bP$は既約で,ある状態$i\in I$について,$\exists_{n_2\in\N}\;\forall_{n\ge n_2}\;p_{ii}^{(n)}>0$.
    \end{enumerate}
\end{lemma}

\begin{example}\mbox{}
    \begin{enumerate}
        \item 円周の$N$等分点上のランダムウォークは,$N$が奇数ならばエルゴード的であるが,偶数ならば既約であっても非周期的にはならない.
        \item $p_{ii}=1$を満たす$i\in I$をtrapという.これがあるMarkov過程はエルゴード的でない.
    \end{enumerate}
\end{example}

\begin{theorem}[有限状態Markov過程のエルゴード定理]
    Markov連鎖$((X_n),I,\bP)$が$\abs{I}<\infty$を満たし,$\bP$はエルゴード的であるとする.
    このとき,(1)を満たす$I$上の確率分布$\pi$が一意的に存在する.この$\pi$は(2),(3)も満たす.
    \begin{enumerate}
        \item 定常性:$\pi\bP=\pi$.
        \item 極限分布:$\forall_{i,j\in I}\;\lim_{n\to\infty}p_{ij}^{(n)}=\pi_j$.
        \item 混合性:(2)の収束は指数関数的である:$\exists_{C>0}\;\exists_{0<\lambda<1}\;\forall_{n\in\N}\;\forall_{i,j\in I}\;\abs{p_{ij}^{(n)}-\pi_j}\le C\lambda^n$.
    \end{enumerate}
\end{theorem}

\begin{definition}[invariant measure]
    定理の性質を満たす分布$\pi\in P(\Sigma)$を\textbf{不変分布}または\textbf{定常分布}という:
    \[\sum_{x\in\Sigma}\pi(x)p(x,y)=\pi(y).\]
\end{definition}

\subsection{可算Markov連鎖の再帰性}

\begin{definition}[recurrent, transient]
    $\{X_n\}\subset L(\Om;\Sigma)$を可算Markov連鎖,$\tau_x^+:=\min\Brace{n\in\N^+\mid X_n=x}$を到達時刻とする.
    \begin{enumerate}
        \item 状態$x\in\Sigma$が\textbf{再帰的}とは,$P_x[\tau_x^+<\infty]=1$を満たすことをいう.
        \item 状態$x\in\Sigma$が\textbf{非再帰的}とは,$P_x[\tau_x^+<\infty]<1$を満たすことをいう.
        \item 状態$x\in\Sigma$が\textbf{正再帰的}とは,$E_x[\tau_x^+]<\infty$を満たすことをいう.
        \item 状態$x\in\Sigma$が\textbf{零再帰的}とは,$E_x[\tau_x^+]=\infty$を満たすことをいう.
    \end{enumerate}
\end{definition}

\begin{proposition}
    既約なMarkov連鎖について,次は同値:
    \begin{enumerate}
        \item ある状態$x\in\Sigma$は再帰的である.
        \item 任意の状態$x\in\Sigma$は再帰的である.
    \end{enumerate}
    「再帰的」を「非再帰的」「正再帰的」「零再帰的」に変えても成り立つ.
\end{proposition}

\begin{proposition}\mbox{}
    既約なMarkov連鎖について,
    \begin{enumerate}
        \item $x\in\Sigma$について再帰的であるとする.このとき,初期状態に依らずに,殆ど確実に$x$を無限回訪れる.
        \item $x\in\Sigma$について非再帰的とする.このとき,初期状態に依らずに,殆ど確実に$x$は有限回しか訪れない.
    \end{enumerate}
\end{proposition}

\begin{proposition}[再帰性・非再帰性の十分条件]
    既約な可算Markov連鎖について,
    \begin{enumerate}
        \item ある$x\in\Sigma$と非空集合$A\subset\Sigma$について,$P_x[\tau_A<\infty]<1$を満たすならば,$(X_n)$は非再帰的である.
        \item ある非空な有限集合$A\subset\Sigma$と任意の$x\in\Sigma\setminus A$について$P_x[\tau_A<\infty]=1$を満たすならば,$(X_n)$は再帰的である.
    \end{enumerate}
\end{proposition}

\subsection{Markov過程に対する大数の弱法則}

\begin{tcolorbox}[colframe=ForestGreen, colback=ForestGreen!10!white,breakable,colbacktitle=ForestGreen!40!white,coltitle=black,fonttitle=\bfseries\sffamily,
title=]
    Markov連鎖がエルゴード的ならば,独立性の代わりになり,大数の法則が成り立つ.
\end{tcolorbox}

\begin{theorem}[大数の弱法則]\label{thm-law-of-large-number-of-Markov-chain}
    Markov連鎖$((X_n),I,\bP)$が$\abs{I}<\infty$を満たし,$\bP$はエルゴード的であるとする.$\pi$を不変分布とすると,
    関数$f:I\to\R$について,
    \[\frac{1}{n}\sum_{k=1}^nf(X_k)\xrightarrow{p}E^\pi[f].\]
\end{theorem}

\begin{definition}[number of visit]
    $i\in I$に関して,$f(j):=1_{\Brace{j=i}}$と定めると,$\sum^n_{k=1}f(X_k)$とは時刻$n$までの$i$への訪問回数$\tau^{(n)}_i$を表す.
    滞在時間ともいう.
\end{definition}

\begin{corollary}
    $\frac{\tau_i^{(n)}}{n}\xrightarrow{p}\pi_i$.
\end{corollary}

\begin{definition}[stationarity]\mbox{}
    \begin{enumerate}
        \item Markov連鎖$(X_n)_{n\in\N}$が\textbf{定常的}であるとは,$(X_n)_{n\in\N}$と$(X_{n+1})_{n\in\N}$との分布が等しいことをいう.
        \item 初期分布を$\pi$とするエルゴード的なMarkov連鎖を\textbf{定常Markov連鎖}という.
    \end{enumerate}
\end{definition}

\begin{theorem}[高次元化]
    Markov連鎖$((X_n),I,\bP)$が$\abs{I}<\infty$を満たし,$\bP$はエルゴード的であるとする.$\pi$を不変分布とすると,
    関数$f:I^l\to\R\;(l\ge1)$について,
    \[\frac{1}{n}\sum_{k=1}^nf(X_k,\cdots,X_{k+l-1})\xrightarrow{p}E^\pi[f]\]
    ただし,$E^\pi$は定常Markov連鎖$(\o{X}_n)_{n\in[l]}$に関する期待値である.
\end{theorem}

\subsection{Markov連鎖の極限定理2:Doobの定理}

\begin{definition}
    時間的に一様なMarkov過程$(X_n)$のMarkov核$P(x;A)$について,
    確率分布$\Pi\in P(\R)$が\textbf{不変確率分布}であるとは,$X_0\sim\Pi\Rightarrow X_1\sim\Pi$を満たすことをいう.
\end{definition}

\begin{theorem}[Doob]
    次の2条件を仮定する.
    \begin{enumerate}[({A}1)]
        \item $\forall_{x,y\in\R}\;P(x;-),P(y;-)$は互いに特異でない.
        \item 不変確率分布$\Pi$が存在する.
    \end{enumerate}
    このとき,次が成り立つ:
    \begin{enumerate}
        \item 任意の$A$に関して,$A$上一様に$P^m(x,A)\to\Pi(A)\;(\Pi\das\;x\in\R)$.すなわち,全変動ノルムに関して収束する.
        \item 次の収束が,積分$I:=\Pi f$が存在する限り成り立つ:
        \[\frac{1}{M}\sum_{n\in M}f(X_m)\xrightarrow{M\to\infty} I\]
    \end{enumerate}
\end{theorem}

\begin{remarks}[MCMC]
    $\Pi$-不変なMarkov連鎖を疑似乱数で生成し,積分$I=\Pi f$をDoobの定理の収束によって近似する手法を\textbf{Markov連鎖Monte Carlo法}という.
\end{remarks}

\section{可算Markov連鎖の例:酔歩}

\begin{tcolorbox}[colframe=ForestGreen, colback=ForestGreen!10!white,breakable,colbacktitle=ForestGreen!40!white,coltitle=black,fonttitle=\bfseries\sffamily,
title=]
    極めて直感的かつ,最先端の道具の動機が詰まった例である\cite{Popov21-RandomWalk}.
    この例について,原点への到達確率の解析を試みる.
\end{tcolorbox}

\subsection{正方格子上の酔歩の定義}

\begin{definition}[SRW: simple random walk]
    Markov過程$\{X_n\}\subset L(\Om;\Z^d)$について,
    \begin{enumerate}
        \item 初期分布$\nu\in P(\Z^d)$と時空間に一様な遷移確率$(p_z)_{z\in\Z^d}$によって定まるとき,これを\textbf{酔歩}という.
        これを,$Z_1,Z_2,\cdots\sim p_z\;\iid$として,$X_n=\sum_{i\in[n]}Z_i$と表す.
        \item さらに遷移確率が$p_z=\frac{1}{2d}1_{\abs{z}=1}$を満たすとき,\textbf{単純酔歩}という.
    \end{enumerate}
\end{definition}
\begin{remarks}[単純酔歩という対象]
    パリから出発した1歩1mの2次元の酔歩は,銀河系を脱出するまでに30回ほど戻ってくる.
    それ故,シミュレーションによる酔歩の性質の解析は困難な部分も多い.
    $d=2,3$で大きく振る舞いが違う数学的な背景の一つに,無限級数
    \[\sum_{n\in\N^+}\frac{1}{n^{d/2}}\]
    の収束性が関与する.
\end{remarks}

\subsection{再帰性とその十分条件}

\begin{tcolorbox}[colframe=ForestGreen, colback=ForestGreen!10!white,breakable,colbacktitle=ForestGreen!40!white,coltitle=black,fonttitle=\bfseries\sffamily,
title=]
    酔歩$X_n:=\sum_{i\in[n]}Z_i$の
    平均的な一歩$E[Z_1]$が零ベクトルでない場合,
    酔歩は非再帰的である.
    また,再帰的であることと各段階で$0$を踏む確率の和が発散することは同値である.
\end{tcolorbox}

\begin{notation}
    $\nu=\delta_0$を初期分布として持つ酔歩を考える.
    \begin{enumerate}
        \item 時刻$n\in\N$で初めて原点に戻るという事象を$A_n$,その確率を$q_n$とする:
        \[A_n:=\Brace{X_n=0,\forall_{1\le k\le n-1}\;X_k\ne0},\qquad q_n:=P[A_n].\]
        \item ある時刻$n\in\N$が存在して原点に戻る確率を$0\le q\le 1$とする:
        \[q:=\sum_{n\in\N^+}q_n=\sum_{n\in\N^+}P[A_n]=P[\Brace{\exists_{n\in\N}\;X_n=0}].\]
        \item 一般に,時刻$n\in\N$で原点に居る確率を
        \[p_n:=P[X_n=0]\]
        で表す.
    \end{enumerate}
\end{notation}

\begin{definition}[recurrent, transient]
    酔歩が
    \begin{enumerate}
        \item $q=1$を満たすとき\textbf{再帰的}であるという.
        \item $q<1$のとき\textbf{非再帰的}であるという.
    \end{enumerate}
\end{definition}

\begin{theorem}[再帰性の特徴付け]
    初期分布$\nu=\delta_0$を持つ酔歩について,次の2条件は同値.
    \begin{enumerate}
        \item 再帰的である.
        \item 各時点において原点に戻る確率の和が発散する:$\sum^\infty_{n=1}P[X_n=0]=\infty$.
    \end{enumerate}
\end{theorem}
\begin{Proof}\mbox{}
    \begin{description}
        \item[準備1:母関数の定義]  母関数
        \[q(\xi):=\sum_{n=1}^\infty q_n\xi^n,\qquad p(\xi):=\sum_{n=0}^\infty p_n\xi^n,\qquad(\abs{\xi}<1)\]
        を考える.
        係数列$(q_n),(p_n):\N^+\to\R_+$は非負で,$\xi\nearrow1$について単調増加であるから,単調収束定理より,$\xi\nearrow1$の極限を持ち,
        \[q=\sum_{n=1}^\infty q_n,\qquad q=\sum_{n=0}^\infty q_n.\]
        が成り立つ.
        \item[準備2:母関数の間の関係] 時刻$n$で原点に居る事象は,時点$n-k$で原点に居る確率と時点$k$で初めて原点に至る確率との積の$k\in\N$に関する畳み込みで表せる:
        \[p_n=\sum_{k=1}^nq_kp_{n-k}\qquad(n\in\N^+).\]
        これより,母関数の関係
        \begin{align*}
            p(\xi)=\sum_{n=0}^\infty p_n\xi^n&=1+\sum_{n=1}^\infty\paren{\sum_{k=1}^nq_kp_{n-k}}\xi^n\\
            &=1+p(\xi)q(\xi).
        \end{align*}
        を得て,いずれも連続関数であるから,特に$\xi=1$でも一致する:
        \[q=1-\frac{1}{p}.\]
    \end{description}
    \begin{description}
        \item[(1)$\Rightarrow$(2)] $p<\infty$ならば$q<1$より非再帰的である.
        \item[(2)$\Rightarrow$(1)] $p=\infty$ならば$q=1$より再帰的である.
    \end{description}
\end{Proof}
\begin{remarks}[母関数について]
    二項展開・二項係数や多項係数と同様に,関数の乗算に現れる組合せ論的な性質に仮託する技術である.

\end{remarks}

\begin{corollary}[非再帰性の十分条件]
    次の2条件が成り立つとき,酔歩$(X_n)$は非再帰的である:
    \begin{enumerate}[{[A}1{]}]
        \item 歩幅は有限:$R:=\max\Brace{\abs{z}\in\R\mid z\in\Z^d,p_{0z}>0}<\infty$.
        \item 平均的な一歩が偏っている:$m:=\sum_{z\in\Z^d}p_{0z}z=E[Z_1]\ne0\in\R^d$.
    \end{enumerate}
\end{corollary}
\begin{Proof}\mbox{}
    \begin{enumerate}[{Step}1]
        \item $E[X_n]=nE[Z_1]$より,
        Chebyshevの不等式から,
        \[P[X_n=0]\le P\Square{\abs{X_n-nE[Z_1]}\ge\frac{n\abs{E[Z_1]}}{2}}\le\frac{16}{n^4\abs{E[Z_1]}^4}E[\abs{X_n-nE[Z_1]}^4].\]
        \item 右辺を
        $Z_i$で表すと,
        \begin{align*}
            \abs{X_n-nE[Z_1]}^4&=\paren{\sum_{k_1,k_2=1}^n(Z_{k_1}-E[Z_1]|Z_{k_2}-E[Z_1])}^2\\
            &=\sum_{k_1,k_2,l_1,l_2}^n(Z_{k_1}-E[Z_1]|Z_{k_2}-E[Z_1])(Z_{l_1}-E[Z_1]|Z_{l_2}-E[Z_1]).
        \end{align*}
        \item この右辺の評価を通じて,
        \[\sum_{n=1}^\infty P[X_n=0]\le\frac{16 C}{\abs{E[Z_1]}^4}\sum_{n=1}^\infty\frac{1}{n^2}<\infty.\]
    \end{enumerate}
\end{Proof}

\subsection{単純ランダムウォークの再帰性と非再帰性}

\begin{tcolorbox}[colframe=ForestGreen, colback=ForestGreen!10!white,breakable,colbacktitle=ForestGreen!40!white,coltitle=black,fonttitle=\bfseries\sffamily,
title=]
    単純酔歩において,
    \begin{enumerate}
        \item $d=1,2$のとき,$p_{2n}\sim\frac{1}{(\pi n)^{d/2}}$.
        \item $d=3$のとき,$p_{2n}\le\frac{1}{2}\paren{\frac{3}{\pi n}}^{3/2}$.
    \end{enumerate}
    の評価を得ることが出来る.
\end{tcolorbox}

\begin{theorem}[Polya 1921]
    $d$次元の単純酔歩は,$d=1,2$のとき再帰的であり,$d\ge3$のとき再帰的でない.
\end{theorem}
\begin{history}
    Polya自身は母関数の方法によって証明した.
\end{history}

\chapter{拡散過程}

\begin{quotation}
    $C$-過程でもある強Markov過程を拡散過程という.
    確率微分方程式は,真に確率論的な方法で拡散過程を構成することを可能にする.

    元々Markov過程論から純粋に数学的に生じた問題意識の解決のために確率微分方程式が
    開発された.
    しかし,確率論的な現象は自然界にありふれている.
    常微分方程式の定める流れに沿って輸送された物理量は,移流方程式と呼ばれる1階の偏微分方程式を満たす.
    Brown運動に沿って輸送された物理量(熱など)は,熱伝導方程式・拡散方程式と呼ばれる2階の偏微分方程式を満たす.
    この対応関係は確率微分方程式を導入することでさらに一般化され,2階の放物型・楕円形の偏微分方程式の解を
    確率的に表示することが出来るようになる.
    こうして,確率微分方程式は,ポテンシャル論・偏微分方程式論や微分幾何学との架け橋になる.

    Brown運動は,空間的に一様な確率場での積分曲線だと思えば,さらに一般に空間的な一様性の仮定を取った場合が確率微分方程式であり,
    これはBrown運動の変形として得られるというのが伊藤清のアイデアである.
\end{quotation}

\section{確率微分方程式の概観}

\subsection{最適輸送理論}

\begin{definition}
    
\end{definition}

\subsection{移流方程式}

\begin{tcolorbox}[colframe=ForestGreen, colback=ForestGreen!10!white,breakable,colbacktitle=ForestGreen!40!white,coltitle=black,fonttitle=\bfseries\sffamily,
title=]
    Brown運動は,位置を忘れて粒子の視点から見た,確率ベクトル場から受ける「流れ」だと理解できる.
    したがってその背景には確率ベクトル場がある.
\end{tcolorbox}

\begin{definition}[advection, 時間的に一様なベクトル場による輸送方程式]\mbox{}
    \begin{enumerate}
        \item 物理量のスカラー場や物質がベクトル場によって輸送されること・経時変化することを,\textbf{移流}という.
        \item 定ベクトル$b=(b^1,b^2)\in\R^2$が定める空間一様な移流に関する初期値問題$u=u(t,x)$は,
        \begin{align*}
            \pp{u}{t}+b\cdot\nabla u=0,\quad(t>0,x=(x^1,x^2)\in\R^2),\\
            u(0,x)=f(x)\in C^1(\R^2)
        \end{align*}
        と表される.このとき,$X_t(x):=x-bt$は,時刻$t$に$x$に居る粒子が,時刻$0$にどこにいたかを表す.よって,初期状態として与えられた物理量$f:\R^2\to\R$を用いて,時刻$t$に位置$x\in\R^2$で観測される物理量は$f(X_t(x))$で表される.
        これは大数の法則に物理的な意味論を与える\ref{thm-law-of-large-number-of-Markov-chain}.
    \end{enumerate}
\end{definition}

\begin{discussion}[変数係数の移流問題]
    ベクトル場$b=b(t,x):\R\times\R^2\to\R^2$に対して,方程式
    \[\pp{u}{t}+b(t,x)\cdot\nabla u=0\quad(t>0,x\in\R^2)\]
    を考える.$y\in\R^2$から出発した粒子の位置を表す変数$Y_t\in\R^2$を固定して,そこでの時間変化を表す
    常微分方程式に関する初期値問題$\dot{Y}_t=b(t,Y_t)\;(t\in\R_+);Y_0=y\in\R^2$を考える.
    すると,解$u(t,x)$は次を満たす必要がある:
    \begin{align*}
        \pp{}{u}\paren{u(t,Y_t(y))}&=\pp{u}{t}(t,Y_t(y))+\dot{Y}_t(y)\cdot\nabla u(t,Y_t(y))\\
        &=\paren{\pp{u}{t}+b\cdot\nabla u}(t,Y_t(y))=0.
    \end{align*}
    よって,$u(t,Y_t(y))=u(0,Y_0(y))=f(y)$なる第一積分が見つかったことになる.

    こうして元の偏微分方程式は,$y\in\R^2$に居た粒子が受けることになる流れ$Y_t$に沿った輸送を記述する方程式であったと理解できる.
    または,ベクトル場による移流方程式とは,ある移流$Y_t$の第一積分が満たすべき方程式とも見れる.
    \footnote{これがKolmogorovが発見したものだった?}
\end{discussion}
\begin{remark}[2つの関連]
    いま,$Y_t:\R^2\times\R_+\to\R^2$は,各$y\in\R^2$に対して,これが受けることになる力の時系列$Y_t(y)$を与えている.
    これが可逆であるとする:$X_t:=Y_t^{-1}:\R^2\to\R^2$.すると,出発点$y$が$t$時刻にいる位置$x$について,$y=X_t(x)$と辿る確率過程になる.
    これは解について$u(t,x)=f(X_t(x))$なる表示を与える.

    ベクトル場が時間に依存しないとき,$Y_t$は可逆であり,逆$X_t$は常微分方程式$\dot{X}_t=-b(X_t),X_0=x$を満たす.
    定数係数の場合は,これを解いたものと見れるから,たしかに一般化となっている.
\end{remark}

\subsection{Brown運動が定める確率ベクトル場}

\begin{tcolorbox}[colframe=ForestGreen, colback=ForestGreen!10!white,breakable,colbacktitle=ForestGreen!40!white,coltitle=black,fonttitle=\bfseries\sffamily,
title=]
    移流方程式は,時間変化するベクトル場$\R_+\to\Map(\R^2,\R^2)$による物理量の輸送を考えた.
    もし,ベクトル場が確率的であったら?前節の例はひとつの見本道に過ぎないとしたら?
    すなわち,前節ではスタート地点$y\in\R^2$を根源事象とみて,時間変化するベクトル場を確率過程と見たが,ここに新たにランダム性の発生源を加えるのである.
\end{tcolorbox}

\subsection{確率微分方程式}

\begin{example}[空間的に一様な場合]
    $b\in\R^2,\al=(\al^i_k)_{i,k\in[2]}$が定める確率過程$X_t(x):=x-\al B_t-bt\;(t\in\R_+,x\in\R^2)$を考える.
    $X_t$は,定数係数ベクトル場による輸送と,Brown運動とを合成した運動に他ならない.
    実際,仮に$B_t$が$t$について微分可能であるならば,$\dot{X}_t=-\al\dot{B}_t-b$となるから,時間発展するベクトル場$-\al\dot{B}_t-b$による移流であると考えられる.
    $\al$がBrown運動の強さと歪みの情報を含んでいる.
    ただし,Brown運動の時刻$t$における$y\in\R^2$に関する分布は
    \[P[B_t\in dy]=p(t,y)dy:=\frac{1}{2\pi t}e^{-\abs{y}^2/2t}dy\]
    と表せる.

    すると,$X_t$による輸送の平均値を
    \[u(t,x):=E[f(X_t(x))]\]
    とおけば,これは
    \[q(t,x,z):=\frac{1}{2\pi t\abs{\det\al}}\exp\paren{-\frac{\abs{\al^{-1}(z-x+bt)}^2}{2t}}\]
    とおくことで
    \begin{align*}
        u(t,x)&=\int_{\R^2}f(x-al-bt)p(t,y)\\
        &=\int_{\R^2}f(z)q(t,x,z)dz.
    \end{align*}
    と整理出来る.これは熱伝導方程式と呼ばれる放物型方程式の一般解であり,$q(t,x,z)$が一般解と呼ばれるものである:
    \[\pp{u}{t}=\frac{1}{2}\sum^2_{i,j=1}a^{ij}\pp{^2u}{x^i\partial x^j}-b\cdot\nabla u,\quad(t>0,x\in\R^2).\]
\end{example}

\begin{example}[変数係数の場合]
    $\al,b$が共に$x\in\R^2$に依存する場合,任意の$t>0$について,常微分方程式
    \[\dot{X}_t=\al(X_t)\dot{B}_t+b(X_t)\quad(X_0=x\in\R^2)\]
    を考えたい.しかし$B_t$は微分可能でないから,積分形で代わりに表現することを考える.
    すなわち,$\dot{B}_sds=\dd{B_s}{s}ds=dB_s$とし,この右辺を定義することを考える.
    \[X_t=x+\int_0t\al(X_s)dB_s+\int^t_0b(X_s)ds.\]
    すると次の問題は,右辺第2項がStieltjesの意味での積分として定義することは出来ない.
    $B_t$は有界変動でないので,線積分の発想はお門違いである.
    これは確率論的な考察で乗り越えることが出来る.

    この確率微分方程式の解$X_t$を求めれば,平均値$u(t,x)$は放物型偏微分方程式
    \[\pp{u}{t}=\frac{1}{2}\sum^2_{i,j=1}a^{ij}(x)\pp{^2u}{x^i\partial x^j}+b(x)\cdot\nabla u\quad(t>0,x\in\R^2,a(x):=\al(x){}^t\!\al(x))\]
    を満たすことになる.
    これはいわば,確率過程$X_t$から得られる平均処置効果のような統計量の1つが,偏微分方程式で記述される物理法則を満たすということに過ぎない.
    $X_t$は非常に豊かな情報を湛えていて,偏微分方程式はその1断面に過ぎないと言えるだろう.
\end{example}

\subsection{無限次元確率微分方程式}

\begin{tcolorbox}[colframe=ForestGreen, colback=ForestGreen!10!white,breakable,colbacktitle=ForestGreen!40!white,coltitle=black,fonttitle=\bfseries\sffamily,
title=無限次元確率微分方程式]
    確率微分方程式はランダムなゆらぎを持つ常微分方程式であるから,同様にランダムなゆらぎを持つ偏微分方程式に当たる概念も自然に現れるはずである.
\end{tcolorbox}

\section{1次元拡散過程}

\begin{tcolorbox}[colframe=ForestGreen, colback=ForestGreen!10!white,breakable,colbacktitle=ForestGreen!40!white,coltitle=black,fonttitle=\bfseries\sffamily,
title=]
    与えられた領域$U$をほとんど確実に出ていく強Markov過程を拡散過程という.
\end{tcolorbox}

\begin{notation}
    $W:=C(\R_+)$上に確率測度の族$(P_x)_{x\in\R}$を考える.
    射影を$X_t:=\pr_t:W\to\R;\om\mapsto\om(t)\;(t\in\R_+)$と表すと,この見本過程$\R_+\to\Map(W,\R)$は任意の$x\in\R$について$(W,P_x)$上連続.
    $\B_t:=\sigma[X_s;s\le t]$とする.
\end{notation}

\begin{definition}[diffusion process]
    確率過程$(X_t)_{t\in\R_+}$が$P_x[X_0=x]=1$と次の条件を満たすとき,$\M:=(\M_x)_{x\in\R},\M_x:=\Brace{X_t(\om)\in\R\mid t\in\R_+,\om\in(W,P_x)}$を,$x$から始まる\textbf{一様な連続強Markov過程}または\textbf{拡散過程}という.
    \begin{quote}
        (強Markov性) $x\in\R$と有限な$(B_t)$-Markov時刻$\tau:W\to\R_+$,$E\in\B^1(\R^1)$について,
        $P_x[X_{\tau+t}\in E\mid\B_\tau]=P_x[X_t\in E]|_{x=X_\tau}$.
    \end{quote}
\end{definition}

\begin{definition}[regular point, regular diffusion process]
    拡散過程$(\M_x)_{x\in\R}$について,
    \begin{enumerate}
        \item $x$は$\M$の\textbf{正則点}であるとは,$P_x[\exists_{t\in\R_+}\;X_t>x]>0,P_x[\exists_{t\in\R_+}\;X_t<x]>0$が成り立つことをいう.
        \item 任意の$x\in\R$が正則点であるとき,$\M$は\textbf{正則}であるという.
    \end{enumerate}
\end{definition}

\chapter{定常過程と時系列解析}



\section{定常過程}

\begin{tcolorbox}[colframe=ForestGreen, colback=ForestGreen!10!white,breakable,colbacktitle=ForestGreen!40!white,coltitle=black,fonttitle=\bfseries\sffamily,
title=]
    定常過程については,スペクトル分解とエルゴード定理が証明できる.
    弱定常過程は決定的な成分と純非決定的な成分とに分解出来る(Wold分解).
\end{tcolorbox}

\begin{definition}[weak / strong stationary stochastic process]
    確率過程$(X_t)_{t\in\R}$について,
    \begin{enumerate}
        \item $(X_t)$は$L^2(\Om)$-過程で,
        $\forall_{t,s,h\in\R}\;m(t+h)=m(t),\Gamma(t+h,s+h)=\Gamma(t,s)$が成り立つとき,すなわち$m$が定数で$\Gamma(s,t)$は$\abs{t-s}$の関数であるとき,\textbf{弱定常過程}という.
        $\R\to L^2(\Om)$の連続性を仮定することもある.
        \item 自身の任意の平行移動$(X_{t+h})_{t\in\R}\;(h\in\R)$に分布同等であるとき,\textbf{強定常過程}という.
        \item $\M_t(X)$を$\Brace{X_s}_{s\in[0,t]}\subset L^2(\Om)$の閉部分空間とする.これが$t\in\R$に依らないとき,\textbf{決定的}であるといい,$\bigcap_{t\in\R}M_t(X)=\R$のとき純非決定的であるという.
    \end{enumerate}
\end{definition}

\begin{lemma}
    過程$(X_t)_{t\in\R}$について,
    \begin{enumerate}
        \item 強定常かつ$X_0\in L^2(\Om)$のとき,弱定常である.
        \item $(X_t)_{t\in\R}$がGaussであるとき,弱定常性と強定常性とは同値.
    \end{enumerate}
\end{lemma}

\section{信号処理の用語}

\begin{tcolorbox}[colframe=ForestGreen, colback=ForestGreen!10!white,breakable,colbacktitle=ForestGreen!40!white,coltitle=black,fonttitle=\bfseries\sffamily,
title=]
The problem of optimal non-linear filtering (even for the non-stationary case) was solved by Ruslan L. Stratonovich (1959,[1] 1960[2]).\footnote{\url{https://en.wikipedia.org/wiki/Filtering_problem_(stochastic_processes)}}
\end{tcolorbox}

\begin{definition}[filtering problem]
    
\end{definition}

\begin{definition}[innovation]
    時系列$(X_t)_{t\in T}$について,$X_t$の観測値と,$\F_s\;(s<t)$による予測の値との差が定める過程を\textbf{新生過程}という.
    この過程が白色雑音になることは,予測可能な成分をすべて除去しきったとみなせるため,予測として理想的であると考えられる.
\end{definition}

\chapter{ミキシング過程}

\begin{quotation}
    非可逆な熱力学的現象のモデルとして,物理学から最初にモデル化された.
\end{quotation}

\chapter{超過程}

\begin{quotation}
    時間パラメータ$\R_+$を高次元化して,確率場を得た.
    さらに,これを無限次元空間,特に試験関数の空間とすると,これは見本道を超関数とする確率過程と見れる.
    そこでこれを\textbf{確率超過程}といい,超関数の空間上に測度を押し出す!
\end{quotation}

\section{定義と構成}

\begin{definition}
    写像$X:\Om\times\D(T)\to\R$について,
    \begin{enumerate}
        \item 次の2条件を満たす$X$を\textbf{狭義の超過程}という:
        \begin{enumerate}
            \item $\forall_{\phi\in\D(T)}\;X(-,\phi)\in\L(\Om)$.
            \item $X(\om,-)\in\D'\;\as\;\om\in\Om$
        \end{enumerate}
        \item 次の2条件を満たす$X$を\textbf{広義の超過程}という:
        \begin{enumerate}
            \item $\forall_{\phi\in\D(T)}\;X(-,\phi)\in L^2(\Om)$.
            \item 対応$\phi\mapsto X(-,\phi)$が殆ど確実に超関数$\D\to L^2(\Om)$を定める.
        \end{enumerate}
    \end{enumerate}
\end{definition}

\begin{definition}
    超過程$X$の\textbf{特性汎関数}とは,次で定まる$C_X:E\to\R$をいう:
    \[C_X(\xi):=\int e^{iX_\xi(\om)}dP(\om),\quad\xi\in E.\]
\end{definition}

\subsection{Bochner-Minlosの定理}

\begin{notation}
    $E$を加算なHilbert空間で,核型であり,$T\subset\R_+$について$E\subset L^2(T)=:H\subset E^*$を満たすとする.
\end{notation}

\begin{theorem}
    汎関数$C:E\to\R$は次の3条件を満たし,ある$n>p$について埋め込み$T^n_p:E_n\to E_p$がHilbert-Schmidt作用素になるとき,$m$の拡張である$(E^*,\B)$上の測度$\mu$が一意に存在して,その台は$E_n^*$に含まれる.
    \begin{enumerate}
        \item ある$p$について,$L^p(\Om)$-連続である.
        \item 正定値.
        \item $C(0)=1$.
    \end{enumerate}
\end{theorem}

\begin{corollary}
    $E$上に3条件を満たす汎関数$C:E\to\R$が与えられたとき,$(E^*,\B)$上に
    \[C(\xi)=\int_{E^*}e^{i\brac{x,\xi}}d\mu(x)\]
    を満たす確率測度$\mu$が唯一つ定まる.
\end{corollary}

\begin{definition}
    $C$が定める超過程$X:\Om\to E^*$に対して,系が定める測度$\mu\in P(E^*)$を\textbf{分布}という.
\end{definition}

\subsection{例}

\begin{example}
    $C_{\sigma^2}(\xi)=e^{-\frac{\sigma^2}{2}\norm{\xi}^2}\;(\xi\in\S)$は特性汎関数である.よって$\S'$上に確率測度$\mu_\sigma$を定めるが,これを\textbf{分散$\sigma^2$のホワイトノイズ}という.
    \begin{enumerate}
        \item $L^2$-連続な弱定常過程$X_t$について,$dM$をランダム測度として
        \[X_t=\int e^{it\lambda}dM(\lambda)\]
        とスペクトル分解され,これが定める定常超過程のスペクトル分解は
        \[\brac{x,\xi}=\int\wh{\xi}(\lambda)dM(\lambda)\]
        となる.この定常超過程がホワイトノイズであるとき,$E[\abs{dM(\lambda)}^2]=d\lambda$であり,
        したがってあらゆるスペクトルが一様に出てくる.この性質を\textbf{白色}といい,通信工学での雑音のモデルとなる.
        \item 特に,白色雑音はGauss型とは限らず,例えばPoisson型などもある.
    \end{enumerate}
\end{example}

\subsection{定常超過程}

\begin{definition}
    $(S_t\xi)(u):=\xi(u-t)$を$E$上の平行移動作用素とする.
    任意の$t$について$S_t$は$E$の自己同型であり,$C(S_t\xi)=C(\xi)$を満たすとき,$C$の定める超過程は\textbf{定常超過程}であるという.
\end{definition}

\begin{proposition}
    ホワイトノイズは定常超過程である.
\end{proposition}

\section{ノイズ}

一般に,物理的な干渉過程のモデルをノイズ過程という.
そして工学系はノイズ過程の合成にさらされており,ここから真の予測可能成分を分離することが普遍的な目標となる(エアバッグの作動など).
また通信理論は,ノイズに対して頑健な符号化法の開発が重要な問題となる.
\begin{quote}
    The
    better the probabilistic model of a noise process, the better the chance to
    avoid unpredictable consequences of noise. Therefore, it is indispensable
    that mathematicians provide effective noise models. On the other hand, it is
    indispensable that engineers are familiar with the mathematical background
    of noise modelling in order to handle noise models in an optimal way.\cite{Schaffler}
\end{quote}

\section{点過程}

\begin{tcolorbox}[colframe=ForestGreen, colback=ForestGreen!10!white,breakable,colbacktitle=ForestGreen!40!white,coltitle=black,fonttitle=\bfseries\sffamily,
title=]
    確率過程の概念を一般化し,
    空間にランダムに点を打ちたいとする.
    それも一点ではなく多粒子系を考えたい.
    そこで,
    測度をランダム化することを考える.
    任意の見本過程が可算な定義域を持つとき,点過程という.すなわち,$\bP(T,U)$-値過程をいう.
\end{tcolorbox}

\begin{history}
    一般の可測空間上の点過程(discrete chaos)を考察したのは\cite{Wiener-Wintner43-DiscreteChaos}で,
    点測度を用いた現代的な定式化を与えたのは\cite{Moyal62-PointProcess}である.
\end{history}

\subsection{点関数の定義と性質}

\begin{tcolorbox}[colframe=ForestGreen, colback=ForestGreen!10!white,breakable,colbacktitle=ForestGreen!40!white,coltitle=black,fonttitle=\bfseries\sffamily,
title=]
    粒子系を$\R^n$で考えるのではなく,$\R$に点を打って考えるというモデルの転換がある.
    そこで可算粒子系を考えて見ると,これは$\o{\N}$-値の完備$\sigma$-有限測度とも,有限の台を持つ関数ともみなせる.
\end{tcolorbox}

\begin{lemma}
    $(S,\B(S),\mu)$を完備可分距離空間上の完備な$\sigma$-有限測度空間とする.次が成り立つ:
    \begin{enumerate}
        \item $\forall_{\ep>0}\;\exists_{E'\subset E}\;0<\mu(E')<\ep$.
        \item $\forall_{\al\in[0,1]}\;\exists_{E'\subset E}\;\mu(E')=\al\mu(E)$.
        \item $\forall_{\ep>0}\;\exists_{\{E_i\}\subset\S}\;E=\sum_{k\in[n]}E_k,\mu(E_k)<\ep$.
    \end{enumerate}
\end{lemma}

\begin{definition}[configuration, proper, configuration space, point function]
    完備可分距離空間上の可測空間$(S,\B(S))$について,
    \begin{enumerate}
        \item この上の$\o{\N}$-値の完備$\sigma$-有限測度を\textbf{配置}という.\footnote{Radon測度に制限することもある.}
        \item 配置が$\forall_{x\in S}\;m(\{x\})\in\{0,1\}$を満たすとき,\textbf{固有}であるという.
        \item 配置の全体を$\M(S)$で表し,漠位相を入れたものを\textbf{配置空間}という.
        \item $p:S\to\o{\N}$が\textbf{点関数}であるとは,可算な台$D_p:=\{x\in S\mid p(x)>0\}\subset S$を持つことをいう.
    \end{enumerate}
\end{definition}

\begin{lemma}\mbox{}
    \begin{enumerate}
        \item 任意の配置$m$は,可算個の点(重複を許す)$\{s_i\}\subset S$を用いて$m=\sum_{i\in\N}\delta_{s_i}$と表せる.
        \item $S$上の配置と点関数とは1対1対応する.
    \end{enumerate}
\end{lemma}

\begin{notation}\mbox{}
    \begin{enumerate}
        \item $T_1,T_2,\cdots$を$[l,r]\;(-\infty<l<r\le\infty)$という形の区間とし,Borel $\sigma$-代数$\cT_1,\cT_2,\cdots$によって可測空間とみなす.
        \item $U_1,U_2,\cdots$を$\U_1,\U_2,\cdots$を$\sigma$-代数とする可測空間とし,\textbf{状態空間}または\textbf{相空間}という.
    \end{enumerate}
\end{notation}

\begin{definition}[point function, trivial, discrete, graph, restriction]\mbox{}
    \begin{enumerate}
        \item $p:T\to U$が\textbf{点関数}であるとは,可算な部分集合$D_p\subset T$上で定義された部分関数をいう.
        \item $D_p=\emptyset$のとき\textbf{自明}であるといい,$D_p$が集積点を持たないとき\textbf{離散}であるという.
        \item 点関数の\textbf{グラフ}とは$G(p):=\Brace{(t,p(t))\in T\times U\mid t\in D_p}$をいう.これは可算集合である.
        \item 集合$E\subset T\times U$内にある$G(p)$の点の数を$N(p;E):=\abs{G(p)\cap E}\in\o{\N}\;(E\subset T\times U)$と表す.これは$p$の\textbf{制限}$p|_E$の定義域の濃度に等しい.
        \item 点関数の全体を$\bP(T,U)$で表す.$\{p\in\bP\mid N(p,E)=k\}_{E\in\cT\times\U,k\in\o{\N}}$が生成する$\sigma$-代数によって可測空間とみなす.
    \end{enumerate}
\end{definition}

\subsection{点過程の定義と例と性質}

\begin{tcolorbox}[colframe=ForestGreen, colback=ForestGreen!10!white,breakable,colbacktitle=ForestGreen!40!white,coltitle=black,fonttitle=\bfseries\sffamily,
title=]
    測度値の過程を点過程というのである.したがって,点過程は測度空間上の確率測度となる.
    白色雑音も$(\S',\B(\S'))$上の確率測度である.
\end{tcolorbox}

\begin{definition}\mbox{}
    \begin{enumerate}
        \item 確率変数$X:\Om\to\M(S)$を,$S$上の\textbf{偶然配置}または\textbf{ランダム点測度}または\textbf{点過程}または\textbf{確率点場}という.
        \item $\mu(E):=E[X(E)]\;(E\in\S)$は$(S,\S)$上の測度を定め,これを点過程$X$の\textbf{平均(測度)}という.
    \end{enumerate}
\end{definition}

\begin{example}[多項偶然測度]
    点過程$X:\Om\to\M(S)$が\textbf{多項配置}であるとは,
    任意の分割$S=\sum_{r=1}^dE_r\;\{E_r\}\subset\S$に対して,結合分布$(X(E_1),\cdots,X(E_d)):\Om\to\o{\N}^d$が$d$次元の多項分布に従うことをいう.
\end{example}

\begin{lemma}\mbox{}
    \begin{enumerate}
        \item 平均$\mu$に対して,対応する多項配置が法則同等を除いて一意に定まる.
        \item 任意の$\sigma$-有限完備測度$\mu$に対して,これを平均に持つ多項配置が存在する.
    \end{enumerate}
\end{lemma}

\begin{definition}[point process / random point function, sample point function]
    可測関数$X:\Om\to \bP(T,U)$を\textbf{点過程}または\textbf{ランダム点関数}という.
    $X_\om$を\textbf{見本点関数}という.
\end{definition}

\begin{lemma}
    2つの点過程$X_1,X_2:T\to U$について,次の2条件は同値.
    \begin{enumerate}
        \item 法則同等である:$P^{X_1}=P^{X_2}$.
        \item $\forall_{n\in\N}\;\forall_{\{k_i\}\subset\N}\;\forall_{\{E_i\}\subset T\times U}\;P[\forall_{i\in[n]}\;N(X_1,E_i)=k_i]=P[\forall_{i\in[n]}\;N(X_2,E_i)=k_i]$.
    \end{enumerate}
\end{lemma}

\begin{definition}[discrete, differential, stationary]
    点過程$X:T\to U$について,
    \begin{enumerate}
        \item 離散であるとは,$X_\om$は殆ど確実に離散であることをいう.
        \item $\sigma$-離散であるとは,増大列$\{U_n\}\subset\U$が存在して,$X|_{U_n}$が離散で,$X=X|_{\cup_nU_n}\;\as$が成り立つことをいう.
        \item 微分であるとは,任意の互いに素な集合$\{T_i\}\subset\cT$について,$(X|_{T_i})_{i\in[n]}$が独立であることをいう.
        \item 定常であるとは,平行移動に関して確率分布が不変であることをいう.
    \end{enumerate}
\end{definition}

\subsection{Poisson点過程}

\begin{definition}[Poisson偶然測度]
    完備可分距離空間上の原子を持たない$\sigma$-有限測度空間$(S,\B(S),m)$について,\textbf{強度$m$のPoissonランダム測度}とは,
    体積確定集合に添字付けられた確率過程$\{M(A)\}_{A\in\M^1}\subset L^2(\Om)$であって,次を満たすものをいう:
    \begin{enumerate}
        \item $\forall_{A\in\M^1}\;M(A)\sim\Pois(m(A))$.
        \item $\forall_{A_1,\cdots,A_n\in\M^1}\;$が互いに素ならば,$M(A_1),\cdots,M(A_n)$は独立である.
    \end{enumerate}
\end{definition}

\begin{theorem}[Poisson測度の構成と表示]
    完備可分距離空間上の原子を持たない$\sigma$-有限測度空間$(S,\B(S),m)$について,
    \begin{enumerate}
        \item $m$を強度とするPoisson偶然測度が存在する.
        \item 次のように表せる:
        \[M(A)=\sum_{j\in[N]}\delta_{X_j}(A)\quad(A\in\M^1,X_j\in L(\Om;S),N\in L(\Om;[0,\infty]),\forall_{j\ge k}\;X_j\ne X_k\;\as).\]
    \end{enumerate}
\end{theorem}

\begin{example}\mbox{}
    \begin{enumerate}
        \item 強度$dt\otimes \mu$で与えられる$(\R_+\times S,\B(\R_+)\otimes\B(S))$上のPoisson点測度を,$U$上の\textbf{定常Poisson点過程}という.
        \item $S=\R^d$で$\mu$がLebesgue測度のとき,付随するPoisson点過程はさまざまな不変性を持ち,配置空間$\M(S)$上のLebesgue測度の役割を果たす.
    \end{enumerate}
\end{example}


\begin{lemma}[Poisson point process (点関数からみたPoisson点過程)]
    $X:\R_+\to U$がPoisson点過程とは,$\sigma$-離散的かつ微分的かつ定常的な点過程に同値.
\end{lemma}

\subsection{Gauss点過程}

\begin{tcolorbox}[colframe=ForestGreen, colback=ForestGreen!10!white,breakable,colbacktitle=ForestGreen!40!white,coltitle=black,fonttitle=\bfseries\sffamily,
title=]
    $S\subset\R^2$上のGauss点測度を白色雑音という.
    換言すれば,試験関数の空間$\D(\R)$上の測度の空間(したがって双対空間)上のGauss確率測度を白色雑音という.
    実はこれは,(定常増分過程たる)Brown運動の超関数微分(測度のようなもの)として得られる可分な定常過程とみなせる.
    ホワイトノイズとは花粉にとっての水中の微粒子である.
\end{tcolorbox}

\section{確率超過程}

\begin{tcolorbox}[colframe=ForestGreen, colback=ForestGreen!10!white,breakable,colbacktitle=ForestGreen!40!white,coltitle=black,fonttitle=\bfseries\sffamily,
title=]
    点過程の概念を一般化して,ランダム測度の過程を考える.
    これは逆に体積確定測度で添字付けられた確率場ともみなせる.
    この2つの見方を併せてノイズと呼ぶ.
\end{tcolorbox}

\subsection{確率超過程}

\begin{tcolorbox}[colframe=ForestGreen, colback=ForestGreen!10!white,breakable,colbacktitle=ForestGreen!40!white,coltitle=black,fonttitle=\bfseries\sffamily,
title=]
    試験関数の空間$\D$の双対空間$\D'$に値を取る過程を\textbf{確率超過程}という.
    これは,いくつかの条件を満たす,$\D$で添字付けられた実過程とも見れる.
\end{tcolorbox}

\begin{lemma}
    次の2条件を満たす過程$(X_\varphi)_{\varphi\in\D(\R)}$が定める$\Om\to\D'(\R)$は可測になる:
    \begin{enumerate}
        \item 線形性:$X_{a\varphi+b\psi}=aX_\varphi+bX_\psi\;\as$
        \item 連続性:$\D$での収束列$(\varphi_i)$は,法則収束する確率変数列$(X_{\varphi_i})$を定める.
    \end{enumerate}
\end{lemma}

\subsection{ノイズ}

\begin{tcolorbox}[colframe=ForestGreen, colback=ForestGreen!10!white,breakable,colbacktitle=ForestGreen!40!white,coltitle=black,fonttitle=\bfseries\sffamily,
title=]
    測度確定集合で添字付けられた
    中心化されたGauss過程であって,互いに相関を持たない性質の良いものとしてノイズを定義する.
    すると$\Om$上の測度値確率変数とも見れる.超関数は測度の一般化なのであった.
\end{tcolorbox}

\begin{definition}
    $(\Om,\S,P)$を確率空間,$(H,\H,\mu)$を測度空間,その測度確定な集合を$\H_\mu:=\{M\in\H\mid\mu(M)<\infty\}$で表す.
    \begin{enumerate}
        \item 体積確定集合で添字付けられた確率過程$\{X_M\}_{M\in\H_\mu}\subset\L(\Om;\R)$,または,測度値確率変数$\Om\to\M(H,\H)$が次の条件をみたすとき,\textbf{$\mu$-ノイズ}という:
        \begin{enumerate}[(a)]
            \item 中心化:$\forall_{M\in\H_\mu}\;E[X_M]=0$.
            \item 分散:$\forall_{M\in\H_\mu}\;E[(X_M)^2]=\mu(M)$.
            \item $\forall_{M_1,M_2\in\H_\mu}\;M_1\cap M_2=\emptyset\Rightarrow X_{M_1\sqcup M_2}=X_{M_1}+X_{M_2}\;\as$.
            \item $\forall_{M_1,M_2\in\H_\mu}\;M_1\cap M_2=\emptyset\Rightarrow E[X_{M_1}X_{M_2}]=0\;\as$.
        \end{enumerate}
    \end{enumerate}
\end{definition}

\subsection{ホワイトノイズ}

\begin{tcolorbox}[colframe=ForestGreen, colback=ForestGreen!10!white,breakable,colbacktitle=ForestGreen!40!white,coltitle=black,fonttitle=\bfseries\sffamily,
title=]
    一般の2階の確率超過程はホワイトノイズの線形変換とみなせる.
\end{tcolorbox}

\section{ホワイトノイズ解析}

\begin{quotation}
    \begin{quote}
        時系列として独立同分布の確率変数列で絵あり,定常時系列として持つスペクトルはフラットすなわち無色である.
        このような偶然量はノイズと呼ぶのにふさわしい.それ絵はノイズとして見ると厄介なものかもしれないが,最大の情報量をもつので通信の理論には大いに活躍の場がある.
        そして,一般のガウス過程の中でも元素的なもののとして特徴付けられる.それは「揺らぎ」の典型であり,基本となる.
    \end{quote}
    An innovative approach to random fields: Applications of white noise theory.
\end{quotation}

\chapter{参考文献}

\bibliography{../mathematics.bib,../statistics.bib}
\begin{thebibliography}{99}
    %%% 第一章から今後の指針にした書籍:
    \item
    David Nualart and Eulalia Nualart (2018). \textit{Introduction to Malliavin Calculus}. Cambridge University Press.
    \item
    伊藤清 (1991) 『確率論』(岩波基礎数学選書).
    \item
    Olav Kallenberg. (2021). \textit{Foundations of Modern Probability}. 3rd. Springer.
    \item
    Lipster, R. S. and Shiryayev, A. N. (1989). \textit{Theory of Martingale}. Kluwer Academic Publishers.
    \item %第一章のまとめが良い
    Jean Jacod, and Albert N. Shiryaev. (2003). \textit{Limit Theorems for Stochastic Processes}.
    \item
    Liptser, R. S. and, Shiryaev, A. N. (1974). \textit{Ststistics of Random Processes, I}. General Theory. Springer. (Second Revised and Expanded Edition 2001).
    \bibitem{Scheutzow}
    Michael Scheutzo. (2018). \textit{Stochastic Processes}. Lecture Notes, Technische Universitat Berlin.
    \item
    Pierre Brémaud. (2020). \textit{Probability Theory And Stochastic Processes}. Springer.

    1.マルチンゲール
    \item %独学に極めて向いている.Paris 6の新世代騎手.
    Jean-Francois Le Gall. (2013). \textit{Brownian Motion, Martingales, and Stochastic Calculus.} Springer.
    \item
    Daniel Revuz, and Marc Yor. (1999). \textit{Continuous Martingales and Brownian Motion}, 3rd. Springer.
    \item
    Schilling, R. L. (2005). \textit{Measures, Integrals and Martingales}. Cambridge Univ. Press.
    \item
    Schilling, R. L., and Partzsch, L. (2012). \textit{Brownian Motion: An Introduction to Stochastic Processes}. De Gruyter.
    \bibitem{Chow}
    Chow, Y. S., and Teicher, H. (2003). \textit{Probability Theory: Independence, Interchangeability, Martingales}. Springer Texts in Statistics.
    \item
    Meyer, P. A. (1966). \textit{Probability and Potentials}. Blaisdell Publishing Company.
    \item
    Dellacherie, C., and Meyer, P. A. (1978). \textit{Probabilities and Potential}. North Holland.
    \item
    Dellacherie, C. (1972). \textit{Capacités et processus stochastiques}. Springer Verlag, Berlin, Heidelberg.
    \item
    Doob, J. L. (1953). \textit{Stochastic Processes}. John Wiley \& Sons.

    3.確率過程の応用を扱った書籍は以下.
    \bibitem{Karlin}
    Karlin, S. (1966). \textit{A First Course in Stochastic Processes}. Academic Press.

    4.Gauss過程
    \bibitem{Hida-Gauss}
    飛田武幸,櫃田倍之. (1976). 『ガウス過程 表現と応用』.(紀伊國屋数学叢書9).
    \bibitem{Hida-Brown}
    飛田武幸. (1975). 『ブラウン運動』.(岩波書店).

    5.ブラウン運動と確率解析について.
    \item
    Peter Mörters (Univ. of Bath), and Yuval Peres (Microsoft Research). (2010). \textit{Brownian Motion}.
    \item
    松本裕行, 谷口説男 (2016). \textit{Stochastic Analysis: Itô and Malliavin Calculus in Tandem}. (Cambridge Studies in Advanced Mathematics).

    確率過程の逆問題
    \bibitem{国田}
    国田寛 (1976) 『確率過程の推定』(数理解析とその周辺14,産業図書).

    その他
    \bibitem{Williams}
    Williams - Probability with Martingales
    \bibitem{Schaffler}
    Schaffler - Generalized Stochastic Processes
    \bibitem{Durrett}
    Durrett - Probability: Theory and Examples
\end{thebibliography}

\begin{history}[Russia学派]
    Kolmogorovの居たMoscow州立大学と,Steklov数学研究所が中心.
    \begin{enumerate}
        \item Kolmogorov 03-87:Moscow州立大学を卒業後もそこで教えて,Arnold, Dynkin, Gelfand, Martin-Lof, Sinai, Prokhorov, Shiryaevらを教える.
        \begin{enumerate}
            \item Luzinが指導教官であったが,大粛清にあい,Luzinに対して盗用・身内贔屓などの証言をして職を奪った.これが事実なのか,なにかの力が働いたのかは不明で,関与者は一切この話をしたがらなかった.
            \item Moscow大学卒業後,2年ゲッティンゲンやミュンヘン,パリを旅行して,すぐにそこの教授に.
        \end{enumerate}
        \item Robert Liptser 36-19:確率過程とその逆問題.
        \begin{enumerate}
            \item Ukraine生まれで,モスクワ航空大学(Moscow Aviation Institute)で電気工学を学ぶが,6年後にMoscow州立大学で数学で2つ目の学士を取る.MIPT(Moscow Institute of Physics and Technology)\footnote{Landau}でPh.D.を取ってそのまま研究するが,
            58歳のときにロシアを出てイスラエルのテルアビブ大学へ.
            \item 主な応用先は確率制御などの工学とフィルター問題などの統計学で,マルチンゲールと条件付きGauss過程とに貢献がある.
        \end{enumerate}
        \item Albert Shiryaev 34-:
        \begin{enumerate}
            \item Moscow州立大学を卒業してからずっとSteklov数学研究所\footnote{Arnold, Perelman}.
        \end{enumerate}
        \item Ildar A. Ibragimov 32-:
        \begin{enumerate}
            \item Leningrad州立大学でYuri Linnikに学ぶ.
            \item LinnikをついでSteklovの統計手法ラボの所長に.
        \end{enumerate}
        \item Vladimir I. Bogachev 61-:Moscow州立大学を卒業後,そこの教授に.最も論文が引用されているロシアの数学者の一人.
        \item Alexei Borodin 75-:現在MIT.
        \begin{enumerate}
            \item ウクライナのドネツィク州立大学の数学教授の息子に生まれ,数学オリンピックで銀メダル.
            \item その後モスクワ州立大学へ.学部を出たら米国のPennsylvania大学へ.
        \end{enumerate}
    \end{enumerate}
\end{history}

\begin{history}[France学派]
    Malliavinが教えていたパリ第六大学(Paris 6, UPMC, Pierre and Marie Curie University)が中心.
    \begin{enumerate}
        \item Paul Malliavin 25-10:調和解析と確率解析.
        \begin{enumerate}
            \item Strookが指摘している通り,調和解析から確率論に入ったのはWienerに似ている.
        \end{enumerate}
        \item Jacques Neveu 32-16 パリ第六大学で教えていた,戦後のパリ学派の復興者の一人.
        \item Jean Jacod 44-:Neveuに学ぶ.パリ第六大学で教える.
        \item \textbf{Marc Yor} 49-14:パリ第六大学で学び,その後もそこで教え続ける.
        Brown運動とその数理ファイナンスへの応用の大家.
        \item \textbf{Jean-François Le Gall} 59-:Yorに学ぶ.
        \item Fabrice Baudoin:Yorに学び,現在Connecticut大学教授.
    \end{enumerate}
    エコール・ノルマルもいる.
    \begin{enumerate}
        \item Maurice Frechet 78-73:高等師範学校.
        \item Levy 86-71:高等師範学校.HadamardとVolterraに学ぶ.
        \item Lucien Le Cam 24-:Berkeleyで漸近理論を打ち立てる.接触性,漸近正規性は彼による.
        \begin{enumerate}
            \item 農家の出身で学ぶのに極めて苦労したがParis大学へ.学士を取るとhydroelectric utilityとして働きながらPrais大学の統計セミナーに5年出続ける.
            \item 26でUC Berkeleyに呼ばれる.もともと1年居たらutilityとして戻って働く予定であったが,Ph.D.まで取る.
        \end{enumerate}
        \item Paul-Andre Meyer 34-03:Strasbourg(パリ大学の別名)派の創始者.
        \begin{enumerate}
            \item Ecole Normaleで学び,Levyの弟子であるMichel Loeveに学ぶ.
            \item 博士が終わると渡米しDoobと研究.
            \item フランスに返ってStrasbourgでセミナーをし,確率過程論の中心地とした.
        \end{enumerate}
        \item Daniel Revuz 36-:Jacques NeveuとMeyerに学ぶ.Yorとの局所マルチンゲールの研究で有名.
        \item Claude Dellacherie 43-
        \begin{enumerate}
            \item Strasbourg大学でMeyerの下で学ぶ.
            \item IASとCNRSで働き,現在はRouen大学.
        \end{enumerate}
        \item Pierre Brémaud 45-(?):通信学者
        \begin{enumerate}
            \item ポリテクニーク出身で,72年にUC Berkeleyで電気工学とCSのPh.D!
            \item エコール・ポリテクニック応用数学科講師なども経験したが,現在はスイスのPEFL(連邦工科大学Lausanne校)の通信システム分野の3つのコアコースのうちの1つを教えている.
        \end{enumerate}
    \end{enumerate}
\end{history}

\begin{history}[U.S.学派]
    Birkhoff, Bochnerらが居たHarvard大学で,調和解析が盛んであった.そこ中心.
    Illinois大学にはDoobとRanga Raoら.
    Chicago大学にはBillingsleyら,PrincetonにFellerで,MITにWiener.
    NY大学にはCourant数理科学研究所があり,そこにはVaradhan, Donsker, McKeanら.
    Lehmann, Bickel, Le Camら統計学者はNeymanが統計学部を作ったUC Berkeleyから.
    \begin{enumerate}[({H}1)]
        \item Joseph L. Doob 10-04:Wiener, Malliavin同様,調和解析出身.
        \begin{enumerate}
            \item HarvardでJoseph L. Walshの指導を受ける.
            \item その後はPrincetonで3年ポスドクをしたのみで,Illinois大学で定年まで教える.
            \item しかしその後1929の大恐慌の影響で仕事が見つからず,Koopmanの助言で統計学者のHarold Hotellingの予算で彼と働くことになる.そこで,Doobは確率論の分野に入ることとなった.
            \item その後33年にKolmogorovの公理が発表されるのは,伊藤清と同じ道を辿る.
        \end{enumerate}
        \item William Gemmell Cochran 09-80:英国出身で,CambridgeでWishartに学び(Karl Pearsonの弟子),RothamstedというFisherな路線を辿った後に渡米し,主にHarvardで教える.このCochranに学んだのがRubinである.
        \item Donald Rubin 43-:Harvard出身でそこで教える.
        \begin{enumerate}
            \item HarvardでPh.D.プログラムに2回参加している.初めは心理学で,その後統計で入り直す.
        \end{enumerate}
        \item 他にも,James RobinsとAndrea RotnitzkyはHarvardの公衆衛生部門と生物統計部門にいる.
    \end{enumerate}
    なお,Rubinとの共著が多いPaul R. RosenbaumはPennsylvania大学に勤めていた.
    \begin{enumerate}[({NY}1)]
        \item Henry P. McKean, Jr. 30-:NY大学で教える.
        \begin{enumerate}
            \item Willian Fellerの下でPrincetonでPh.D.を得る.Daniel Stroockも教える.
        \end{enumerate}
        \item Daniel W. Stroock 40-:楠岡重雄とMalliavin解析を,Varadhanと拡散過程を研究.
        \begin{enumerate}
            \item HarvardからMark Kacの居たRockefeller大学へ.
            \item NY大学から現在MITへ.
        \end{enumerate}
        \item Donsker, M. D. 24-91:NY大学.
        \begin{enumerate}
            \item Varadhanとの共著が増えてから,NY大学へ移動.
        \end{enumerate}
    \end{enumerate}
    次はStroockもいるMIT.
    \begin{enumerate}[({MIT}1)]
        \item Daniel W. Stroock 40-:
        \begin{enumerate}
            \item 博士(Rockefeller大学)はMark Kacに教わる.Courant数学研究所,Colorado大学で教えたのち,MITにて教授.伊藤清の確率解析にいち早く反応し,楠岡茂雄とMalliavin解析に基本的な貢献をした先駆者で,その後はVaradhanとの拡散過程の研究で有名.
        \end{enumerate}
        \item Whitney K. Newey 54-:計量経済学.
        \item Victor Chenozhukov 74-
        \begin{enumerate}
            \item Illinois大学で統計を,Stanfordで経済学を学ぶ.
        \end{enumerate}
    \end{enumerate}
    統計は西海岸を中心に.UC BerkeleyにはJerzy Neymanが居た.
    \begin{enumerate}[({UCB}1)]
        \item Erich Leo Lehmann 17-09:UC BerkeleyでJerzy Neymannの下で学ぶ.
        \item Peter J. Bickel 40-:UC Berkeley
        \item Mark van der Laan 67-:Utrecht大学出身だが,UC Berkeleyで教えている.
    \end{enumerate}
\end{history}

\begin{history}[米へ移った学派]
    \begin{enumerate}
        \item Rabi Bhattacharya 37- :バングラデシュ出身.
        \begin{enumerate}
            \item Chicago大学のBillingsleyの下でPh.D. 多次元CLTの収束速度の問題を解く.
            \item Arizona大学を中心に教えていた.78年にGhosh, J. K.と共に形式的Edgeworth展開の正当性の問題を解く.
        \end{enumerate}
        \item Olav Kallenberg 39-:Sweden出身.交換可能な確率過程.
        \begin{enumerate}
            \item Auburn大学へ渡米する.
            \item ところでLars HormanderもSweden.卒業も教員も
        \end{enumerate}
        \item David Nualart 51- スペイン
        \begin{enumerate}
            \item Barcelona大学を卒業してそこで教えてから,Kansas大学へ渡米.
        \end{enumerate}
        \item Judea Pearl 36-:計算機科学者,哲学者.
        \item Yuval Peres 63-:イスラエル出身.
        \begin{enumerate}
            \item Hebrew大学でHillel Furstenbergに学ぶ.
            \item その後ポスドクで米国に渡り,31歳でUC Berkeleyで教授に.
            \item 06年にMicrosoft Researchの理論グループに加入し,principal researcherに.
        \end{enumerate}
    \end{enumerate}
\end{history}

\begin{history}[インド学派]
    Raoの教えていたISI (Indian Statistical Institute)中心.
    Raoの弟子である(2)~(5)は"famous four"と呼ばれる.
    \begin{enumerate}
        \item C. R. Rao 20-:Cambridge大学でRonald Fisher 90-29に学ぶ.
        \item S. R. S. Varadhan 40-:Strookと拡散過程,Donskerと大偏差原理の研究.
        \begin{enumerate}
            \item Raoの下で博士号を取るが,そのdegenceのときにRaoがKolmogorovも呼んだ.
        \end{enumerate}
        \item Parthasarathy, K. R. 36-:SteklovなどでKolmogorovと働く.
        \item Varadarajan V. S. 37-19:確率論から表現論へ移り,そこで活躍.
        \item Ramaswamy Ranga Rao -21:Illinois大学教授,Lie群・Lie代数と統計.
    \end{enumerate}
\end{history}

\begin{history}[ドイツ語圏]\mbox{}
    \begin{enumerate}
        \item Peter J. Huber 34-:ETH Zurich大学でEchmannという純粋数学者に指導を受ける.
        \item Pierre Bremaud:スイス連邦工科大学ローザンヌ校EPFL(Ecole polytechnique federale de Lausanne)
        \item Michael Scheutzow 54- :ドイツBerlin工科大学(TU Berlin)
        \item René Schilling:ドイツDresden大学(TUD).
    \end{enumerate}
\end{history}

\begin{history}[イタリア学派]\mbox{}
    \begin{enumerate}
        \item Giuseppe Da Prato :Roma Sapienza大学(ローマにある国立大学3つのうちの最も古いもの,他2つは戦後).
        \item Stefano M. Iacus:Milan大学教授.
        \item Paolo Baldi:Roma "Tor Vergata"大学
    \end{enumerate}
\end{history}

\begin{history}[オランダ学派]
    \begin{enumerate}
        \item Richard D. Gill 51-:Utrecht大学.計数過程と生存改正.
        \item Aad van der Vaart 59-:アムステルダム自由大学(Vrije Universiteit Amsterdam)→Leiden大学→21年にDelft大学.
        \item Mark van der Laan 67-:Utrecht大学出身だが,UC Berkeleyで教えている.Gillと,Bickelにも学ぶ.
        博士論文からセミパラメトリックモデルの研究をしている.
    \end{enumerate}
\end{history}



\end{document}