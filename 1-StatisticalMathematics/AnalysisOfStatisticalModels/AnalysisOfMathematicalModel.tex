\documentclass[uplatex,dvipdfmx]{jsreport}
\title{数理モデルの解析学}
\author{}
\pagestyle{headings} \setcounter{secnumdepth}{4}
\input{/Users/Hirofumi Shiba/NatureOfStatistics/preamble_no_fonts.tex}
%\input{/Users/hirofumi.shiba48/NatureOfStatistics/preamble_no_fonts.tex}
\usepackage[math]{anttor}
\begin{document}
\tableofcontents

\begin{quotation}
    感度分析,影響関数などの手法を用いて,頑健な数理モデルを作ろうとする試みは,
    セミパラメトリックモデルや因果推論など,統計数理を横断して考えられている.
    モデルの頑健性は物理学では採用され得ない視座であることから,新しい数理科学の最初の挑戦と言えるかもしれない.
    
    数理モデルはFrechet多様体の概念に収まる.
    古典力学系の相空間はEuclid多様体になり,パラメータが無限次元になってもFrechet多様体になる.
    その微分理論は,Banach空間の間の作用素の微分の理論で事足りるはずである.
\end{quotation}

\chapter{研究の軌跡}

\section{データ融合}

\begin{thebibliography}{9}
    \bibitem{}
    Dynamic confounding and Long term treatment effect estimation by data combination: point and partial identification
\end{thebibliography}



\chapter{参考文献}

\begin{thebibliography}{99}
    
\end{thebibliography}

\end{document}