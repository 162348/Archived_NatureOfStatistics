\documentclass[uplatex,dvipdfmx]{jsreport}
\title{ノンパラメトリック手法}
\author{司馬博文}
\date{\today}
\pagestyle{headings} \setcounter{secnumdepth}{4}
\input{/home/hirofumi/StatisticalNature/preamble_no_fonts.tex}
%\input{/Users/hirofumi.shiba48/StatisticalNature/preamble_no_fonts.tex}
\usepackage[math]{anttor}
\begin{document}
\tableofcontents

\chapter{古典的推測}

\begin{quotation}
    史上初めての分布に仮定を置かない統計手法は,順位検定であった.
\end{quotation}

\section{順位検定}

\chapter{カーネル法}

\section{密度関数のカーネル推定}

\chapter{標本分割}

\begin{quotation}
    統計的リサンプリング法と,その発展としての標本分割法を考える.
\end{quotation}

\section{ジャックナイフ法}

\section{ブートストラップ法}

\section{$V$-統計量の論文}

\subsection{Introduction}

\begin{definition}
    統計量とは,モデル$\P$上の汎関数$\P\to\R$に他ならない.
    特に,論文\cite{von Mises}では,\textbf{統計的汎関数}を,経験分布関数$S_n$(repartitionとよんでいる)上の汎関数として定義している.
    最も基本的なものは,$\psi(x)$を定め,これに対する$S_n$による積分として線型汎関数を定義することである.
    また,線型なものはすべて積分によって表現できる(例えば$L^2$空間上では).
    すると,中心極限定理とは,線型汎関数に関する分布の理論だ,ということになる.
\end{definition}

\begin{discussion}[動機]
    $V$をモデル$\P$上を動く累積分布関数として,
    一般の汎関数$f(V(x))$を考え,上の例はある点での値$f(S_n(x))$に過ぎない,と考える.
    すると,汎関数の考え方の源流であるVito Volterra (1887)に沿って,可微分統計汎関数$f(S_n(x))$の漸近分布は,延長された汎関数$f(V_n(x))$の点$\o{V}_n(x):=\frac{1}{n}\sum_{i=1}^nV_i(x)$におけるTaylor展開の第一項に等しい.
    1次の項が消えないならば,漸近分布は正規である.この消息を正則条件の下で導くのが中心極限定理である.
    1次の項が消えて高次の項が現れる場合,漸近分布は正規でなくなるのである.

    特に,関数$f(V(x))$と分布列$V_1(x),V_2(x),\cdots$とが独立に定義されているとき,$f$の$\o{V}_n(x)$における微分係数が消えることはない.
    標本平均を取っているからであろうか.$\bE$と$f$とが交換可能だからだろうか.
\end{discussion}

\begin{discussion}
    型$m\;(m=1,2,\cdots)$の分布で表示
    \[n^{m/2}(f(S_n(x))-f(\o{V}_n(x)))\]
    を持つものは,有限の平均と分散を持つ関数となる.
    $m$が奇数のとき,分布は$0$に関して対称になる.
\end{discussion}

\chapter{漸近正規統計量}

\begin{quotation}
    ノンパラメトリック手法の比較・評価は漸近的手法に頼り切ることになる.
\end{quotation}

\chapter{参考文献}

\begin{thebibliography}{99}
    \bibitem{Tsybakov}
    Tsybakov \textit{Introduction To Nonparametric Estimation}.
    \bibitem{前園}
    前園宜彦 (2019). 『ノンパラメトリック統計』(数学の輝き,共立出版).

    \bibitem{von Mises}
    von Mises, R. (1947). \textit{On the asymptotic distribution of differentiable statistical functions}.
\end{thebibliography}

\end{document}