\documentclass[uplatex,dvipdfmx]{jsreport}
\title{函数解析}
\author{}
\pagestyle{headings} \setcounter{secnumdepth}{4}
%%%%%%%%%%%%%%% 数理文書の組版 %%%%%%%%%%%%%%%

\usepackage{mathtools} %内部でamsmathを呼び出すことに注意.
%\mathtoolsset{showonlyrefs=true} %labelを附した数式にのみ附番される設定.
\usepackage{amsfonts} %mathfrak, mathcal, mathbbなど.
\usepackage{amsthm} %定理環境.
\usepackage{amssymb} %AMSFontsを使うためのパッケージ.
\usepackage{ascmac} %screen, itembox, shadebox環境.全てLATEX2εの標準機能の範囲で作られたもの.
\usepackage{comment} %comment環境を用いて,複数行をcomment outできるようにするpackage
\usepackage{wrapfig} %図の周りに文字をwrapさせることができる.詳細な制御ができる.
\usepackage[usenames, dvipsnames]{xcolor} %xcolorはcolorの拡張.optionの意味はdvipsnamesはLoad a set of predefined colors. forestgreenなどの色が追加されている.usenamesはobsoleteとだけ書いてあった.
\setcounter{tocdepth}{2} %目次に表示される深さ.2はsubsectionまで
\usepackage{multicol} %\begin{multicols}{2}環境で途中からmulticolumnに出来る.
\usepackage{mathabx}\newcommand{\wc}{\widecheck} %\widecheckなどのフォントパッケージ

%%%%%%%%%%%%%%% フォント %%%%%%%%%%%%%%%

\usepackage{textcomp, mathcomp} %Text Companionとは,T1 encodingに入らなかった文字群.これを使うためのパッケージ.\textsectionでブルバキに!
\usepackage[T1]{fontenc} %8bitエンコーディングにする.comp系拡張数学文字の動作が安定する.

%%%%%%%%%%%%%%% 一般文書の組版 %%%%%%%%%%%%%%%

\definecolor{花緑青}{cmyk}{1,0.07,0.10,0.10}\definecolor{サーモンピンク}{cmyk}{0,0.65,0.65,0.05}\definecolor{暗中模索}{rgb}{0.2,0.2,0.2}
\usepackage{url}\usepackage[dvipdfmx,colorlinks,linkcolor=花緑青,urlcolor=花緑青,citecolor=花緑青]{hyperref} %生成されるPDFファイルにおいて、\tableofcontentsによって書き出された目次をクリックすると該当する見出しへジャンプしたり、さらには、\label{ラベル名}を番号で参照する\ref{ラベル名}やthebibliography環境において\bibitem{ラベル名}を文献番号で参照する\cite{ラベル名}においても番号をクリックすると該当箇所にジャンプする.囲み枠はダサいので,colorlinksで囲み廃止し,リンク自体に色を付けることにした.
\usepackage{pxjahyper} %pxrubrica同様,八登崇之さん.hyperrefは日本語pLaTeXに最適化されていないから,hyperrefとセットで,(u)pLaTeX+hyperref+dvipdfmxの組み合わせで日本語を含む「しおり」をもつPDF文書を作成する場合に必要となる機能を提供する
\usepackage{ulem} %取り消し線を引くためのパッケージ
\usepackage{pxrubrica} %日本語にルビをふる.八登崇之(やとうたかゆき)氏による.

%%%%%%%%%%%%%%% 科学文書の組版 %%%%%%%%%%%%%%%

\usepackage[version=4]{mhchem} %化学式をTikZで簡単に書くためのパッケージ.
\usepackage{chemfig} %化学構造式をTikZで描くためのパッケージ.
\usepackage{siunitx} %IS単位を書くためのパッケージ

%%%%%%%%%%%%%%% 作図 %%%%%%%%%%%%%%%

\usepackage{tikz}\usetikzlibrary{positioning,automata}\usepackage{tikz-cd}\usepackage[all]{xy}
\def\objectstyle{\displaystyle} %デフォルトではxymatrix中の数式が文中数式モードになるので,それを直す.\labelstyleも同様にxy packageの中で定義されており,文中数式モードになっている.

\usepackage{graphicx} %rotatebox, scalebox, reflectbox, resizeboxなどのコマンドや,図表の読み込み\includegraphicsを司る.graphics というパッケージもありますが,graphicx はこれを高機能にしたものと考えて結構です(ただし graphicx は内部で graphics を読み込みます)
\usepackage[top=15truemm,bottom=15truemm,left=10truemm,right=10truemm]{geometry} %足助さんからもらったオプション

%%%%%%%%%%%%%%% 参照 %%%%%%%%%%%%%%%
%参考文献リストを出力したい箇所に\bibliography{../mathematics.bib}を追記すると良い.

%\bibliographystyle{jplain}
%\bibliographystyle{jname}
\bibliographystyle{apalike}

%%%%%%%%%%%%%%% 計算機文書の組版 %%%%%%%%%%%%%%%

\usepackage[breakable]{tcolorbox} %加藤晃史さんがフル活用していたtcolorboxを,途中改ページ可能で.
\tcbuselibrary{theorems} %https://qiita.com/t_kemmochi/items/483b8fcdb5db8d1f5d5e
\usepackage{enumerate} %enumerate環境を凝らせる.

\usepackage{listings} %ソースコードを表示できる環境.多分もっといい方法ある.
\usepackage{jvlisting} %日本語のコメントアウトをする場合jlistingが必要
\lstset{ %ここからソースコードの表示に関する設定.lstlisting環境では,[caption=hoge,label=fuga]などのoptionを付けられる.
%[escapechar=!]とすると,LaTeXコマンドを使える.
  basicstyle={\ttfamily},
  identifierstyle={\small},
  commentstyle={\smallitshape},
  keywordstyle={\small\bfseries},
  ndkeywordstyle={\small},
  stringstyle={\small\ttfamily},
  frame={tb},
  breaklines=true,
  columns=[l]{fullflexible},
  numbers=left,
  xrightmargin=0zw,
  xleftmargin=3zw,
  numberstyle={\scriptsize},
  stepnumber=1,
  numbersep=1zw,
  lineskip=-0.5ex
}
%\makeatletter %caption番号を「[chapter番号].[section番号].[subsection番号]-[そのsubsection内においてn番目]」に変更
%    \AtBeginDocument{
%    \renewcommand*{\thelstlisting}{\arabic{chapter}.\arabic{section}.\arabic{lstlisting}}
%    \@addtoreset{lstlisting}{section}
%    }
%\makeatother
\renewcommand{\lstlistingname}{算譜} %caption名を"program"に変更

\newtcolorbox{tbox}[3][]{%
colframe=#2,colback=#2!10,coltitle=#2!20!black,title={#3},#1}

% 証明内の文字が小さくなる環境.
\newenvironment{Proof}[1][\bf\underline{[証明]}]{\proof[#1]\color{darkgray}}{\endproof}

%%%%%%%%%%%%%%% 数学記号のマクロ %%%%%%%%%%%%%%%

%%% 括弧類
\newcommand{\abs}[1]{\lvert#1\rvert}\newcommand{\Abs}[1]{\left|#1\right|}\newcommand{\norm}[1]{\|#1\|}\newcommand{\Norm}[1]{\left\|#1\right\|}\newcommand{\Brace}[1]{\left\{#1\right\}}\newcommand{\BRace}[1]{\biggl\{#1\biggr\}}\newcommand{\paren}[1]{\left(#1\right)}\newcommand{\Paren}[1]{\biggr(#1\biggl)}\newcommand{\bracket}[1]{\langle#1\rangle}\newcommand{\brac}[1]{\langle#1\rangle}\newcommand{\Bracket}[1]{\left\langle#1\right\rangle}\newcommand{\Brac}[1]{\left\langle#1\right\rangle}\newcommand{\bra}[1]{\left\langle#1\right|}\newcommand{\ket}[1]{\left|#1\right\rangle}\newcommand{\Square}[1]{\left[#1\right]}\newcommand{\SQuare}[1]{\biggl[#1\biggr]}
\renewcommand{\o}[1]{\overline{#1}}\renewcommand{\u}[1]{\underline{#1}}\newcommand{\wt}[1]{\widetilde{#1}}\newcommand{\wh}[1]{\widehat{#1}}
\newcommand{\pp}[2]{\frac{\partial #1}{\partial #2}}\newcommand{\ppp}[3]{\frac{\partial #1}{\partial #2\partial #3}}\newcommand{\dd}[2]{\frac{d #1}{d #2}}
\newcommand{\floor}[1]{\lfloor#1\rfloor}\newcommand{\Floor}[1]{\left\lfloor#1\right\rfloor}\newcommand{\ceil}[1]{\lceil#1\rceil}
\newcommand{\ocinterval}[1]{(#1]}\newcommand{\cointerval}[1]{[#1)}\newcommand{\COinterval}[1]{\left[#1\right)}


%%% 予約語
\renewcommand{\iff}{\;\mathrm{iff}\;}
\newcommand{\False}{\mathrm{False}}\newcommand{\True}{\mathrm{True}}
\newcommand{\otherwise}{\mathrm{otherwise}}
\newcommand{\st}{\;\mathrm{s.t.}\;}

%%% 略記
\newcommand{\M}{\mathcal{M}}\newcommand{\cF}{\mathcal{F}}\newcommand{\cD}{\mathcal{D}}\newcommand{\fX}{\mathfrak{X}}\newcommand{\fY}{\mathfrak{Y}}\newcommand{\fZ}{\mathfrak{Z}}\renewcommand{\H}{\mathcal{H}}\newcommand{\fH}{\mathfrak{H}}\newcommand{\bH}{\mathbb{H}}\newcommand{\id}{\mathrm{id}}\newcommand{\A}{\mathcal{A}}\newcommand{\U}{\mathfrak{U}}
\newcommand{\lmd}{\lambda}
\newcommand{\Lmd}{\Lambda}

%%% 矢印類
\newcommand{\iso}{\xrightarrow{\,\smash{\raisebox{-0.45ex}{\ensuremath{\scriptstyle\sim}}}\,}}
\newcommand{\Lrarrow}{\;\;\Leftrightarrow\;\;}

%%% 注記
\newcommand{\rednote}[1]{\textcolor{red}{#1}}

% ノルム位相についての閉包 https://newbedev.com/how-to-make-double-overline-with-less-vertical-displacement
\makeatletter
\newcommand{\dbloverline}[1]{\overline{\dbl@overline{#1}}}
\newcommand{\dbl@overline}[1]{\mathpalette\dbl@@overline{#1}}
\newcommand{\dbl@@overline}[2]{%
  \begingroup
  \sbox\z@{$\m@th#1\overline{#2}$}%
  \ht\z@=\dimexpr\ht\z@-2\dbl@adjust{#1}\relax
  \box\z@
  \ifx#1\scriptstyle\kern-\scriptspace\else
  \ifx#1\scriptscriptstyle\kern-\scriptspace\fi\fi
  \endgroup
}
\newcommand{\dbl@adjust}[1]{%
  \fontdimen8
  \ifx#1\displaystyle\textfont\else
  \ifx#1\textstyle\textfont\else
  \ifx#1\scriptstyle\scriptfont\else
  \scriptscriptfont\fi\fi\fi 3
}
\makeatother
\newcommand{\oo}[1]{\dbloverline{#1}}

% hslashの他の文字Ver.
\newcommand{\hslashslash}{%
    \scalebox{1.2}{--
    }%
}
\newcommand{\dslash}{%
  {%
    \vphantom{d}%
    \ooalign{\kern.05em\smash{\hslashslash}\hidewidth\cr$d$\cr}%
    \kern.05em
  }%
}
\newcommand{\dint}{%
  {%
    \vphantom{d}%
    \ooalign{\kern.05em\smash{\hslashslash}\hidewidth\cr$\int$\cr}%
    \kern.05em
  }%
}
\newcommand{\dL}{%
  {%
    \vphantom{d}%
    \ooalign{\kern.05em\smash{\hslashslash}\hidewidth\cr$L$\cr}%
    \kern.05em
  }%
}

%%% 演算子
\DeclareMathOperator{\grad}{\mathrm{grad}}\DeclareMathOperator{\rot}{\mathrm{rot}}\DeclareMathOperator{\divergence}{\mathrm{div}}\DeclareMathOperator{\tr}{\mathrm{tr}}\newcommand{\pr}{\mathrm{pr}}
\newcommand{\Map}{\mathrm{Map}}\newcommand{\dom}{\mathrm{Dom}\;}\newcommand{\cod}{\mathrm{Cod}\;}\newcommand{\supp}{\mathrm{supp}\;}


%%% 線型代数学
\newcommand{\vctr}[2]{\begin{pmatrix}#1\\#2\end{pmatrix}}\newcommand{\vctrr}[3]{\begin{pmatrix}#1\\#2\\#3\end{pmatrix}}\newcommand{\mtrx}[4]{\begin{pmatrix}#1&#2\\#3&#4\end{pmatrix}}\newcommand{\smtrx}[4]{\paren{\begin{smallmatrix}#1&#2\\#3&#4\end{smallmatrix}}}\newcommand{\Ker}{\mathrm{Ker}\;}\newcommand{\Coker}{\mathrm{Coker}\;}\newcommand{\Coim}{\mathrm{Coim}\;}\DeclareMathOperator{\rank}{\mathrm{rank}}\newcommand{\lcm}{\mathrm{lcm}}\newcommand{\sgn}{\mathrm{sgn}\,}\newcommand{\GL}{\mathrm{GL}}\newcommand{\SL}{\mathrm{SL}}\newcommand{\alt}{\mathrm{alt}}
%%% 複素解析学
\renewcommand{\Re}{\mathrm{Re}\;}\renewcommand{\Im}{\mathrm{Im}\;}\newcommand{\Gal}{\mathrm{Gal}}\newcommand{\PGL}{\mathrm{PGL}}\newcommand{\PSL}{\mathrm{PSL}}\newcommand{\Log}{\mathrm{Log}\,}\newcommand{\Res}{\mathrm{Res}\,}\newcommand{\on}{\mathrm{on}\;}\newcommand{\hatC}{\widehat{\C}}\newcommand{\hatR}{\hat{\R}}\newcommand{\PV}{\mathrm{P.V.}}\newcommand{\diam}{\mathrm{diam}}\newcommand{\Area}{\mathrm{Area}}\newcommand{\Lap}{\Laplace}\newcommand{\f}{\mathbf{f}}\newcommand{\cR}{\mathcal{R}}\newcommand{\const}{\mathrm{const.}}\newcommand{\Om}{\Omega}\newcommand{\Cinf}{C^\infty}\newcommand{\ep}{\epsilon}\newcommand{\dist}{\mathrm{dist}}\newcommand{\opart}{\o{\partial}}\newcommand{\Length}{\mathrm{Length}}
%%% 集合と位相
\renewcommand{\O}{\mathcal{O}}\renewcommand{\S}{\mathcal{S}}\renewcommand{\U}{\mathcal{U}}\newcommand{\V}{\mathcal{V}}\renewcommand{\P}{\mathcal{P}}\newcommand{\R}{\mathbb{R}}\newcommand{\N}{\mathbb{N}}\newcommand{\C}{\mathbb{C}}\newcommand{\Z}{\mathbb{Z}}\newcommand{\Q}{\mathbb{Q}}\newcommand{\TV}{\mathrm{TV}}\newcommand{\ORD}{\mathrm{ORD}}\newcommand{\Tr}{\mathrm{Tr}}\newcommand{\Card}{\mathrm{Card}\;}\newcommand{\Top}{\mathrm{Top}}\newcommand{\Disc}{\mathrm{Disc}}\newcommand{\Codisc}{\mathrm{Codisc}}\newcommand{\CoDisc}{\mathrm{CoDisc}}\newcommand{\Ult}{\mathrm{Ult}}\newcommand{\ord}{\mathrm{ord}}\newcommand{\maj}{\mathrm{maj}}\newcommand{\bS}{\mathbb{S}}\newcommand{\PConn}{\mathrm{PConn}}

%%% 形式言語理論
\newcommand{\REGEX}{\mathrm{REGEX}}\newcommand{\RE}{\mathbf{RE}}
%%% Graph Theory
\newcommand{\SimpGph}{\mathrm{SimpGph}}\newcommand{\Gph}{\mathrm{Gph}}\newcommand{\mult}{\mathrm{mult}}\newcommand{\inv}{\mathrm{inv}}

%%% 多様体
\newcommand{\Der}{\mathrm{Der}}\newcommand{\osub}{\overset{\mathrm{open}}{\subset}}\newcommand{\osup}{\overset{\mathrm{open}}{\supset}}\newcommand{\al}{\alpha}\newcommand{\K}{\mathbb{K}}\newcommand{\Sp}{\mathrm{Sp}}\newcommand{\g}{\mathfrak{g}}\newcommand{\h}{\mathfrak{h}}\newcommand{\Exp}{\mathrm{Exp}\;}\newcommand{\Imm}{\mathrm{Imm}}\newcommand{\Imb}{\mathrm{Imb}}\newcommand{\codim}{\mathrm{codim}\;}\newcommand{\Gr}{\mathrm{Gr}}
%%% 代数
\newcommand{\Ad}{\mathrm{Ad}}\newcommand{\finsupp}{\mathrm{fin\;supp}}\newcommand{\SO}{\mathrm{SO}}\newcommand{\SU}{\mathrm{SU}}\newcommand{\acts}{\curvearrowright}\newcommand{\mono}{\hookrightarrow}\newcommand{\epi}{\twoheadrightarrow}\newcommand{\Stab}{\mathrm{Stab}}\newcommand{\nor}{\mathrm{nor}}\newcommand{\T}{\mathbb{T}}\newcommand{\Aff}{\mathrm{Aff}}\newcommand{\rsub}{\triangleleft}\newcommand{\rsup}{\triangleright}\newcommand{\subgrp}{\overset{\mathrm{subgrp}}{\subset}}\newcommand{\Ext}{\mathrm{Ext}}\newcommand{\sbs}{\subset}\newcommand{\sps}{\supset}\newcommand{\In}{\mathrm{in}\;}\newcommand{\Tor}{\mathrm{Tor}}\newcommand{\p}{\b{p}}\newcommand{\q}{\mathfrak{q}}\newcommand{\m}{\mathfrak{m}}\newcommand{\cS}{\mathcal{S}}\newcommand{\Frac}{\mathrm{Frac}\,}\newcommand{\Spec}{\mathrm{Spec}\,}\newcommand{\bA}{\mathbb{A}}\newcommand{\Sym}{\mathrm{Sym}}\newcommand{\Ann}{\mathrm{Ann}}\newcommand{\Her}{\mathrm{Her}}\newcommand{\Bil}{\mathrm{Bil}}\newcommand{\Ses}{\mathrm{Ses}}\newcommand{\FVS}{\mathrm{FVS}}
%%% 代数的位相幾何学
\newcommand{\Ho}{\mathrm{Ho}}\newcommand{\CW}{\mathrm{CW}}\newcommand{\lc}{\mathrm{lc}}\newcommand{\cg}{\mathrm{cg}}\newcommand{\Fib}{\mathrm{Fib}}\newcommand{\Cyl}{\mathrm{Cyl}}\newcommand{\Ch}{\mathrm{Ch}}
%%% 微分幾何学
\newcommand{\rE}{\mathrm{E}}\newcommand{\e}{\b{e}}\renewcommand{\k}{\b{k}}\newcommand{\Christ}[2]{\begin{Bmatrix}#1\\#2\end{Bmatrix}}\renewcommand{\Vec}[1]{\overrightarrow{\mathrm{#1}}}\newcommand{\hen}[1]{\mathrm{#1}}\renewcommand{\b}[1]{\boldsymbol{#1}}

%%% 函数解析
\newcommand{\HS}{\mathrm{HS}}\newcommand{\loc}{\mathrm{loc}}\newcommand{\Lh}{\mathrm{L.h.}}\newcommand{\Epi}{\mathrm{Epi}\;}\newcommand{\slim}{\mathrm{slim}}\newcommand{\Ban}{\mathrm{Ban}}\newcommand{\Hilb}{\mathrm{Hilb}}\newcommand{\Ex}{\mathrm{Ex}}\newcommand{\Co}{\mathrm{Co}}\newcommand{\sa}{\mathrm{sa}}\newcommand{\nnorm}[1]{{\left\vert\kern-0.25ex\left\vert\kern-0.25ex\left\vert #1 \right\vert\kern-0.25ex\right\vert\kern-0.25ex\right\vert}}\newcommand{\dvol}{\mathrm{dvol}}\newcommand{\Sconv}{\mathrm{Sconv}}\newcommand{\I}{\mathcal{I}}\newcommand{\nonunital}{\mathrm{nu}}\newcommand{\cpt}{\mathrm{cpt}}\newcommand{\lcpt}{\mathrm{lcpt}}\newcommand{\com}{\mathrm{com}}\newcommand{\Haus}{\mathrm{Haus}}\newcommand{\proper}{\mathrm{proper}}\newcommand{\infinity}{\mathrm{inf}}\newcommand{\TVS}{\mathrm{TVS}}\newcommand{\ess}{\mathrm{ess}}\newcommand{\ext}{\mathrm{ext}}\newcommand{\Index}{\mathrm{Index}\;}\newcommand{\SSR}{\mathrm{SSR}}\newcommand{\vs}{\mathrm{vs.}}\newcommand{\fM}{\mathfrak{M}}\newcommand{\EDM}{\mathrm{EDM}}\newcommand{\Tw}{\mathrm{Tw}}\newcommand{\fC}{\mathfrak{C}}\newcommand{\bn}{\boldsymbol{n}}\newcommand{\br}{\boldsymbol{r}}\newcommand{\Lam}{\Lambda}\newcommand{\lam}{\lambda}\newcommand{\one}{\mathbf{1}}\newcommand{\dae}{\text{-a.e.}}\newcommand{\das}{\text{-a.s.}}\newcommand{\td}{\text{-}}\newcommand{\RM}{\mathrm{RM}}\newcommand{\BV}{\mathrm{BV}}\newcommand{\normal}{\mathrm{normal}}\newcommand{\lub}{\mathrm{lub}\;}\newcommand{\Graph}{\mathrm{Graph}}\newcommand{\Ascent}{\mathrm{Ascent}}\newcommand{\Descent}{\mathrm{Descent}}\newcommand{\BIL}{\mathrm{BIL}}\newcommand{\fL}{\mathfrak{L}}\newcommand{\De}{\Delta}
%%% 積分論
\newcommand{\calA}{\mathcal{A}}\newcommand{\calB}{\mathcal{B}}\newcommand{\D}{\mathcal{D}}\newcommand{\Y}{\mathcal{Y}}\newcommand{\calC}{\mathcal{C}}\renewcommand{\ae}{\mathrm{a.e.}\;}\newcommand{\cZ}{\mathcal{Z}}\newcommand{\fF}{\mathfrak{F}}\newcommand{\fI}{\mathfrak{I}}\newcommand{\E}{\mathcal{E}}\newcommand{\sMap}{\sigma\textrm{-}\mathrm{Map}}\DeclareMathOperator*{\argmax}{arg\,max}\DeclareMathOperator*{\argmin}{arg\,min}\newcommand{\cC}{\mathcal{C}}\newcommand{\comp}{\complement}\newcommand{\J}{\mathcal{J}}\newcommand{\sumN}[1]{\sum_{#1\in\N}}\newcommand{\cupN}[1]{\cup_{#1\in\N}}\newcommand{\capN}[1]{\cap_{#1\in\N}}\newcommand{\Sum}[1]{\sum_{#1=1}^\infty}\newcommand{\sumn}{\sum_{n=1}^\infty}\newcommand{\summ}{\sum_{m=1}^\infty}\newcommand{\sumk}{\sum_{k=1}^\infty}\newcommand{\sumi}{\sum_{i=1}^\infty}\newcommand{\sumj}{\sum_{j=1}^\infty}\newcommand{\cupn}{\cup_{n=1}^\infty}\newcommand{\capn}{\cap_{n=1}^\infty}\newcommand{\cupk}{\cup_{k=1}^\infty}\newcommand{\cupi}{\cup_{i=1}^\infty}\newcommand{\cupj}{\cup_{j=1}^\infty}\newcommand{\limn}{\lim_{n\to\infty}}\renewcommand{\l}{\mathcal{l}}\renewcommand{\L}{\mathcal{L}}\newcommand{\Cl}{\mathrm{Cl}}\newcommand{\cN}{\mathcal{N}}\newcommand{\Ae}{\textrm{-a.e.}\;}\newcommand{\csub}{\overset{\textrm{closed}}{\subset}}\newcommand{\csup}{\overset{\textrm{closed}}{\supset}}\newcommand{\wB}{\wt{B}}\newcommand{\cG}{\mathcal{G}}\newcommand{\Lip}{\mathrm{Lip}}\DeclareMathOperator{\Dom}{\mathrm{Dom}}\newcommand{\AC}{\mathrm{AC}}\newcommand{\Mol}{\mathrm{Mol}}
%%% Fourier解析
\newcommand{\Pe}{\mathrm{Pe}}\newcommand{\wR}{\wh{\mathbb{\R}}}\newcommand*{\Laplace}{\mathop{}\!\mathbin\bigtriangleup}\newcommand*{\DAlambert}{\mathop{}\!\mathbin\Box}\newcommand{\bT}{\mathbb{T}}\newcommand{\dx}{\dslash x}\newcommand{\dt}{\dslash t}\newcommand{\ds}{\dslash s}
%%% 数値解析
\newcommand{\round}{\mathrm{round}}\newcommand{\cond}{\mathrm{cond}}\newcommand{\diag}{\mathrm{diag}}
\newcommand{\Adj}{\mathrm{Adj}}\newcommand{\Pf}{\mathrm{Pf}}\newcommand{\Sg}{\mathrm{Sg}}

%%% 確率論
\newcommand{\Prob}{\mathrm{Prob}}\newcommand{\X}{\mathcal{X}}\newcommand{\Meas}{\mathrm{Meas}}\newcommand{\as}{\;\mathrm{a.s.}}\newcommand{\io}{\;\mathrm{i.o.}}\newcommand{\fe}{\;\mathrm{f.e.}}\newcommand{\F}{\mathcal{F}}\newcommand{\bF}{\mathbb{F}}\newcommand{\W}{\mathcal{W}}\newcommand{\Pois}{\mathrm{Pois}}\newcommand{\iid}{\mathrm{i.i.d.}}\newcommand{\wconv}{\rightsquigarrow}\newcommand{\Var}{\mathrm{Var}}\newcommand{\xrightarrown}{\xrightarrow{n\to\infty}}\newcommand{\au}{\mathrm{au}}\newcommand{\cT}{\mathcal{T}}\newcommand{\wto}{\overset{w}{\to}}\newcommand{\dto}{\overset{d}{\to}}\newcommand{\pto}{\overset{p}{\to}}\newcommand{\vto}{\overset{v}{\to}}\newcommand{\Cont}{\mathrm{Cont}}\newcommand{\stably}{\mathrm{stably}}\newcommand{\Np}{\mathbb{N}^+}\newcommand{\oM}{\overline{\mathcal{M}}}\newcommand{\fP}{\mathfrak{P}}\newcommand{\sign}{\mathrm{sign}}\DeclareMathOperator{\Div}{Div}
\newcommand{\bD}{\mathbb{D}}\newcommand{\fW}{\mathfrak{W}}\newcommand{\DL}{\mathcal{D}\mathcal{L}}\renewcommand{\r}[1]{\mathrm{#1}}\newcommand{\rC}{\mathrm{C}}
%%% 情報理論
\newcommand{\bit}{\mathrm{bit}}\DeclareMathOperator{\sinc}{sinc}
%%% 量子論
\newcommand{\err}{\mathrm{err}}
%%% 最適化
\newcommand{\varparallel}{\mathbin{\!/\mkern-5mu/\!}}\newcommand{\Minimize}{\text{Minimize}}\newcommand{\subjectto}{\text{subject to}}\newcommand{\Ri}{\mathrm{Ri}}\newcommand{\Cone}{\mathrm{Cone}}\newcommand{\Int}{\mathrm{Int}}
%%% 数理ファイナンス
\newcommand{\pre}{\mathrm{pre}}\newcommand{\om}{\omega}

%%% 偏微分方程式
\let\div\relax
\DeclareMathOperator{\div}{div}\newcommand{\del}{\partial}
\newcommand{\LHS}{\mathrm{LHS}}\newcommand{\RHS}{\mathrm{RHS}}\newcommand{\bnu}{\boldsymbol{\nu}}\newcommand{\interior}{\mathrm{in}\;}\newcommand{\SH}{\mathrm{SH}}\renewcommand{\v}{\boldsymbol{\nu}}\newcommand{\n}{\mathbf{n}}\newcommand{\ssub}{\Subset}\newcommand{\curl}{\mathrm{curl}}
%%% 常微分方程式
\newcommand{\Ei}{\mathrm{Ei}}\newcommand{\sn}{\mathrm{sn}}\newcommand{\wgamma}{\widetilde{\gamma}}
%%% 統計力学
\newcommand{\Ens}{\mathrm{Ens}}
%%% 解析力学
\newcommand{\cl}{\mathrm{cl}}\newcommand{\x}{\boldsymbol{x}}

%%% 統計的因果推論
\newcommand{\Do}{\mathrm{Do}}
%%% 応用統計学
\newcommand{\mrl}{\mathrm{mrl}}
%%% 数理統計
\newcommand{\comb}[2]{\begin{pmatrix}#1\\#2\end{pmatrix}}\newcommand{\bP}{\mathbb{P}}\newcommand{\compsub}{\overset{\textrm{cpt}}{\subset}}\newcommand{\lip}{\textrm{lip}}\newcommand{\BL}{\mathrm{BL}}\newcommand{\G}{\mathbb{G}}\newcommand{\NB}{\mathrm{NB}}\newcommand{\oR}{\o{\R}}\newcommand{\liminfn}{\liminf_{n\to\infty}}\newcommand{\limsupn}{\limsup_{n\to\infty}}\newcommand{\esssup}{\mathrm{ess.sup}}\newcommand{\asto}{\xrightarrow{\as}}\newcommand{\Cov}{\mathrm{Cov}}\newcommand{\cQ}{\mathcal{Q}}\newcommand{\VC}{\mathrm{VC}}\newcommand{\mb}{\mathrm{mb}}\newcommand{\Avar}{\mathrm{Avar}}\newcommand{\bB}{\mathbb{B}}\newcommand{\bW}{\mathbb{W}}\newcommand{\sd}{\mathrm{sd}}\newcommand{\w}[1]{\widehat{#1}}\newcommand{\bZ}{\boldsymbol{Z}}\newcommand{\Bernoulli}{\mathrm{Ber}}\newcommand{\Ber}{\mathrm{Ber}}\newcommand{\Mult}{\mathrm{Mult}}\newcommand{\BPois}{\mathrm{BPois}}\newcommand{\fraks}{\mathfrak{s}}\newcommand{\frakk}{\mathfrak{k}}\newcommand{\IF}{\mathrm{IF}}\newcommand{\bX}{\mathbf{X}}\newcommand{\bx}{\boldsymbol{x}}\newcommand{\indep}{\raisebox{0.05em}{\rotatebox[origin=c]{90}{$\models$}}}\newcommand{\IG}{\mathrm{IG}}\newcommand{\Levy}{\mathrm{Levy}}\newcommand{\MP}{\mathrm{MP}}\newcommand{\Hermite}{\mathrm{Hermite}}\newcommand{\Skellam}{\mathrm{Skellam}}\newcommand{\Dirichlet}{\mathrm{Dirichlet}}\newcommand{\Beta}{\mathrm{Beta}}\newcommand{\bE}{\mathbb{E}}\newcommand{\bG}{\mathbb{G}}\newcommand{\MISE}{\mathrm{MISE}}\newcommand{\logit}{\mathtt{logit}}\newcommand{\expit}{\mathtt{expit}}\newcommand{\cK}{\mathcal{K}}\newcommand{\dl}{\dot{l}}\newcommand{\dotp}{\dot{p}}\newcommand{\wl}{\wt{l}}\newcommand{\Gauss}{\mathrm{Gauss}}\newcommand{\fA}{\mathfrak{A}}\newcommand{\under}{\mathrm{under}\;}\newcommand{\whtheta}{\wh{\theta}}\newcommand{\Em}{\mathrm{Em}}\newcommand{\ztheta}{{\theta_0}}
\newcommand{\rO}{\mathrm{O}}\newcommand{\Bin}{\mathrm{Bin}}\newcommand{\rW}{\mathrm{W}}\newcommand{\rG}{\mathrm{G}}\newcommand{\rB}{\mathrm{B}}\newcommand{\rN}{\mathrm{N}}\newcommand{\rU}{\mathrm{U}}\newcommand{\HG}{\mathrm{HG}}\newcommand{\GAMMA}{\mathrm{Gamma}}\newcommand{\Cauchy}{\mathrm{Cauchy}}\newcommand{\rt}{\mathrm{t}}
\DeclareMathOperator{\erf}{erf}

%%% 圏
\newcommand{\varlim}{\varprojlim}\newcommand{\Hom}{\mathrm{Hom}}\newcommand{\Iso}{\mathrm{Iso}}\newcommand{\Mor}{\mathrm{Mor}}\newcommand{\Isom}{\mathrm{Isom}}\newcommand{\Aut}{\mathrm{Aut}}\newcommand{\End}{\mathrm{End}}\newcommand{\op}{\mathrm{op}}\newcommand{\ev}{\mathrm{ev}}\newcommand{\Ob}{\mathrm{Ob}}\newcommand{\Ar}{\mathrm{Ar}}\newcommand{\Arr}{\mathrm{Arr}}\newcommand{\Set}{\mathrm{Set}}\newcommand{\Grp}{\mathrm{Grp}}\newcommand{\Cat}{\mathrm{Cat}}\newcommand{\Mon}{\mathrm{Mon}}\newcommand{\Ring}{\mathrm{Ring}}\newcommand{\CRing}{\mathrm{CRing}}\newcommand{\Ab}{\mathrm{Ab}}\newcommand{\Pos}{\mathrm{Pos}}\newcommand{\Vect}{\mathrm{Vect}}\newcommand{\FinVect}{\mathrm{FinVect}}\newcommand{\FinSet}{\mathrm{FinSet}}\newcommand{\FinMeas}{\mathrm{FinMeas}}\newcommand{\OmegaAlg}{\Omega\text{-}\mathrm{Alg}}\newcommand{\OmegaEAlg}{(\Omega,E)\text{-}\mathrm{Alg}}\newcommand{\Fun}{\mathrm{Fun}}\newcommand{\Func}{\mathrm{Func}}\newcommand{\Alg}{\mathrm{Alg}} %代数の圏
\newcommand{\CAlg}{\mathrm{CAlg}} %可換代数の圏
\newcommand{\Met}{\mathrm{Met}} %Metric space & Contraction maps
\newcommand{\Rel}{\mathrm{Rel}} %Sets & relation
\newcommand{\Bool}{\mathrm{Bool}}\newcommand{\CABool}{\mathrm{CABool}}\newcommand{\CompBoolAlg}{\mathrm{CompBoolAlg}}\newcommand{\BoolAlg}{\mathrm{BoolAlg}}\newcommand{\BoolRng}{\mathrm{BoolRng}}\newcommand{\HeytAlg}{\mathrm{HeytAlg}}\newcommand{\CompHeytAlg}{\mathrm{CompHeytAlg}}\newcommand{\Lat}{\mathrm{Lat}}\newcommand{\CompLat}{\mathrm{CompLat}}\newcommand{\SemiLat}{\mathrm{SemiLat}}\newcommand{\Stone}{\mathrm{Stone}}\newcommand{\Mfd}{\mathrm{Mfd}}\newcommand{\LieAlg}{\mathrm{LieAlg}}
\newcommand{\Sob}{\mathrm{Sob}} %Sober space & continuous map
\newcommand{\Op}{\mathrm{Op}} %Category of open subsets
\newcommand{\Sh}{\mathrm{Sh}} %Category of sheave
\newcommand{\PSh}{\mathrm{PSh}} %Category of presheave, PSh(C)=[C^op,set]のこと
\newcommand{\Conv}{\mathrm{Conv}} %Convergence spaceの圏
\newcommand{\Unif}{\mathrm{Unif}} %一様空間と一様連続写像の圏
\newcommand{\Frm}{\mathrm{Frm}} %フレームとフレームの射
\newcommand{\Locale}{\mathrm{Locale}} %その反対圏
\newcommand{\Diff}{\mathrm{Diff}} %滑らかな多様体の圏
\newcommand{\Quiv}{\mathrm{Quiv}} %Quiverの圏
\newcommand{\B}{\mathcal{B}}\newcommand{\Span}{\mathrm{Span}}\newcommand{\Corr}{\mathrm{Corr}}\newcommand{\Decat}{\mathrm{Decat}}\newcommand{\Rep}{\mathrm{Rep}}\newcommand{\Grpd}{\mathrm{Grpd}}\newcommand{\sSet}{\mathrm{sSet}}\newcommand{\Mod}{\mathrm{Mod}}\newcommand{\SmoothMnf}{\mathrm{SmoothMnf}}\newcommand{\coker}{\mathrm{coker}}\newcommand{\Ord}{\mathrm{Ord}}\newcommand{\eq}{\mathrm{eq}}\newcommand{\coeq}{\mathrm{coeq}}\newcommand{\act}{\mathrm{act}}

%%%%%%%%%%%%%%% 定理環境(足助先生ありがとうございます) %%%%%%%%%%%%%%%

\everymath{\displaystyle}
\renewcommand{\proofname}{\bf\underline{[証明]}}
\renewcommand{\thefootnote}{\dag\arabic{footnote}} %足助さんからもらった.どうなるんだ?
\renewcommand{\qedsymbol}{$\blacksquare$}

\renewcommand{\labelenumi}{(\arabic{enumi})} %(1),(2),...がデフォルトであって欲しい
\renewcommand{\labelenumii}{(\alph{enumii})}
\renewcommand{\labelenumiii}{(\roman{enumiii})}

\newtheoremstyle{StatementsWithUnderline}% ?name?
{3pt}% ?Space above? 1
{3pt}% ?Space below? 1
{}% ?Body font?
{}% ?Indent amount? 2
{\bfseries}% ?Theorem head font?
{\textbf{.}}% ?Punctuation after theorem head?
{.5em}% ?Space after theorem head? 3
{\textbf{\underline{\textup{#1~\thetheorem{}}}}\;\thmnote{(#3)}}% ?Theorem head spec (can be left empty, meaning ‘normal’)?

\usepackage{etoolbox}
\AtEndEnvironment{example}{\hfill\ensuremath{\Box}}
\AtEndEnvironment{observation}{\hfill\ensuremath{\Box}}

\theoremstyle{StatementsWithUnderline}
    \newtheorem{theorem}{定理}[section]
    \newtheorem{axiom}[theorem]{公理}
    \newtheorem{corollary}[theorem]{系}
    \newtheorem{proposition}[theorem]{命題}
    \newtheorem{lemma}[theorem]{補題}
    \newtheorem{definition}[theorem]{定義}
    \newtheorem{problem}[theorem]{問題}
    \newtheorem{exercise}[theorem]{Exercise}
\theoremstyle{definition}
    \newtheorem{issue}{論点}
    \newtheorem*{proposition*}{命題}
    \newtheorem*{lemma*}{補題}
    \newtheorem*{consideration*}{考察}
    \newtheorem*{theorem*}{定理}
    \newtheorem*{remarks*}{要諦}
    \newtheorem{example}[theorem]{例}
    \newtheorem{notation}[theorem]{記法}
    \newtheorem*{notation*}{記法}
    \newtheorem{assumption}[theorem]{仮定}
    \newtheorem{question}[theorem]{問}
    \newtheorem{counterexample}[theorem]{反例}
    \newtheorem{reidai}[theorem]{例題}
    \newtheorem{ruidai}[theorem]{類題}
    \newtheorem{algorithm}[theorem]{算譜}
    \newtheorem*{feels*}{所感}
    \newtheorem*{solution*}{\bf{[解]}}
    \newtheorem{discussion}[theorem]{議論}
    \newtheorem{synopsis}[theorem]{要約}
    \newtheorem{cited}[theorem]{引用}
    \newtheorem{remark}[theorem]{注}
    \newtheorem{remarks}[theorem]{要諦}
    \newtheorem{memo}[theorem]{メモ}
    \newtheorem{image}[theorem]{描像}
    \newtheorem{observation}[theorem]{観察}
    \newtheorem{universality}[theorem]{普遍性} %非自明な例外がない.
    \newtheorem{universal tendency}[theorem]{普遍傾向} %例外が有意に少ない.
    \newtheorem{hypothesis}[theorem]{仮説} %実験で説明されていない理論.
    \newtheorem{theory}[theorem]{理論} %実験事実とその(さしあたり)整合的な説明.
    \newtheorem{fact}[theorem]{実験事実}
    \newtheorem{model}[theorem]{模型}
    \newtheorem{explanation}[theorem]{説明} %理論による実験事実の説明
    \newtheorem{anomaly}[theorem]{理論の限界}
    \newtheorem{application}[theorem]{応用例}
    \newtheorem{method}[theorem]{手法} %実験手法など,技術的問題.
    \newtheorem{test}[theorem]{検定}
    \newtheorem{terms}[theorem]{用語}
    \newtheorem{solution}[theorem]{解法}
    \newtheorem{history}[theorem]{歴史}
    \newtheorem{usage}[theorem]{用語法}
    \newtheorem{research}[theorem]{研究}
    \newtheorem{shishin}[theorem]{指針}
    \newtheorem{yodan}[theorem]{余談}
    \newtheorem{construction}[theorem]{構成}
    \newtheorem{motivation}[theorem]{動機}
    \newtheorem{context}[theorem]{背景}
    \newtheorem{advantage}[theorem]{利点}
    \newtheorem*{definition*}{定義}
    \newtheorem*{remark*}{注意}
    \newtheorem*{question*}{問}
    \newtheorem*{problem*}{問題}
    \newtheorem*{axiom*}{公理}
    \newtheorem*{example*}{例}
    \newtheorem*{corollary*}{系}
    \newtheorem*{shishin*}{指針}
    \newtheorem*{yodan*}{余談}
    \newtheorem*{kadai*}{課題}

\raggedbottom
\allowdisplaybreaks
%%%%%%%%%%%%%%%% 数理文書の組版 %%%%%%%%%%%%%%%

\usepackage{mathtools} %内部でamsmathを呼び出すことに注意.
%\mathtoolsset{showonlyrefs=true} %labelを附した数式にのみ附番される設定.
\usepackage{amsfonts} %mathfrak, mathcal, mathbbなど.
\usepackage{amsthm} %定理環境.
\usepackage{amssymb} %AMSFontsを使うためのパッケージ.
\usepackage{ascmac} %screen, itembox, shadebox環境.全てLATEX2εの標準機能の範囲で作られたもの.
\usepackage{comment} %comment環境を用いて,複数行をcomment outできるようにするpackage
\usepackage{wrapfig} %図の周りに文字をwrapさせることができる.詳細な制御ができる.
\usepackage[usenames, dvipsnames]{xcolor} %xcolorはcolorの拡張.optionの意味はdvipsnamesはLoad a set of predefined colors. forestgreenなどの色が追加されている.usenamesはobsoleteとだけ書いてあった.
\setcounter{tocdepth}{2} %目次に表示される深さ.2はsubsectionまで
\usepackage{multicol} %\begin{multicols}{2}環境で途中からmulticolumnに出来る.
\usepackage{mathabx}\newcommand{\wc}{\widecheck} %\widecheckなどのフォントパッケージ

%%%%%%%%%%%%%%% フォント %%%%%%%%%%%%%%%

\usepackage{textcomp, mathcomp} %Text Companionとは,T1 encodingに入らなかった文字群.これを使うためのパッケージ.\textsectionでブルバキに!
\usepackage[T1]{fontenc} %8bitエンコーディングにする.comp系拡張数学文字の動作が安定する.

%%%%%%%%%%%%%%% 一般文書の組版 %%%%%%%%%%%%%%%

\definecolor{花緑青}{cmyk}{1,0.07,0.10,0.10}\definecolor{サーモンピンク}{cmyk}{0,0.65,0.65,0.05}\definecolor{暗中模索}{rgb}{0.2,0.2,0.2}
\usepackage{url}\usepackage[dvipdfmx,colorlinks,linkcolor=花緑青,urlcolor=花緑青,citecolor=花緑青]{hyperref} %生成されるPDFファイルにおいて、\tableofcontentsによって書き出された目次をクリックすると該当する見出しへジャンプしたり、さらには、\label{ラベル名}を番号で参照する\ref{ラベル名}やthebibliography環境において\bibitem{ラベル名}を文献番号で参照する\cite{ラベル名}においても番号をクリックすると該当箇所にジャンプする.囲み枠はダサいので,colorlinksで囲み廃止し,リンク自体に色を付けることにした.
\usepackage{pxjahyper} %pxrubrica同様,八登崇之さん.hyperrefは日本語pLaTeXに最適化されていないから,hyperrefとセットで,(u)pLaTeX+hyperref+dvipdfmxの組み合わせで日本語を含む「しおり」をもつPDF文書を作成する場合に必要となる機能を提供する
\usepackage{ulem} %取り消し線を引くためのパッケージ
\usepackage{pxrubrica} %日本語にルビをふる.八登崇之(やとうたかゆき)氏による.

%%%%%%%%%%%%%%% 科学文書の組版 %%%%%%%%%%%%%%%

\usepackage[version=4]{mhchem} %化学式をTikZで簡単に書くためのパッケージ.
\usepackage{chemfig} %化学構造式をTikZで描くためのパッケージ.
\usepackage{siunitx} %IS単位を書くためのパッケージ

%%%%%%%%%%%%%%% 作図 %%%%%%%%%%%%%%%

\usepackage{tikz}\usetikzlibrary{positioning,automata}\usepackage{tikz-cd}\usepackage[all]{xy}
\def\objectstyle{\displaystyle} %デフォルトではxymatrix中の数式が文中数式モードになるので,それを直す.\labelstyleも同様にxy packageの中で定義されており,文中数式モードになっている.

\usepackage{graphicx} %rotatebox, scalebox, reflectbox, resizeboxなどのコマンドや,図表の読み込み\includegraphicsを司る.graphics というパッケージもありますが,graphicx はこれを高機能にしたものと考えて結構です(ただし graphicx は内部で graphics を読み込みます)
\usepackage[top=15truemm,bottom=15truemm,left=10truemm,right=10truemm]{geometry} %足助さんからもらったオプション

%%%%%%%%%%%%%%% 参照 %%%%%%%%%%%%%%%
%参考文献リストを出力したい箇所に\bibliography{../mathematics.bib}を追記すると良い.

%\bibliographystyle{jplain}
%\bibliographystyle{jname}
\bibliographystyle{apalike}

%%%%%%%%%%%%%%% 計算機文書の組版 %%%%%%%%%%%%%%%

\usepackage[breakable]{tcolorbox} %加藤晃史さんがフル活用していたtcolorboxを,途中改ページ可能で.
\tcbuselibrary{theorems} %https://qiita.com/t_kemmochi/items/483b8fcdb5db8d1f5d5e
\usepackage{enumerate} %enumerate環境を凝らせる.

\usepackage{listings} %ソースコードを表示できる環境.多分もっといい方法ある.
\usepackage{jvlisting} %日本語のコメントアウトをする場合jlistingが必要
\lstset{ %ここからソースコードの表示に関する設定.lstlisting環境では,[caption=hoge,label=fuga]などのoptionを付けられる.
%[escapechar=!]とすると,LaTeXコマンドを使える.
  basicstyle={\ttfamily},
  identifierstyle={\small},
  commentstyle={\smallitshape},
  keywordstyle={\small\bfseries},
  ndkeywordstyle={\small},
  stringstyle={\small\ttfamily},
  frame={tb},
  breaklines=true,
  columns=[l]{fullflexible},
  numbers=left,
  xrightmargin=0zw,
  xleftmargin=3zw,
  numberstyle={\scriptsize},
  stepnumber=1,
  numbersep=1zw,
  lineskip=-0.5ex
}
%\makeatletter %caption番号を「[chapter番号].[section番号].[subsection番号]-[そのsubsection内においてn番目]」に変更
%    \AtBeginDocument{
%    \renewcommand*{\thelstlisting}{\arabic{chapter}.\arabic{section}.\arabic{lstlisting}}
%    \@addtoreset{lstlisting}{section}
%    }
%\makeatother
\renewcommand{\lstlistingname}{算譜} %caption名を"program"に変更

\newtcolorbox{tbox}[3][]{%
colframe=#2,colback=#2!10,coltitle=#2!20!black,title={#3},#1}

% 証明内の文字が小さくなる環境.
\newenvironment{Proof}[1][\bf\underline{[証明]}]{\proof[#1]\color{darkgray}}{\endproof}

%%%%%%%%%%%%%%% 数学記号のマクロ %%%%%%%%%%%%%%%

%%% 括弧類
\newcommand{\abs}[1]{\lvert#1\rvert}\newcommand{\Abs}[1]{\left|#1\right|}\newcommand{\norm}[1]{\|#1\|}\newcommand{\Norm}[1]{\left\|#1\right\|}\newcommand{\Brace}[1]{\left\{#1\right\}}\newcommand{\BRace}[1]{\biggl\{#1\biggr\}}\newcommand{\paren}[1]{\left(#1\right)}\newcommand{\Paren}[1]{\biggr(#1\biggl)}\newcommand{\bracket}[1]{\langle#1\rangle}\newcommand{\brac}[1]{\langle#1\rangle}\newcommand{\Bracket}[1]{\left\langle#1\right\rangle}\newcommand{\Brac}[1]{\left\langle#1\right\rangle}\newcommand{\bra}[1]{\left\langle#1\right|}\newcommand{\ket}[1]{\left|#1\right\rangle}\newcommand{\Square}[1]{\left[#1\right]}\newcommand{\SQuare}[1]{\biggl[#1\biggr]}
\renewcommand{\o}[1]{\overline{#1}}\renewcommand{\u}[1]{\underline{#1}}\newcommand{\wt}[1]{\widetilde{#1}}\newcommand{\wh}[1]{\widehat{#1}}
\newcommand{\pp}[2]{\frac{\partial #1}{\partial #2}}\newcommand{\ppp}[3]{\frac{\partial #1}{\partial #2\partial #3}}\newcommand{\dd}[2]{\frac{d #1}{d #2}}
\newcommand{\floor}[1]{\lfloor#1\rfloor}\newcommand{\Floor}[1]{\left\lfloor#1\right\rfloor}\newcommand{\ceil}[1]{\lceil#1\rceil}
\newcommand{\ocinterval}[1]{(#1]}\newcommand{\cointerval}[1]{[#1)}\newcommand{\COinterval}[1]{\left[#1\right)}


%%% 予約語
\renewcommand{\iff}{\;\mathrm{iff}\;}
\newcommand{\False}{\mathrm{False}}\newcommand{\True}{\mathrm{True}}
\newcommand{\otherwise}{\mathrm{otherwise}}
\newcommand{\st}{\;\mathrm{s.t.}\;}

%%% 略記
\newcommand{\M}{\mathcal{M}}\newcommand{\cF}{\mathcal{F}}\newcommand{\cD}{\mathcal{D}}\newcommand{\fX}{\mathfrak{X}}\newcommand{\fY}{\mathfrak{Y}}\newcommand{\fZ}{\mathfrak{Z}}\renewcommand{\H}{\mathcal{H}}\newcommand{\fH}{\mathfrak{H}}\newcommand{\bH}{\mathbb{H}}\newcommand{\id}{\mathrm{id}}\newcommand{\A}{\mathcal{A}}\newcommand{\U}{\mathfrak{U}}
\newcommand{\lmd}{\lambda}
\newcommand{\Lmd}{\Lambda}

%%% 矢印類
\newcommand{\iso}{\xrightarrow{\,\smash{\raisebox{-0.45ex}{\ensuremath{\scriptstyle\sim}}}\,}}
\newcommand{\Lrarrow}{\;\;\Leftrightarrow\;\;}

%%% 注記
\newcommand{\rednote}[1]{\textcolor{red}{#1}}

% ノルム位相についての閉包 https://newbedev.com/how-to-make-double-overline-with-less-vertical-displacement
\makeatletter
\newcommand{\dbloverline}[1]{\overline{\dbl@overline{#1}}}
\newcommand{\dbl@overline}[1]{\mathpalette\dbl@@overline{#1}}
\newcommand{\dbl@@overline}[2]{%
  \begingroup
  \sbox\z@{$\m@th#1\overline{#2}$}%
  \ht\z@=\dimexpr\ht\z@-2\dbl@adjust{#1}\relax
  \box\z@
  \ifx#1\scriptstyle\kern-\scriptspace\else
  \ifx#1\scriptscriptstyle\kern-\scriptspace\fi\fi
  \endgroup
}
\newcommand{\dbl@adjust}[1]{%
  \fontdimen8
  \ifx#1\displaystyle\textfont\else
  \ifx#1\textstyle\textfont\else
  \ifx#1\scriptstyle\scriptfont\else
  \scriptscriptfont\fi\fi\fi 3
}
\makeatother
\newcommand{\oo}[1]{\dbloverline{#1}}

% hslashの他の文字Ver.
\newcommand{\hslashslash}{%
    \scalebox{1.2}{--
    }%
}
\newcommand{\dslash}{%
  {%
    \vphantom{d}%
    \ooalign{\kern.05em\smash{\hslashslash}\hidewidth\cr$d$\cr}%
    \kern.05em
  }%
}
\newcommand{\dint}{%
  {%
    \vphantom{d}%
    \ooalign{\kern.05em\smash{\hslashslash}\hidewidth\cr$\int$\cr}%
    \kern.05em
  }%
}
\newcommand{\dL}{%
  {%
    \vphantom{d}%
    \ooalign{\kern.05em\smash{\hslashslash}\hidewidth\cr$L$\cr}%
    \kern.05em
  }%
}

%%% 演算子
\DeclareMathOperator{\grad}{\mathrm{grad}}\DeclareMathOperator{\rot}{\mathrm{rot}}\DeclareMathOperator{\divergence}{\mathrm{div}}\DeclareMathOperator{\tr}{\mathrm{tr}}\newcommand{\pr}{\mathrm{pr}}
\newcommand{\Map}{\mathrm{Map}}\newcommand{\dom}{\mathrm{Dom}\;}\newcommand{\cod}{\mathrm{Cod}\;}\newcommand{\supp}{\mathrm{supp}\;}


%%% 線型代数学
\newcommand{\vctr}[2]{\begin{pmatrix}#1\\#2\end{pmatrix}}\newcommand{\vctrr}[3]{\begin{pmatrix}#1\\#2\\#3\end{pmatrix}}\newcommand{\mtrx}[4]{\begin{pmatrix}#1&#2\\#3&#4\end{pmatrix}}\newcommand{\smtrx}[4]{\paren{\begin{smallmatrix}#1&#2\\#3&#4\end{smallmatrix}}}\newcommand{\Ker}{\mathrm{Ker}\;}\newcommand{\Coker}{\mathrm{Coker}\;}\newcommand{\Coim}{\mathrm{Coim}\;}\DeclareMathOperator{\rank}{\mathrm{rank}}\newcommand{\lcm}{\mathrm{lcm}}\newcommand{\sgn}{\mathrm{sgn}\,}\newcommand{\GL}{\mathrm{GL}}\newcommand{\SL}{\mathrm{SL}}\newcommand{\alt}{\mathrm{alt}}
%%% 複素解析学
\renewcommand{\Re}{\mathrm{Re}\;}\renewcommand{\Im}{\mathrm{Im}\;}\newcommand{\Gal}{\mathrm{Gal}}\newcommand{\PGL}{\mathrm{PGL}}\newcommand{\PSL}{\mathrm{PSL}}\newcommand{\Log}{\mathrm{Log}\,}\newcommand{\Res}{\mathrm{Res}\,}\newcommand{\on}{\mathrm{on}\;}\newcommand{\hatC}{\widehat{\C}}\newcommand{\hatR}{\hat{\R}}\newcommand{\PV}{\mathrm{P.V.}}\newcommand{\diam}{\mathrm{diam}}\newcommand{\Area}{\mathrm{Area}}\newcommand{\Lap}{\Laplace}\newcommand{\f}{\mathbf{f}}\newcommand{\cR}{\mathcal{R}}\newcommand{\const}{\mathrm{const.}}\newcommand{\Om}{\Omega}\newcommand{\Cinf}{C^\infty}\newcommand{\ep}{\epsilon}\newcommand{\dist}{\mathrm{dist}}\newcommand{\opart}{\o{\partial}}\newcommand{\Length}{\mathrm{Length}}
%%% 集合と位相
\renewcommand{\O}{\mathcal{O}}\renewcommand{\S}{\mathcal{S}}\renewcommand{\U}{\mathcal{U}}\newcommand{\V}{\mathcal{V}}\renewcommand{\P}{\mathcal{P}}\newcommand{\R}{\mathbb{R}}\newcommand{\N}{\mathbb{N}}\newcommand{\C}{\mathbb{C}}\newcommand{\Z}{\mathbb{Z}}\newcommand{\Q}{\mathbb{Q}}\newcommand{\TV}{\mathrm{TV}}\newcommand{\ORD}{\mathrm{ORD}}\newcommand{\Tr}{\mathrm{Tr}}\newcommand{\Card}{\mathrm{Card}\;}\newcommand{\Top}{\mathrm{Top}}\newcommand{\Disc}{\mathrm{Disc}}\newcommand{\Codisc}{\mathrm{Codisc}}\newcommand{\CoDisc}{\mathrm{CoDisc}}\newcommand{\Ult}{\mathrm{Ult}}\newcommand{\ord}{\mathrm{ord}}\newcommand{\maj}{\mathrm{maj}}\newcommand{\bS}{\mathbb{S}}\newcommand{\PConn}{\mathrm{PConn}}

%%% 形式言語理論
\newcommand{\REGEX}{\mathrm{REGEX}}\newcommand{\RE}{\mathbf{RE}}
%%% Graph Theory
\newcommand{\SimpGph}{\mathrm{SimpGph}}\newcommand{\Gph}{\mathrm{Gph}}\newcommand{\mult}{\mathrm{mult}}\newcommand{\inv}{\mathrm{inv}}

%%% 多様体
\newcommand{\Der}{\mathrm{Der}}\newcommand{\osub}{\overset{\mathrm{open}}{\subset}}\newcommand{\osup}{\overset{\mathrm{open}}{\supset}}\newcommand{\al}{\alpha}\newcommand{\K}{\mathbb{K}}\newcommand{\Sp}{\mathrm{Sp}}\newcommand{\g}{\mathfrak{g}}\newcommand{\h}{\mathfrak{h}}\newcommand{\Exp}{\mathrm{Exp}\;}\newcommand{\Imm}{\mathrm{Imm}}\newcommand{\Imb}{\mathrm{Imb}}\newcommand{\codim}{\mathrm{codim}\;}\newcommand{\Gr}{\mathrm{Gr}}
%%% 代数
\newcommand{\Ad}{\mathrm{Ad}}\newcommand{\finsupp}{\mathrm{fin\;supp}}\newcommand{\SO}{\mathrm{SO}}\newcommand{\SU}{\mathrm{SU}}\newcommand{\acts}{\curvearrowright}\newcommand{\mono}{\hookrightarrow}\newcommand{\epi}{\twoheadrightarrow}\newcommand{\Stab}{\mathrm{Stab}}\newcommand{\nor}{\mathrm{nor}}\newcommand{\T}{\mathbb{T}}\newcommand{\Aff}{\mathrm{Aff}}\newcommand{\rsub}{\triangleleft}\newcommand{\rsup}{\triangleright}\newcommand{\subgrp}{\overset{\mathrm{subgrp}}{\subset}}\newcommand{\Ext}{\mathrm{Ext}}\newcommand{\sbs}{\subset}\newcommand{\sps}{\supset}\newcommand{\In}{\mathrm{in}\;}\newcommand{\Tor}{\mathrm{Tor}}\newcommand{\p}{\b{p}}\newcommand{\q}{\mathfrak{q}}\newcommand{\m}{\mathfrak{m}}\newcommand{\cS}{\mathcal{S}}\newcommand{\Frac}{\mathrm{Frac}\,}\newcommand{\Spec}{\mathrm{Spec}\,}\newcommand{\bA}{\mathbb{A}}\newcommand{\Sym}{\mathrm{Sym}}\newcommand{\Ann}{\mathrm{Ann}}\newcommand{\Her}{\mathrm{Her}}\newcommand{\Bil}{\mathrm{Bil}}\newcommand{\Ses}{\mathrm{Ses}}\newcommand{\FVS}{\mathrm{FVS}}
%%% 代数的位相幾何学
\newcommand{\Ho}{\mathrm{Ho}}\newcommand{\CW}{\mathrm{CW}}\newcommand{\lc}{\mathrm{lc}}\newcommand{\cg}{\mathrm{cg}}\newcommand{\Fib}{\mathrm{Fib}}\newcommand{\Cyl}{\mathrm{Cyl}}\newcommand{\Ch}{\mathrm{Ch}}
%%% 微分幾何学
\newcommand{\rE}{\mathrm{E}}\newcommand{\e}{\b{e}}\renewcommand{\k}{\b{k}}\newcommand{\Christ}[2]{\begin{Bmatrix}#1\\#2\end{Bmatrix}}\renewcommand{\Vec}[1]{\overrightarrow{\mathrm{#1}}}\newcommand{\hen}[1]{\mathrm{#1}}\renewcommand{\b}[1]{\boldsymbol{#1}}

%%% 函数解析
\newcommand{\HS}{\mathrm{HS}}\newcommand{\loc}{\mathrm{loc}}\newcommand{\Lh}{\mathrm{L.h.}}\newcommand{\Epi}{\mathrm{Epi}\;}\newcommand{\slim}{\mathrm{slim}}\newcommand{\Ban}{\mathrm{Ban}}\newcommand{\Hilb}{\mathrm{Hilb}}\newcommand{\Ex}{\mathrm{Ex}}\newcommand{\Co}{\mathrm{Co}}\newcommand{\sa}{\mathrm{sa}}\newcommand{\nnorm}[1]{{\left\vert\kern-0.25ex\left\vert\kern-0.25ex\left\vert #1 \right\vert\kern-0.25ex\right\vert\kern-0.25ex\right\vert}}\newcommand{\dvol}{\mathrm{dvol}}\newcommand{\Sconv}{\mathrm{Sconv}}\newcommand{\I}{\mathcal{I}}\newcommand{\nonunital}{\mathrm{nu}}\newcommand{\cpt}{\mathrm{cpt}}\newcommand{\lcpt}{\mathrm{lcpt}}\newcommand{\com}{\mathrm{com}}\newcommand{\Haus}{\mathrm{Haus}}\newcommand{\proper}{\mathrm{proper}}\newcommand{\infinity}{\mathrm{inf}}\newcommand{\TVS}{\mathrm{TVS}}\newcommand{\ess}{\mathrm{ess}}\newcommand{\ext}{\mathrm{ext}}\newcommand{\Index}{\mathrm{Index}\;}\newcommand{\SSR}{\mathrm{SSR}}\newcommand{\vs}{\mathrm{vs.}}\newcommand{\fM}{\mathfrak{M}}\newcommand{\EDM}{\mathrm{EDM}}\newcommand{\Tw}{\mathrm{Tw}}\newcommand{\fC}{\mathfrak{C}}\newcommand{\bn}{\boldsymbol{n}}\newcommand{\br}{\boldsymbol{r}}\newcommand{\Lam}{\Lambda}\newcommand{\lam}{\lambda}\newcommand{\one}{\mathbf{1}}\newcommand{\dae}{\text{-a.e.}}\newcommand{\das}{\text{-a.s.}}\newcommand{\td}{\text{-}}\newcommand{\RM}{\mathrm{RM}}\newcommand{\BV}{\mathrm{BV}}\newcommand{\normal}{\mathrm{normal}}\newcommand{\lub}{\mathrm{lub}\;}\newcommand{\Graph}{\mathrm{Graph}}\newcommand{\Ascent}{\mathrm{Ascent}}\newcommand{\Descent}{\mathrm{Descent}}\newcommand{\BIL}{\mathrm{BIL}}\newcommand{\fL}{\mathfrak{L}}\newcommand{\De}{\Delta}
%%% 積分論
\newcommand{\calA}{\mathcal{A}}\newcommand{\calB}{\mathcal{B}}\newcommand{\D}{\mathcal{D}}\newcommand{\Y}{\mathcal{Y}}\newcommand{\calC}{\mathcal{C}}\renewcommand{\ae}{\mathrm{a.e.}\;}\newcommand{\cZ}{\mathcal{Z}}\newcommand{\fF}{\mathfrak{F}}\newcommand{\fI}{\mathfrak{I}}\newcommand{\E}{\mathcal{E}}\newcommand{\sMap}{\sigma\textrm{-}\mathrm{Map}}\DeclareMathOperator*{\argmax}{arg\,max}\DeclareMathOperator*{\argmin}{arg\,min}\newcommand{\cC}{\mathcal{C}}\newcommand{\comp}{\complement}\newcommand{\J}{\mathcal{J}}\newcommand{\sumN}[1]{\sum_{#1\in\N}}\newcommand{\cupN}[1]{\cup_{#1\in\N}}\newcommand{\capN}[1]{\cap_{#1\in\N}}\newcommand{\Sum}[1]{\sum_{#1=1}^\infty}\newcommand{\sumn}{\sum_{n=1}^\infty}\newcommand{\summ}{\sum_{m=1}^\infty}\newcommand{\sumk}{\sum_{k=1}^\infty}\newcommand{\sumi}{\sum_{i=1}^\infty}\newcommand{\sumj}{\sum_{j=1}^\infty}\newcommand{\cupn}{\cup_{n=1}^\infty}\newcommand{\capn}{\cap_{n=1}^\infty}\newcommand{\cupk}{\cup_{k=1}^\infty}\newcommand{\cupi}{\cup_{i=1}^\infty}\newcommand{\cupj}{\cup_{j=1}^\infty}\newcommand{\limn}{\lim_{n\to\infty}}\renewcommand{\l}{\mathcal{l}}\renewcommand{\L}{\mathcal{L}}\newcommand{\Cl}{\mathrm{Cl}}\newcommand{\cN}{\mathcal{N}}\newcommand{\Ae}{\textrm{-a.e.}\;}\newcommand{\csub}{\overset{\textrm{closed}}{\subset}}\newcommand{\csup}{\overset{\textrm{closed}}{\supset}}\newcommand{\wB}{\wt{B}}\newcommand{\cG}{\mathcal{G}}\newcommand{\Lip}{\mathrm{Lip}}\DeclareMathOperator{\Dom}{\mathrm{Dom}}\newcommand{\AC}{\mathrm{AC}}\newcommand{\Mol}{\mathrm{Mol}}
%%% Fourier解析
\newcommand{\Pe}{\mathrm{Pe}}\newcommand{\wR}{\wh{\mathbb{\R}}}\newcommand*{\Laplace}{\mathop{}\!\mathbin\bigtriangleup}\newcommand*{\DAlambert}{\mathop{}\!\mathbin\Box}\newcommand{\bT}{\mathbb{T}}\newcommand{\dx}{\dslash x}\newcommand{\dt}{\dslash t}\newcommand{\ds}{\dslash s}
%%% 数値解析
\newcommand{\round}{\mathrm{round}}\newcommand{\cond}{\mathrm{cond}}\newcommand{\diag}{\mathrm{diag}}
\newcommand{\Adj}{\mathrm{Adj}}\newcommand{\Pf}{\mathrm{Pf}}\newcommand{\Sg}{\mathrm{Sg}}

%%% 確率論
\newcommand{\Prob}{\mathrm{Prob}}\newcommand{\X}{\mathcal{X}}\newcommand{\Meas}{\mathrm{Meas}}\newcommand{\as}{\;\mathrm{a.s.}}\newcommand{\io}{\;\mathrm{i.o.}}\newcommand{\fe}{\;\mathrm{f.e.}}\newcommand{\F}{\mathcal{F}}\newcommand{\bF}{\mathbb{F}}\newcommand{\W}{\mathcal{W}}\newcommand{\Pois}{\mathrm{Pois}}\newcommand{\iid}{\mathrm{i.i.d.}}\newcommand{\wconv}{\rightsquigarrow}\newcommand{\Var}{\mathrm{Var}}\newcommand{\xrightarrown}{\xrightarrow{n\to\infty}}\newcommand{\au}{\mathrm{au}}\newcommand{\cT}{\mathcal{T}}\newcommand{\wto}{\overset{w}{\to}}\newcommand{\dto}{\overset{d}{\to}}\newcommand{\pto}{\overset{p}{\to}}\newcommand{\vto}{\overset{v}{\to}}\newcommand{\Cont}{\mathrm{Cont}}\newcommand{\stably}{\mathrm{stably}}\newcommand{\Np}{\mathbb{N}^+}\newcommand{\oM}{\overline{\mathcal{M}}}\newcommand{\fP}{\mathfrak{P}}\newcommand{\sign}{\mathrm{sign}}\DeclareMathOperator{\Div}{Div}
\newcommand{\bD}{\mathbb{D}}\newcommand{\fW}{\mathfrak{W}}\newcommand{\DL}{\mathcal{D}\mathcal{L}}\renewcommand{\r}[1]{\mathrm{#1}}\newcommand{\rC}{\mathrm{C}}
%%% 情報理論
\newcommand{\bit}{\mathrm{bit}}\DeclareMathOperator{\sinc}{sinc}
%%% 量子論
\newcommand{\err}{\mathrm{err}}
%%% 最適化
\newcommand{\varparallel}{\mathbin{\!/\mkern-5mu/\!}}\newcommand{\Minimize}{\text{Minimize}}\newcommand{\subjectto}{\text{subject to}}\newcommand{\Ri}{\mathrm{Ri}}\newcommand{\Cone}{\mathrm{Cone}}\newcommand{\Int}{\mathrm{Int}}
%%% 数理ファイナンス
\newcommand{\pre}{\mathrm{pre}}\newcommand{\om}{\omega}

%%% 偏微分方程式
\let\div\relax
\DeclareMathOperator{\div}{div}\newcommand{\del}{\partial}
\newcommand{\LHS}{\mathrm{LHS}}\newcommand{\RHS}{\mathrm{RHS}}\newcommand{\bnu}{\boldsymbol{\nu}}\newcommand{\interior}{\mathrm{in}\;}\newcommand{\SH}{\mathrm{SH}}\renewcommand{\v}{\boldsymbol{\nu}}\newcommand{\n}{\mathbf{n}}\newcommand{\ssub}{\Subset}\newcommand{\curl}{\mathrm{curl}}
%%% 常微分方程式
\newcommand{\Ei}{\mathrm{Ei}}\newcommand{\sn}{\mathrm{sn}}\newcommand{\wgamma}{\widetilde{\gamma}}
%%% 統計力学
\newcommand{\Ens}{\mathrm{Ens}}
%%% 解析力学
\newcommand{\cl}{\mathrm{cl}}\newcommand{\x}{\boldsymbol{x}}

%%% 統計的因果推論
\newcommand{\Do}{\mathrm{Do}}
%%% 応用統計学
\newcommand{\mrl}{\mathrm{mrl}}
%%% 数理統計
\newcommand{\comb}[2]{\begin{pmatrix}#1\\#2\end{pmatrix}}\newcommand{\bP}{\mathbb{P}}\newcommand{\compsub}{\overset{\textrm{cpt}}{\subset}}\newcommand{\lip}{\textrm{lip}}\newcommand{\BL}{\mathrm{BL}}\newcommand{\G}{\mathbb{G}}\newcommand{\NB}{\mathrm{NB}}\newcommand{\oR}{\o{\R}}\newcommand{\liminfn}{\liminf_{n\to\infty}}\newcommand{\limsupn}{\limsup_{n\to\infty}}\newcommand{\esssup}{\mathrm{ess.sup}}\newcommand{\asto}{\xrightarrow{\as}}\newcommand{\Cov}{\mathrm{Cov}}\newcommand{\cQ}{\mathcal{Q}}\newcommand{\VC}{\mathrm{VC}}\newcommand{\mb}{\mathrm{mb}}\newcommand{\Avar}{\mathrm{Avar}}\newcommand{\bB}{\mathbb{B}}\newcommand{\bW}{\mathbb{W}}\newcommand{\sd}{\mathrm{sd}}\newcommand{\w}[1]{\widehat{#1}}\newcommand{\bZ}{\boldsymbol{Z}}\newcommand{\Bernoulli}{\mathrm{Ber}}\newcommand{\Ber}{\mathrm{Ber}}\newcommand{\Mult}{\mathrm{Mult}}\newcommand{\BPois}{\mathrm{BPois}}\newcommand{\fraks}{\mathfrak{s}}\newcommand{\frakk}{\mathfrak{k}}\newcommand{\IF}{\mathrm{IF}}\newcommand{\bX}{\mathbf{X}}\newcommand{\bx}{\boldsymbol{x}}\newcommand{\indep}{\raisebox{0.05em}{\rotatebox[origin=c]{90}{$\models$}}}\newcommand{\IG}{\mathrm{IG}}\newcommand{\Levy}{\mathrm{Levy}}\newcommand{\MP}{\mathrm{MP}}\newcommand{\Hermite}{\mathrm{Hermite}}\newcommand{\Skellam}{\mathrm{Skellam}}\newcommand{\Dirichlet}{\mathrm{Dirichlet}}\newcommand{\Beta}{\mathrm{Beta}}\newcommand{\bE}{\mathbb{E}}\newcommand{\bG}{\mathbb{G}}\newcommand{\MISE}{\mathrm{MISE}}\newcommand{\logit}{\mathtt{logit}}\newcommand{\expit}{\mathtt{expit}}\newcommand{\cK}{\mathcal{K}}\newcommand{\dl}{\dot{l}}\newcommand{\dotp}{\dot{p}}\newcommand{\wl}{\wt{l}}\newcommand{\Gauss}{\mathrm{Gauss}}\newcommand{\fA}{\mathfrak{A}}\newcommand{\under}{\mathrm{under}\;}\newcommand{\whtheta}{\wh{\theta}}\newcommand{\Em}{\mathrm{Em}}\newcommand{\ztheta}{{\theta_0}}
\newcommand{\rO}{\mathrm{O}}\newcommand{\Bin}{\mathrm{Bin}}\newcommand{\rW}{\mathrm{W}}\newcommand{\rG}{\mathrm{G}}\newcommand{\rB}{\mathrm{B}}\newcommand{\rN}{\mathrm{N}}\newcommand{\rU}{\mathrm{U}}\newcommand{\HG}{\mathrm{HG}}\newcommand{\GAMMA}{\mathrm{Gamma}}\newcommand{\Cauchy}{\mathrm{Cauchy}}\newcommand{\rt}{\mathrm{t}}
\DeclareMathOperator{\erf}{erf}

%%% 圏
\newcommand{\varlim}{\varprojlim}\newcommand{\Hom}{\mathrm{Hom}}\newcommand{\Iso}{\mathrm{Iso}}\newcommand{\Mor}{\mathrm{Mor}}\newcommand{\Isom}{\mathrm{Isom}}\newcommand{\Aut}{\mathrm{Aut}}\newcommand{\End}{\mathrm{End}}\newcommand{\op}{\mathrm{op}}\newcommand{\ev}{\mathrm{ev}}\newcommand{\Ob}{\mathrm{Ob}}\newcommand{\Ar}{\mathrm{Ar}}\newcommand{\Arr}{\mathrm{Arr}}\newcommand{\Set}{\mathrm{Set}}\newcommand{\Grp}{\mathrm{Grp}}\newcommand{\Cat}{\mathrm{Cat}}\newcommand{\Mon}{\mathrm{Mon}}\newcommand{\Ring}{\mathrm{Ring}}\newcommand{\CRing}{\mathrm{CRing}}\newcommand{\Ab}{\mathrm{Ab}}\newcommand{\Pos}{\mathrm{Pos}}\newcommand{\Vect}{\mathrm{Vect}}\newcommand{\FinVect}{\mathrm{FinVect}}\newcommand{\FinSet}{\mathrm{FinSet}}\newcommand{\FinMeas}{\mathrm{FinMeas}}\newcommand{\OmegaAlg}{\Omega\text{-}\mathrm{Alg}}\newcommand{\OmegaEAlg}{(\Omega,E)\text{-}\mathrm{Alg}}\newcommand{\Fun}{\mathrm{Fun}}\newcommand{\Func}{\mathrm{Func}}\newcommand{\Alg}{\mathrm{Alg}} %代数の圏
\newcommand{\CAlg}{\mathrm{CAlg}} %可換代数の圏
\newcommand{\Met}{\mathrm{Met}} %Metric space & Contraction maps
\newcommand{\Rel}{\mathrm{Rel}} %Sets & relation
\newcommand{\Bool}{\mathrm{Bool}}\newcommand{\CABool}{\mathrm{CABool}}\newcommand{\CompBoolAlg}{\mathrm{CompBoolAlg}}\newcommand{\BoolAlg}{\mathrm{BoolAlg}}\newcommand{\BoolRng}{\mathrm{BoolRng}}\newcommand{\HeytAlg}{\mathrm{HeytAlg}}\newcommand{\CompHeytAlg}{\mathrm{CompHeytAlg}}\newcommand{\Lat}{\mathrm{Lat}}\newcommand{\CompLat}{\mathrm{CompLat}}\newcommand{\SemiLat}{\mathrm{SemiLat}}\newcommand{\Stone}{\mathrm{Stone}}\newcommand{\Mfd}{\mathrm{Mfd}}\newcommand{\LieAlg}{\mathrm{LieAlg}}
\newcommand{\Sob}{\mathrm{Sob}} %Sober space & continuous map
\newcommand{\Op}{\mathrm{Op}} %Category of open subsets
\newcommand{\Sh}{\mathrm{Sh}} %Category of sheave
\newcommand{\PSh}{\mathrm{PSh}} %Category of presheave, PSh(C)=[C^op,set]のこと
\newcommand{\Conv}{\mathrm{Conv}} %Convergence spaceの圏
\newcommand{\Unif}{\mathrm{Unif}} %一様空間と一様連続写像の圏
\newcommand{\Frm}{\mathrm{Frm}} %フレームとフレームの射
\newcommand{\Locale}{\mathrm{Locale}} %その反対圏
\newcommand{\Diff}{\mathrm{Diff}} %滑らかな多様体の圏
\newcommand{\Quiv}{\mathrm{Quiv}} %Quiverの圏
\newcommand{\B}{\mathcal{B}}\newcommand{\Span}{\mathrm{Span}}\newcommand{\Corr}{\mathrm{Corr}}\newcommand{\Decat}{\mathrm{Decat}}\newcommand{\Rep}{\mathrm{Rep}}\newcommand{\Grpd}{\mathrm{Grpd}}\newcommand{\sSet}{\mathrm{sSet}}\newcommand{\Mod}{\mathrm{Mod}}\newcommand{\SmoothMnf}{\mathrm{SmoothMnf}}\newcommand{\coker}{\mathrm{coker}}\newcommand{\Ord}{\mathrm{Ord}}\newcommand{\eq}{\mathrm{eq}}\newcommand{\coeq}{\mathrm{coeq}}\newcommand{\act}{\mathrm{act}}

%%%%%%%%%%%%%%% 定理環境(足助先生ありがとうございます) %%%%%%%%%%%%%%%

\everymath{\displaystyle}
\renewcommand{\proofname}{\bf\underline{[証明]}}
\renewcommand{\thefootnote}{\dag\arabic{footnote}} %足助さんからもらった.どうなるんだ?
\renewcommand{\qedsymbol}{$\blacksquare$}

\renewcommand{\labelenumi}{(\arabic{enumi})} %(1),(2),...がデフォルトであって欲しい
\renewcommand{\labelenumii}{(\alph{enumii})}
\renewcommand{\labelenumiii}{(\roman{enumiii})}

\newtheoremstyle{StatementsWithUnderline}% ?name?
{3pt}% ?Space above? 1
{3pt}% ?Space below? 1
{}% ?Body font?
{}% ?Indent amount? 2
{\bfseries}% ?Theorem head font?
{\textbf{.}}% ?Punctuation after theorem head?
{.5em}% ?Space after theorem head? 3
{\textbf{\underline{\textup{#1~\thetheorem{}}}}\;\thmnote{(#3)}}% ?Theorem head spec (can be left empty, meaning ‘normal’)?

\usepackage{etoolbox}
\AtEndEnvironment{example}{\hfill\ensuremath{\Box}}
\AtEndEnvironment{observation}{\hfill\ensuremath{\Box}}

\theoremstyle{StatementsWithUnderline}
    \newtheorem{theorem}{定理}[section]
    \newtheorem{axiom}[theorem]{公理}
    \newtheorem{corollary}[theorem]{系}
    \newtheorem{proposition}[theorem]{命題}
    \newtheorem{lemma}[theorem]{補題}
    \newtheorem{definition}[theorem]{定義}
    \newtheorem{problem}[theorem]{問題}
    \newtheorem{exercise}[theorem]{Exercise}
\theoremstyle{definition}
    \newtheorem{issue}{論点}
    \newtheorem*{proposition*}{命題}
    \newtheorem*{lemma*}{補題}
    \newtheorem*{consideration*}{考察}
    \newtheorem*{theorem*}{定理}
    \newtheorem*{remarks*}{要諦}
    \newtheorem{example}[theorem]{例}
    \newtheorem{notation}[theorem]{記法}
    \newtheorem*{notation*}{記法}
    \newtheorem{assumption}[theorem]{仮定}
    \newtheorem{question}[theorem]{問}
    \newtheorem{counterexample}[theorem]{反例}
    \newtheorem{reidai}[theorem]{例題}
    \newtheorem{ruidai}[theorem]{類題}
    \newtheorem{algorithm}[theorem]{算譜}
    \newtheorem*{feels*}{所感}
    \newtheorem*{solution*}{\bf{[解]}}
    \newtheorem{discussion}[theorem]{議論}
    \newtheorem{synopsis}[theorem]{要約}
    \newtheorem{cited}[theorem]{引用}
    \newtheorem{remark}[theorem]{注}
    \newtheorem{remarks}[theorem]{要諦}
    \newtheorem{memo}[theorem]{メモ}
    \newtheorem{image}[theorem]{描像}
    \newtheorem{observation}[theorem]{観察}
    \newtheorem{universality}[theorem]{普遍性} %非自明な例外がない.
    \newtheorem{universal tendency}[theorem]{普遍傾向} %例外が有意に少ない.
    \newtheorem{hypothesis}[theorem]{仮説} %実験で説明されていない理論.
    \newtheorem{theory}[theorem]{理論} %実験事実とその(さしあたり)整合的な説明.
    \newtheorem{fact}[theorem]{実験事実}
    \newtheorem{model}[theorem]{模型}
    \newtheorem{explanation}[theorem]{説明} %理論による実験事実の説明
    \newtheorem{anomaly}[theorem]{理論の限界}
    \newtheorem{application}[theorem]{応用例}
    \newtheorem{method}[theorem]{手法} %実験手法など,技術的問題.
    \newtheorem{test}[theorem]{検定}
    \newtheorem{terms}[theorem]{用語}
    \newtheorem{solution}[theorem]{解法}
    \newtheorem{history}[theorem]{歴史}
    \newtheorem{usage}[theorem]{用語法}
    \newtheorem{research}[theorem]{研究}
    \newtheorem{shishin}[theorem]{指針}
    \newtheorem{yodan}[theorem]{余談}
    \newtheorem{construction}[theorem]{構成}
    \newtheorem{motivation}[theorem]{動機}
    \newtheorem{context}[theorem]{背景}
    \newtheorem{advantage}[theorem]{利点}
    \newtheorem*{definition*}{定義}
    \newtheorem*{remark*}{注意}
    \newtheorem*{question*}{問}
    \newtheorem*{problem*}{問題}
    \newtheorem*{axiom*}{公理}
    \newtheorem*{example*}{例}
    \newtheorem*{corollary*}{系}
    \newtheorem*{shishin*}{指針}
    \newtheorem*{yodan*}{余談}
    \newtheorem*{kadai*}{課題}

\raggedbottom
\allowdisplaybreaks
\usepackage[math]{anttor}
\begin{document}
\tableofcontents

\chapter{Banach空間}

\begin{quotation}
    Banach空間とは,CompMetの中の線型空間である.
    この圏Banの射はLipschitz定数が1以下の有界線形写像であるが,
    距離構造を忘れ,位相線型空間の構造を特に重要視して,$i:\Ban\mono\TVS$の像の充満部分圏$\Ban_\TVS$に注目することも多い.
    
    $\TVS$の射は有界線形写像である.
    位相線型空間とは,連続群のように,Topの中の線型空間である.
    任意の位相線形空間は自然に有界型空間の構造が入り,線型作用素が連続であることと有界であることとLipschitz連続写像であることとが同値になる.
\end{quotation}

\begin{notation}\mbox{}
    \begin{enumerate}
        \item $B(0,r)$を閉球とする.$B$を単位閉球とする.$B^*$を双対空間の単位閉球とする.$rB:=\Brace{rx\mid x\in B}$と表す.
        \item 具体的に$\bF=\R,\C$とするが,ほとんどの結果は離散的でない付値体一般,または局所体(local field),すなわち,局所コンパクトHausdorffな体一般に拡張できる.
        \item ノルム位相についての閉包を$\oo{Y}$で表す.
        \item $X$を局所コンパクトハウスドルフ空間とし,その上のコンパクト台を持つ実数値連続関数のなすノルム代数$C_c(X)$を考える.
        \item $X$のコンパクト部分集合の全体を$\cC\subset P(X)$で表す.
        \item $X$の可測集合全体を$\M$で表し,体積確定集合の全体を$\M^1$で表す\ref{def-measurable-sets}.
        \item $X$上のBorel関数のクラスを$\B(X)$,可測関数の全体を$\L(X)$で表す.$\B\subset\M$より$\B(X)\subset\L(X)$である\ref{def-measurable-function}.$X$上の局所可積分関数の空間を$\L^1_\loc(X)$で表す\ref{def-locally-integrable-function}.
        \item $X$上の有界なRadon電荷$\Phi$全体に全変動ノルム$\norm{\Phi}:=\abs{\Phi}(1)$を考えたBanach空間を$M(X)$で表す.
        \item $X$上の零関数を$\cN(X)$,零集合を$\cN$で表す\ref{def-null-set}.
        \item $C_c(X)$の単調増加ネットの極限として得られる関数全体のなす正錐を
        \[C_c(X)^m:=\Brace{f:X\to\R\cup\{\infty\}\mid C_c(X)\text{上の単調増加ネット}(f_\lambda)_{\lambda\in\Lambda}\text{が存在して}\forall_{x\in X}\;f(x)=\sup f_\lambda(x)\text{を満たす}}\]
        とする.
        また,$C_c(X)_m:=\Brace{f:X\to\R\cup\{-\infty\}\mid\exists_{\{f_\lambda\}\subset C_c(X)}\;f_\lambda\searrow f}$とすると,$C_c(X)_m=-C_c(X)^m$であり,$C_c(X)^m\cap C_c(X)_m=C_c(X)$である.
        \item 関数クラス$M(X)\subset\Map(X,\R)$に対して$M(X)_+$とは,非負関数のなす部分空間$M(X)\cap\Map(X,\R_{\ge 0})$を表す.
        \item 特性関数を$[A]=\chi_A=1_A$で表す.
    \end{enumerate}
\end{notation}

\section{ノルム空間の定義と例}

\begin{tcolorbox}[colframe=ForestGreen, colback=ForestGreen!10!white,breakable,colbacktitle=ForestGreen!40!white,coltitle=black,fonttitle=\bfseries\sffamily,
title=]
    連続と線型が出会うと有界と変化すること,内部homを持つことなど,とにかく美しい描像がある.
    
    ノルム空間の例は一般化すると,位相空間上の体値関数に,適切なノルムを入れることで構成する.
    ほとんどの例はその退化と見れる.
\end{tcolorbox}

\subsection{ノルム空間の定義と射}

\begin{tcolorbox}[colframe=ForestGreen, colback=ForestGreen!10!white,breakable,colbacktitle=ForestGreen!40!white,coltitle=black,fonttitle=\bfseries\sffamily,
title=]
    ノルム空間が定める位相,すなわち$\{\lambda B+x\}_{\lambda>0,x\in X}$が生成する位相は,ノルムを連続にする最弱位相になる.
    この位相について完備なノルム空間をBanach空間という.
    するとノルム空間の間の線型写像が連続であることと有界であることは同値で,これが射となる.
    なお,作用素論では,作用素の語は部分関数,線型写像の語は写像として,定義を使い分ける.
    作用素$x:X\to Y$の定義域を$D(x)\subset X$と表す,という具合である.
\end{tcolorbox}

\begin{definition}[norm, seminorm, equivalence, Banach space]
    $X$を$k$-線型空間とし,体$k$には絶対値$\abs{\;}:X\to\R_+$が備わっているとする.
    \begin{enumerate}
        \item 実数値関数$\norm{\;}:X\to\R_+$が\textbf{ノルム}であるとは,次の3条件を満たすことをいう.
        \begin{enumerate}[(a)]
            \item (positivity / faithfulness) $\forall_{u\in X}\;\norm{u}=0\Rightarrow u=0$.\footnote{これは通常$\forall_{u\in X}\;\norm{u}\ge 0$と等号成立条件が$u=0$と分けて書かれる.この主張だけで十分である理由は,(2)より$\norm{-u}=\norm{u}$であり,(3),(1)より$\norm{0}\le 2\norm{u}$が従うので,非負値であることが3条件から従う.}
            \item (linearity / homogeneity) $\forall_{\alpha\in k}\;\forall_{u\in X}\;\norm{\alpha u}=\abs{\alpha}\norm{u}$.
            \item (triangle inequality / subadditivity) $\forall_{u,v\in X}\;\norm{u+v}\le\norm{u}+\norm{v}$.
        \end{enumerate}
        \item 条件(1)が成り立たない場合,\textbf{セミノルム}または\textbf{半ノルム}という.
        \item ノルムが同値であるとは,$\exists_{C_1,C_2\in\R_{>0}}\;\forall_{u\in X}\;C_1\norm{u}_1\le\norm{u}_2\le C_2\norm{u}_1$.\footnote{これはノルムが定める距離が同値であることに同値.これは,$\id$についてのLipschitz連続性の条件と見れば良い.}
        \item ノルム空間が,ノルムが定める距離について完備であるとき,\textbf{Banach空間}という.\footnote{局所凸空間$(X,\F)$において,Cauchyネットの定義は$\forall_{r>0}\;\forall_{m\in\F}\;\forall_{j,k\in D}\;\exists_{j_0\in D}\;j,k\ge j_0\Rightarrow m(x_j-x_k)<r$に同値.これが必ず極限を持つことは通常の意味でのノルム空間の完備性と同値.}
        \item ノルム空間の間の線型写像を(線型)\textbf{作用素}という.
    \end{enumerate}
\end{definition}
\begin{remarks}[ノルム位相とは,ノルムが定める始位相であり,弱位相はセミノルム族が定める始位相である]
    $\norm{x}-\norm{y}\le\norm{x-y}=\norm{y-x}\ge\norm{y}-\norm{x}$より,
    $\abs{\norm{x}-\norm{y}}\le\norm{x-y}$であるため,ノルムは(Lipschitz)連続である.
    また,ノルムが定める位相とは,$\R_+$の開球の引き戻し,すなわち$\norm{x-y}<\ep$なる集合が生成する位相にほかならないから,ノルム$X\to\R_+$定める始位相にほかならない.
    これが弱位相の発想を導く.
\end{remarks}

\begin{definition}[$F$-norm, $F$-space / quasi-normed space, Frechet space]
    $V$を$K$-線型空間とする.$\norm{-}:V\to K$を関数とする.
    \begin{enumerate}
        \item 次の3条件を満たすとき,$\norm{-}$を\textbf{$G$-[セミ]ノルム}という.\footnote{ノルム空間のunderlying spaceをアーベル群について一般化した時に適切となるノルム概念である.}
        \begin{enumerate}[(a)]
            \item 正:$\norm{0_V}=0$.$[\norm{x}=0\Rightarrow x=0_V]$
            \item \textbf{$-1$の作用}:$\forall_{x\in V}\;\norm{-x}=\norm{x}$.
            \item 三角不等式:$\forall_{x,y\in V}\;\norm{x+y}\le\norm{x}+\norm{y}$.
        \end{enumerate}
        \item $G$-[セミ]ノルムが$V$に定める擬距離が定める位相について,スカラー乗法$K\times V\to V;(a,x)\mapsto ax$が連続であるとき,これを\textbf{$F$-[セミ]ノルム}という.
        \item そのノルム位相が完備な$F$-ノルム空間を\textbf{$F$-空間}という:すなわち,任意のCauchyネットが収束する$F$-ノルム空間を$F$-空間という.
        \item $F$-空間のノルム位相が局所凸であるとき,これを\textbf{Fréchet空間}という.射を連続線型写像とすると,Fréchet空間は圏TVSの充満部分圏をなす.
    \end{enumerate}
\end{definition}

\begin{proposition}[有界性:連続性の特徴付け]
    $X,Y$をノルム空間,
    $T:X\to Y$を線型作用素とする.次の6条件は同値.
    \begin{enumerate}
        \item $T$は連続である.
        \item ある$x\in X$において$T$は連続である.
        \item $T$は\textbf{有界}である:$\exists_{\al\ge 0}\;\forall_{x\in X}\;\norm{Tx}\le\al\norm{x}$.
        \item $T$はLipschitz連続.
        \item $T$は一様連続.
        \item 作用素ノルム$\norm{T}$が有限.
    \end{enumerate}
\end{proposition}
\begin{Proof}\mbox{}
    \begin{description}
        \item[(1)$\Rightarrow$(2)] 自明.
        \item[(2)$\Rightarrow$(3)] $x\in X$で連続であるから,$\exists_{\delta>0}\;\forall_{y\in X}\;\norm{x-y}<\delta\Rightarrow\norm{Tx-Ty}\le 1$.
        いま,$\forall_{z\in X\setminus 0}\;\Norm{\paren{\delta\frac{z}{\norm{z}}+x}-x}=\delta\le\delta$より,
        \[\Norm{T\delta\frac{z}{\norm{z}}+Tx-Tx}\le 1\Lrarrow\norm{Tz}\le\delta^{-1}\norm{z}.\]
        \item[(3)$\Rightarrow$(4)]
        $\forall_{x,y\in X}\;\norm{Ty-Tx}=\norm{T(y-x)}\le\al\norm{y-x}$より,Lipschitz連続である.
        \item[(4)$\Rightarrow$(5)$\Rightarrow$(1)] 明らか.
        \item[(3)$\Leftrightarrow$(6)] 明らか.
    \end{description}
\end{Proof}
\begin{remark}
    ここで,線形写像は零でない限り,有界には決してならない.
    作用素が「有界」であるとは,bornologyについて言う\ref{def-bounded-function}.
    実際,有界作用素は,有界集合を有界集合に写す.
\end{remark}

\subsection{射の指定}

\begin{tcolorbox}[colframe=ForestGreen, colback=ForestGreen!10!white,breakable,colbacktitle=ForestGreen!40!white,coltitle=black,fonttitle=\bfseries\sffamily,
title=]
    有限次元線形空間では,基底の行き先を指定する毎に線形写像が構成出来た.
    ノルム空間でも,稠密部分空間の中で,任意の有限次元部分空間上に有界作用素を定めるならば,同様の構成が出来る.
    一番の違いは,位相の用語が肝要になる点である.
\end{tcolorbox}

\begin{theorem}[線型作用素の「基底」による構成]
    $X,Y$をノルム空間,$\{x_j\}_{j\in J}\subset X,\{y_j\}_{j\in J}\subset Y$を元の族とし,生成する空間$\brac{x_j}_{j\in J}$は$X$上稠密とする.
    このとき,次の2条件は同値:
    \begin{enumerate}
        \item $\exists_{T\in B(X,Y)}\;\forall_{j\in J}Tx_j=y_j$.
        \item $\exists_{\al}\;\forall_{\lambda\subset J}\;\forall_{\al_j\in\bF}\;\abs{\lambda}<\infty\Rightarrow\Norm{\sum_{j\in\lambda}\al_jy_j}\le\al\Norm{\sum_{j\in\lambda}\al_jx_j}$.
    \end{enumerate}
    なお,このとき$T$は一意的である.
\end{theorem}

\begin{proposition}[線型汎関数の連続性の特徴付け]
    線型空間$X$上の
    線型汎函数$f:X\to\bF$について,
    \begin{enumerate}
        \item $f\ne 0$ならば,$X\simeq\Ker f\oplus\bF$.
        \item $X$が局所凸とする.$f$が連続であることと,$\Ker f$が閉部分空間であることとは同値.
    \end{enumerate}
\end{proposition}
\begin{Proof}\mbox{}
    \begin{enumerate}
        \item $f\ne 0$のとき,$f$は全射だから,$x\in f^{-1}(1)$が取れる.すると任意の$y\in X$について,$y=(y-f(y)x)+f(y)x\in\Ker f+\C x$である.実際,$f(y-f(y)x)=f(y)-f(y)f(x)=0$.
    \end{enumerate}
\end{Proof}

\begin{proposition}[有界作用素の可逆性の特徴付け]
    $H$をHilbert空間$T\in B(H)$とする.次の2条件は同値:
    \begin{enumerate}
        \item $T$は可逆である.
        \item $T$は下に有界:$\exists_{M>0}\;\forall_{x\in H}\;\norm{Tx}\ge M\norm{x}$.
    \end{enumerate}
\end{proposition}
\begin{Proof}\mbox{}
    \begin{description}
        \item[(2)$\Rightarrow$(1)] このとき,$\norm{T}\ge M$より,
    \end{description}
\end{Proof}

\subsection{作用素のノルム空間の完備性の十分条件}

\begin{tcolorbox}[colframe=ForestGreen, colback=ForestGreen!10!white,breakable,colbacktitle=ForestGreen!40!white,coltitle=black,fonttitle=\bfseries\sffamily,
title=Banには内部homを持つ閉圏としての構造がある]
    Banach空間の射は有界線型作用素とした,
    「閉単位球をどれくらい飛ばすか」という目安が考え得る.
    実際これによってノルムが定まり,
    これについて$\Hom_{\Ban_\TVS}(X,Y)$はBanach空間となる.
    この射集合は乗法を合成として代数の構造をもち,Banach代数と呼ばれる.
    Banは$\R$を単位として閉圏をなす:$\Hom_\Ban(\R,V)\simeq B:=\Brace{v\in V\mid\norm{v}\le 1}$.\footnote{\url{https://ncatlab.org/nlab/show/Banach+space}}
\end{tcolorbox}

\begin{definition}[operator norm, normed algebra]\mbox{}
    \begin{enumerate}
        \item ノルム空間$X,Y$の間の有界作用素の集合を$B(X,Y):=\Hom_{\TVS}(X,Y)$で表す.
        \item $B(X,Y)$は\textbf{作用素ノルム}
        \[\norm{T}:=\sup\Brace{\norm{Tx}\in\R_{\ge 0}\mid x\in X,\norm{x}\le 1}=\inf_{x\in X}\frac{\norm{Tx}}{\norm{x}}=\inf_{x\in X}\Norm{T\frac{x}{\norm{x}}}\]によってノルム空間になる.
        \item \textbf{ノルム代数}または\textbf{ノルム環}とは,連続な双線型写像$\cdot :A\times A\to A$によって結合代数の構造も持つノルム空間$A$のことをいう.\footnote{積が連続とは,$\exists_{C\in\R_{>0}}\;\norm{ST}\le C\norm{S}\norm{T}$を含意する.}
        \item ノルム空間としての$A$が完備である場合,これを\textbf{Banach代数}という.
        \item $B(X)$は作用素の合成を乗法として,
        劣乗法性$\norm{ST}\le\norm{S}\norm{T}$が成り立つ\footnote{$\norm{T}$が1以上か1以下かで場合分けすれば良い}.
        この劣乗法性により,乗法はノルムの定める位相について(両側)連続になるので,
        $B(X)$はノルム空間でありながら,かつ,ノルム代数の構造も持つ.
    \end{enumerate}
\end{definition}
\begin{remarks}
    作用素ノルムは達成されないことも十分にある.この場合,$B$は非閉集合に写される.
    なお,双対空間の作用素ノルムが達成され得るかは,$X$の回帰性に同値\ref{thm-James}.
\end{remarks}

\begin{proposition}[作用素ノルムの特徴付け]
    次の3つのノルムは等しい(任意の$T\in B(X,Y)$について同じ値を取る).
    \begin{enumerate}
        \item $\norm{T}_1=\sup_{x\in B}\norm{Tx}$.
        \item $\norm{T}_2=\sup_{x\in\partial B}\norm{Tx}$.
        \item $\norm{T}_3=\sup_{x\ne 0}\frac{\norm{Tx}}{\norm{x}}$.
    \end{enumerate}
\end{proposition}
\begin{Proof}
    $T\in B(X,Y)$を任意に取る.
    すると,$x/\norm{x}$のノルムが1であることに注意すると,$\norm{T}_3\le\norm{T}_2\le\norm{T}_1$は明らか.
    ここで,$\norm{T}_1\le\norm{T}_3$でもある.これは,任意の$x\in B$について,$\norm{Tx}\le\norm{Tx}/\norm{x}$であるため.
\end{Proof}

\begin{proposition}\label{prop-internal-hom}
    $X,Y$をノルム空間,$Y$を完備とする.このとき,$B(X,Y)$はBanach空間である.
    特に$B(X)$はBanach代数である.
\end{proposition}
\begin{Proof}
    $B(X,Y)$の任意のCauchy列$(T_n)$が収束することを示せば良い.
    任意の$x\in X$について,$(T_nx)$も$Y$のCauchy列であり,$Y$は完備であるから,極限$\lim_{n\to\infty}T_nx$が定まる.これを$Tx:=\lim_{n\to\infty}T_nx$として,対応$T:X\to Y$を定め,これが$T\in B(X,Y)$で$\lim_{n\to\infty}T_n=T$であることを示す.
    \begin{description}
        \item[線形性] $T_n$と極限の線形性による:$T(x+y)=\lim_{n\to\infty}T_n(x+y)=\lim_{n\to\infty}(T_nx+T_ny)=Tx+Ty$.
        \item[有界性と$T_n\to\ T$] 
        任意の$n\in\N$と$x\in X$について,
        \begin{align*}
            \norm{Tx-T_nx}&=\lim_{m\to\infty}\norm{T_mx-T_nx}=\lim_{m\to\infty}\Norm{(T_m-T_n)\frac{x}{\norm{x}}\norm{x}}&ノルムの連続性\\
            &\le\limsup_{m\to\infty}\norm{T_m-T_n}\cdot\norm{x}
        \end{align*}
        が成り立つ.
        $(T_n)$はCauchy列としたから,$\limsup_{n,m\to\infty}\norm{T_m-T_n}=0$.
        また,$(\norm{T-T_n})_{n\in\N}$は$0$に収束する列より有界列だから,$T$が有界であることもわかる:
        \[\norm{Tx}\le\norm{Tx-T_nx}+\norm{T_nx}\le(C_1+C_2)\norm{x}.\quad(\exists_{C_1,C_2\in\R}).\]
    \end{description}
\end{Proof}



\subsection{体の積と数列空間}

\begin{example}[体の積]\mbox{}
    \begin{enumerate}
        \item $\bF^n$には$p\in[1,\infty]$について,$\norm{x}_p:=\paren{\sum\abs{x_k}^p}^{1/p}\;(p<\infty)$,$\norm{x}_\infty=\max\abs{x_k}$などのノルムが入り,いずれも同値である.
        これは特殊な例で,普通Banach空間には自然なノルムが一意に定まる.距離空間に同値な距離が大量にあるのと対照的である.
        \item 体$\bF$の添字集合$J$に関する直積($l^\infty$-直和)を$l^\infty(J)$で表す.
        \item 体$\bF$の添字集合$J$に関する直和($l^1$-直和)を$l^1(J)$または$c_0(J)$で表す.
        \begin{enumerate}[(i)]
            \item $J=\N$のとき,これを省略して書く.$l^\infty$は有界列の空間,$l^1$は絶対収束列の空間である.$l^1\subsetneq(l^\infty)^*$が成り立ち,$l^\infty$は可分でない.
            \item $c_0$は絶対収束する級数を定める数列,すなわち$0$に収束する数列の空間である.
            収束列は有界列であるから,上限ノルムについて$l^\infty$の可分な閉部分空間となる.したがって$c_0$もBanachであり,その双対空間を考えると,$\N$上の有界な符号付き測度であるから,$(c_0)^*=l^1$となる\ref{exp-Riesz-Markov-theorem-to-sequence-spaces}.また,$c_0$はいかなるBanach空間のpredualにもならない\ref{cor-predual-of-c0}.
            \item 収束列の空間$c$は$l^\infty$の可分な閉部分空間となる.$\forall_{n\in\N}\;e_n=1$を満たす元を$e\in l^\infty$とすると,$c=\bF e+c_0$と表せる.
            \item $a_k=0\;\fe$を満たす数列全体の空間を$c_c$または$c_{00}$で表す.これはコンパクト収束位相\footnote{任意の有限集合上で収束する}について局所凸空間である.
        \end{enumerate}
        \item 体$\bF$の添字集合$J$に関する$l^p$-直和($l^1$-直和)を$l^p(J)$と表すが,これは$J$の数え上げ測度に関する$L^p(J)$にほかならない.
    \end{enumerate}
\end{example}

\subsection{位相空間上の有界連続関数の空間}

\begin{tcolorbox}[colframe=ForestGreen, colback=ForestGreen!10!white,breakable,colbacktitle=ForestGreen!40!white,coltitle=black,fonttitle=\bfseries\sffamily,
title=]
    局所コンパクトハウスドルフ空間$X$について,
    $C_c(X)\subset C_0(X)\subset C_b(X)$が成り立ち,後2つが一様ノルムについて完備になる.
    $C_c(X)$の一様ノルムに関する完備化は$C_0(X)$で,$p$-ノルムに関する完備化が$L^p(X)$である.

    一方で$C(X),H(X),C^\infty(X),D_K$はノルム付け可能でないが,いずれもFrechet空間となる.
\end{tcolorbox}

\begin{example}[位相空間上の体値関数]\label{exp-Banach-spaces}
    $X$を位相空間として,$C_c(X),L^p(\Om),C_b(X),C_0(X)$を定める.
    \begin{enumerate}
        \item 局所コンパクトハウスドルフ空間$X$について,$C_c(X):=\Brace{f\in C(X)\mid\supp f\text{ is compact}}$は,
        一様ノルムについて,
        完備でないノルム空間.
        \begin{enumerate}[(i)]
            \item $X$がコンパクトのとき,
            $C_b(X,\C)=C_c(X,\C)$で,これは$C^*$-環である.
            これを\textbf{連続関数環}という.
            \item 連続関数環の閉部分環であって,定数を含み,$X$の点を分離するものを\textbf{一様環}という.
            これは再び単位的なBanachとなる.
            \item Euclid空間の局所コンパクトハウスドルフな部分空間$\Om\subset\R^n$\footnote{局所コンパクトハウスドルフ性は弱遺伝的で,開集合と閉集合とはこれを満たす}の場合,$C_c(\Om)$には
            Riemann積分を通じて$p\in\cointerval{1,\infty}$-ノルムも入る.
            この完備化は,\textbf{Lebesgue空間}$L^p(\Om)$となり,Banach空間である.
        \end{enumerate}
        \item 位相空間$X$について,$C_b(X)$は
        一様ノルムについてBanach空間となり,各点ごとの乗法についてBanach代数になる.
        \begin{enumerate}[(i)]
            \item 離散空間$X$は局所コンパクトである.このとき,$X$上の任意の関数は連続になる:$C(X)=\Map(X,\bF)$.この場合,$C_b(X)$を$l^\infty(X)$と表す.
        \end{enumerate}
        \item 局所コンパクトハウスドルフ空間$X$について,
        $C_0(X):=\Brace{f\in C(X)\mid\forall_{\ep>0}\;\Brace{x\in X\mid\abs{f(x)}\ge\ep}\text{ is compact}}$
        は可換な$C^*$-環で,一様ノルムについて$\o{C_c(X)}=C_0(X)\subset C_b(X)$が成り立つ.
        \begin{enumerate}[(i)]
            \item $X$がコンパクトであることとBanach環$C_0(X)$が単位的であることは同値.
        \end{enumerate}
        \item $X$がコンパクトであるとき,$C_c(X)=C_0(X)=C_b(X)=C(X)$が成り立つ.
    \end{enumerate}
\end{example}

\subsection{位相空間上の連続関数の空間}

\begin{example}[コンパクト開位相は距離化可能だがノルム付け可能でない]
    $\Om\osub\R^m$を非空な開集合とし,$C(\Om), H(\Om)$の位相を考える.
    \begin{enumerate}
        \item $\Om$は$\sigma$-コンパクトだから,$\Om$に至るコンパクト集合の増大列$(K_n)$が存在する.
        これが定める半ノルムの族$(p_n(f):=\sup_{x\in K_n}\abs{f(x)})_{n\in\N}$が誘導する局所凸位相を入れたものは完備で,したがってFrechet空間となる.
        \begin{enumerate}[(i)]
            \item 距離は
            \[d(f,g)=\max_{n\in\N}\frac{2^{-n}p_n(f-g)}{1+p_n(f-g)}\]
            が与える.$\max_{n\in\N}$の代わりに$\sum_{n\in\N}$としても良いことに注意.
            \item この距離についてのCauchy列は,任意の半ノルム$\rho_n$について$0$に収束するから,すなわち任意の$K_n$上で一様収束する.
            よってこの位相はコンパクト開位相にほかならない.
            \item しかし局所有界でないため,ノルム付け可能ではない\ref{thm-character-of-TVS}.
        \end{enumerate}
        \item 正則関数のなす部分空間$H(\Om)\subset C(\Om)$は閉部分空間だから,やはりFrechet空間である.この空間はHeine-Borel性を持つため,局所有界でなく,したがってノルム付け可能でない.
    \end{enumerate}
\end{example}

\subsection{試験関数の空間と分布論}

\begin{example}[滑らかな関数の空間と試験関数の空間]
    $K\subset\R^n$をコンパクト集合とし,$K\subset\Om\osub\R^n$を開集合として,$C^\infty(\Om),D_K(\Om),D(\Om)$を定める.
    $D_K(\Om),D(\Om)$は現代的には$C_c^\infty(K),C_c^\infty(\Om)$とも書く.\footnote{$D$の記法はSchwartz以来伝統的である.}
    \begin{enumerate}
        \item  $C^\infty(\Om)$は$\Om$に届くコンパクト集合の増大列$(K_i)$について,
        \[p_N(f):=\max\Brace{\abs{D^\al f(x)}\in\R_+\mid x\in K_N,\abs{\al}\le N}\]
        として得る半ノルム列$(p_N)_{N\in\N}$に関して,完備になる.よって,Frechet空間である.
        \begin{enumerate}[(i)]
            \item これはHeine-Borel性を持つため,局所有界でなく,ノルム付け可能でない.
            \item $\ev_x:C^\infty(\Om)\to\R$は連続.
            \item このことからわかるように,この位相は,任意の多重指数$\al\in\N^n$に関する導関数$D^\al f$が一様収束する位相である.
            \item この空間の上に,任意の多重指数$\al$について,$D^\al:C^\infty(\Om)\to C^\infty(\Om)$は連続単射になる.
        \end{enumerate}
        \item $K$に台を持つ$\Om$上の滑らかな関数の全体を
        \[D_K(\Om):=\Brace{f\in C^\infty(\Om)\mid \supp f\subset K}\]
        とすると,これは$C^\infty(\Om)$の閉部分空間となり,再びFrechet空間である.
        \begin{enumerate}[(i)]
            \item $D_K(\Om):=\bigcap_{y\in\Om\setminus K}\Ker\ev_y$だから,たしかに閉部分空間である.
        \end{enumerate}
        \item ネット$(D_K)_{K\in\cC}$の帰納極限$D(\Om):=\bigcup_{K\in\cC(\Om)}D_K(\Om)=C_c^\infty(\Om)$は再び局所凸で完備であり,$F$-空間でもあるが,Frechet空間ではない.この空間を\textbf{コンパクト台を持つ試験関数の空間}という.
        \begin{enumerate}[(i)]
            \item $\norm{\phi}_N:=\max_{\abs{\al}\le N,x\in\Om}\abs{D^\al\phi(x)}$はノルムを定める.このノルムの$D_K$への制限はセミノルムであり,$D_K$の位相を定める.
            \item このノルムについて,$D(\Om)$は局所凸な距離化可能空間になるが,完備ではない.
            \item 一方で,局所凸で完備であるが,距離化可能ではない位相$\tau$も取れる.
            \item いずれにしろ,$D(\Om)$の位相からみて,$D_K(\Om):=\bigcap_{x\in\Om\setminus K}\Ker\delta_x$はいずれも閉疎集合で,$\Om$の$\sigma$-コンパクト性よりこの部分空間の可算合併で表せるから,$D(\Om)$もBaire空間ではなく,完備距離空間たり得ない.
            つまり,位相$\tau$は完備な時点で距離化可能ではない.
        \end{enumerate}
    \end{enumerate}
    $D(\Om)$上,次の位相に関して連続な線型汎関数を\textbf{分布}または\textbf{超関数}という.
\end{example}

\begin{theorem}[連続関数の滑らかな近似]
    任意の$f\in C_c(\R^n)$は,$C_c^\infty(\R^n)$の列によって$\R^n$上一様に近似可能である.
\end{theorem}

\begin{lemma}
    次の位相$\tau$は$D(\Om)$を完備にするが,距離化可能でない.
    \begin{enumerate}
        \item $\tau_K\subset P(D_K)$をFrechet空間$D_K$の位相とする.
        \item $\beta:=\Brace{W\subset D(\Om)\mid W\text{は絶対凸集合で}\forall_{K\compsub\Om}\;D_K\cap W\in\tau_K}$.
        \item $W\in\beta$の$\varphi\in D(\Om)$-平行移動$\varphi+W$で表せる集合の合併で表せる集合の全体を$\tau$とする.
    \end{enumerate}
\end{lemma}

\begin{theorem}\mbox{}
    \begin{enumerate}
        \item $\tau$は$D(\Om)$に位相を定め,$\beta$は局所基底となる.
        \item $(D(\Om),\tau)$は局所凸空間である.
    \end{enumerate}
\end{theorem}

\begin{theorem}[試験関数の空間の位相の特徴付け]\mbox{}
    \begin{enumerate}
        \item 絶対凸集合$V\subset D(\Om)$について,開であることと$V\in\beta$であることとは同値.
        \item $\tau_K$は$\tau$の相対位相に等しい.
        \item $D(\Om)$の有界集合$E$について,あるコンパクト集合$K\subset\Om$が存在して$E\subset D_K$を満たし,また,正整数列$(M_N)_{N\in\N}$が存在して$\forall_{\varphi\in E}\;\forall_{N\in\N}\;\norm{\varphi}_N\le M_N$が成り立つ.
        \item $D(\Om)$はHeine-Borel性を持つ.
        \item $\{\varphi_i\}\subset D(\Om)$がCauchy列ならば,あるコンパクト集合$K\subset\Om$が存在して$\{\varphi_i\}\subset D_K$でもあり,$\forall_{N\in\N}\;\lim_{i,j\to\infty}\norm{\varphi_i-\varphi_j}_N=0$.
        \item $\varphi_i\to0$ならば,あるコンパクト集合$K\subset\Om$が存在して$\cup_{i\in\N}\supp\varphi_i\subset K$を満たし,$\forall_{\al\in\N^n}\;D^\al\varphi_i\to0$は一様収束する.
        \item $D(\Om)$は完備である.
    \end{enumerate}
\end{theorem}

\begin{theorem}[試験関数の空間上の線形作用素の連続性の特徴付け]
    $Y$を局所凸空間,$\Lambda:D(\Om)\to Y$を線型作用素とする.次の4条件は同値:
    \begin{enumerate}
        \item $\Lambda$は連続.
        \item $\Lambda$は有界.
        \item $\varphi_i\to0$ならば,$\Lambda\varphi_i\to0$.
        \item 任意の$K\compsub\Om$について,制限$\Lambda|_{D_K}$は連続.
    \end{enumerate}
\end{theorem}

\begin{definition}[distribution]
    $D(\Om)$上の連続な線型汎関数を\textbf{$\Om$上の分布}という.
    分布の全体を$\D'(\Om):=\Hom_\TVS(\Om,\R)$で表す.
\end{definition}

\begin{theorem}[連続性の特徴付け]
    $\Lambda:D(\Om)\to\C$を線型汎関数とする.次の2条件は同値:
    \begin{enumerate}
        \item $\Lambda\in\D'(\Om)$.
        \item 任意のコンパクト集合$K\subset\Om$に対して自然数$N\in\N$と定数$C\in\R$が存在して,$\forall_{\varphi\in D_K}\;\abs{\Lambda\varphi}\le C\norm{\varphi}_N$を満たす.
    \end{enumerate}
\end{theorem}

\subsection{急減少・緩増加関数の空間}

\begin{tcolorbox}[colframe=ForestGreen, colback=ForestGreen!10!white,breakable,colbacktitle=ForestGreen!40!white,coltitle=black,fonttitle=\bfseries\sffamily,
title=]
    試験関数の空間を
    $D(\R^n)\subset\S_n$に拡張して,対応する双対空間を少し小さくしてみることを考える.
\end{tcolorbox}

\begin{definition}[rapidly decreasing function, the Schwartz space]
    $f\in C^\infty(\R^n)$が\textbf{急減少偏導関数を持つ滑らかな関数}であるとは,次の同値な条件を満たすことをいう:
    \begin{enumerate}
        \item $\forall_{n,m\in\N}\;(\id)^nf^{(m)}\in L^1$.
        \item $\forall_{N\in\N}\;\sup_{\abs{\al}\le N}\sup_{x\in\R^n}(1+\abs{x}^2)N\abs{(D_\al f)(x)}<\infty$.
        \item $\forall_{n\in\N,\al\in\N^n}\;\forall_{P\in\R[x]}\;P\cdot D_\al f\in l^\infty(\R^n)$.
    \end{enumerate}
    ノルムの列$\sup_{\abs{\al}\le N}\sup_{x\in\R^n}(1+\abs{x}^2)N\abs{(D_\al f)(x)}\;(N\in\N)$は,$\S_n$上に局所凸な位相を定める.
    この位相線形空間を\textbf{Schwartz空間}といい,$\S_n:=\S(\R^n)$で表す.
\end{definition}

\begin{theorem}\mbox{}
    \begin{enumerate}
        \item $\S_n$はFrechet空間である.
        \item 包含$i:\D(\R^n)\to\S_n$は連続であり,像は稠密.
        \item 多項式$P$との積$f\mapsto Pf$,積$(f,g)\mapsto fg$,偏微分$f\mapsto D_\al f$は$\S_n$上の連続線型写像を定める.
    \end{enumerate}
\end{theorem}

\begin{definition}[tempered distribution]
    連続線型汎関数$L\in\S_n'$に対して,定理より$u_L:=L\circ i\in\D'(\R^n)$も連続線型汎関数となる.
    また,$\D(\R^n)$の稠密性より,$L\mapsto u_L$の対応は単射である.こうして埋め込み$\S'_n\ni L\mapsto u_L\in\D'(\R^n)$が定まる.
    この像をなす$\D'(\R^n)$の元を\textbf{緩増加分布}という.
    同一視により$\S'_n\subset\D'(\R^n)$とする.
    すなわち,$u\in\D'(\R^n)$であって,$\S_n$上への連続な延長を持つものをいう.
\end{definition}

\begin{theorem}
    Fourier変換$\S_n'\to\S_n'$を緩増加分布に適切に定めると,周期4のTVSの同型となる.
\end{theorem}

\begin{example}[任意の可積分関数は緩増加である]\mbox{}
    \begin{enumerate}
        \item コンパクトな台を持つ超関数は緩増加である.
        \item $\R^n$上のBorel測度$\mu$であって,$\exists_{k\in\N}\;\int_{\R^n}(1+\abs{x}^2)^{-k}d\mu(x)<\infty$を満たすものは緩増加である.
        \item $g\in\L(\R^n)$が$\exists_{N>0,p\in[1,\infty)}\;\int_{\R^n}\abs{(1+\abs{x}^2)^{-N}g(x)}^pdm_n(x)=C<\infty$を満たすならば緩増加である.
        \item 任意の$g\in L^p(\R^n)\;(p\in[1,\infty])$は緩増加である.
    \end{enumerate}
    $e^x\cos(e^x)$は緩増加であるが,$e^x$はそうではない.
\end{example}

\subsection{連続性・可微分性の階層}

\begin{tcolorbox}[colframe=ForestGreen, colback=ForestGreen!10!white,breakable,colbacktitle=ForestGreen!40!white,coltitle=black,fonttitle=\bfseries\sffamily,
title=]
    $\Lip^\al([a,b])$は特殊な$l^1$-ノルムについてBanach空間になる.
    同様な発想で,有界微分可能な関数の空間$C_b^m(\Om)$もBanach空間となる.
    連続微分可能な関数の空間$C^{m,\al}(X)$も局所凸になり,$\Om$がコンパクトなときBanach空間となる.
\end{tcolorbox}

\begin{example}[Lipschitz連続な関数の空間]
    $I:=[a,b]$上の$\al$-Holder連続な関数の全体
    \[\Lip^\al(I):=\Brace{f\in C(I)\mid L(f)<\infty},\quad L(f):=\sup_{x\ne y}\frac{\abs{f(x)-f(y)}}{\abs{x-y}^\al}=\sup_{\delta>0}\frac{\om(\delta;f)}{\delta^\al}.\]
    は,2つの同値なノルム
    \[\norm{f}:=L(f)+\abs{f(a)}\]
    \[\nnorm{f}:=L(f)+\norm{f}_\infty\]
    についてBanach空間となる.
    $L(f)$を$\al$-次Holder定数という.
    \begin{enumerate}[(i)]
        \item さらに,連続度が$\om(\delta;f)=o(\delta^\al)\;(\delta\to0)$を満たすものからなる部分集合$\lip^\al(I):=\Brace{f\in\Lip^\al(I)\mid\lim_{\delta\to0}\frac{\om(\delta;f)}{\delta^\al}=0}$は閉部分空間をなす.
    \end{enumerate}
\end{example}

\begin{example}[有界変動関数と絶対連続関数の空間]
    $f\in\BV([a,b])$について,
    \begin{enumerate}
        \item $\norm{f}=\abs{f(a)}+V(f)$はノルムを定め,これについてBanach空間となる.
        \item $\AC([a,b])\subset\BV([a,b])$は閉部分空間をなす.
    \end{enumerate} 
    \begin{remark}\mbox{}
        \begin{enumerate}
            \item $\AC([a,b])\subsetneq\BV([a,b])\cap C([a,b])$であることに注意.
            反例はCantor関数で,連続かつ単調増加であるが,絶対連続ではない.
            \item また$C([a,b])\setminus\BV([a,b])\ne\emptyset$である.例は$f(x)=1_{\Brace{x\ne0}}x\sin\paren{\frac{1}{x}}$.
        \end{enumerate}
    \end{remark}
\end{example}

\begin{example}[導関数ノルムを備えたBanach空間]\label{exp-Banach-space-with-derivative-norm}
    $\Om\osub\R^n$上の$C^m$-級関数の空間$C^m(\Om)$,$m$階以下の導関数がすべて$\Om$上有界である関数の空間$B^m(\Om)$または$C_b^m(\Om)$を定義する.
    \begin{enumerate}
        \item $C^m(\Om)$はセミノルムの族$(p_{K,m}(f)=\sup_{\abs{s}\le m,x\in K}\abs{D^sf(x)})$に関してFrechet空間となる.
        \item $C^m_b(\Om)$はある種の1-ノルム$\norm{u}_{C^1}=\sum_{\abs{\al}\le m}\norm{D^\al u}_\infty$について,Banach空間となる.
        \begin{enumerate}[(i)]
            \item これはある種の1-ノルム$\norm{u}_{C^1}=\sum_{\abs{\al}\le m}\norm{D^\al u}_\infty$について,Banach空間となる.
            \item $C_b^0(\Om)$は$C_b(\Om)$に一致する.
            \item $C_c^m(\Om)\subset C_b^m(\Om)$は変わらない.
        \end{enumerate}
    \end{enumerate}
\end{example}

\begin{example}[Holder空間]
    局所コンパクトハウスドルフ空間$X$上の,$k$階$\al$-Holder連続微分可能な関数の空間を
    \[C^{k,\al}(X):=\Brace{f\in C^k\mid\forall_{n\in[k]}\abs{f^{(k)}}_{C^{0,k}}<\infty}\quad\abs{f}_{C^{0,\al}}:=\sup_{x\ne y\in\Om}\frac{\abs{f(x)-f(y)}}{\abs{x-y}^\al}\]
    と定めると,これは半ノルムの族$\norm{f}_{C^{k,\al}}:=\norm{f}_{C^k}+\sup_{\abs{\beta}=k}\abs{D^\beta f}_{C^{0,\al}}$,ただし$\norm{f}_{C^k}:=\sup_{\abs{\beta}\le k}\sup_{x\in\Om}\abs{D^\beta f(x)}$についてFrechet空間(たぶん)となり,$X$がコンパクトであるときさらにBanach空間となる.
    \begin{enumerate}[(i)]
        \item 包含$C^{0,\beta}(X)\mono C^{0,\al}(X)\;(0<\al<\beta\le 1)$が存在する.
        \item Ascoli-Arzelaの定理より,$\norm{-}_{0,\beta}$ノルムにおける有界集合は,$\norm{-}_{-\al}$ノルムにおける相対コンパクト集合に送られる.
        \item これは,$C^k(X)$空間の,$k$に分数を許したバージョンだとみなせる.
    \end{enumerate}
\end{example}

\subsection{正則関数の空間}

\begin{example}
    $\Om\osub\C$上の正則関数の空間を$H(\Om)$とする.
    \begin{enumerate}
        \item $H(\Om)\subset C(\Om)$は閉部分空間である.コンパクト一様収束の位相について閉じているのは明らか.したがって$H(\Om)$もFrechet空間である.
        \item $H(\Om)$はHeine-Borel性を持つ.
        \item $H(\Om)$は局所有界でないから,ノルム付け可能ではない.
    \end{enumerate}
\end{example}

\begin{example}
    2乗可積分な正則関数のなす部分空間を
    $A^2(\Om):= H(\Om)\cap\L^2(\Om)$とする.このとき,次が成り立つ.
    \begin{enumerate}
        \item 任意の閉円板$[\Delta(z,r)]\subset\Om$と$f\in A^2(\Om)$について,
        \[f(z)=\frac{1}{\pi r^2}\int_{B(z,r)}f(x)dx.\]
        \item 通常の内積について$A^2(\Om)$もHilbert空間をなす.その位相は,コンパクト一様収束の位相に一致する.
        \item 特に$\Om=\Delta$のとき,$(\wt{e}_n(z):=(n+1)^{1/2}\pi^{-1/2}z^n)_{n\in\N}$は正規直交基底をなす.
    \end{enumerate}
\end{example}

\begin{example}[コンパクト空間上のHardy空間]
    $L^2(T)$の正規直交基底の一部$(e_n(z)=(2\pi)^{-1/2}z^n)_{n\ge0}$が生成する閉部分空間を$H^2(T)\subset L^2(T)$と表す.したがってそれ自体Hilbert空間である.
    \begin{enumerate}
        \item $f=\sum\al_ne_n\in H^2$について,Fourier変換$\wt{f}:\Delta\to\C$を$\wt{f}(z):=(2\pi)^{-1/2}\sum\al_nz^n$で定める.
        Hardy空間はFourier変換を通じて,$l^2(\N)$で表現される.すなわち,次の2条件は同値:
        \begin{enumerate}[(a)]
            \item 正則関数$g\in\H(\Delta)$のTaylor級数$g(z)=\sum\beta_nz^n$の係数は$\sum\abs{\beta_n}^2<\infty$を満たす.
            \item ある$f\in H^2$について,$g=\wt{f}$.
        \end{enumerate}
        \item 2乗可積分な正則関数$g\in A^2$に対して$g_t(z):=g(tz)\;(t\in(0,1))$と定めると,ある$f_t\in H^2$について$g_t=\wt{f_t}$.
        \item またこのとき,ある$f\in H^2$について$g=\wt{f}$であることと,$\sup_{t\in(0,1)}\norm{f_t}<\infty$とは同値.このとき,$f_t\to f\;\in H^2$.
    \end{enumerate}
\end{example}

\begin{example}[上半平面上のHardy空間]
    こうして,上半平面上には,Hardy空間$H^2(H^*)$を次のように定義することも出来る:
    \[H^2(H^+):=\Brace{f\in H(H^+)\;\middle|\;\norm{f}_{H^2}:=\sup_{y>0}\paren{\int\abs{f(x+ip)}^2dx}^{1/2}<\infty}\]
    この定義は一般の$H^p\;(1\le p<\infty)$に一般化出来る.
    一般にBanach空間で,$p=2$のときHilbertとなる.
\end{example}

\subsection{測度空間上の関数と積分}

\begin{tcolorbox}[colframe=ForestGreen, colback=ForestGreen!10!white,breakable,colbacktitle=ForestGreen!40!white,coltitle=black,fonttitle=\bfseries\sffamily,
title=]
    Radon積分論によれば,
    一般の局所コンパクトハウスドルフ空間上に
    $L^p(X)$が定まる.$(L^\infty(X))^*$のみが,$L^p$と同型な双対空間を持たない.
    $p\in(0,1)$のとき,$X$が可算でない限り,Frechet空間でさえない$F$-空間となる.
\end{tcolorbox}

\begin{example}[測度空間上の体値関数の同値類]\label{exp-Banach-space-of-Radon-integrable-functions}
    $\Om\subset\R^n$を一般の集合,$X$を一般の位相空間とし,$L^p(\Om),L^p(X)$を定める.
    \begin{enumerate}
        \item Lebesgue積分により,$L^p(\Om)\;p\in[1,\infty]$はBanach空間となる.
        \item 局所コンパクトハウスドルフ空間$X$上のRadon積分$\int:C_c(X)\to\R$について,
        \begin{align*}
            \L^p(X)=\left\{f\in\L(X)\;\middle|\;\int\abs{f}^p<\infty\quad(p\in[1,\infty))\right\},\quad\L^\infty(X)=\Brace{f\in\L(X)\;\middle|\;\esssup\abs{f}<\infty}
        \end{align*}
        は,セミノルム
        \[\norm{f}_p:=\paren{\int\abs{f}^p}^{1/p}\;(1\le p<\infty),\quad\norm{f}_\infty:=\esssup\abs{f}=\inf\left\{s\in\R\;\middle|\;\int(\abs{f}-\abs{f}\land s)=0\right\}\]
        を持つ.
        $\cN(X)$を零関数$\int\abs{f}=0$のなす部分空間とすると,$L^p(X):=\L^p(X)/\cN(X)$はノルム空間となり,さらにBanach空間となる(Riesz-Fischerの定理\ref{thm-Riesz-Fischer}).
    \end{enumerate}
    なお,固定された$f$について,ノルム$\norm{f}_p$は$p>1$について連続.
    このとき,$\norm{f}_p\xrightarrow{p\to\infty}\norm{f}_\infty$と,本質的上限に収束する.
\end{example}
    
\begin{theorem}[$L^p$空間の描像]\label{thm-Lp-of-dual-spaces}
    $(X,\Om,\mu)$を測度空間とし,$1\le p<\infty,1/p+1/q=1$を共役指数とする.
    $g\in L^q(\mu)$に対して,$F_g:L^p(\mu)\to\bF$を$F_g(f):=\int fgd\mu$で定める.
    \begin{enumerate}
        \item $1<p<\infty$のとき,$F:L^q(\mu)\iso L^p(\mu)^*$は等長同型を定める.特に回帰的である.
        \item $p=1$で$(X,\Om,\mu)$が$\sigma$-有限のとき,$F:L^\infty(\mu)\iso L^1(\mu)^*$は等長同型を定める.
        \item ただし,$L^1(\mu)\subsetneq (L^\infty(\mu))^*$が成り立つ.
    \end{enumerate}
\end{theorem}

\begin{example}
    $I:=[0,1].p\in(0,1)$について,
    \[L^p(I):=\Brace{f\in\Meas(I,\R)\;\middle|\;\Delta(f):=\int^1_0\abs{f(t)}^pdt<\infty}\]
    上に,平行移動不変な距離$d(f,g):=\Delta(f-g)$が定まる.$I$でなく可算集合である場合$l^p$などは,これはFrechet空間になるが,
    $[0,1]$上では局所有界な$F$-空間であるが,局所凸ではないため,Frechet空間とはならない.
    すなわち,次が成り立つ:$\emptyset,L^p$を除いて凸開集合は存在しない.
\end{example}

\subsection{Sobolev空間}

\subsection{Orlicz空間}

\begin{definition}\mbox{}
    \begin{enumerate}
        \item 関数$\Phi:\R_+\to[0,\infty]$が\textbf{Young関数}であるとは,ある左連続な広義単調増加関数$\varphi:\R_+\to[0,\infty],\varphi(0)=0$に対して
        \[\Phi(t)=\int^t_0\varphi(s)ds\;(t\in\R_+)\]
        と表せることをいう.
        \item この$\varphi$を\textbf{左導関数}という.
        \item Young関数$\Phi$が特に$\varphi(\R_{>0})\subset\R_{>0}$を満たし,$\lim_{t\to+0}\varphi(t)=0,\lim_{t\to\infty}\varphi(t)=\infty$を満たすとき,\textbf{$N$-関数}という.
    \end{enumerate}
\end{definition}

\begin{definition}
    $(X,\mu)$を$\sigma$-有限な測度空間,$\Phi$をYoung関数とし,$\ep>0$とする.
    \begin{enumerate}
        \item $\Phi(\ep L):=\Brace{f\in\Map(X,\R)\;\middle|\;\int_X\Phi(\ep\abs{f(x)})d\mu<\infty}$をOrliczクラスという.
        \item $L^\Phi(X,\mu):=\bigcup_{0<\ep<\infty}\Phi(\ep L)$を\textbf{Orlicz空間}という.
    \end{enumerate}
\end{definition}

\begin{lemma}[Orlicz空間はBanach空間になる]\mbox{}
    \begin{enumerate}
        \item Orlicz空間は,Orliczクラス$\Phi(L)$を含む最小の線型空間となる.
        \item 次はノルムを定める(Luxemburg-Nakanoノルム)
        \[\norm{f}_{L^\Phi(X,\mu)}:=\inf\Brace{\lambda>0\;\middle|\;\int_X\Phi\paren{\frac{1}{\lambda}\abs{f(x)}}d\mu\le 1}.\]
        \item このノルムは完備である.
    \end{enumerate}
\end{lemma}

\subsection{Morrey–Campanato空間}



\subsection{群上の関数}

\begin{example}[群上のBanach algebra]
    特別なクラスである.
    \begin{enumerate}
        \item 局所コンパクト群$G$上のRadon測度$(C(G))^*$はBanach環をなす.積は測度の畳み込みとする.
        \item Lebesgue空間$L^1(\R)$は畳み込みを積として非単位的なBanach代数をなす.単位元はDirac関数に相当する.
        $\R$は一般の局所コンパクトハウスドルフな位相群$G$,Lebesgue測度はHaar測度に一般化出来る.
        $L^1(G)$が単位的であることは,$G$が離散群であることに同値.
    \end{enumerate}
\end{example}

\section{ノルム空間の性質と構成}

\subsection{ノルム空間の完備性の特徴付け}

\begin{tcolorbox}[colframe=ForestGreen, colback=ForestGreen!10!white,breakable,colbacktitle=ForestGreen!40!white,coltitle=black,fonttitle=\bfseries\sffamily,
title=]
    これまではノルム空間論であった.線形写像が連続であることは有界性と同値だが,線形空間が完備であることは何を引き起こすか?
    実は,絶対収束級数というものの見方は,Cauchy列と表裏一体である.

    自己射の不動点は,作用している群に関する不変量の概念の卵である.
\end{tcolorbox}

\begin{theorem}[絶対収束級数なるクラスの定義]
    ノルム空間$X$と任意の列$(x_n)$について,次の2条件は同値.
    \begin{enumerate}
        \item $(x_n)$が定める級数が有界$\sum^\infty_{n=1}\norm{x_n}<\infty$ならば収束する.
        \item $X$は完備である.
    \end{enumerate}
\end{theorem}
\begin{Proof}\mbox{}
    \begin{description}
        \item[(2)$\Rightarrow$(1)] 絶対収束級数はCauchy列を定めるため.
        \item[(1)$\Rightarrow$(2)] Cauchy列$(x_n)$を任意に取る.
        $\norm{x'_{n+1}-x'_n}\le\frac{1}{2^n}$を満たす部分列$(x'_n)$が取れる.
        これについて,
        \[\sum_{n=1}^\infty\norm{x'_n}\le\norm{x'_1}+\sum_{n=1}^\infty\norm{x'_{n+1}-x'_n}<\infty\]
        より,部分列$(x'_n)$は収束する.よって,元のCauchy列$(x_n)$も収束する.
    \end{description}
\end{Proof}

\begin{theorem}[不動点定理]
    $X$をBanach空間,$Y$をその閉部分空間とする.自己写像$f:Y\to Y$が縮小写像ならば($1$より小さいLipschitz定数を持つならば),$Y$内に不動点が一意的に存在する.
\end{theorem}

\subsection{ノルム空間のテンソル積}

\begin{tcolorbox}[colframe=ForestGreen, colback=ForestGreen!10!white,breakable,colbacktitle=ForestGreen!40!white,coltitle=black,fonttitle=\bfseries\sffamily,
title=]
    Banach空間の台となる線型空間のテンソル積にノルムを入れる方法はいくつかある.
    projectiveとinjectiveの2つが代表的である.
    $\wh{\otimes}_\pi$によってBanは閉な対称モノイダル圏となる.
    Hilbの場合はさらに複数のノルムによって,対称モノイダル圏になる.
\end{tcolorbox}

\begin{definition}
    $V\otimes W$上のクロスノルム$\chi$とは,次の2条件を満たすものを言う:
    \begin{enumerate}
        \item $\forall_{v\in V,w\in W}\;\norm{v\otimes w}_\chi=\norm{v}_V\norm{w}_W$.
        \item $\forall_{\lambda\in V^*,\mu\in W^*}\;\norm{\lambda\otimes\mu}_{\chi^*}=\norm{\lambda}_{V^*}\norm{\mu}_{W^*}$.
    \end{enumerate}
\end{definition}

\begin{definition}[projective tensor product of Banach space]
    2つのBanach空間$X,Y$について,線型空間としてのテンソル積$V\otimes W$上に\textbf{射影的クロスノルム}
    \[\Norm{x}_\pi:=\inf\Brace{\sum_i\abs{\al_i}\norm{v_i}_V\norm{w_i}_W\in\R_{\ge 0}\;\middle|\; x=\sum_i\al_iv_iw_i}.\]
    を入れて完備化して得たものを$V\hat{\otimes}_\pi W$といい,\textbf{射影的テンソル積}という.
\end{definition}

\begin{definition}[injective tensor product]
    $\lambda,\mu$をそれぞれ$V,W$上の線型汎関数とする.
    テンソル積$V\otimes W$をノルム
    \[\norm{x}_\ep:=\sup\Brace{\abs{(\lambda\otimes\mu)(x)}\in\R_{\ge 0}\;\middle|\;\norm{\lambda}_{V^*},\norm{\mu}_{W^*}\le 1}\]
    について完備化したもの$V\hat{\otimes}_\ep W$を\textbf{入射的テンソル積}という.
\end{definition}

\begin{definition}[tensor product of Hilbert space]
    $V,W$をHilbert空間とする.
    $\brac{v_1w_1,v_2w_2}:=\brac{v_1,v_2}\brac{w_1,w_2}$によって定まる内積が定めるノルム$\norm{x}_\sigma$に関する内積空間$V\otimes W$の完備化を,テンソル積$V\hat{\otimes}_\sigma W$という.
\end{definition}

\begin{proposition}[入射的・射影的クロスノルムの特徴付け (2002)]
    $V,W$をBanach空間,$\chi$を$V\otimes W$上のノルムとする.次の2条件は同値.
    \begin{enumerate}
        \item $\chi$はクロスノルムである.
        \item $\forall_{x\in V\otimes W}\;\norm{x}_\ep\le\norm{x}_\chi\le\norm{x}_\pi$.
    \end{enumerate}
\end{proposition}

\subsection{双線型写像}

\begin{tcolorbox}[colframe=ForestGreen, colback=ForestGreen!10!white,breakable,colbacktitle=ForestGreen!40!white,coltitle=black,fonttitle=\bfseries\sffamily,
title=]
    積位相の普遍性から,位相線型空間の積からの双線型写像が連続ならば,各成分毎にも連続である.
    この逆は一般のノルム空間について成り立つ.
    すなわち,$\BIL(X\times Y)\simeq B(X,Y^*)\simeq B(Y,X^*)$.
\end{tcolorbox}

\begin{theorem}
    $X,Y,Z\in\TVS$について,
    $B:X\times Y\to Z$は双線型で,それぞれの引数について連続であるとする.
    $X$が$F$-空間ならば,任意の積空間$X\times Y$の収束列$((x_n,y_n))$について,$B$による像も$Z$上の収束列を定める.
    特に,$Y$も距離化可能ならば,$B$は連続である.
\end{theorem}

\subsection{射影的テンソル積の普遍性}

\begin{definition}[bounded linear operator]
    Banach空間の間の写像$B:X\times Y\to Z$が\textbf{有界双線型作用素}であるとは,$\forall_{y\in Y}\;B(-,y)\in B(X,Z)\land\forall_{x\in X}\;B(x,-)\in B(Y,Z)$が成り立つことをいう.
    この全体を$\BIL(X\times Y,Z)$で表す.$Z=\bF$のとき,$\BIL(X\times Y)$と表す.
\end{definition}

\begin{proposition}[有界双線型作用素の空間]\mbox{}
    \begin{enumerate}
        \item $\sup_{x\in B_X,y\in B_Y}\norm{B(x,y)}<\infty$が成り立つ.
        \item これをノルムにして,$\BIL(X\times Y,Z)$はBanach空間となる.
    \end{enumerate}
\end{proposition}

\begin{proposition}
    $B\in\BIL(X\times Y)$に対して,$S_B\in B(Y,X^*),T_B\in B(X,Y^*)$を
    \[\brac{x,S_B(y)}=B(x,y)=\brac{y,T_B(x)}\]
    で定める.
    \begin{enumerate}
        \item $S_-:\BIL(X\times Y)\to B(Y,X^*)$はBanの同型である.
        \item $T_-:\BIL(X\times Y)\to B(X,Y^*)$はBanの同型である.
    \end{enumerate}
\end{proposition}

\begin{proposition}[tensor product of elements]\mbox{}
    \begin{enumerate}
        \item ある$\otimes\in\BIL(X\times Y,\BIL(X\times Y)^*)$が存在して,$\forall_{B\in\BIL(X\times Y)}\;\forall_{x\in X,y\in Y}\;\brac{B,x\otimes y}=B(x,y)$を満たす.
        \item ある$\otimes\in\BIL(X^*\times Y^*,\BIL(X\times Y))$が存在して,$\forall_{(\varphi,\psi)\in X^*\times Y^*}\;\forall_{x\in X,y\in Y}\;\varphi\otimes\psi(x,y)=\varphi(x)\psi(y)$を満たす.
        \item さらに,$\norm{x\otimes y}=\norm{x}\norm{y}$かつ$\norm{\varphi\otimes\psi}=\norm{\varphi}\norm{\psi}$が成り立つ.
    \end{enumerate}
\end{proposition}

\begin{definition}[projective tensor product]
    $X\otimes Y\subset(\BIL(X\times Y))^*$をノルム閉な部分空間であって,$\{x\times y\}_{(x,y)\in X\times Y}$によって生成されるものとする.
    $\varphi\in(X\otimes Y)^*,B\in\BIL(X\times Y)$に対して,$B_\varphi(x,y):=\varphi(x\otimes y),\varphi_B(a):=\brac{a,B}$と定める.
\end{definition}

\begin{proposition}[射影的テンソル積の普遍性]\mbox{}
    \begin{enumerate}
        \item $\varphi\mapsto B_\varphi,B\mapsto\varphi_B$は互いに逆であり,$(X\otimes Y)^*\simeq_\Ban\BIL(X\times Y)$を導く.
        \item $\forall_{Z\in\Ban}\;\forall_{B\in\BIL(X\times Y,Z)}\;\exists!_{\wt{B}\in B(X\otimes Y,Z)}\forall_{x\in X,y\in Y}\;\wt{B}(x\otimes y)=B(x,y)$.
    \end{enumerate}
\end{proposition}

\subsection{商ノルム空間}

\begin{tcolorbox}[colframe=ForestGreen, colback=ForestGreen!10!white,breakable,colbacktitle=ForestGreen!40!white,coltitle=black,fonttitle=\bfseries\sffamily,
title=]
    ノルム空間での商は,閉部分空間に対してのみ達成される.その
    商ノルム空間では,新しい原点$Y$への最短距離をノルムとする.
    すると,そのままBanの商空間の定義にもなる(商写像は計量写像(short map)である).
    完備性の遺伝については,環論で見た完全列を通じた議論だ,双対性を感じる.
\end{tcolorbox}

\begin{proposition}[商空間]\label{prop-quotient-Banach-space}
    $Y\subset X$をノルム空間の部分空間とし,$Q:X\epi X/Y$を商空間への商写像とする.
    \begin{enumerate}
        \item 商ノルム$\norm{Qx}:=\inf\Brace{\norm{x-y}\in\R\mid y\in Y}$は代表元$x+Y$の取り方に依らず,$X/Y$上にセミノルムを定める.
        \item $Y$が$X$のノルム閉集合であることと,これがノルムとなることは同値.\footnote{定める位相がHausdorffであるかの議論と全くパラレルだ.}
        \item $Q$はノルム減少的である:$\norm{Qx}\le\norm{x}$.
        \item $X$がBanach空間で$Y$がその閉部分空間であるとき,$X/Y$もBanach空間である.
        \item $X$がBanach空間で$Y$がその閉部分空間であるとき,$Q$は開写像である.
        \item この商ノルムが定める位相は,$Q$が定める終位相に一致する.
    \end{enumerate}
\end{proposition}
\begin{Proof}\mbox{}
    \begin{enumerate}
        \item まず,商ノルムの定義から次が成り立つ:
        \begin{align*}
            \forall_{x_1,x_2\in X}\;\forall_{\ep>0}\;\exists_{y_1,y_2\in Y}\quad\norm{Qx_1}+\norm{Qx_2}+\ep&\ge\norm{x_1-y_1}+\norm{x_2-y_2}\\
            &\ge\norm{(x_1+x_2)-(y_1+y_2)}\ge\norm{Q(x_1+x_2)}.
        \end{align*}
        よって,$\norm{Qx_1+Qx_2}\le\norm{Qx_1}+\norm{Qx_2}$からノルムの劣加法性,斉次性は商写像$Q$の線形性から従う.
        \item $\forall_{x\in X}\;\norm{Qx}=0\Leftrightarrow x\in\oo{Y}$であるが,これが$Qx=0+Y$すなわち$x\in Y$と同値になるための必要十分条件を導けば良い.
        これはあきらかに$Y=\oo{Y}$となることである.
        \item 明らか.
        \item $X/Y$のCauchy列$(z_n)$を取る.すると,$\norm{z'_{n+1}-z'_n}<2^{-n}$を満たす部分列が取れる.
        これに対して,$\norm{x_{n+1}-x_n}<2^{-n}$を満たす$x_n\in Q^{-1}(z'_n)$が存在する.実際,ある$x'_n\in X$について$Qx'_n=z'_n$であるが,ある$y\in Y$が存在して$\norm{x'_n-x_{n-1}-y}<2^{-n}$であるから,$x_n:=x'_n-y$と定めれば良いことと,帰納法により従う.
        よって$(x_n)$はCauchy列だから収束し,$Q$は連続だから$(z'_n)$も収束する.$(z_n)$はCauchy列だから,これも$Qx$に収束する.
        \item 開写像定理より.
        \item 連続全射が開写像ならば商写像である.というのも,実際,
        (3)より$\O_{X/Y}\subset Q_*\O_X$.逆に$\O_*\O_X\subset\O_{X/Y}$を示せば良いが,これは(5)からわかる.
    \end{enumerate}
\end{Proof}
\begin{remark}[連続延長の失敗と単位閉球のコンパクト性]
    商写像$Q$は$X$の単位開球$B_X^\circ$を$X/Y$の単位開球$B_{X/Y}^\circ$に全射に写すが,
    単位閉球$B$は一般にはそうとは限らない.
    実際,$z\in X/Y$が$\norm{z}=r$とは,$x\in Q^{-1}(z)$について$r=\inf\Brace{\norm{x-y}\ge 0\mid y\in Y}$ということだから,$r<1$ならば,ある$y'\in Y$について$\norm{x-y'}<1$と逆像を見つけることができるが,
    $r=1$の場合は逆像が単位閉球内にみつかるとは限らない.
    実際,一般の作用素について\ref{lemma-image-of-ball}のような描像がある.境界上で特異的な振る舞いをする.
\end{remark}

\begin{proposition}[商空間の普遍性]
    $T\in B(X,Y)$と閉部分空間$Z\subset X$について,次の2条件は同値:
    \begin{enumerate}
        \item $Z\subset\Ker T$.
        \item 下図
        を可換にする$\o{T}$がただ一つ存在し,$\norm{\o{T}}=\norm{T}$を満たす.
        \[\xymatrix{
            X\ar[r]^-T\ar[d]_-Q&Y\\
            X/Z\ar@{.>}[ur]_-{\o{T}}
        }\]
    \end{enumerate}
\end{proposition}
\begin{Proof}
    $\o{T}$が$\norm{\o{T}}=\norm{T}$を満たすことを除いて,(1)$\Leftrightarrow$(2)の部分は位相空間の商空間の普遍性から従う.
    \begin{description}
        \item[$\norm{\o{T}}\le\norm{T}$] 任意の$z\in Z$について,
        \[\norm{\o{T}Qx}=\norm{T}=\norm{T(x-z)}\le\norm{T}\norm{x-z}\]
        より,$\norm{\o{T}Qx}\le\norm{T}\norm{Qx}$.
        \item[$\norm{\o{T}}\ge\norm{T}$] $Q$がノルム減少的であるため.
    \end{description}
\end{Proof}

\begin{proposition}
    上の命題の(1)が成り立つとする.
    このとき,次の2条件は同値.
    \begin{enumerate}
        \item $\o{T}$は開写像である.
        \item $T$は開写像である.
    \end{enumerate}
\end{proposition}

\subsection{ノルム空間の完全列}

\begin{proposition}[Banach空間の標準分解]
    ノルム空間$X$とその部分空間$Y$について,$Y$と$X/Y$がBanach空間ならば,$X$もBanach空間である.
\end{proposition}
\begin{Proof}
    $X$の任意のCauchy列$(x_n)$を取る.
    \begin{enumerate}[(a)]
        \item $Q$はノルム減少的だから,$(Qx_n)$は$X/Y$のCauchy列を定め,完備性より極限$\exists_{x\in X}\;Qx$を持つ.
        \item 商ノルムの定義から,$\forall_{n\in\N}\;\exists_{y_n\in Y}\;\norm{x_n-x-y_n}<1/n+\norm{Q(x_n-x)}$が成り立つから,こうして$Y$の列$(y_n)$が取れた.これはCauchy列である.よって,収束先$y\in Y$を持つ.
        \item $x_n\to x+y$となる.
    \end{enumerate}
\end{Proof}

\subsection{部分ノルム空間}

\begin{tcolorbox}[colframe=ForestGreen, colback=ForestGreen!10!white,breakable,colbacktitle=ForestGreen!40!white,coltitle=black,fonttitle=\bfseries\sffamily,
title=連続線形延長の算譜]
    Banach空間論も,有限の範囲では,$k^n$だと思って扱える.
    この消息を正しく捉えるには「稠密」がキーワードになる.
\end{tcolorbox}

\begin{proposition}[完備性の弱遺伝性]
    Banach空間$X$の部分空間$M$について,次の2条件は同値.
    \begin{enumerate}
        \item $M$は$X$のノルムについてBanach空間である.
        \item $M$は閉である.
    \end{enumerate}
    同様のことはHilbert空間とFrechet空間にも成り立つ.
\end{proposition}
\begin{Proof}
    完備な距離空間の完備な部分集合は,閉部分集合と同値であるため.
\end{Proof}

\subsection{有限次元性の特徴付け}

\begin{proposition}[自明なBanach部分空間]\label{prop-finite-subspaces}
    位相線形空間$X$の任意の有限次元部分空間$Y$はBanach空間であり,特に閉である.
    また,$\dim Y=n$ならば,任意の線型同型$k^n\iso Y$は位相同型でもある.
\end{proposition}

\begin{corollary}
    $X,Y$をノルム空間,$Y$は有限次元とする.次の2条件は同値;
    \begin{enumerate}
        \item 作用素$T:X\to Y$は連続.
        \item $\Ker T$は閉部分空間である.
    \end{enumerate}
\end{corollary}

\begin{proposition}
    ノルム空間$X$について,次の2条件は同値:
    \begin{enumerate}
        \item $X$は有限次元である.
        \item $B$はノルム位相についてコンパクトである.
    \end{enumerate}
\end{proposition}

\subsection{線形作用素の有界延長定理}

\begin{tcolorbox}[colframe=ForestGreen, colback=ForestGreen!10!white,breakable,colbacktitle=ForestGreen!40!white,coltitle=black,fonttitle=\bfseries\sffamily,
title=]
    距離空間は正規であるから,任意の閉集合上に定められた有界連続関数は,一様ノルムを保ったまま延長する(Tietze).
    実はこれは$C_b$上の一様ノルムに限らず,任意のBanach空間上でも成り立つが,稠密部分空間に限る.
\end{tcolorbox}

\begin{proposition}[Bounded Linear Transformation theorem]\label{prop-extension-of-operator-on-dense-subset}
    $X,Y$をBanach空間,$X_0$を$X$の稠密な部分空間とする.このとき,任意の作用素$T_0\in B(X_0,Y)$は一意的な延長$T\in B(X,Y)$をもち,さらにこれは$\norm{T}=\norm{T_0}$を満たす.
\end{proposition}
\begin{Proof}\mbox{}
    \begin{enumerate}[(a)]
        \item 任意の$x\in X$に対して,これに収束する$X_0$の列$(x_n)$が取れる.$(T_0x_n)$も収束列だから,極限$y\in Y$を持つ.これに対して,$Tx:=y$と定める.収束先の一意性より,$(x_n)$の取り方に依らない.
        \item すると,$T$は$T_0$の延長になっている.$T$は$X$上線型であることは,極限の線形性と$T_0$の線形性から従う.
        \item さらに有界性は,$\norm{T_0}\le\norm{T}$は明らかだが,逆$\norm{T}=\sup_{x\in B}\norm{Tx}\le\norm{T_0}$も,$X_0$の稠密性からわかる.
    \end{enumerate}
\end{Proof}

\begin{proposition}
    任意のノルム空間$X$について,Banaxh空間$\o{X}$であって,$X$を稠密な部分空間として含むものが同型を除いて一意的に存在する.
\end{proposition}
\begin{Proof}\mbox{}
    \begin{description}
        \item[一意性] 
        \item[存在] 再双対空間\ref{def-bidual}への包含$X\mono X^{**}$の像の閉包を取れば良い.
    \end{description}
\end{Proof}

\subsection{ノルム空間の直和と直積}

\begin{tcolorbox}[colframe=ForestGreen, colback=ForestGreen!10!white,breakable,colbacktitle=ForestGreen!40!white,coltitle=black,fonttitle=\bfseries\sffamily,
title=]
    直積には$l^\infty$-直和,直和には$l^1$-直和を採用すると,これがBanの普遍構成である.
    一方で,構成論的には代数的直和の完備化として捉えられることが肝要になる.

    Hilbert空間の直和は,$l^2$-直和をいう.
\end{tcolorbox}

\subsubsection{3つの直和}

\begin{definition}
    $W\subset\Ban$をBanach空間の族とする.
    \begin{enumerate}
        \item $l^p$-直和とは,$\bigoplus_{i}^pW_i:=\Brace{(w_i)_i\in\prod_{i}W_i\;\middle|\;\sqrt[q]{\sum_i\norm{w_i}^p}<\infty}$.
        \item $l^\infty$-直和とは,$\bigoplus_{i}^pW_i:=\Brace{(w_i)_i\in\prod_{i}W_i\;\middle|\;\sup_{i}\norm{w_i}<\infty}$.
        \item $l^1$-直和とは,$\bigoplus_{i}^pW_i:=\Brace{(w_i)_i\in\prod_{i}W_i\;\middle|\;\sum_i\norm{w_i}<\infty}$.
    \end{enumerate}
\end{definition}

\begin{definition}
    Banach空間の$l^\infty$-直和を\textbf{直積}といい,$l^1$-直和を\textbf{直和}という.
    これは実際Ban上の直和・直積の普遍性を満たす.
\end{definition}

\begin{proposition}[well-definedness]
    Banach空間の族の直積・直和はたしかにBanach空間である.
\end{proposition}

\begin{example}
    $\forall_{j\in J}\;X_j=\bF$の場合を考える.
    \begin{enumerate}
        \item 直積を$l^\infty(J):=\prod_{j\in J}\bF$と表すと,これは$J$上の有界関数の空間となる.
        \item $l^p$-直和について$l^p(J)=L^p(J)$が成り立つ,ただし$J$の測度は数え上げ測度とした.
        \item 直和を$c_0(J):=\coprod_{j\in J}\bF$と表すと,$J$に離散位相を入れたときの$C_0(J)$に等しい.
    \end{enumerate}
    $J=\N$のとき,$(J)$の部分が省略される.
\end{example}

\subsubsection{代数的直和の完備化としての特徴付け}

\begin{tcolorbox}[colframe=ForestGreen, colback=ForestGreen!10!white,breakable,colbacktitle=ForestGreen!40!white,coltitle=black,fonttitle=\bfseries\sffamily,
title=]
    $l^1$-直和は,代数的直和空間(加群としての直和)の完備化として特徴付けられる.
\end{tcolorbox}

\begin{definition}
    ノルム空間の族$(X_j)$に対して,
    \[\sum_{j\in J}X_j:=\Brace{x\in\prod_{j\in J}X_j\;\middle|\;\pr_j(x)=0\;\fe}\]
    を\textbf{代数的直和}という.ここには明らかに,$1\le p\le\infty$について,$p$-ノルムが入る.
\end{definition}

\begin{proposition}[well-definedness]\label{prop-completion-of-algebraic-direct-product}
    Banach空間の族$(X_j)$について,代数的直和空間$\sum_{j\in J}X_j$を考える.
    \begin{enumerate}
        \item $p\in[1,\infty)$のとき,$\sum_{j\in J}X_j$の完備化は$\Brace{x\in\prod_{j\in J}X_j\;\middle|\;\sum_{j\in J}\norm{\pr_j(x)}^p<\infty}$に一致する.
        \item $p=\infty$のとき,$\sum_{j\in J}X_j$の完備化は$\Brace{x\in\prod_{j\in J}X_j\;\middle|\;\norm{\pr_-(x)}:J\to\R \in C_0(J),ただしJは離散空間とする}$に一致する.
    \end{enumerate}
\end{proposition}
\begin{remarks}
    $\sum_{j\in J}X_j$の$1$-ノルムに関する完備化は,$l^1$-直和にほかならないことがわかる.これを\textbf{直和}という.
\end{remarks}

\subsection{ノルム位相の性質}

\begin{proposition}[有限次元空間の特徴付け]\label{prop-unit-ball-in-normed-space}
    ノルム空間$X$の閉単位球$B$について,次の2条件は同値.
    \begin{enumerate}
        \item $B$はノルム位相についてコンパクトである.
        \item $X$は有限次元である.
    \end{enumerate}
\end{proposition}
\begin{remark}
    一方で,弱位相についてはコンパクトになる.
\end{remark}

\section{Baireの定理の帰結}

\begin{tcolorbox}[colframe=ForestGreen, colback=ForestGreen!10!white,breakable,colbacktitle=ForestGreen!40!white,coltitle=black,fonttitle=\bfseries\sffamily,
title=]
    ここではノルム空間$X,Y$はBanachであるとして,そのノルム位相の性質を調べる.
    閉グラフ定理を除いては,複素解析学の世界の一般化にも見える.
\end{tcolorbox}

\subsection{Baireの定理}


\begin{definition}[nowhere dense, first category, Baire space]\mbox{}
    \begin{enumerate}
        \item 閉包が内点を持たない集合$A$を\textbf{疎集合}という:$\o{X\setminus\o{A}}=X$.
        集合$A$が疎であることと,その補集合$X\setminus A$が稠密であることは同値.
        \item 可算個の疎集合の合併として表せる集合を\textbf{第一類}という.
        \item そうでない集合を\textbf{第二類}という.
        これは,任意の稠密開集合の可算共通部分は稠密であることに同値.これが成り立つことを\textbf{Baire空間}であるという.
    \end{enumerate}
\end{definition}

\begin{theorem}[Baire空間の特徴付け]
    位相空間$X$について,次の3条件は同値.
    \begin{enumerate}
        \item $X$はBaire空間である:任意の稠密開集合の可算共通部分は稠密である.
        \item $X$の任意の非空開集合は第二類である:可算個の疎集合の合併としては表せない.
        \item $X$の閉集合列$(F_i)_{i\in\N}$が$\paren{\cup_{i\in\N}F_i}^\circ\ne\emptyset$ならば,$\exists_{i\in\N}\;F_i^\circ\ne\emptyset$.
    \end{enumerate}
\end{theorem}
\begin{Proof}\mbox{}
    \begin{description}
        \item[(1)$\Rightarrow$(2)] $X$の任意の非空開集合$H\in\O_X$が,$X$の疎集合の合併$H=\cup_{n\in\N}N_n$で表せると仮定して矛盾を示す.
        各$N_n$は疎だから,$X\setminus\o{N_n}$は$X$上稠密である.
        すると,$\cap_{n\in\N}X\setminus\o{N_n}\subset X\setminus H$も稠密である必要があるから,$X=X\setminus H$.これは$H\ne\emptyset$に矛盾.
        \item[(2)$\Rightarrow$(3)] $\paren{\cup_{i\in\N}F_i}^\circ=:H\in\O_X$とおき,$\forall_{i\in\N}F_i^\circ=\emptyset$と仮定して矛盾を示す.すると,各$F_i$は疎だから,$H=\cup_{i\in\N}H\cap F_i$も素.よって,$H=\emptyset$となり,矛盾.
        \item[(3)$\Rightarrow$(1)] $(G_i)$を$X$の稠密開集合の列とする.$F_i:=X\setminus G_i$とすると,$(\cup_{i\in\N}F_i)^\circ=\emptyset$だから,結論が従う.
    \end{description}
\end{Proof}

\begin{proposition}[Baire's theorem]
    完備距離空間$(X,d)$はBaire空間である.
\end{proposition}
\begin{Proof}
    任意の稠密開集合の列$(A_n)$を取る.
    $\cap_{n\in\N}A_n$が任意の半径$r>0$の閉球$B_0$と非空な共通部分を持つならば,稠密であると示せる.
    そこで,任意の閉球$B_0:=B(a,r)\;(a\in X,r>0)$を取る.
    \begin{enumerate}[(a)]
        \item 閉球の列$(B_n)$を次のように構成する.$A_1$は稠密より,非空開集合$A_1\cap B_0^\circ$には半径$r/2$より小さい閉球$B_1$が取れる.これを繰り返すと,$\diam(B_n)<2^{-n}r$が成り立つ.
        \item $(x_n)\in\prod_{n\in\N}B_n$はCauchy列だから極限$x\in X$を持ち,また$\{x\}=\cap B_n$が成り立つ.いま,$\cap B_n\subset B_0\cap\paren{\cap A_n}$である.
    \end{enumerate}
\end{Proof}

\begin{corollary}[至る所微分不可能な関数の存在]
    $X:=C([0,1])$を$\infty$-ノルムによってBanach空間とみなし,
    \[\F_n:=\Brace{f\in X\mid\exists_{x\in[0,1]}\;\forall_{y\in[0,1]}\;\abs{f(x)-f(y)}\le n\abs{x-y}}\]
    と定める.
    \begin{enumerate}
        \item $\F_n$は$X$の閉な疎集合である.
        \item $X$には,$[0,1]$上至る所微分不可能な関数からなる稠密部分集合が存在する.
    \end{enumerate}
\end{corollary}

\subsection{開写像定理}

\begin{tcolorbox}[colframe=ForestGreen, colback=ForestGreen!10!white,breakable,colbacktitle=ForestGreen!40!white,coltitle=black,fonttitle=\bfseries\sffamily,
title=]
    開球をどう写すかを考える.
    実は,一般に$F$-空間の間の全射は開写像である.
\end{tcolorbox}

\begin{lemma}[単位閉球の像で特異的な振る舞いをするのは境界部分のみである]\label{lemma-image-of-ball}
    $X,Y$をBanach空間とする.
    $T\in B(X,Y)$による単位閉球$B_X(0,1)$の像が,$Y$のある球$B_Y(0,r)\;(r>0)$の中で稠密であるとする:$B_Y(0,r)\subset\o{T(B_X(0,1))}$.この時,$\forall_{\ep\in(0,1)}\; B_Y(0,(1-\ep)r)\subset T(B_X(0,1))$.
\end{lemma}
\begin{Proof}\mbox{}
    \begin{description}
        \item[方針] $A:=T(B(0,1))$と表すと,これは$B(0,r)$上稠密である.任意の$y\in B(0,r)$と$\ep\in(0,1)$を取り,$y\in(1-\ep)^{-1}A$が従うことを示せば良い.
        \item[構成] まず,$y$に収束する$A$の列$(y_n)$を構成する.$A$は$B(0,r)$上稠密だから,$\exists_{y_1\in A}\;\norm{y-y_1}<\ep r$を満たす.
        次に,$y-y_1\in B(0,\ep r)$であることに注目すると,$\ep A$はこの上で稠密だから,$\exists_{y_2\in\ep A}\;\norm{y-y_1-y_2}<\ep^2r$を満たす.
        これを繰り返すことで,$y_n\in\ep^{n-1}A,\Norm{y-\sum^n_{k=1}y_k}<\ep^nr$を満たす$A$の列$(y_n)$が取れ,対応する$X$の列$(x_n)$が$\norm{x_n}\le\ep^{n-1},Tx_n=y_n$を満たすように取れる.
        \item[証明] するとこの列$(x_n)$は絶対収束級数$x:=\sum_{n\in\N}x_n$を定めるが,$Tx=y$であり,また$\norm{x}\le\sum_{n\in\N}\ep^{n-1}=(1-\ep)^{-1}$を満たすから,$y\in(1-\ep)^{-1}A$.
    \end{description}
\end{Proof}

\begin{theorem}[開写像定理]\label{thm-open-mapping-theorem}
    Banach空間の間の全射な有界作用素$T\in B(X,Y)$は開である.
\end{theorem}
\begin{Proof}\mbox{}
    \begin{description}
        \item[方針] $T$の線形性と,$X$の位相は開球を基として生成されることより,$T(B(0,1))$が$0$を内点に持つことを示せれば十分である.
        実際このとき,任意の開集合の基底$U(x,r)$について,$B(x,\delta)\subset T(B(x,r))\subset\oo{T(U(x,r))}$より,$U(x,\delta)\subset T(U(x,r))$が取れることがわかる.
        他の点についても,$X,Y$の各点の等質性より従う.\footnote{原点に引き戻して考えるのは,位相群と同じ.}
        \item[証明] 全射性より,$Y=T(X)=\cup_{n\in\N}\oo{T(B(0,n))}$.
        Baireの定理より,$\exists_{n\in\N}\;B(y,\ep)\subset\oo{T(B(0,n))}$.
        よって,$T(B(0,n))$は$B(y,\ep)$上稠密,
        $T(B(0,1))$は$B(y/n,\ep/n)$上稠密である.
        $2B(0,\ep/n)\subset B(y/n,\ep/n)-B(y/n,\ep/n)$と,像$T(B(0,1))$が対称凸であることより,$B(0,\ep/n)$上稠密でもある.
        よって補題より,$\forall_{\delta\in(0,\ep/n)}\;B(0,\delta)\subset T(B(0,1))$.
    \end{description}
\end{Proof}

\begin{theorem}[一般化された開写像定理]
    $X$を$F$-空間,$Y$を位相線型空間,$T:X\to Y$を連続作用素とする.像$T(X)\subset Y$が第二類であるならば,次が成り立つ:
    \begin{enumerate}
        \item $T$は全射である;$T(X)=Y$.
        \item $T$は開写像である.
        \item $Y$も$F$-空間である.
    \end{enumerate}
\end{theorem}

\subsection{閉部分空間の和}

\begin{tcolorbox}[colframe=ForestGreen, colback=ForestGreen!10!white,breakable,colbacktitle=ForestGreen!40!white,coltitle=black,fonttitle=\bfseries\sffamily,
title=]
    開写像定理から,閉部分空間の和の消息が聞こえる.
    Hilbert空間の任意の閉部分空間は補空間を持つが,Banach空間ではそうとは限らない.
\end{tcolorbox}

\begin{theorem}
    $E$をBanach空間,$G,L\subset E$を閉部分空間で,$G+L$も閉となるとする.
    このとき,$\exists_{C\ge0}\;\forall_{z\in G+L}\;\exists_{x\in G,y\in L}\;\norm{x}\le C\norm{z}\land\norm{y}\le C\norm{z}\land z=x+y$.
\end{theorem}

\begin{corollary}
    2つの閉部分空間$G,L\subset E$について,次の2条件は同値:
    \begin{enumerate}
        \item $G+L$も閉部分空間である.
        \item $\exists_{C\in\R}\;\forall_{x\in E}\;\dist(x,G\cap L)\le C(\dist(x,G)+\dist(x,L))$.
    \end{enumerate}
\end{corollary}

\begin{corollary}[Banach空間の補空間分解]
    $Y,Z\subset X$をBanach空間の閉部分空間とする.次の2条件は同値で,これを満たす$Z$を$Y$の\textbf{(位相的)補空間}といい,$X=Y\oplus Z$で表す.
    \begin{enumerate}
        \item 任意の$x\in X$は一意な分解$x=y+z\;(y\in Y,z\in Z)$を持つ.
        \item $Y+Z=X$かつ$Y\cap Z=0$.
    \end{enumerate}
    またこのとき,$\exists_{\al>0}\;\norm{y}+\norm{z}\le\al\norm{x}$.
\end{corollary}

\begin{proposition}[補空間分解と射影]
    $Y,Z\subset X$のBanach空間の閉部分空間とする.
    このとき,次の2条件は同値:(実は$F$-空間に成り立つ)
    \begin{enumerate}
        \item 射影$P:X\to Y$は連続である.
        \item $Y=P(X)$と$Z=\Ker P$は互いに補空間である.
    \end{enumerate}
    またこの条件を満たすとき,次の2条件も同値:
    \begin{enumerate}
        \item $T\in B(X)$は$P$と可換である.
        \item $P$の定める分解のそれぞれが$T$-安定:$T(Y)\subset Y$かつ$T(Z)\subset Z$.
    \end{enumerate}
    (2)は上三角行列であることを意味している.
\end{proposition}

\begin{remarks}
    Lindenstrauss, J., and Tzafriri, L.によると,Hilbert空間と同型でない任意のBanach空間は,補空間を持たない閉部分空間を持つ.
\end{remarks}

\subsection{逆写像定理と部分逆}

\begin{tcolorbox}[colframe=ForestGreen, colback=ForestGreen!10!white,breakable,colbacktitle=ForestGreen!40!white,coltitle=black,fonttitle=\bfseries\sffamily,
title=]
    $F$-空間には,本質的にはノルムは1つしかはいらない.
    特に,確率変数の空間$L^2(\Om)$のノルム位相と,$L^1(\Om)$の相対位相とは一致する($L^2$は$L^1$の閉部分空間ならば).
    $L^2$-収束するならば$L^1$-収束するから,少なくとも$L^1$-位相の方が弱いが,$L^2(\Om)$-が$L^1$-完備ならば逆も言える.
    だがこの前提は違うのか??
\end{tcolorbox}

\begin{corollary}[逆写像定理]\label{cor-inverse-mapping-theorem}
    Banach空間の間の任意の全単射な有界作用素の逆写像は,有界作用素である.
\end{corollary}
\begin{Proof}
    全単射線型写像の逆も線型であるから,位相同型であることを示せば良い.
    全単射な開写像は位相同型である.
\end{Proof}

\begin{corollary}[ノルムが同値であることの十分条件]
    線型空間$X$が,2つのノルム$\norm{-}_1,\norm{-}_2$についてBanach空間をなし,$\exists_{\al>0}\;\norm{-}_1\le\al\norm{-}_2$が成り立つとする.
    この時,$\beta>0$が存在して,$\norm{-}_2\le\beta\norm{-}_1$も満たす.
\end{corollary}
\begin{Proof}
    $X_1:=(X,\norm{-}_1),X_2:=(X,\norm{-}_2)$とすると,$\id:X_2\to X_1$はLipschitz連続であるから全単射な有界作用素であるから,有界な逆を持つ.これはLipschitz連続であることと同値.よって,$X_1\simeq X_2$.
\end{Proof}
\begin{remarks}
    2つの位相$(X,\tau_1),(X,\tau_2)$について,大小関係$\tau_1\subset\tau_2$があり,いずれも$F$-空間ならば,$\tau_1=\tau_2$である.
\end{remarks}

\begin{theorem}
    $E,F$をBanach空間,
    $T\in B(E,F)$を全射とする.このとき,次の2
    条件は同値:
    \begin{enumerate}
        \item $T$は右逆を持つ(エピ射である).
        \item 核$\Ker T$は$E$内に補空間を持つ.
    \end{enumerate}
    また,単射$T\in B(E,F)$について次の2条件も同値:
    \begin{enumerate}
        \item $T$は左逆を持つ(モノ射である).
        \item $\Im T$は閉で,$F$内に補空間を持つ.
    \end{enumerate}
\end{theorem}
\begin{remarks}
    Hilbではエピ射と全射は一致するが,Banではそうとは限らない!
\end{remarks}

\subsection{閉グラフ定理}

\begin{tcolorbox}[colframe=ForestGreen, colback=ForestGreen!10!white,breakable,colbacktitle=ForestGreen!40!white,coltitle=black,fonttitle=\bfseries\sffamily,
title=全空間で定義された閉作用素は有界である]
    位相空間$X$がHausdorffであることと,任意の位相空間$T$からの連続写像$f:T\to X$は閉なグラフを持つこととは同値である.
    コンパクトハウスドルフ空間の間の写像は,グラフが閉ならば連続である.
    グラフが閉である作用素は\textbf{閉作用素}という.
    グラフが閉で,全空間$X$上で定義された線型作用素は有界であるが,部分集合$D(T)\subsetneq X$上で定義された線型作用素については一般には有界とは限らない.
\end{tcolorbox}

\begin{theorem}[closed graph theorem]\label{thm-closed-graph-theorem}
    作用素$T:X\to Y$のグラフ$G(T)\subset X\times Y$が閉ならば,$T$は有界である.
\end{theorem}
\begin{Proof}
    仮定より,$G(T)$は$X\times Y$内の閉部分空間をなす,特にBanachである.
    このとき,$\pr_1,\pr_2$はいずれもノルム減少的であるから,特に有界である.また$\pr_1$は全単射を定めるから,逆写像定理より,有界な逆$\pr_1^{-1}\in B(X,G(T))$を持つ.
    $T=\pr_2\circ\pr_1^{-1}$より,$T$も有界.
\end{Proof}

\subsection{Banach-Steinhausの定理}

\begin{tcolorbox}[colframe=ForestGreen, colback=ForestGreen!10!white,breakable,colbacktitle=ForestGreen!40!white,coltitle=black,fonttitle=\bfseries\sffamily,
title=]
    $B(X,Y)$のノルム有界集合と同程度連続な有界作用素族とは同値.
\end{tcolorbox}

\begin{definition}
    $X,Y$を位相線型空間,$\Gamma\subset\Hom_\bF(X,Y)$を集合とする.
    $\Gamma$が\textbf{同程度連続}とは,$\forall_{W\in\O(0_Y)}\;\exists_{V\in\O(0_X)}\;\forall_{T\in\Gamma}\;T(V)\subset W$が成り立つことをいう.
\end{definition}

\begin{theorem}[同程度連続ならば一様有界]
    $X,Y\in\TVS,\Gamma\subset\Hom_\bF(X,Y)$を同程度連続な線型写像集合とする.このとき,次の意味で一様有界である:$\forall_{E\in\B(X)}\;\exists_{F\in\B(Y)}\;\forall_{T\in\Gamma}\;T(E)\subset F$.
    ただし,$\B(E),\B(F)$は有界集合系とした.
\end{theorem}

\begin{theorem}[Banach-Steinhaus]
    $X,Y\in\TVS,\Gamma\subset\Hom_\TVS(X,Y)$を連続線型写像の集合,$B:=\Brace{x\in X\mid \{Tx\}_{T\in\Gamma}\in\B(X)}$とする.
    $B\subset X$がBaire空間ならば,$B=X$で,$\Gamma$は同程度連続である.
\end{theorem}

\begin{corollary}[各点有界ならば同程度連続]
    $X$を$F$-空間,$Y$を位相線型空間とし,$\Gamma\subset\Hom_\TVS(X,Y)$を連続線形写像の集合とする.
    任意の$\forall_{x\in X}\;\Gamma(x):=\Brace{Tx\in Y\mid T\in\Gamma}$が有界ならば,$\Gamma$は同程度連続である.
\end{corollary}

\begin{corollary}
    $X,Y$をBanach空間,$E\subset B(X,Y)$とする.次の2条件は同値:
    \begin{enumerate}
        \item $E$は同程度連続.
        \item $E$はノルム有界.
    \end{enumerate}
\end{corollary}

\subsection{一様有界性の原理}

\begin{tcolorbox}[colframe=ForestGreen, colback=ForestGreen!10!white,breakable,colbacktitle=ForestGreen!40!white,coltitle=black,fonttitle=\bfseries\sffamily,
title=]
    有界作用素の族が各点で有界ならば,同程度連続であり,一様有界であり,したがって作用素ノルムについても有界である.
\end{tcolorbox}

\begin{theorem}[作用素族は各点有界ならば(一様)有界]
    $X,Y$をBanach空間とする.\footnote{$X$が$F$-空間,$Y$が位相線型空間で十分.}
    任意の集合$\{T_\lambda\}\subset B(X,Y)$が
    各点有界:$\forall_{x\in X}\;\{T_\lambda x\}_{\lambda\in\Lambda}\in\B(Y)$ならば,
    有界集合である:$\sup_{\lambda\in\Lambda}\norm{T_\lambda}<\infty$.
\end{theorem}
\begin{Proof}
    $Y_\Lambda:=\prod_{\lambda\in\Lambda}Y$を直積空間,$T:=\prod_{\lambda\in\Lambda}T_\lambda:X\to Y_\Lambda$を積作用素とすると,
    仮定より$\forall_{x\in X}\;\norm{T_\lambda x}<\infty$だから,$T$はたしかにwell-definedである.
    この$T$が有界であることを示せば,積作用素の普遍性$\forall_{\lambda\in\Lambda}\;T_\lambda=\pr_\lambda\circ T$より,$\forall_{\lambda\in\Lambda}\;\norm{T_\lambda}\le\norm{T}<\infty$が従う.

    $(x,y)\in X\times Y_\Lambda$に収束する列$(x_n,Tx_n)$を任意にとり,$Tx=y$を示せば良い.
    各$\lambda\in\Lambda$について$T_\lambda$は連続であるから,$T_\lambda x=\pr_\lambda(y)$.これは$y=Tx$を意味する.
\end{Proof}

\subsection{作用素のネットが収束する十分条件}

\begin{corollary}[作用素ネットが収束するための十分条件]
    $B(X,Y)$のネット$(T_\lambda)_{\lambda\in\Lambda}$が,任意の$x\in X$について$Y$のネット$(T_\lambda x)_{\lambda\in\Lambda}$は有界で収束するならば,
    このネットは各点収束先を持つ:$\exists_{T\in B(X,Y)}\;\forall_{x\in X}\;T_\lambda x\to Tx$.
\end{corollary}
\begin{Proof}
    $Tx:=\lim_{\lambda\in\Lambda}T_\lambda x$によって定まる作用素$T:X\to Y$が有界であることを示せば良い.
    この$\{T_\lambda\}$は各点有界族だったから一様有界でもある:$\exists_{\alpha\in\R}\;\forall_{\lambda\in\Lambda}\;\norm{T_\lambda}\le\al$.
    したがって,$\forall_{x\in X}\;\norm{Tx}\le\al\norm{x}$.すなわち,$T$は有界である.
\end{Proof}
\begin{remarks}
    命題\ref{prop-extension-of-operator-on-dense-subset}より,$X$の稠密な部分集合上で任意のネットが有界で収束することを示せば十分.
    さらに$(T_n)$が数列のときは収束だけ確認すれば良い.
\end{remarks}

\begin{corollary}
    複素Banach空間$X:=\Brace{f\in C([0,2\pi];\C)\mid f(0)=f(2\pi)}$上の作用素$T_n\in B(X)$を
    \[T_nf:=\sum_{k=-n}^n\paren{\frac{1}{2\pi}\int^{2\pi}_0f(t)e_{-k}(t)dt} e_k,\qquad(e_k(x):=\exp(ikx))\]
    と定めると,
    \begin{enumerate}
        \item Dirichlet核$D_n$について,$T_nf=D_n*f$である:
        \[D_n(x)=\sum^n_{k=-n}e_k(x)=\frac{\sin((n+1/2)x)}{\sin(x/2)}.\]
        \item $\norm{T_n}=\norm{D_n}_1>4\pi^{-2}\sum^n_{k=1}k^{-1}$が成り立つ.
        \item $\norm{T_n}\to\infty$.
        \item Fourier級数が一様収束しないような$f\in X$が存在する.
        \item 集合$\Brace{f\in L^2([0,2\pi])\mid\lim_{n\to\infty}T_nf\text{が存在する}}$は$L^2([0,2\pi])$上稠密な第一類集合である.
    \end{enumerate}
\end{corollary}

\begin{corollary}
    $X:=L^1([0,2\pi])$とすると,Riemann-Lebesgueの補題より,作用素$T:X\to c_0(\Z)$が
    \[Tf(n):=\frac{1}{2\pi}\int^{2\pi}_0f(t)\exp(-int)dt\]
    により定まる.
    \begin{enumerate}
        \item $T$は単射で有界である.
        \item $T$は全射たりえない.
    \end{enumerate}
\end{corollary}

\section{Hahn-Banachの優延長定理}

\begin{tcolorbox}[colframe=ForestGreen, colback=ForestGreen!10!white,breakable,colbacktitle=ForestGreen!40!white,coltitle=black,fonttitle=\bfseries\sffamily,
title=]
    According to Helmut H. Schaefer, "the study of a locally convex space in terms of its dual is the central part of the modern theory of topological vector spaces, for it provides the deepest and most beautiful results of the subject."
    殆どの関数論は,関手$S:=\Hom(-,\bF)$によって豊かな理論を得る.これが層の理論であり,双対空間論である.
\end{tcolorbox}

\subsection{Hahn-Banachの線型汎関数の延長定理}

\begin{tcolorbox}[colframe=ForestGreen, colback=ForestGreen!10!white,breakable,colbacktitle=ForestGreen!40!white,coltitle=black,fonttitle=\bfseries\sffamily,
title=双対空間の元の存在]
    超平面の分離定理は,これの特別な場合である.
    このことも含めて,$\abs{X^*}\ne\emptyset$を主張する位相線型空間論でのHahn-Banachの定理は,位相空間論におけるUrysohnの補題と同じ立ち位置である.
    Urysohnの補題は正規空間において閉集合を分離するという主張であるが,連続関数を構成する際にも用いられる.
    It ensures that such a space will have enough continuous linear functionals such that the topological dual space is interesting.\footnote{\url{https://ncatlab.org/nlab/show/Hahn-Banach+theorem}}
    さらに言えば,選択公理と同じ役割をするともみれる.
\end{tcolorbox}

\begin{tcolorbox}[colframe=ForestGreen, colback=ForestGreen!10!white,breakable,colbacktitle=ForestGreen!40!white,coltitle=black,fonttitle=\bfseries\sffamily,
title=]
    完備距離空間の稠密部分空間や,正規空間の閉集合上からは,一様ノルムを変えない延長が存在したが,
    同様にノルム空間の部分空間上には,作用素ノルムを変えない延長が存在する.
    一般の線型空間の部分空間の上からは,任意のMinkowski汎関数に対して,これを超えない延長が存在する.
\end{tcolorbox}

\begin{definition}[Minkowski functional]
    $X$を$\bF$-ノルム空間とする.
    関数$m:X\to\R$が\textbf{Minkowski汎関数}または\textbf{劣線形汎関数}であるとは,次の2条件を満たすことをいう:
    \begin{enumerate}[(a)]
        \item (劣加法性) $m(x+y)\le m(x)+m(y)$.
        \item (正斉次性) $\forall_{t\in\R_{\ge 0}}\;m(tx)=tm(x)$.
    \end{enumerate}
    セミノルムはMinkowski汎関数である.
\end{definition}

\begin{lemma}[fundamental lemma (AC)]
    $m:X\to\R$を実線型空間$X$上のMinkowski汎関数とし,$\varphi:Y\to\R$を部分空間$Y\subset X$上の線型汎関数で$m$で抑えられるものとする:$\forall_{y\in Y}\;\varphi(y)\le m(y)$.
    このとき,$X$上の汎関数$\wt{\varphi}:X\to\R$であって$m$によって抑えられる延長$\wt{\varphi}|_Y=\varphi$が存在する.
\end{lemma}
\begin{Proof}\mbox{}
    \begin{description}
        \item[$m$以下の延長の存在] いま$Y$を任意の部分空間,$\varphi$をその上の線型汎関数とする.
        このとき,任意の$x\in X\setminus Y$に対して,延長$\wt{\varphi}:Y+\R x\to\R$であって$m$によって抑えられるものが存在することを示す.

        いま,$\al:=\wt{\varphi}(x)$の定め方であって,$\forall_{s\in\R,y\in Y}\;\wt{\varphi}(y+sx)=\varphi(y)+s\al$かつ$\varphi(y)+s\al\le m(y+sx)$を$s=\pm 1$の場合について満たすことが必要十分.
        これは,$\forall_{y,z\in Y}\;\varphi(y)-m(y-x)\le\al\le-\varphi(z)+m(z+x)$が必要十分.
        仮定より,
        \begin{align*}
            -\varphi(z)+m(z+x)-\varphi(y)+m(y-x)&=m(y-x)+m(z+x)-\varphi(y+z)\\
            &\ge m(y+z)-\varphi(y+z)\ge 0
        \end{align*}
        だから,$\Square{\sup_{y\in Y}\Brace{\varphi(y)-m(y-x)},\sup_{z\in Y}\Brace{-\varphi(z)+m(z+x)}}\ne\emptyset$より,たしかに条件を満たす延長は存在する.
        \item[Zornの補題により極大な延長を探す]
        こうして,$Y\subset Z\subset X$を満たす部分空間$Z$と,その上の$\varphi$の$m$以下の延長$\psi$の組$(Z,\psi)$全体からなる集合$\Lambda$を考えると,これは空でない.
        また,$(Z_1,\psi_1)\le(Z_2,\psi_2):\Leftrightarrow Z_1\subset Z_2\land\psi_2|_{Z_1}=\psi_1$と定めると,$\Lambda$は順序集合をなす.
        さらにこれは帰納的であることを示す.
        $N=\Brace{(Z_\mu,\psi_\mu)}_{\mu\in M}\subset\Lambda$を全順序部分集合としたとき,$Z:=\cup_{\mu\in M}Z_\mu,\psi(z):=\psi_\mu(z)\;(z\in Z_\mu)$と定めると,$(Z,\psi)\in\Lambda$で,$N$の上界である.
        よってZornの補題より,$\Lambda$の極大元$(Z,\wt{\varphi})$が存在するが,仮に$Z\ne X$としたら,$Z+\R x\;(x\in X\setminus Z)$上の延長を考えることで$Z$の極大性に矛盾する.よって,$Z=X$.
    \end{description}
\end{Proof}

\begin{theorem}[Hahn-Banach extension theorem]
    $m:X\to\R$を線型空間上のセミノルム,$\varphi$を部分空間$Y\subset X$上の汎関数であり$\abs{\varphi}\le m$を満たすとする.
    このとき,汎関数$\wt{\varphi}$であって,$\abs{\wt{\varphi}}\le m$を満たし,$\wt{\varphi}|_Y=\varphi$を満たす延長が存在する.
\end{theorem}
\begin{Proof}
    $\bF=\R$の場合,セミノルムはMinkowski汎関数で,$\abs{\varphi}\le m\Rightarrow\varphi\le m$だから,これは補題の特別な場合である.
    よって,$\bF=\C$の場合を考える.
\end{Proof}
\begin{remarks}[有界線型汎関数の延長]
    $X$がノルム空間で,$f:Y\to\C$がその上の有界線型汎関数であるとき,特に$\norm{\o{f}}=\norm{f}$を満たす延長が可能.
    $\norm{f}\le\norm{\o{f}}$は明らか.
    また,セミノルムを$m(x):=\norm{f}\norm{x}$とすれば,$\forall_{x\in Y}\;\o{f}(x)\le\norm{f}\norm{x}$を満たすから,$\norm{\o{f}}\le\norm{f}$も成立.
\end{remarks}

\subsection{双対写像}

\begin{tcolorbox}[colframe=ForestGreen, colback=ForestGreen!10!white,breakable,colbacktitle=ForestGreen!40!white,coltitle=black,fonttitle=\bfseries\sffamily,
title=]
    延長定理により,種々の興味深い有界線型汎関数が$B^*\subset X^*$上に取れる.
    \begin{enumerate}
        \item $x\in X\setminus\{0\}$に対して,この方向の射線を$\varphi\paren{\al\frac{x}{\norm{x}}}=\al$と写すもの$B^*$上に取れる.
        \item 閉部分空間$Y\subset X$に対して,$Y$からの距離を返すようなものが$B^*$上に取れる.
    \end{enumerate}
    特に(1)の対応を双対写像という\cite{Brezis}.
\end{tcolorbox}

\begin{corollary}[汎関数の構成]\label{cor-Hahn-Banach}
    ノルム空間$X$の
    任意の元$x\in X\setminus\{0\}$に対して,
    線型汎関数$\varphi_x\in X^*$が存在して,$\norm{\varphi_x}=1$かつ$\varphi_x(x)=\norm{x}$を満たす.
\end{corollary}
\begin{Proof}
    $\varphi:\bF x\to\bF$を,$\varphi(\al x)=\al\norm{x}$で定めると,$\norm{\varphi}=1$である.
    これはノルム$\norm{-}$より大きくないままの$X$上への延長が存在するが,$\norm{\al x}=1$を満たす$\al x\in\bF x$について$\varphi(\al x)=1$であるから,$\norm{\varphi}=1$のままである.
\end{Proof}
\begin{remarks}[duality map]
    この対応は,双対空間$X^*$の単位円周$\partial B^*$上の点を取る標準的な対応$\varphi_-:X\setminus\{0\}\to\partial B^*$を定めており,$\varphi_x$は$x$方向の単位ベクトル$\frac{x}{\norm{x}}$を$1$に写す:$\varphi\paren{\frac{x}{\norm{x}}}=1$.
    この条件を満たす$\varphi_x$は一般には一意でないが,$X^*$が厳密に凸(Hilbert空間や$L^p\;(p\in(1,\infty))$空間)のとき,一意的である.
    純粋な存在定理として,汎関数による点の分離能力は十分に高いことを表現していると捉えられる.定理\ref{thm-Spectrum-is-compact}などに使う.
\end{remarks}

\begin{corollary}[閉部分空間からの距離関数]\label{cor-distance-from-closed-subspace}
    ノルム空間$X$の任意の閉部分空間$Y\subset X$と点$x\in X\setminus Y$に関して,線型汎関数$\varphi_Y\in X^*$が存在して,$\norm{\varphi_Y}=1$かつ$\varphi_Y|_Y=0$かつ$\varphi_Y(x)=\inf_{y\in Y}\norm{x-y}$を満たす.
\end{corollary}
\begin{remarks}
    \ref{prop-標準写像の随伴}に関連する.
    この時点ですでに,分離定理にステートメントが似てきた.
    この系は,部分空間$Y\subset X$について,次の2条件が同値である,というようにも定式化出来る.
    \begin{enumerate}
        \item $x_0\in \oo{Y}$.
        \item $Y$上で零だが,$f(x_0)\ne0$を満たす有界線型汎関数$f\in X^*$は存在しない.
    \end{enumerate}
\end{remarks}

\begin{corollary}[作用素ノルムの双対空間による特徴付け]
    $x\in X$のノルムは,$X^*$を用いて次のようにも表せる:
    \[\norm{x}=\sup_{f\in B^*}\abs{f(x)}=\max_{f\in B^*}\abs{f(x)}.\]
    特に,$x\in X$に関する評価写像は$\ev_x\in X^*$で,$\norm{\ev_x}=\norm{x}$.
\end{corollary}
\begin{Proof}
    系\ref{cor-Hahn-Banach}による.
\end{Proof}
\begin{remarks}
    作用素ノルムの場合とは違って,この特徴付けでは,supは達成される.
\end{remarks}

\subsection{複素線型空間の扱い}

\begin{proposition}[複素線型空間は実線形空間]
    $V$を複素線型空間とする.
    \begin{enumerate}
        \item $f\in V^*$の実部を$u\in V^*$とする.このとき,$\forall_{x\in V}\;f(x)=u(x)-iu(ix)$.
        \item $u\in B(V,\R)$に対して,$f(x):=u(x)-iu(ix)$と定めると,$f\in V^*$である.
        \item $V$はノルム空間であり,$f,u\in V^*$は$\forall_{x\in V}\;f(x)=u(x)-iu(ix)$を満たすとする.このとき,$\norm{f}=\norm{u}$.
    \end{enumerate}
\end{proposition}

\section{回帰性}

\begin{tcolorbox}[colframe=ForestGreen, colback=ForestGreen!10!white,breakable,colbacktitle=ForestGreen!40!white,coltitle=black,fonttitle=\bfseries\sffamily,
title=]
    双対空間の単位閉球$B^*$に標準的な元の取り方を得た.これによって調べることが出来る.
    \begin{enumerate}
        \item $X$や$X^*$上の部分集合に対して,零化作用素$A^\perp$,極作用素$A^\circ$が存在し,種々の位相的性質を特徴付けることが出来る.
        \item まず第一に埋め込み$i:X\mono X^{**}$が見つかる.
        \item ペアリング$\brac{-,-}:X\times Y\to\bF$を用いて,他の空間の知識を別の空間に流入させることが出来る.
        \item 射集合の間に,随伴作用素$*:B(X,Y)\to B(Y^*,X^*)$や,転置などの関手が考えられる.
    \end{enumerate}
\end{tcolorbox}

\subsection{零化空間}

\begin{tcolorbox}[colframe=ForestGreen, colback=ForestGreen!10!white,breakable,colbacktitle=ForestGreen!40!white,coltitle=black,fonttitle=\bfseries\sffamily,
title=自然なペアリングに関する直交空間]
    Hahn-Banachの拡張定理より,標準的な単射$\kappa_X:X\mono (X^*)^*$は等長写像であることが従う.
    これがBanのisoでもあるとき,$X$を回帰的という(Hahn-Banachの定理より等長写像であることが従うから,Banの同型でもある).

    annihilatorの概念は一般の加群について定義され,可換環に対しては随伴と関係が深く,内積に関する場合を直交補空間という.
\end{tcolorbox}

\begin{definition}[annihilator]
    部分空間$Y\subset X$と$Z\subset X^*$について,
    \begin{enumerate}
        \item $Y$の零化空間とは,$Y^\perp=\Brace{\varphi\in X^*\mid\forall_{y\in Y}\;\varphi(y)=0}=\cap_{y\in Y}\Ker\ev_y$を指す.
        \item $Z$の零化空間とは,$Z^\perp=\Brace{x\in X\mid\forall_{\varphi\in Z}\;\varphi(x)=0}=\cap_{\varphi\in Z}\Ker\varphi$を指す.
    \end{enumerate}
    自然なペアリング$\brac{\varphi,y}=\varphi(y)$に関する直交空間ともいう.
\end{definition}

\begin{lemma}[零化空間の基本性質]\mbox{}
    \begin{enumerate}
        \item 零化空間は必ず閉部分空間である($X$のノルム閉,$X^*$の$w^*$-閉).
        \item 一般に$Y\subset(Y^\perp)^\perp,Z\subset(Z^\perp)^\perp$である.
        \item $Y$が閉部分空間である時,$Y=(Y^\perp)^\perp$である.特に,$(Y^\perp)^\perp=\o{Y}$でもある.
        \item 一方で,$Z$が閉部分空間であっても$Z=(Z^\perp)^\perp$とは限らないが,$w^*$-閉ならば$Z=(Z^\perp)^\perp$である.
    \end{enumerate}
    もちろん,$X$が有限次元の場合はいずれの等号も常に成り立つ.
\end{lemma}
\begin{Proof}\mbox{}
    \begin{enumerate}
        \item いずれも閉部分空間の共通部分であるため.
        \item 任意に$y_0\in Y$を取る.任意に$\varphi\in T^\perp$を取ると,$\forall_{y\in Y}\;\varphi(y)=0$を満たすのだから,当然$\varphi(y_0)=0$.
        \item $Y$は閉だから,系\ref{cor-distance-from-closed-subspace}より,ここからの距離を測る関数$\varphi_Y\in X^*$が取れる.任意に$x\in (Y^\perp)^\perp$を取ると,これは特に$\varphi_Y(x)=0$を満たす必要があるが,これは$x\in Y$を意味する.
        \item 系\ref{cor-weak-star-closed-subspace}の証明より.
    \end{enumerate}
\end{Proof}

\begin{proposition}[再零化空間の閉包による特徴付け]
    $X$をBanach空間とする.
    \begin{enumerate}
        \item $(Z^\perp)^\perp=\oo{Z}$.
        \item 一$w^*$-閉包について$(Z^\perp)^\perp=\o{Z}$.
    \end{enumerate}
\end{proposition}

\begin{proposition}[一般の集合の零化空間]\label{prop-dense-subspace-of-Banach-space}
    局所凸空間$X$の部分集合$A$が生成する部分空間を$Y$とする.
    \begin{enumerate}
        \item $A^\perp=Y^\circ$である.
        \item $A$が$X$上線形稠密である($\o{Y}=X$)ことと,$A^{\perp\perp}=X$は同値.
    \end{enumerate}
\end{proposition}
\begin{Proof}\mbox{}
    \begin{enumerate}
        \item a
        \item 系\ref{cor-dense-subspace}と同様に証明できるはず.
    \end{enumerate}
\end{Proof}

\subsection{部分空間の束との関係}

\begin{proposition}
    $X$をノルム空間,$X_1,X_2$を部分空間とする.$(H_1+H_2)^\perp=H_1^\perp\cap H_2^\perp$.
\end{proposition}
\begin{Proof}
    $(H_1+H_2)^\perp\subset H_1^\perp\cap H_2^\perp$は$H_1+H_2$と直交するとき,特に$H_1,H_2$とも直交するため.
    逆の$H_1^\perp\cap H_2^\perp\subset (H_1+H_2)^\perp$は,2つの空間のいずれとも直交するなら,その和で表される元のすべてと直交する.
\end{Proof}

\subsection{零化空間と双対空間}

\begin{tcolorbox}[colframe=ForestGreen, colback=ForestGreen!10!white,breakable,colbacktitle=ForestGreen!40!white,coltitle=black,fonttitle=\bfseries\sffamily,
title=]
    $M\csub X$をBanach空間の閉部分空間とすると,$M^*simeq X^*/M^\perp$かつ$(X/M)^*\simeq M^\perp$が成り立つ.
\end{tcolorbox}

\begin{theorem}
    $M\csub X$をBanach空間の閉部分空間とする.
    \begin{enumerate}
        \item $m^*\in M^*$は一意に$x^*\in X^*$に延長するから,これを用いて$\sigma:M^*\to X^*/M^\perp$を$\sigma m^*=x^*+M^\perp$とすると,これはBanの同型である.
        \item $\pi:X\to X/M=:Y$を用いて,$\tau:Y^*\to M^\perp$を$\tau y^*=y^*\pi$とすると,これはBanの同型である.
    \end{enumerate}
\end{theorem}

\subsection{再双対空間}

\begin{definition}[bidual space, double-dual embedding, reflexive]\label{def-bidual}
    $X$をノルム空間とすると,$X^*,X^{**}$はBanach空間である.
    \begin{enumerate}
        \item $i:X\to X^{**}$を値写像$i(x):=\ev_x$と定めると,これは単射線型作用素である.
        \item すると$i$はノルム減少作用素(特に有界)であり,さらに等長写像である(系\ref{cor-Hahn-Banach}).
        よって,$i$はBanの埋め込みである.
        \item $i$が全射でもある(したがって$i$はBanの同型である)とき,Banach空間$X$を\textbf{回帰的}という.
    \end{enumerate}
\end{definition}
\begin{Proof}\mbox{}
    \begin{enumerate}
        \item $X^*$は$X$の相異なる2点を分離するので,$i$が単射である.
        \item $\norm{i(x)}=\sup_{\varphi\in B^*}\varphi(x)\le\norm{x}$であるからノルム減少的.ここで系\ref{cor-Hahn-Banach}より,$\varphi(x)=x$を満たす$\varphi\in B^*$が取れるから,等号は達成される.
    \end{enumerate}
\end{Proof}
\begin{example}[回帰的空間の例]\mbox{}
    \begin{enumerate}
        \item 有限次元線型空間は回帰的である.
        \item $L^p$空間は$1<p<\infty$のとき,回帰的である.
        \item Rieszの表現定理より,任意のHilbert空間は回帰的である.
    \end{enumerate}
\end{example}

\begin{lemma}[Goldstine]
    $X$をBanach空間とする.
    $\iota(B)\subset X^{**}$は$\sigma(X^{**},X^*)$-位相について$B^**$上稠密である.
\end{lemma}

\subsection{回帰性の弱位相による特徴付け}

\begin{proposition}
    Banach空間$X$について,次の2条件は同値.
    \begin{enumerate}
        \item $X$は回帰的である.
        \item $X^*$は回帰的である.
    \end{enumerate}
\end{proposition}

\begin{theorem}[Kakutani]\label{thm-characterization-of-reflexive-Banach-spaces}
    Banach空間$X$について,次の2条件は同値.
    \begin{enumerate}
        \item $X$は回帰的である.
        \item ノルム閉単位球$B$は,$\sigma(X,X^*)$-位相についてコンパクト.
    \end{enumerate}
\end{theorem}

\begin{corollary}\mbox{}
    \begin{enumerate}
        \item $X$が回帰的であるとき,任意の有界線型作用素$E\to X,X\to F$について,合成$E\to F$は弱コンパクトである.
        \item (Davis-Figiel-Johnson-Pelczynski 74) 任意の弱コンパクトな有界線型作用素$E\to F$について,ある回帰的なBanach空間$X$が存在してこれに沿って分解する.
    \end{enumerate}
\end{corollary}

\begin{theorem}[Eberlein-Smulian]
    $E$をBanach空間とする.次の2条件は同値:
    \begin{enumerate}
        \item $E$は回帰的である.
        \item 任意の有界列$\{x_n\}\subset E$は弱収束する部分列を持つ.
    \end{enumerate}
\end{theorem}

\subsection{回帰性と最適化}

\begin{theorem}[James, R. C.]\label{thm-James}
    $X$をBanach空間とする.
    \begin{enumerate}
        \item 任意の$f\in X^*$について,その作用素ノルム$\norm{f}=\sup_{x\in B}\abs{f(x)}$の上限が達成される.
        \item $X$は回帰的である.
    \end{enumerate}
\end{theorem}
\begin{remarks}
    このように,双対空間を考えるとsupが達成され得ることは,凸解析の双対理論でも利用される\ref{thm-Fenchel-Rockafellar}.
\end{remarks}

\begin{corollary}
    $E$を回帰的Banach空間,$K\subset E$を閉凸有界集合とする.$K$は$\sigma(E,E^*)$-コンパクトである.
\end{corollary}
\begin{Proof}
    $K$は凸だから,弱閉であることとノルム閉であることとが同値であることに注意すれば,
    角谷の定理から,$K$は弱コンパクト集合の閉部分集合と見れる.
\end{Proof}

\begin{corollary}
    $E$を回帰的Banach空間,$A\subset E$を非空閉凸集合,$\varphi:A\to\R_\infty$を真凸関数で$\lim_{\norm{x}\to\infty,x\in A}\varphi(x)=\infty$とする.このとき,$\varphi$は$A$上で最小値を取る.
\end{corollary}

\subsection{回帰性の一様凸性による十分条件}

\begin{tcolorbox}[colframe=ForestGreen, colback=ForestGreen!10!white,breakable,colbacktitle=ForestGreen!40!white,coltitle=black,fonttitle=\bfseries\sffamily,
title=]
    一様凸性は,$\forall_{\ep>0}\;\exists_{\delta>0}\;\forall_{x,y\in B}\;\norm{x-y}>\ep\Rightarrow\Norm{\frac{x+y}{2}}\le1-\delta$とも定義できる.
\end{tcolorbox}

\begin{definition}[uniformly convex, strictly convex]
    Banach空間$X$が
    \begin{enumerate}
        \item \textbf{一様凸}とは,任意の$\partial B$の列$(x_n),(y_n)$が$\norm{x_n+y_n}\to 2$を満たすならば$\norm{x_n-y_n}\to0$を満たすことをいう.
        \item \textbf{厳密に凸}とは,$\forall_{x,y\in X}\;\norm{x+y}=\norm{x}+\norm{y}\Rightarrow\exists_{\al\in\bF}\; x =\al y$を満たすことをいう.
    \end{enumerate}
\end{definition}

\begin{proposition}[Milman]
    $X$を一様凸とする.
    \begin{enumerate}
        \item 任意の有界線型汎関数$f\in X^*$は作用素ノルムの上限を達成する.
        \item 任意の閉凸集合は最小のノルムを持つ点が一意に定まる.
        \item 弱収束列$(x_n)$のノルムも収束するならば,$\norm{x_n-x}\to0$.
        \item $X$は回帰的である.
    \end{enumerate}
\end{proposition}

\begin{example}
    任意のHilbert空間と$L^p\;(1<p<\infty)$空間は一様凸である.
\end{example}

\begin{proposition}
    Banach空間$X$について,次の2条件は同値:
    \begin{enumerate}
        \item $X$は厳密に凸である.
        \item 任意の球面上の点$x\in\partial B$は,$B$の極点である:$\partial B\subset\Ex(B)$.
    \end{enumerate}
\end{proposition}

\subsection{回帰性の遺伝}

\begin{proposition}
    $X$を回帰的Banach空間,
    $Y\subset X$を閉部分空間とする.
    $Y,X/Y$も回帰的である.
\end{proposition}

\section{共役理論}

\begin{tcolorbox}[colframe=ForestGreen, colback=ForestGreen!10!white,breakable,colbacktitle=ForestGreen!40!white,coltitle=black,fonttitle=\bfseries\sffamily,
title=]
    Hilbert空間にて最も美しい躍動を見るが,内積ではなくとも,任意の双線型形式によって共役理論が展開出来る.
\end{tcolorbox}

\subsection{双対ペアによる誘導位相}

\begin{tcolorbox}[colframe=ForestGreen, colback=ForestGreen!10!white,breakable,colbacktitle=ForestGreen!40!white,coltitle=black,fonttitle=\bfseries\sffamily,
title=]
    2つの線型空間上のペアリングと呼ばれる双線型汎函数を与え,これを用いて互いに他へ位相を流入させることを考える.
    そのとき現れる有限次元的性格を持つ位相のうち,よく使われるのが弱$*$位相とMackey位相である.
    ペアリングはHilbert空間における射影を,内積がない空間でも模倣したものになる.
    その内積がある空間では直交補空間$X^\perp$に当たる概念が零化空間で,これを支える道具が極集合である.零化空間は$A^\perp=\brac{A}^\circ$と特徴づけられる.\ref{prop-dense-subspace-of-Banach-space}.
\end{tcolorbox}

\begin{definition}[algebraic duality / algebraic dual pair, duality / dual pair]\mbox{}
    \begin{enumerate}
        \item 2つの線型空間$X,Y$が\textbf{代数的双対ペア}であるとは,双線型写像$\brac{-,-}:X\times Y\to\bF$であって,部分空間$\brac{-,Y}\subset\Hom_\bF(X,\bF)$は$X$上の点を分離し,$\brac{X,-}\subset\Hom_\bF(Y,\bF)$は$Y$上の点を分離するようなものが存在することをいう.\footnote{これは$\forall_{x\in X\setminus\{0\}}\;\exists_{y\in Y}\;\brac{x,y}\ne 0$に同値で,双線型形式が非退化であるともいう.pairing bilinear formという.}
        \item $X,Y$がノルム空間でもあり,さらに$\brac{-,Y}\subset X^*,\brac{X,-}\subset Y^*$を満たすとき,単に\textbf{双対ペア}という.
    \end{enumerate}
\end{definition}

\begin{example}[線型空間の自然なペアリング]\label{exp-dual-pair}
    双対空間の間には評価によって自然な双線型写像が存在する.
    \begin{enumerate}
        \item 組$(X,X^*)$は双対ペアである.双線型形式は$\brac{x,\varphi}=\varphi(x)$で与えられる.
        \item 組$(X^{**},X^*)$も双対ペアである.双線型形式は$\brac{z,\varphi}=z(\varphi)$で与えられる.
    \end{enumerate}
\end{example}

\begin{definition}[代数的双対が定める位相]
    $(X,Y)$を代数的双対ペアとする.
    \begin{enumerate}
        \item $X$において,任意の$y\in Y$を用いて$p_y(-):=\abs{\brac{-,y}}$とおけば,これは$X$上の半ノルムを定める.族$(p_y)_{y\in Y}$が$X$上に定める始位相を\textbf{$\sigma(X,Y)$-位相}または\textbf{$Y$による弱位相}という.
        すなわち,$\sigma(X,Y)$位相とは,$Y$の元を(双対を通じて)$X$上の線型汎関数とみなしたとき,これらを連続にする最弱の位相である.
        \item 同様に,$Y$上にも$\sigma(Y,X)$位相が考えられる.
    \end{enumerate}
\end{definition}

\begin{example}[線型空間の自然な誘導位相]
    $E$における$\sigma(E,E^*)$位相も,$E^*$における$\sigma(E^*,E^{**})$位相も\textbf{弱位相}と呼ぶ.
    一方で,$\sigma(E^*,E)$による$E^*$上の位相を\textbf{弱$*$-位相}\ref{def-weak-star-topology}という.
    埋め込み$i:X\mono X^{**}$の存在より,弱$*$位相は弱位相より弱い.
    が,いずれの場合も局所凸である.
\end{example}

\subsection{ノルム空間の随伴作用素}

\begin{tcolorbox}[colframe=ForestGreen, colback=ForestGreen!10!white,breakable,colbacktitle=ForestGreen!40!white,coltitle=black,fonttitle=\bfseries\sffamily,
title=]
    随伴とは,標準的な双対双線型形式$\ev_(-)$を介して,互いに相等する関係にある元をいう.
\end{tcolorbox}

\begin{definition}[adjoint operator]
    ノルム空間の射$T\in B(X,Y)$に対して,\textbf{随伴作用素}$T^*:Y^*\to X^*$を
    自然なペアリングを用いて$\brac{x,T^*\varphi}_X=\brac{Tx,\varphi}_Y$で定まる.
\end{definition}

\begin{lemma}[$*$-作用素の反変関手性]
    $S\in B(X,Y),R\in B(Y,Z),\al\in\bF$について,
    \begin{enumerate}
        \item $(\al T+S)^*=\al T^*+S^*$.
        \item $(RT)^*=T^*R^*$.
    \end{enumerate}
\end{lemma}
\begin{Proof}
    ペアリングの双線型性による.
\end{Proof}

\begin{proposition}[$*$-作用素の等長性]
    ノルム空間の射$T\in B(X,Y)$の随伴作用素も$T^*\in B(Y^*,X^*)$であり,$\norm{T^*}=\norm{T}$.
\end{proposition}

\begin{proposition}[随伴であるための必要条件]
    $X,Y$をBanach空間,$T:X\to Y,S:Y^*\to X^*$を作用素とする.
    $\forall_{x\in X,\varphi\in Y^*}\;\brac{Tx,\varphi}_Y=\brac{x,S\varphi}_X$を満たすならば,
    $S,T$はいずれも有界で,$S=T^*$である.
\end{proposition}
\begin{remarks}[随伴ならば有界]
    これは,非有界な作用素を,その随伴を通じて調べる試みの失敗も意味する.
    非有界な自己共役作用素の理論は,必ず部分写像(not everywhere defined)の考え方がついてくることがわかる.
\end{remarks}

\begin{example}[transpose]
    線型作用素$x:X\to Y$に対して,前合成${}^t\!x:=x^*:Y^*\to X^*;f\mapsto f\circ x$を\textbf{転置写像}という.
    $X,Y$をノルム空間とする.$\norm{{}^t\!x}=\norm{x}$.
\end{example}

\subsection{Banach空間の随伴作用素}

\begin{tcolorbox}[colframe=ForestGreen, colback=ForestGreen!10!white,breakable,colbacktitle=ForestGreen!40!white,coltitle=black,fonttitle=\bfseries\sffamily,
title=]
    随伴の多くの性質は完備性のみにより,開写像定理から示され,内積である必要まではないのである.
    重要な非対称性としては,$T$の全射性は$T^*$の単射性と$\Im T^*$の閉性で特徴付けられるが,$T$の単射性は随伴の言葉では特徴付けられない.
\end{tcolorbox}

\begin{theorem}[像・核と零化空間]
    $X,Y$をBanach空間,$T\in B(X,Y)$とする.このとき,
    \begin{enumerate}
        \item $\Ker(T^*)=(\Im T)^\perp$.
        \item $\Ker T=(\Im T^*)^\perp$.
    \end{enumerate}
    また,$\Im T$が$Y$の閉集合ならば,次も成り立つ:
    \begin{enumerate}\setcounter{enumi}{2}
        \item $\Im T^*=(\Ker T)^\perp$.
    \end{enumerate}
\end{theorem}

\begin{corollary}\mbox{}
    \begin{enumerate}
        \item $\Ker(T^*)$は$Y^*$上$w^*$-閉である.
        \item $\Im T$が$Y$上稠密であることは,$T^*$が単射であることに同値.
        \item $T$が単射であることは,$\Im(T^*)$が$X^*$上$w^*$-稠密であることに同値.
    \end{enumerate}
\end{corollary}

\begin{theorem}
    $X,Y$をBanach空間,$T\in B(X,Y)$とする.次の3条件は同値:
    \begin{enumerate}
        \item $\Im(T)$は$Y$閉.
        \item $\Im(T^*)$は$X^*$上$w^*$-閉.
        \item $\Im(T^*)$は$X^*$上ノルム閉.
    \end{enumerate}
\end{theorem}

\begin{corollary}[closed range theorem]
    $X,Y$をBanach空間,$T\in B(X,Y)$とする.このとき,次の2条件は同値:
    \begin{enumerate}
        \item $T$は全射.
        \item $T^*$が単射で$\Im(T^*)$はノルム閉.
    \end{enumerate}
\end{corollary}

\subsection{極集合}

\begin{tcolorbox}[colframe=ForestGreen, colback=ForestGreen!10!white,breakable,colbacktitle=ForestGreen!40!white,coltitle=black,fonttitle=\bfseries\sffamily,
title=]
    一般の位相線形空間でも,双対ペアがあれば,Hilbert空間のように扱うことで幾何学的な考察が可能になる.
    これは普遍的な威力を持つ強力な手法である.
    直交性の概念よりも,凸錐が重要な位置を占める.
\end{tcolorbox}

\begin{definition}[real polar, bipolar]\label{def-polar}
    双対ペア$(X,Y)$と部分集合$A\subset X$に対して,
    \begin{enumerate}
        \item 対応する$Y$の部分集合$\Brace{y\in Y\mid \forall_{x\in A}\;\Re(x,y)\ge-1}$を$A$の\textbf{極集合}といい,$A^\circ$または$A^r$または$P(A)$で表す.
        \item $A^\circ$の極集合を$A^{\circ\circ}$と表し,\textbf{双極集合}という.
    \end{enumerate}
\end{definition}
\begin{remark}
    極集合の定義に単位円板$\sup_{x\in X}\abs{\brac{x,y}}\le1$を用いることもあり,この時をabsolute polarという.$\partial BA=A$を満たす集合$A$(これは$A$が均衡集合であることに同値\footnote{\url{https://en.wikipedia.org/wiki/Polar_set}})については両定義は一致するが,そうでない場合は違うが,その後の議論では同種の命題が成り立つことが知られている.
\end{remark}

\begin{lemma}[極集合の性質]\mbox{}\label{lemma-polar}
    \begin{enumerate}
        \item $A^\circ$は凸で,$\sigma(Y,X)$-閉である.
        \item $A\subset A^{\circ\circ}$で,$A$が閉凸集合ならば等号成立.
        \item $A$が絶対凸であるときは$A^\circ$も絶対凸であり,$A^\circ=\Brace{y\in Y\mid\forall_{x\in A}\;\abs{\brac{x,y}}\le 1}$と表せる.
        \item $A$が部分空間であるときは$A^\circ$も$Y$の部分空間であり,$A^\circ=\Brace{y\in Y\mid\forall_{x\in A}\;\brac{x,y}=0}$と表せる.
        \item $A$が$0$に頂点を持つ凸錐であるときは$A^\circ$も$0$に頂点を持つ凸錐であり,$A^\circ=\Brace{y\in Y\mid\forall_{x\in A}\;\brac{x,y}\ge0}$と表せる.
    \end{enumerate}
\end{lemma}

\begin{theorem}[双極定理]\label{thm-polar-theorem}
    $(X,Y)$を双対ペアとする.部分集合$A\subset X$について,双極集合$A^{\circ\circ}$は$A\cup\{0\}$の$\sigma(X,Y)$-閉凸包である.
\end{theorem}

\begin{corollary}
    局所凸空間$X$は,再双対空間$X^{**}$上で,$\sigma(X^{**},X^*)$-稠密である.
\end{corollary}

\subsection{共役関数}

\begin{tcolorbox}[colframe=ForestGreen, colback=ForestGreen!10!white,breakable,colbacktitle=ForestGreen!40!white,coltitle=black,fonttitle=\bfseries\sffamily,
title=]
    関数解析では,エピグラフが閉な下半連続関数と,エピグラフが凸な凸関数とが重要なサブクラスとなる.
    下半連続関数を閉関数ともいう.
\end{tcolorbox}

\begin{definition}[effective domain, proper function, conjugate function]
    $\R_\infty:=(-\infty,\infty]$,$E$をノルム空間,$E^*$をペアリング$\brac{-,-}$に関する双対として,関数$f:E\to\R_\infty$を考える.
    \begin{enumerate}
        \item $\Dom(f):=\Brace{x\in E\mid f(x)<\infty}$を\textbf{実効定義域}といい,これが空でない関数を\textbf{非自明な関数}または\textbf{真の関数}という.
        \item 非自明な関数$f$の\textbf{共役関数}または\textbf{Legendre-Fenchel変換}\footnote{極関数,またはYoung-Fenchel-Moreau変換とも呼ぶ}とは,
        \[f^*(\varphi):=\sup_{x\in E}(\brac{\varphi,x}-f(x))\]
        により定まる関数$\varphi^*:E^*\to\R_\infty$をいう.
    \end{enumerate}
    随伴と峻別するために,$f^\star$ともかく.
\end{definition}
\begin{remark}
    伝統的には,関数を$f:E\to\o{\R}$とし,\textbf{真凸関数}と言ったときは$-\infty<f\ne\infty$を満たすものに限る,とする.
\end{remark}


\begin{lemma}\mbox{}
    \begin{enumerate}
        \item 対応$f\mapsto\brac{f,x}-\varphi(x)$は$E^*$上凸で連続である.
        \item 任意の非自明な関数$f\in\Map(E,\R_\infty)$について,$f^\star$は凸な下半連続関数である.
        \item (Youngの不等式) $\forall_{(x,f)\in E\times E^*}\;\brac{f,x}\le\varphi(x)+\varphi^\star(f)$.
    \end{enumerate}
\end{lemma}
\begin{remarks}
    $E=E^*=\R$とし,
    $\varphi(t)=\abs{t}^p/p$とすると,$p$の共役指数$q$について$\varphi^\star(s)=\abs{s}^q/q$となる.このとき,Youngの不等式は
    \[\forall_{a,b\ge 0}\;\forall_{p\in(1,\infty)}\;ab\le\frac{a^p}{p}+\frac{b^q}{q}\]
    となる.$t=1,p=q=2$としたら相加相乗不等式である.
    また,畳み込みの劣乗法性もYoungの不等式という\ref{lemma-Young-for-convolution}.
\end{remarks}

\begin{theorem}[双共役定理 (Fenchel-Moreau)]
    局所凸空間$E$上の関数$\varphi:E\to\R_\infty$に対して,次の2条件は同値:
    \begin{enumerate}
        \item $\varphi^{\star\star}=\varphi$.
        \item $f$は下半連続な凸関数であるか,自明な関数$f=\infty$である.
    \end{enumerate}
\end{theorem}
\begin{remarks}
    これは双極定理の一般化とみなせる.最適化問題において,主問題と双対問題を関連付ける摂動関数が$F^{\star\star}=F$を満たすとき,主問題と双対問題の解は等しくなる.これを\textbf{強双対性}という.
\end{remarks}

\begin{theorem}[双対性定理 (Fenchel-Rockafellar)]
    $\varphi,\psi:E\to\R_\infty$を真凸関数とし,$x_0\in\Dom(\varphi)\cap\Dom(\psi)$にて$\varphi$は連続とする.このとき,
    \begin{align*}
        \inf_{x\in E}(\varphi(x)+\psi(x))&=\sup_{f\in E^*}(-\varphi^\star(-f)-\psi^\star(f))\\
        &=\max_{f\in E^*}(-\varphi^\star(-f)-\psi^\star(f))=-\min_{f\in E^*}(\varphi^\star(-f)+\psi^\star(f)).
    \end{align*}
    特に,双対問題は必ず解を持つ.
\end{theorem}

\subsection{数列空間の研究}

\begin{tcolorbox}[colframe=ForestGreen, colback=ForestGreen!10!white,breakable,colbacktitle=ForestGreen!40!white,coltitle=black,fonttitle=\bfseries\sffamily,
title=]
    $L^p$には積分なる双線型形式を通じた自然なペアリングが存在する.
\end{tcolorbox}

\begin{example}[数列空間の間のペアリング]
    数列の内積$\brac{x,y}=\sum x_ny_n$によって,$l^1\simeq_\Ban(c_0)^*$と$l^\infty\simeq_\Ban(l^1)^*$を導くことが出来る.
    一方で,$c_0,c$は回帰的ではない\ref{cor-predual-of-c0}.
\end{example}

\begin{proposition}[Schur's property 1921]\label{prop-Schur}
    $l^1$の弱収束列はノルム収束する.
\end{proposition}
\begin{remarks}
    弱位相に関する同様の結論は,定理について$X\mono X^{**}$を考えることで,$X^*$が可分ならば単位閉球$B\subset X$は距離化可能であることがわかる.
    しかし,双対空間が可分である例は(回帰的な例を除くと)少なく,代表的なのは$X=c_0,X^*=l^1$の場合のみである.
    なお,$X$が可分だからといって$B$は弱位相に関して距離化可能ではないことが,この命題からわかる.
    $l^1$の閉単位球が弱位相について距離化可能だとすると,$l^1$上の弱位相とノルム位相が一致してしまい,矛盾.
    これが,弱位相を調べる際には点列のみを考えては粒度が足りない好例である.
\end{remarks}

\begin{definition}[shift operator]
    有界作用素$S\in B(l^\infty)$を$(Sx)_n=x_{n+1}$で定める.これを\textbf{シフト作用素}という.
\end{definition}

\begin{proposition}[極限作用素の存在]
    ある有界線型汎関数$L\in(l^\infty)^*$が存在して,次の2条件を満たす:
    \begin{enumerate}
        \item $\forall_{x\in l^\infty}\;L(Sx)=L(x)$.
        \item $\forall_{x\in l^\infty}\;\liminf x_n\le L(x)\le\limsup x_n$.
    \end{enumerate}
\end{proposition}

\begin{proposition}[部分和作用素]
    $T:l^1\to c_0$を$(Tx)_n:=\sum_{m\ge n}x_m$で定めると,$T\in B(l^1,c_0)$を満たす.
\end{proposition}

\section{最適化問題}

\begin{tcolorbox}[colframe=ForestGreen, colback=ForestGreen!10!white,breakable,colbacktitle=ForestGreen!40!white,coltitle=black,fonttitle=\bfseries\sffamily,
title=]
    物理・数学・工学をまたがって変分問題は長い歴史を持つ.
\end{tcolorbox}

\subsection{枠組み}

\begin{definition}[feasible region, objective function, optimazation problem]
    位相線型空間$X$,部分集合(\textbf{実行可能領域}という)$S\subset X$,目的関数$f:S\to\R$について,$\min_{x\in S}f(x)$を求める問題を\textbf{最適化問題}という.
    \begin{enumerate}
        \item $S$がaffine関数のみによって定まるとする.さらに$f$もaffineなとき,\textbf{線型計画問題}といい,$f$が2次関数であるとき2次計画問題という.
        \item $S$が凸集合で,$f$も凸関数であるとき,\textbf{凸計画問題}という.
    \end{enumerate}
\end{definition}
\begin{example}
    $X$が関数空間であるとき,変分問題や最適制御問題は最適化問題の例である.
    $X$が行列全体,$S$を半正定値行列全体のなす凸錐とするときの凸最適化問題を\textbf{半正定値計画問題}という.
    $f$は経済分野では効用関数,物理分野ではエネルギー汎関数ともいう.
    $f$が種々の関数の和$f=f_1+\cdots+f_n$で表せたときの局所最適解は\textbf{Pareto最適解}ともいい,この汎関数に関する最適化を\textbf{多目的最適化}という.
\end{example}

\subsection{変分不等式問題}

\begin{tcolorbox}[colframe=ForestGreen, colback=ForestGreen!10!white,breakable,colbacktitle=ForestGreen!40!white,coltitle=black,fonttitle=\bfseries\sffamily,
title=]
    Lax-Milgramが最小化問題をEuler方程式に還元するために,
    最適化問題の一般化と捉えられる枠組みである.
\end{tcolorbox}

\begin{definition}[bounded, coercive]
    Hilbert空間上の双線型形式$a:H\times H\to\R$が
    \begin{enumerate}
        \item 有界であるとは,$\exists_{C\in\R}\;\forall_{u,v\in H}\;\abs{a(u,v)}\le C\abs{u}\abs{v}$を満たすことをいう.
        \item \textbf{強圧的}であるとは,$\exists_{\al>0}\;\forall_{v\in H}\;a(v,v)\ge\al\abs{v}^2$を満たすことをいう.
    \end{enumerate}
\end{definition}

\begin{theorem}[Stampacchia]
    $a:H\times H\to\R$を有界で強圧的な双線型形式,$K\csub H$を非空な閉凸集合とする.このとき,次が成り立つ:
    \begin{enumerate}
        \item $\forall_{\varphi\in H^*}\;\exists!_{u\in K}\;\forall_{v\in K}\;a(u,v-u)\ge\brac{\varphi,v-u}$.
        \item $a$が対称ならば,$u\in K$は次を満たすものとして特徴付けられる:
        \[\frac{1}{2}a(u,u)-\brac{\varphi,u}=\min_{v\in K}\paren{\frac{1}{2}a(v,v)-\brac{\varphi,v}}.\]
    \end{enumerate}
\end{theorem}

\begin{corollary}[Lax-Milgram]
    $a:H\times H\to\R$を有界で強圧的な双線型形式とする.このとき,次が成り立つ:
    \begin{enumerate}
        \item $\forall_{\varphi\in H^*}\;\exists!_{u\in H}\;\forall_{v\in H}\;a(u,v)=\brac{\varphi,v}$.
        \item $a$が対称ならば,$u\in H$は次の条件で特徴付けられる:
        \[\frac{1}{2}a(u,u)-\brac{\varphi,u}=\min_{v\in H}\paren{\frac{1}{2}a(v,v)-\brac{\varphi,v}}.\]
    \end{enumerate}
\end{corollary}
\begin{remarks}
    最小化問題(2)と対応するEuler方程式(1)との同値性を主張する.$F(v):=\frac{1}{2}a(v,v)-\brac{\varphi,v}$とおくと,$F'(v)=0$とも表せる.
\end{remarks}

\section{位相線型空間論}

\begin{tcolorbox}[colframe=ForestGreen, colback=ForestGreen!10!white,breakable,colbacktitle=ForestGreen!40!white,coltitle=black,fonttitle=\bfseries\sffamily,
title=]
    位相群論との相違(係数体の作用の構造)は,端的に「凸性」の概念に集約される.
\end{tcolorbox}

\subsection{定義と性質まとめ}

\begin{definition}[topological vector space]
    線型空間$X$とその上の$T_1$-位相$\tau$の組
    $(X,\tau)$が\textbf{位相線型空間}であるとは,$\tau$が線型空間の演算$+,\cdot$が定める終位相より強いことをいう.
    \begin{enumerate}
        \item 位相線型空間の近傍系$\O_0$とは,$0$の近傍系をいう.
        \item その理由は,平行移動$T_a:X\to X;x\mapsto a+x\;(a\in X)$と拡大$M_\lambda:X\to X;x\mapsto\lambda x\;(\lambda\in\bF)$とが$X$の位相同型を定めるためである.
        \item $X$上の距離$d$が\textbf{不変}であるとは,$\forall_{z\in X}\;d(x+z,y+z)=d(x,y)$を満たすことをいう.
    \end{enumerate}
\end{definition}

\begin{definition}
    位相線型空間$(X,\tau)$について,
    \begin{enumerate}
        \item \textbf{局所凸}とは,$\O_0$が凸集合からなる基本系を持つことをいう.
        \item \textbf{局所有界}とは,$0$に有界な近傍が存在することをいう.
        \item \textbf{局所コンパクト}とは,$0$に相対コンパクトな近傍が存在することをいう.
        \item \textbf{$F$-空間}であるとは,$\tau$がある不変な距離$d$が生成する完備な位相であることをいう.
        \item \textbf{Frechet空間}であるとは,局所凸な$F$-空間をいう.
        \item \textbf{Heine-Borel性を持つ}とは,任意の有界閉集合がコンパクトになることをいう.
    \end{enumerate}
\end{definition}

\begin{theorem}[位相線型空間の性質まとめ]\label{thm-character-of-TVS}
    $X$を位相線形空間とする.
    \begin{enumerate}
        \item $X$が局所有界ならば$X$は第1可算である\ref{thm-neighborhood-is-absorbant}.
        \item $X$が距離化可能であることと,第1可算であることとは同値.
        \item $X$がノルム付け可能であることと,局所凸で局所有界であることは同値.
        \item $X$が有限次元であることと,$X$が局所コンパクトであることとは同値.
        \item 局所有界な位相線形空間$X$について,Heine-Borel性を持つことと有限次元であることとは同値.
    \end{enumerate}
\end{theorem}

\subsection{位相線型空間はHausdorff}

\begin{lemma}[コンパクト集合と閉集合の開近傍による分離]
    $X$を位相線型空間,$K,C\subset X$を互いに素な部分集合で,それぞれコンパクト・閉であるとする.
    このとき,$0$の近傍$V\in\O_0$が存在して,$(K+V)\cap (C+V)=\emptyset$が成り立つ.
    ただし,$K+V:=\cup_{x\in K}(x+V)$とした.
\end{lemma}

\begin{corollary}
    位相線型空間$X$の近傍系$\O_0$の任意の元$U\in\O_0$に対して,$\exists_{V\in\O_0}\;\o{V}\subset U$.
\end{corollary}

\begin{corollary}
    任意の位相線型空間はHausdorffである.
\end{corollary}

\subsection{均衡集合と絶対凸集合}

\begin{tcolorbox}[colframe=ForestGreen, colback=ForestGreen!10!white,breakable,colbacktitle=ForestGreen!40!white,coltitle=black,fonttitle=\bfseries\sffamily,
    title=]
    集合$C$が定める計測関数
    \[m_C(x):=\inf\Brace{s>0\mid s^{-1}x\in C}\]
    が半ノルムになるために$C$に必要な条件が絶対凸性である.回転や鏡映・反転に関して安定な集合をいう.
\end{tcolorbox}

\begin{definition}[balanced, absolutely convex]
    $\bF$-線型空間$X$の部分集合$C\subset X$について,
    \begin{enumerate}
        \item $\forall_{\al\in\bF}\;\abs{\al}\le 1\Rightarrow\al C\subset C$を満たすとき,$C$を\textbf{均衡集合}または\textbf{円板}という.これは$\forall_{\al\in\bF}\;\abs{\al}=1\Rightarrow\al C=C$と同値.
        \item 均衡な凸集合を\textbf{絶対凸}という.
        \item $C$を含む絶対凸集合全体の共通部分を,絶対凸包という.
    \end{enumerate}
\end{definition}

\begin{lemma}[絶対凸性の特徴付け]
    $\bF$-線型空間$X$の部分集合$C$について,次の2条件は同値.
    \begin{enumerate}
        \item $C$は絶対凸である.
        \item $\forall_{x,y\in C}\;\forall_{\lambda_1,\lambda_2\in\bF}\;\abs{\lambda_1}+\abs{\lambda_2}\le 1\Rightarrow \lambda_1x+\lambda_2y\in C$.
    \end{enumerate}
\end{lemma}

\begin{lemma}[均衡性]
    $C$を均衡集合とする.
    \begin{enumerate}
        \item $C=-C$.
        \item $\forall_{r,\lambda\in\R\setminus\{0\}}\;\lambda x\in rC\Leftrightarrow x\in\frac{r}{\abs{\lambda}}C$.
        \item $C$が定める計量関数$m_C$について,$\forall_{\lambda\in\R}\;m_C(\lambda x)=\abs{\lambda}m_C(x)$.
    \end{enumerate}
\end{lemma}
\begin{Proof}\mbox{}
    \begin{enumerate}
        \item $-C\subset C$.$a\ne0$のとき$ax\in C\Leftrightarrow x\in a^{-1}C$だから,$x\in C\Rightarrow -x\in C\Rightarrow x\in -C$より,$C\subset-C$も得る.
        \item (1)より,$\lambda x\in rC\Leftrightarrow\abs{\lambda}x\in rC\Leftrightarrow\frac{r}{\abs{\lambda}}C$.
    \end{enumerate}
\end{Proof}

\subsection{位相線型空間の位相的性質}

\begin{theorem}
    $X$を位相線型空間とする.
    \begin{enumerate}
        \item 任意の部分集合$A\subset X$について,$\o{A}=\bigcap_{V\in\O_0}(A+V)$.
        \item $A,B\subset X$について,$\o{A}+\o{B}\subset\o{A+B}$.
        \item $Y\subset X$が部分空間ならば,$\o{Y}$も部分空間である.
        \item $C\subset X$が凸ならば,内部$C^\circ$も閉包$\o{C}$も凸である.
        \item $B\subset X$が均衡集合ならば,$\o{B}$も均衡で,さらに$0\in B^\circ$ならば$B^\circ$も均衡である.
        \item $E\subset X$が有界ならば,$\o{E}$も有界である.
    \end{enumerate}
\end{theorem}

\begin{theorem}
    位相線型空間$X$において,
    \begin{enumerate}
        \item 任意の$0$の近傍は,均衡な$0$の近傍を含む.
        \item 任意の$0$の凸近傍は,絶対凸な$0$の近傍を含む.
    \end{enumerate}
\end{theorem}

\begin{corollary}\mbox{}\label{cor-locally-convex-space}
    \begin{enumerate}
        \item 任意の位相線型空間は,均衡な集合からなる基本系を持つ.
        \item 任意の局所凸位相線型空間は,絶対凸な集合からなる基本系を持つ.
    \end{enumerate}
\end{corollary}

\subsection{併呑集合}

\begin{tcolorbox}[colframe=ForestGreen, colback=ForestGreen!10!white,breakable,colbacktitle=ForestGreen!40!white,coltitle=black,fonttitle=\bfseries\sffamily,
title=]
    任意の併呑集合は$0$を含むが,任意の$0$の近傍は併呑である.
\end{tcolorbox}

\begin{definition}[absorbant, barrelled]\label{def-barrel}
    $E$を線型空間とする.
    \begin{enumerate}
        \item $M\subset E$が\textbf{吸収的集合}または\textbf{併呑集合}であるとは,$\cup_{t>0}tM=E$を満たすことをいう.
        \item 任意の閉じた絶対凸な併呑集合(これを\textbf{樽型集合}という)が,$0$の近傍であるとき,局所凸線型空間$E$を\textbf{樽型空間}であるという.
    \end{enumerate}
\end{definition}

\begin{theorem}[近傍は併呑]\label{thm-neighborhood-is-absorbant}
    $X$を位相線型空間,$V\in\O_0$を近傍とする.
    \begin{enumerate}
        \item 無限に発散する単調増加な正数列$(r_n)$について,$X=\cup_{n\in\N}r_nV$.
        \item 任意のコンパクト集合$K\subset X$は有界である.
        \item $V$が有界ならば,$0$に収束する正数列$(\delta_n)$について,$\{\delta_nV\}_{n\in\N}$は基本系をなす.
    \end{enumerate}
\end{theorem}

\subsection{Minkowski汎関数}

\begin{tcolorbox}[colframe=ForestGreen, colback=ForestGreen!10!white,breakable,colbacktitle=ForestGreen!40!white,coltitle=black,fonttitle=\bfseries\sffamily,
title=]
    任意のセミノルムは,ある併呑な絶対凸集合$A\subset X$の定めるMinkowski汎関数である.
\end{tcolorbox}

\begin{definition}
    併呑凸集合$A\subset X$の定めるMinkowski汎関数$\mu_A:X\to\R_+$とは,
    \[\mu_A(x):=\inf\Brace{t>0\mid t^{-1}x\in A}\]
    をいう.併呑でないと,well-definedではない.
\end{definition}

\begin{theorem}
    $p:X\to\R_+$をセミノルムとする.
    \begin{enumerate}
        \item $p(0)=0$.
        \item $\abs{p(x)-p(y)}\le p(x-y)$.
        \item $p\ge0$.
        \item $p^{-1}(0)$は$X$の部分空間である.
        \item 集合$B:=p^{-1}([0,1))$は絶対凸な併呑集合であり,$p=\mu_B$を満たす.
    \end{enumerate}
\end{theorem}

\begin{theorem}
    $A$を線型空間$X$の併呑凸集合とする.
    \begin{enumerate}
        \item $\mu_A(x+y)\le\mu_A(x)+\mu_A(y)$.
        \item $\forall_{t\in\R_+}\;\mu_A(tx)=t\mu_A(x)$.
        \item $A$が均衡集合でもあるならば,$\mu_A$はセミノルムである.
        \item $B:=\mu_A^{-1}([0,1)),C:=\mu_A^{-1}([0,1])$とすれば,$B\subset A\subset C$であり,$\mu_B=\mu_A=\mu_C$である.
    \end{enumerate}
\end{theorem}

\subsection{局所凸性のセミノルム空間としての特徴付け}

\begin{tcolorbox}[colframe=ForestGreen, colback=ForestGreen!10!white,breakable,colbacktitle=ForestGreen!40!white,coltitle=black,fonttitle=\bfseries\sffamily,
title=]
    絶対凸集合は,セミノルムに関する「円板」として得られる集合にほかならないのであった.
    局所凸位相線型空間は,絶対凸な集合からなる基本系を持つ\ref{cor-locally-convex-space}.
    すなわち,セミノルムの族が定める位相線型空間こそが,局所凸位相線型空間にほかならない.
    こうしてノルム空間論の延長で議論出来る.
    セミノルム空間の位相は,ノルム空間の位相はノルムを連続にする最弱位相である点を一般化する.
\end{tcolorbox}

\begin{theorem}
    位相線型空間$X$は,絶対凸な基本系$\B$を持つとすると,対応するMinkowski汎関数の族$(\mu_V)_{V\in\B}$について,
    \begin{enumerate}
        \item $\forall_{V\in\B}\;V=\Brace{x\in X\mid\mu_V(x)<1}$.
        \item $\Brace{\mu_V}_{V\in\B}$は$X$上の連続なセミノルムからなる分離族である.
    \end{enumerate}
\end{theorem}

\begin{theorem}
    $\P\subset\Map(X,\bF)$を$X$上のセミノルムからなる分離族とする.任意の$p\in\P$に対して,
    \[V(p,n):=\Brace{x\in X\;\middle|\;p(x)<\frac{1}{n}}\]
    とすると,$\{V(p,n)\}_{p\in\P,n\in\N}$の有限交叉の全体$\B$は$X$のある位相$\tau$の,絶対凸な基本系であり,$(X,\tau)$は局所凸であり,次が成り立つ:
    \begin{enumerate}
        \item 任意の$p\in\P$は連続.
        \item $E\subset X$が有界であることと,任意の$p\in\P$について$p(E)$は有界であることとは同値.
    \end{enumerate}
\end{theorem}
\begin{remarks}
    特に,可算なセミノルムが定める局所凸線型空間は距離化可能である.
\end{remarks}

\begin{definition}[seminorm topology]\label{def-seminorm-topology}
    線型空間$X$とこれを分離するセミノルムの族$\F$を考える.すなわち,関数族$\F\times X\to\Map(X,\R);(m,y)\mapsto m(\cdot-y)$は$X$の点を分離する.
    これが$X$に定める始位相を,\textbf{$\F$が定めるセミノルム位相}とする.\footnote{一般に,擬距離の族によって定められた位相を持つ位相空間を,\textbf{ゲージ空間}という.}
    換言すれば,次のフィルター準基が各$\O(x)$に生成する位相である:
    \[\Brace{y\in X\mid\abs{m(y-z)-m(x-z)}<\ep,\;m\in\F,z\in Z,\ep>0}\]
    すると,$\R$はHausdorffであるから,セミノルム位相はHausdorffとなる\ref{prop-for-initial-topology-begin-Hausdorff}.
\end{definition}
\begin{remarks}
    セミノルムはそのままではHausdorffな位相を定めないから,このように族$\F$を用意して定義とする.
    $X$を分離する族$\F$が単元集合$\{m\}$であるためには,$m$がノルムであることが必要十分.
    任意の$\F$の元$m\in\F$が,$X$の汎関数$\varphi\in Y\subset X^*$の族$Y$を用いて$m=\abs{\varphi}$と表せるとき,このセミノルム位相を\textbf{$Y$が定める弱位相}$\sigma(X,Y)$という.
\end{remarks}

\begin{lemma}[セミノルム位相の性質]
    $\F$が定めるセミノルム空間$X$において,
    \begin{enumerate}
        \item ある有限集合$\{m_1,\cdots,m_n\}\subset\F$について,
        次のように表される集合全体が,近傍フィルター$\O(x)$の基底をなす:
        \[\bigcap_{k\in[n]}\Brace{y\in X\mid m_k(y-x)<\ep,\ep>0}\]
        \item ネット$(x_\lambda)$が収束することは,$\forall_{m\in\F}\;m(x_\lambda-x)\to0$に同値.
        \item $f:Y\to X$が連続であることは,任意の$Y$の収束ネット$(y_\lambda)$に対して,$\forall_{m\in\F}\;m(f(y_\lambda)-f(y))\to 0$が成り立つことに同値.
    \end{enumerate}
\end{lemma}
\begin{Proof}\mbox{}
    \begin{enumerate}
        \item 特に$z=x$の場合を考えると,$\{y\in X\mid m(y-x)<\ep\}$は準基である.しかし,$m$の劣加法性(三角不等式)より,$m(y-x)<\ep$ならば$\abs{m(y-z)-m(x-z)}<\ep$が従う.
        よって準基としてはこの形のもののみを考えれば良いから,これらの有限交叉の全体は基底をなす.
        \item \ref{prop-net-in-initial-topology}より.
        \item \ref{cor-continuous-function-to-initial-topology}より.
    \end{enumerate}
\end{Proof}

\begin{proposition}[局所凸性の特徴付け]\label{prop-characterization-of-locally-convex-spaces}
    位相線型空間$X$について,次の2条件は同値.
    \begin{enumerate}
        \item $X$は局所凸である.
        \item $X$の位相はセミノルム位相である.
    \end{enumerate}
\end{proposition}
\begin{Proof}\mbox{}
    \begin{description}
        \item[(2)$\Rightarrow$(1)] 任意の$X$の収束ネット$x_\lambda\to x,y_\lambda\to y$について,$x_\lambda+y_\lambda\to x+y$を示せば良い.
    \end{description}
\end{Proof}
\begin{remarks}
    この命題より,ノルム空間の理論の延長として局所凸空間を考えることができる.
    以降,線型空間$X$とその上の半ノルムの族$\F$との組$(X,\F)$を\textbf{局所凸空間}といい,混用する.
    実は開写像定理\ref{thm-open-mapping-theorem}の証明ですでに線形性と凸性との関係を使っている.
\end{remarks}

\subsection{ノルム空間の特徴付け}

\begin{theorem}
    位相線型空間$X$について,次の2条件は同値:
    \begin{enumerate}
        \item $X$はノルム付可能である.
        \item $X$は有界凸な近傍を持つ.
    \end{enumerate}
\end{theorem}

\subsection{有界集合}

\begin{proposition}
    $X$を位相線型空間,$A,B\subset X$を部分集合とする.
    \begin{enumerate}
        \item $A,B$が有界なら,$A+B$も有界である.
        \item $A,B$がコンパクトなら,$A+B$もコンパクトである.
        \item $A$がコンパクト,$B$が閉ならば,$A+B$も閉である.
    \end{enumerate}
    2つの閉集合の和は閉とは限らない.
\end{proposition}

\begin{proposition}[有界性の特徴付け]
    $X$を位相線型空間,$A\subset X$を部分集合とする.次の2条件は同値:
    \begin{enumerate}
        \item $A$は有界である.
        \item $A$の任意の可算部分集合は有界である.
    \end{enumerate}
\end{proposition}

\subsection{射の定義}

\begin{theorem}[連続性の特徴付け]\label{thm-morphism-of-TVS}
    位相線形空間$X$上の零でない線型汎関数$\Lambda:X\to\K$について,次の4条件は同値:
    \begin{enumerate}
        \item $\Lambda$は連続.
        \item 核$N(\Lambda):=\Lambda^{-1}(0)$は閉である.
        \item $N(\Lambda)$は$X$上稠密でない.
        \item $\Lambda$は$0$のある近傍$V$上で有界である.
    \end{enumerate}
\end{theorem}

\subsection{有限次元となる十分条件}

\subsection{距離化可能性}

\begin{theorem}
    $X$を第1可算な位相線型空間とする.このとき,距離$d$が存在して,次を満たす:
    \begin{enumerate}
        \item $d$は$X$の位相を生成する.
        \item $0$を中心とする開球は均衡である.
        \item $d$は平行移動不変である.
        \item $X$が局所凸であるとき,さらに任意の開球が凸になるように$d$を選べる.
    \end{enumerate}
\end{theorem}

\subsection{有界作用素}

\begin{theorem}
    $X,Y$を位相線形空間,$\Lambda:X\to Y$を線型写像とする.
    (1)$\Rightarrow$(2)$\Rightarrow$(3)が成り立つ.
    $X$が距離化可能であるとき,(3)$\Rightarrow$(4)$\Rightarrow$(1)も成り立ち,4条件が同値になる.
    \begin{enumerate}
        \item $\Lambda$は連続である.
        \item $\Lambda$は有界である.
        \item $x_n\to 0$ならば$\{\Lambda x_n\}_{x\in\N}$は有界である.
        \item $x_n\to0$ならば$\Lambda x_n\to0$.
    \end{enumerate}
    特にFrechet空間は距離化可能な完備局所凸線型空間であるから,上の4条件は同値になる.
\end{theorem}

\section{凸解析}

\subsection{凸集合}

\begin{proposition}[凸集合の特徴付けと伝播]
    $X$を線型空間,$A\subset X$を部分集合とする.次の2条件は同値.
    \begin{enumerate}
        \item $A$は凸である:$\forall_{a,b\in A}\;[a,b]\subset A$.
        \item $\forall_{s,t>0}\;(s+t)A=sA+tA$.
    \end{enumerate}
    また,次が成り立つ.
    \begin{enumerate}
        \item 凸集合の任意交叉は凸.
        \item 凸集合の増大族の極限は凸.
        \item $A,B$が凸集合のとき,$A+B$も凸.
    \end{enumerate}
    なお,この3条件は,部分空間が満たす性質と全く同じである.
\end{proposition}

\begin{proposition}[凸包の特徴付け]
    $X$を線型空間,$A\subset X$を部分集合とする.次の2条件は同値.
    \begin{enumerate}
        \item $\Conv(A)$は$A$を含む凸集合全体の共通部分である.
        \item $\Conv(A)=\Brace{x=\sum_{j\in J}\lambda_jx_j\in X\;\middle|\;\forall_{j\in J}\;x_j\in A\land \lambda_j\in\R_+\land\lambda_j=0\fe,\;\sum_{j\in J}\lambda_j=1}$である.
    \end{enumerate}
\end{proposition}

\subsection{凸集合の極点}

\begin{tcolorbox}[colframe=ForestGreen, colback=ForestGreen!10!white,breakable,colbacktitle=ForestGreen!40!white,coltitle=black,fonttitle=\bfseries\sffamily,
title=]
    図形$C$の
    極集合$F$とは,$C\setminus F$の2点を結ぶ線分の内点とはならない集合をいう.多角形の辺などをいう.
    極点とは,一点集合となるような極集合である.
    極点全体の集合を$\Ex(C)$で表し,ここから元の集合$C$を復元する問題を考える.
\end{tcolorbox}

\begin{definition}[extreme set / face, extreme point, extremal boundary]
    線型空間$X$の部分集合$C$について,
    \begin{enumerate}
        \item $C$の\textbf{極集合}とは,部分集合$F\subset C$であって,$\forall_{\lambda\in(0,1),x,y\in C}\;\lambda x+(1-\lambda)y\in F\Rightarrow x\in F\land y\in F$を満たすものをいう.
        \item 特に,一点集合からなる面を,\textbf{極点}または\textbf{端点}という.
        \item $C$の極点全体からなる集合を,\textbf{極境界}といい,$\partial C$や$\Ex(C)$で表す.
    \end{enumerate}
\end{definition}

\begin{proposition}[極点の特徴付け]
    線型空間$X$の凸集合$K$と点$p\in K$について,次の5条件は同値.
    \begin{enumerate}
        \item $p\in K$は極点である.
        \item $K\setminus\{p\}$は凸集合である.
        \item $p$は$K$内の非退化な開線分上の点として表せない.
        \item $\forall_{x\in X}\;p+x\in K\land p-x\in K\Rightarrow x=0$.
        \item $\{p\}$は$K$の面である.
    \end{enumerate}
\end{proposition}

\begin{example}[凸集合の極点]\mbox{}
    \begin{enumerate}
        \item コンパクトハウスドルフ空間上の連続関数の空間$C(X)$の単位球は,
        $X$が無限集合のとき,
        $\infty$-ノルムについてコンパクトでない\ref{prop-unit-ball-in-normed-space}.
        しかし一般に
        $\forall_{x\in X}\;\abs{f(x)}=1$を満たす関数$f\in C(X)$が極点となるが,
        $\bF=\R$で$X$が連結ならば,これはただ2つで,$F=\C$ならば,これはユニタリ関数を意味し,ノルム閉単位球の中で一様に稠密である.
        ($C(X)$は回帰的でないから,ノルム閉単位球は弱コンパクトではないのに!?)
        \item $L^1(X)\;(X\subset\R^n)$について,単位球はコンパクトでなく,極点を持たない.
        \item $L^p(X)\;(p\in(1,\infty))$は回帰的だから,閉単位球は$w^*$-コンパクトである.極境界は位相境界に一致する.これは$p$-ノルムが「一様に丸」くて尖った点のない球を与えるという幾何的消息を示唆している.
        \item $L^\infty(X)$の単位球の極境界は$\Brace{f\in \L^\infty(X)\mid\abs{f(x)}=1\;\ae}$.
        \item 単調増加関数の集合は凸で,各点収束位相についてコンパクトである.極点は$\Im f\subset\{0,1\}$をみたす関数.
        \item 開集合$\Om\subset\C$上の正則関数で$\norm{f}_\infty\le1$を満たすものがなす凸集合$\O(\Om)$の極点は$f(z)=\frac{\al}{z-z_0}\;(z_0\notin\Om,\abs{\al}=d(z_0,\Om))$と表せる関数.これはCauchyの積分公式で対称となる関数のクラスにほかならない!
        \item $M_n(\bF)=B(\bF^n)$に$\bF^n$の$2$-ノルムから定まる作用素ノルムを考え,これについてノルム単位球は$\bF^n$の等長同型.すなわち,$\bF=\R$のときは直交行列で,$\bF=\C$のときはユニタリ行列.
        \item $B(H)\;(\dim H\ge\aleph_0)$は回帰的で($B^1(H)$の双対空間と考えられる),単位球は$w^*$-コンパクトである($\sigma$-弱位相という).その極点は$T^*T=I$を満たす等長同型か,$TT^*=I$を満たす余等長同型である.
        また,この極点の凸包はそのまま閉単位球となる.実際,ユニタリ作用素の凸包ですでに開単位球を含む.
        \item $B(H)_\sa$は実Banach空間で,$\sigma$-弱位相の双対空間である.この極点は対称変換$S=S^*,S^2=I$である.
        \item $B(H)_+$は閉錐で,$B(H)$を生成する.この単位球の極点は直交射影$P=P^*,P^2=P$である.このクラスの作用素に対する結果がスペクトル定理である.
        \item $P(X)$の極点はDirac測度である.
    \end{enumerate}
\end{example}

\subsection{位相線型空間の凸集合}

\begin{proposition}
    $X$を位相線型空間とする.
    \begin{enumerate}
        \item 開集合の凸包は開である.
        \item $X$が局所凸ならば,任意の有界集合の凸包は有界である.
    \end{enumerate}
\end{proposition}

\begin{proposition}[凸集合の正則性]
    $X$を位相線型空間,$C\subset X$を凸集合とする.
    \begin{enumerate}
        \item $x\in C^\circ,y\in\o{C}$ならば,$\forall_{0<\lambda\le1}\;z=\lambda x+(1-\lambda y)\in C^\circ$.
        \item $C^\circ\ne\emptyset$ならば,$\o{(C^\circ)}=C$かつ$(C^-)^\circ=C^\circ$.
    \end{enumerate}
\end{proposition}

\subsection{凸錐}

\begin{tcolorbox}[colframe=ForestGreen, colback=ForestGreen!10!white,breakable,colbacktitle=ForestGreen!40!white,coltitle=black,fonttitle=\bfseries\sffamily,
title=]
    錐を基本言語として一般の集合を解析していく.
\end{tcolorbox}

\begin{definition}[convex cone]
    順序体$\R$上の局所凸線型空間内の集合$A$について,
    \begin{enumerate}
        \item $A$が\textbf{錐}であるとは,$\R_+A\subset A$を満たすことをいう.それぞれの$\R_+x$を\textbf{射線}という.
        \item $A$が\textbf{凸錐}であるとは,正係数の線型結合に閉じていることをいう:$A+A\subset A,\R_+A\subset A$.
        \item 凸錐$A$が$A\cap -A=0$を満たすとき,$0$に頂点を持つという.
    \end{enumerate}
\end{definition}
\begin{example}\mbox{}
    \begin{enumerate}
        \item $C_p:=\Brace{(t,x)\in\R\times\R^n\mid t\ge\norm{x}_p}$は凸錐である.$C_2$を\textbf{二次錐}という.
        \item 対称行列の空間$\R^{n(n+1)/2}$内で,半正定値行列が作る集合は凸錐である.一般に,自己共役作用素$B(H)_\sa$内で半正定値作用素は閉凸錐をなす.実軸の非負部分$\R_+$は凸錐であるが,この例は退化したものと捉えられる.
    \end{enumerate}
    これらはいずれも自己共役錐である.
\end{example}

\begin{definition}[polar cone, dual cone, self-adjoint cone]
    錐$C\subset\R^n$に対して,
    \begin{enumerate}
        \item $C^\circ:=\Brace{s\in\R^n\mid\forall_{x\in C}\;s^\top x\le0}$を\textbf{極錐}という.これは錐$C$の極集合である\ref{lemma-polar}.
        \item $C^*:=-C^\circ=\Brace{s\in\R^n\mid\forall_{x\in C}\; s^\top x\ge0}$を\textbf{双対錐}という.
        \item $C^*=C$を満たすとき,錐$C$を\textbf{自己共役錐}という.
    \end{enumerate}
\end{definition}

\begin{corollary}[双極定理]
    錐$C,C_1,C_2\subset\R^n$について,
    \begin{enumerate}
        \item $C^\circ$は閉凸錐.
        \item $C_1\subset C_2\Rightarrow C_1^\circ\supset C_2^\circ$.
        \item $C$が凸のとき,$C^{\circ\circ}=\Cl\;C$.
    \end{enumerate}
\end{corollary}
\begin{Proof}
    双極定理\ref{thm-polar-theorem}の錐に関する特殊化である.錐は必ず$0$を含むことに注意.
\end{Proof}

\begin{definition}
    集合$S\subset\R^n$とその点$x\in S$について,
    \begin{enumerate}
        \item $d\in\R^n$が$x$における$S$の\textbf{触ベクトル}または接ベクトルであるとは,$x$に収束する$S$の列$\{x_k\}\subset S$と$0$に上から収束する数列$(t_k)$が存在して,$d=\lim_{k\to\infty}\frac{x_k-x}{t_k}$と表せることをいう.
        \item 接ベクトル全体の集合$T_S(x)$は閉錐をなし,\textbf{接錐}という.$x\in\Int S\Rightarrow T_S(x)=\R^n$に注意.
        \item 接錐の極錐$N_S(x):=T_S(x)^\circ$を\textbf{法線錐}という.
    \end{enumerate}
\end{definition}

\subsection{凸関数}

\begin{proposition}
    可微分関数$f:X\to\R$について,次の2条件は同値.
    \begin{enumerate}
        \item $f$は凸である.
        \item $\forall_{x,y \in X}\;f(y)\ge f(x)+\brac{\nabla f(x),y-x}$.
    \end{enumerate}
\end{proposition}

\begin{proposition}
    連続微分可能関数$f:X\to\R$について,次の2条件は同値:
    \begin{enumerate}
        \item $f$は凸である.
        \item $\nabla f$は単調である:$\forall_{x,y\in X}\;\brac{\nabla f(x)-\nabla f(y),x-y}\ge0$.
    \end{enumerate}
\end{proposition}

\section{Hahn-Banachの分離定理}

\begin{tcolorbox}[colframe=ForestGreen, colback=ForestGreen!10!white,breakable,colbacktitle=ForestGreen!40!white,coltitle=black,fonttitle=\bfseries\sffamily,
title=関数空間の弱位相に対するHausdorffの公理の役割を果たす]
    2つの非空な凸集合は,位相線型空間$X$が有限次元のときは常に分離出来る.
    一般の位相線型空間の場合,どちらかが開集合であるときに分離可能で,
    局所凸空間の場合,それぞれ閉集合とコンパクト集合であるときに厳密に分離出来る.
\end{tcolorbox}

\subsection{超平面の性質}

\begin{definition}[affine space]
    affine空間$H$とは,線型空間$V$と全射線形作用素$p:V\to\bF$との組$(V,p)$である.ファイバー$H=p^{-1}(1)$もaffine空間と呼ぶ.
    すると,包含$\Vect\mono\Aff$が存在し,対象写像は全射であるが,射写像は充満ではない.$\Aff$の射は線形写像と定数関数との和である.
\end{definition}

\begin{definition}[affine map]
    凸集合$K$と線型空間$Y$について,$T:K\to Y$が次を満たすとき\textbf{affine}であるという:
    \[\forall_{\lambda\in(0,1)}\quad T((1-\lambda)x+\lambda y)=(1-\lambda)Tx+\lambda Ty.\]
\end{definition}

\begin{definition}[affine hyperplane]
    $X$を位相線型空間とする.$H\subset X$が\textbf{affine超平面}であるとは,ある零でない線型汎関数$f:X\to\bF$が存在して$H=f^{-1}(\al)\;(\al\in\bF)$と表せることをいう.
    以降,次の記法を用いる:
    \[H^-=(f^{-1}(\al))^-:=\Brace{x\in X\mid f(x)\le\al},\quad H^+=(f^{-1}(\al))^+:=\Brace{x\in X\mid f(x)\ge\al}\]
\end{definition}

\begin{definition}\mbox{}
    \begin{enumerate}
        \item affine空間$H$が定める半空間$H^+$の有限個の共通部分として表せる集合を\textbf{多面体}という.
        \item 集合$S\subset X$の\textbf{相対的内点}とは,$S$の\textbf{affine包}$\Aff(S)$の相対位相に関して内点であることをいう.
    \end{enumerate}
\end{definition}

\begin{proposition}
    $H\subset X$を超平面とする.次の2条件は同値.
    \begin{enumerate}
        \item $H$は閉集合である.
        \item 線型汎関数$f$は有界である.
    \end{enumerate}
\end{proposition}

\begin{definition}[separate, strictly separate]
    $A,B\subset X$を部分集合,$H=f^{-1}(\al)\subset X$を超平面とする.
    \begin{enumerate}
        \item $H$が$A,B$を\textbf{分離する}とは,$A\subset H^-\land B\subset H^+$が成り立つことをいう.
        \item $H$が$A,B$を\textbf{厳密に分離する}とは,$\exists_{\ep>0}\;A\subset (f^{-1}(\al-\ep))^-\land B\subset(f^{-1}(\al+\ep))^+$が成り立つことをいう.
    \end{enumerate}
\end{definition}

\subsection{位相線型空間上の結果1}

\begin{tcolorbox}[colframe=ForestGreen, colback=ForestGreen!10!white,breakable,colbacktitle=ForestGreen!40!white,coltitle=black,fonttitle=\bfseries\sffamily,
title=]
    位相線型空間の任意の凸開近傍は,あるMinkowski汎関数が定める「開球」になる.
    この性質を用いて,一般の位相空間上にHahn-Banachの分離定理が成り立つ.
    Minkowski汎関数の劣加法性は凸性に対応していたのである.
    また,$X$上の半ノルムとこれを計測関数として定める絶対凸集合との間には全単射の対応$m_C\leftrightarrow C$が存在する!
\end{tcolorbox}

\begin{theorem}[幾何形のHahn-Banachの分離定理]
    $X$をノルム空間,
    $A,B\subset X$を互いに素な凸集合とする.
    片方が開集合ならば,ある閉超平面が存在して,$A$と$B$を分離する.
\end{theorem}

\begin{lemma}[gauge]
    $X$を位相線型空間,$C\osub X$を$0$における凸開近傍とする.
    関数$m_C:X\to\R_+$を
    \[m_C(x):=\inf\Brace{s>0\mid s^{-1}x\in C}\]
    と定めると,これはMinkowski汎関数で$\exists_{M\in\R_+}\forall_{x\in X}\;0\le m_C(x)\le M\norm{x}$を満たし,$C=\Brace{x\in X\mid m_C(x)<1}$と表せる.
    この$m_C$を,開近傍$C$の\textbf{計測写像}という.
\end{lemma}
\begin{Proof}\mbox{}
    \begin{description}
        \item[well-definedness] $C$は$0$の開近傍としたから,ある$N\in\N$が存在して,$\forall_{x\in X}\;\forall_{n\ge N}\;n^{-1}x\in C$.よって,$\forall_{x\in X}\;m(x)<\infty$であり,たしかに関数が定まる.なお,このことから,任意の$0$の近傍は併呑集合であることがわかる.
        \item[正斉次性] $\forall_{t\ge 0}\;m(tx)=\inf_{tx\in sC,s>0}s=\inf_{x\in t^{-1}sC,s<0}s=tm(x)$.
        \item[劣加法性] 任意の$x,y\in X$を取る.すると,ある正実数$s,t>0$が存在して,$s^{-1}x\in C$かつ$t^{-1}y\in C$.
        $C$の凸性より,
        \[\frac{s}{s+t}s^{-1}x+\frac{t}{s+t}t^{-1}y=(s+t)^{-1}(x+y)\in C\]
        であるから,$m$の定義より,$m(x+y)\le s+t$.$s,t$の取り方に依らないから,$m(x+y)\le m(x)+m(y)$が従う.
        \item[$C$の表現] $C$は開集合であるから,$x\in C$ならば,ある$\ep>0$が存在して,$(1+\ep)x\in C$.よって,$m(x)\le(1+\ep)^{-1}<1$.
        逆に,$x\in X$は$m(x)<1$を満たすならば,ある$s<1$について$s^{-1}x\in C$ということだから,$C$が$0$の近傍であることと凸性より,
        \[x=(1-s)0+s(s^{-1}x)\in C.\]
    \end{description}
\end{Proof}
\begin{remarks}[絶対凸性の計測写像による特徴付け]\label{remarks-seminorm-and-absolutely-convex-sets}
    $C$の凸性は$m$の劣加法性に対応している.
    \footnote{実は,$\Brace{s>0\mid s^{-1}x\in C}$が空でないという仮定の下で,$C$が凸であるということは$C$が吸収的であるということと同値\ref{lemma-characterizing-absorbant}.なお,任意の位相線型空間において,$0$の近傍は併呑である.}
    このときのMinkowski汎関数$m$が半ノルムにもなるため(正斉次性だけでなく,さらに強い同次性を満たすため)の$C$の条件を\textbf{絶対凸性}という.
    $C$が単位球であるとき,$m_C$はノルムを定める.
\end{remarks}

\begin{lemma}[凸開集合と点の分離]
    $C\subset X$を非空な凸開集合,$x_0\in X\setminus C$を点とする.
    ある有界汎関数$f\in X^*$であって$\forall_{x\in C}\;f(x)<f(x_0)$を満たすものが存在する.
    特に,超平面$f^{-1}(f(x_0))$は$\{x_0\}$と$C$とを分離する.
\end{lemma}

\begin{lemma}[計測写像の核の特徴付け]
    $C$を絶対凸とする.
    $\Ker m_C=\bigcap_{t>0}tC$は,$C$に含まれる最大の線型部分空間である.
\end{lemma}

\begin{theorem}[Hahn-Banach separation theorem]\label{thm-Hahn-Banach-separation-theorem}
    $X$を位相線型空間,
    $A,B\subset X$を非空で互いに素な凸集合とする.
    $A$が開ならば,$\varphi\in X^*$と$t\in\R$が存在して,$\forall_{x\in A,y\in B}\;\Re\varphi(x)<t\le\Re\varphi(y)$を満たす.
\end{theorem}

\subsection{局所凸空間上の結果2}

\begin{theorem}
    $X$を局所凸空間,
    $A,B\subset X$を互いに素な凸集合とする.
    片方が閉集合でもう片方がコンパクト集合ならば,ある閉超平面が存在して,$A$と$B$を厳密に分離する.
\end{theorem}

\begin{corollary}[部分空間が稠密でないことの特徴付け \cite{Conway}IV.3.14]
    $X$を局所凸空間,$Y\subset X$を部分空間とする.次の2条件は同値.
    \begin{enumerate}
        \item $\o{F}\ne E$.
        \item ある零でない$f\in X\setminus\{0\}$が存在して,$f|_F=0$を満たす.
    \end{enumerate}
    特に,この$f$は任意の$x_0\notin X\setminus\o{Y}$に対して$f(x_0)=1$を満たすように取れる.
\end{corollary}

\begin{corollary}[双対空間の稠密部分空間の特徴付け]\label{cor-separating-subspace-is-dense}
    $X$を局所凸空間,$Y\subset X^*$を部分空間とする.次の2条件は同値.
    \begin{enumerate}
        \item $X$の分離族である.
        \item $Y$は$X^*$上$w^*$-稠密である.
    \end{enumerate}
\end{corollary}
\begin{Proof}
    $X^*$上$w^*$-稠密でないと仮定すると,$Y$上で$0$だが,恒等的に$0$ではないような$w^*$-連続な線型汎関数が存在することになるが,
    $w^*$-位相を備えた空間$X^*$上で連続な汎関数は$(\ev_x)_{x\in X^*}$に限る.
\end{Proof}

\begin{corollary}[連続延長定理]
    $X$を局所凸空間,$f:Y\to\bF$をその部分空間上の連続線型汎関数とする.
    このとき,連続な延長$\o{f}\in X^*$が存在する:$\o{f}|_Y=f$.
\end{corollary}

\begin{remarks}
    こうして,TVSの局所凸性は,Topの正規性に似た消息だと理解できる.
\end{remarks}

\begin{corollary}[Mazur]
    $X$を局所凸空間,$B$を絶対凸な閉集合とする.
    $x_0\in X\setminus B$ならば,ある$f\in X^*$が存在して,$f(B)\subset[0,1]$かつ$f(x_0)>1$を満たす.
\end{corollary}

\section{弱位相}

\begin{tcolorbox}[colframe=ForestGreen, colback=ForestGreen!10!white,breakable,colbacktitle=ForestGreen!40!white,coltitle=black,fonttitle=\bfseries\sffamily,
title=]
    作用素環の意味での弱位相/強位相とは,混用されるが全くの別物である.
    区別の意味で,こちらはBanach空間としての弱位相などと呼ばれる.

    $X^*$(の部分集合)が$X$に定める始位相を弱位相といい,$X^{**}$(の部分集合)が$X^*$に定める始位相を$*$-弱位相という.
    いずれの位相も極限が一意であることは,Hahn-Banachの分離定理から従う.
\end{tcolorbox}

\subsection{弱位相を定める関数族の特徴付け}

\begin{tcolorbox}[colframe=ForestGreen, colback=ForestGreen!10!white,breakable,colbacktitle=ForestGreen!40!white,coltitle=black,fonttitle=\bfseries\sffamily,
title=]
    $\sigma(X,Y)$-位相に関する双対空間は$X^*=Y$となる.
    $Y$として評価写像を取れば,これは収束に関する言葉で特徴づけられることになる(弱位相の収束の特徴付け\ref{prop-characterization-of-convergence-in-wtopology},$w^*$-位相の収束の特徴付け\ref{lemma-characterization-of-convergence-in-w*topology}).
\end{tcolorbox}

\begin{lemma}
    $\varphi,\varphi_1,\cdots,\varphi_n\in X^*$を線型空間$X$上の汎関数とする.次の3条件は同値.
    \begin{enumerate}
        \item $\exists_{\Brace{\al_1,\cdots,\al_n}\subset\bF}\;\varphi=\sum_{k\in[n]}\al_k\varphi_k$.
        \item $\exists_{\al>0}\;\forall_{x\in X}\;\abs{\varphi(x)}\le\al\max_{k\in[n]}\abs{\varphi_k(x)}$.
        \item $\cap_{k\in[n]}\Ker\varphi_k\subset\Ker\varphi$.
    \end{enumerate}
\end{lemma}

\begin{proposition}[$\sigma(X,Y)$-位相の双対空間]\label{prop-dual-space-of-TVS-with-weak-topology}
    線型空間$X$と,$X$の点を分離する線型汎関数のなす線型空間$Y$を考える.
    $X$に$Y$定める弱位相$\sigma(X,Y)$を入れて考える.
    このとき,線型汎関数$\varphi$がこの弱位相について連続ならば,$\varphi\in Y$である.
\end{proposition}
\begin{corollary}[対称な同値命題]
    線型空間の代数的双対ペア$(X,Y)$について,$\brac{-,Y},\brac{X,-}$がそれぞれ$X,Y$に定める弱位相は
    局所凸で,$X^*=Y,Y^*=X$を満たす.
\end{corollary}

\subsection{弱収束の特徴付け}

\begin{tcolorbox}[colframe=ForestGreen, colback=ForestGreen!10!white,breakable,colbacktitle=ForestGreen!40!white,coltitle=black,fonttitle=\bfseries\sffamily,
title=]
    ノルム収束列は弱収束する.
\end{tcolorbox}

\begin{proposition}[弱収束の特徴付け]\label{prop-characterization-of-convergence-in-wtopology}
    $X$上のネット$(x_\lambda)_{\lambda\in\Lambda}$について,次の2条件は同値.
    \begin{enumerate}
        \item $x$に弱位相$\sigma(X,X^*)$について収束する.
        \item $\forall_{\varphi\in X^*}\;\varphi(x_\lambda)\to\varphi(x)$.
    \end{enumerate}
\end{proposition}

\begin{theorem}[ノルム空間の弱収束の特徴付け]
    $X$をノルム空間とし,$\{x_n\}\subset X$点列とする.次の2条件は同値.
    \begin{enumerate}
        \item $(x_n)$は弱収束する.
        \item \begin{enumerate}[(a)]
            \item ノルム有界:$\sup_{n\in\N}\norm{x_n}<\infty$.
            \item 任意のノルム稠密な部分集合$Y\subset X^*$について,$\forall_{f\in Y}\;f(x_n)\to f(x)$.
        \end{enumerate}
    \end{enumerate}
\end{theorem}


\begin{theorem}[ノルム空間の弱収束の必要条件]
    $X$をノルム空間とし,$\{x_n\}\subset X$を点列,$x\in X$とする.(1)$\Rightarrow$(2)が成り立つ.
    \begin{enumerate}
        \item $\{x_n\}$は$x$に弱収束する.
        \item ノルムの弱位相に関する半連続性:$\{x_n\}$はノルム有界で:$\sup_{n\in\N}\norm{x_n}<\infty$,$\norm{x}\le\liminf_{n\to\infty}\norm{x_n}$を満たす.
    \end{enumerate}
\end{theorem}

\subsection{強収束との関係}

\begin{theorem}
    $\{u_n\}\subset X$を弱収束列とする.
    \begin{enumerate}
        \item $\{u_n\}$は$V^*$のノルム位相についてコンパクト一様収束する.
        \item $\{u_n\}$がさらにノルム収束するための必要十分条件は,$B^*\subset V^*$上一様に収束することである.
    \end{enumerate}
\end{theorem}

\begin{theorem}
    Hilbert空間における弱収束列が,ノルムを通じても収束するとき,ノルム収束もする.
\end{theorem}

\subsection{弱閉集合}

\begin{tcolorbox}[colframe=ForestGreen, colback=ForestGreen!10!white,breakable,colbacktitle=ForestGreen!40!white,coltitle=black,fonttitle=\bfseries\sffamily,
    title=]
    弱位相は一般にノルム位相よりも弱いが,双対空間は$X^*$で一致するために,
    凸集合について,閉性は同値になる(閉凸集合を共有する).
\end{tcolorbox}

\begin{theorem}
    $X$を局所凸空間,$C\subset X$を凸集合とする.
    $C$の弱閉法は,元の位相に関する閉包と一致する.
\end{theorem}

\begin{proposition}\label{prop-closedness-of-convex-sets}
    任意のノルム空間$X$の凸部分集合$C\subset X$について,次の2条件は同値.
    \begin{enumerate}
        \item ノルム閉である.
        \item 弱閉である.
    \end{enumerate}
    また,ノルム稠密であることと弱稠密であることも同値.
\end{proposition}
\begin{Proof}
    (1)$\Rightarrow$(2)は明らかだから,(2)$\Rightarrow$(1)を示す.
    $x\in X\setminus C$を任意に取る.
    すると,$C$と互いに素な$x$中心の開球$B$が取れる.
    Hahn-Banachの分離定理\ref{thm-Hahn-Banach-separation-theorem}より,
    $U:=\Brace{y\in X\mid\Re\varphi(y)\ge t}$なる$C$の弱位相における閉近傍が見つかる.
    この補集合は,$x\in C$の開近傍である.よって,$X\setminus C$は弱位相について開いている.
\end{Proof}
\begin{remarks}[同じ双対空間を持つということの帰結]
    弱位相はノルム位相よりも開集合の数は同じか少ない.
    閉集合の数も同様であるが,凸集合については保たれる.
    この議論を一般化すると,\textbf{凸集合が閉であるという条件は,同じ双対を持つどの局所凸位相についても同値である}.
\end{remarks}

\begin{corollary}
    $X$を距離化可能な局所凸空間,$\{x_n\}\subset X$を$x\in X$に弱収束する点列とする.
    このとき,列$\{y_i\}\subset X$が存在して,次を満たす:
    \begin{enumerate}
        \item 任意の$y_i$は有限個の$x_n$の凸結合である.
        \item 元来の位相に関して,$y_i\to x$.
    \end{enumerate}
\end{corollary}

\subsection{弱閉包}



\begin{theorem}[Mazur, S.]
    $X$を局所凸空間,$E\subset X$を凸集合とする.
    弱閉包$\o{E}$と元来の閉包$\oo{E}$とは等しい.
\end{theorem}

\begin{corollary}
    局所凸空間の凸な部分集合について,
    \begin{enumerate}
        \item 元来の閉集合と弱閉集合とは同値.
        \item 元来の稠密集合と弱稠密集合とは同値.
    \end{enumerate}
\end{corollary}

\begin{corollary}
    $X$をノルム空間とし,$\{x_n\}\subset X$を$x$に弱収束する点列とする.
    任意の$\ep>0$に対して,$x_n$の凸結合が存在して,
    \[\Norm{x-\sum_{j=1}^n\al_jx_j}\le\ep.\]
\end{corollary}
\begin{Proof}
    ここでは直接示してみる.
    \begin{description}
        \item[方針] $x_n\wto x$は$x_n-x_0\wto x_\infty-x_0$と同値だから,改めて$x_n-x_0$を$x_n$と取り直すことで,$x_0=0$を仮定しても一般性は失われない.
        このとき,
        \[M_1:=\Brace{\sum^n_{i=1}\al_ix_i\in X\;\middle|\;\al_i\ge0,\sum^n_{i=1}\al_i=1}\]
        とすると,$x_0=0$の仮定より$0\in M_1$である.$\exists_{\ep>0}\;\forall_{u\in M}\;\norm{x_\infty-u}>\ep$と仮定して矛盾を導く.
        \item[優越するMinkowski汎関数の構成] 
        \[M:=\Brace{v\in X\mid\exists_{u\in M_1}\;\norm{v-u}\le\ep/2}\]
        とすると,$M_1\subset M$を満たす$0$の凸近傍である.よって,$\mu_M(x):=\inf\Brace{t>0\mid t^{-1}x\in M}$とおくと,これはMinkowski汎関数である.
        \item[Hahn-Banachの定理による帰謬]
        いま$\forall_{v\in M}\;\norm{x_\infty-v}>\ep/2$なので,$\mu_M(x_\infty)>1$より,ある$\mu_M(u_0)=1$を満たす$u_0\in X$と$\beta>1$を用いて$x_\infty=\beta u_0$と表せる.
        ここで,
        \[X_1:=\Brace{x\in X\mid\exists_{\gamma\in\R}\;x=\gamma u_0}\]
        とすると,$x_\infty\in X_1$を満たす部分空間である.
        この上の有界線型汎関数を$f_1(x)=\gamma\;(x=\gamma u_0\text{のとき})$で定めると,これは$f_1\le\mu_M\;\on X$を満たす.よって,$f\le p$を満たす$X$上への有界線型な延長$f:X\to\R$が存在する:$f\in X^*$.
        これまでの議論より
        \[\sup_{x\in M_1}f(x)\le\sup_{x\in M}f(x)\le\sup_{x\in M}\mu_M(x)=1<\beta=f(\beta u_0)=f(x_\infty)\]
        であるから,$f(x_n)\to f(x_\infty)$に矛盾.
    \end{description}
\end{Proof}

\subsection{点列完備性}

\begin{tcolorbox}[colframe=ForestGreen, colback=ForestGreen!10!white,breakable,colbacktitle=ForestGreen!40!white,coltitle=black,fonttitle=\bfseries\sffamily,
title=]
    $w^*$-位相は点列完備であったが,弱位相はそうとは限らない.
    仮に$\{x_n\}\subset X$が$\forall_{T\in X^*}\;T(x_n)\to T(x)$を満たしたとしても,$x^*(T):=\lim_{n\to\infty}T(x_n)$としたとき$x^*\in X^{**}$であっても,元の$X$に入るとは限らない.
\end{tcolorbox}

\begin{theorem}
    Banach空間$X$が回帰的ならば,弱位相に関して点列完備である.
\end{theorem}

\subsection{弱有界集合}

\begin{tcolorbox}[colframe=ForestGreen, colback=ForestGreen!10!white,breakable,colbacktitle=ForestGreen!40!white,coltitle=black,fonttitle=\bfseries\sffamily,
title=]
任意の$V:=\Brace{x\in X\mid\abs{T_ix}<r_i,T_i\in X^*,r_i>0}$の形をした$0$の開近傍は,基本系$\B$をなす.
\end{tcolorbox}

\begin{proposition}
    $X^*$は位相線型空間$X$の点を分離するとする(局所凸ならば成立).
    このとき,部分集合$A\subset X$について,次の3条件は同値.
    \begin{enumerate}
        \item $A$は弱有界である.
        \item $\forall_{V\in\B}\;\exists_{t\in>0}\;tA\subset V$.
        \item $\forall_{T\in X^*}\;\exists_{\gamma\in\R_+}\;\forall_{x\in A}\;\abs{Tx}\le\gamma$.
        \item 任意の$T\in X^*$は$A$上有界である.
    \end{enumerate}
\end{proposition}

\begin{corollary}
    $X$は無限次元ならば,任意の$0$の弱近傍は無限次元部分空間を含む.
    特に,弱位相は局所有界でなく,従ってノルム付け可能ではない.
\end{corollary}

\begin{proposition}
    局所凸空間$X$において,弱有界集合と元来の位相について有界な集合とは一致する.
\end{proposition}

\begin{corollary}
    $X$をノルム空間,$A\subset X$を$\forall_{f\in X^*}\;\sup_{x\in A}\abs{fx}<\infty$を満たす集合とする.
    このとき,$A$はノルム有界.
\end{corollary}

\subsection{$w^*$-位相に関する収束}

\begin{tcolorbox}[colframe=ForestGreen, colback=ForestGreen!10!white,breakable,colbacktitle=ForestGreen!40!white,coltitle=black,fonttitle=\bfseries\sffamily,
title=]
    汎関数の空間$X^*$には標準的な分離族$X\mono X^{**}$が考えられ,これについての始位相を$*$-弱位相という.
    これは,$x\in X$での評価写像$\ev_x$を連続にする最弱の位相である.
    $X^*$の作用素ノルムが定める位相より弱く,弱位相よりも弱いが,
    $w^*$-位相は各点収束位相として特徴付けられるため,自然な対象である.
    Alaogluの定理より,$\dim(X)<\infty$の場合のみノルム位相と$w^*$-位相は一致する.
    回帰的な場合のみ弱位相と$w^*$-位相は一致する.
\end{tcolorbox}

\begin{definition}[weak-star topology]\label{def-weak-star-topology}
    ノルム空間$X$は,\ref{exp-dual-pair}(2)の埋め込み$i:X\mono X^{**}$により,線型汎函数の空間$X^*$上の分離族とみれる.
    こうして$X^*$に引き起こされる$\sigma(X^*,X)$位相を,\textbf{$*$-弱位相}という.
    \footnote{すなわち,二重共役空間の部分空間$X\mono X^{**}$に属する写像を全て連続にする最弱の位相.}
    こうして,$X^*$は局所凸な位相線型空間となり,$X$を双対空間と同一視できる(命題\ref{prop-dual-space-of-TVS-with-weak-topology}).
\end{definition}
\begin{remarks}
    $*$-弱位相の始位相としての定義より,$X^*$にノルムが定める位相よりも,弱位相$\sigma(X^*,X^{**})$よりも弱い.
    このことから名前がついた.
\end{remarks}

\begin{lemma}[$*$-弱収束は各点収束]\label{lemma-characterization-of-convergence-in-w*topology}
    $*$-弱位相における収束は各点収束である.すなわち,$X^*$上のネット$(\varphi_\lambda)_{\lambda\in\Lambda}$について,次の2条件は同値.
    \begin{enumerate}
        \item $w^*$-位相について$\varphi$に収束する.
        \item $\forall_{x\in X}\;\varphi_\lambda(x)\to\varphi(x)$.
    \end{enumerate}
\end{lemma}

\subsection{$w^*$-閉部分空間の特徴付け}

\begin{proposition}
    $X$をノルム空間,$Z\subset X^*$を$w^*$-閉部分空間とする.
    任意の$\varphi\in X^*\setminus Z$について,$x\in Z^\perp$が存在して,$\brac{x,\varphi}\ne 0$を満たす.
\end{proposition}
\begin{Proof}\mbox{}
    \begin{enumerate}[(a)]
        \item 任意の$\varphi\in X^*\setminus Z$を取ると,ノルム空間は正規より,近傍によって分離出来るが,$X^*$は$w^*$-位相に関して局所凸だから,特に$w^*$-開な近傍$U\in\O(\varphi)$が取れる.
        \item 位相線型空間$(X^*,\sigma(X^*,X))$の凸開集合$C$と凸集合$Z$に関するHahn-Banachの分離定理より,ある連続線型汎関数,すなわち,ある$x\in X$が存在して,$\exists_{t\in\R}\Re\brac{x,\varphi}<t\le\Re\brac{x,y}\;(\forall_{y\in Z})$が成り立つ.$X^*$上の連続汎関数は必ず$\exists_{x\in X}\;\brac{x,-}$と表せることに注意.
        \item $Z$は部分空間だったから,$x\in Z^\perp$かつ$t\le0$である.仮に$\exists_{y\in Z}\;\Re\brac{x,Z}\ne0$ならば,$\Re\brac{x,Z}=\R$となって矛盾.
    \end{enumerate}
\end{Proof}

\begin{corollary}[$w^*$-閉な部分空間の表現]\label{cor-weak-star-closed-subspace}
    $X^*$の任意の$w^*$-閉な部分空間について,あるノルム閉な部分空間$Y\subset X$が存在して,$Y^\perp$と表せる.
\end{corollary}
\begin{Proof}
    $Z\subset X^*$は$w^*$-閉とする.$Z=X^*$ならば$0^\perp=Z$だから,$Z\subsetneq X^*$とすると,
    命題は$(Z^\perp)^\perp\subset Z$を含意する.$Z\subset(Z^\perp)^\perp$は常に成り立つから,$Z=(Z^\perp)^\perp$が成り立つ.$Z^\perp$はノルム閉である.
\end{Proof}


\subsection{随伴と$w^*$-位相}

\begin{tcolorbox}[colframe=ForestGreen, colback=ForestGreen!10!white,breakable,colbacktitle=ForestGreen!40!white,coltitle=black,fonttitle=\bfseries\sffamily,
title=]
    随伴とは$\sigma(X,X^*)$-位相に関する対応であるから,$\sigma(X^*,X)$の言葉で特徴づけられるのは自然なことである.
\end{tcolorbox}

\begin{proposition}[随伴の特徴付け]
    ${-}^*:B(X,Y)\to\Hom_\TVS(Y^*,X^*)$は全単射である.すなわち,
    \begin{enumerate}
        \item Banach空間$X,Y$間の作用素$T\in B(X,Y)$について,その随伴$T^*:Y^*\to X^*$は$w^*$-連続である.
        \item 任意の$w^*$-連続な作用素$S:Y^*\to X^*$\footnote{有界性,すなわち$S\in B(Y^*,X^*)$は仮定しない}について,ある$T\in B(X,Y)$が存在して,$S=T^*$と表せる.特に,$S\in B(Y^*,X^*)$.
    \end{enumerate}
\end{proposition}

\begin{proposition}\label{prop-標準写像の随伴}
    ノルム空間$X$の閉部分空間$Y$と,包含写像$I:Y\to X$と商写像$Q:X\to X/Y$について,
    \begin{enumerate}
        \item 随伴$Q^*$と包含写像$J:Y^\perp\to X^*$とを同一視できる.
        \item 随伴$I^*$と商写像$R:X^*\to X^*/Y^\perp$とを同一視できる.
    \end{enumerate}
\end{proposition}

\section{弱コンパクト集合}



\subsection{双対空間の単位閉球は$w^*$-コンパクト}

\begin{tcolorbox}[colframe=ForestGreen, colback=ForestGreen!10!white,breakable,colbacktitle=ForestGreen!40!white,coltitle=black,fonttitle=\bfseries\sffamily,
title=双対理論を支えている消息]
    位相線型空間は局所コンパクトならば有限次元だから,
    $0$は相対コンパクトな近傍を持ってはならない.
    しかし,$w^*$-位相では$B^*$はコンパクトである.
    実は$B^*$は$0$の$w^*$-開近傍を含まない.
\end{tcolorbox}

\begin{lemma}
    ノルム空間$X$について,
    \begin{enumerate}
        \item $B^*$は$w^*$-閉集合である.
        \item $B^*$が$0$の$w^*$-近傍であるならば,$X$は有限次元である.
    \end{enumerate}
\end{lemma}
\begin{Proof}\mbox{}
    \begin{enumerate}
        \item 任意の$B^*$の$w^*$-収束ネット$(\varphi_\lambda)$に対して,収束先$\varphi\in X^*$が$B^*$内に入ることを示せば良い.
        すなわち,$\forall_{x\in X}\;\varphi_\lambda(x)\to\varphi(x)$.いま,$(\varphi_\lambda)$は$B^*$のネットだから,$\forall_{\lambda\in\Lambda}\;\varphi_\lambda(x)\le\norm{x}$.よって,$\varphi(x)\le\norm{x}$より,$\varphi\in B^*$.
        \item $B^*$が$0$の$w^*$-近傍ならば,$X$は$w^*$-位相について局所コンパクトである.よって,$X$は有限次元\ref{thm-character-of-TVS}.
    \end{enumerate}
\end{Proof}

\begin{theorem}[Alaoglu's theorem]\label{thm-Alaoglu}
    ノルム空間$X$について,$B^*$は$w^*$-コンパクトである.
\end{theorem}
\begin{remarks}[Bourbaki-Alaoglu]
    実は一般に,$X$における$0$の絶対凸近傍$U$の極集合$U^\circ$が$\sigma(X^*,X)$-コンパクトになる.
    これはBourbaki-AlaogluまたはBanach-Alaogluの定理という.
    上の消息は,自然なペアリングに関する極集合$B^\circ=\Brace{\varphi\in X^*\mid\forall_{x\in B}\abs{\varphi x}\le1}$に関する結果である.
\end{remarks}
\begin{remarks}
    $B^*$が$\sigma(X^*,X)$-コンパクトであることは,$w^*$-位相による$X^*$の双対空間が$X$になることに起因する.\footnote{凸集合についてはノルム閉と弱閉が同値である議論と並行だろう.}
    ノルム空間$X$の単位閉球$B$がコンパクトになることが,$X$が回帰的であることに同値であることと同じ消息である.
\end{remarks}

\subsection{弱コンパクト凸集合の凸閉包による復元}

\begin{tcolorbox}[colframe=ForestGreen, colback=ForestGreen!10!white,breakable,colbacktitle=ForestGreen!40!white,coltitle=black,fonttitle=\bfseries\sffamily,
title=コンパクト凸集合の代数的特徴付けを与える]
    凸集合には,凸結合(convex conbination)を線型結合のようなものだと思うと「基底」なるべき概念があり,それらの情報だけで全体の形を復元できる.
    極点に当たる関数は大抵特殊な振る舞いをするので,取っ掛かりになる.
    その後,凸結合に対する安定性(これは線形性が十分条件となることが注意)と連続性を確認すれば,目標の凸集合上での成立を示せる.
    実際,デルタ分布だけでなく,Cauchyの積分表示が対象とする関数クラス,ユニタリ行列,直交行列はすべてある凸集合の極点として特徴付けられる.
\end{tcolorbox}

\begin{theorem}[Krein-Milman]
    $X^*$は位相線型空間$X$の点を分離するとする(局所凸ならば成立\ref{cor-separating-subspace-is-dense}).
    弱コンパクトな凸集合$C\subset X$について,極境界$\partial C$の凸包は$C$上弱稠密である:$C=\o{\Conv(\Ex(C))}$.
\end{theorem}
\begin{remarks}
    $C$を凸としたのは,$\o{\Conv(\Ex(C))}$をコンパクトにするためである.
    $X$が局所凸ならば,次のようになる.
\end{remarks}

\begin{theorem}
    $X$を局所凸空間,$K\subset X$をコンパクト部分集合とする.
    このとき,$K\subset\o{\Conv(\Ex(K))}(=\o{\Conv(K)})$.
\end{theorem}

\begin{corollary}\label{cor-predual-of-c0}
    $X^*\simeq c_0$となるようなBanach空間$X$は存在しない.
\end{corollary}
\begin{remarks}
    Krein-Milmanの定理は,このように$X^*$上の$w^*$-コンパクト集合に使うことで重要な系を得る.
    他にも,$L^1([0,1])$もpredualを持たない.
\end{remarks}

\subsection{確率測度のなす部分空間}

\begin{tcolorbox}[colframe=ForestGreen, colback=ForestGreen!10!white,breakable,colbacktitle=ForestGreen!40!white,coltitle=black,fonttitle=\bfseries\sffamily,
title=]
    例として,確率測度がBanach代数の双対空間$(C(X))^*$の中でなす部分空間としての性質を見る.
    測度を,関数の双対空間の元と考える.つまり,セミパラ理論でも見た,関数への作用素としての測度の理解である.
    これは,積分を有界線形作用素とみる世界観の始まりでもある.「加法的集合関数」なる概念は時代遅れである.
\end{tcolorbox}

\begin{corollary}\label{cor-space-of-probability-measures}
    $\infty$-ノルムを備えたコンパクトハウスドルフ空間上のBanach代数$C(X)$を考える.
    $C(X)$の双対空間を$M(X)$,$P(X):=\Brace{\mu\in M(X)\mid\norm{\mu}\le 1,\mu(1)=1}$を確率測度のなす部分空間とする.
    \begin{enumerate}
        \item $P(X)$は$M(X)$の凸集合である.
        \item $P(X)$は$w^*$-コンパクトである.
        \item $P(X)$の極点はDirac測度$\delta_x\;(x\in X),\forall_{f\in C(X)}\;\delta_x(f)=f(x)$である.
    \end{enumerate}
\end{corollary}
\begin{Proof}\mbox{}
    \begin{enumerate}
        \item $\mu_1,\mu_2\in P(X)$と$\lambda\in(0,1)$を任意に取ると,$\lambda\mu_1+(1-\lambda)\mu_2\in P(X)$がわかる.
        \item 線型汎函数$\ev_1:M(X)\to\bF;\mu\mapsto\mu(1)$は$w^*$-位相について連続である
        $M(X)=(C(X))^*$の閉単位球$B^*$は$w^*$-コンパクト\ref{thm-Alaoglu}である.
    \end{enumerate}
\end{Proof}
\begin{remarks}
    Krein-Milmanの定理と併せ見ると,
    コンパクトハウスドルフ空間$X$上の確率測度は,
    $(C(X))^*$上において,
    有限な台を持つ測度(=Dirac測度の凸結合)によって
    各点近似($w^*$-位相は各点収束の位相\ref{lemma-characterization-of-convergence-in-w*topology})が出来る.
    これは経験過程に対する大数の法則である.
\end{remarks}

\begin{proposition}
    $X$をコンパクトハウスドルフ空間,$\{f_n\}\subset C(X)$を有界列とする.
    $(f_n)$が$f\in C(X)$に各点収束するならば,任意の$\ep>0$に対して凸結合$g:=\sum\lambda_nf_n$が存在して,$\norm{f-g}_\infty<\ep$を満たす.
\end{proposition}

\subsection{Radon測度の集合}

\begin{tcolorbox}[colframe=ForestGreen, colback=ForestGreen!10!white,breakable,colbacktitle=ForestGreen!40!white,coltitle=black,fonttitle=\bfseries\sffamily,
title=]
    $C[0,1]$の双対空間は$[0,1]$上の有限な全変動を持つBaire測度全体の空間になる.
    一般に,$(L^\infty(\Om;\C))^*$は代数$L^\infty$のスペクトル$K$上の複素Radon測度の空間と同型になる.これは$L^1(\Om)$より大きいのである.
\end{tcolorbox}

\subsection{凸集合の$w^*$-コンパクト性の特徴付け}

\begin{tcolorbox}[colframe=ForestGreen, colback=ForestGreen!10!white,breakable,colbacktitle=ForestGreen!40!white,coltitle=black,fonttitle=\bfseries\sffamily,
title=$w^*$-閉性の特徴付けは,単位閉球の言葉によってなされる.]
    ある凸集合が弱$*$-閉であることを示したいとき,弱位相での特徴付け\ref{prop-closedness-of-convex-sets}は弱$*$-位相が弱位相より真に弱い時には失敗するから(ノルム閉だが$w^*$-閉でない集合が存在する:$\Ker x^{**}\;(x^{**}\in X^{**}\setminus X)$など),
    別の特徴付けを探したい.
    Krein-Smulianの定理は,弱$*$-位相一般について成り立つ.
\end{tcolorbox}

\begin{theorem}[Krein-Smulian theorem]
    $X$をBanach空間,
    $C\subset X^*$を凸集合とする.次の2条件は同値である.
    \begin{enumerate}
        \item $\forall_{r>0}\;rB^*\cap C$は$w^*$-閉である.従って,$rB^*$は$w^*$-コンパクトだから,$rB^*\cap C$も$w^*$-コンパクトである.
        \item $C$は$w^*$-閉である.
    \end{enumerate}
\end{theorem}
\begin{Proof}\mbox{}
    \begin{description}
        \item[方針] (2)$\Rightarrow$(1)は明らかだから,(1)$\Rightarrow$(2)を示す.
        $\forall_{r>0}\;rB^*\cap C$が$w^*$-閉であるとき,$C$はノルム閉でもある.\footnote{実際,$C$の任意のノルム収束列$(x_n)$を取ると,これは有界列だから,$\exists_{r>0}\;\{x_n\}\subset rB^*\cap C$.$rB^*\cap C$は$w^*$-閉という仮定よりノルム閉でもあるから,$(x_n)$は$C$上でノルム収束する.}
        よって,$\varphi\in X^*\setminus C$のとき,$0\notin C-\varphi$だから,ある$r>0$が存在して$rB^*\cap(C-\varphi)=\emptyset$が成り立つ.
        そこで,$r=1$とし,$C$を$C-\varphi$として取り直すことで,$B^*\cap C=\emptyset$なる状況について,必ず$0\notin\o{C}$であることを示せば良い.
        \item[主張] $X^*$の閉球は,極集合として$(rB)^\circ=r^{-1}B^*$と表せる.
        
        $r^{-1}B^*\subset(rB)^\circ$は明らか.実際,任意の$\varphi\in r^{-1}B^*$について,$\norm{\varphi}\le r^{-1}$より,任意の$x\in rB$に対して,$\abs{\varphi(x)}\le\norm{x}\norm{\varphi}\le1$だから,特に$\Re\varphi(x)\ge-1$.よって,$\varphi\in(rB)^\circ$.

        $r^{-1}B^*\supset(rB)^\circ$について,任意に$\varphi\notin r^{-1}B^*$を取る.
        $w^*$-位相を備えた空間$X^*$の双対空間は$X$であることに注意すると,$\varphi$の$r^{-1}B^*$と交わらない任意の開球と$r^{-1}B^*$とについて,
        Hahn-Banachの分離定理\ref{thm-Hahn-Banach-separation-theorem}より,$x\in X$と$t\in\R$が存在して,
        \[\Re\brac{x,\varphi}<t\le\Re\brac{x,r^{-1}B^*}\]
        を満たす.ここで,$\Re\brac{x,r^{-1}B^*}=[-r^{-1}\norm{x},r^{-1}\norm{x}]\subset\R$だから(作用素ノルムの定義と系\ref{cor-Hahn-Banach}より),
        $x\in rB$を$\norm{x}=r$を満たすように上式を正規化することで,$\Re\brac{x,\varphi}<t\le-1$,特に$\varphi\notin(rB)^\circ$を得る.
        \item[有界線型作用素の構成]
        $X$の有限部分集合の列$(F_n)$を次のように定める:
        \begin{enumerate}
            \item $F_1=\{0\}$.
            \item $D:=(n+1)B^*\cap C\cap P(F_1)\cap\cdots\cap P(F_n)=\emptyset$とおくと,$w^*$-コンパクトな凸集合の有限共通部分だから$D$も$w^*$-コンパクトな凸集合で\ref{lemma-polar},帰納的に$D\cap nB^*=\emptyset$を満たす($n=1$のときは$C\cap B^*=\emptyset$は仮定した).
            すなわち,$\emptyset=D\cap P(n^{-1}B)=\cap_{x\in n^{-1}B}D\cap P(\{x\})$であるから,$D$のコンパクト性より,$D\cap P(F_{n+1})=\emptyset$を満たす有限部分集合$F_{n+1}\subset n^{-1}B$が取れる.
        \end{enumerate}
        こうして構成した$(F_n)$は
        \[F_n\subset(n-1)^{-1}B\quad nB^*\cap C\cap P(F_1)\cap\cdots\cap P(F_n)=\emptyset\]
        を満たす.
        したがって,$\cup_{n\in\N}F_n=\{x_n\}$は$0$に収束する点列と思える.
        こうして,次の写像\ref{exp-direct-sum-of-norm-spaces}
        \[\xymatrix@R-2pc{
            T:X^*\ar[r]&c_0(\N)\\
            \rotatebox[origin=c]{90}{$\in$}&\rotatebox[origin=c]{90}{$\in$}\\
            \varphi\ar@{|->}[r]&(\brac{x_n,\varphi})_{n\in\N}
        }\]
        は有界線型作用素となる.
        \item[数列の空間への対応を用いて証明]
        $\varphi\in C$について,十分大きな$m\in\N$についても$mB^*\cap\{\varphi\}\cap P(\{x_n\mid n\in\N\})=\emptyset$より,$\inf_{n\in\N}\Re\brac{x_n,\varphi}\le-1$,よって特に$\norm{T\varphi}_\infty\ge1$.
        すなわち,像$T(C)$も凸であるが,$c_0(\N)$における単位開球$B_0$と互いに素である.
        よってHahn-Banachの分離定理\ref{thm-Hahn-Banach-separation-theorem}より,ある$\lambda=(\lambda_n)\in (c_0(\N))^*=l^1$と$t\in\R$が存在して,
        \[\Re\brac{B_0,\lambda}<t\le\Re\brac{T(C),\lambda}\]
        を満たす.上式を$\norm{\lambda}_1=1$を満たすように正規化すると,$1\le t$をみたすように$t$を取れる.
        よって,$x:=\sum_{n\in\N}\lambda_nx_n$とおくと$x\in X$で,任意の$\varphi\in C$について
        \[\Re\brac{x,\varphi}=\sum_{n\in\N}\Re\brac{\lambda_nx_n,\varphi}=\Re\brac{T\varphi,\lambda}\ge t\ge1\footnote{$X$と$X^*$上の双線型形式から,$l^1$上の内積に写している.}\]
        が成り立つ.特に,$0\notin C$で,$C$の$w^*$-閉包にも含まれないことがわかる.
    \end{description}
\end{Proof}
\begin{remarks}
    Banach空間$X$のノルム位相と弱位相について,凸集合$A\subset X$が弱閉であることと$\forall_{r>0}\;A\cap rB$が弱閉であることは同値.なぜならば,$\Rightarrow$:凸集合$rB$はノルム閉であるから弱閉でもある\ref{prop-closedness-of-convex-sets}.$\Leftarrow$:$A$がノルム閉であることを示せば十分だが,$A$のノルム収束列は有界で,あるノルム閉集合$\exists_{r>0}\;A\cap rB$に含まれるから,結局$A$はノルム閉.
    $X$が回帰的である場合は,弱$*$-位相は弱位相と同じ強さだから,全く同様の事実が成り立つ.しかし,弱$*$-位相が弱い場合は?
\end{remarks}

\begin{corollary}
    部分空間$Z\subset X^*$について,次の2条件は同値.
    \begin{enumerate}
        \item $Z$は$w^*$-閉である.
        \item $Z\cap B^*$は$w^*$-閉である.
    \end{enumerate}
    このとき,$Z\cap B^*$は$w^*$-閉コンパクト集合の閉部分集合だから,やはり$w^*$-コンパクトである.
\end{corollary}
\begin{Proof}
    (1)$\Rightarrow$(2)は明らか.(2)$\Rightarrow$(1)を考える.
    部分空間$Z$は凸集合である.
    $Z\cap B^*$が$w^*$-閉ならば,$rB^*\;(r>0)$も$w^*$-閉だから,$Z\cap rB^*$も$w^*$-閉.よってKrein-Smulianの定理より$Z$は$w^*$-閉.
\end{Proof}

\begin{corollary}\label{cor-characterization-of-weak-star-continuousness}
    $X^*$上の線型汎関数$x:X^*\to\bF$について,次の2条件は同値.
    \begin{enumerate}
        \item $w^*$-連続である.すなわち,$x\in X$\ref{prop-dual-space-of-TVS-with-weak-topology}.
        \item 閉単位球への制限$x|_{B^*}$が$w^*$-連続である.
    \end{enumerate}
\end{corollary}
\begin{Proof}
    (1)$\Rightarrow$(2)は明らかだから(2)$\Rightarrow$(1)を示す.
    $x:X^*\to\bF$が零であるとき,(2)$\Rightarrow$(1)は成り立つ.
    任意の凸な閉集合$E\subset\bF$に対して,$(x|_{B^*})^{-1}(E)=x^{-1}(E)\cap B^*$は$w^*$-閉集合だから,$x^{-1}(E)$も$w^*$-閉集合である.
\end{Proof}

\subsection{Banach空間の弱コンパクト集合}

\begin{theorem}[Krein-Smulian]
    $X$をBanach空間,$K$を弱コンパクト集合とする.
    閉凸包$\o{\Conv(K)}$は引き続き弱コンパクトである.
\end{theorem}

\begin{proposition}
    $X$を可分ノルム空間,$C\subset X^*$を$w^*$-コンパクト凸集合とする.
    極境界$\Ex(C)$は,$C$の$G_\delta$-集合,すなわち開集合の可算共通部分で表せる集合である.
\end{proposition}

\subsection{不動点定理}

\begin{theorem}[Markov's fixed point theorem]
    $X^*$は位相線型空間$X$の点を分離するとし(局所凸ならば成立\ref{cor-separating-subspace-is-dense}),
    $C\subset X$を弱コンパクトな凸集合とする.$\cT\subset\Hom_\TVS(X,X)$は,互いに可換な弱連続な作用素の族であって,$\forall_{T\in\cT}\;T(C)\subset C$を満たすならば,ある点$x\in X$が存在して$\forall_{T\in\cT}\;Tx=T$を満たす.
\end{theorem}

\begin{theorem}[Kakutani's fixed point theorem]
    $X$を局所凸空間,$K\subset X$を非空なコンパクト凸集合,$G\subset\Aff(K)$を$K$のaffine変換のなす同程度連続な群とする.
    このとき,$G$は$K$上に共通の不動点を持つ.
\end{theorem}

\section{可分空間}

\subsection{可分Banach空間の消息}

\begin{tcolorbox}[colframe=ForestGreen, colback=ForestGreen!10!white,breakable,colbacktitle=ForestGreen!40!white,coltitle=black,fonttitle=\bfseries\sffamily,
title=]
    収束列の空間$l^1$には,可算なSchauder基底が存在する.$c_0\mono c_\infty\mono c_b=l^\infty$で,$l^\infty$は可分でない.
    この包含関係は稠密からは程遠い.なお,$c_\infty$は収束列の空間とした.
    また,$c_c\mono l^p\mono k^q\mono c_0\;(0<p<q<\infty)$が成り立つ,包含関係は後者の位相について稠密である.
    なお,上述の包含関係はすべて真に成り立つ.
\end{tcolorbox}

\begin{example}[Tsirelson space (74)]
    どの部分空間も$l^p\;(1\le p<\infty)$とも$c_0$とも同型でないようなBanach空間が存在する.
\end{example}

\begin{theorem}[Bessaga and Pelczynski]\label{thm-Bessaga-and-Pelczynski}
    Banach空間$X$の列$(x_n)$について,これが生成する閉部分空間$\Span\Brace{x_n}$が$c_0$に同型な部分空間を含むための十分条件は,
    \begin{enumerate}
        \item $\inf_{n\in\N}\norm{x_n}>0$.
        \item $\exists_{C\ge0}\;\forall_{k\ge1}\;\forall_{\ep_j\in\{\pm 1\}}\;\Norm{\sum^k_{j=1}\ep_jx_j}\le C$.
    \end{enumerate}
\end{theorem}

\begin{theorem}
    任意の可分Banach空間は,ある$l^1$の商空間と等長同型である.
\end{theorem}

\subsection{双対空間が可分ならば可分}

\begin{theorem}
    Banach空間$X$の双対空間$X^*$は可分とする.このとき,$X$も可分である.
\end{theorem}
\begin{remark}
    $L^1$は可分だが,$(L^1)^*=L^\infty$は可分ではない.
\end{remark}

\subsection{可分性の$w^*$-位相の距離化可能性による特徴付け}

\begin{tcolorbox}[colframe=ForestGreen, colback=ForestGreen!10!white,breakable,colbacktitle=ForestGreen!40!white,coltitle=black,fonttitle=\bfseries\sffamily,
title=]
    弱位相と$w^*$-位相は決して距離化可能でないが,有界な集合に限ると距離化可能たり得る\cite{Conway}V.5.1.

    また,$B$が弱距離化可能であることと$X^*$が可分であることが同値で,$B$が弱コンパクトであることと$X$が回帰的であることが同値(Kakutani,\ref{thm-characterization-of-reflexive-Banach-spaces}).
\end{tcolorbox}

\begin{theorem}[$X$の可分性は双対空間の$w^*$-距離化可能性で分かる]\label{thm-metrizability-of-ball}
    $X$をBanach空間とする.次の2条件は同値.
    \begin{enumerate}
        \item $X$は可分である.
        \item $B^*$は$w^*$位相に関して距離化可能である.
    \end{enumerate}
    このとき,$X^*$も可分である.
\end{theorem}
\begin{remarks}
    一般に位相線型空間の射はある近傍上で有界ならば連続である\ref{thm-morphism-of-TVS}が,これはこれと全く関係のない例外的な消息である.
\end{remarks}

\begin{theorem}[双対命題]
    $X$をBanach空間とする.次の2条件は同値:
    \begin{enumerate}
        \item $X^*$は可分である.
        \item $B$は弱距離化可能である.
    \end{enumerate}
\end{theorem}

\begin{corollary}
    $E$を可分Banach空間,$\{f_n\}\subset E^*$を有界列とする.このとき,$\{f_n\}$は$\sigma(E^*,E)$-相対コンパクトである.すなわち,$w^*$-収束する部分列が存在する.
\end{corollary}


\subsection{$L^p$の可分性}


\begin{theorem}
    Lebesgue空間$L^p(\Om)\;(1\le p<\infty)$は可分である.
\end{theorem}
\begin{Proof}\mbox{}
    \begin{description}
        \item[方針] $\Om:=\R^n$の場合について示せば,一般の領域$\Om$については,$L^p(\Om)\mono L^p(\R^n)$を,$\R^n\setminus\Om$上で零と定めることにより部分集合とみなすことで,可分性が従う.
        いま,$\R^n$上の閉矩形全体のなす集合を
        \[\cR:=\Brace{\prod_{k=1}^n[a_k,b_k]\in P(\R^n)\;\middle|\;a_k< b_k\in\Q}\]
        と定めるとこれは可算集合で,$\sigma(\cR)=\B(\R^n)$を満たす.
        $\E\subset L^p(\R^n)$を,各閉矩形$R$の定義関数$1_R$が生成する$\Q$-線型空間とすると,これは可算集合である.
        あとは,$L^p(\Om)$上稠密であることを示せばよい.
        \item[$\E$が$L^p(\Om)$上稠密であることの証明] 任意の$f\in L^p(\R^n)$と$\ep>0$について,
        $C_c(\Om)$は$L^p(\Om)$上稠密であることより,$\exists_{f_1\in C_c(\R^n)}\;\norm{f-f_1}_p<\ep$.
        すると,$\supp f_1$はコンパクトだから,ある閉矩形$R\in\cR$が存在して,$\supp f_1\subset R$を満たす.
        このとき,$f_1$は連続だから,$R$を十分細かく$\cR$の元の和として$R=\cup_{k=1}^NR_d$かつ$\exists_{i\ne j}\;x\in R_i$かつ$x\in R_j$ならば$\exists_{i\in[d]}\;x\in\partial R_i$を満たすように取れる.
        こうして,任意の$\delta>0$に対して,$f_2\in\E$の値を各$R_j^\circ$上$\min_{x\in R_j}f_1(x)\le f_2\le\max_{x\in R_j}f_1(x)$を満たすように定め,各$\partial R_j$上では$0$とすれば,
        $\norm{f_1-f_2}_\infty<\delta$を満たすように出来る.とくに,$\delta<\frac{\ep}{\abs{R}^{1/p}}$を満たすように取れば,ある$f_2\in\E$について$\norm{f_1-f_2}_\infty<\ep$を満たすように取れる.

        以上より,$\norm{f-f_2}_p\le\norm{f-f_1}_p+\norm{f-f_2}<2\ep$.
        よって,$\E$は$L^p(\Om)$上稠密である.
    \end{description}
\end{Proof}

\begin{theorem}
    $L^\infty(\Om)$は可分でない.
\end{theorem}
\begin{Proof}\mbox{}
    \begin{enumerate}
        \item $\Om=\cup_{n\in\N}S_n$かつ$\forall_{n\in\N}\;\abs{S_n}>0$を満たすような$\Om$の可測な分割$\{S_n\}\subset\B(\Om)$が存在する.
        実際,$S_0:=\Om\setminus(2^{-1}\Om)$とすると,$\abs{S_0}=(1/2)^n\abs{\Om}$.同様に,$S_m:=S_{m-1}\setminus2^{-1}S_{m-1}$としていくと,各$(S_n)$は互いに素で,正の測度を持つ.
        \item 任意の$I\subset\N$に対して,$f_I:=1_{\cup_{n\in I}S_n}$を集合$\cup_{n\in I}S_n\subset\Om$の特性関数とすると,任意の$I\ne J\subset\N$に対して$\norm{f_I-f_J}_\infty=1$より,$\{B_{1/2}(f_I)\}_{I\in P(\N)}\subset L^\infty(\Om)$は互いに素な開集合の非可算無限族となる.
        ただし$B_{1/2}(f_I)$とは,$f_I\in L^\infty(\Om)$を中心とした半径$1/2$の開球とした.
        \item $X\subset L^\infty(\Om)$を稠密部分集合とすると,$\{X\cap B_{1/2}(f_I)\}$は非空集合の族であるから,選択公理より元$\{x_I\}_{I\in P(\N)}$が選び出せて,$x_I\in X\cap B_{1/2}(f_I)$を満たす.
        これにより,単射$P(\N)\to X;I\mapsto x_I$が定まったことになるから,$X$は非可算集合である.
    \end{enumerate}
\end{Proof}

\begin{theorem}
    $L^1(\Om)\subsetneq (L^\infty(\Om))^*$.
\end{theorem}
\begin{Proof}\mbox{}
    \begin{enumerate}
        \item $L^1(\Om)\ne (L^\infty(\Om))^*$である.仮に等号が成立するならば,$L^\infty(\Om)$は可分であることが必要だが,これは矛盾.
        \item $L^1(\Om)\subset (L^\infty(\Om))^*$である.任意の$u\in L^1(\Om)$に対して,対応
        \[\xymatrix@R-2pc{
            T_u:L^\infty(\Om)\ar[r]&\R\\
            \rotatebox[origin=c]{90}{$\in$}&\rotatebox[origin=c]{90}{$\in$}\\
            f\ar@{|->}[r]&T_u(f):=\int_\Om fudx
        }\]
        は$T_u\in(L^\infty(\Om))^*$を満たすことを示せば良い.$T_u$は明らかに線形作用素であり,
        またこれはHolderの不等式より,$\abs{T_uf}\le\norm{u}_1\norm{f}_\infty$が成り立つから,有界でもある.
    \end{enumerate}
\end{Proof}

\section{$L^p$の研究}

\begin{tcolorbox}[colframe=ForestGreen, colback=ForestGreen!10!white,breakable,colbacktitle=ForestGreen!40!white,coltitle=black,fonttitle=\bfseries\sffamily,
title=]
    特に$L^p(X)$空間について,
    弱収束と漸近収束との関係を考える.
\end{tcolorbox}

\subsection{弱収束の特徴付け}

\begin{theorem}[ノルム空間の弱収束の特徴付け]
    $X$をノルム空間とし,$\{x_n\}\subset X$点列とする.次の2条件は同値.
    \begin{enumerate}
        \item $(x_n)$は弱収束する.
        \item \begin{enumerate}[(a)]
            \item ノルム有界:$\sup_{n\in\N}\norm{x_n}<\infty$.
            \item 任意のノルム稠密な部分集合$Y\subset X^*$について,$\forall_{f\in Y}\;f(x_n)\to f(x)$.
        \end{enumerate}
    \end{enumerate}
\end{theorem}

\begin{corollary}[$L^1$の場合]
    可積分関数列$\{f_n\}\subset L^1(X)$について,弱収束することは次の2条件に同値:
    \begin{enumerate}
        \item $\{\norm{f_n}\}$は有界.
        \item $\forall_{A\in\A}\;\lim_{n\to\infty}\int_Af_nd\mu\in\R$が存在する.
    \end{enumerate}
\end{corollary}

\subsection{弱収束の十分条件}

\begin{proposition}
    $\{f_n\}\subset L^p(\R)\;(1<p<\infty)$は有界列で,$f\in L^p(\R)$に概収束するとする.このとき,$f_n\wto f$は弱収束もする.
\end{proposition}

\subsection{弱収束列がノルム収束する条件}

\begin{corollary}
    ある$f_\infty\in L^1(X,\M,\mu)$に弱収束する可積分関数列$\{f_n\}\subset L^1(X,\M,\mu)$について,ノルム収束することは次の条件に同値:
    \[\forall_{B\in\M^1}\;\forall_{\ep>0}\;\mu\paren{\Brace{x\in B\mid\abs{f_n(x)-f_\infty(x)}\ge\ep}}\xrightarrow{x\to\infty}0.\]
\end{corollary}
\begin{remarks}
    $l^1$で弱収束列はノルム収束するというSchurの定理はこの系である.
    全く同様に,超関数の空間$D'(\Om)$の列も,$w^*$-収束するならば強収束もする.
\end{remarks}

\subsection{弱コンパクト集合}

\begin{tcolorbox}[colframe=ForestGreen, colback=ForestGreen!10!white,breakable,colbacktitle=ForestGreen!40!white,coltitle=black,fonttitle=\bfseries\sffamily,
title=]
    $L^p(\Om)\;(1<p\le\infty)$の有界集合は弱コンパクトである.
    $L^1(\Om)$はperdualを持たないために,有界集合はさらに一様可積分でないと弱コンパクト性を持たない.
    しかし,$L^1(\Om)$の有界集合は$M(\Om)$から定まる相対位相=漠位相についてはコンパクトである.
    すなわち,唯一の問題は収束先が$L^1(\Om)$から出てしまう場合のみである.
\end{tcolorbox}

\begin{theorem}
    回帰的Banach空間は弱点列完備である.
\end{theorem}

\begin{corollary}
    任意の$L^p(\Om)\;(1<p\le\infty)$の有界集合は,$\sigma(L^p.L^{p^*})$-位相に対して相対コンパクトである.
\end{corollary}

\begin{theorem}[Dunford-Pettis]
    $\Om\osub\R^N$を有界開集合,$\F\subset L^1(\Om)$を有界とする.このとき,次の2条件は同値:
    \begin{enumerate}
        \item $\F$は弱位相$\sigma(L^1,L^\infty)$について相対コンパクトである.
        \item $\F$は一様可積分である:$\forall_{\ep>0}\;\exists_{\delta>0}\;\forall_{A\subset\Om}\;\forall_{f\in\F}\;\abs{A}<\delta\Rightarrow\int_A\abs{f}<\ep$.
    \end{enumerate}
\end{theorem}

\begin{proposition}[$w^*$-収束]
    埋め込み$T:L^1(\Om)\mono M(\Om)$を,$f\in L^1(\Om)$に対して
    \[\brac{Tf,u}:=\int fu\quad(u\in C_c(\Om))\]
    で定めると,これは等長写像になり,$L^1(\Om)$の有界集合は$w^*$-位相$\sigma(M,C_c)$-相対コンパクトである.
\end{proposition}

\begin{corollary}
    任意の有界列$\{f_n\}\subset L^1(\Om)$に対して,ある部分列$\{f_{n_k}\}$が存在して位相$\sigma(M,C_c)$に関して測度$\mu\in M(\Om)$に収束する.
\end{corollary}

\subsection{ノルム収束の十分条件}

\begin{theorem}
    Hilbert空間において,$\{f_n\}\subset X$を$f_\infty$に弱収束する列とする.次の2条件は同値:
    \begin{enumerate}
        \item $\{f_\infty\}$に強収束もする.
        \item $\norm{f_n}\xrightarrow{n\to\infty}\norm{f_\infty}$.
    \end{enumerate}
\end{theorem}

\begin{theorem}
    $X$を一様凸なBanach空間とし,$\{x_n\}$は$x$に弱収束するとする.このときさらに$\limsup_{n\in\N}\norm{x_n}\le\norm{x}$ならば,ノルム収束もする.
\end{theorem}

\subsection{ノルムコンパクト部分集合}

\begin{theorem}[Ascoli]
    $K$をコンパクト距離空間,$H\subset C(K)$を部分集合とする.
    $H$はノルム有界で,一様同程度連続ならば,$C(K)$で相対コンパクトである.
\end{theorem}

\begin{theorem}[Frechet-Kolmogorov]
    $L^p(\Om)\;(\Om\osub\R^N,1\le p<\infty)$,$\om\osub\Om$を$\o{\om}\subset\Om$を満たす開集合とする.
    有界集合$\F\subset L^p(\Om)$が次を満たすならば,$\F|_\om$は$L^p(\om)$で相対コンパクトである:
    \[\forall_{\ep>0}\;\exists_{\dist(\om,\Om^\comp)>\delta>0}\;\forall_{h\in\R^N}\;\forall_{f\in\F}\;\abs{h}<\delta\Rightarrow\norm{\tau_hf-f}_{L^p(\om)}<\ep\]
\end{theorem}

\begin{corollary}
    $\F\subset L^p(\Om)\;(\Om\osub\R^N,1\le p<\infty)$を有界集合とする.
    次の2条件は同値:
    \begin{enumerate}
        \item $\F$は$L^p(\Om)$-相対コンパクトである.
        \item \begin{enumerate}[(a)]
            \item $\forall_{\ep>0}\;\forall_{\om\subset\subset\Om}\;\exists_{\dist(\om,\Om^\comp)>\delta>0}\;\forall_{h\in\R^N}\;\forall_{f\in\F}\;\abs{h}<\delta\Rightarrow\norm{\tau_hf-f}_{L^p(\om)}<\ep$.
            \item $\forall_{\ep>0}\;\exists_{\om\subset\subset\Om}\;\forall_{f\in\F}\;\norm{f}_{L^p(\Om\setminus\om)}<\ep$.
        \end{enumerate}
    \end{enumerate}
\end{corollary}

\begin{corollary}
    ある$G\in L^1(\R^N)$と有界集合$\B\subset L^p(\R^N)\;(1\le p<\infty)$に対して,$\F:=G*\B$の任意の有界開集合$\om\osub\R^N$への制限$\F|_\om$は,$L^p(\om)$-コンパクトである.
\end{corollary}

\subsection{$L^p$のノルム収束}

\begin{theorem}[概収束部分列の存在]
    $\{f_n\}\subset L^p(\Om)\;(\Om\osub\R^n)$は$f$に$L^p$-ノルム収束するとする.このとき,ある部分列と$h\in L^p(\Om)$が存在して,$h$を優関数として概収束する:$\forall_{k\in\N}\;\abs{f_{n_k}(x)}\le h(x)\;\ae$かつ$f_{n_k}\to f\;\ae$
\end{theorem}

\subsection{$L^p$の有限次元部分空間}

\begin{theorem}[Grothendieck (\cite{Rudin},Th'm 5.2)]
    任意の$0<p<\infty$について,確率空間$(\Om,\F,P)$上の閉部分空間$S\subset L^p(P)$が$S\subset L^\infty(P)$も満たすならば,有限次元である.
\end{theorem}



\subsection{有界列の収束}

\begin{theorem}
    $\{f_n\}\subset X^*$を有界列とする.$X$が可分ならば,$\{f_n\}$は$w^*$-相対コンパクトである:$\exists_{\{f_{n_k}\}\subset\{f_n\}}\;\exists_{f\in X^*}\;w^*\text{-}\lim_{k\to\infty}f_{n_k}=f$.
\end{theorem}

\begin{theorem}
    $\{x_n\}\subset X$を有界列とする.$X$が回帰的ならば,$\{x_n\}$は弱相対コンパクトである:$\exists_{\{x_{n_k}\}\subset\{x_n\}}\;\exists_{x\in X}\;w\text{-}\lim_{k\to\infty}x_{n_k}=x$.
\end{theorem}



\section{Banach空間値関数}

\subsection{Banach空間値関数の可測性}

\begin{tcolorbox}[colframe=ForestGreen, colback=ForestGreen!10!white,breakable,colbacktitle=ForestGreen!40!white,coltitle=black,fonttitle=\bfseries\sffamily,
title=]
    有限次元空間では退化していた3つの可測性の概念が考え得る.
    \begin{description}
        \item[可測性] $X^*$が$X$に誘導する$\sigma$-代数はBorel $\sigma$-代数$\B(X)$より真に小さくなり得る.
        \item[強可測性] Lebesgue積分の定義で肝要であるのは,可測関数のクラスは単関数の各点収束閉包である事実である.これを抽出する.
    \end{description}
    強可測性の特徴付けをPettis measurability theoremが与える.
    このとき,弱可測性との関係は,タイト性に似た,可分性のバージョン「可分値」が鍵を握る.
    これより,弱可測$\Rightarrow$可測$\Rightarrow$強可測で,強可測ならば可測かつ可分値.
\end{tcolorbox}

\begin{notation}
    $(S,\A)$を可測空間,$X$をBanach空間とする.
    部分集合$Y\subset X^*$が$X$に誘導する$\sigma$-代数を$\sigma(Y)$で表すと,
    \[\sigma(Y)=\Brace{\cC(Y)\in P(Y)\mid\exists_{x_1^*,\cdots,x_n^*\in Y}\;\forall_{B\in\B(\K^n)}\;\cC(Y)=\Brace{x\in X\mid (\brac{x,x_1^*},\cdots,\brac{x,x_n^*})\in B}}.\]
    また,$\K$-値関数の線型空間$F\subset\Map(S,\K)$について,
    \[F\otimes X:=\Brace{\sum^N_{n=1}f_n\otimes x_n\in\Map(S,X)\mid\forall_{n\in[N]}\;f_n\in F,x_n\in X,N\in\N}\]
\end{notation}

\subsubsection{可測性}

\begin{definition}[measurable]
    関数$f:(S,\A)\to (X,\B(X))$について,
    \begin{enumerate}
        \item $f$が可測であるとは,$\forall_{B\in\B(X)}\;f^{-1}(B)\in\A$.
        \item 
    \end{enumerate}
\end{definition}

\begin{proposition}[可分ならば縮退する]
    $X$を可分,$Y\subset X^*$は$w^*$-稠密な部分空間とする.このとき,$\sigma(Y)=\sigma(X^*)=\B(X)$.
\end{proposition}

\begin{corollary}[可測性の特徴付け]
    $X$は可分であるとき,関数$f:S\to X$について次の2条件は同値.
    \begin{enumerate}
        \item $f$は可測.
        \item 任意の$x^*\in X^*$について,$\brac{f(-),x^*}:S\to\K$は可測.
    \end{enumerate}
\end{corollary}

\subsubsection{強可測性}

\begin{definition}[simple]
    関数$f:S\to X$が\textbf{単純}であるとは,$\exists_{N\in\N}\;\forall_{n\in[N]}\;\exists_{A_n\in\A,x_n\in X}\;f=\sum_{n=1}^N1_{A_n}\otimes x_n$.
\end{definition}

\begin{definition}[strongly measurable]
    関数$f:S\to X$が\textbf{強可測}または\textbf{Bochner可測}であるとは,単関数列$\{f_n\}\subset\Map(S,X)$が存在して,その各点極限である:$\lim_{n\to\infty}f_n=f$.
\end{definition}

\begin{lemma}
    $X$が可分であるとき,関数$f:S\to X$について次の2条件は同値.
    \begin{enumerate}
        \item $f$は強可測.
        \item $f$は可測.
    \end{enumerate}
\end{lemma}

\subsubsection{弱可測性と特徴付け}

\begin{definition}[separably valued, weakly measurable]
    関数$f:S\to X$について,
    \begin{enumerate}
        \item $f$が\textbf{可分値}であるとは,$\Im f$が可分であることをいう.
        \item $f$が\textbf{弱可測}であるとは,任意の$x^*\in X^*$について,$S$上の$\K$-値関数$s\mapsto\brac{f,x^*}(s):=\brac{f(s),x^*}$は可測であることをいう.すなわち,$\forall_{g\in X^*}\;g\circ f:S\to\K$が可測であることをいう.
    \end{enumerate}
\end{definition}

\begin{theorem}[Pettis measurability theorem]
    $Y\subset X^*$を$w^*$-稠密な部分空間とする.関数$f:S\to X$について,次の3条件は同値.
    \begin{enumerate}
        \item $f$は強可測.
        \item $f$は可分値で,弱可測.
        \item $f$は可分値で,$\forall_{g\in Y^*}\;g\circ f:S\to\K$は可測.
    \end{enumerate}
    また,ある閉部分空間$X_0\subset X$について$\Im f\subset X_0$が成り立つとき,$f$は$X_0$-値単関数の列の各点収束極限である.
\end{theorem}

\begin{corollary}[単関数列は単調増加に取れる]
    $f:S\to X$が強可測ならば,単関数列$(f_n)$が存在して,$\norm{f_n(x)}\le\norm{f(x)}$かつ$\forall_{x\in X}\;f_n(x)\to f(x)$.
\end{corollary}

\begin{corollary}
    $f:S\to X$について,ある閉部分空間$X_0\subset X$について$\Im f\subset X_0$が成り立つとする.
    このとき,次の2条件は同値.
    \begin{enumerate}
        \item $f:S\to X$は強可測.
        \item $f:S\to X_0$は強可測.
    \end{enumerate}
\end{corollary}

\begin{corollary}
    強可測関数の列の各点収束極限は強可測である.
\end{corollary}

\begin{corollary}
    関数$f:S\to X$について,次の2条件は同値.
    \begin{enumerate}
        \item $f$は強可測.
        \item $f$は可分値で,可測.
    \end{enumerate}
\end{corollary}

\subsubsection{可測性の保存}

\begin{lemma}
    $E$を可分距離空間,$F$を距離空間とする.
    可測関数$f:E\to F$の像は可分である.
\end{lemma}

\begin{corollary}
    関数$f:S\to X$は強可測で,$\phi:X\to Y$は可測とする.このとき,$\phi\circ f:S\to Y$も強可測.
\end{corollary}

\subsection{測度空間上の関数}

\begin{tcolorbox}[colframe=ForestGreen, colback=ForestGreen!10!white,breakable,colbacktitle=ForestGreen!40!white,coltitle=black,fonttitle=\bfseries\sffamily,
title=]
    Banach空間値関数については,可測性の概念が3つに分かれるだけでなく,積分の定義に際しても新たな注意が必要になる.
    それは,面積確定でない可測集合$A_n$上に値を持つ単関数を省く必要がある.
    Lebesgue積分では高々$\infty,-\infty$しか出てこないため,非負単関数を別に考えれば良かったが,今回は$\infty x+\infty y\;(x,y\in X)$は全く定義不可能な演算になる.
    こうして,$\sigma$-有限性がさらに肝要になる.

    定義域に測度$\mu$が定まっているとき,これを用いて定義を強めることが出来る.
    これが可積分性の概念に繋がる.ここでは,$\mu$-可測性なる概念として定義する.
    この方がむしろ見通しが良い.
\end{tcolorbox}

\begin{notation}
    測度空間$(S,\A,\mu)$を考える.
\end{notation}

\begin{definition}[strongly $\mu$-measurable]\mbox{}
    \begin{enumerate}
        \item $X$値の$\mu$-単関数とは,$x_n\in X$と$\mu(A_n)<\infty$を満たす$A_n\in\A$を用いて$f=\sum_{n=1}^N1_{A_n}\otimes x_n$と表せるものをいう.
        \item $f:S\to X$が\textbf{$\mu$-強可測}であるとは,$f$に$\mu$-概収束する$\mu$-単関数列が存在することをいう.$\mu$-強可測関数全体の集合の同値類がなす線型空間を$L^0(S;X)$で表す.
        \item $f:S\to X$が\textbf{$\mu$-本質的に可分値}であるとは,可分な閉部分空間$X_0\subset X$が存在して,$f(s)\in X_0\;\mu\dae$が成り立つことをいう.
        \item $f:S\to X$が\textbf{$\mu$-弱可測}であるとは,任意の$g\in X^*$について,$g\circ f:S\to\R$が$\mu$-可測であることをいう.
    \end{enumerate}
\end{definition}

\begin{example}
    定数関数$1$は常に強可測であるが,これが$\mu$-強可測でもあることは$\mu$が$\sigma$-有限であることに同値.
\end{example}

\begin{proposition}[強可測関数は$\sigma$-有限な空間に本質的な台を持つ]
    $f:S\to X$は$\mu$-強可測であるとする.このとき,可測な分割$S=S_0\sqcup S_1\;(S_0,S_1\in\A)$が存在して,次の2条件を満たす:
    \begin{enumerate}
        \item $f=0\;\mu\dae\;\on S_0$.
        \item $\mu$は$S_1$上$\sigma$-有限.
    \end{enumerate}
\end{proposition}

\begin{proposition}[$\mu$-強可測性と一般の強可測性]
    関数$f:S\to X$について,
    \begin{enumerate}
        \item $f$が$\mu$-強可測ならば,$f$はある強可測関数に殆ど至る所等しい.
        \item $\mu$が$\sigma$-有限で,$f$が殆ど至る所ある強可測関数に等しいならば,$f$は$\mu$-強可測である.
    \end{enumerate}
\end{proposition}

\subsection{Bochner積分}

\begin{tcolorbox}[colframe=ForestGreen, colback=ForestGreen!10!white,breakable,colbacktitle=ForestGreen!40!white,coltitle=black,fonttitle=\bfseries\sffamily,
    title=]
    Banach空間に測度を導入して積分を考えたい.
    Lebesgue積分に相当する概念がBochner積分である.
    $X$-値関数のLebesgue関数$L^p(S;X)$をBochner空間という.
    一方で,汎関数$\brac{f,x^*}$の通常の意味でのLebesgue積分と可換になるべき連続な線型汎関数はただ一つに定まる.これがPettis積分である.

    Bochner積分は代数的場の量子論で頻繁に用いられる.
    優収束定理は引き続き成り立つが,Radon-Nikodymnoの定理は成り立たなくなる.
\end{tcolorbox}

\subsubsection{定義と特徴付け}

\begin{notation}
    $(S,\A,\mu)$を可測空間とする.
    $\mu$-単関数$f=\sum^N_{n=1}1_{A_n}\otimes x_n$について,
    \[\int_Sfd\mu:=\sum^N_{n=1}\mu(A_n)x_n\]
    とする.
\end{notation}

\begin{definition}[Bochner integral]
    $\mu$-強可測関数$f:S\to X$が\textbf{$\mu$についてBochner積分可能}であるとは,$\mu$-単関数列$(f_n)$が存在して次を満たすことをいう:
    \[\lim_{n\to\infty}\int_S\norm{f-f_n}d\mu=0.\]
    実際,この条件が成り立つとき,$\paren{\int_Sf_nd\mu}$は$\K$上のCauchy列をなす.この極限を$\int_Sfd\mu$とする.
\end{definition}

\begin{lemma}
    $f:S\to X$は$\mu$-強可測とする.
    \begin{enumerate}
        \item $f$がBochner可積分であることと,$\int_S\norm{f}d\mu<\infty$とは同値.
        \item $f$がBochner可積分であるとき,$\Norm{\int_Sfd\mu}\le\int_S\norm{f}d\mu(<\infty)$.
    \end{enumerate}
\end{lemma}

\subsubsection{作用素との可換性}

\begin{tcolorbox}[colframe=ForestGreen, colback=ForestGreen!10!white,breakable,colbacktitle=ForestGreen!40!white,coltitle=black,fonttitle=\bfseries\sffamily,
title=]
    Bochner積分により,線型汎関数$L^1(S,X)\to\K$が定まった.
    これと,作用素$T:L^1(S,X)\to L^2(S,Y)$との相互関係を考える.
\end{tcolorbox}

\begin{discussion}
    $f:S\to X$がBochner可積で,$T:X\to Y$が有界線型作用素であるとき,$Tf:S\to Y$もBochner可積で
    \[T\int_Sfd\mu=\int_STfd\mu.\]
    特に,$Y=\K$である場合について,
    \[\forall_{x^*\in E^*}\quad\Brac{\int_Sfd\mu,x^*}=\int_S\brac{f,x^*}d\mu.\]
    この結果を一般化する.
\end{discussion}

\begin{theorem}[Hille]
    $f:S\to X$をBochner可積で,$T:D(T)\to Y$を部分空間上の閉線型作用素とする.また,$f$は殆ど至る所$D(T)$に値を取り,殆ど至る所で定義された関数$Tf:S\to Y$もBochner可積であるとする.このとき,
    \begin{enumerate}
        \item $f:S\to D(T)$はBochner可積である.
        \item $\int_Sfd\mu\in D(T)$.
        \item $T\int_Sfd\mu=\int_STfd\mu$.
    \end{enumerate}
\end{theorem}

\subsection{Bochner空間}

\begin{tcolorbox}[colframe=ForestGreen, colback=ForestGreen!10!white,breakable,colbacktitle=ForestGreen!40!white,coltitle=black,fonttitle=\bfseries\sffamily,
title=]
    こうして,Lebesgue空間とは,スカラー値関数についてのBochner空間として相対化された.
    このとき,$\mu$-強可測性は,$\mu$-可測性に退化する.
\end{tcolorbox}

\begin{definition}\mbox{}
    \begin{enumerate}
        \item $L^p(S;X)\;(1\le p<\infty)$を,$\int_S\norm{f}^pd\mu<\infty$を満たす$\mu$-強可測関数$f:S\to X$の同値類がなす線型空間とする.
        \item $L^\infty(S;X)$を,$\exists_{r\ge0}\;\mu\Brace{\norm{f}>r}=0$を満たす$\mu$-強可測関数$f:S\to X$全体がなす線型空間とする.
    \end{enumerate}
    これらは$p$-ノルムについてBanach空間をなす.scaler-valued case $L^p(S):=L^p(S;\K)$をLebesgue空間という.
    また,$\sigma$-部分代数$\F$について,可測空間$(S,\F,\mu|_\F)$上のBochner空間を$L^p(S,\F;X)$と表すと,これは$L^p(S;X)$の$\F$上$\mu$-強可測な関数のなす部分空間である.
    $L^p(S,\F):=L^p(S,\F;\K)$と表す.
\end{definition}

\begin{definition}[convergence in measure]
    関数列$(f_n)$が\textbf{測度収束}するとは,$\forall_{r>0}\;\forall_{A\in\A}\;\mu(A)<\infty\Rightarrow\lim_{n\to\infty}\mu(A\cap\Brace{\abs{f_n-f}>r})=0$.
\end{definition}

\begin{lemma}
    $p\in[1,\infty]$とする.
    \begin{enumerate}
        \item $\mu$-単関数は$L^p(S;X)\;(1\le p<\infty)$上稠密である.特に,$L^p(S)\otimes X$は稠密である.
        \item $\mu$-単関数は$L^\infty(S;X)$上,測度収束の位相について稠密である.すなわち,任意の$f\in L^\infty(S;X)$と測度確定な$A\in\A$について,任意の$\ep>0$に対して$\mu$-単関数$g:S\to X$と$\mu(A\setminus A')<\ep$を満たす可測集合$A'\subset A$が存在して,$\norm{g}_\infty\le\norm{f}_\infty$かつ$\sup_{s\in A'}\norm{f(s)-g(s)}<\ep$を満たす.
    \end{enumerate}
\end{lemma}

\subsection{Pettis積分}\label{subsection-Pettis-integral}

\begin{tcolorbox}[colframe=ForestGreen, colback=ForestGreen!10!white,breakable,colbacktitle=ForestGreen!40!white,coltitle=black,fonttitle=\bfseries\sffamily,
title=]
    Pettis積分は,双対ペアを介した弱位相のアイデアを,積分の定義に用いたものである.
    導入したのはGelfandらしい.
\end{tcolorbox}

\subsubsection{弱積分の定義}

\begin{notation}
    $(S,\A,\mu)$を可測空間,$Y\subset X^*$を部分空間とする.
    関数$f:S\to X$は任意の$x^*\in Y$に対して$\brac{f,x^*}$が可積分だとする.
    この条件を,$Y=X^*$のとき\textbf{弱可積分}という.
    このとき,$T_f:Y\to L^1(S)$を$T_fx^*:=\brac{f,x^*}\;(x^*\in Y)$と定めると,これは有界線型作用素である.
    この随伴$T_f^*:L^\infty(S)\subset(L^1(S))^*\to Y^*$が鍵となる.なお,$\mu$が$\sigma$-有限のとき,$(L^1(S))^*=L^\infty(S)$.
\end{notation}

\begin{definition}[Pettis integrable]
    任意の$A\in\A$に対して,
    \[\tau(X,Y)\text{-}\int_Afd\mu:=T_f^*1_A\quad\in Y^*\]
    とする.これを,$f$の$A$上の\textbf{$\tau(X,Y)$-積分}と呼ぶ.
    $\tau(X,X^*)$-積分を弱積分という.さらにこの$Y=X^*$のとき,
    この$T_f^*$の像が$X(\subset X^{**})$に含まれるとき,$\mu$-弱可積分関数は\textbf{Pettis可積分}であるという.
\end{definition}

\begin{lemma}[弱積分の特徴付け]
    $\tau(X,Y)$-積分は,次の条件を満たすただ一つの$Y^*$の元である:
    \[\forall_{x^*\in Y}\quad\Brac{x^*,\tau(X,Y)\text{-}\int_Afd\mu}=\int_A\brac{f,x^*}d\mu.\]
\end{lemma}

\begin{proposition}
    弱積分可能な関数$f:S\to X$について,次の2条件は同値.
    \begin{enumerate}
        \item $f$は$\mu$についてPettis可積分.
        \item $\forall_{A\in\A}\;\exists_{x_A\in X}\;\forall_{x^*\in X^*}\;\brac{x_A,x^*}=\int_A\brac{f,x^*}d\mu$.
    \end{enumerate}
    この同値な条件が成り立つとき,$x_A=T^*_f1_A=:(P)\text{-}\int_Afd\mu$と表し,$f$の$A$上の\textbf{Pettis積分}という.
\end{proposition}
\begin{remarks}
    Bochner可積な関数はPettis可積で,各集合$A\in\A$上でその値は一致する.
\end{remarks}

\subsubsection{Pettis可積性の特徴付け}

\begin{tcolorbox}[colframe=ForestGreen, colback=ForestGreen!10!white,breakable,colbacktitle=ForestGreen!40!white,coltitle=black,fonttitle=\bfseries\sffamily,
title=]
    $\mu$-強可測かつ弱可積分ならば,ほぼPettis可積分であり,例外は$c_0$の非回帰性が引き起こすタイプのもののみである.
\end{tcolorbox}

\begin{theorem}[Pettis可積性の十分条件]
    $1<p\le\infty,1\le q<\infty$を共役指数とする.
    $\mu$-強可測関数$f:S\to X$が$\forall_{x^*\in X^*}\;\brac{f,x^*}\in L^p(S)$を満たすならば,任意の$\phi\in L^q(S)$に対して,関数$\phi\otimes f:S\to\K;s\mapsto \phi(s)f(s)$はPettis可積分である.
\end{theorem}

\begin{corollary}
    $(S,\A,\mu)$を有限測度空間とし,$p>1$とする.$f:S\to X$が強可測で$\forall_{x^*\in X^*}\;\brac{f,x^*}\in L^p(S)$が成り立つならば,$f$はPettis可積である.
\end{corollary}

\begin{example}[弱可積であるがPettis可積分でない例]
    $\{A_n\}\subset P((0,1))$を,互いに素な区間の列で,正の測度$\abs{A_n}>0$をもつものとする.
    関数$f:(0,1)\to c_0$を,$c_0$の標準Schauder基底$(e_n)$について
    \[f:=\sum_{n\in\N}\frac{1_{A_n}\otimes e_n}{\abs{A_n}}\]
    で定める.このとき,$f$は明らかに強可測で,弱可積分でもある.実際,任意の絶対収束列$a=(a_n)\in l^1=(c_0)^*$について,
    \[\int^1_0\brac{f(x),a}dx=\sum_{n\in\N}\abs{A_n}\frac{a_n}{\abs{A_n}}=\sum_{n\in\N}a_n<\infty\]
    より,$\brac{f(-),a}\in L^1(S;\R)$.
    ここで,$x_{**}:=T_f^*1_{(0,1)}$を$f$の$(0,1)$上の弱積分とする.
    すると,
    \[\forall_{n\in\N}\quad\brac{e_n^*,x^{**}}=\int^1_0\brac{f(x),e_n^*}dx=\int^1_0\sum_{m\in\N}\frac{\brac{1_{A_m}(x)e_m(x),e_n^*}}{\abs{A_n}}dx=1\]
    より,Fourier係数の一意性から$x^{**}=1\in l^\infty$である.しかし,これは有界列であっても,$0$へ収束はしない:$x^{**}\notin c_0$.よってPettis可積分ではない.
\end{example}
\begin{remarks}
    弱積分$\tau(X,Y)\td\int\in X^{**}$が$X$に入っていない例を作れ良いので,$c_0\subsetneq(c_\infty\subsetneq)(c_0)^{**}=l^\infty$を用いた.
\end{remarks}

\begin{proposition}
    $(S,\A,\mu)$を測度空間,$X$は$c_0$と同型な閉部分空間を持たないとする.このとき,$f:S\to X$が$\mu$-強可測で弱可積ならば,$f$はPettis可測である.
\end{proposition}
\begin{Proof}
    \ref{thm-Bessaga-and-Pelczynski}による.
\end{Proof}

\subsubsection{Analysis Now}

\begin{tcolorbox}[colframe=ForestGreen, colback=ForestGreen!10!white,breakable,colbacktitle=ForestGreen!40!white,coltitle=black,fonttitle=\bfseries\sffamily,
    title=Pettis積分]
    局所コンパクトハウスドルフ空間$X$からBanach空間$\fX$への関数$f:X\to\fX$を考える.
    $X$上のRadon積分$\int:C_c(X)\to\R$について,積分$\int f\in\fX$にあたる元を考えたい.
    $\R^n$-値積分が成分毎であるのと同様に,全ての連続な線型形式(射影みたいなもの)について,可換性
    \[\forall_{\varphi\in\fX^*}\;\Bracket{\int f,\varphi}=\int\brac{f(-),\varphi}\]
    を満たしてほしい.
    実はこの性質だけで積分は特徴付けられる,ということを議論する.
    1つ目の命題は双対ペア$(X^*,X)$についてで,2つ目は$(X,X^*)$についてである.
    ただし,左辺がこれから定義しようとしている$\fX$上の積分,右辺が$X$上のRadon積分となる.
\end{tcolorbox}

\begin{remark}[integrable norm]
    $\norm{f(-)}:X\to\R$は$X$上の可積分関数を定めるとする.このノルムを可積分ノルムと呼ぶ.
\end{remark}

\begin{lemma}
    関数$f:X\to\fX$に対して,$\fY_f:=\Brace{\varphi\in\fX^*\mid\brac{f(\cdot),\varphi}:X\to\fX\to\R は可測}$と定める.
    このとき,
    \begin{enumerate}
        \item $\fY_f$は$w^*$-閉,特にノルム閉な$\fX^*$の部分空間である.
        \item $\fY_f$が$\fX$の点を分離するならば,高々1つの元$\int f\in\fX$が存在して,
        \[\forall_{\varphi\in\fY_f}\;\Bracket{\int f,\varphi}=\int\brac{f(-),\varphi}\]
        を満たす.
    \end{enumerate}
\end{lemma}
\begin{Proof}\mbox{}
    \begin{enumerate}
        \item $\fY_f$の列$(\varphi_n)$であって,$\varphi\in X^*$に$w^*$-収束するものを任意に取る.
        すると,任意の$x\in X$について,$\fX^*\ni\varphi_n\mapsto\brac{f(x),\varphi_n}\in\bF$は連続だから,
        $\{\brac{f(-),\varphi_n}\}\subset\Meas(X,\bF)$は$\brac{f(-),\varphi}$に各点収束する.
        このとき,極限である$\brac{f(-),\varphi}$は可測.
        よって,$\varphi\in\fY_f$.
        したがって,$\fY_f$は$w^*$-閉で,特にノルム閉である.
        線型部分空間であることは明らか.
        \item 作用素ノルムの定義から$\abs{\brac{f(-),\varphi}}\le\norm{f(-)}\norm{\varphi}$であり,右辺の関数は仮定より可積分であるから,$\brac{f(-),\varphi}:X\to\bF$も可積分である.
        すると,
        $\fY_f=\Brace{\varphi\in X^*\mid\brac{f(-),\varphi}\in\L^1(X)}$とも表せる.
        分離的な部分空間$\fY_f$は$w^*$-稠密である\ref{cor-separating-subspace-is-dense}から,
        $X^*$上への一意的な延長を持つ\ref{prop-extension-of-operator-on-dense-subset}.
        よって一意に定まる.
    \end{enumerate}
\end{Proof}

\begin{proposition}
    $\fY$をBanach空間で,$f:X\to\fY^*$は$w^*$-可測で,ノルム$\norm{f(-)}$は可積分であるとする.
    このとき,ただ一つの元$\int f\in\fY^*$が存在して,
    \[\forall_{y\in\fY}\;\Bracket{y,\int f}=\int\brac{y,f(-)}\]
    を満たす.
\end{proposition}
\begin{Proof}
    $\fX:=\fY^*$とすると,$f$は$w^*$-可測で,任意の$y\in\fY$に対して$\brac{y,-}:\fY^*\to\bF$も$w^*$-連続より$w^*$-可測だから,$\brac{y,f(-)}:X\to\bF$も可測.よって,$\fY\subset\fY_f$.
    したがって補題より,任意の$y\in\fY$について$\brac{f(-),y}\in\L^1(X)$だから,$\int\brac{y,f(-)}\in\bF$.
    この対応$\fY\to\R;y\mapsto\int\brac{y,f(-)}$は明らかに$\fY$上で線型である.この対応を$\int f$と表すと,一意性は補題より従うから,連続性すなわち$\int f\in\fY^*$を示せば良い.

    \[\Abs{\Bracket{y,\int f}}=\Abs{\int\brac{y,f(-)}}=\norm{y}\int\Bracket{\frac{y}{\norm{y}},f(-)}\le\norm{y}\int\norm{f(-)}\]
    より,線型汎函数$\int f$は有界である.よって,$\int f\in\fY^*$.
\end{Proof}

\begin{proposition}
    $f:X\to\fX$を弱可測関数で,$\fX$を可分とする.$\norm{f(-)}\in\L^1(X)$とする.
    このとき,ただ一つの$\int f\in\fX$が存在して,
    \[\forall_{\varphi\in\fX^*}\;\Bracket{\int f,\varphi}=\int\brac{f(-),\varphi}\]
    を満たす.
\end{proposition}
\begin{Proof}\mbox{}
    \begin{description}
        \item[方針] $f:X\to\fX$は弱可測としたから,$\brac{-,\varphi}:X\to\bF$は弱連続より弱可測であることと併せると,$\fY_f=\fX^*$.
        よって,$f:X\to\fX\mono\fX^{**}$も弱可測で特に$w^*$-可測だから,命題より,$\int f\in\fX^{**}$が存在して,
        \[\forall_{\varphi\in\fX^*}\;\Bracket{\int f,\varphi}=\int\brac{f(-),\varphi}\]
        を満たす.あとはこれが$\int f\in\fX$であること,すなわち,$\int f:\fX^*\to\bF$が$w^*$-連続であることを示せば良い.
        \item[証明] 
        $\fX$は可分としたから,$\fX^*$の単位閉球$B^*$は$w^*$-位相についても第2可算であり,距離化可能\ref{thm-metrizability-of-ball}.
        よって,$B^*$の列$(\varphi_n)$が$\varphi$に$w^*$-収束する,すなわち,$\L^1(X)$の列$(\brac{f(-),\varphi_n})$が$\brac{f(-),\varphi}$に各点収束する($\forall_{x\in\fX}\;\brac{f(x),-}:\fX^*\to\bF$は$w^*$-連続)と仮定して,$\paren{\int\brac{f(-),\varphi_n}}$も$\bF$上で$\int\brac{f(-),\varphi}$に収束することを示せば良い\ref{cor-characterization-of-weak-star-continuousness}.
        が,これは$\norm{f(-)}$という可積分な非負値の優関数が見つかるから,Lebesgueの優収束定理より従う.
    \end{description}
\end{Proof}

\chapter{Hilbert空間}

\begin{quotation}
    完備な距離を内積が誘導する場合をHilbert空間という.
    これは,標準的なペアリングが存在するBanach空間論と見れる.
    些細な違いかもしれないが,理論がフルパワーを発するのはこの場合のみである.
    特に,Banach空間上の作用素にはほとんど一般論が成り立たない.
    例えばHilbert空間には,標準的なペアリングを通じてRieszの表現が存在する.

    Banach空間上に,内積$L^p(X)\times L^q(X)\ni(f,g)\mapsto\brac{f,g}:=\int f\o{g}$によって,
    $(L^p(X))^*\simeq L^q(X)$なる同型が誘導される\ref{thm-duality-of-Lp}.
    2次のノルムがHilbert空間として重要である理由は,自身と共役だからである.
\end{quotation}

\section{内積が拓く描像}

\begin{tcolorbox}[colframe=ForestGreen, colback=ForestGreen!10!white,breakable,colbacktitle=ForestGreen!40!white,coltitle=black,fonttitle=\bfseries\sffamily,
title=]
    Hilbert空間には内積という名前の標準的なペアリングが存在する.
    これを通じてRieszの表現が存在し,また直交性の概念が定義される.
    これは基底論にも影響を与え,Hilbert空間の分類は有限次元線型空間論の延長に落ちる.
\end{tcolorbox}

\begin{history}
    はじめは,Fredholmの積分方程式論,Fourier級数・積分論を綜合して,固有値問題を一般的に扱うために生まれた理論であり,
    von Neumannによる量子力学の基礎付けの研究を通じて整えられた.
\end{history}

\subsection{半双線型形式とセミノルムの関係}

\begin{tcolorbox}[colframe=ForestGreen, colback=ForestGreen!10!white,breakable,colbacktitle=ForestGreen!40!white,coltitle=black,fonttitle=\bfseries\sffamily,
title=]
    Hilbert空間とは,Banach空間であって,ノルムが中線定理
    \[\norm{x+y}^2+\norm{x-y}^2=2(\norm{x}^2+\norm{y}^2)\]
    を満たすような空間である.
    中線定理が成り立つとき,
    \[4(x|y)=\sum^3_{k=0}i^k\norm{x+i^ky}^2=\norm{x+y}+i\norm{x+iy}-\norm{x-y}-i\norm{x-iy}\]
    に従えばこれは内積を定める.このような,ノルムと内積の関係を\textbf{極化恒等式}という.
    一方で双線型形式は二次形式と対応し,これも極化恒等式と呼ばれる.
\end{tcolorbox}

\begin{definition}[sesquilinear form, adjoint form, self-adjoint, positive, semi-inner product, inner product]
    $\bF$-線型空間$X$について,
    \begin{enumerate}
        \item 写像$(-|-):X\times X\to\bF$が\textbf{半双線型形式}であるとは,第一引数について線型で,第二引数について共役線型であることをいう.\footnote{数理物理では逆.}$\bF=\R$のときは双線型性に同値.
        \item 半双線型形式$(-|-)$の\textbf{随伴形式}$(-|-)^*$とは,$(x|y)^*:=\o{(y|x)}$で定まる半双線型形式をいう.
        \item $(-|-)^*=(-|-)$を満たす半双線型形式を\textbf{自己共役}という.$\bF=\R$であるときは対称ともいう.
        \item $\forall_{x\in X}\;(x|x)\ge 0$を満たすとき,$(-|-)$を\textbf{半正定値}という.
        \item 半正定値な自己共役半双線型形式を,$X$上の\textbf{半内積}という.
        \item $(x|x)=0\Rightarrow x=0$を満たす半内積を,$X$上の\textbf{内積}という.
    \end{enumerate}
\end{definition}

\begin{lemma}[自己共役性の特徴付け]\label{lemma-characterization-of-self-adjointness}
    $\C$-線型空間上の半双線型形式について,
    \begin{enumerate}
        \item $4(x|y)=\sum^3_{k=0}i^k(x+i^ky|x+i^ky)$.
        \item $(-|-)$が自己共役であることは,$\forall_{x\in X}\;(x|x)\in\R$に同値.特に,$(-|-)$が正定値ならば自己共役である.\footnote{実数値の正定値双線型形式が対称とは限らない.}
    \end{enumerate}
\end{lemma}
\begin{Proof}\mbox{}
    \begin{enumerate}
        \item \begin{align*}
            &(x+y|x+y)+i(x+iy|x+iy)+(-1)(x-y|x-y)+(-i)(x-iy|x-iy)\\
            =&(x+y|x+y)+(x+iy|y-ix)+(x-y|y-x)+(x-iy|y+ix)=4(x|y).
        \end{align*}
        \item $(-|-)$が自己共役とすると,$y=0$と任意の$x\in X$について,(1)を用いて展開すると
        \begin{align*}
            (x|0)&=(x|x)+i(x|x)+(-1)(x|x)+(-i)(x|x)\\
            \o{(0,x)}&=\o{(x|x)}+(-i)\o{(x|x)}+(-1)\o{(x|x)}+i\o{(x|x)}
        \end{align*}
        と展開できるから,$(x|x)\in\R$である.
        一方で$(x|x)\in\R$を仮定すると,(1)を用いて,$(x|x)\in\R$に注意して$a+ib\;(a,b\in\R)$の形に整理すると,
        \begin{align*}
            (x|y)&=(x+y|x+y)+i(x+iy|x+iy)+(-1)(x-y|x-y)+i(-x+iy|x-iy)\\
            &=(2y|2x)+i\underbrace{(2iy|2x)}_{\in\R}\\
            (y|x)&=(x+y|x+y)+i(y+ix|y+ix)+(-1)(y-x|y-x)+i(y-ix|ix-y)\\
            &=(2y|2x)+i\underbrace{(2y|2ix)}_{\in\R}
        \end{align*}
        となるが,それぞれの虚部について$(2iy|2x)=i(2y|ix),(2y|2ix)=-i(2y|2x)$だから,たしかに$(x|y)=\o{(y|x)}$を満たす.
    \end{enumerate}
\end{Proof}
\begin{remarks}
    一変数じゃないが,鏡像の原理を奥に感じる.
\end{remarks}

\begin{proposition}[polarization identity, parallellogram law]\label{prop-polarization-identity}
    半内積$(-|-):X\times X\to\bF$について,
    \begin{enumerate}
        \item 関数$\norm{-}:X\to\R_+$を$\norm{x}:=(x|x)^{1/2}$で定めると,これは斉次関数である.
        \item 次の\textbf{極化恒等式}が成り立つ:
        \begin{enumerate}[(a)]
            \item $\bF=\C$のとき,$4(x|y)=\sum^3_{k=0}i^k\norm{x+i^ky}^2$.
            \item $\bF=\R$のとき,$4(x|y)=\norm{x+y}^2-\norm{x-y}^2$.
            \item $\bF=\C$のとき,実部と虚部に分けて$\norm{x+y}^2=\norm{x}^2+2\Re\brac{x,y}+\norm{y}^2$も極化恒等式と呼ぶ.
        \end{enumerate}
        \item (Cauchy-Bunyakowsky-Schwarz) $\abs{(x|y)}\le\norm{x}\norm{y}$.
        特に,$\norm{-}:X\to\R_+$は劣加法的であり,$X$上のセミノルムを定める.
        $(-|-)$が内積であるとき,$\norm{-}$はノルムを定める.
        \item セミノルム$\norm{-}$について中線定理が成り立つ:$\norm{x+y}^2+\norm{x-y}^2=2(\norm{x}^2+\norm{y}^2)$.
        \item (Frechet-von Neumann-Jordan) ノルム$\norm{-}$が中線定理を満たすとき,(2)の極化恒等式によって定まる半双線型形式は内積を定める.
    \end{enumerate}
\end{proposition}
\begin{Proof}\mbox{}
    \begin{enumerate}
        \item 半内積$(-|-):X\times X\to\bF$は特に半正定値だから,特に$\forall_{x\in X}\;(x|x)\ge 0$.よって確かに$\norm{-}:X\to\R_+$はwell-defined.
        $\forall_{a\in\bF}\;\norm{ax}=(\abs{a}^2(x|x))^{1/2}=\abs{a}\norm{x}$.
        \item (a)は補題(1)より.(b)は
        \begin{align*}
            (x+y|x+y)-(x-y|x-y)&=(x|x)+(x|y)+(y|x)+(y|y)-((x|x)-(y|x)-(x|y)+(y|y))=4(x|y)
        \end{align*}
        より.(c)は
        \begin{align*}
            \norm{x+y}^2&=\norm{x}^2+(y|x)+(x|y)+\norm{y}^2=\norm{x}^2+2\Re(x|y)+\norm{y}^2.
        \end{align*}
        \item 
        \begin{description}
            \item[Cauchy-Schwarz] 
        (2)(c)と同様にして,
        \[\forall_{\al\in\bF}\;\abs{\al}^2\norm{x}^2+2\Re\al(x|y)+\norm{y}^2=\norm{\al x+y}^2\ge 0\]
        を得る.これを$\al$の方程式と見ると,実数解は高々1つであるから,判別式は非正でなくてはならない.よって,
        \[(\Re(x|y))^2-\norm{x}^2\norm{y}^2\le 0\]
        であるが,ここで$(x|y)=be^{i\theta}\;(b\ge 0)$とおいたとき,$\al:=te^{-i\theta}\;(t\ge0)$を考えることで,
        \[\norm{x}^2t^2+2bt+\norm{y}^2\ge0\]
        を得る.特に,左辺を実数$t$についての二次方程式と見たときに解は高々1つだから,$b^2-\norm{x}^2\norm{y}^2\le 0$が必要.ここから$\abs{(x|y)}\le\norm{x}^2\norm{y}^2$を得る.
            \item[劣加法性] Cauchy-Schwarzの不等式より$\Re(x|y)\le\abs{(x|y)}\le\norm{x}\norm{y}$と(2)(c)を併せると,$\norm{x+y}^2\le(\norm{x}+\norm{y})^2$を得る.
            \item[斉次性] (1)で示した.
            \item[分離] $\norm{x}=0$とすると,$\norm{x}^2=(x|x)=0$.$(-|-)$が内積であるとき,これは$x=0$を導く.
        \end{description}
        \item 省略.
        \item 
    \end{enumerate}
\end{Proof}
\begin{remarks}
    本質的には次の4式のみである:
    \begin{align*}
        (x+y|x+y)&=(x|x)+(y|y)+2\Re(x|y),&(x-y|x-y)&=(x|x)+(y|y)-2\Re(x|y),\\
        (x-y|x+y)&=(x|x)-(y|y)+2\Im(x|y)i,&(x+y|x-y)&=(x|x)-(y|y)-2\Im(x|y)i,\\
        (x+iy|x+iy)&=(x|x)+(y|y)+2\Im(x|y),&(x-iy|x-iy)&=(x|x)+(y|y)-2\Im(x|y).
    \end{align*}
    第2行も同じくらい示唆的ではあるが,ノルムは作り出せないのであろう.
\end{remarks}

\begin{definition}[orthogonal]
    $(-|-):X\times X\to\bF$を半双線型形式とする.
    \begin{enumerate}
        \item ベクトル$x,y$が直交する$x\perp y$とは,$(x|y)=0$を満たすことをいう.
        \item 部分集合$Y,Z\subset X$が直交する$Y\perp Z$とは,$\forall_{y\in Y,z\in Z}\;y\perp z$を満たすことをいう.
        \item 部分集合$X\subset H$について,$X^\perp:=\Brace{x^\perp\in H\mid x^\perp\perp X}$と表すと,$X^\perp$は閉部分空間である.
    \end{enumerate}
\end{definition}

\begin{lemma}[Pythagoras identity]\label{lemma-Pythagoras-identity}
    $X$が実線型空間であるとき,
    $x,y\in X$について,次の2条件は同値.
    \begin{enumerate}
        \item $x\perp y=0$.
        \item $\norm{x+y}^2=\norm{x}^2+\norm{y}^2$.
    \end{enumerate}
\end{lemma}
\begin{Proof}
    (1)$\Rightarrow$(2)は極化恒等式(c)より.
    (2)$\Rightarrow$(1)は,極化恒等式(c)より$\Re(x|y)=0$と,$(x|iy)=0$より$\Im(x|y)=0$を,別々に得る.
\end{Proof}

\subsection{直交分解}

\begin{tcolorbox}[colframe=ForestGreen, colback=ForestGreen!10!white,breakable,colbacktitle=ForestGreen!40!white,coltitle=black,fonttitle=\bfseries\sffamily,
title=]
    直交分解を距離の言葉によって議論するところは,商ノルムの定義と同じ作戦である.
\end{tcolorbox}

\begin{lemma}
    $C$をHilbert空間$H$の非空な閉凸集合とする.このとき,任意の$y\in H$に対して,距離$d(y,C)$を最小にするときの点$x=\argmin_{x\in C}d(y,x)\in C$が唯一つ存在する.
\end{lemma}
\begin{Proof}
    $C$を$C-y$と取り直すことで,$d(0,C)$を最小にする$x\in C$を考えても,一般性は失われない.
    \begin{description}
        \item[存在] $\al:=\inf\Brace{\norm{x}\ge0\int x\in C}$とおき,$\norm{x_n}\to\al$を満たす$C$の点列$(x_n)$を任意に取る.
        $C$の凸性より$\forall_{y,z\in C}\;(y+z)/2\in C$だから,中線定理\ref{prop-polarization-identity}より,
        \[\forall_{y,z\in C}\;\quad2(\norm{y}^2+\norm{z}^2)=\norm{y+z}^2+\norm{y-z}^2\ge4\al^2+\norm{y-z}^2.\]
        特に$y=x_n,z=x_m$の場合を考えることで,$(x_n)$はCauchy列であることが分かる.よって,$C$は閉集合だから,ある$x\in C$が存在して,$\lim_{n\to\infty}x_n=x,\norm{x}=\al$を満たす.
        \item[一意性] $z\in C$も$\norm{z}=\al$を満たすとする.すると,中線定理より,$4\al^2\ge4\al^2+\norm{x-z}^2$だから,$x=z$を得る.
    \end{description}
\end{Proof}

\begin{theorem}
    任意の閉部分空間$X\subset H$について,
    \begin{enumerate}
        \item 任意の元$y\in H$は一意的な分解$y=x+x^\perp\in X\oplus X^\perp$を持つ.
        また,$H=X\oplus X^\perp$と直交直和で表せる.
        \item この$x\in X$は$y$に一番近い点$\argmin_{x\in X} d(y,x)$であり,$x^\perp\in X^\perp$も$y$に一番近い点$\argmin_{x^\perp\in X^\perp} d(y,x^\perp)$である.
        \item $(X^\perp)^\perp=X$が成り立つ.
    \end{enumerate}
\end{theorem}
\begin{Proof}\mbox{}
    \begin{description}
        \item[(1)] 
        任意に$y\in H$をとり,$x:=\argmin_{x\in X}d(y,X)$とおく.
        \begin{description}
            \item[存在] $x^\perp:=y-x\in X^\perp$を示す.
            \begin{align*}
                \forall_{z\in X}\;\forall_{\ep>0}\quad\norm{x^\perp}^2&=\norm{y-x}^2\le\norm{y-(x+\ep x)}^2&\because xの取り方\\
                &=\norm{x^\perp-\ep z}^2=\norm{x^\perp}^2-2\ep\Re(x^\perp|z)+\ep^2\norm{z}^2&\because 極化恒等式\ref{prop-polarization-identity}
            \end{align*}
            より,$\forall_{z\in X}\;\forall_{\ep>0}\;2\Re(x^\perp|z)\le\ep\norm{z}^2$.
            よって,$\forall_{z\in X}\;\Re(x^\perp|z)\le 0$.$z,-z,iz,-iz$について考えることで,$\forall_{z\in X}\;(x^\perp|z)=0$を得る.
            以上より,対応$\Phi:H\to X\oplus X^\perp;y\mapsto x+x^\perp$が定まった.
            明らかに全射で,Pythagorasの恒等式\ref{lemma-Pythagoras-identity}より,等長写像であることも分かる.あとは単射性を示せば良い.
            \item[一意性] $y=z+z^\perp\;(z\in X,z\in X^\perp)$とも表せたとする.このとき$0=(x-z)+(x^\perp-z^\perp)$であるが,Pythagorasの恒等式\ref{lemma-Pythagoras-identity}より,$0=\norm{x-z}^2+\norm{x^\perp-z^\perp}^2$.
            すなわち,$x=z\land x^\perp=z^\perp$.
            よって,$\Phi:H\iso X\oplus X^\perp$は等長同型である.
        \end{description}
        \item[(3)] $X\perp X^\perp$より$X\subset (X^\perp)^\perp$であるから,$X\supset (X^\perp)^\perp$を示せば良い.
        任意に$y\in(X^\perp)^\perp$をとると,一意的な分解$y=z+z^\perp$を持つ.
        $x^\perp=y-x\in X^\perp\cap (X^\perp)^\perp$より,
        $x^\perp=0$が従い,$X=(X^\perp)^\perp$が分かる.
        \item[(2)] $(X^\perp)^\perp=X$より,(1)の証明を$X$を$X^\perp$として行うことより,$x^\perp=\argmin_{x^\perp\in X^\perp}d(y,x^\perp)$も分かる.
    \end{description}
\end{Proof}

\begin{corollary}[closed linear span]\label{cor-expression-of-closed-linear-span}
    任意の部分集合$X\subset H$について,$X$を含む最小の閉部分空間は$(X^\perp)^\perp$である.
    特に,$X$が$H$の部分空間ならば,$\dbloverline{X}=(X^\perp)^\perp$.
\end{corollary}
\begin{Proof}
    $X^\perp$は閉部分空間であるから,定理より$(X^\perp)^\perp$も閉部分空間である.これが$X$を含む最小の閉部分空間であることを示す.
    $X\subset Y$を$H$の閉部分空間とすると,$Y^\perp\subset X^\perp$で,$(X^\perp)^\perp\subset (Y^\perp)^\perp=Y$が従う.

    $H$の部分空間$X$を含む最小の閉集合が$\dbloverline{X}$で,$\dbloverline{X}$自身も部分空間であるから,これは最初の閉部分空間でもある.
\end{Proof}

\begin{corollary}[部分空間の稠密性の特徴付け]\label{cor-dense-subspace}
    $X$を$H$の部分空間とする.次の2条件は同値.
    \begin{enumerate}
        \item $X$は$H$上稠密である.
        \item $X^\perp=0$.
    \end{enumerate}
\end{corollary}
\begin{Proof}
    部分空間$X$について,$X^\perp$は$H$の閉部分空間だから,$X^\perp\oplus(X^\perp)^\perp=H$.系より$(X^\perp)^\perp=\dbloverline{X}$であるから,$\dbloverline{X}=H$は$X^\perp=0$と同値.
\end{Proof}

\subsection{Rieszの表現定理}

\begin{tcolorbox}[colframe=ForestGreen, colback=ForestGreen!10!white,breakable,colbacktitle=ForestGreen!40!white,coltitle=black,fonttitle=\bfseries\sffamily,
title=]
    Rieszの表現定理は変種が多々あり,いずれもある種の位相線型空間とその双対空間との間に(反)同型を取る知識である.

    なお,
    Metの同型は全射な等距離写像である(全単射な等長写像は,等長な逆を持つ).
    これは擬距離空間やRiemann多様体では成り立たない.
\end{tcolorbox}

\begin{proposition}[Riesz representation theorem]\label{prop-isometry-of-Hilbert-dual}
    写像$\Phi:H\to H^*;x\mapsto(-|x)$は,共役線型な等長同型である.
\end{proposition}
\begin{Proof}\mbox{}
    \begin{description}
        \item[共役線形性] $\Phi(ax)=(-|ax)=\o{a}(-|x)=\o{a}\Phi(x)$.
        \item[等長写像] Cauchy-Schwarzの不等式\ref{prop-polarization-identity}より,
        \[\forall_{x\in H}\;\norm{\Phi(x)}=\sup_{y\in B}\abs{(y|x)}\le\sup_{y\in B}\norm{y}\norm{x}=\norm{x}.\]
        また一方で,
        \[\forall_{x\in H}\;\norm{x}^2=\Phi(x)(x)=\Phi(x)\paren{\norm{x}\frac{x}{\norm{x}}}\le\norm{x}\norm{\Phi(x)}\]
        より$\norm{x}\le\norm{\Phi(x)}$.
        \item[全単射] 等長写像は単射であるから,全射性を示せば良い.
        任意に$\varphi\in H^*\setminus\{0\}$をとって逆像が像が空で無いことを示す.
        $X:=\Ker\varphi$とおくと,$\varphi\ne 0$よりこれは$H$の真の閉部分空間をなすから,ある$x\in X^\perp$が存在して$\varphi(x)=1$を満たす.
        すると,$\forall_{y\in H}\;y-\varphi(y)x\in X$より,
        \[(y|x)=(y-\varphi(y)x+\varphi(y)x|x)=\varphi(y)\norm{y}^2\]
        が任意の$y\in H$について成り立つから,$\Phi^{-1}(\varphi)\ni\frac{x}{\norm{x}^2}$.
    \end{description}
\end{Proof}

\begin{corollary}[$L^2$上の有界線形関数の積分表現]
    $(X,\Om,\mu)$を測度空間とする.任意の有界線型汎函数$F:L^2(\mu)\to\bF$に対して,ただ一つの元$h_0\in L^2(\mu)$が存在して,
    \[\forall_{h\in L^2(\mu)}\quad F(h)=\int h\o{h_0}d\mu\]
    と表せる.
\end{corollary}

\subsection{作用素の弱連続性}

\begin{definition}[weak topology on Hilbert space]\label{def-weak-topology-on-H(B)}
    Hilbert空間$H$上の弱位相とは,有界な線型汎函数の集合$H^*$が定める始位相をいう.
    これは,標準的な同型$\Phi:H\to H^*$ \ref{prop-isometry-of-Hilbert-dual}により,$H^*$上の$w^*$-位相を引き戻したものに一致する.
    よって特に,Alaogluの定理\ref{thm-Alaoglu}より,$H$内の単位球は弱コンパクトである.
\end{definition}

\begin{lemma}\label{lemma-characterization-of-bounded-operator-on-Hilbert-space}
    任意のHilbert空間の作用素$T:H\to H$について,次の2条件は同値.
    \begin{enumerate}
        \item $T\in B(H)$である($H$のノルム位相について連続,すなわち,有界).
        \item $T$は弱位相について連続である(weak-weak continuous).
        \item $T$はノルム-弱連続である(norm-weak continuous).
    \end{enumerate}
\end{lemma}
\begin{Proof}\mbox{}
    \begin{description}
        \item[(1)$\Rightarrow$(2)] 任意の$y\in H$を取れば,随伴作用素\ref{thm-existence-of-adjoint-operator}を考えることにより,$T^*y\in H$が存在して,次の図式は可換.
        \[\xymatrix{
            H\ar[r]^-T\ar[dr]_-{(-|T^*y)}&H\ar[d]^-{(-|y)}\\
            &\bF
        }\]
        $H$上に弱位相を考えるとき,$(-|T^*y),(-|y)$はいずれも連続だから,$T:H\to H$も連続である.
        \item[(2)$\Rightarrow$(1)] 
        グラフ$G(T):=\Brace{(x,y)\in H\times H\mid Tx=y}$がノルム閉集合であることを示せば良い\ref{thm-closed-graph-theorem}.
        任意に,$G(T)$の列$(x_n,Tx_n)$を取り,これは$(x,y)$にノルム収束するとする.
        すると,
        ノルム位相は弱位相よりも強いため,特に弱収束する.
        よって,$T$が弱位相について連続であるとすると,弱位相について$Tx_n\to Tx$であるから,これは$Tx=y$を含意する.すなわち,$(x,y)\in G(T)$.
        \item[(2)$\Rightarrow$(3)] ノルム位相は弱位相よりも強いため.
        \item[(3)$\Rightarrow$(1)] (2)$\Rightarrow$(1)と同様の議論によって示せる.
    \end{description}
\end{Proof}

\begin{proposition}
    weak-norm連続な作用素は,有限なランクを持つ.
\end{proposition}

\subsection{正規直交系}

\begin{tcolorbox}[colframe=ForestGreen, colback=ForestGreen!10!white,breakable,colbacktitle=ForestGreen!40!white,coltitle=black,fonttitle=\bfseries\sffamily,
title=]
    任意の線型空間には,極大な線型独立系が存在するという意味で,Hamel基底を持つ.
    Hilbert空間では,内積から定まる基底が存在し,これは常にHamel基底とは異なる.
    正規直交基底を用いて,Hilbert空間の同型を特徴づけることが出来る.
\end{tcolorbox}

\begin{definition}[orthonormal, complete / orhonormal basis]
    Hilbert空間$H$の部分集合$\{e_j\}_{j\in J}$について,
    \begin{enumerate}
        \item $\{e_j\}_{j\in J}$が\textbf{正規直交}であるとは,$\forall_{j\in J}\;\norm{e_j}=1$かつ$(e_j|e_i)=\delta_{ij}$を満たすことをいう.
        \item $\{e_j\}_{j\in J}$が\textbf{完全}または\textbf{正規直交基底}であるとは,生成する部分空間が$H$上稠密であることをいう:$\dbloverline{\brac{e_j}_{j\in J}}=H$.これは,$\oplus_{j\in J}\bF e_j=H$に同値\ref{def-orthogonal-sum-of-Hilbert-spaces}.すなわち,2-ノルムで収束する表示$x=\sum_{j\in J}\al_j e_j$が存在する.
    \end{enumerate}
\end{definition}
\begin{Proof}
    (3)は,$(x|x)=\sum_{j\in J}\abs{\al_j}^2$であるが,内積$(-|-):H\times H\to\bF$が連続であることより,右辺は収束する.
\end{Proof}

\begin{proposition}[正規直交基底の特徴付け]
    $\{x_i\}_{i\in I}\subset H$を正規直交系であるとする.次は同値:
    \begin{enumerate}
        \item $\{x_i\}$は完全である:$\oo{\brac{x_i}_{i\in I}}=H$.
        \item Fourier展開:$\forall_{x\in H}\;\sum_{i\in I}(x|x_i)x_i$.
        \item Fourier変換の等長性(Parseval):$\forall_{x\in H}\;\norm{x}^2=\sum_{j\in J}\abs{\al_j}^2\;\paren{\al_i:=(x|x_i)}$.
        \item 内積:$\forall_{x,y\in H}\;(x|y)=\sum_{i\in I}(x|x_i)\o{(y|x_i)}$.
        \item 線型独立性:任意の$x\in H$について,$\forall_{i\in I}\;(x|x_i)=0\Rightarrow x=0$.
    \end{enumerate}
\end{proposition}

\begin{proposition}[正規直交系の基底への延長]
    Hilbert空間$H$の任意の正規直交系は,正規直交基底へと拡大できる.
\end{proposition}
\begin{Proof}\mbox{}
    \begin{description}
        \item[方針] $\{e_j\}_{j\in J_0}\subset H$を正規直交系とする.
        $\{e_j\}_{j\in J_0}$を含む$H$の正規直交系全体からなる集合は,包含関係について帰納的順序集合を定めるから,Zornの補題より極大元$\{e_j\}_{j\in J}\;J_0\subset J$が存在する.
        これが生成する閉部分空間を$X$としたとき,$X=H$を示せば良い.
        \item[証明] 
        $X\ne H$のとき,$X^\perp\ne 0$だから,ある$e\in X^\perp$が存在して,$\forall_{j\in J}\;(e_j|e)=0,(e|e)=1$を満たす.
        これは$\{e_j\}_{j\in J}$の極大性に矛盾する.
    \end{description}
\end{Proof}

\begin{corollary}[Gram-Schmidt Orthogonalization Process]
    $\{h_n\}_{n\in\N}$を線型独立な部分集合とする.このとき,正規直交系$\{e_n\}_{n\in\N}$が存在して,$\forall_{n\in\N}\;\brac{e_1,\cdots,e_n}=\brac{h_1,\cdots,h_n}$.
\end{corollary}

\begin{proposition}[Hilbert空間の同型類]\label{prop-characterization-of-isomorphism-of-Hilbert-spaces}
    $H,K$をHilbert空間とし,$\{e_i\}_{i\in I},\{f_j\}_{j\in J}$をそれぞれの正規直交基底とし,$I$と$J$は集合として同型であるとする.
    この時,等長同型$U:H\to K$が存在して,$\forall_{x,y\in H}\;(Ux|Uy)=(x|y)$を満たすものが存在する.
\end{proposition}
\begin{Proof}\mbox{}
    \begin{description}
        \item[稠密部分集合上の作用素] 全単射$\gamma:I\to J$をとる.これを用いて,
        作用素$U_0:\sum_{i\in I}\bF_ie_i\to\sum_{j\in J}\bF_jf_j$を
        $U_0x:=\sum_{i\in I}a_if_{\gamma(i)}$と定めると,これは全射な線型作用素で,Parseval恒等式より等長写像である:$\norm{U_0x}=\sum_{i\in I}\abs{\al_i}=\norm{x}$.
        特にこれは連続写像であるから,一意な延長$U:H\to K$を持つ\ref{prop-extension-of-operator-on-dense-subset}.
        これは再び全射であり,極化恒等式\ref{prop-polarization-identity}より,
        \begin{align*}
            4(Ux|Uy)&=\sum_{k=0}^3i^k\norm{U(x+i^ky)}^2\\
            &=\sum_{k=0}^3i^k\norm{x+i^ky}^2=4(x|y).
        \end{align*}
    \end{description}
\end{Proof}
\begin{remarks}
    無限次元の可分なHilbert空間は,全て可算な正規直交基底を持つため,全て同型である.
    標準的に$l^2$を考えると良い.これはEuclid空間の非常に自然な一般化となっている.
    Hilbert空間論の今後の主な対象は,その上の作用素となり,これが同型類よりもさらに細かい構成を特徴付ける.
\end{remarks}

\subsection{Hilbert空間の同型類のFourier係数の空間による分類}

\begin{tcolorbox}[colframe=ForestGreen, colback=ForestGreen!10!white,breakable,colbacktitle=ForestGreen!40!white,coltitle=black,fonttitle=\bfseries\sffamily,
title=]
    Hilbert空間の,正規直交基底$\{e_\al\}$が定まる係数空間への対応$H\to l^2(A)$をFourier変換といい,これが内積を保つことと$\{e_\al\}$が実際に正規直交基底であることは同値(Parsevalの等式).
    そこで,Hilbert空間の同型類は,係数の空間$l^2(A)$を分類すれば良い.
    これが抽象的なFourier変換論である.
\end{tcolorbox}

\begin{definition}[pre-Hilbert space]
    内積空間$H$が,付随するノルムについてBanach空間をなすとき,これを\textbf{Hilbert空間}という.
    また内積空間を前Hilbert空間ともいう.
\end{definition}

\begin{example}[二乗可積分関数の空間:特殊から一般へ]\mbox{}
    \begin{enumerate}
        \item Euclid空間$\bF^n$は,通常の内積についてHilbert空間をなし,付随するノルムは2-ノルムである.
        \item $l^2(\Z):=\Brace{(a_n)_{n\in\Z}\in\prod_{n\in\Z}\bF\;\middle|\;\sum_{n\in\Z}\abs{a_n}^2<\infty}$は,内積$((a_n)|(b_n))=\sum_{n\in\Z}a_n\o{b_n}$に関してHilbert空間をなす.
        \item コンパクト台を持つ関数の空間$C_c(\R^n)$は,内積$(f|g):=\int f(x)\o{g(x)}dx$についてpre-Hilbert空間をなし,付随するノルムは2-ノルムである.これを完備化したものは$L^2(\R^n)$であった\ref{exp-Banach-spaces}が,これがHilbert空間である.
        \item 一般に,局所コンパクトハウスドルフ空間$X$上のRadon積分$\int$について二乗可積分な関数のなす空間の完備化$L^2(X)$\ref{exp-Banach-space-of-Radon-integrable-functions}は,内積$(f|g)=\int f\o{g}$についてHilbert空間となる.
    \end{enumerate}
\end{example}

\begin{definition}[orthogonal sum / direct sum of Hilbert space]\label{def-orthogonal-sum-of-Hilbert-spaces}
    Hilbert空間の族$(H_j)_{j\in J}$について,
    \begin{enumerate}
        \item 代数的直和$\sum_{i\in J}H_j$には,$(x|y):=\sum_{j\in J}(\pr_jx|\pr_jy)$によって内積が定まる.
        \item この内積に付随するノルムは2-ノルムであり,これについての完備化を\textbf{(直交)直和}といい,$\oplus_{j\in J}H_j$と表す.
        \item 命題\ref{prop-completion-of-algebraic-direct-product}より,集合としては
        \[\bigoplus_{j\in J}=\Brace{x\in\prod_{j\in J}H_j\;\middle|\;\sum_{j\in J}\norm{\pr_j(x)}^2<\infty}\]
        と表せる.特に,Hilbert空間の直和$\oplus H_j$の元$x$は,可算個の$j\in J$を除いて$\pr_j(x)=0$である.
    \end{enumerate}
\end{definition}
\begin{remarks}
    これがHamel基底を超える,待ち望まれた無限次元の直和空間の構成である.
    有限の場合はHamel基底と変わらないが,無限の場合は,無限和が二乗収束するものからなる空間を考えると完備な内積が定まる.
\end{remarks}

\begin{corollary}[数ベクトルへの対応こそがFourier変換]
    $H$をHilbert空間,
    $\{e_n\}_{n\in\N}\subset H$を正規直交な列とする.
    \begin{enumerate}
        \item (Besselの不等式) 係数への対応$\F:H\to l^2(\N);x\mapsto(x|e_n)_{n\in\N}$はノルム減少的で全射な有界作用素である:$\sum_{n\in\N}\abs{(x|e_n)}^2\le\norm{x}$.
        \item (Riesz-Fischerの定理) $H$が可分のとき,$\F:H\to l^2(\N)$は等長でもある.すなわち,Hilbert空間の同型を定める.
        \item (Parsevalの等式) $H$が可分のとき,条件$\forall_{x,y\in H}\;(\F x|\F y)_{l^2(\N)}=\sum_{n\in\N}\F x\o{\F x}=(x|y)_H$は$\{e_n\}$が正規直交基底をなすことに同値.
    \end{enumerate}
\end{corollary}

\subsection{Hilbert空間の直積と基底}

\begin{proposition}
    $L^2(X),L^2(Y)$の正規直交基底$\phi_n,\psi_n$について,$\Brace{\phi_m\psi_n}_{(m,n)\in\N^2}$は$L^2(X\times Y)$の正規直交基底である.
\end{proposition}

\subsection{幾何的性質}

\begin{theorem}[Hilbert空間はHaine-Borel性を持たない]
    $\{u_n\}\subset H$を正規直交系とする.
    \begin{enumerate}
        \item (Hilbertの立方体) $Q:=\Brace{x\in H\;\middle|\;x=\sum^\infty_{n=1}c_nu_n,\abs{c_n}\le\frac{1}{n}}$はコンパクトである.
        \item $S:=\Brace{x\in H\;\middle|\;x=\sum^\infty_{n=1}c_nu_n,\abs{c_n}\le\delta_n}$とする.数列$\{\delta_n\}\subset\R_{>0}$が$l^2(\N)$の元であることと,$S$がコンパクトであることとは同値.
    \end{enumerate}
\end{theorem}

\section{Hilbert空間上の作用素}

\begin{tcolorbox}[colframe=ForestGreen, colback=ForestGreen!10!white,breakable,colbacktitle=ForestGreen!40!white,coltitle=black,fonttitle=\bfseries\sffamily,
title=具体的$C^*$-環論は抽象的$C^*$-論と等価]
    任意の単位的$C^*$-代数$\A$に対して,Hilbert空間$H$が存在して,$*$-等長同型$\Phi:\A\mono B(H)$が存在する.
    \begin{enumerate}
        \item $B(H)$は単位的$C^*$-環になる.
        \item 自己共役な作用素のなす部分空間$B(H)$は,$*$-作用素について不変な閉部分空間$B(H)_\sa$である.数域半径は作用素ノルムと同値なノルムを定めるが,特に$B(H)_\sa$上では値が一致する.
        \item 正作用素のなす閉凸錐$B(H)_+$は,$\forall_{x\in H}\;(Tx|x)\ge0$を満たす$B(H)_\sa$の部分集合である.これを通じて,$B(H)_\sa$には順序が定まる.
        \item 正射影の空間$B(H)_p$は,$0\le P\le I$を満たす$B(H)_+$の部分集合であり,束の構造を持つ.これは特性関数に当たる.$B_0(H)$には直交射影からなる近似的単位元が存在する.
        \item 正規直交基底$(e_j)_{j\in J}$に関して有界な係数集合$\{\lambda_j\}\subset\bF$による和で表せる作用素を対角化可能といい,その全体を$B(H)_\diag$で表す.これは可測関数のようなものである.
        \item 正規作用素の空間$B(H)_\normal$は$B(H)_\diag$のノルム閉包である.この上ではスペクトル半径と数域半径が一致する.すると,対角化可能で固有値が無限遠で消えるクラスは,正確に正規なコンパクト作用素$B_0(H)\cap B_\normal(H)$であると分かる.
        \item ユニタリ作用素のなす群$U(H)\subset B(H)$は弧状連結である.$B(H)$の開球の元はユニタリ作用素の凸結合(特に平均)として得られる(Russo-Dye-Gardner).
        \item 跡類作用素とHilbert-Schmidt作用素$B_f(H)\subset B^1(H)=\Span\Brace{T\in B_0(H)\mid T\ge0,\Tr(T)<\infty}\subset B^2(H):=\Brace{T\in B_0(H)\mid\Tr(T)<\infty}$は$B(H)$内の自己共役なイデアルをなす.跡が定めるペアリング$\brac{S,T}:=\Tr(ST)$により,$(B_0(H))^*=B^1(H)$かつ$(B^1(H))^*=B(H)$の関係がある.
    \end{enumerate}
    さらに一般に,$B(H)$を$C^*$-代数$\A$,$B(H)_\sa$を$\Re\A$に一般化して考察する.
\end{tcolorbox}

\begin{notation}
    $H$をHilbert空間,$(-|-)$をその内積,$B(H)$をその上の有界な自己準同型,$I\in B(H)$を恒等写像$\id_H$とする.
\end{notation}

\begin{history}
    この一般化に際して,Hilbertは二次形式/双線型形式の不変式論を04-10に研究していて,\footnote{不変式論のテーマは,BooleからCayleyに引き継がれてから,大陸を渡ってHilbertに届いた.2人ともlogicに入る前は不変式論の研究をしていた.}その時にスペクトル理論を構築したが,
    基底の選択と無限次元行列としての表現,そして積は畳み込みとして成分ごとに計算するというのはあまりにも煩雑であった.
    von Neumannの成功は,補題の同型を渡って,作用素の概念の方に注目したことが大きい.
    これが作用素の研究の第一歩となる.
\end{history}

\subsection{随伴作用素}

\begin{tcolorbox}[colframe=ForestGreen, colback=ForestGreen!10!white,breakable,colbacktitle=ForestGreen!40!white,coltitle=black,fonttitle=\bfseries\sffamily,
title=]
    Hilbert空間では,標準的なペアリングを通じて任意の有界線型作用素を表現できるのであった.
    これは随伴と呼ばれる対合$*:B(H)\to B(H)$を作用素の間に定める.
    こちらの構造はさらに奥が深い.
    これは行列の共役転置の一般化である.
    またこうして自己共役性の概念が内積から作用素へ流入する.
\end{tcolorbox}

\begin{lemma}[有界作用素の内積による特徴付け]\label{lemma-correspondence-between-sesquilinearform-and-operator}
    Hilbert空間$H$上の有界線形作用素と,これが内積$(-|-)$を通じて定める有界な半双線型形式との
    次の対応は等長同型である:
    \[\xymatrix@R-2pc{
        B(H)\ar[r]&\{\brac{-|-}\in\Map(H\times H,\bF)\mid\brac{-|-}は半双線型\}\\
        \rotatebox[origin=c]{90}{$\in$}&\rotatebox[origin=c]{90}{$\in$}\\
        T\ar@{|->}[r]&B_T(x,y):=(x|Ty)
    }\]
    ただし,終域となっている空間のノルムは作用素ノルム$\norm{B_T}:=\sup\Brace{\abs{B_T(x,y)}\in\R_+\mid\norm{x}\le 1,\norm{y}\le1}$とする.
\end{lemma}
\begin{Proof}\mbox{}
    \begin{description}
        \item[$\to$がwell-definedな等長写像] 
        $T\in B(H)$ならば,$B_T:=(-,T-)$は明らかに半双線型形式である.
        あとは$B_T$が有界であり,$\norm{T}=\norm{B_T}$であることを示せば良い.

        まず,Cauchy-Schwarzの不等式より,
        \begin{align*}
            \norm{B_T}&=\sup\Brace{\abs{B_T(x,y)}\in\R\mid\norm{x}\le1,\norm{y}\le1}\\
            &=\sup\Brace{\abs{(x|Ty)}\in\R\mid\norm{x}\le1,\norm{y}\le1}\le\norm{T}
        \end{align*}
        よって$B_T$は有界である.次に,任意の$x\in H$について,
        \begin{align*}
            \norm{Tx}^2&=(Tx|Tx)=B_T(Tx,x)=B_T\paren{\norm{Tx}\frac{Tx}{\norm{Tx}},\norm{x}\frac{x}{\norm{x}}}\\
            &\le\norm{B_T}\norm{Tx}\norm{x}\le(\norm{B_T}\norm{T})\norm{x}^2
        \end{align*}
        と評価できるから,$\norm{T}^2\le\norm{B_T}\norm{T}$より,$\norm{T}\le\norm{B_T}$.
        \item[$\leftarrow$がwell-definedな切断] 
        $B$を$H$上の有界な半双線型形式とする.任意の$y\in H$に対して,$B(-,y)\in H^*$だから,Rieszの表現定理より,$\exists!_{Ty\in H}\;B(-,y)=(-|Ty)$.
        こうして写像$T:H\to H;y\mapsto Ty$が定まる.これが有界な線形作用素であることと,$\rightarrow\circ\leftarrow=\id$であることを示せば良い.

        線形性は明らか:$B(-,ay)=\o{a}B(-,y)=\o{a}(-|Ty)=(-|a\cdot Ty)$.
        有界性は,既に行った評価$\forall_{y\in H}\;\norm{Ty}^2\le\norm{B}\norm{T}\norm{y}^2$より,$\norm{T}\le\norm{B}$で抑えられる.
        よって,$T\in B(H)$.
        また,この有界作用素$T$が定める半双線型形式は$B_T=B$に他ならない.
    \end{description}
\end{Proof}

\begin{theorem}[随伴作用素の存在]\mbox{}\label{thm-existence-of-adjoint-operator}
    \begin{enumerate}
        \item 任意の$T\in B(H)$に対して,$\forall_{x,y\in H}\;(Tx|y)=(x|T^*y)$を満たす$T^*\in B(H)$が唯一つ存在する.
        \item これが定める全単射な対応${}^*:B(H)\to B(H)$について,
        \begin{enumerate}[(a)]
            \item 対合的(周期2)である:$T^{**}=T$.
            \item 共役線型である:$(aT)^*=\o{a}T^*$.
            \item 乗法について反変的である:$(ST)^*=T^*S^*$.\footnote{antimultiplicativeと表現されている.homomorphismが積を保つのに対し,antihomomorphismとは積を逆にする.}
            \item 等長写像である:$\norm{T}=\norm{T^*}$.
            \item $\norm{T^*T}=\norm{T}^2$を満たす.
        \end{enumerate}
    \end{enumerate}
\end{theorem}
\begin{Proof}\mbox{}
    \begin{enumerate}
        \item 任意の$T\in B(H)$に対して,$(T-|-):H^2\to\bF$は有界な半双線型形式であるから,ただ一つの$T^*\in B(H)$が存在して$(T-|-)=(-|T^*-)$を満たす.
        \item \begin{enumerate}[(a)]
            \item 内積は自己共役であるから,任意の$x,y\in H$について,
            \begin{align*}
                (x|T^{**}y)&=(T^*x|y)=\o{(y|T^*x)}\\
                &=\o{(Ty|x)}=(x|Ty).
            \end{align*}
            より,$T,T^{**}$は同一の半双線型形式を定める.よって,$T=T^**$.
            \item 同様にして,
            \begin{align*}
                \forall_{x,y\in H}\;\quad(x|(aT)^*y)=(aTx|y)=a(Tx|y)=(x|\o{a}T^*y)
            \end{align*}
            より,$(aT)^*=\o{a}T^*$.
            \item \begin{align*}
                \forall_{x,y\in H}\quad(x|(ST)^*y)&=(STx|y)=(Tx|S^*y)=(x|T^*(S^*(y)))
            \end{align*}
            \item 作用素ノルムの劣乗法性より,$\norm{T^*T}\le\norm{T}\norm{T^*}$である.また,
            \begin{align*}
                \forall_{x\in H}\quad\norm{Tx}^2=(Tx|T^{**}x)=(T^*Tx|x)\le\norm{T^*T}\norm{x}^2
            \end{align*}
            より,$\norm{T}^2\le\norm{T^*T}(\le\norm{T}\norm{T^*})$でもある.よって2つ併せて,$\norm{T}\le\norm{T^*}(\le\norm{T^{**}}=\norm{T})$を得るから,$\norm{T}=\norm{T^*}$.
            \item $\norm{T^*T}=\norm{T}^2$は,$B$を$H$の閉単位球として$K=T(B)$とおくと,
            $\sup_{y\in K}\abs{T^*y}=\sup_{y\in K}\abs{Ty}$と見ると明らか.
        \end{enumerate}
    \end{enumerate}
\end{Proof}

\subsection{自己共役作用素}

\begin{tcolorbox}[colframe=ForestGreen, colback=ForestGreen!10!white,breakable,colbacktitle=ForestGreen!40!white,coltitle=black,fonttitle=\bfseries\sffamily,
title=]
    自己共役作用素はノルムに対して特殊な振る舞いをする.
    特に,作用素ノルムの特徴付け$\norm{T}=\sup_{x\in\partial B}\abs{(Ax|x)}\;(T\in B(H)_\sa)$は本質的であり,
    自己共役作用素を調べるのに,半内積$(T-|-)$の構造と,これについての一般化Cauchy-Schwarzの不等式$\abs{(Ax|y)}\le\sqrt{(Ax|x)(Ay|y)}$を多用する.
\end{tcolorbox}

\begin{definition}[self-adjoint / hermitian]\mbox{}
    \begin{enumerate}
        \item 乗法についての反準同型$*:B(H)\to B(H)$が対合な共役線型写像で,等長同型でもあるとき,$(B(H),*)$を$B^*$-代数という.$C^*$-性$\norm{T^*T}=\norm{T}^2$も満たすとき,$C^*$-代数という.\footnote{最初に定義したI. E. Segal in 1947で,$C$は"Closed"から取られた.\url{https://en.wikipedia.org/wiki/C*-algebra}}
        \item $T\in B(H)$が$T=T^*$を満たすとき,これを自己共役作用素という.これを$B_\sa(H)=B(H)^{\Brace{*}}$\footnote{群作用に対する不変式の記法を踏襲した.}と表すと,$B(H)$の閉な実部分空間となる.
    \end{enumerate}
\end{definition}

\begin{example}[随伴]\mbox{}
    \begin{enumerate}
        \item 乗算作用素\ref{operator-multiplication}については$M^*_\varphi=M_{\o{\varphi}}$となる.よって,$\varphi\in L^\infty(^mu)$が実数値である場合に限り,自己共役である.$\abs{\phi}=1$である場合に限り,ユニタリである.
        \item $k$を核とする積分作用素$K$\ref{operator-integral-transformation}については$K^*$は$k^*(x,y)=\o{k(y,x)}$を核とする積分作用素となる.よって,$k(x,y)=\o{k(y,x)}\;\ae[\mu\times\mu]$である場合に限り,自己共役である.
        \item shift作用素\ref{operator-unilateral-shift}は方向が逆になる:$S^*(\al_1,\al_2,\cdots)=(\al_2,\al_3,\cdots)$.これを\textbf{後方シフト(backward shift)}と呼ぶ.
        \item 複素Hilbert空間$H$上の作用素$A\in B(H)$について,$B:=\frac{A+A^*}{2},C:=\frac{A-A^*}{2}$をそれぞれ実部と虚部と呼び,自己共役である.
    \end{enumerate}
\end{example}

\begin{corollary}[自己共役作用素の特徴付け]\label{cor-characterization-of-self-adjointness}
    $\bF=\C$のとき,次の2条件は同値.
    \begin{enumerate}
        \item $T=T^*$.
        \item $\forall_{x\in H}\;(Tx|x)\in\R$.
    \end{enumerate}
\end{corollary}
\begin{remark}[自己共役作用素の固有値は実数]
    極化恒等式による証明\ref{lemma-characterization-of-self-adjointness}参照.
    $H$が実Hilbert空間という仮定のみでは,その上の任意の作用素$A\in B(H)$に対して$\brac{Ah,g}\in\R$が常に成り立つので,特徴付けにならないことに注意.
\end{remark}

\begin{proposition}[自己共役作用素の作用素ノルム]\label{prop-operator-norm-of-self-adjoint-operator}
    $A$を自己共役作用素とする.このとき,作用素ノルムは
    \[\norm{A}=\sup\Brace{\abs{(Ax|x)}\in\R\mid\norm{x}=1}=\sup_{x\ne0}\frac{\abs{(Ax|x)}}{\norm{x}^2}.\]
    とも表せる.
\end{proposition}
\begin{remarks}[数域]
    これはRayleigh商の値域になっている.数域半径と深い関係にある.
\end{remarks}

\begin{corollary}[半内積の非退化性]\label{cor-nondegeneratedness-of-semi-inner-product}
    $A$を自己共役作用素とする.このとき,$\forall_{x\in H}\;(Ax|x)=0\Rightarrow A=0$.
    なお,$H$が複素Hilbert空間ならば,一般の$A$について成り立つ.
\end{corollary}
\begin{Proof}\mbox{}
    \begin{enumerate}
        \item 自己共役作用素の作用素ノルムの性質\ref{prop-operator-norm-of-self-adjoint-operator}により,$\forall_{x\in H}\;(Ax|x)=0\Rightarrow\norm{A}=0\Rightarrow A=0$が従うことによる.
        \item ??
    \end{enumerate}
\end{Proof}

\subsection{Bantの同型}

\begin{tcolorbox}[colframe=ForestGreen, colback=ForestGreen!10!white,breakable,colbacktitle=ForestGreen!40!white,coltitle=black,fonttitle=\bfseries\sffamily,
title=]
    ${}^*:B(H)\to B(H)$というのはある種${}^\op$のように使える.
    作用素の可逆性は,随伴の消息を以て特徴付けることが出来る.
\end{tcolorbox}

\begin{proposition}\label{prop-Ker-of-adjoint-operator}
    任意の$T\in B(H)$に対して,$\Ker T^*=(\Im T)^\perp$.
\end{proposition}
\begin{Proof}
    $\forall_{x,y\in H}\;(x|T^*y)=(Tx|y)$の下で,
    $y\in\Ker T^*\Rightarrow (T(H)|y)=0$より,$y\in(T(H))^\perp$.
    逆に$y\in(T(H))^\perp\Rightarrow (H|T^*y)=0\Rightarrow T^*y\in H^\perp=\{0\}$.
\end{Proof}
\begin{remark}
    双対的な関係は$(\Ker A)^\perp=\oo{\Im A^*}$までしか成り立たない.
\end{remark}

\begin{proposition}\label{prop-characterization-of-invertibleness-of-operator}
    $T\in B(H)$について,次の6条件は同値.
    \begin{enumerate}
        \item $T$は可逆:$T^{-1}\in B(H)$.
        \item $T^*$は可逆.
        \item $T,T^*$はbounded away from zero:$\exists_{\ep>0}\;\forall_{x\in H}\;\norm{Tx}\ge\ep\norm{x}$.%$\inf d(0,\Im T)>0$.
        \item $T,T^*$は単射で,$\Im T$はノルム閉.
        \item $T$は全単射.
        \item $T,T^*$は全射.
    \end{enumerate}
\end{proposition}
\begin{Proof}\mbox{}
    \begin{description}
        \item[(1)$\Leftrightarrow$(2)] $(T^{-1}T)=(TT^{-1})=I$であるとき,${}^*:B(H)\to B(H)$の劣乗法性より$(T^{-1})^*$が$T^*$の逆射である.また$T^*T^{*-1}=T^{*-1}T^*=I$のときも同様に$*$を作用させれば良い.
        \item[(1)$\Rightarrow$(3)] 任意の$x\in H$について$\norm{x}=\norm{T^{-1}Tx}\le\norm{T^{-1}}\norm{Tx}$より,$\ep:=\norm{T^{-1}}^{-1}$と取れば良い.
        \item[(3)$\Rightarrow$(4)] $\exists_{\ep>0}\;\forall_{x\in H}\;\norm{Tx}\ge\ep\norm{x}$は特に$\norm{Tx-Ty}\ge\ep\norm{x-y}$より,単射性$Tx=Ty\Rightarrow x=y$を含意する.またこれより,$T(H)$の任意のCauchy列$(Tx_i)_{i\in\N}$は$H$上のCauchy列$(x_i)_{i\in H}$を定めることより,$Tx$に$T(H)$は完備で,特に閉集合.
        \item[(4)$\Rightarrow$(5)] ノルム閉包の直交補空間による特徴付け\ref{cor-expression-of-closed-linear-span}より,$T(H)=\dbloverline{T(H)}=(T(H)^\perp)^\perp$.命題より,$(T(H)^\perp)^\perp=(\Ker T^*)^\perp=0^\perp=H$.
        \item[(5)$\Rightarrow$(1)] 開写像定理の系\ref{cor-inverse-mapping-theorem}より.
        \item[(6)$\Rightarrow$(5)] $\Ker T=(T^*(H))^\perp=0$より.
        \item[(1)$\Rightarrow$(6)] 明らか.
    \end{description}
\end{Proof}

\subsection{Hilbの同型}

\begin{tcolorbox}[colframe=ForestGreen, colback=ForestGreen!10!white,breakable,colbacktitle=ForestGreen!40!white,coltitle=black,fonttitle=\bfseries\sffamily,
title=]
    Hilbの射は等長同型(=全写な等長写像)であった.
    一般にMetでは計量写像(short map)を射とする.
    そこでは,等長写像がモノ射となる.
    全射な等長写像がHilbの同型となり\textbf{ユニタリー作用素}と呼ばれる.
\end{tcolorbox}

\begin{proposition}[モノ射の内積による特徴付け]\label{prop-characterization-of-isometry}
    $V:H\to K$を線型写像とする.
    次の2条件は同値.
    \begin{enumerate}
        \item $V$は等長写像である.
        \item $\forall_{h,g\in H}\;(Vh|Vg)=(h|g)$.
    \end{enumerate}
\end{proposition}
\begin{Proof}\mbox{}
    \begin{description}
        \item[(2)$\Rightarrow$(1)] $\forall_{h\in H}\;\norm{Vh}^2=(Vh|Vh)=(h|h)=\norm{h}^2$より,$V$は等長である.
        \item[(1)$\Rightarrow$(2)] $\forall_{\lambda\in\bF}\;\norm{h+\lambda g}^2=\norm{Vh+\lambda Vg}^2$について,極化恒等式を用いると,
        \[\norm{h}^2+2\Re\o{\lambda}(h|g)+\abs{\lambda}^2\norm{g}^2=\norm{Vh}^2+2\Re\o{\lambda}(Vh|Vg)+\abs{\lambda}^2\norm{Vg}^2\]
        より,$\forall_{\lambda\in\bF}\;\Re\o{\lambda}(h|g)=\Re\o{\lambda}(Vh|Vg)$.
        $\bF=\R$のとき,$\lambda=1$とすると結論を得る.$\bF=\C$のとき,$\lambda=1,i$とすると結論を得る.
    \end{description}
\end{Proof}
\begin{remarks}
    内積とノルムをつなぐものが極化恒等式である.
\end{remarks}

\begin{theorem}
    $U\in B(H)$をユニタリ作用素とする.このとき,自己共役作用素$T\in B(H)$が存在して,$U=\exp(iT)$と表せる.
\end{theorem}

\begin{theorem}
    ユニタリ作用素の群$U(H)\subset B(H)$はノルム位相について弧状連結である.
\end{theorem}

\subsection{正規作用素}

\begin{tcolorbox}[colframe=ForestGreen, colback=ForestGreen!10!white,breakable,colbacktitle=ForestGreen!40!white,coltitle=black,fonttitle=\bfseries\sffamily,
title=真に$C^*$-代数の性質を特徴づける作用素のクラスである]
    正規であるという可換性条件は,距離的な特徴付けがある.
    随伴とノルム的に判別不可能ならば,正規である.
    スペクトル定理は,正規作用素なるクラスが正確に対角化可能であることを主張する.
\end{tcolorbox}

\begin{definition}[normal]
    作用素$T\in B(H)$について,次の2条件は同値.$\F=\C$のとき,(3)も同値.
    \begin{enumerate}
        \item $T^*T=TT^*$.
        \item (metrically identical) $\norm{Tx}=\norm{T^*x}$.
        \item $T$の実部と虚部は可換である.
    \end{enumerate}
    この同値な条件を満たすとき,$T$を\textbf{正規作用素}という.
\end{definition}
\begin{Proof}\mbox{}
    \begin{description}
        \item[(1)$\Rightarrow$(2)] $\norm{Tx}=(T^*Tx|x)^{1/2}=(TT^*x|x)^{1/2}=\norm{T^*x}$.
        \item[(2)$\Rightarrow$(1)] 極化恒等式から$\forall_{x,y\in H}\;(T^*Tx|y)=(TT^*x|y)$を得る.具体的には,
        \[\norm{Tx}^2-\norm{T^*x}^2=(Tx|Tx)-(T^*x|T^*x)=((TT^*-T^*T)x|x)\]
        であるが,交換子作用素$TT^*-T^*T$は自己共役であるから,任意の$x\in H$について,この式が常に$0$になるのは$TT^*=T^*T$と等価.
        \item[(1)$\Leftrightarrow$(3)] $A=B+iC$とおく.
        \begin{align*}
            A^*A&=B^2+C^2+(BC-CB)i\\
            AA^*&=B^2+C^2+(CB-BC)i
        \end{align*}
        より,$A^*A=AA^*\Leftrightarrow BC=CB$.
    \end{description}
\end{Proof}

\begin{example}
    自己共役作用素とユニタリ作用素は正規である.多くの書籍はスペクトル分解をこのクラスに限る.
    また,半正定値作用素\ref{def-positive-operator}も正規である.
\end{example}

\begin{proposition}[等長写像の特徴付け]\label{prop-characterization-of-isometry-2}
    $A\in B(H)$について,次の3条件は同値.
    \begin{enumerate}
        \item $A$は等長写像である.
        \item $A^*A=I$.
        \item $\forall_{h,g\in H}\;(Ah|Ag)=(h|g)$.
    \end{enumerate}
\end{proposition}
\begin{Proof}\mbox{}
    \begin{description}
        \item[(1)$\Leftrightarrow$(3)] 等長写像の特徴付け\ref{prop-characterization-of-isometry}に他ならない.
        \item[(2)$\Leftrightarrow$(3)] $\forall_{h,g\in H}\;(A^*Ah|g)=(Ah,Ag)$より,$A^*A=I$と同値.
    \end{description}
\end{Proof}

\begin{proposition}[同型の特徴付け]
    $A\in B(H)$について,次の3条件は同値.
    \begin{enumerate}
        \item $A^*A=AA^*=I$.
        \item $A$は全写な等長写像,すなわちユニタリである.
        \item $A$は正規な等長写像である.
    \end{enumerate}
\end{proposition}
\begin{Proof}\mbox{}
    \begin{description}
        \item[(1)$\Rightarrow$(2)] 等長写像の特徴付け\ref{prop-characterization-of-isometry-2}より$A$は等長で,また可逆であるから特に全写.
        \item[(2)$\Rightarrow$(3)] $A$について,ユニタリならば等長だから,$A^*A=I$.また一般に,全写な等長写像$A$の逆写像$A^{-1}$も全写な等長写像である.よって,$(A^{-1})^*A^{-1}=I\Leftrightarrow(AA^*)^{-1}=I$.
        \item[(3)$\Rightarrow$(1)] 等長写像の特徴付け\ref{prop-characterization-of-isometry-2}より,$A^*A=I$.$A$は正規だから,$AA^*=I$も成り立つ.
    \end{description}
\end{Proof}

\begin{corollary}
    $A$が自己共役ならば,$\forall_{h\in H}\;(Ah|h)=0\Rightarrow A=0$.
\end{corollary}

\begin{corollary}[正規作用素の可逆性]\label{prop-characterization-of-invertibleness-of-normal-operator}
    正規作用素$T\in B(H)$について,次の2条件は同値.
    \begin{enumerate}
        \item $T$は可逆.
        \item $T$ is bounded away from zero.
    \end{enumerate}
\end{corollary}
\begin{Proof}
    作用素の可逆性の特徴付け\ref{prop-characterization-of-invertibleness-of-operator}と,正規性の特徴付け$\norm{Tx}=\norm{T^*x}$より.
\end{Proof}

\subsection{半正定値作用素}

\begin{tcolorbox}[colframe=ForestGreen, colback=ForestGreen!10!white,breakable,colbacktitle=ForestGreen!40!white,coltitle=black,fonttitle=\bfseries\sffamily,
title=]
    半正定値作用素は,density matrix formalismを通じて,量子状態を表す.
    自己共役作用素のうち,特別なクラスである.
    大雑把には,複素平面内で,自己共役作用素$B(H)_\sa$は実数,正作用素$B(H)_+$は非負実数と見れる.
\end{tcolorbox}

\begin{definition}[positive (semi-definite)]\label{def-positive-operator}
    $T\in B(H)$が\textbf{半正定値}であるとは,$T=T^*$かつ$\forall_{x\in H}\;(Tx|x)\ge 0$を満たすことをいう.\footnote{2つ条件があるように見えるが,$\bF=\C$かつ$T$が$H$全域で定義されているとき,半正定値性は自己共役性を含意する\ref{lemma-characterization-of-self-adjointness}.これは極化恒等式による.}
    これを「\textbf{作用素$T$は正である}」と省略して良い,$T\ge 0$と表す.正定値であることは$T>0$で表す.
\end{definition}

\begin{lemma}[正作用素の閉凸錐]\label{lemma-positive-closed-cone-of-positive-operator}
    $T_1,T_2\in B(H)$を半正定値とする.
    \begin{enumerate}
        \item $T_1+T_2$も半正定値:半正定値は作用素は凸錐をなす.
        \item 正作用素の凸錐は閉集合である.
        \item $T_1T_2$は一般には半正定値とも自己共役とも限らないが,この積が自己共役であるとき,正である.
    \end{enumerate}
\end{lemma}
\begin{Proof}\mbox{}
    \begin{enumerate}
        \item $\forall_{x\in H}\;(T_1x+T_2x|x)=(T_1x|x)+(T_2x|x)\ge0$.
        \item 閉性は,Cauchy列$(T_n)$を取ることによる.極限$T:=\lim_{n\to\infty}T_n$は正ではないと仮定する:$\exists_{x\in H}\;(Tx|x)<0$.$\ep:=-(Tx|x)>0$とおくと,$\exists_{N\in\N}\;\forall_{n\ge N}\;\norm{T_n-T}\le\ep/\norm{x}^2$だから,これについて,
        \[\Abs{(T_nx|x)-(Tx|x)}=\abs{((T_n-T)x|x)}\le\ep.\]
        よって,$T_n$が正であることに矛盾.
        \item 二乗根補題\ref{prop-square-root-lemma}より,$ST=(S^{1/2})^2T=S^{1/2}TS^{1/2}\ge S^{1/2}0S^{1/2}=0$.
        $\forall_{x\in H}\;(ABx|x)=(\sqrt{A}\sqrt{A}Bx|x)=(B\sqrt{A}x|\sqrt{A})\ge0$と言ってもよい.
    \end{enumerate}
\end{Proof}

\begin{lemma}[半正定値作用素の自己共役性]
    半正定値作用素は,$\bF=\C$のとき自己共役であるが,$\bF=\R$のときそうとは限らない.
\end{lemma}
\begin{Proof}
    $\bF=\C$の場合は特徴付け\ref{cor-characterization-of-self-adjointness}より明らか.
    $\bF=\R$の場合は$A:\R^2\to\R^2$を$\varphi\in(-\pi/2,\pi/2)$回転とすると,$A^*=A^{-1}\ne A$となる.
\end{Proof}

\subsection{自己共役作用素の順序}

\begin{tcolorbox}[colframe=ForestGreen, colback=ForestGreen!10!white,breakable,colbacktitle=ForestGreen!40!white,coltitle=black,fonttitle=\bfseries\sffamily,
title=]
    幾何学的には,正作用素の全体$B(H)_+$は,自己共役作用素のなす閉・実・順序部分空間$B(H)_\sa$の内部で閉凸錐をなす.
    しかしこの順序は束にさえならず,より深い考察には二乗根補題を必要とする.
\end{tcolorbox}

\begin{proposition}[ordering by positive operator]\mbox{}
    \begin{enumerate}
        \item 自己共役な作用素は,$B(H)$の閉な実部分空間をなす.これを$B(H)_\sa$と表す.
        \item $B(H)_\sa$は,順序$S\le T:\Leftrightarrow T-S\ge 0$を備える.
    \end{enumerate}
\end{proposition}
\begin{Proof}\mbox{}
    \begin{enumerate}
        \item $0\in B(H)$.$*:B(H)\to B(H)$は共役線型であるから,実数倍は関係$T=T^*$を変えず,和についても閉じている.
        また,ノルム閉性については,$(T_n)$を$B(H)_\sa$のCauchy列とすると,収束先$T:=\lim_{n\to\infty}T_n$を持つ.$*:B(H)\to B(H)$は等長写像だから,像$(T_n^*)$もCauchy列で,収束先$T^*$を持ち,$*:B(H)\to B(H)$は特に連続だから,$T=\lim_{n\to\infty}T_n=\lim_{n\to\infty}T_n^*=T^*$.
        \item \begin{enumerate}[(a)]
            \item $S\le S\Leftrightarrow S-S=0\ge 0$は,零写像は半正定値だから,常に成り立つ.
            \item $S\le T$かつ$T\le S$のとき,$\forall_{x\in H}\;(Tx|x)=(Sx|x)$であるから,$T=S$.これは,自己共役作用素の作用素ノルムの性質\ref{prop-operator-norm-of-self-adjoint-operator}により,$\forall_{x\in H}\;(Ax|x)=0\Rightarrow\norm{A}=0\Rightarrow A=0$が従うことによる.
            \item 推移律は明らか.
        \end{enumerate}
    \end{enumerate}
\end{Proof}

\begin{lemma}[数域:自己共役作用素に特徴的な対象]\label{lemma-numerical-range}
    写像
    \[\xymatrix@R-2pc{
        B:B(H)_\sa\ar[r]&\Map(H,\R)\\
        \rotatebox[origin=c]{90}{$\in$}&\rotatebox[origin=c]{90}{$\in$}\\
        T\ar@{|->}[r]&(T-|-)=:B_T(-)
    }\]
    を考える.
    \begin{enumerate}
        \item これは線型社像ではない,計量写像である.特に,有界.
        \item 単射である.
    \end{enumerate}
\end{lemma}
\begin{Proof}
    等長同型であることを示せば,特に有界であること,すなわちこの写像のwell-definednessが従う.

    まず,$\forall_{x\in B}\;\norm{(Tx|x)}\le\norm{Tx}\norm{x}\le\norm{x}^2\norm{T}\le\norm{T}$より,$\norm{(T-|-)}\le\norm{T}$.
    逆は,任意の$x\in H$について,$(Tx|Tx)$を合成$x\mapsto Tx\mapsto (Tx|Tx)$と見ると,
\end{Proof}

\begin{proposition}[自己共役作用素の順序]\label{prop-order-of-self-adjoint-operator}
    $S,T\in B(H)_\sa$について,
    \begin{enumerate}
        \item $S\le T$ならば,$\forall_{A\in B(H)}\;A^*SA\le A^*TA$.
        \item $0\le S\le T$ならば,$\norm{S}\le\norm{T}$.特に,正作用素$S\ge$について,$S\le I$と$\norm{S}\le1$とは同値.
        \item $-I\le T\le I$と$\norm{T}\le 1$は同値.
    \end{enumerate}
\end{proposition}
\begin{Proof}
    $\bF=\R,\C$のいずれの場合も,自己共役作用素$T$について,$\forall_{x\in H}\;(Tx|x)\in\R$を満たすことに注意.
    \begin{enumerate}
        \item \begin{align*}
            \forall_{y\in H}\;((T-S)y|y)\ge0&\Leftrightarrow\forall_{x\in H}\;((T-S)Ax|Ax)\ge0\\
            &\Leftrightarrow\forall_{x\in H}\;(A^*(T-S)Ax|x)\ge0.
        \end{align*}
        なお,$A^*TA-A^*SA=A^*(T-S)A$は線型写像についてしか成り立たないことに注意.
        \item 半内積$(S-|-):H\times H\to\bF$についてのCauchy-Schwarzの不等式,仮定$0\le S\le T$より,任意の単位ベクトル$x,y\in\partial B$について,
        \begin{align*}
            \abs{(Sx|y)}^2\le(Sx|x)(Sy|y)\le(Tx|x)(Ty|y)\le\norm{T}^2
        \end{align*}
        であるから,$\norm{S}^2\le\norm{T}^2$が従う.なお,左辺の上限が$\norm{S}^2$に等しいことは,等長同型\ref{lemma-correspondence-between-sesquilinearform-and-operator}による.
        特に,$T=I$とすると,$0\le S\le I\Leftrightarrow\norm{S}\le 1$.
        \item $\norm{T}\le 1$ならば,Cauchy-Schwarzの不等式より$\abs{(Tx|x)}\le\norm{x}^2=(x|x)$.
        絶対値の場合分けより,$-I\le T\le I$が従う.
        よって,$-I\le T\le I$を仮定して$\norm{T}\le 1$を導く.
        仮定より,
        \begin{align*}
            (T(x+y)|x+y)&\le(I(x+y)|x+y),&(T(x-y)|x-y)&\ge(-I(x-y)|x-y)\\
            \Leftrightarrow\quad(T(x+y)|x+y)&\le\norm{x+y}^2,&-(T(x-y)|x-y)&\le\norm{x-y}^2.
        \end{align*}
        これと中線定理を併せて,特に単位ベクトル$x,y\in\partial B$について,
        \begin{align*}
            &(T(x+y)|x+y)-(T(x-y)|x-y)&\le &2(\norm{x}^2+\norm{y}^2)\\
            \Leftrightarrow\quad&4\Re(Tx|y)&\le&4.
        \end{align*}
    \end{enumerate}
    $x,y\in\partial B$は任意に取ったから,等長同型対応\ref{lemma-correspondence-between-sesquilinearform-and-operator}を通じて,
    $\norm{T}\le 1$を得る.
\end{Proof}

\begin{theorem}[有界単調収束定理\cite{Eidelman}]
    $A_0\le A_1\le\cdots\le A_n\le\cdots\le A$を$B(H)_\sa$の有界な単調増加列とする.
    このとき,ある有界作用素$B$が存在して,これに各点収束する.
\end{theorem}

\begin{proposition}[\cite{Eidelman}]
    $A\in B(H)_\sa$は$\exists_{m,M\in\R}\;m I\le A\le MI$を満たすとし,$P\in\R[x]$を$\forall_{z\in[m,M]}\;P(z)\ge0$を満たす多項式とする.
    このとき,$P(A)\ge0$.
\end{proposition}

\subsection{二乗根補題}

\begin{tcolorbox}[colframe=ForestGreen, colback=ForestGreen!10!white,breakable,colbacktitle=ForestGreen!40!white,coltitle=black,fonttitle=\bfseries\sffamily,
title=]
    正作用素は正規作用素の代表例だが,その正規たる所以を解明する.
    基本的にはスペクトル定理の帰結でもある.

    正作用素は凸錐をなすが,その中に二乗根が必ず見つかる.
    可逆性は,順序構造を用いて特徴付けることが出来る.
    また正作用素の逆は再び正である.
\end{tcolorbox}

\begin{lemma}
    正係数を持つ多項式の列$\{p_n\}\subset\R_{>0}[x]$であって,
    和$\sum_{n\in\N}p_n$が$[0,1]$上で関数$t\mapsto 1-(1-t)^{1/2}$に一様収束するものが存在する.
\end{lemma}

\begin{proposition}[square root lemma]\label{prop-square-root-lemma}
    半正定値な作用素$T\in B(H)$について,
    \begin{enumerate}
        \item ただ一つの半正定値な作用素$T^{1/2}\ge0$が存在して,$(T^{1/2})^2=T$を満たす.
        \item $A\in B(H)$が$T$と可換ならば,$T^{1/2}$と可換である.
    \end{enumerate}
\end{proposition}
\begin{Proof}\mbox{}
    \begin{enumerate}
        \item 
        \begin{description}
            \item[存在] 任意の$\al\ge 0$に対して,$(\al T)^{1/2}:=\al^{1/2}T^{1/2}$と対応させれば良いから,$T\le I$を仮定して存在を示せば良い.
            $S:=I-T$とすると,$0\le S\le I$を満たす.
            いま.補題の多項式列$\{p_n\}\subset\R_{\ge 0}[x]$に対して$S_n:=p_n(S)$と定めると,和$\sum_{n\in\N}p_n$は一様収束するから,
            \[\forall_{\ep>0}\;\exists_{n_0\in\N}\;\forall_{m\ge n_0}\;\forall_{t\in[0,1]}\;0\le\sum_{n=n_0}^mp_n(t)=\sum\al_kt^k\le\ep.\]
            $t=S$を代入すると,
            \[\Abs{\sum^m_{n=n_0}S_n}\le\sum\al_k\norm{S^k}\le\sum\al_k\le\ep.\]
            よって,$\sum S_n$はCauchy列だから,$0\le R\le I$を満たすある作用素$R$に一様収束する.
            これについて,$T^{1/2}:=I-R$と定めれば良い.
            \item[一意性]
        \end{description}
        \item $T^{1/2}$は,$I,T$からなる多項式の一様収束極限として構成したから,$T$と可換ならば$T^{1/2}$とも可換である.
    \end{enumerate}
\end{Proof}

\begin{corollary}
    $A\ge 0$かつ$(Ax,x)=0$ならば,$Ax=0$.
\end{corollary}
\begin{Proof}
    $A$の二乗根$X:=\sqrt{A}$を取ると,$(Ax|x)=(Xx|Xx)=0$.内積の非退化性より,$Xx=0$.よって,$Ax=0$.
\end{Proof}
\begin{remark}
    一般の場合は\ref{cor-nondegeneratedness-of-semi-inner-product}に示してあるが,正作用素については二乗根補題より直接従う.
\end{remark}

\begin{proposition}[正作用素の可逆性]
    半正定値な作用素$T\in B(H)$について,
    \begin{enumerate}
        \item $T$が可逆であることと,$\exists_{\ep>0}\;T\ge\ep I$は同値.
        \item $T$が可逆であるとき,逆作用素も正$T^{-1}\ge 0$で,$T^{1/2}$も可逆で,$(T^{-1})^{1/2}=(T^{1/2})^{-1}=:T^{-1/2}$.
        \item 可逆な$T$に対して$T\le S$ならば$S$も可逆で,$S^{-1}\le T^{-1}$.
    \end{enumerate}
\end{proposition}
\begin{Proof}\mbox{}
    \begin{enumerate}
        \item \begin{description}
            \item[$\Leftarrow$] $T\ge\ep I$と仮定する.すると,2つの作用素$T-\ep I,T+\ep I$はいずれも正で,互いに可換であるから,積$T^2-\ep^2I$も正である:$T^2\ge\ep^2I$.
            よって,
            \[\norm{Tx}^2=(T^2x|x)\ge\ep^2(x|x)=\ep^2\norm{x}^2.\]
            すなわち,$0$に対して有界であるから,$T$は可逆\ref{prop-characterization-of-invertibleness-of-operator}.
            \item[$\Rightarrow$]
            $T$が可逆ならば$0$に対して有界であるから,$\exists_{\ep>0}\;(T^2x|x)\norm{Tx}^2\ge^2\norm{x}^2=\ep^2(x|x)$より,$T^2\ge\ep^2I$.
            いま,$\Leftarrow$より$T+\ep I$は可逆であり,その逆も正((2)の結果を先に用いた).
            すると,$T-\ep I=(T^2-\ep^2I)(T+\ep I)^{-1}$と見ると,2つの可換な正作用素の積だから,これも正である.
        \end{description}
        \item $\forall_{x\in H}\;(T^{-1}Tx|Tx)=(Tx|x)\ge0$であるから,$T$が可逆であるとき特に$\Im T=H$であることと併せると,$T^{-1}\ge0$を得る.
        (1)の$\Rightarrow$で$T^2\ge\ep^2I$から$T\ge\ep I$を導いた議論を繰り返すことにより,$T^{1/2}\ge\ep^{1/2}I$を得るから,(1)より可逆で正な逆作用素$(T^{1/2})^{-1}$を持つ.
        二乗根の一意性より,$(T^{1/2})^{-1}=(T^{-1})^{1/2}$.
        \item $T\le S$とすると,(1)より$S$も可逆で,また(2)より$S^{-1/2}$も存在する.
        自己共役作用素の順序の性質\ref{prop-order-of-self-adjoint-operator}(1)より,$S^{-1/2}TS^{-1/2}\le S^{-1/2}SS^{-1/2}=I$.
        $C^*$-性と自己共役作用素の順序の性質\ref{prop-order-of-self-adjoint-operator}(2)より,
        \[\norm{T^{1/2}S^{-1/2}}^2=\norm{(T^{1/2}S^{-1/2})^*T^{1/2}S^{-1/2}}=\norm{S^{-1/2}TS^{-1/2}}\le 1.\]
        すると再び$C^*$性と$*:B(H)\to B(H)$が等長写像であることより,
        \[\norm{T^{1/2}S^{-1}T^{1/2}}=\norm{S^{-1/2}T^{1/2}}^2=\norm{T^{1/2}S^{-1/2}}^2\le1.\]
        よって,$T^{1/2}S^{-1}T^{1/2}\le I$を得る.
        再び自己共役作用素の順序の性質\ref{prop-order-of-self-adjoint-operator}(1)より,$S^{-1}\le T^{-1/2}IT^{-1/2}=T^{-1}$.
    \end{enumerate}
\end{Proof}
\begin{remarks}
    正規作用素の可逆性はbounded away from zeroのみであった\ref{prop-characterization-of-invertibleness-of-normal-operator}.
    正作用素については,これをさらに緩めることができる.
\end{remarks}

\subsection{直交射影と対角化}

\begin{tcolorbox}[colframe=ForestGreen, colback=ForestGreen!10!white,breakable,colbacktitle=ForestGreen!40!white,coltitle=black,fonttitle=\bfseries\sffamily,
title=対角化とは,直交射影への標準分解をいう.]
    直交射影は,正作用素の代表例であり,$0\le P\le I$に位置する.
    また作用素の中でも最も基本的なもので,実解析における特性関数のような役割を持つ.
    単関数は直交射影の有限和に当たる.
    これへの分解が対角化の理論である.
\end{tcolorbox}
\begin{remarks}
    この定義は,射影$P$が自己共役であることと,$\Im P=(\Ker P^*)^\perp=(\Ker P)^\perp$が成り立つことが同値であるためである.
\end{remarks}

\begin{definition}[orthogonal projection / orthoprojection]
    一般に,冪等律を満たす線型作用素$P:H\to H,P^2=P$を射影という.
    射影が自己共役であるとき,\textbf{正射影}または\textbf{直交射影}であるという.\footnote{前者を冪等作用素,後者を射影と呼び分けることもある.この場合直交射影とは,直交分解が定める射影のことを指す.}
    射影全体の空間を$B(H)_p$で表す.射影を定める部分空間の束構造から,$B(H)_p$にも束の構造が入る.
\end{definition}
\begin{example}[直交射影の直交分解による特徴付け]
    自己共役性がなぜ「直交」なのかというと,自己共役な射影は全て直交分解が定めるからである.
    \begin{enumerate}
        \item 閉部分空間$X\le H$について,直交分解$H=X+X^\perp$が導く作用素$P:H\to H;y\mapsto x$は$\norm{P}\le 1$かつ$P^2=P$を満たす半正定値作用素である.よってこれは\textbf{直交射影}である.
        \item 逆に,任意の自己共役な冪等作用素$P:H\to H$に対して,$X:=\Im P$とおくとこれは閉部分空間で,任意の$x^\perp\in X^\perp$に対して$\norm{Px^\perp}^2=(x^\perp|P^2x\perp)=0$より,$P$は直交射影である.$I-P$は$\Im(I-P)=X^\perp$を満たす直交射影である.
    \end{enumerate}
    以上の観察より,直交射影$P$について,$\Im P\perp\Ker P$すなわち$\Im P\perp\Im(I-P)$が成り立つ.
\end{example}
\begin{remarks}[diagonalizable, eigenfunction expansion]\label{remarks-diagonalizability}
    行列の対角化とJordan標準形の理論を,作用素論の言葉を用いて捉え直す.
    \begin{enumerate}
        \item 射影は実解析における特性関数にあたる.
        \item 単関数は$T=\sum_{i\in[n]}\lambda_iP_i$にあたる.
        \item 一般に,作用素$T$が\textbf{対角化可能}であるとは,ある正規直交基底$(e_j)_{j\in J}$と有界集合$\{\lambda_j\}_{j\in J}\subset\bF$が存在して,$\forall_{x\in H}\;Tx=\sum_{j\in J}\lambda_j(x|e_j)e_j=\sum_{j\in J}\lambda_jP_j\;(P_j:H\epi\bF e_j)$と表せることをいう.このとき,$(e_j)$を固有ベクトル,$(\lambda_j)$を固有値という.$(x|e_j)$は$x$の$j$-座標である.
        
        なお,無限和$T=\sum_{j\in J}\lambda_jP_j$は作用素の強位相については収束する(各点収束)が,そのほかは保証されない.
        \item $A\in M_n(\C)$を自己共役(Hermite)行列とする.すると,固有値$\lambda_1,\cdots,\lambda_m\in\C$の属する固有空間$W_1,\cdots,W_m$は互いに直交するのであった(ユニタリ行列によって対角化可能なので).
        そこで,各$W_i$への直交射影を$P_i:\C^n\epi W_i$で表すと,$A=\sum^m_{i=1}\lambda_iP_i$が成り立つ.これを作用素論的な「対角化可能性」の定義とすれば良い.
        \item 一般の行列$A\in M_n(\C)$について,固有多項式$p(\lambda)=\det(\lambda I-A)$の根を$\{\lambda_1,\cdots,\lambda_m\}$とし,最小多項式$q(\lambda)$におけるそれぞれの重複度を$k_i\in\N$とする.
        広義固有空間を$\wt{W}_i:=\Brace{u\in\C^n\mid(A-\lambda_i I)^{k_i}u=0}$と定めても,これらは$\C^n$の直和分解ではあるが互いに直交するとは限らない.
        このとき,$\wt{W}_i$への直交とは限らない射影を$\wt{P}_i$とすると,$\wt{W}_i$上の冪零作用素$N_i$を用いて,$A=\sum_{i=1}^m(\lambda_i\wt{P}_i+N_i\wt{P}_i)$と表せる.いわば,冪零・冪等分解である.
    \end{enumerate}
    (4)の式を,$A$の\textbf{固有関数展開}または\textbf{スペクトル展開}といい,(5)の式を\textbf{Jordan分解}という.
\end{remarks}

\begin{lemma}[一般の射影の性質]\label{lemma-property-of-projection}
    $P^2=P\in B(E)$を満たすとする.$E_1:=\Im P,E_2:=\Ker P$とする.
    \begin{enumerate}
        \item $P|_{E_1}=\id_{E_1}$.
        \item $Q:=I-P$は$\Im P=\Ker Q,\Ker P=\Im Q$を満たす射影である.
        \item $E=E_1\oplus E_2$.
    \end{enumerate}
\end{lemma}

\begin{lemma}[直交射影の特徴付け]
    $E$を射影とし,$E\ne 0$とする.次の6条件は同値.
    \begin{enumerate}
        \item $\Ker E=(\Im E)^\perp$.
        \item $E$はある閉部分空間$M$についての直交射影である.
        \item $\norm{E}=1$.
        \item $E$は自己共役である.
        \item $E$は正規である.
        \item $E$は正である.
    \end{enumerate}
\end{lemma}

\subsection{対角化可能作用素と正規作用素}

\begin{lemma}[対角化可能作用素の性質]
    $T$が対角化可能であるとする:$T=\sum_{j\in J}\lambda_jP_j$.
    \begin{enumerate}
        \item $T^*$も対角化可能で,固有ベクトルは$(e_j)$,固有値は$(\o{\lambda_j})$である.
        \item $T$は正規:$TT^*=T^*T$.
        \item $T$が自己共役であることと,固有値が全て実数であることは同値.
        \item $T$が半正定値であることと,固有値が全て非負実数であることは同値.
    \end{enumerate}
\end{lemma}
\begin{Proof}\mbox{}
    \begin{enumerate}
        \item $T^*x=\sum\o{\lambda_j}(x|e_j)e_j$より.
    \end{enumerate}
\end{Proof}

\begin{proposition}[有限次元線型空間論]
    $H$が有限次元であるとき,任意の正規な作用素は対角化可能である.
    また,互いに可換な正規な作用素は,同時対角化可能である.
\end{proposition}
\begin{Proof}
    有限次元線型空間論による.
\end{Proof}

\begin{lemma}
    任意の正規作用素$T\in B(H)$と$\ep>0$について,組ごとに直交な射影の有限集合$\{P_n\}$が存在して和が$I$になるものと$\{\lambda_n\}\subset\C$とが存在して,$\Norm{T-\sum\lambda_nP_n}\le\ep$を満たす.
\end{lemma}

\begin{theorem}
    $B(H)$内の対角化可能な作用素の集合$B_\diag(H)$のノルム閉包は正規作用素の集合である.
\end{theorem}

\subsection{Hilbの(自己)同型}

\begin{tcolorbox}[colframe=ForestGreen, colback=ForestGreen!10!white,breakable,colbacktitle=ForestGreen!40!white,coltitle=black,fonttitle=\bfseries\sffamily,
title=ユニタリ作用素]
    いままではHilbert空間を位相線形空間の延長として扱い,射は有界線型写像としてきたが,
    ここで距離構造に注目する.ユニタリ作用素は,真の意味での自己射である.
    可逆な有界線型写像がユニタリであるための必要十分条件は$V^{-1}=V^*$である.

    全ての基底変換はユニタリ作用素で記述されるのであった\ref{prop-characterization-of-isomorphism-of-Hilbert-spaces}.
    よって,この分だけ条件を緩めることに価値がある.
    ユニタリー部分群
    \[U(H)\mono\GL(H)=B(H)^\times\]
    が考えられ,この$B(H)$への作用による共役類を考える.
    これをユニタリー同値という.
\end{tcolorbox}

\begin{definition}[unitary operator / orthogonal operator]
    $H$の自己等長同型を\textbf{ユニタリ作用素}という(全写な等長写像であることと同値):$\forall_{x\in H}\;\norm{Ux}=\norm{x}$かつ$\Im U=H$.実線型空間については,直交作用素ともいう.
\end{definition}

\begin{lemma}[ユニタリ作用素の特徴付け]\label{lemma-characterization-of-unitary-operator}
    ユニタリ作用素$U\in B(H)$について,
    \begin{enumerate}
        \item $U$は内積を保つ.
        \item $U^*U=I$(実はこれは等長写像の特徴付け).
        \item 正規である:$UU^*=U^*U=I$.
        \item 一般の作用素$V\in B(H)$が可逆でかつ$V^{-1}=V^*$を満たすなら,$V$はユニタリである.
    \end{enumerate}
\end{lemma}
\begin{Proof}\mbox{}
    \begin{enumerate}
        \item 極化恒等式\ref{prop-polarization-identity}より,等長写像ならば内積を保つこと\ref{prop-characterization-of-isometry}に注意.
        \item $(Ux|Uy)=(x|U^*Uy)=(x|y)$より,補題\ref{lemma-correspondence-between-sesquilinearform-and-operator}の全単射から,$U^*U=I$.
        \item 明らか.
        \item $\norm{Ux}^2=(U^*Ux|x)=\norm{x}^2$より等長.可逆なら全写.
    \end{enumerate}
\end{Proof}

\begin{definition}[unitary equivalent]\label{def-unitary-equivalent}
    2つの作用素$S,T\in B(H)$について$\exists_{U\in U(H)}\;S=UTU^*$を満たすとき,これらは\textbf{ユニタリー同値}であるという.
\end{definition}

\begin{lemma}
    ユニタリー同値は,作用素ノルム,自己共役性,正規性,対角化可能性,ユニタリー性を保つ.
\end{lemma}

\subsection{部分等長作用素と極分解}

\begin{tcolorbox}[colframe=ForestGreen, colback=ForestGreen!10!white,breakable,colbacktitle=ForestGreen!40!white,coltitle=black,fonttitle=\bfseries\sffamily,
title=]
    自己共役作用素は実数,正作用素は正実数と見れるのであった.
    $C^*$-代数の元の標準形がほしい.
    $\forall_{\lambda\in\C}\;\lambda=\abs{\lambda}e^{i\theta}$に対応する作用素の概念は,$\abs{A}:=(A^*A)^{1/2}$と表せるが,$e^{i\theta}$に対応する作用素のクラスは,新たに用意する必要がある.
\end{tcolorbox}

\begin{definition}[partial isometry, initial subspace, final subspace]
    作用素$U\in B(H)$が\textbf{部分等長作用素}であるとは,閉部分集合$X=(\Ker U)^\perp=\oo{\Im U^*}$が存在して,制限$U|_X$は等長写像で,$U|_{X^\perp}=0$を満たすことをいう.
    $X$を\textbf{初期部分空間},$\Im U$を\textbf{最終部分空間}という.
\end{definition}
\begin{remarks}
    $\Ker U=X^\perp$は閉であるから,$X=(\Ker U)^\perp=\oo{\Im U^*}$は必然的に閉である.
\end{remarks}
\begin{lemma}[部分等長写像の特徴付け]\mbox{}
    \begin{enumerate}
        \item $P:=U^*U$とおくと,$\forall_{x\in X}\;Px=x,\;\forall_{x^\perp\in X^\perp}\;Px^\perp=0$.すなわち,$P$は直交射影である:$P^2=P\land P^*=P$.
        \item (1)は部分等長作用素を特徴付ける.すなわち,作用素$U\in B(H)$について$U^*U$が直交射影となるとき,$U,U^*$は部分等長作用素である.$U^*$を$U$の\textbf{部分逆}(partial inverse)という.
    \end{enumerate}
    結局,次の6条件は同値である.
    \begin{enumerate}
        \item $U$は部分等長作用素である.
        \item $U^*$は部分等長作用素である.
        \item $U^*U$は直交射影である.
        \item $UU^*$は直交射影である.
        \item $UU^*U=U$.
        \item $U^*UU^*=U^*$.
    \end{enumerate}
\end{lemma}
\begin{Proof}\mbox{}
    \begin{enumerate}
        \item $x^\perp\in X^\perp=\Ker U$に対して,$U^*Ux=U^*0=0$は明らか.
        等長写像の特徴付け\ref{lemma-characterization-of-unitary-operator}より,$X$上で$U^*U=I$である.
        \item $X:=\Im U^*U$とおくと,射影はこの上で恒等だから,$\forall_{x\in X}\;\norm{Ux}^2=(Ux|Ux)=(U^*Ux|x)=(x|x)=\norm{x}^2$より,$X$上で等長.$X^\perp=(\Im U^*U)=\Ker U^*U$より,$\forall_{x\in X^\perp}\;(Ux|Ux)=(U^*Ux|x)=(0|x)=0$.
        
        一般に,$U$が部分等長写像であるとき,$U^*$も部分等長写像である.
        $U(I-P)=0$である.よって,$(UU^*)^2=UU^*UU^*=UPU^*=UU^*$から,$UU^*$も直交射影である.
        よって,前述の議論を繰り返せば良い.
    \end{enumerate}
\end{Proof}
\begin{remarks}
    部分等長写像は,$0$と等長写像とに分解できる作用素をいう.
    等長写像はepi:$U^*U=I$として特徴付けられるから,射影の言葉で部分等長写像が特徴付けられることは自然である.
\end{remarks}

\begin{example}[unilateral shift operator]
    可分Hilbert空間$H$上のシフト作用素\ref{operator-unilateral-shift}
    \[S\paren{\sum\al_ne_n}=\sum\al_ne_{n+1}\]
    は等長作用素だが,ユニタリーではない(すなわち全射でない)例となっている.
    また$S,S^*$は部分等長作用素である.
    $S:H\to\{e_1\}^\perp$は等長作用素であり,$S^*S=I$(たしかに射影である).
    一方で,$SS^*=P_{\Brace{e_1}^\perp}$であり,$S^*$は$\{e_1\}^\perp$上では$H$に値を取る部分等長写像である.
\end{example}

\begin{theorem}[polar decomposition (von Neumann)]\mbox{}\label{thm-polar-decomposition}
    \begin{enumerate}
        \item 任意の作用素$T\in B(H)$に対して,ただ一つの半正定値作用素$\abs{T}:=(T^*T)^{1/2}\in B(H)$が存在して,$\forall_{x\in H}\;\norm{Tx}=\norm{\abs{T}x}$を満たす.特に,$\Ker T=\Ker\abs{T}$.
        \item ただ一つの部分等長写像$U$が存在して,$\Ker U=\Ker T,U\abs{T}=T$を満たす.
        \item $U^*U\abs{T}=U^*T=\abs{T},\;UU^*T=T$が成り立つ.
    \end{enumerate}
\end{theorem}
\begin{Proof}\mbox{}
    \begin{enumerate}
        \item 一意性を示せば良い.
        ある正作用素$S\ge0$が$\forall_{x\in H}\;\norm{Tx}=\norm{Sxx}$を満たすならば,$(S^2x|x)=\norm{Sx}^2=\norm{Tx}^2=(T^*Tx|x)$である.これより,$T^*T$も正だから,$S^2=T^*T$が従う\ref{cor-nondegeneratedness-of-semi-inner-product}.
        二乗根の一意性より,$S=(T^*T)^{1/2}$である.
        \item $T,\abs{T}$は任意の$x\in H$に対して像のノルムが等しいから,核も等しい.よって,$\Ker T=\Ker\abs{T}=(\Im\abs{T})^\perp$\ref{prop-Ker-of-adjoint-operator}.
        \begin{description}
            \item[存在] まず,任意の$y=\abs{T}x\in\Im\abs{T}$に対して,$U_0y=U_0\abs{T}x:=Tx$と定めると,これは等長作用素$U_0:\Im\abs{T}\to\Im T$である:$\norm{U_0y}=\norm{Tx}=\norm{\abs{T}x}=\norm{y}$.
            連続延長\ref{prop-extension-of-operator-on-dense-subset}により,等長な延長$\o{U_0}:\oo{\Im\abs{T}}\to\oo{\Im T}$が得られる.
            さらに,$\Ker T=(\Im\abs{T})^\perp$上では零とすることで,$U\abs{T}=T$を満たす部分等長写像$U$を得る.これはたしかに$\Ker U=\Ker T$を満たす.
            \item[一意性] 
            $V$を,$\Ker V=\Ker T,V\abs{T}=T$を満たす部分等長写像とする.
            すると,まず$U=V\on\;\oo{\Im\abs{T}}$が必要で,その直交補空間ではいずれも$0$であるから,$U=V$が必要.
        \end{description}
        \item $U^*U$は構成から$\oo{\Im\abs{T}}$への直交射影だから,$U^*U\abs{T}=\abs{T}$.
        $T=U\abs{T}$より,$U^*T=\abs{T}$.
        この両辺に$U$を乗じると$UU^*T=U\abs{T}=T$.
    \end{enumerate}
\end{Proof}

\begin{corollary}[随伴作用素の極分解]
    $T=U\abs{T}$を極分解とすると,$T^*=\abs{T}U^*=U^*(U\abs{T}U^*)$より,符号は$U^*$で,絶対値は$U\abs{T}U^*$である.
\end{corollary}
\begin{remark}
    和,積の極分解には殆ど法則がない.
\end{remark}

\begin{corollary}[Schmidt展開]
    $H$を可分,$T\in B_0(H)$とする.
    \begin{enumerate}
        \item 半正定値作用素を$0\le A\in B_0(H)$と,等長作用を$U:\o{R(T)}\to H$と取れて,$T=UA$.
        \item $H$の正規直交系$\{\phi_n\},\{\psi_n\}$と正数$s_n\to0$が存在して,
        \[Tu=\sum_{n\in\N}s_n(u|\psi_n)\phi_n\]
        と書ける.$\{s_n\}$を$T$の\textbf{特異数}という.
    \end{enumerate}
\end{corollary}

\begin{theorem}
    $T\in B(H)$について,次の2条件は同値.
    \begin{enumerate}
        \item あるユニタリ作用素$U$について,$T=U\abs{T}$と極分解される.
        \item $T,T^*$の核が同次元である:$\Ker T\simeq_\Hilb\Ker T^*$.
    \end{enumerate}
\end{theorem}

\begin{corollary}\label{cor-polar-decomposition-of-invertible-operators}
    $T\in B(H)$が可逆であるとき,極分解に含まれる部分等長写像$U$はユニタリである.
\end{corollary}
\begin{Proof}
    ここでは定理に依らず,単独で証明する.

    $T=U\abs{T}$を極分解とすると,$\Ker U=\Ker T$で,$\Im U=\oo{\Im T}$.
    $T$が可逆のとき$\Ker U=0,\Im U=H$が従うから,部分等長写像$U$はユニタリである.
\end{Proof}

\begin{corollary}
    $T\in B(H)$を正規とする.
    このとき,$T,T^*,\abs{T}$と可換なユニタリ作用素$W$が存在して,$T=W\abs{T}$と極分解できる.
\end{corollary}

\subsection{Russo-Dye-Gardner定理}

\begin{tcolorbox}[colframe=ForestGreen, colback=ForestGreen!10!white,breakable,colbacktitle=ForestGreen!40!white,coltitle=black,fonttitle=\bfseries\sffamily,
title=ユニタリ作用素への分解]
    極分解をしたなら,$\C\simeq\R^2$に対応する分解も欲しくなる.これは$B\cap B(H)_\sa$上のみで存在する.
    結局,$\norm{S}\le 1$を満たす有界作用素は,ユニタリ作用素の凸結合(特に平均)として表せる.
    さらに,$B(H)_\sa$は$B(H)$を生成するから,任意の作用素はユニタリ作用素の線型結合である.
\end{tcolorbox}

\begin{lemma}[単位球内の自己共役作用素の表示]
    $T=T^*$かつ$\norm{T}\le 1$ならば,作用素$U:=T+i(I-T^2)^{1/2}$はユニタリであり,$T=\frac{1}{2}(U+U^*)$と表せる.
\end{lemma}
\begin{Proof}
    $I-T\ge0$かつ$I+T\ge0$で互いに可換であるから,$I-T^2\ge0$である\ref{lemma-positive-closed-cone-of-positive-operator}(2),\ref{prop-order-of-self-adjoint-operator}(3).
    よってたしかに二乗根を持つ.
    $UU^*=U^*U=I$がわかるから,$U$はユニタリである.
    正作用素は自己共役であることに注意すれば,$T=\frac{1}{2}(U+U^*)$は明らか.
\end{Proof}

\begin{lemma}
    $S\in B(H)$かつ$\norm{S}<1$ならば,任意のユニタリ作用素$U$に対して,ユニタリ作用素$U_1,V_1$が存在して,$S+U=U_1+V_1$を満たす.
\end{lemma}
\begin{Proof}
    $U=I$の場合について示せば,両辺に$U^*$を左から乗ずることで一般の結論を得る.

    $\norm{S}=\norm{S^*}<1$の仮定より,三角不等式$\norm{x}-\norm{y}\le\norm{x-y}$から,
    \[\norm{Sx+x}\ge\norm{x}-\norm{Sx}\ge(1-\norm{S})\norm{x}\]
    より,$I+S,I+S^*$はいずれも$0$に対して有界である.よって,作用素の可逆性の特徴付け\ref{prop-characterization-of-invertibleness-of-operator}より,
    $I+S$は可逆である.
    よって,極分解$I+S=V\abs{I+S}$において,$V$はユニタリである\ref{cor-polar-decomposition-of-invertible-operators}.
    $\norm{I+S}\le\norm{I}+\norm{S}<2$より,補題から,正作用素$\abs{I+S}$はあるユニタリ作用素$W$を用いて$\abs{I+S}=W+W^*$と表せる.
    以上より,
    \[S+I=V(W+W^*)=VW+VW^*.\]
\end{Proof}

\begin{proposition}[Russo-Dye-Gardner theorem (66)]
    $T\in B(H)$かつ$\exists_{n>2}\norm{T}<1-\frac{2}{n}$ならば,ユニタリ作用素$U_1,\cdots,U_n$が存在して,
    \[T=\frac{1}{n}(U_1+U_2+\cdots+U_n).\]
\end{proposition}
\begin{Proof}
    $S:=\frac{1}{n-1}(nT-I)$とおくと,$nT=(n-1)S+IT$で,
    \[\norm{S}\le\frac{1}{n-1}(n\norm{T}-1)<\frac{1}{n-1}(n-3)<1\]
    を満たすから,補題を$n-1$回繰り返し適用すると,
    \begin{align*}
        nT&=(n-1)S+I=(n-2)S+(S+I)=(n-2)S+(V_1+U_1)\\
        &=(n-3)S+(S+V_1)+U_1=(n-3)S+(V_2+U_2)+U_1\\
        &=(n-4)S+(S+V_2)+U_2+U_1\\
        &=(n-4)S+(V_3+U_3)+U_2+U_1=\cdots\\
        &=(S+V_{n-2})+U_{n-2}+\cdots+U_1=U_n+U_{n-1}+U_{n-2}+\cdots+U_1.
    \end{align*}
\end{Proof}
\begin{remarks}
    $B(H)$の開球の元は,ユニタリ作用素の凸結合(特に平均)で表せる.
    球面上の元はこの限りではなく,シフト作用素が反例である.
\end{remarks}

\begin{corollary}[同じ論文で紹介されている系]
    $C^*$-代数$A$のユニタリ元の全体を$U(A)$で表す.ノルム空間$B$への線型写像$f:A\to B$について,$\sup_{U\in U(A)}\norm{f(U)}$.
\end{corollary}
\begin{remarks}
    the norm of an operator can be calculated using only the unitary elements of the algebra. \footnote{\url{https://en.wikipedia.org/wiki/Russo\%E2\%80\%93Dye_theorem}}
\end{remarks}

\subsection{数域半径}

\begin{tcolorbox}[colframe=ForestGreen, colback=ForestGreen!10!white,breakable,colbacktitle=ForestGreen!40!white,coltitle=black,fonttitle=\bfseries\sffamily,
title=]
    作用素ノルムと$(T-|-)$なる半内積(の制限)との関係の観察\ref{prop-operator-norm-of-self-adjoint-operator}から,
    $C^*$-代数の手法に数域$\Im(T-|-)$を考えることは自然である(参考:\ref{lemma-numerical-range}).
    その半径は作用素ノルムと同値なノルムを定め,正規作用素については作用素ノルムと同じノルムを定める.
    さらにスペクトル半径と数域半径が一致するとき,spectraloid作用素といい,正規作用素はこれにも当てはまる.
\end{tcolorbox}

\begin{definition}[numerical radius]
    $H$を複素Hilbert空間,$T\in B(H)$とする.$T$の\textbf{数域半径}とは,
    \[\nnorm{T}=\sup\Brace{\abs{(Tx|x)}\in\R_+\mid x\in H,\norm{x}\le 1}\]
    をいう.なお,$T\in B(H)_\sa$のとき,作用素ノルムは
    \[\norm{T}=\sup\Brace{\abs{(Tx|x)}\in\R_+\mid x\in H,\norm{x}=1}\]
    と表せること\ref{prop-operator-norm-of-self-adjoint-operator}に注意.
    一般に命題が成り立つ.
\end{definition}
\begin{remark}[numerical range]
    $W:=\Brace{(Tx|x)\in\C\mid x\in H,\norm{x}= 1}=\Brace{\frac{x^*Tx}{x^*x}\in\C\;\middle|\;x\in H,x\ne 0}$を数域という.
\end{remark}

\begin{notation}[自己共役部分]
    $\Re(T):=\frac{1}{2}(T+T^*)$.$T$が自己共役のとき,これは$T$自身に一致する.
\end{notation}

\begin{proposition}[数域半径は同値なノルムを定める]
    $\nnorm{\cdot}$は$B(H)$上のノルムで,$\forall_{T\in B(H)}\;\frac{1}{2}\norm{T}\le\nnorm{T}\le\norm{T}$を満たす.
\end{proposition}
\begin{Proof}
    $\nnorm{T}$の斉次性と劣加法性は明らかで,非退化性は後半の主張から従う.また,Cauchy-Schwarzの不等式より$\nnorm{T}\le\norm{T}$もわかる.
    よって,$\norm{T}\le 2\nnorm{T}$を示せば良い.
    任意の単位ベクトル$x,y\in H$について,極化恒等式と中線定理より
    \begin{align*}
        4\abs{(Tx|y)}&=\Abs{\sum^3_{k=0}i^k(T(x+i^ky)|x+i^ky)}\\
        &\le\nnorm{T}(\norm{x+y}^2+\norm{x-y}^2+\norm{x+iy}^2+\norm{x-iy}^2)\\
        &=\nnorm{T}2(\norm{x}^2+\norm{y}^2+\norm{x}^2+\norm{iy}^2)=8\nnorm{T}
    \end{align*}
    これで,半双線型形式のノルムについて$\norm{(T-|-)}\le2\nnorm{T}$を得たが,等長同型\ref{lemma-correspondence-between-sesquilinearform-and-operator}より結論を得る.
\end{Proof}
\begin{remarks}
    また,一般の線型写像について,有界であることと$\nnorm{T}<\infty$になることは同値になる.
\end{remarks}

\begin{proposition}
    $\forall_{T\in B(H)}\;\nnorm{T}=\max\Brace{\norm{\Re(\theta T)}\in\R_+\mid\theta\in\C,\abs{\theta}=1}$.
\end{proposition}

\begin{proposition}
    任意の$T\in B(H)$について,
    \begin{enumerate}
        \item $\nnorm{T^2}\le\nnorm{T}^2$.
        \item $T$が正規であるとき,等号成立.
    \end{enumerate}
\end{proposition}

\subsection{数域}

\begin{tcolorbox}[colframe=ForestGreen, colback=ForestGreen!10!white,breakable,colbacktitle=ForestGreen!40!white,coltitle=black,fonttitle=\bfseries\sffamily,
title=]
    自己共役作用素$T$に対して,$R(T,x):=\frac{(Tx|x)}{(x|x)}$をRayleigh商といい\ref{prop-operator-norm-of-self-adjoint-operator},固有値の数値計算に利用される.\footnote{\url{https://ja.wikipedia.org/wiki/レイリー商}}
    そしてこの値域を数域という.
    $T$がエルミート行列$M$のとき数域は実数に含まれ,$R(M,x)\in[\lambda_\mathrm{min},\lambda_\mathrm{max}]$となる.
\end{tcolorbox}

\begin{proposition}[Hausdorff-Toeplitz]
    $T\in B(H)$について,数域$W(T)$は凸集合である.$H$が有限次元のとき,$W(T)$はコンパクトである.
\end{proposition}

\begin{proposition}
    $T\in B(H)$を正規作用素とする.数域の閉包$\oo{W(T)}$とスペクトルの閉凸包$\oo{\Conv(\sigma(T))}$とは等しい.また,$\oo{W(T)}$の極点$x$について,次の2条件は同値.
    \begin{enumerate}
        \item $x\in W(T)$.
        \item $x$は$T$の固有値である.
    \end{enumerate}
\end{proposition}
\begin{remark}
    工学では,この定理を利用して,$T$の固有値を$W(T)$の形から推定する.\footnote{\url{https://en.wikipedia.org/wiki/Numerical_range}}
\end{remark}

\subsection{作用素の例}

\begin{tcolorbox}[colframe=ForestGreen, colback=ForestGreen!10!white,breakable,colbacktitle=ForestGreen!40!white,coltitle=black,fonttitle=\bfseries\sffamily,
title=]
    乗算作用素は対角行列の概念を一般化する.任意のHilbert空間上の自己共役作用素は,$L^2$空間上のある乗算作用素とユニタリ同値である,という主張がスペクトル定理である.
    核が超関数になることも許せば,全ての線型作用素は積分作用素として表せる,という主張がSchwartzの核定理である.
    そのほか,合成作用素と転送作用素の随伴,シフト作用素=平行移動作用素(時系列解析ではラグ作用素)などもある.

    また,作用素の随伴は,「転置」の要素は隠れて,複素共役を取ることに似る.
\end{tcolorbox}

\begin{theorem}[multiplication operator and its symbol]\label{operator-multiplication}
    $(X,\Om,\mu)$を$\sigma$-有限な測度空間とし,$H:=L^2(X,\Om,\mu)=:L^2(\mu)$とする.任意の$\varphi\in L^\infty(\mu)$に対して,これと積を取る写像
    \[\xymatrix@R-2pc{
        M_\varphi:L^2(\mu)\ar[r]&L^2(\mu)\\
        \rotatebox[origin=c]{90}{$\in$}&\rotatebox[origin=c]{90}{$\in$}\\
        f\ar@{|->}[r]&\varphi\cdot f
    }\]
    は,有界で$M_\varphi\in B(L^2(\mu))$,等長である:$\norm{M_\varphi}=\norm{\varphi}_\infty$.
    $\varphi$を\textbf{乗算作用素}$M_\varphi$の\textbf{記号}という.

    乗算作用素は対角行列の一般化であり,スペクトル定理によると,Hilbert空間上の自己共役作用素は,$L^2$-空間上の乗算作用素とユニタリ同値になる.
\end{theorem}

\begin{theorem}[Fredholm integral operator / transform, kernel]\label{operator-integral-transformation}
    $(X,\Om,\mu)$を測度空間,$k:X\times X\to\bF$を次を満たす$\Om\times\Om$-可測関数とする:
    \begin{align*}
        \int_X\abs{k(x,y)}d\mu(y)\le c_1,&\ae[\mu],&\int_X\abs{k(x,y)}d\mu(x)\le c_2,&\ae[\mu].
    \end{align*}
    このとき,写像
    \[\xymatrix@R-2pc{
        K:L^2(\mu)\ar[r]&L^2(\mu)\\
        \rotatebox[origin=c]{90}{$\in$}&\rotatebox[origin=c]{90}{$\in$}\\
        f\ar@{|->}[r]&Kf(x):=\int_Xk(x,y)f(y)d\mu(y)
    }\]
    は有界線型作用素で,ノルムは$\norm{K}\le(c_1c_2)^{1/2}$を満たす.
\end{theorem}

\begin{example}[Volterra integral operator, Volterra operator]\label{operator-Volterra}
    Voltera積分作用素は不定積分に関する積分変換全般を指すが,Volterra作用素と言った場合は,冪零作用素の一般化概念をいう.
    前者は後者の例となっている.
    \begin{enumerate}
        \item Fredholm積分作用素$K$が定積分を定める積分作用素ならば,Volterra積分作用素は不定積分を定める積分作用素である.\footnote{ヴォルテラ積分方程式は、人口学や、粘弾性物質の研究、保険数学に現れる再生方程式などへと応用されている。}
        \item $k:[0,1]\times[0,1]\to2\mono\R$を集合$\{(x,y)\in[0,1]\times[0,1]\mid y<x\}$の特性関数とする.
        これを核とする積分作用素$V:L^2(0,1)\to L^2(0,1);f\mapsto Vf(x)=\int^1_0k(x,y)f(y)dy=\int^x_0f(y)dy$を\textbf{Volterra作用素}という.
        別の抽象的な定義としては,スペクトル半径が$0$なコンパクト作用素をVoltarra作用素と定める.
    \end{enumerate}
    Volterra積分作用素はコンパクトであるが,固有値を持たず,擬冪零$\Sp(V)=\{0\}$である.
\end{example}

\begin{example}[Hilbert-Schmidt integral operator, Hilbert-Schmidt operator]\label{exp-Hilbert-Schmidt}
    積分核が2乗可積分である$k\in L^2(\Om\times\Om;\C)$とき,この積分作用素を\textbf{Hilbert-Schmidt積分作用素}といい,コンパクト作用素になる\ref{def-trace-class-Hilbert-Schmidt}.
    2乗可積分な積分核を\textbf{Hilbert-Schmidt核}という.
    $k$が自己共役ならば$K$も自己共役で,従ってスペクトル定理が適用できる.
    なお,これはHilbert-Schmidt作用素\ref{def-trace-class-Hilbert-Schmidt}の例である.
\end{example}

\begin{remark}
    \begin{align*}
        \int^b_aK(x,y)f(y)dy-\lambda f(x)&=g(x)\\
        \int^x_aK(x,y)f(y)dy-\lambda f(x)&=g(x)
    \end{align*}
    をそれぞれ,Fredholm型,Volterra型積分方程式という.
\end{remark}

\begin{example}[matrix multiplication]
    行列乗算は,離散空間上での積分変換と捉えられる.
    これが行列という形式の普遍性を説明しているのではなかろうか?
    線形代数と微分積分の概念はここに交錯する.
    積分変換はより一般に多項式関手(polynomial functor)の特別な場合で,多項式関手とは,多項式概念の関手化である.
    Volterra作用素は,不定積分概念の作用素化であろうか.
\end{example}

\begin{example}[unilateral shift]\label{operator-unilateral-shift}
    \[\xymatrix@R-2pc{
        S:l^2\ar[r]&l^2\\
        \rotatebox[origin=c]{90}{$\in$}&\rotatebox[origin=c]{90}{$\in$}\\
        (\al_1,\al_2,\cdots)\ar@{|->}[r]&(0,\al_1,\al_2,\cdots)
    }\]
    は全射でない等長作用素となる(すなわち,ノルム1の有界線型写像).
    これをシフト作用素という.
\end{example}


\section{コンパクト作用素}

\begin{tcolorbox}[colframe=ForestGreen, colback=ForestGreen!10!white,breakable,colbacktitle=ForestGreen!40!white,coltitle=black,fonttitle=\bfseries\sffamily,
title=]
    ($X$が局所コンパクトハウスドルフ空間である時,)
    有界連続関数の中でコンパクト台を持つもの$C_c(X)\subset C_b(X)$の関係と,有界作用素の中で有限ランクを持つもの$B_f(H)\subset B(H)$の関係は非常に似ている.
    These classes describe local phenomena on $H$ and on $X$.\cite{Pedersen}
    そこで,$C_c(X)$の完備化として得た$C_0(X)$に対応するクラス"$B_0(H)$"を,作用素論でも構成することを考える.
    このクラスはある種の有限的な性質を持ち,$B(H)$の閉イデアルをなす.

    このクラスには,行列はもちろん,積分作用素が含まれる.
    そこで,この順に,対角化の一般理論と,積分方程式の一般理論を調べていく.
\end{tcolorbox}

\subsection{定義と特徴付け}

\begin{tcolorbox}[colframe=ForestGreen, colback=ForestGreen!10!white,breakable,colbacktitle=ForestGreen!40!white,coltitle=black,fonttitle=\bfseries\sffamily,
title=]
    積分作用素も例に含むが,より行列の類似的性質が見やすい狭いクラスの作用素を定義する.
    作用素に「有界集合は相対コンパクト集合に写す」という有限性条件を課す.
    Hilbertは「完全連続作用素」と呼んだ.
\end{tcolorbox}

\begin{definition}[finite rank]
    Hilbert空間$H$上の作用素$T:H\to H$について,
    \begin{enumerate}
        \item $T$が\textbf{有限階数}または\textbf{退化}であるとは,$\Im T$が$H$の有限次元部分空間であることをいう(したがって特に閉\ref{prop-finite-subspaces}).
        \item 有限階数な有界作用素全体$B_f(H)$は,$B(H)$内の部分代数であり,かつイデアルである.
        \item $B_f(H)$はイデアルとして自己共役である:$(B_f(H))^*=B_f(H)$.
    \end{enumerate}
\end{definition}
\begin{Proof}\mbox{}
    \begin{enumerate}\setcounter{enumi}{1}
        \item 任意の$S,T\in B_f(H),U\in B(H),a\in\bF$について,$aS,S+T,ST\in B_f(H)$である:$\Im aS=\Im S,\Im(S+T)\subset\Im S\oplus\Im T,\Im(SU)\subset\Im S,\dim\Im(US)\le\dim\Im(S)$.
        \item $T\in B_f(H)\Leftrightarrow T^*\in B_f(H)$を示す.直交分解$H=\Im T\oplus\Im T^\perp$を随伴によって表現することより,$H=\Im T\oplus\Ker T^*$の関係がある.よって,$\Im T^*=T^*(\Im T)$より,$T^*$も有限ランクである.
    \end{enumerate}
\end{Proof}

\begin{lemma}[近似的単位元の構成]
    $B_f(H)$には射影からなるネット$(P_\lambda)_{\lambda\in\Lambda}$が存在し,$\forall_{x\in H}\;\norm{P_\lambda x-x}\to 0$を満たす.
\end{lemma}
\begin{Proof}
    $H$の正規直交基底${(e_j)}_{j\in J}$を取り,$\Lambda:=\Brace{\lambda\in P(J)\mid \abs{\lambda}<\infty}$からのネット$(P_\lambda:=\pr\{\brac{e_j}_{j\in J})$を考える.
    すると,$\forall_{\lambda\in\Lambda}\;\dim(\Im(P_\lambda))<\infty$より確かに$B_f(H)$上のネットなっている.
    任意の$x=\sum_{j\in J}\al_je_j\in H$について,
    系\ref{cor-well-definedness-of-Bessel's-identity}より,任意の$0\in\cointerval{0,\infty}$の開近傍の基本系の元$\cointerval{0,\ep}$に対して,ある有限集合$J$が存在して,$\norm{P_\lambda x-x}^2=\sum_{j\in\Lmd}\abs{\al_j}^2<\ep$(Parseval's identity)が成り立つ.
    よって,ネットとして,$\norm{P_\lmd x-x}$は$0$に収束する.
\end{Proof}

\begin{theorem}[stability of compact operator]
    $T\in B(H)$について,次の5条件は同値.
    \begin{enumerate}
        \item $T\in\dbloverline{B_f(H)}$.
        \item $T|_B:B\to H$は弱-ノルム連続な関数である.\footnote{すなわち,列$(f_n)$が$f$に弱収束するならば,像$(Kf_n)$は$Kf$にノルム収束する.}
        \item $T(B)$は$H$でコンパクトである.
        \item $\dbloverline{T(B)}$は$H$でコンパクトである.
        \item $B$上の任意のネットは,$T$での像が$H$上で強収束するような部分ネットを持つ.\footnote{$H$は距離空間だから,すなわち,$H$の任意の有界列$(f_n)$に対して,$(Kf_n)$は収束する部分列を持つ.}
        \item (Hilbert) 双線型形式$\Phi(f,g):=(Tf,g)$は弱連続である.\footnote{すなわち,弱収束列を保存する.}
    \end{enumerate}
\end{theorem}
\begin{Proof}\mbox{}
    \begin{description}
        \item[(1)$\Rightarrow$(2)] 
        $x$に弱収束する$B$のネット$(x_\lambda)_{\lambda\in\Lambda}$を任意に取る.任意の$\ep>0$に対して,仮定より,$S\in B_f(H)$が存在して$\norm{S-T}<\ep/3$を満たすから,
        \begin{align*}
            \norm{Tx_\lambda-Tx}&=\norm{(T-S)x_\lambda-(T-S)x+Sx_\lambda-Sx}\\
            &\le2\norm{T-S}+\norm{Sx_\lambda-Sx}\\
            &\le\frac{2}{3}\ep+\norm{Sx_\lambda-Sx}.
        \end{align*}
        いま,任意の$B(H)$の元は弱-弱連続\ref{lemma-characterization-of-bounded-operator-on-Hilbert-space}だから,$Sx_\lambda$は$Sx$に弱収束する.
        $\Im S$は有限次元であるから,この正規直交基底を$\{e_1,\cdots,e_n\}$とおくと,各線型汎関数$(-|e_j)$は弱連続だから,
        ノルムと内積の関係性
        \[\norm{Sx_\lambda-Sx}^2=\sum_{j\in[n]}\abs{(S(x_\lambda-x)|e_j)}^2\to0\]
        より,ノルムについても収束する.
        以上より,$\norm{Tx_\lambda-Tx}<\ep$であるから,$T$は弱-ノルム連続である.
        \item[(2)$\Rightarrow$(3)]
        任意のHilbert空間は回帰的で,$H$の単位球$B$は弱コンパクト\ref{def-weak-topology-on-H(B)}であったから,$T(B)$はノルムコンパクトである.
        \item[(3)$\Rightarrow$(4)]
        一般に,任意のHausdorff空間のコンパクト集合は閉であるから,$T(B)=\oo{T(B)}$.
        \item[(4)$\Rightarrow$(5)]
        $T(B)$は相対コンパクトであるから,任意の$T(B)$のネットは収束する部分ネットを持つ.したがって,その像も収束する.
        \item[(5)$\Rightarrow$(1)]
        補題の通りの射影からなるネット$(P_\lambda)_{\lambda\in\Lambda}$を取る.
        これについて,$(P_\lambda T)$は$B_f(H)$のネットになるが,これが$T$にノルム収束することを示す.
        仮に収束しないと仮定すると,$\ep>0$が存在して,任意の$\lambda\in\Lambda$に対して単位ベクトル$x_\lambda$が存在して,$\norm{(P_\lambda T-T)x_\lambda}\ge\ep$.
        仮定より,ネット$(Tx_\lambda)$はある極限$y$にノルム収束すると仮定してよく,このとき補題より,
        \begin{align*}
            \ep&\le\norm{(I-P_\lambda)Tx_\lambda}\le\norm{(I-P_\lambda)(Tx_\lambda-y)}+\norm{(I-P_\lambda)y}\\
            &\le\norm{Tx_\lambda-y}+\norm{(I-P_\lambda)y}\to0.
        \end{align*}
        よって矛盾.
    \end{description}
\end{Proof}
\begin{remarks}
    $B_f(H)$の元の像は有限次元であるから,そこではノルム位相と弱位相が一致する.
    これが(1)の消息となる.
\end{remarks}

\begin{definition}[compact operator]
    定理の同値な条件を満たす作用素を\textbf{コンパクト作用素}という.
    無限遠で消えるため,コンパクト作用素の空間は$B_0(H)$と表すが,$K(H),C(H)$も一般的である.
\end{definition}

\begin{example}[Hilbert-Schmidt作用素は非自明な例]
    可分な$H$の正規直交基底$(e_n),(f_n)$を用いて,
    \[\norm{K}_{H.S.}^2:=\sum_{n,m\ge1}\abs{(Ke_n|f_m)}^2=\sum_{n,m\ge1}\abs{(K^*f_m,e_n)}^2\]
    と定めると,Parsevalの等式より,$\norm{K}^2_{H.S.}=\sum^\infty_{n=1}\norm{Ke_n}^2=\sum^\infty_{n=1}\norm{K^*f_m}^2$.
    このノルムが有限な作用素をHilbert-Schmidt作用素といい,これはコンパクト作用素である.
    実際,$K_nf:=\sum^n_{m=1}(Kf,e_m)e_m$とすると明らかに
    $B_0(H)$の列であるが,$K$はこのノルム極限として得る.
\end{example}

\subsection{コンパクト作用素の空間}

\begin{lemma}[コンパクト作用素の空間の描像]
    コンパクト作用素の空間$B_0(H)=\oo{B_f(H)}\subset B(H)$は
    \begin{enumerate}
        \item ノルム閉で自己共役なイデアルである.
        \item (1)の条件を満たすもので最小のものである.特に$H$が可分である場合は唯一の非自明な閉イデアルである.
        \item $H$が無限次元であるとき,非単位的な部分代数である:$I\notin B_0(H)$.が,その場合でも,有限ランクの射影からなる近似的単位元を持つ.
        \item $H$が可分であるとき,$B_f(H)$は$B_0(H)$の中で稠密である.\footnote{これは$C_c(X)$と$C_0(X)$のアナロジーが完成する.}
    \end{enumerate}
\end{lemma}
\begin{Proof}\mbox{}
    \begin{enumerate}
        \item 自己共役性が保たれる.
        \item 
    \end{enumerate}
\end{Proof}

\begin{proposition}
    任意のBanach空間$X,Y$について,
    \begin{enumerate}
        \item $B_0(X)=B(X)$であることと,$X$が有限次元であることとは同値.
        \item $B_0(X,Y)$は$B(X,Y)$の閉部分空間である.
        \item $u\in B_0(X,Y)\Rightarrow u^*\in B_0(Y^*,X^*)$.
    \end{enumerate}
\end{proposition}

\subsection{正規なコンパクト作用素に関するスペクトル定理}

\begin{tcolorbox}[colframe=ForestGreen, colback=ForestGreen!10!white,breakable,colbacktitle=ForestGreen!40!white,coltitle=black,fonttitle=\bfseries\sffamily,
title=]
    作用素の対角化とは,直交射影による表示としたのであった\ref{remarks-diagonalizability}.
    ちょうど$B_0(H)$には\underline{直交}射影からなる近似的単位元が存在することを確認したが,
    正規コンパクト作用素は,固有値の言葉で特徴付けることが出来る.
\end{tcolorbox}

\begin{lemma}
    $T\in B(H)$が対角化可能であるとき,
    \begin{enumerate}
        \item $T$はコンパクトである.
        \item 正規直交基底$(e_j)_{j\in J}$に対応する固有値$(\lambda_j)_{j\in J}$は$c_0(J)$の元である.
    \end{enumerate}
\end{lemma}

\begin{lemma}
    $x\in H$を正規作用素$T\in B(H)$の固有ベクトルとし,対応する固有値を$\lambda\in\C$とする.
    \begin{enumerate}
        \item $x$は$T^*$の固有ベクトルであり,対応する固有値は$\o{\lambda}$である.
        \item $T$の他の固有値に対応する固有ベクトルは$x$に直交する.
    \end{enumerate}
\end{lemma}

\begin{lemma}
    複素Hilbert空間$H$上の正規なコンパクト作用素$T$は,$\abs{\lambda}=\norm{T}$を満たす固有値$\lambda\in\C$を持つ.
\end{lemma}

\begin{theorem}[正規コンパクト作用素の特徴付け]\label{thm-Spectral-Theorem-for-normal-compact-operator}
    $H$を複素Hilbert空間,$T\in B(H)$とする.次の2条件は同値.
    \begin{enumerate}
        \item $T$は正規なコンパクト作用素である.
        \item $T$は対角化可能で,その固有値は無限遠で消える.\footnote{従って,$H$には$T$の固有ベクトルからなる正規直交基底を取れることとも同値.}
    \end{enumerate}
\end{theorem}

\begin{notation}[ベクトルの定める直交射影]
    $x,y\in H$に対して,$x\odot y\in B(H)$を$(x\odot y)z=(z|y)x\;(z\in H)$で定まる$\Im x\odot y=\bF x$を満たす階数$1$の作用素とする.
    この構成は半双線型写像$H\times H\to B_f(H);(e_i,e_j)\mapsto e_i\odot e_j$を定める.
    すると,$\norm{e}=1$について,$e\odot e$は$\C e$への直交射影である.
\end{notation}

\begin{remarks}[正規なコンパクト作用素についてのスペクトル定理としての消息]
    正規なコンパクト作用素は,ある正規直交基底$(e_j)_{j\in J}$が存在して,ノルム収束する級数$T=\sum_{j\in J}\lambda_je_j\odot e_j$と表せるクラスである.
    これは,$J_0:=\Brace{j\in J\mid\lambda_j\ne0}$が有限集合であるか,係数列$(\lambda_j)_{j\in J_0}$は$0$に収束するかのいずれかであるからである.
    このとき,$T$のスペクトルは集積点$0$を併せて,$\Sp(T)=\{\lambda_j\}_{j\in J}\cup\{0\}$と表せる.
    ここで,$*$-代数の等長準同型
    \[\xymatrix@R-2pc{
        C(\Sp(T))\ar[r]&B(H)\\
        \rotatebox[origin=c]{90}{$\in$}&\rotatebox[origin=c]{90}{$\in$}\\
        f\ar@{|->}[r]&f(T):=\sum_{j\in J}f(\lambda_j)e_j\odot e_j
    }\]
    を考えると,$f(T)\in B_0(H)$であることは,$f(0)=0$であることに同値であることがわかる.
    特に$f(z)=\sum\al_{nm}z^n\o{z}^m$と表せる場合を考えると,$f(T)=\sum\al_{nm}T^nT^{*m}$となる.
\end{remarks}

\subsection{Fredholm作用素}

\begin{tcolorbox}[colframe=ForestGreen, colback=ForestGreen!10!white,breakable,colbacktitle=ForestGreen!40!white,coltitle=black,fonttitle=\bfseries\sffamily,
title=Hilbert空間の同値]
    コンパクト作用素の違い(=コンパクトな摂動)を無視して,作用素を見た世界を
    Calkin代数$B(H)\epi B(H)/B_0(H)$といい,この上に同じ像を定める作用素は,同値類を定める.
    これは応用上重要で,
    この同値類についての不変量が重要な意味を持つこととなる.

    そこで,「コンパクト代数の分だけ緩めて可逆」という準可逆な作用素をFredholm作用素とし,これを調べる.
    これは,核と余核が高々有限次元で,像が閉であることに同値.
\end{tcolorbox}

\begin{definition}[Calkin algebra]
    $B_0(H)$は閉イデアルであるから,商$B(H)/B_0(H)$は商ノルムについてBanach代数を定める.
    これを\textbf{Calkin代数}という.\footnote{Wikipediaによると$H$が可分である場合のみを指す,指数理論と作用素環論の対象である.}
\end{definition}

\begin{lemma}
    Calkin代数$B(H)/B_0(H)$は$C^*$-代数である.
\end{lemma}

\begin{definition}[compact perturbation]
    $S,T\in B(H)$が$S-T\in B_0(H)$を満たすとき,片方はもう片方の\textbf{コンパクトな摂動}であるという.
    すなわち,Calkin代数上に同じ像を定めることをいう.
\end{definition}

\begin{proposition}[Atkinson's theorem]
    $T\in B(H)$について,次の4条件は同値.これらを満たす作用素を\textbf{Fredholm作用素}といい,その全体を$F(H)$で表す.
    \begin{enumerate}
        \item ある作用素$S\in B(H)$が一意的に存在して$\Ker S=\Ker T^*$かつ$\Ker S^*=\Ker T$が成り立ち,$ST,TS$はそれぞれ$(\Ker T)^\perp,(\Ker T^*)^\perp$上への有限な余次元を持つ射影を定める.
        \item ある作用素$S\in B(H)$が存在して,$ST-I,TS-I$はいずれもコンパクトである.
        \item $T$は$B(H)/B_0(H)$で見れば可逆である.
        \item $\Ker T,\Ker T^*$は有限次元で,$\Im T$は閉である.
    \end{enumerate}
\end{proposition}
\begin{remarks}
    $F(H)$は積で閉じており,自己共役である($*$作用素について閉じている).
\end{remarks}

\begin{definition}[index, nullity, defect]
    $T\in F(H)$をFredholm作用素とする.
    \begin{enumerate}
        \item $\Index T:=\dim\Ker T-\dim\Ker T^*$を\textbf{指数}という.$\dim\Ker T$をnullity,$\dim\Ker T^*$をdefectという.
        \item このとき,射影$P,Q$を$ST=I-P,TS=I-Q$と定めると,$\Index T=\rank P-\rank Q$でもある.
        \item $F(H)$の部分集合を
        \[F_n(H):=\Brace{T\in F(H)\mid\Index T=n}\quad n\in\Z\]
        で定めると,$\forall_{n\in\Z}\;F_n(H)\ne\emptyset$.シフト作用素\ref{operator-unilateral-shift}$S$について,$\forall_{n\in\N}\;S^n\in F_{-n}(H),S^{*n}\in F_n(H)$が成り立つため.
    \end{enumerate}
\end{definition}
\begin{example}
    任意の正方行列は,指数$0$のFredholm作用素である:$\dim\Ker A=\dim\Ker A^*$.
\end{example}

\begin{lemma}
    $T\in F(H)$とする.
    \begin{enumerate}
        \item $R\in B(H)$を可逆とすると,$\Index RT=\Index TR=\Index T$.
        \item $\Index T^*=-\Index T$.
        \item $T$の部分逆を$S$とすると,$\Index S=-\Index T$.
    \end{enumerate}
\end{lemma}

\subsection{Fredholmの交代定理}

\begin{tcolorbox}[colframe=ForestGreen, colback=ForestGreen!10!white,breakable,colbacktitle=ForestGreen!40!white,coltitle=black,fonttitle=\bfseries\sffamily,
title=コンパクト作用素のスペクトルとは固有値の集合である]
    無限線型系に対して最初に正確な議論をしたのがFredholmであった.
    Dirichlet問題は積分方程式に帰着され,これは無限の連立1次方程式とみれる.
    積分方程式が解を持つための条件を,Fredholm行列式$\delta:\C\to\C$が零でないこととして特徴付けた定理がFredholmの交代定理/択一定理である.
    これは現代の言葉では,コンパクト作用素のスペクトル$\Sp(T)$の固有値の集合としての特徴付けである.
\end{tcolorbox}

\begin{lemma}
    $A\in B_f(H)$ならば,$I+A\in F_0(H)$.
\end{lemma}
\begin{remark}
    任意の$\lambda\ne0$に対して,コンパクト作用素との和$\lambda+K$は指数$0$のFredholm作用素である.
\end{remark}

\begin{lemma}
    任意の$T\in F_0(H)$について,
    \begin{enumerate}
        \item 部分等長作用素$V\in B_f(H)$が存在して,$T+V$は可逆である.
        \item $A\in B_0(H)$ならば,$T+A\in F_0(H)$.
    \end{enumerate}
\end{lemma}

\begin{corollary}[Fredholm alternative]
    $A\in B_0(H)$,$\lambda\in\C\setminus\{0\}$とする.
    このとき,次のいずれか一方のみが成り立つ.
    \begin{enumerate}
        \item $\lambda I-A\in B(H)$は可逆である.すなわち,$\lambda$はレゾルベント集合に属する.
        \item $\lambda$は$A$の有限な重複度を持った固有値である.
    \end{enumerate}
    後者の場合,複素共役$\o{\lambda}$は$A^*$の固有値で,同じ重複度を持つ.
\end{corollary}
\begin{remarks}[Riesz-Schauder]
    コンパクト作用素$T$のスペクトルは,$\{0\}$と,$T$の固有値のみからなる.
    特に,$\C$の可算部分集合で,(集積点があるとするならば)$0$のみを集積点とする.
    また,$\sigma(T)=\o{\sigma(T^*)}$(閉包ではなく実軸反転)で,
    $\dim\Ker(T-\lambda I)=\dim\Ker(T^*-\o{\lambda}I)<\infty$.
\end{remarks}

\begin{tbox}{red}{}
    $T:=I-\lambda^{-1} A$が指数$0$のFredholm作用素である,ということが古典的なFredholm交代定理の肝である.
    こうして,正方行列の基本的な事実(特に連立一次方程式系の可解性に関連して)は,この恒等作用素とコンパクト作用素の差で表される作用素クラス$T\in F_0(H)$に一般化される.
\end{tbox}

\subsection{Fredholm作用素の指数}

\begin{theorem}
    任意のFredholm作用素$T\in F(H)$とコンパクト作用素$A\in B_0(H)$について,$\Index(T+A)=\Index T$.
\end{theorem}

\begin{proposition}
    任意のFredholmクラス$F_n(H)$は$B(H)$の開集合である.
\end{proposition}

\begin{proposition}
    $T_1\in F_n(H),T_2\in F_m(H)$について,$T_1T_2\in F_{n+m}(H)$.
\end{proposition}

\begin{remarks}
    $G:=B(H)/B_0(H)$を乗法群とし,$G_0$を単位元を含む連結成分とする.すなわち,ある$A\in B(H)/B_0(H)$について$\exp A$と表せる元が生成する部分群とする.
    このとき,$G_0$は$G$内で開かつ閉で,$G/G_0$は離散群で$G$の連結成分を特徴付ける.
    今回,$G/G_0\simeq\Z$である.
    そして,Fredholm作用素$T$の指数とは,作用素$F(H)\epi G\epi G/G_0=\Z$による$T$の像である.
\end{remarks}

\subsection{Hilbert-Schmidtの定理}

\begin{theorem}
    $H$を可分Hilbert空間,$T\in B_0(H)$を自己共役とする.このとき,$T$の固有ベクトルからなる正規直交基底$\{\phi_n\}$が存在する.
    $H$が無限次元ならば,対応する固有値を$\{\lambda_n\}\in c_0$とすると,
    \[\forall_{u\in H}\;Tu=\sum_{n\in\N}\lambda_n(u|\phi_n)\phi_n.\]
\end{theorem}

\subsection{Banach空間上での再論}

\begin{definition}[ascent, descent]
    $X$を線型空間,$u:X\to X$を線型写像とすると,$(\Ker(u^n))_{n\in\N}$は部分空間の増大列で,$(u^n(X))_{n\in\N}$は部分空間の減少列である.
    \begin{enumerate}
        \item $\forall_{n\in\N}\;\Ker(u^n)\ne\Ker(u^{n+1})$のとき,$\Ascent(u)=\infty$とし,それ以外のとき,$\Ascent(u):=\min\Brace{p\in\N\mid\Ker(u^p)=\Ker(u^{p+1})}$と定める.
        \item $\Descent(u):=\Brace{p\in\N\mid u^p(X)=u^{p+1}(X)}$とする.
    \end{enumerate}
\end{definition}

\begin{theorem}
    $X$をBanach空間,$u\in B_0(X)$をコンパクト作用素,$\lambda\in\C\setminus\{0\}$とする.
    \begin{enumerate}
        \item $u-\lambda$は有限なascentとdescentを持つ.
        \item $u-\lambda$はFredholm指数$0$を持つ.
        \item $p:=\Ascent(u)$とすると,$X=\Ker(u-\lambda)^p\oplus\Im(u-\lambda)^p$.
        \item (Fredholm alternative) $u-\lambda$が単射であることと全射であることとは同値.
        \item $\Sp(u)$は可算集合で,$\Sp(u)\setminus\{0\}$は$u$の固有値の集合であり,孤立点のみからなる.
    \end{enumerate}
\end{theorem}

\section{跡}

\begin{quotation}
    関数論とヒルベルト空間上の作用素論との類比において,$C_0,B_0$と$C_c,B_f$は対応するが,
    $B(H)$は2つの役割を持つ.1つは$C_b$であるが,もう一つは$L^\infty(X)$である.そのためにはLebesgue測度にあたる概念が必要であるが,これが跡である.
    このような測度論との交差が特に美しい.いつしか積分もただの線型作用素であったし,級写像も積分作用素の特殊な例なのであった.

    跡が積分に当たることを強烈に示唆する例が,Shannonのエントロピーが事象$\rho$に対して$-E[\rho\log\rho]$であるのに対して,von Neumannエントロピーが密度行列$\rho$に対して$-\Tr(\rho\log\rho)$である.
\end{quotation}

\begin{tcolorbox}[colframe=ForestGreen, colback=ForestGreen!10!white,breakable,colbacktitle=ForestGreen!40!white,coltitle=black,fonttitle=\bfseries\sffamily,
title=跡は関数空間上の測度とみなせる]
    前節では$B_f(H)$を$C_c(X)$に,$B_0(H)$を$C_0(X)$に,$B(H)$を$C_b(X)$に見立てたが,$B(H)$は同時に$L^\infty(X)$に似た振る舞いもする.
    では$X$上のLebesgue測度にあたる$H$上の構造はというと,跡である!
    跡は関数空間上の測度と思えるのか!

    有限次元線型空間論では,跡だけがまるで異邦人のような特異な存在であった.
\end{tcolorbox}

\subsection{定義と性質}

\begin{definition}[trace]
    $H$をHilbert空間,$(e_j)_{j\in J}$をその正規直交基底とする.
    正作用素$T\in B(H)$の\textbf{跡}とは,
    \[\Tr(T):=\sum_{j\in J}(Te_j|e_j)\]
    をいう.$\Tr:B(H)_+\to[0,\infty]$である.
\end{definition}

\begin{proposition}
    $\forall_{T\in B(H)}\;\Tr(TT^*)=\Tr(T^*T)$.
\end{proposition}

\begin{corollary}[well-definedness]
    $T\in B(H)$を正作用素とする.
    任意のユニタリ作用素$U$について,$\Tr(UTU^*)=\Tr(T)$.
    
    特に,跡の定義は正規直交基底の取り方に依らず,$\norm{T}\le\Tr(T)$を満たす.
\end{corollary}

\subsection{跡の延長}

\begin{tcolorbox}[colframe=ForestGreen, colback=ForestGreen!10!white,breakable,colbacktitle=ForestGreen!40!white,coltitle=black,fonttitle=\bfseries\sffamily,
title=]
    跡が測度の類似物ならば,なんらかの意味で測度確定であることはコンパクト性を含意しそうである.
    測度確定なコンパクト作用素をHilbert-Schmidt作用素という.
    さらに中線定理や極化恒等式も成り立ち,$B(H)$は幾何学味を帯びてくる.
\end{tcolorbox}

\begin{lemma}
    正作用素$T\in B(H)$が$\exists_{p>0}\;\Tr(\abs{T}^p)<\infty$を満たすならば,コンパクト作用素である.
\end{lemma}

\begin{definition}[trace class operator, Hilbert-Schmidt operator]\mbox{}\label{def-trace-class-Hilbert-Schmidt}
    \begin{enumerate}
        \item $B^1(H):=\Span\Brace{T\in B_0(H)\mid T\ge0,\Tr(T)<\infty}$の元を\textbf{跡類作用素}という.
        \item $B^2(H):=\Brace{T\in B_0(H)\mid\Tr(T)<\infty}$.
    \end{enumerate}
\end{definition}

\begin{lemma}[跡の延長]
    $T\in B^1(H)$について,
    \begin{enumerate}
        \item ある正作用素$T_0,\cdots,T_3\ge0$が存在して,$T=\sum_{k=0}^3i^kT_k$.
        \item これを用いて$\Tr(T):=\sum_{k=0}^3i^k\Tr(T_k)$と定めると,これは線型汎函数である.
    \end{enumerate}
    以降,$\Tr$の定義域は$B(H)_++B^1(H)$とする.
\end{lemma}

\begin{lemma}[作用素の中線定理と極化恒等式]
    任意の$S,T\in B(H)$について,
    \begin{enumerate}
        \item (parallelogram law) $(S+T)^*(S+T)+(S-T)^*(S-T)=2(S^*S+T^*T)$.
        \item $(S+T)^*(S+T)\le 2(S^*S+T^*T)$.
        \item (polarization identity) $4T^*S=\sum^3_{k=0}i^k(S+i^kT)^*(S+i^kT)$.
    \end{enumerate}
\end{lemma}

\subsection{Hilbert-Schmidt作用素のHilbert空間}

\begin{tcolorbox}[colframe=ForestGreen, colback=ForestGreen!10!white,breakable,colbacktitle=ForestGreen!40!white,coltitle=black,fonttitle=\bfseries\sffamily,
title=]
    単に幾何学的なだけではなく,$B^2(H)$は実際にHilbert空間をなす.
\end{tcolorbox}

\begin{proposition}\mbox{}
    \begin{enumerate}
        \item $B^1(H),B^2(H)$は$B(H)$内の自己共役なイデアルである.
        \item $B_f(H)\subset B^1(H)\subset B^2(H)\subset B_0(H)$.
    \end{enumerate}
\end{proposition}

\begin{theorem}
    Hilbert-Schmidt作用素のイデアル$B^2(H)$は内積$(S|T)_{\tr}:=\Tr(T^*S)$についてHilbert空間をなす.
\end{theorem}

\subsection{跡類作用素のBanach代数}

\begin{lemma}
    $T\in B^1(H),S\in B(H)$について,$\abs{\Tr(ST)}\le\norm{S}\Tr(\abs{T})$.
\end{lemma}

\begin{lemma}
    $S,T\in B^2(H)$について,$\Tr(ST)=\Tr(TS)$.
\end{lemma}

\begin{theorem}
    跡類作用素のイデアル$B^1(H)$はノルム$\norm{T}:=\Tr(\abs{T})$についてBanach代数をなす.
\end{theorem}

\begin{theorem}
    双線型形式$\brac{S,T}:=\Tr(ST)$は,Banach空間$B_0(H),B^1(H)$の間のペアリングであり,また$B^1(H),B(H)$の間のペアリングでもある.
    特に,$(B_0(H))^*=B^1(H)$かつ$(B^1(H))^*=B(H)$.
\end{theorem}

\subsection{Hilbert-Schmidt作用素の表現}

\begin{tcolorbox}[colframe=ForestGreen, colback=ForestGreen!10!white,breakable,colbacktitle=ForestGreen!40!white,coltitle=black,fonttitle=\bfseries\sffamily,
title=]
    これを通じて,例\ref{exp-Hilbert-Schmidt}との同値性がわかる.
\end{tcolorbox}

\begin{proposition}
    Hilbert空間$H$の任意の正規直交基底$(e_j)_{j\in J}$について,$1$階の作用素の集合$\Brace{e_i\odot e_j\in H\mid i,j\in J}$は$B^2(H)$の正規直交基底である.
\end{proposition}
\begin{remarks}
    特に,Hilbert空間$H$が$X$上のRadon積分に関する2乗可積分関数の空間$L^2(X)$であったとき,$\int\otimes\int$とは$X^2$上の積積分(product integral)のことだから,
    これはHilbert-Schmidt作用素を$L^2(X^2)$上に実現していることとなる.
    そして,$L^2(X)$の基底$(e_j)_{j\in J}$に対して,$e_i\otimes\o{e_j}(x,y)=e_i(x)e_j(y)$が$L^2(X^2)$上の正規直交基底である.
    こうして,$U:L^2(X^2)\iso B^2(L^2(X));e_i\otimes\o{e_j}\mapsto e_i\odot e_j$は等長同型である.
    この等長同型を具体的に表現すると次の通り.
\end{remarks}

\begin{proposition}[上述の命題の$L^2$における場合での描像]\mbox{}
    \begin{enumerate}
        \item $k\in L^2(X^2)$を核とする積分作用素
        \[\xymatrix@R-2pc{
            T_k:L^2(X)\ar[r]&L^2(X)\\
            \rotatebox[origin=c]{90}{$\in$}&\rotatebox[origin=c]{90}{$\in$}\\
            f\ar@{|->}[r]&T_kf(x):=\int_{y\in X}k(x,y)f(y)
        }\]
        はHilbert-Schmidt作用素である:$T_k\in B^2(L^2(X))$.
        \item 対応$T:L^2(X^2)\to B^2(L^2(X));k\mapsto T_k$は,$L^2(X^2)$上の2-ノルムについて等長同型になる.
        \item $k^*(x,y)=\o{k(x,y)}$とすると,$T_{k^*}=T^*_k$.すなわち,$T_k$が自己共役であることと,$k$が共役対称であることとは同値.
    \end{enumerate}
\end{proposition}

\subsection{Fredholm積分方程式}

\begin{proposition}
    Fredholm型の積分方程式
    \[\int_{y\in X}k(x,y)f(y)-\lambda f(x)=g(x)\]
    について,
    \begin{enumerate}
        \item $g\in L^2(X)$かつ$k\in L^2(X^2)$が共役対称ならば,解は存在する.
        \item さらに$\forall_{j\in J}\;\lambda\ne\lambda_j$ならば,解は一意であり,次のように表示される:
        \[f=\sum_{j\in J}\frac{(g|e_j)}{\lambda_j-\lambda}e_j\]
    \end{enumerate}
\end{proposition}
\begin{Proof}\mbox{}
    \begin{enumerate}
        \item $T_k$は自己共役なコンパクト作用素(Hilbert-Schmidt作用素)になるから,
        ある直交基底$(e_i)_{i\in J}$に対して,
        スペクトル分解が存在する:
        \[T_k=\sum\lambda_je_j\odot e_j\;\in B^2(L^2(X))\qquad\lambda_j\in\R,\sum\abs{\lambda_j}^2=\norm{T_k}^2=\norm{k}^2_2\]
        なお,$\lambda_i=(T_ke_i|e_i)\in\R$\ref{cor-characterization-of-self-adjointness}より,自己共役作用素の固有値は実数である.
        また,$\norm{T_k}=\norm{k}_2$は,$T:L^2(X^2)\to B^2(L^2(X))$が等長同型であることによる.
        よって,積分方程式は,次の式に同値:
        \[\sum_{j\in J}(\lambda_j-\lambda)(f|e_j)e_j=g.\]
        \item 明らか.
    \end{enumerate}
\end{Proof}

\subsection{Strum-Liouville問題への応用}

\begin{tcolorbox}[colframe=ForestGreen, colback=ForestGreen!10!white,breakable,colbacktitle=ForestGreen!40!white,coltitle=black,fonttitle=\bfseries\sffamily,
title=]
    微分方程式の境界値問題の解はGreen関数を積分核とする積分作用素の像として与えられる.
\end{tcolorbox}

\begin{problem}[Strum-Liouville問題]
    $I:=[a,b]$上の2階線型微分方程式を,3種類考える:
    \begin{enumerate}
        \item $(pf')'+qf=0$.
        \item $(pf')'+qf=\lambda f$.
        \item $(pf')'+qf=\lambda f+h$.
    \end{enumerate}
    ただし,$p,q,h\in C(I;\R),p>0,\lambda\in\R$とした.
\end{problem}

\begin{proposition}[斉次の場合]
    次の2条件を満たす$u,v$が存在するとき,(1)は$u,v$によって生成される2次元の完備な解空間を持つ.
    \begin{enumerate}
        \item 境界条件:$\al,\beta,\gamma,\delta\in\R$について,$\al u(a)+\beta p(a)u'(a)=0$かつ$\gamma v(b)+\delta p(b)v'(b)=0$.
        \item 特殊解の線型独立性:Wronskianはある$c\ne0$について$p(uv'-u'v)=c$を満たす.
    \end{enumerate}
\end{proposition}

\begin{proposition}[非斉次の場合]
    上の命題の状況下で,Green関数を
    \[g(x,y):=\begin{cases}
        c^{-1}u(x)v(y),&a\le x\le y\le b\\
        c^{-1}u(y)v(x),&a\le y\le x\le b
    \end{cases}\]
    と定めると,$h\in L^2(I)$ならば($h\in C(I)$はこれを満たす),Hilbert-Schmidt作用素の像$f:=T_gh$が(3)の$\lambda=0$の場合の,次の境界条件を満たすただ一つの解である.
    \begin{quote}
        (B):$\al f(a)+\beta p(a)f'(a)=\gamma f(b)+\delta p(b)f'(b)=0$.
    \end{quote}
\end{proposition}

\begin{proposition}[Hilbert-Schmidt作用素の問題への還元]
    \begin{enumerate}
        \item (2)の境界条件(B)を満たす解が存在するための必要十分条件は,$\exists_{n\in\N}\;\lambda=\lambda_n^{-1}$.このとき,解は$e_n$である.
        \item (2)の境界条件(B)を満たす解が存在するための必要十分条件は,$\exists_{n\in\N}\;\lambda=\lambda_n^{-1}\;\land h\perp e_n$または$\forall_{n\in\N}\;\lambda\ne\lambda_n^{-1}$である.このとき,解は
        \[f=\sum\frac{\lambda_n(h|e_n)}{1-\lambda\lambda_n}e_n\]
    \end{enumerate}
\end{proposition}
\begin{discussion}
    いま,Green関数は共役対称だから,$L^2(I)$の基底$(e_n)_{n\in\N}$が存在して,スペクトル分解
    \[T_g=\sum\lambda_ne_n\odot e_n,\quad\{\lambda_n\}\subset\R\setminus\{0\},\sum\abs{\lambda_n}^2=\norm{g}^2_2.\]
\end{discussion}

\section{半正定値関数論}

\begin{tcolorbox}[colframe=ForestGreen, colback=ForestGreen!10!white,breakable,colbacktitle=ForestGreen!40!white,coltitle=black,fonttitle=\bfseries\sffamily,
title=]
    半正定値性は接頭辞の"semi-"を落として,単に"positive"または"positive definite"とも呼ばれる.
    Radon積分を特徴付けたのも正性であった.
    Abel $*$-半群$(S,+,*)$上の半正定値関数論を考える.
    Laplace, Fourier変換,確率母関数はいずれも半正定値性を持つ.
    これは特定の半正定値性を持つ関数の積分変換であるから当然である.
    そしてその半正定値性を確率論の言葉で特徴付けるのがHoeffdingの不等式である.
    Bochner, Bernstein-Widder, Hamburgerの定理はそれぞれ,$(\R,+,x^*=-x),(\R_+,+,x^*=x),(\N,+,n^*=N)$に対する特殊化である.
\end{tcolorbox}

\subsection{定義と特徴付け}


\begin{tcolorbox}[colframe=ForestGreen, colback=ForestGreen!10!white,breakable,colbacktitle=ForestGreen!40!white,coltitle=black,fonttitle=\bfseries\sffamily,
    title=]
    Fourier解析をさらに一般のHilbert空間で考えるとき,調和解析という.\cite{吉田}
\end{tcolorbox}

\begin{definition}[positive definite function]
    関数$\varphi:\R\to\C$が\textbf{正の定符号関数}であるとは,
    \begin{enumerate}
        \item $t=0$において連続.
        \item $\forall_{t_i\in\R}\;\forall_{\xi_k\in\C}\;\forall_{n\in\N^+}\;\sum_{i,j=0}^n\varphi(t_i-t_j)\xi_i\o{\xi_j}\ge0$.
    \end{enumerate}
    を満たすことをいう.
\end{definition}

\begin{lemma}
    任意の正の定符号関数$\varphi$は,次の3条件を満たす:
    \begin{enumerate}
        \item $\varphi(0)\ge\abs{\varphi(t)}$.
        \item $\R$上で一様連続.
        \item 連続関数$h\in C(\R)$に対して,積分
        \[\int_\R\int_\R\varphi(s-t)h(s)\o{h(t)}dsdt\]
        が存在すれば,$\ge0\;\fe$
    \end{enumerate}
\end{lemma}

\begin{theorem}[Bochner (1932)\footnote{Vorlesungen uber Fouriersche Integrale, Leipzig}]
    正の定符号関数$\varphi$に対して,有界で単調増加かつ右連続な$v(\lambda)$が存在して,
    \[\varphi(t)=\int_\R e^{it\lambda}dv(\lambda)\]
    が成り立つ.この$v$はさらに条件$v(-\infty)=0$を課せば,一意に取れる.
\end{theorem}

\begin{theorem}[Khintchine]
    $\varphi\in C_b(\R;\C)$を有界連続関数であるとする.次の2条件は同値:
    \begin{enumerate}
        \item $\varphi$は正の定符号関数である.
        \item $\norm{f_n}_{L^2(\R)}^2\le\varphi(0)$を満たす列$\{f_n\}\subset L^2(\R)$が存在して,コンパクト一様収束$\lim_{n\to\infty}\int_\R f_n(t+s)\o{f_n(s)}ds=\varphi(t)$が成り立つ.
    \end{enumerate}
\end{theorem}

\section{再生核}

\begin{tcolorbox}[colframe=ForestGreen, colback=ForestGreen!10!white,breakable,colbacktitle=ForestGreen!40!white,coltitle=black,fonttitle=\bfseries\sffamily,
title=]
    積分方程式に関する研究から,振る舞いのよい積分核のクラスを確定させる研究が派生し,Moore, E. H.がこれに自覚的になり抽象的に扱って核の再生性を定義したところから,楕円型偏微分方程式への応用がみつかり,そこで再生核の理論が確立された.
    また,Cartan, E.やWeyl, Kreinの等質空間上の調和解析でも中心的役割を演じた.
    確率論では確率過程の共分散として現れる.
\end{tcolorbox}

\begin{history}\mbox{}
    \begin{enumerate}
        \item 正定値核は,第II種Fredholm積分方程式に関するHilbert (1904)の仕事を作用素論的に解釈したJames Mercer (1909)が定義した.
        正定値カーネルが次元を上げれば標準内積で表現出来るというMercerの定理が,現代の線型アルゴリズムを非線型アルゴリズムに変換するKernelトリックの基礎となっている(Aizerman 1964).
        \item 一方で,Mathias, M.とBochner, S.は正定値核を知らずに,独立に正定値関数を定義したと考えられる.
        \item 再生核:N. Aronszajnの標記理論\footnote{Aronszajn, N. (1950) \textit{Theory of reproducing kernels}. Trans. Amer. Math. Soc. 68: 337-404.}が,複素数関数論や偏微分方程式論に関連して有効性が認められてきたという(吉田1953\cite{吉田}).
        そもそも,超関数以外にもデルタ関数を積分表示で与える手法として提案されたものである.
    \end{enumerate}
\end{history}

\subsection{定義と存在}

\begin{definition}[reproducing kernel]
    $\Om\in\Set,K:\Om^2\to\bF$とHilbert空間$X\subset\Map(\Om,\bF)$について,
    \begin{enumerate}
        \item $K$が\textbf{$X$-再生核}であるとは,
        \begin{enumerate}[(a)]
            \item $\forall_{y\in\Om}\;K(-,y)\in X$.
            \item $\forall_{y\in\Om}\;\forall_{f\in X}\;f(y)=(f(x)|K(x,y))$.
        \end{enumerate}
    \end{enumerate}
\end{definition}

\begin{theorem}[存在と一意性]
    Hilbert空間$X\subset\Map(\Om,\bF)$について,次の2条件は同値:
    \begin{enumerate}
        \item $X$-再生核$K$が存在する.
        \item $\forall_{y_0\in\Om}\;\exists_{C\in\R}\;\forall_{f\in X}\;\abs{f(y_0)}\le C\norm{f}$.
    \end{enumerate}
    また,この条件が成り立つとき,再生核$K$は一意的に存在する.
\end{theorem}

\begin{corollary}
    Hilbert空間$X\subset\Map(\Om,\bF)$に再生核$K$が存在するとする.このとき,
    \begin{enumerate}
        \item $\max_{f\in\partial B}\abs{f(y_0)}=K(y_0,y_0)^{1/2}$.
        \item 等号成立条件は$f_0(x)=\rho\frac{K(x,y_0)}{K(y_0,y_0)^{1/2}}\;(\rho\in\partial\Delta)$のときに限る.
    \end{enumerate}
\end{corollary}

\subsection{半正定値性による特徴付け}

\begin{theorem}[半正定値性による特徴付け]
    $\Om\in\Set,K:\Om\times\Om\to\bF$について,次の2条件は同値.
    \begin{enumerate}
        \item $K$は半正定値である.
        \item あるHilbert空間$\F\subset\Map(\Om,\bF)$が存在して,$K$はその再生核である.
    \end{enumerate}
\end{theorem}

\subsection{再生核の表示}

\begin{theorem}
    $\F$を可分,$K$をその再生核とする.$\F$の任意の正規直交系$(\varphi_n)$に対して,
    \begin{enumerate}
        \item $\sum_{n\in\N}\varphi_n(x)\o{\varphi_n}(y)=K(x,y)$.
        \item $x$の関数として$K(x,y)=\lim_{n\to\infty}\sum_{i=1}^n\varphi_i(x)\o{\varphi_i(y)}$は各点収束する.
    \end{enumerate}
\end{theorem}

\subsection{Bergmanの核関数}

\begin{definition}
    $\Om\osub\C$を領域とすると,$H(\Om)\cap L^2(\Om)$は可分Hilbert空間をなし,再生核を持つ.
\end{definition}

\begin{theorem}
    $\Om\osub\C$を単連結領域,$f:E\to\Delta(0,r)$を双正則写像とする.
    \[f(z_0)=0,\quad\dd{f}{z}(z_0)=1\quad(z_0\in\Om)\]
    を満たすならば,次が成り立つ:
    \[f(z)=\frac{1}{K(z_0,z)}\int^z_{z_0}K(t,z_0)dt.\]
\end{theorem}

\subsection{Stoneの定理}

\begin{theorem}[Stone]
    ユニタリ作用素の系$\{U_t\}_{t\in\R}\subset\Aut(H)$が群の性質$U_tU_s=U_{t+s},U_0=I$と$H$上の各点収束に関する連続性$\lim_{t\to t_0}U_t=U_{t_0}$を満たすならば,単位の分解$(E(\lambda))$が存在して,
    \[U_t=\int^\infty_{-\infty}e^{it\lambda}dE(\lambda).\]
\end{theorem}
\begin{remarks}
    これはBochnerの定理から導くことも出来,逆もたどれる.よって,2つの主張は同等であると考えて良い\cite{吉田}.
\end{remarks}

\subsection{調和解析}

\begin{theorem}
    有界連続関数$f\in C_b(\R;\C)$が,ある右連続かつ有界変動な関数$v\in\BV(\R;\C)$を用いて
    \[f(t)=\int^\infty_{-\infty}e^{i\lambda t}dv(\lambda)\]
    と調和振動$e^{i\lambda t}$の畳み込みとして表されるための必要十分条件は,次が与える:
    \[\sup_{n\in\N}\int_\R\Abs{\int_\R\paren{\frac{\sin(t/n)}{t/n}}^2f(t)e^{-i\lambda t}dt}d\lambda.\]
\end{theorem}
\begin{remarks}
    絶対収束列$(a_n)\in l^1(\N;\C)$は,
    \[\sum_{n=\infty}^\infty a_ne^{i\lambda_nt}\]
    によって振動を定める.しかし,$(a_n)$が一様収束しかしなくても,上の条件を満たさない「振動」が定義出来る.
    このようなクラスを特徴付けるのが,H. Bohrの概周期関数である.
\end{remarks}

\section{概周期関数}

\subsection{定義と特徴付け}

\begin{definition}[almost periodic function]
    $f\in C(\R;\C)$と$\ep>0$について,
    \begin{enumerate}
        \item \textbf{$\ep$に属する概周期}とは,$P_\ep:=\Brace{\tau\in\R\mid\sup_{t\in\R}\abs{f(t+\tau)-f(t)}\le\ep}$をいう.
        \item $\forall_{\ep>0}\;\exists_{l>0}\;\forall_{\al\in\R}\;(\al,\al+l)\cap P_\ep\ne\emptyset$ならば,$f$を\textbf{概周期関数}という.
    \end{enumerate}
\end{definition}

\begin{theorem}
    概周期関数$f\in C(\R;\C)$について,
    \begin{enumerate}
        \item $f\in C_b(\R)$である.
        \item $f\in UC(\R)$である.
    \end{enumerate}
\end{theorem}

\begin{theorem}[Bochner-Favard]
    $f\in C(\R;\C)$について,次は同値:
    \begin{enumerate}
        \item $f$は概周期的である.
        \item 任意の$\{a_n\}\subset\R$について,関数列$\{f_n(t):=f(t+a_n)\}$を作ると,ある部分列が$\R$上一様収束する.
    \end{enumerate}
\end{theorem}

\subsection{Weierstrassの近似定理}

\begin{theorem}
    概周期関数$f$と任意の$\ep>0$について,ある三角多項式$P_\ep$が存在して,$\norm{f-P_\ep}<3\ep$.
\end{theorem}

\chapter{スペクトル理論}

\begin{quotation}
    解析学の理論であるが,幾何学と代数学の双対性であるIsbell双対性の一つの例・Gelfand dualityに到達する.
    任意の$C^*$-代数$\A$は,あるGelfandスペクトル$\Sp(A)$と呼ばれる位相空間上の連続関数のなす$C^*$-代数に等しい.
    群と体の間の写像を調べる営みが解析学であるとしたら,どこに存在するのか.

    Banach代数$\A$のfunction calculusとは,あるコンパクトハウスドルフ空間$X$上の連続関数のなす代数$C\subset C(X)$との間の同型$\Iso(C,A)$の部分集合をいう.
    $C$が$C(X)$の中で大きいほど,よいfunction calculusだとみなされる.

    ある同型$\Phi:C\to A$に対して,

    Spectra in algebraic geometry : 
    Grothendieck has defined a prime spectrum of commutative unital ring having in mind Gel'fand's spectrum of a commutative $C^*$-algebra
\end{quotation}

\section{Categorical Settings}

\begin{tcolorbox}[colframe=ForestGreen, colback=ForestGreen!10!white,breakable,colbacktitle=ForestGreen!40!white,coltitle=black,fonttitle=\bfseries\sffamily,
title=]
    Gelfand Duality\footnote{\url{https://ncatlab.org/nlab/show/Gelfand+duality}}
\end{tcolorbox}

\begin{notation}\mbox{}
    \begin{enumerate}
        \item $C^*\Alg$で,単位的な$C^*$-代数のなす圏を表す.
        \item $C^*\Alg_\nonunital$で,単位的でない$C^*$-代数のなす圏を表す.
        \item $\Top_\cpt$で,コンパクトハウスドルフ空間のなす$\Top_\Haus$の充満部分圏を表す.
        \item $*/\Top_\cpt$で,$\Top_\cpt$のpointed objectのなす圏とする.
        \item $\Top_{\lcpt,\infinity}$で,射を無限遠点で消える連続写像とする局所コンパクトハウスドルフ空間のなす圏とする.
        \item $\Top_{\lcpt,\proper}$で,射をproper mapとする局所コンパクトハウスドルフ空間のなす圏とする.
    \end{enumerate}
\end{notation}

\begin{notation}[functors]\mbox{}
    \begin{enumerate}
        \item 関手$C:\Top_\cpt\to C^*\Alg^\op_{\com}$を,$C(X):=\Brace{f\in\Map(X,\C)\mid f\text{\;continuous}}$に写す関手,
        ,$C_0:*/\Top_\cpt\to C^*\Alg^\op_{\com,\nonunital}$を,$x_0\in X$で消える連続関数のなす空間に写す関手とする.
        \item $\Spec:C^*\Alg^\op_\com\to\Top_\cpt$で,可換な$C^*$-代数$\A$を,スペクトル位相によって位相空間と見た指標の空間に対応させる関手とする.同様に$\Spec:C^*\Alg^\op_{\com,\nu}\to\Top_\lcpt$も定まる.
    \end{enumerate}
\end{notation}

\begin{theorem}[unital Gelfand duality theorem]\label{thm-Gelfand-duality-theorem}
    $\Spec:C^*\Alg^\op_\com\to\Top_\cpt$は圏の同値であり,$C$が準逆である.
\end{theorem}
\begin{corollary}
    $\Spec:C^*\Alg^\op_{\com,\nu}\to */\Top_\cpt$は圏の同値であり,$C_0$が準逆である.
\end{corollary}

\begin{lemma}
    コンパクトハウスドルフ空間の開集合は,局所コンパクトである.
\end{lemma}
\begin{remarks}
    これより,一点コンパクト化によって,連続関数は「無限遠点で消える連続関数」に対応するから,圏の反変同値$\Top_{\lcpt,\infinity}\Leftrightarrow C^*\Alg_{\com,\nonunital}$が引き起こされる.
    この双対性は,基礎体が$\C$でも$\R$でも成り立つ.
\end{remarks}

\section{Banach代数}

\begin{tcolorbox}[colframe=ForestGreen, colback=ForestGreen!10!white,breakable,colbacktitle=ForestGreen!40!white,coltitle=black,fonttitle=\bfseries\sffamily,
    title=]
    Banach環は何故か解析学が展開される場として,$\C$の一般化としても扱える.
    \begin{quote}
        冪級数を介して定義されるいくつかの初等関数は、任意の単位的バナッハ環において定義されうる。そのような例として、指数関数や三角関数、さらに一般的な任意の整関数が挙げられる(特に、指数写像は抽象指数群(英語版)を定義するために用いられる)。幾何級数の公式は、一般の単位的バナッハ環においても依然として有効である。二項定理もまた、バナッハ環の二つの可換な元に対して成立する\footnote{\url{https://ja.wikipedia.org/wiki/バナッハ環}}。
    \end{quote}
\end{tcolorbox}

\subsection{定義}

\begin{tcolorbox}[colframe=ForestGreen, colback=ForestGreen!10!white,breakable,colbacktitle=ForestGreen!40!white,coltitle=black,fonttitle=\bfseries\sffamily,
title=]
    Banach代数はBanのモノイド対象であるが,定義に単位性は入らない(半群対象).
    しかし,標準的な単位化が存在するから,ここでは単位性を暗黙に仮定する.
    ということで,単位的なBanach代数とは,ノルムが$1$な単位元が存在するノルム代数を言う.
    一般に,ノルム代数とは劣乗法的なノルムを備えた代数をいう.
    さらに一般に代数とは,体による作用を備えた環,または環の構造も持つ線型空間を言う.
\end{tcolorbox}

\begin{definition}[algebra, normed -, Banach -, unital, essential ideal]\mbox{}
    \begin{enumerate}
        \item \textbf{代数}$\A$とは,ある結合的な双線型写像$\cdot:\A\times \A\to \A$について(単位的とは限らない)環の構造も持つ線型空間$\A$をいう.
        \item 部分代数とは,乗法について閉じているような部分線型空間をいう.
        \item $\A$が\textbf{ノルム代数}であるとは,劣乗法性を満たすノルム$\norm{xy}\le\norm{x}\norm{y}$を備えた代数を言う.
        \item ノルム代数$\A$がそのノルム位相について完備であるとき,\textbf{Banach代数}という.
        \item ある元$I\in\A$が存在して$\forall_{A\in\A}\;IA=AI=A$を満たすとき,ノルム代数$\A$は単位的であるという.
        \item イデアル$\I\subset\A$が,$\A$の任意のイデアルに対して零でない共通部分を持つとき,これを\textbf{本質イデアル}という.
    \end{enumerate}
\end{definition}

\begin{lemma}[単位元の性質について]\mbox{}
    \begin{enumerate}
        \item 単位元$I$は一意的であり,$\A$が零でないならば$\norm{I}\ge 1$を満たす.
        \item 単位的なノルム代数$\A$について,$\norm{I}=1$ならば,写像$\bF\ni\al\mapsto\al I\in\A$は単射な準同型を定め,さらに等長写像である:$\norm{\al I}=\abs{\al}$.\footnote{これをBanach代数の指標の自動連続性という.Frechet代数の基礎体への準同型も自動的に連続であるかは未解決.}
    \end{enumerate}
\end{lemma}
\begin{Proof}\mbox{}
    \begin{enumerate}
        \item ノルムの劣乗法性より,$\forall_{A\in\A}\;\norm{A}\le\norm{A}\norm{I}$.$\A$が零でないとき,$A\ne 0$に取れるから,$1\le\norm{I}$.
        \item 略.
    \end{enumerate}
\end{Proof}
\begin{remarks}
    (2)は行列で言えばスカラー行列への埋め込みであるが,これが同型になる場合は限られることがGelfand-Mazur定理\ref{cor-Gelfand-Mazur-thm}となる.
\end{remarks}

\begin{lemma}[Banach代数のノルムの劣乗法性の十分条件]
    代数$\A$がBanach空間でもある(ノルムが定まっており,ノルム位相について完備)とする.このとき,(1)$\Rightarrow$(2)$\Rightarrow$(3)が成り立つ.
    \begin{enumerate}
        \item ノルムが劣乗法性を満たす:$\forall_{A,B\in\A}\;\norm{AB}\le\norm{A}\norm{B}$.
        \item 乗法$\A\times\A\to\A$はノルム位相について連続である.
        \item 劣乗法性を満たすノルムであって元のノルムと同値なものが存在する.
    \end{enumerate}
\end{lemma}
\begin{Proof}\mbox{}
    \begin{description}
        \item[(1)$\Rightarrow$(2)] ノルム位相に関して収束する$\A\times\A$の点列$((A_n,B_n))_{n\in\N}\xrightarrow{n\to\infty}(A,B)\in\A\times\A$を取り,積もノルム収束すること:$\norm{AB-A_nB_n}\xrightarrow{n\to\infty}0$を示せば良い.
        \begin{align*}
            \norm{AB-A_nB_n}&=\norm{AB-A_nB+A_nB-A_nB_n}\\
            &\le\norm{(A-A_n)B}+\norm{A_n(B-B_n)}\\
            &\le\norm{A-A_n}\norm{B}+\norm{A_n}\norm{B-B_n}\xrightarrow{n\to\infty}0.&\because\text{劣乗法性}
        \end{align*}
        \item[(2)$\Rightarrow$(3)]
        $\A_1:=\A\times\bF$を単位化とする.
        $a\in\A$に対して,
        \[\xymatrix@R-2pc{
            L_a:\A_1\ar[r]&\A_1\\
            \rotatebox[origin=c]{90}{$\in$}&\rotatebox[origin=c]{90}{$\in$}\\
            (x,\zeta)\ar@{|->}[r]&(ax+\zeta a,0)
        }\]
        とすると,$L_a\in B(\A_1)$である:$\norm{L_a(x,\zeta)}=\norm{ax+\zeta a}\le\norm{a}\norm{(x,\zeta)}$.
        これに対して,$\nnorm{a}:=\norm{L_a}$と定めると,これは$\A$の元々のノルムと同値であり,$\A$はこのノルムについてBanach代数である(劣乗法性を満たす).
        実際,劣乗法性は作用素ノルムの劣乗法性から従い,ノルムの同値性も,
        $\A\to\A;x\mapsto ax$は連続であるから,$\exists_{C\in\R}\;\norm{ax}\le C\norm{x}$に注意すると,
        特に$(x,\zeta)=(0,1)$は$\norm{(0,1)}\le 1$を満たすから,
        \[\norm{a}=\norm{a\cdot 0+1\cdot a}\le\sup_{\norm{(x,\zeta)}\le 1}\norm{ax+\zeta a}=\nnorm{a}\]
        と,
        \begin{align*}
            \nnorm{a}&=\sup_{\norm{(x,\zeta)}\le 1}\norm{ax+\zeta a}\\
            &\le\sup_{\norm{(x,\zeta)}\le 1}(\norm{ax}+\abs{\zeta}\norm{a})\\
            &\le\sup_{\norm{(x,\zeta)}\le 1}(C\norm{x}+\abs{\zeta}\norm{a})\\
            &=\norm{a}\sup_{\norm{x}+\abs{\zeta}\le 1}\underbrace{\paren{\frac{C}{\norm{a}}\norm{x}+\abs{\zeta}}}_{\in\R}
        \end{align*}
        とより判る.
    \end{description}
\end{Proof}
\begin{remarks}[作用素ノルムの立ち位置]
    後に議論する単位化によって$\norm{I}=1$として良いように,ノルムの劣乗法性は作用素ノルムが模範となっている.
\end{remarks}

\subsection{代数の射}

\begin{tcolorbox}[colframe=ForestGreen, colback=ForestGreen!10!white,breakable,colbacktitle=ForestGreen!40!white,coltitle=black,fonttitle=\bfseries\sffamily,
title=]
    代数の射は,環の射であり,線型写像でもある.
    一方で,環の射として単位的であることは要請しないことが多い.
\end{tcolorbox}

\begin{definition}
    代数$A,B$について,
    \begin{enumerate}
        \item $\varphi:A\to B$が代数の射であるとは,積を保つ線型写像であることをいう.
        \item 単位元を保つとき,$\varphi$を単位的という.
    \end{enumerate}
\end{definition}

\begin{lemma}
    代数の射$\varphi:A\to B$について,
    \begin{enumerate}
        \item $\Ker\varphi$は$A$のイデアルである.
        \item $\Im\varphi$は$B$の部分代数である.
    \end{enumerate}
\end{lemma}

\begin{lemma}
    ノルム代数$A,B$と,その間の2つの連続な代数の準同型$\varphi,\psi:A\to B$について,ある$A$をノルム代数として生成する集合$S$(すなわち,$S$の生成する閉部分代数が$A$である)上で一致するならば,これらは写像として等しい.
\end{lemma}

\begin{example}[disk algebra]
    円板代数$A(D)$に対して,$\lambda\in\o{D}$が定める評価写像$\ev_\lambda:A(D)\to\C$は連続な代数の準同型である.
    また,任意の連続な代数の準同型で零でないもの(これを指標というのであった\ref{def-character})はこの形を持つ.
    この条件を満たすBanach代数,すなわち指標がすべて評価者像となるBanach代数を,自然な一様環\ref{def-uniform-algebra}という.
    これは,恒等写像と包含写像$z:\o{D}\mono\C$とが生成する閉部分代数は$A(D)$自身であることによる.
\end{example}

\begin{example}
    単位的代数$A$に対して,多項式$p\in\C[z]$に$a\in A$での値を対応させる写像$\C[z]\to A;p\mapsto p(a)$は単位的準同型である.
\end{example}

\subsection{部分代数と生成}

\begin{tcolorbox}[colframe=ForestGreen, colback=ForestGreen!10!white,breakable,colbacktitle=ForestGreen!40!white,coltitle=black,fonttitle=\bfseries\sffamily,
title=]
    2つの代数の射$\varphi,\psi:A\to B$が,$A$の生成系$S$で一致するならば写像として一致するのは,ファイバー$\Brace{a\in A\mid\varphi{a}=\psi(a)}$は部分代数をなすためである.
\end{tcolorbox}

\begin{lemma}\mbox{}
    \begin{enumerate}
        \item 部分代数の閉包は再び部分代数である.
        \item Banach代数の閉じた部分代数はBanach代数である.
    \end{enumerate}
\end{lemma}

\begin{lemma}
    $(B_\lambda)_{\lambda\in\Lambda}$を部分代数の族とする.
    \begin{enumerate}
        \item $\cap_{\lambda\in\Lambda}B_\lambda$も部分代数である.
        \item 任意の集合$S\subset A$に対して,これを含む最小の部分代数が定まる.
    \end{enumerate}
\end{lemma}

\begin{example}
    一点$a\in A$が生成する部分代数は$\Brace{a^n\in A\mid n\in\N}$によって与えられる.
\end{example}

\begin{example}[Stone-Weierstrassの定理]
    単位円周$\T:=\partial\Delta$上の連続関数全体の代数$C(\T)$内で,包含写像$z:\T\mono\C$とその共役$\o{z}:\T\mono\C$とが生成する閉部分代数は$C(\T)$自身になる.
    すなわち,この2つの生成する部分代数は$C(\T)$内で稠密である.
\end{example}

\subsection{Banach代数の単位化}

\begin{tcolorbox}[colframe=ForestGreen, colback=ForestGreen!10!white,breakable,colbacktitle=ForestGreen!40!white,coltitle=black,fonttitle=\bfseries\sffamily,
title=]
    Banach代数は単位的にできる.これは位相的には一点コンパクト化に対応する.
\end{tcolorbox}

\begin{proposition}[Banach代数の単位化]
    $\A$を単位元のない零でないBanach代数とする.
    このとき,基礎体との直和空間$\A_I:=\A\oplus\bF$に次のように積を定めると,単位元$I:=(0,1)$を持つBanach代数となり,写像$i:\A\to\A_I:A\mapsto(A,0)$は余次元1を持つ本質閉イデアルに$\A$を等長に埋め込む.
    \begin{quote}
        $(A,\al)(B,\beta):=(AB+\al B+\beta A,\al\beta)$.
    \end{quote}
\end{proposition}
\begin{Proof}
    $\A_I$が代数になること,$i:\A\to\A_I$が代数の埋め込みであり,等長写像になることは認める.
    \begin{description}
        \item[劣乗法性] \begin{align*}
            \norm{(A,\al)(B,\beta)}&=\norm{(AB+\al B+\beta A,\al\beta)}\\
            &=\norm{AB+\al B+\beta A}+\abs{\al\beta}\\
            &\le\norm{A}\norm{B}+\abs{\al}\norm{B}+\abs{\beta}\norm{A}+\abs{\al}\abs{\beta}\\
            &=(\norm{A}+\abs{\al})(\norm{B}+\abs{\beta})=\norm{(A,\al)}\norm{(B,\beta)}.
        \end{align*}
        \item[完備性] 命題\ref{prop-completion-of-algebraic-direct-product}より.
        \item[像は余次元$1$の本質イデアルである] 準同型定理より$\A_I/\A\simeq\bF$だから余次元$1$で,また$\A$は極大イデアルである.
        $\B$を零でないイデアルとする.
        $\A\cap\B=0$と仮定して矛盾を導く.
        このとき$\exists_{\al\in\bF^\times}\;(B,\al)\in\B$が成り立つが,$\A_1\B=\B$より,特に$\al=1$と考えて良い.
        また,$\forall_{A\in\A}\;(A,0)(B,1)=(AB+A,0)\in\B$であるが,このとき$AB+A\ne0$である.$AB+A=0$ならば,$\forall_{A\in\A}\;AB^2=A$が従い,$\A$に単位元がないことに矛盾.よって,$AB+A\ne0$であるが,これは$\A\cap\B=0$に矛盾.
        \footnote{$\B\subset\A$ならば明らかに$\A\cap\B\ne0$だから,$\B\setminus\A\ne\emptyset$とすると,$\exists_{\al\in\bF^\times,A\in\A}\;(A,\al)\in\B$であるが,$(A,\al^{-1})$との積を考えることより,特に$\al=1$として良い.
        すると,$(-A,0)+(A,1)=(0,1)\in \A+\B$より,$\A+\B=\A_I$だから,2つのイデアル$\A,\B$は互いに素である.
        よって中国剰余定理より,$\A_I/\A\cap\B\simeq\A_I/\A\times\A_I/\B\simeq\bF\times\A_I/\B$であり,$\B\ne0$より,$\A_I/\A\cap\B\subsetneq\A_I$.
        特に$\A\cap\B\ne0$.}
        \item[像はノルム閉である]
        明らか.また像は極大イデアルだから,極大イデアルは閉であること\ref{cor-maximal-ideal-is-closed}からもわかる.
    \end{description}
\end{Proof}
\begin{remarks}[一点コンパクト化との対応]
    一点コンパクト化に次の図式を可換にするという意味で「対応」する操作である.
    \[\xymatrix{
        C^*\Alg_{\com,\nonunital}^\op\ar[r]^-{\Spec}\ar[d]_-{u}&\Top_{\lcpt,\infinity}\ar[d]^-c\\
        C^*\Alg^\op_\com\ar[r]^-{\Spec|}&\Top_\cpt
    }\]
    ただし,$u:C^*\Alg^\op_{\com,\nonunital}\to C^*\Alg^\op_{\com}$は$C^*$-代数の単位化が定める関手,$c:\Top_{\lcpt,\infinity}\to\Top_\cpt$は一点コンパクト化が定める関手とした.\footnote{\url{https://math.stackexchange.com/questions/2084557/why-is-adjoining-a-unit-the-algebraic-counterpart-to-the-one-point-compactificat}}
\end{remarks}

\begin{remark}[極大スペクトルとの関係]
    なお,指標$\A\to\C$の核も,余次元$1$の閉部分空間で,極大イデアルになる.
    特に,Gelfandスペクトルは,環や(離散)代数の極大スペクトル$\Spec_mA$の位相的な類比でもある.
    極大スペクトルの方について,$R$を体$k$上の有限生成な単位的で可換なNoether環で冪零元を持たないものとしたとき,$\Spec_mA$はZariski位相によってNoether位相空間となる.
    こちらでは,より一般の環について素スペクトルを考えていく.
\end{remark}

\begin{example}
    局所コンパクトハウスドルフ空間上の代数$C_0(X)$はBanach代数で,単位的であることは$X$がコンパクトであることに同値.
\end{example}

\subsection{$*$-代数の定義と例}

\begin{tcolorbox}[colframe=ForestGreen, colback=ForestGreen!10!white,breakable,colbacktitle=ForestGreen!40!white,coltitle=black,fonttitle=\bfseries\sffamily,
title=]
    Banach代数と$C^*$-代数の橋渡しとなる概念をここで提示しておく.ここでは$\bF=\C$とする.
\end{tcolorbox}

\subsubsection{定義}

\begin{definition}[involution, anti-involution, star algebra, Banach star algebra]\mbox{}
    \begin{enumerate}
        \item 二乗が恒等写像となるような準同型を\textbf{対合}という.恒等射自身も対合である.反準同型でもある対合を\textbf{反対合}と呼ぶ.
        \item 対合を備えた可換環$K$について,$K$-$*$-代数$A$とは,$K$-双線型写像$\cdot:A\times A\to A$と$K$-反線型写像${}^*:A\to A$を備えた$K$-加群$A$であって,次の2条件を満たすものをいう:
        \begin{enumerate}[(a)]
            \item ${}^*$は台となる$K$-加群$A$上に対合を定める:$\forall_{x\in A}\;x^{**}=x$.
            \item ${}^*$は乗法$*$上に反準同型を定める:$\forall_{x,y\in A}\;(xy)^*=y^*x^*$.
        \end{enumerate}
        \item $K=\C$であり,$C$-$*$-代数$A$がBanach代数でもあるとき,これを\textbf{Banach $*$-代数}という.$*:A\to A$は等長同型である条件$\norm{xx^*}=\norm{x}^2$(これを$B^*$-条件という)も課すとき,$B^*$-代数ともいう.
        \item Banach $*$-代数$B$について,半対合${}^*:B\to B$が台となるBanach代数$B$上のノルムと両立するとき:$\norm{a^*a}=\norm{a}\norm{a^*}$(これを$C^*$-性という),これを特に$C^*$-代数と呼ぶ.
        \item 実は$C^*$-条件と$B^*$-条件とは同値になるため,現在は前者の用語しか使われない.
    \end{enumerate}
\end{definition}

\begin{example}\mbox{}
    \begin{enumerate}
        \item 
    \end{enumerate}
\end{example}

\subsubsection{$*$-準同型}

\begin{lemma}
    任意の$*$-準同型は正である.
\end{lemma}

\subsubsection{可換な例}

\begin{example}[位相空間上の関数の代数]\mbox{}
    \begin{enumerate}
        \item 集合$S$上の有界な関数$l^\infty(S)$は,各点毎の和・積・スカラー倍と一様ノルムによって,単位的なBanach代数をなす.
        \item 位相空間$\Om$上の有界連続関数$C_b(\Om)$は,その閉部分代数をなす.よって,単位的Banach代数である.
        \item 局所コンパクトハウスドルフ空間$\Om$上の無限遠点で消える関数$C_0(\Om):=\Brace{f\in C_b(\Om)\mid\forall_{\ep>0}\;\{\abs{f(\om)}\ge\ep\}\text{ is compact}}$は,この閉部分代数で,よってBanach代数である.
        これが単位的であることは$\Om$がコンパクトであることに同値.これは$C_0(\Om)=C(\Om)$であることに同値.
        \item これらはいずれも,各点毎の複素共役を対合として,$C^*$-代数となる.
    \end{enumerate}
\end{example}

\begin{example}[可測関数の代数]\mbox{}
    \begin{enumerate}
        \item 測度空間$(\Om,\mu)$上の本質的に有界な可測関数全体の集合$L^\infty(\Om,\mu)$は,上述の代数と同じ演算と本質的上限ノルムによって,単位的Banach代数をなす.
        \item $\Om$を可測空間とすると,その上の有界な可測関数全体の空間$B_\infty(\Om)$は,一般の有界な関数の空間$l^\infty(\Om)$の中の閉部分代数をなす.よって,単位的Banach代数である.
    \end{enumerate}
\end{example}

\begin{example}[正則関数の代数(disk / disc algebra, Hardy space]\mbox{}
    \begin{enumerate}
        \item 単位開円板$D:=\Delta$上で正則な関数全体の空間$A(D)$は連続関数全体の空間$C(\o{D})$の閉部分代数で,したがって単位的Banach代数である.
        これを\textbf{円板代数}という.これが,Banach代数上の理論(Neumann級数展開など)の霊性源となる.
        実は,$A(D)=H^\infty(D)\cap C(\o{D})$が成り立ち,Hardy空間の閉部分代数でもある.
        \item 単位開円板$D$上の有界正則関数のなす空間$H^\infty(D)$は,一様ノルムについてBanach代数をなす.これを\textbf{Hardy代数}という.
    \end{enumerate}
\end{example}

\subsubsection{Gelfand変換の用語}

\begin{tcolorbox}[colframe=ForestGreen, colback=ForestGreen!10!white,breakable,colbacktitle=ForestGreen!40!white,coltitle=black,fonttitle=\bfseries\sffamily,
title=]
    測度のなす空間の弱位相はBanach空間の弱位相の定義の15年前にRadonが考えていた.
    この測度代数(measure algebra)は$C(X)$(の部分環)の双対空間の初期に考えられた例である.
\end{tcolorbox}

\begin{definition}[uniform algebra, natural, Banach function algebra]\label{def-uniform-algebra}
    $X$をコンパクトハウスドルフ空間とする.
    \begin{enumerate}
        \item $X$上の連続関数のなす一様ノルムについての$C^*$-環$C(X)=C_b(X)$の閉部分環(したがってBanach代数)であって,すべての定数関数を含み,$X$の点を分離するものを\textbf{一様環}という.
        \item $X$上の一様環$A$の極大イデアルが,ある点$x\in X$について,そこで消失する関数のイデアル$M_x$であるとき,これを\textbf{自然}であるという.
        \item 一般に,ノルムを考えず,$C(X)$の部分代数を\textbf{関数代数}といい,一様環は$C(X)$とその関数代数に一様ノルムを入れた場合に当たる.
    \end{enumerate}
\end{definition}

\begin{corollary}
    単位的な可換Banach代数$A$が$\forall_{a\in A}\;\norm{a^2}=\norm{a}^2$を満たすとき,あるコンパクトハウスドルフ空間$X$が存在して,$A$はその上のある一様環と同型になる.
\end{corollary}

\begin{example}[測度のなす代数(measure algebra)]
    局所コンパクトハウスドルフな群$G$上のRadon測度全体$M(G)$はBanach空間をなす.
    これは畳み込みを積とするとBanach代数となる.
\end{example}

\subsubsection{非可換な例}

\begin{example}[群環:可測関数の代数で積を合成ではなく畳み込みとしたもの]
    局所コンパクト群$G$上のHaar測度$\mu$について,$\mu$-可積分関数全体はBanach空間$L^1(G)$をなす.
    積を畳み込みとすると,Banach代数となる.
\end{example}

\begin{example}[Banの内部ホム]\mbox{}
    \begin{enumerate}
        \item ノルム空間$X$の有界自己作用素全体の空間$B(X)$は,和・スカラー倍を各点で定め,積を合成とすることで,作用素ノルムについてノルム代数となり,$X$がBanach代数であるとき$B(X)$もBanach代数である.コンパクト作用素はこの閉イデアルをなす.随伴を対合として$C^*$-代数をなす.
        \item 行列代数$M_n(\C)$は作用素の代数$B(\C^n)$に同型であるから,単位的Banach代数である.
        \item 上三角行列全体の空間は,$M_n(\C)$の部分代数をなす.
    \end{enumerate}
\end{example}



\subsection{イデアルと商}

\begin{tcolorbox}[colframe=ForestGreen, colback=ForestGreen!10!white,breakable,colbacktitle=ForestGreen!40!white,coltitle=black,fonttitle=\bfseries\sffamily,
title=]
    非単位的環を扱う際に重要な橋渡しをする概念として,モジュラー性がある.
    イデアル$I$が非単位的環$\A$の\textbf{単模}であるとは,「(存在しない)単位元をあたかも含んでいるイデアル」あるいは「単位元を模したもの」として働くことをいう.
    どのように「存在しない単位元を含んでいるか否か」を捉えるかというと,$A/I$の元$p(I)$が単位元になるかで測る.
\end{tcolorbox}

\subsubsection{単模イデアル}

\begin{definition}[modular / regular]
    イデアル$I$が\textbf{単模}であるとは,$\exists_{u\in A}\;\forall_{a\in A}\;a-au\in I\land a-ua\in I$を満たすことをいう.
\end{definition}
\begin{remarks}[退化した消息]
    実は正しくは「単模イデアルを含む極大イデアルは存在する」であり,一般の単位的環$A$の任意のイデアルは単模であるから一般に極大イデアルを持つが,単位的とは限らない環については,必ずしも極大イデアルを持つとは限らない.
    「単模イデアル」と指定することで,単位元を含んだイデアルの世界だけを選び出して議論することができる.
\end{remarks}

\begin{lemma}[modular性の特徴付け]
    イデアル$I$がモジュラーであることと,$A/I$が単位的であることとは同値.
\end{lemma}

\begin{example}
    局所コンパクトハウスドルフ空間$\Om$上の任意の元$\om\in\Om$に対して,
    \[M_\om:=\Brace{f\in C_0(\Om)\mid f(\om)=0}\]
    はモジュラーイデアルである.
    実際,ある関数$u\in C_0(\Om)$で$u(\om)=1$を満たすものが存在するから,$\forall_{f\in C_0(\Om)}\;f-uf\in M_\om$.
    これは$M\oplus \C u=C_0(\Om)$を満たすから余次元$1$で,極大イデアルでもある.
\end{example}

\subsubsection{代数のイデアル}

\begin{lemma}[quotient]\label{lemma-quotient-of-Banach-algebra}
    $\A$をBanach代数,$\I\subset \A$をイデアルとする.
    \begin{enumerate}
        \item $\I$が閉集合であるとき,商$\A/\I$は再びBanach代数である.
        \item イデアル$\I$について,次の2条件は同値である.
        \begin{enumerate}[(a)]
            \item $\I$は閉集合である.
            \item $\I$はあるノルム減少的な連続線型作用素$\Phi:\A\to \A/\I$の核である.
        \end{enumerate}
    \end{enumerate}
\end{lemma}
\begin{Proof}\mbox{}
    \begin{enumerate}
        \item $\I$が閉集合であるとき,商$\A/\I$は商ノルムについて再びBanach空間である\ref{prop-quotient-Banach-space}.
        また,環$\A/\I$は積$(A+\I)(B+\I)=AB+\I$によって再び代数となる.よって,商ノルムの劣乗法性を示せば良い.
        \begin{align*}
            \norm{A+\I}\norm{B+\I}&=\inf_{S\in\I}\norm{A+S}\inf_{T\in\I}\norm{B+T}\\
            &\ge\inf_{S\in\I}\inf_{T\in\I}\norm{AB+(AT+SB+ST)}\ge\inf_{R\in\I}\norm{AB+R}=\norm{AB+\I}.
        \end{align*}
        \item 明らか.
    \end{enumerate}
\end{Proof}

\begin{corollary}[Banach代数の極大イデアルは閉]\label{cor-maximal-ideal-is-closed}
    $\A$を単位的なBanach代数とする.
    \begin{enumerate}
        \item イデアルのノルム閉包はイデアルである.
        \item 真のイデアルの閉包は再び真のイデアルである.
        \item 極大イデアルは閉である.
        \item $\A$を可換とする.任意の非可逆元$A\in\A\setminus\GL(\A)$に対して,ある極大イデアル$\I$が存在して,$A\in\I$が成り立つ.
    \end{enumerate}
\end{corollary}
\begin{Proof}
    $\GL(\A)$が開集合であること\ref{prop-GA-is-open-in-Banach-algebra}の系として従う.
    \begin{enumerate}
        \item 任意の$x,y\in\oo{I}$に対して,$I$のノルム収束列$(x_n),(y_n)$が取れる.Banach代数の積と和が連続であることから,$x+y=\lim_{n\to\infty}(x_n+y_n)\in\oo{I},ax=\lim_{n\to\infty}ax_n\in\oo{I}$.
        \item 真のイデアル$\I\subset\A$を取ると,$\I\cap\GL(\A)=\emptyset$である.
        すなわち,$\I\subset\A\setminus\GL(\A)$であるが,$\A\setminus\GL(\A)$は閉であるから$\o{\I}\subset\A\setminus\GL(\A)$.
        よって,$\o{\I}$も真のイデアルである.
        \item 任意の極大イデアル$I\subsetneq\A$を取る.
        $I\cap\GL(\A)=\emptyset$であるが,これは$\forall_{A\in I}\;\norm{I-A}\ge 1$を意味する($\norm{I-A}<1$ならば,$A\in\GL(\A)$で,Neumann級数による逆元の表示が可能\ref{lemma-Neumann-series}になってしまう).
        よって,$\oo{I}\ne\A$であるが,$\oo{I}$もイデアルであるから,$\oo{I}=I$でないと極大性に矛盾する.
        \item 任意に$A\in\A\setminus\GL(\A)$を取ると,$I\notin\A\cdot A$.よって,$A$はある真のイデアルの元である(例えば$\A A$).
        $A$を元に持つ真のイデアルの全体は空でない帰納的順序集合をなすから,Zornの補題より極大元が取れるが,これは極大イデアルになる.
    \end{enumerate}
\end{Proof}

\subsubsection{単模性の消息}

Banach代数$A$の可換性条件をおとし,より一般的に述べようとすると,単模イデアルの言葉を使って次のように表せる.

\begin{theorem}
    $I$をBanach代数$A$の単模イデアルとする.
    \begin{enumerate}
        \item $I$が真のイデアルならば,ノルム閉包$\oo{I}$も真のイデアルである.
        \item $I$が極大ならば,$I$は閉である.
    \end{enumerate}
\end{theorem}

\begin{lemma}
    $A$は単位的で可換なBanach代数,
    $I$が単模で極大なイデアルとする.このとき,$A/I$は体である.
\end{lemma}
\begin{Proof}
    極大スペクトルと指標との対応\ref{prop-correspondence-between-maximal-spectrum-and-characters}において示している.
\end{Proof}

\subsection{正則表現}

\begin{tcolorbox}[colframe=ForestGreen, colback=ForestGreen!10!white,breakable,colbacktitle=ForestGreen!40!white,coltitle=black,fonttitle=\bfseries\sffamily,
title=]
    群のCayley表現のように,代数の埋め込み$\A\mono B(\A)$が標準的に存在する.
    これは埋め込みで,近似的単位元が存在するときは等長でもある.

    代数的存在は他の数学的対称に作用させて研究する指針がある(表現論).\footnote{\url{http://nlab-pages.s3.us-east-2.amazonaws.com/nlab/show/regular+representation}}
    特に代数は加群に作用する.このとき,代数は,加群の特別なクラスだと思える,ちょうど群は台集合をもち,代数は台加群を持つのと同様に.
    すると,代数に備わる追加構造である積は,自身を忘却して得る加群の上に標準的な作用を定め,これを\textbf{正則表現}という.
\end{tcolorbox}

\begin{definition}[(left) regular representation]
    Banach代数$\A$の\textbf{左正則表現}$\rho:\A\to B(\A)$とは,左移動$\forall_{A,B\in\A}\;\rho(A)(B)=AB$をいう.
\end{definition}
\begin{lemma}
    $\rho:\A\mono B(\A)$はノルム減少的な代数の準同型である.
    また,$\A$が単位的ならば,$\rho$は位相の埋め込みでもある.
\end{lemma}
\begin{Proof}
    $\rho$が単射な代数準同型を定めることは明らか.
    \begin{description}
        \item[ノルム減少性] $\norm{\rho(A)}\le\norm{A}$は,作用素ノルムの定義
        \[\norm{\rho(A)}=\sup\Brace{\norm{\rho(A)(B)}=\norm{AB}\in\R_{\ge0}\mid B\in\A,\norm{B}\le 1}\]
        と劣乗法性より従う.
        \item[劣乗法性の保存] 作用素ノルムの劣乗法性より明らか:$\norm{\rho(AB)}=\norm{\rho(A)\rho(B)}\le\norm{\rho(A)}\norm{\rho(B)}$.
        が,$\A$が非単位的であるとき,位相の埋め込みとは限らないから,Banach代数の埋め込みであるとは言えない.
        \item[位相の埋め込み] 
        $\norm{A}=\norm{\rho(A)(I)}\le\norm{\rho(A)}\norm{I}\le\norm{A}\norm{I}$は$\rho:\A\to B(\A)$とその逆$\Im\rho\to\A$のLipschitz連続性を表していると読めるから,
        $\rho$は部分空間への位相同型である.
    \end{description}
\end{Proof}
\begin{remark}[adjointing identity from the regular representation]
    Banach代数の埋め込み$\rho:\A\mono B(\A)$を用いて,$\A$に$B(\A)$の作用素ノルムと同値なノルムを入れることで,$\A$が零でないとき$\norm{I}=1$と仮定して良い.
    以降,これを暗黙の仮定とする.
\end{remark}

\subsection{近似的単位元}

\begin{definition}[approximate unit/ identity]
    Banach代数$\A$の\textbf{近似的単位元}とは,単位球$B\subset\A$上のネット$(E_\lambda)_{\lambda\in\Lambda}$で,$\forall_{A\in\A}\;\lim E_\lambda A=\lim AE_{\lambda}=A$を満たすものを言う.
\end{definition}
\begin{lemma}[近似的単位元の特徴付け]
    $\A$をBanach代数とする.次の2条件は同値である.
    \begin{enumerate}
        \item 近似的単位元$(E_\lambda)$が存在する.
        \item ある有界集合$\E\subset\A$が存在して,$\forall_{\ep>0}\;\forall_{A\in\A}\;\exists_{E\in\E}\;\norm{AE-A}+\norm{EA-A}<\ep$.
    \end{enumerate}
\end{lemma}
\begin{lemma}
    Banach代数に近似的単位元が存在するとき,左正則表現$\rho$は等長同型である.
\end{lemma}
\begin{Proof}
    評価$\norm{AE_\lambda}\le\norm{\rho(A)}\norm{E_\lambda}\le\norm{A}\norm{E_\lambda}\le\norm{A}$の極限を考えることより従う.
\end{Proof}
\begin{example}
    局所コンパクトハウスドルフ空間上の無限遠点で消失する連続関数全体の空間$C_0(X)$,
    Hilbert空間上のコンパクト作用素の空間$B_0(H)$,局所コンパクトな群$G$上の可積分関数全体のなす空間$L^1(G)$は近似的単位元を持つ.
    $P_r:\partial\Delta\to\cointerval{0,\infty};z\mapsto\sum^\infty_{n=-\infty}r^{\abs{n}}z^n\;\;(r\in(0,1))$をPoisson核とすると,ネット$(P_r)_{r\in(0,1)}$は$L^1(\partial\Delta)$を畳み込みについてBanach代数とみたときの近似的単位元である.
\end{example}

\subsection{可逆元と乗法部分群}

\begin{tcolorbox}[colframe=ForestGreen, colback=ForestGreen!10!white,breakable,colbacktitle=ForestGreen!40!white,coltitle=black,fonttitle=\bfseries\sffamily,
title=]
    $\norm{I-A}<1$を満たすことは,$A\in\A$が可逆であり,かつ,逆元がvon Neumannの表示を持つための十分条件である.
\end{tcolorbox}

\begin{definition}[invertible]
    単位的なBanach代数$\A$において,元$A\in\A$が\textbf{可逆}であるとは,ある$B,C\in\A$が存在して$BA=AC=I$を満たすことを言う.
    このとき$B=C$が従うが,これを$A^{-1}$と表す.可逆元全体の集合を$\GL(\A)=\A^\times$で表す.
\end{definition}

\begin{lemma}[Neumann seriesによる可逆元の表示が可能な十分条件]\label{lemma-Neumann-series}
    単位的Banach代数$\A$の元$A$が$\norm{A}<1$を満たすとき,
    \begin{enumerate}
        \item $I-A\in\GL(\A)$で,
        \item $(I-A)^{-1}=\sum^\infty_{n=0}A^n$と表せる.
    \end{enumerate}
\end{lemma}
\begin{Proof}
    ノルムの劣乗法性より$\norm{A^n}\le\norm{A}^n$より,$\sum_{n=0}^\infty\norm{A^n}<\infty$.よって,Banach代数$\A$の完備性より,$\sum_{n=0}^{\infty}A^n$も,ある元$B\in\A$に収束する.
    $B_n:=\sum_{i=0}^nA^i$とおくと,$AB_n=B_{n+1}-I$より,$AB=\lim_{n\to\infty}AB_n=\limn B_{n+1}-I=B-I$.同様に$BA=B-I$.
    よって,$I-A$は可逆で,$(I-A)^{-1}=B$.
\end{Proof}

\begin{proposition}\label{prop-GA-is-open-in-Banach-algebra}
    単位的Banach代数$\A$において,乗法群$\GL(\A)$について次が成り立つ.
    \begin{enumerate}
        \item $\GL(\A)$は開集合である.
        \item 写像$A\mapsto A^{-1}$は$\GL(\A)$の位相同型を定める(よって$\GL(\A)$は位相群をなす).
        \item また,この写像は可微分でもある.
    \end{enumerate}
\end{proposition}
\begin{Proof}\mbox{}
    \begin{enumerate}
        \item 任意に$A\in\GL(\A)$を取る.
        \[\forall_{B\in\A}\quad B=A-(A-B)=A(I-A^{-1}(A-B))\]
        より,補題から,$\norm{A^{-1}(A-B)}<1$ならば,$B\in\GL(\A)$である.
        特に,$\ep<\norm{A^{-1}}^{-1}(>0)$を満たす$\ep>0$を取れば,$B(A,\ep)\subset\GL(\A)$を満たす.
        \item 任意の$A,B\in\GL(\A)$について,
        \[B^{-1}=(A(I-A^{-1}(A-B)))^{-1}=\paren{\sum^\infty_{n=0}(A^{-1}(A-B))^n}A^{-1}\]
        が成り立つ.よって,$B\to A$のとき,$B^{-1}\to A^{-1}$.
        よって,${}^{-1}:\GL(\A)\to\GL(\A)$は連続写像.これが対合であることより,同相写像でもある.
        \item \cite{Murphy}
    \end{enumerate}
\end{Proof}

\subsection{スペクトル}

\begin{tcolorbox}[colframe=ForestGreen, colback=ForestGreen!10!white,breakable,colbacktitle=ForestGreen!40!white,coltitle=black,fonttitle=\bfseries\sffamily,
title=]
    いままで代数っぽかったが,ここでいきなり幾何的な空間$\C$に落とす.
    このときの$\C\to X$への帰還.

    任意のBanach代数の元のスペクトルが空でないことを示すにあたって,
    以降,体を$\bF=\C$とする.
    すると,レゾルベントを通じて複素解析学の道具が流入する(Fredholm 1903).
    というよりむしろ,複素解析学が,$\C$の領域上のBanach空間値関数に一般化される.
\end{tcolorbox}

\subsubsection{スペクトルの定義と例}

\begin{tcolorbox}[colframe=ForestGreen, colback=ForestGreen!10!white,breakable,colbacktitle=ForestGreen!40!white,coltitle=black,fonttitle=\bfseries\sffamily,
title=]
    スペクトルとは,即時的には関数の値域と,行列の固有値との2つの概念の一般化である.
    が,局所コンパクトなアーベル群上の可積分関数のなす畳み込みBanach代数$L^1(\T)$や$L^1(\R)$の元にとっては,Fourier級数全体の集合であるから,その一般化とも見れる(こちらが名前の由来である)
    \ref{remark-spectrum-as-Fourier-coefficient}.
\end{tcolorbox}

\begin{definition}[spectrum, spectral radius, resolvent set, resolvent]
    $\A$を単位的Banach代数とする.
    \begin{enumerate}
        \item 元$A\in\A$の\textbf{スペクトル}とは,複素数の部分集合$\Sp(A):=\Brace{\lambda\in\C\mid\lambda I-A\notin\GL(\A)}$をいう.
        \item 元$A\in\A$の\textbf{スペクトル半径}とは,実数$r(A):=\sup\Brace{\abs{\lambda}\in\R_{\ge 0}\mid\lambda\in\Sp(A)}=\min\Brace{r\in\R_{\ge0}\mid\Sp(A)\subset B(0,r)}$を指す.
        \item 補集合$\rho(A):=\C\setminus\Sp(A)$を\textbf{解核集合}という.
        \item 解核集合上の関数$R(A,\lambda):=(\lambda I-A)^{-1}:\C\setminus\Sp(A)\to\A$を\textbf{解核}という.
    \end{enumerate}
\end{definition}

\begin{example}[スペクトルとは,関数の値域と,行列の固有値との2つの概念の一般化である]\mbox{}
    \begin{enumerate}
        \item コンパクトハウスドルフ空間$\Om$上の複素連続関数のBanach代数$C(\Om)$について,スペクトルとは像$\sigma(f)=f(\Om)=\Im f\;(f\in C(\Om))$となる.
        \item 非空集合$S$上の有界関数全体のなすBanach代数$l^\infty(S)$について,スペクトルとは像の閉包$\sigma(f)=\o{f(\Om)}=\o{\Im f}\;(f\in l^\infty(S))$となる.上の例は退化していたのだ.
        \item $A\subset M_n(\C)$を上三角行列のなすBanach代数とすると,スペクトルは対角成分の集合$\sigma(a)=\Brace{\lambda_{11},\lambda_{22},\cdots,\lambda_{nn}}\;(a\in A)$となる.
        \item 正方行列のなすBanach代数$M_n(\C)$について,スペクトルは固有値全体の集合となる.
    \end{enumerate}
\end{example}
\begin{remark}[スペクトルは非可換性を区別しない]
    単位的代数$A$について,$1-ab$が可逆であることと$1-ba$が可逆であることとは同値.実際,$1-ab$の逆元を$c$とすると,$1+bca$は$1-ba$の逆元になる.
    その結果,$\sigma(ab)\setminus\{0\}=\sigma(ba)\setminus\{0\}$が成り立つ.
    ここからGelfand変換の香りがする.
\end{remark}

\begin{example}[レゾルベント]
    一般の行列$A$の場合\ref{remarks-diagonalizability}で考えると,
    $(A-\lambda I)^{-1}$は$\C\setminus\{\lambda_1,\cdots,\lambda_m\}$上の正則関数で,核$\lambda_i$は$k_i$次の極であり,次のようにLaurent(?)展開できる:
    \[(A-\lambda I)^{-1}=-\sum^m_{i=1}\paren{\frac{1}{\lambda-\lambda_i}\wt{P}_i+\frac{1}{(\lambda-\lambda_i)^2}N_i\wt{P}_i+\cdots+\frac{1}{(\lambda-\lambda_i)^{k_i}}N_i^{k_i-1}\wt{P}_i}\]
    従って,今後もスペクトルに注目するが,これは双対的にはレゾルベントなる正則関数の特異点を探す問題でもある.
    $\Sp(A)$は閉なので,$\rho(A)$は開集合であることに注意.
\end{example}

\begin{example}[quasinilpotent]
    冪零作用素のスペクトル半径は$0$であるが,他にも例がある.
    そこで,$\Sp(N)=\{0\}$となる作用素$N$を\textbf{準冪零}であるという.
\end{example}

\subsubsection{スペクトル写像}

\begin{lemma}[spectral mapping theorem:単位的Banach代数のスペクトルと正則関数の可換性]
    $\bigcup_{A\in\A}B(0,\norm{A})$を含む領域上で定義された正則関数$f:\C\supset U\to\C;z\mapsto\sum_{n=0}^\infty\al_nz^n$が定める
    単位的Banach代数の自己写像$f:\A\to\A$について,$\lambda\in B(0,\norm{A})\cap\Sp(A)\Rightarrow f(\lambda)\in\Sp(f(A))$:$f(\Sp(A)\cap \Delta(0,r))\subset\Sp(f(A))$.
\end{lemma}
\begin{Proof}\mbox{}
    \begin{enumerate}
        \item $\sum_{n=0}^\infty\abs{\al_n}\norm{A^n}<\infty$と$\A$の完備性より,写像$f$はwell-definedである.
        \item 対偶を示す.任意の$\lambda\in\C$について,
        $P_{n-1}\in\C[A]$を$P_{n-1}(\lambda,A):=\sum_{k=0}^{n-1}\lambda^kA^{n-k-1}$と定めると,
        \[\norm{P_{n-1}(\lambda,A)}\le\sum^{n-1}_{k=0}\abs{\lambda}^k\norm{A}^{n-k-1}\le nr^{n-1}\]
        より,列$(P_{n-1}(\lambda,A))$は,ある$A$と可換な元$B\in\A$に収束する.
        また,
        \[f(\lambda)I-f(A)=\sum^\infty_{n=1}\al_n(\lambda^nI-A^n)=(\lambda I-A)\sum^\infty_{n=1}\al_nP_{n-1}(\lambda,A)=(\lambda I-A)B.\]
        よって,$f(\lambda)I-f(A)\in\GL(\A)$で逆元$C$を持つならば,$BC$は$\lambda I-A$の逆元となる.
    \end{enumerate}
\end{Proof}
\begin{remarks}
    この形は,実は単位的$C^*$-代数の正則元$a$と連続写像$f\in C(\sigma(a))$に関する性質$\sigma(f(a))=f(\sigma(a))$の前身であるとみれる.
\end{remarks}

\begin{theorem}[一般の代数に関するスペクトル写像定理]
    $A$を単位的代数とし,$p\in\C[z]$を多項式とすると,$\sigma(p(a))=p(\sigma(a))\ne\emptyset$.\cite{Murphy}
\end{theorem}
\begin{remarks}
    $\sigma:\A\to P(\C)$と正則関数$f:\A\to\A$とは可換である.
\end{remarks}

\subsubsection{スペクトル半径の特徴付け}

\begin{lemma}[Beurling]
    任意の単位的Banach代数の元$A\in\A$に関して,$r(A)\le\inf_{n\in\N}\norm{A^n}^{1/n}$.
    特に,$\Sp(A)$は有界である.
\end{lemma}
\begin{Proof}\mbox{}
    \begin{enumerate}[(a)]
        \item 補題\ref{lemma-Neumann-series}より,
        \[\forall_{\lambda\in\C}\quad\abs{\lambda}>\norm{A}\Rightarrow(\lambda I-A)^{-1}=\lambda^{-1}(I-\lambda^{-1}A)^{-1}=\sum^\infty_{n=0}\lambda^{-n-1}A^n\]
        だから,$\abs{\lambda}>\norm{A}\Rightarrow\lambda I-A\in\GL(\A)$,すなわち,$\Sp(A)\subset\Delta(0,r)$,$r(A)<r$.
        よって,$r(A)\le\norm{A}$.
        \item 任意の$\lambda\in\Sp(A)$を取る.$x\mapsto x^n$は整関数だから,補題より,$\lambda^n\in\Sp(A^n)$.(a)での議論より,$\abs{\lambda}^n\le\norm{A^n}\Leftrightarrow\abs{\lambda}\le\norm{A^n}^{1/n}$.
        実際,$\abs{\lambda^n}>\norm{A^n}$ならば,$\lambda^nI-A^n\notin\GL(\A)$に矛盾.
    \end{enumerate}
\end{Proof}
\begin{example}
    $A$を$C^1([0,1])$にノルム$\norm{f}:=\norm{f}_\infty+\norm{f'}_\infty$を定めて得るBanach代数とする.
    包含$x:[0,1]\mono\C$は$A$の元であり,明らかに$r(x)=1$である.
    ここで,$\norm{x^n}=1+n$であるから,$r(x)=\lim_{n\to\infty}(1+n)^{1/n}=1<2=\norm{x}$とも合致する.
\end{example}

\begin{theorem}\label{thm-Spectrum-is-compact}
    単位的Banach代数の任意の元$A\in\A$に関して,
    \begin{enumerate}
        \item スペクトル$\Sp(A)\subset\C$は非空なコンパクト集合である.\footnote{なお,Banach代数が単位的でない場合は,局所コンパクト性のみが成り立つ.}
        \item $A$のスペクトル半径は$r(A)=\lim_{n\to\infty}\norm{A^n}^{1/n}$と表せる.
    \end{enumerate}
\end{theorem}
\begin{Proof}\mbox{}
    \begin{enumerate}
        \item 
        \begin{description}
            \item[コンパクト性] 任意の$A\in\A$を取り,$R:\C\setminus\Sp(A)\to\C$をそのレゾルベントとする.
            任意の$\lambda\notin\Sp(A)$について,$\abs{\zeta}<\norm{R(\lambda)}^{-1}$を満たすように取れば,補題\ref{lemma-Neumann-series}より,
            $\lambda-\zeta\notin\Sp(A)$で,その逆元は
            \begin{align*}
                R(\lambda-\zeta)&=(\lambda I-A-\zeta I)^{-1}\\
                &=((\lambda I-A)(I-R(\lambda)\zeta))^{-1}=\sum^\infty_{n=0}R(\lambda)^{n+1}\zeta^n
            \end{align*}
            と表せる.特に,$\rho(A)=\C\setminus\Sp(A)$は開集合であるから,$\Sp(A)$は閉集合である.補題と併せて,$\Sp(A)$はコンパクトである.
            \item[非空]
            \begin{enumerate}[(a)]
                \item 任意に有界連続汎関数$\varphi\in\A^*$を取り,$f(\lambda):=\varphi(R(\lambda))$と定めると,
                各点$\lambda\in\rho(A)$において,$r>0$が存在して,
                \[f(\lambda-\zeta)=\sum^\infty_{n=0}\varphi(R(\lambda)^{n+1})\zeta^n\quad(\zeta\in\Delta(\lambda,r))\]
                と冪級数表示できるから,$f:\C\setminus\Sp(A)\to\C$は正則関数である.

                $\abs{\lambda}>\norm{A}$とする.
                補題の(a)での議論の通り,$f(\lambda)=\sum_{n=0}^\infty\lambda^{-n-1}\varphi(A^n)$だから,
                \begin{align*}
                    \abs{f(\lambda)}&\le\sum^\infty_{n=0}\abs{\lambda}^{-n-1}\norm{A}^n\norm{\varphi}\\
                    &=\abs{\lambda}^{-1}\norm{\varphi}(1-\abs{\lambda}^{-1}\norm{A})^{-1}=\norm{\varphi}(\abs{\lambda}-\norm{A})^{-1}.
                \end{align*}
                これより,$\abs{f(\lambda)}\xrightarrow{\abs{\lambda}\to\infty}0$.
                \item $\Sp(A)=\emptyset$と仮定して矛盾を導く.このとき$f$は$C_0(\C)$に属する整関数であるが,Liouvilleの定理より,これは定数関数であることが必要だから,$f=0$である.
                したがって,$\forall_{\varphi\in\A^*}\;\varphi((\lambda I-A)^{-1})=0$.
                よって系\ref{cor-Hahn-Banach}より,$(\lambda I-A)^{-1}=0$が必要であるが,これは矛盾.
            \end{enumerate}
        \end{description}
        \item \begin{enumerate}[(a)]
            \item (1)で定義した正則関数$f:\C\setminus\Sp(A)\to\C$は,$\abs{\lambda}>\norm{A}$の範囲で局所的な冪級数展開$f(\lambda)=\sum^\infty_{n=0}\lambda^{-n-1}\varphi(A^n)$を持つ.
            正則関数$f(\lambda^{-1})$は少なくとも$\Delta(0,r(A)^{-1})$上で定義されており,対応する冪級数展開はCauchyの積分表示より,この領域内で広義一様収束する.
            したがって,任意の$r>r(A)$について,$f$の冪級数表示も$\Brace{\lambda\in\C\mid\abs{\lambda}>r}$上で一様収束する(Laurent展開の議論と並行).
            \item よって,$\lambda:=re^{i\theta}$とおくと,正則関数$\lambda^{n+1}f(\lambda)$は$\partial\Delta(0,r)$上で次のように項別積分出来る:
            \begin{align*}
                \int^{2\pi}_0r^{n+1}e^{i(n+1)\theta}f(re^{i\theta})d\theta&=\sum^\infty_{m=0}\int^{2\pi}_0r^{n-m}e^{i(n-m)\theta}\varphi(A^m)d\theta\\
                &=2\pi\varphi(A^n).&m=n\text{の時を除いて積分は}0
            \end{align*}
            また最左辺の積分は
            $M(r):=\sup_{\theta\in[0,2\pi]}\norm{R(re^{i\theta})}$とおくことで
            \begin{align*}
                \int^{2\pi}_0r^{n+1}e^{i(n+1)\theta}f(re^{i\theta})d\theta&\le r^{n+1}\sup_{\theta\in[0,2\pi]}\abs{\varphi(R(re^{i\theta}))}2\pi\\
                &\le r^{n+1}\norm{\varphi}M(r)2\pi
            \end{align*}
            と評価できるから,$\varphi(A^n)\le r^{n+1}M(r)\norm{\varphi}$を得る.
            \item (b)の議論は$\varphi\in\A^*$を任意としたから,特に$\norm{\varphi}=1,\norm{\varphi(A^n)}=\norm{A^n}$をみたすものについて(系\ref{cor-Hahn-Banach}より存在する),
            $\norm{A^n}\le r^{n+1}M(r)\Leftrightarrow\norm{A^n}^{1/n}\le r(rM(r))^{1/n}$.
            よって,$\limsup_{n\to\infty}\norm{A^n}^{1/n}\le r$.$r>r(A)$は任意に取ったから,
            $\limsup_{n\to\infty}\norm{A^n}^{1/n}\le r(A)$.補題より$r(A)\le\liminf_{n\to\infty}\norm{A^n}^{1/n}$と併せると,
            $\norm{A^n}^{1/n}$は収束して,$\lim_{n\to\infty}\norm{A^n}^{1/n}=r(A)$.
        \end{enumerate}
    \end{enumerate}
\end{Proof}
\begin{remarks}
    乗法群$\GL(\A)$が開集合となるような単位的位相代数$\A$を$Q$-代数といい,一般に$Q$-代数の任意の元のスペクトルは非空でコンパクトになる.
    なお,非空なコンパクト集合$K\subset\C$について,補集合$\C\setminus K$の非有界な連結成分はただ一つである.
    $\C\setminus K$の有界な連結成分を\textbf{穴(hole)}という.
\end{remarks}

\begin{corollary}[Gelfand-Mazur theorem (41)]\label{cor-Gelfand-Mazur-thm}
    $\A$が斜体であるとき,すなわち,$\GL(\A)=\A\setminus\{0\}$を満たすとき,$\A=\C$である.
\end{corollary}
\begin{Proof}
    定理より,任意の$A\in\A$について,$\lambda\in\Sp(A)\ne\emptyset$.
    すなわち,$\lambda I-A\notin\GL(\A)$であるが,$\A$が可除環であるとき,これは$A=\lambda I$を意味する.
\end{Proof}
\begin{remarks}
    $\C$以外の複素Banach代数については,開集合$\GL(\A)$は$\A$より真に小さい.
\end{remarks}

\subsubsection{閉部分代数のスペクトル}

\begin{theorem}[閉部分代数のスペクトル]
    $A$を単位的Banach代数,$B$をその単位元を含む閉部分代数とする.
    \begin{enumerate}
        \item $\GL(B)$は$B\cap\GL(A)$の開かつ閉集合である.
        \item $\forall_{b\in B}\;\sigma_A(b)\subset\sigma_B(b)$かつ$\partial\sigma_B(b)\subset\partial\sigma_A(b)$.
        \item $b\in B$について$\sigma_A(b)$が穴をもたないとき,$\sigma_A(b)=\sigma_B(b)$.
    \end{enumerate}
\end{theorem}

\subsubsection{指数関数}

\begin{tcolorbox}[colframe=ForestGreen, colback=ForestGreen!10!white,breakable,colbacktitle=ForestGreen!40!white,coltitle=black,fonttitle=\bfseries\sffamily,
title=]
    行列の指数関数の発想を一般化する.
\end{tcolorbox}

\begin{notation}\label{not-extended-exp}
    単位的Banach代数$A$の元$a\in A$について,
    \[\sum^\infty_{n=0}\Norm{\frac{a^n}{n!}}\le\sum^\infty_{n=0}\frac{\norm{a}^n}{n!}<\infty\]
    より,級数$\sum^\infty_{n=0}\frac{a^n}{n!}$は収束する.こうして,$e:\A\to\A;a\mapsto e^a$を定める.
\end{notation}

\begin{theorem}
    単位的Banach代数$A$について,
    \begin{enumerate}
        \item 写像$f:\R\to A$は可微分でえ,微分方程式$f(0)=1,f'(t)=af(t)\;(a\in A)$を満たすとする.このとき,$f(t)=e^{ta}$である.
        \item $e^a\;(a\in A)$は可逆で,$e^{-a}$は逆元である.
        \item $a,b$が可換ならば,$e^{a+b}=e^ae^b$.
    \end{enumerate}
\end{theorem}
\begin{remark}
    $e^a$の形で表せることは,可逆性の十分条件だが,必要条件ではない.
\end{remark}

\section{Gelfand変換}

\begin{tcolorbox}[colframe=ForestGreen, colback=ForestGreen!10!white,breakable,colbacktitle=ForestGreen!40!white,coltitle=black,fonttitle=\bfseries\sffamily,
title=]
    ここでは,Banach代数$\A$は複素係数で単位的で可換であるとする.
    $\A$に対して,$i:C(X;\C)\mono\A$と表現できるような位相空間$X$を探す問題を考える.
    このとき,$\Im i$は可換になるから,可換な$C^*$-代数に限って理論を構築することは自然である.
    Gelfand変換$\Gamma:\A\to C(\wh{\A})$はある条件下で可逆になり,一つの解決を与える.
\end{tcolorbox}

\subsection{指標の空間}

\begin{tcolorbox}[colframe=ForestGreen, colback=ForestGreen!10!white,breakable,colbacktitle=ForestGreen!40!white,coltitle=black,fonttitle=\bfseries\sffamily,
title=]
    Gelfand双対性とは,
    可換な$C^*$代数の構造は,その1次元表現$\A\epi\C$によって特徴付けられる,という理論である.
    1次元表現$\A\epi\C$のことを指標といい,一般の群についても定義される.
    なお,1次元表現$\epi\bF$といえば,確率測度についてもCramer-Woldが成り立ったりする.
\end{tcolorbox}

\begin{definition}[character, Gelfand spectrum]\label{def-character}
    $\A$を単位的$C^*$-代数とする.
    \begin{enumerate}
        \item (一般には群)$\A$の\textbf{指標}とは,全射(即ち零でない)連続線型準同型$\A\epi\C$のことをいう.\footnote{一般の群と体について定義される.}
        Banach代数については,任意の準同型は連続であることに注意.また,Fourier変換は指標による積を平行移動に写すのであった.
        \item $\A$の指標全体の空間$\hat{\A}:=\Brace{\gamma\in\Hom_{\Ban\Alg}(\A,\C)\mid\dim\Im\gamma=1}$を\textbf{指標空間}または\textbf{スペクトル}という.
        この空間には標準的な位相・\textbf{スペクトル位相}はコンパクトハウスドルフであり,$\A$が単位的でないときこれは高々局所コンパクトとなる.
        この対応は単位的な$C^*$-代数からHausdorff位相空間の圏への関手$\Spec:C^*\Alg^\op\to\Top$を定め,Gelfandスペクトルと呼ばれ,可換なもの$C^*\Alg^\op_\com$に制限すると圏の同値となる.
    \end{enumerate}
\end{definition}
\begin{remark}[他の数学的対象との関係]
    冪集合$P(X)$を$2$上の線型空間だと思えば,集合論の特性関数の用語と一致する.
    また,指標の核は余次元1の部分空間で,また極大イデアルである.
    こうして,Gelfandスペクトルは,環の極大スペクトルの位相的な類似物だと思える.
    そこで,Banach代数$\A$の極大イデアル全体のなす空間を$M(\A)$で表す.
\end{remark}

\begin{proposition}[極大スペクトルとの対応]\label{prop-correspondence-between-maximal-spectrum-and-characters}
    $\A$を可換で単位的なBanach代数とする.
    \begin{enumerate}
        \item 次の対応は集合の同型を定める:
        \[\xymatrix@R-2pc{
            \wh{\A}\ar[r]&M(\A)\\
            \rotatebox[origin=c]{90}{$\in$}&\rotatebox[origin=c]{90}{$\in$}\\
            \gamma\ar@{|->}[r]&\Ker\gamma
        }\]
        \item $\A$の任意の指標$\gamma:\A\mono\C$は連続であり,任意の極大イデアル$I\in M(\A)$は閉である.
        \item 任意の$A\in\A$について,$\Sp(A)=\Brace{\lambda\in\C\mid\lambda I-A\notin\GL(\A)}=\Brace{\brac{A,\gamma}=\gamma(A)\in\C\mid\gamma\in\wh{\A}}$.
    \end{enumerate}
\end{proposition}
\begin{Proof}\mbox{}
    \begin{enumerate}
        \item 任意に$\I\in M(\A)$を取り,商Banach代数$\A/\I$を考えると,これは体になる.単位的Banach代数が可除ならば$\C$である(Gelfand-Mazur\ref{cor-Gelfand-Mazur-thm})から,$\A/\I=\C$.
        よって,この商写像$\gamma:\A\to\A/\I$を対応させれば,これは指標である.また,商写像は連続であることに注意.
        
        逆に,任意の$\gamma\in\wh{\A}$を取ると,$\Ker\gamma$は$\A$の余次元1のイデアルとなるから,従って極大イデアルである.
        こうして,全単射が定まった.
        \item 任意の極大イデアル$\Ker\gamma$は閉\ref{cor-maximal-ideal-is-closed}より,任意の指標$\gamma:\A\to\A/\Ker\gamma$は連続である.
        \item 任意の$\lambda\in\Sp(A)$を取ると,$\lambda I-A\notin\GL(\A)$.このとき,ある極大イデアル$\I$が存在して,$\lambda I-A\in I$が成り立つ\ref{cor-maximal-ideal-is-closed}(4).
        (1)の対応より,ある指標$\gamma$が存在して,$\lambda I-A\in\Ker\gamma$すなわち$\brac{\lambda I-A,\gamma}=0$.
        このとき$\gamma(\lambda I)-\gamma(A)=0\Leftrightarrow\lambda=\gamma(A)$より,$\lambda=\brac{A,\gamma}$.

        逆に,任意の$\gamma\in\wh{\A}$について,$\lambda:=\brac{A,\gamma}$とすると,$\gamma(\lambda I-A)=\lambda-\gamma(A)=0$より,$\lambda I-A\in\Ker\gamma$.すなわち,$\lambda I-A\notin\GL(\A)$.
    \end{enumerate}
\end{Proof}

\subsection{Gelfand変換}

\begin{tcolorbox}[colframe=ForestGreen, colback=ForestGreen!10!white,breakable,colbacktitle=ForestGreen!40!white,coltitle=black,fonttitle=\bfseries\sffamily,
title=]
    一般の可換で単位的なBanach代数$\A$に対しては,Gelfand変換$\Gamma:\A\to C(\wh{\A})$が単射であるかどうかも定かでないようだ.
\end{tcolorbox}

\begin{theorem}[Gelfand変換 (Gelfand 1941)]\label{thm-Gelfand}
    $\A$を可換な単位的Banach代数とする.
    \begin{enumerate}
        \item 指標のなす集合$\hat{\A}$はコンパクトハウスドルフ位相を備える.
        \item 写像
        \[\xymatrix@R-2pc{
            \Gamma:\A\ar[r]&C(\wh{\A})\\
            \rotatebox[origin=c]{90}{$\in$}&\rotatebox[origin=c]{90}{$\in$}\\
            A\ar@{|->}[r]&\Gamma(A)=\hat{A}:=\brac{A,\gamma}
        }\]
        はノルム減少的な代数の準同型である.
        \item 像$\Im\Gamma<C(\hat{\A})$は部分代数であるが,$\hat{\A}$の点を分離する.
        \item 任意の$A\in\A$について,$\hat{A}(\hat{\A})=\Sp(A)$かつ$\norm{\hat{A}}_\infty=r(A)$.\footnote{$\A$が非単位的な場合は,$\Sp(A)=\wh{A}(\wh{\A})\cup\{0\}$となる.}
    \end{enumerate}
\end{theorem}
\begin{remarks}
    一般にスペクトルを,指標上の複素連続関数の値域として特徴づけることができた.
    これがどうやって固有値に繋がるのか?
\end{remarks}

\begin{proposition}[Gelfand変換の核]
    Gelfand変換$\Gamma$の核は,$\A$の根基である:
    $\Ker\Gamma=R(\A)=\bigcap_{\I\in M(\A)}\I=\Brace{A\in\A\mid r(A)=0}$.
\end{proposition}
\begin{remark}[semisimple]
    ほとんどの古典的なBanach代数は半単純である($R(\A)=\{0\}$)から,
    $R(\A)=\A$や$R(\A)$が$\A$の極大イデアルになる場合には当てはまらない.
    しかしながら,像$\Im\Gamma<C(\hat{\A})$を決定する問題は極めて難しい場合が多い.
    使える知識といえば,(3)の$\hat{\A}$の点を分離するということくらい(従って小さすぎない)である場合も多いので,
    そんなにうまくいくことばかりではない.
\end{remark}

\subsection{例}

\begin{example}[$C(X)$のGelfand変換は恒等写像である]
    $X$をコンパクトハウスドルフ空間,$\A:=C(X)$をその上の連続な複素関数を,各点和と積によって代数とみて,一様ノルムでBanach空間とみたBanach代数とする.
    単射$\iota:X\mono\wh{\A}$が,各点$x\in X$での評価を返す写像として$\brac{f,\iota(x)}=f(x)\;(x\in X,f\in C(X))$として定まる.
    これは$w^*$-位相を定める埋め込みに他ならない.つまり,
    空間$\wh{\A}\subset \A^*$には$w^*$-位相を入れると,写像$\iota:X\mono\wh{\A}$は連続で,さらに位相の埋め込みでもある.
    
    実は$\wh{\A}=\iota(X)$であることを示す.
    いま,任意の$\gamma\in\wh{\A}\setminus\iota(X)$について,$\forall_{x\in X}\;\exists_{f\in\A}\;\brac{f,\gamma}=0\land f(x)\ne0$が成り立つ.
    このとき,$X$のコンパクト性より,有限集合$\{f_1,\cdots,f_n\}\subset\Ker\gamma\subset\A$が存在して,余零集合$U_k:=\Brace{x\in X\mid f_k(x)\ne0}$が$X$を被覆する.
    略.

    以上より,$\wh{\A}=\iota(X)$であるから,Gelfand変換は$\A=C(X)\to C(\wh{\A})=C(\iota(X))$.
\end{example}

\begin{example}[絶対収束数列と連続な周期関数の間のGelfand変換]
    $\A:=l^1(\Z)$を畳み込み積$(AB)_n:=\sum^\infty_{-\infty}A_kB_{n-k}$によりBanach代数と見ると,単位的で可換である.
    また,ある$E\in\A$が$E_1=1,\forall_{n\ne1}\;E_n=0$を満たすなら,これは$\A$の生成元である:$\forall_{A\in\A}\;A=\sum^\infty_{-\infty}A_nE^n$.
    $E^2_2=1,E^2_n=0\;(n\ne2)$であることに注意.
    したがって,指標$\gamma:\A\to\C$は$E$の行き先を決めるごとに定まる.

    よってGelfand変換は
    \[\xymatrix@R-2pc{
        \Gamma:l^1(\Z)\ar[r]&C(\T)\\
        \rotatebox[origin=c]{90}{$\in$}&\rotatebox[origin=c]{90}{$\in$}\\
        A\ar@{|->}[r]&\wh{A}(\lambda)=\sum^\infty_{-\infty}A_n\lambda^n
    }\]
    となり,その像$\Gamma(\A)\subset C(\T)$は,Fourier級数が絶対収束する$[0,2\pi]$上の連続周期関数の集合となる.

    応用として,Wienerは,$[0,2\pi]$上の絶対収束するFourier級数を持つ連続な周期関数$f$が$\forall_{x\in[0,2\pi]}\;f(x)\ne0$を満たすとき,$f^{-1}$も絶対収束するFourier級数を持つことを示した.
    ある$A\in l^1(\Z)$が存在して$f=\wh{A}$で,仮定より$0\notin\Sp(A)$なので,$A^{-1}\in l^1(\Z)$かつ$\wh{(A^{-1})}=\wh{A}^{-1}=f^{-1}$が従う.
\end{example}

\begin{example}[絶対収束数列の部分代数と絶対収束するTaylor級数を持つ正則関数の間のGelfand変換]
    先程の$l^1(\Z)$の部分代数$l^1$を考える.生成元$E\in l^1$はこの部分代数では可逆元ではない.
    任意の$\lambda\in[\Delta]$について,指標を
    \[\brac{A,\gamma}:=\sum^\infty_{n=0}A_n\lambda^n\]
    と定めると,$\wh{\A}=\Delta$となる.

    よって,Gelfand変換は
    \[\xymatrix@R-2pc{
        \Gamma:l^1\ar[r]&C([\Delta])\\
        \rotatebox[origin=c]{90}{$\in$}&\rotatebox[origin=c]{90}{$\in$}\\
        A\ar@{|->}[r]&\wh{A}(\lambda)=\sum^\infty_{n=0}A_n\lambda^n
    }\]
    となり,その像$\Gamma(l^1)$は$\Delta$上の正則関数で,Taylor級数が絶対総和可能である関数の空間となる.
\end{example}

\begin{example}[可積分関数のBanach代数に関するGelfand変換はFourier変換に他ならない]
    Banach空間$L^1(\R)$を,合成積
    \[(f\times g)(x):=\int f(y)g(x-y)dy\]
    によってBanach代数とみる.Dirac測度$\delta_0$は合成積に関する中立元として,付加した単位的Banach代数$\A:=L^1(\R)+\C\delta$を考える.


\end{example}
\begin{remark}[Fourier係数全体の集合としてのスペクトル]\label{remark-spectrum-as-Fourier-coefficient}
    $\A=L^1(\T)$とすると$\wh{\A}=\Z$で,
    \[\forall_{A\in L^1(\T),n\in\Z}\;\wh{A}(n)=\int A(x)\exp(-inx)dx\]
    となるから,$\Sp(A)=\wh{A}(\Z)$とは$A$のFourier係数全体の集合に他ならない.

    $\A=L^1(\R)$とすると$\wh{\A}=\R$で,
    \[\Sp(A)=\wh{A}(\R)=\Brace{\int A(x)\exp(-ixy)dx\;\middle|\;y\in\R}\]
    は,$A$のスペクトル分析をするのに欠かせない情報である.
\end{remark}

\section{関数の代数}

\begin{tcolorbox}[colframe=ForestGreen, colback=ForestGreen!10!white,breakable,colbacktitle=ForestGreen!40!white,coltitle=black,fonttitle=\bfseries\sffamily,
title=]
    $C^*$-代数は,Banach代数のうち特に振る舞いのよいもので,Hilbert空間上の作用素の理論を流入させることが出来る.
    そこでは,Stone-Weierstrassの定理を用いて,Gelfand変換の像$\Gamma(\A)$が関数代数$C(\wh{\A})$の中で稠密であることが示せる.
\end{tcolorbox}

\subsection{Stone-Weierstrassの定理}

\begin{tcolorbox}[colframe=ForestGreen, colback=ForestGreen!10!white,breakable,colbacktitle=ForestGreen!40!white,coltitle=black,fonttitle=\bfseries\sffamily,
title=]
    $\Gamma(\A)$は$\A$の点を分離するくらいには大きい.
    Stone-Weierstrassの定理は,さらに$C(\hat{\A})$の中で稠密であるための必要条件を調べるための道具となる.
\end{tcolorbox}

\begin{definition}[self-adjoint]
    複素関数の集合$\A\subset\Map(X,\C)$が\textbf{自己共役}であるとは,$\forall_{f\in\A}\;\o{f}\in\A$を満たすことをいう.
    $f=\frac{1}{2}(f+\o{f})+i\frac{f-\o{f}}{2i}$より,$\A_\sa:=\Brace{f\in\A\mid\Im f\subset\R}$を$\A$の実数値関数がなす部分集合とすると,$\A=\A_\sa+i\A_\sa$と表せることに同値.
\end{definition}

\begin{definition}[sub vector lattice]
    部分空間$\A\subset C(X)$が$\forall_{f,g\in\A}\;(f\lor g\in\A)\land(f\land g\in\A)$を満たすとき,\textbf{部分線型束}という.
\end{definition}

\begin{lemma}
    コンパクトハウスドルフ空間$X$上の部分線型束$\A\subset C(X;\R)$について,
    任意の$X$上の連続関数で$X$の任意の2点において$\A$によって近似できるものは,$\A$によって一様に近似できる.
\end{lemma}

\begin{lemma}
    $\A\subset C_b(X,\R)$を,一様位相について閉じた代数\ref{def-uniform-algebra}とする.
    $\A$は$C(X)$の束演算$f\lor g,f\land g$について閉じている.
\end{lemma}

\begin{theorem}[Stone (1948)]
    $X$をコンパクトハウスドルフ空間,$\A$を自己共役な$C(X)$の部分代数で定数を含み,$X$の点を分離するとする.
    このとき,$\A$は$C(X)$の中で一様に稠密である.
\end{theorem}
\begin{example}\mbox{}
    \begin{enumerate}
        \item Weierstrass (1895):有界閉区間上の実数値連続関数は,多項式によって一様に近似できる.
        \item $[0,2\pi]$上の任意の連続な周期関数は,三角多項式によって一様に近似できる(Fourier級数は必ずしも一様収束しないにも拘らず).
        \item Rungeの近似定理:単位閉円板$[\Delta]$上の開円板$\Delta$上で正則な関数全体のなす集合$H(\Delta)$は,$C(\Delta)$の真の閉部分代数であって,点を分離し,定数を含む.が,自己共役性の要件は満たさず,Cauchy-Riemann方程式なるさらなる条件が必要である.
    \end{enumerate}
\end{example}

\begin{corollary}
    $X$を局所コンパクトハウスドルフ空間,$\A$を$C_0(X)$の自己共役な部分代数で,$X$の点を分離し,$X$のどの点でも同時には消えないとする:$\forall_{x\in X}\;\exists_{f\in\A}\;f(x)\ne0$.
    このとき,$\A$は$C_0(X)$上で一様に稠密である.
\end{corollary}

\subsection{$C^*$-代数の定義と例}

\begin{tcolorbox}[colframe=ForestGreen, colback=ForestGreen!10!white,breakable,colbacktitle=ForestGreen!40!white,coltitle=black,fonttitle=\bfseries\sffamily,
title=]
    対合と両立するBanach代数が$C^*$-代数である.
    対合と両立するHeyting代数がBoole代数であるのと同じように.
\end{tcolorbox}

\begin{definition}[involution, $C^*$-algebra, symmetric]\mbox{}
    \begin{enumerate}
        \item 代数$\A$の\textbf{対合}とは,周期2の写像$*:\A\to\A$であって,共役線型で乗法について反変的なものをいう.
        対合を備えた代数を$*$-代数という.
        \item ノルムについて$\forall_{A\in\A}\;\norm{A^*A}=\norm{A}^2$を満たす対合を備えたBanach代数を,\textbf{$C^*$-代数}という.
        \item 対合$*$が$\forall_{A\in\A}\;A=A^*\Rightarrow\Sp(A)\subset\R$を満たすとき,これを\textbf{対称的}であるという.
        \item 対合$*$が$\forall_{A\in\A}\;\Sp(A^*A)\subset\R_+$を満たすとき,これを\textbf{正}であるという.
    \end{enumerate}
\end{definition}
\begin{remarks}[$C^*$-性は強い条件]
    $\norm{A^*A}=\norm{A}^2$は,劣乗法性より$\norm{A}^2\le\norm{A^*}\norm{A}$から$\norm{A}\le\norm{A^*}$.対合であることより等長性$\norm{A}=\norm{A^*}$を結局は含意する.
    また$C^*$-性は(3)と(4)も含意する.
\end{remarks}
\begin{example}[involution]
    $C(X)$上の複素関数の共役や,$B(\H)$上の共役作用素.
\end{example}
\begin{example}[$C^*$-代数]\mbox{}\label{exp-C*-algebra}
    \begin{enumerate}
        \item 任意の局所コンパクトハウスドルフ空間について,$C_0(X)$は複素共役について$C^*$-代数となる.
        \item $B(H)$は随伴について$C^*$-代数となる.
        \item (Gelfand and Naimark 43) $B(H)$の任意の自己共役な閉部分代数は$C^*$-代数で,任意の$C^*$-代数は,なんらかのこの部分代数に$*$-等長同型になる.
        \item 局所コンパクトなアーベル群$G$上の畳み込みBanach代数$L^1(G)$は,$f^*(x)=\o{f}(-x)$によって等長な対合による$*$-代数の構造を持つが,$C^*$-性は持たない.
        また対称である.
    \end{enumerate}
\end{example}

\subsection{非可換な$C^*$-代数の扱い}

\begin{lemma}[$C^*$-代数の単位化]
    任意の単位的でない$C^*$-代数$\A$に対して,単位的$C^*$-代数$\wt{\A}=\A+\C I$が存在して,$\A\mono\wt{\A}$は余次元$1$の極大イデアルとなる.
\end{lemma}
\begin{remark}
    これにより,非可換な$C^*$-代数の元にも,$\wt{\A}$を通じてスペクトルを定義できるが,このとき$\A$はイデアルなので必ず$0\in\Sp(A)$を満たしてしまう.
\end{remark}

\begin{lemma}[正規な元のスペクトル半径はノルムに等しい]
    $A\in\A$が$C^*$-代数の正規な元であるならば,$r(A)=\norm{A}$.
\end{lemma}

\begin{lemma}[自己共役な元のスペクトルは実数/ユニタリな元のスペクトルは単位円周上にある]\mbox{}
    \begin{enumerate}
        \item $A\in\A$が$C^*$-代数の自己共役な元であるならば,$\Sp(A)\subset\R$(すなわち,対合は対称的である).
        \item $\A$が単位的で$U$がユニタリであるならば,$\Sp(U)\subset\T$.
    \end{enumerate}
\end{lemma}

\subsection{Gelfandの定理}

\begin{tcolorbox}[colframe=ForestGreen, colback=ForestGreen!10!white,breakable,colbacktitle=ForestGreen!40!white,coltitle=black,fonttitle=\bfseries\sffamily,
title=可換な$C^*$代数の構造は1次元表現によって特徴付けられる]
    ここで初めて,Gelfand変換$\Gamma:\A\to C(\wh{\A})$が単射になるための十分条件を得た.

    一般の$C^*$-代数の構造は,任意の$n$次元表現も考えねばならない.そこで,$B(H)$に埋め込むことが考えられる.これをGelfand-Naimark表現という.
    これは$C^*$-代数上の正な線型汎函数と極めて深い関係を持ち,正な線型汎函数の$B(H)$への表現をGelfand-Naimark-Segal表現という.
\end{tcolorbox}

\begin{theorem}[Gelfand's theorem (commutative)]
    任意の可換な単位的$C^*$-代数$\A$は,$\wh{\A}$を指標のなすコンパクトハウスドルフ空間として,$C(\wh{\A})$と等長$*$-同型である:$\A\simeq_* C(\wh{\A})$.
\end{theorem}

\begin{corollary}[Gelfand's theorem (non-commutative)]
    任意の可換で非単位的な$C^*$-代数$\A$は,$\wh{\A}$を指標のなす局所コンパクトハウスドルフ空間として,$C_0(\wh{\A})$と等長$*$-同型である:$\A\simeq_*C_0(\wh{\A})$.
\end{corollary}

\subsection{Gelfandスペクトルの対応}

\begin{tcolorbox}[colframe=ForestGreen, colback=ForestGreen!10!white,breakable,colbacktitle=ForestGreen!40!white,coltitle=black,fonttitle=\bfseries\sffamily,
title=]
    これはスペクトル定理の原型である.
\end{tcolorbox}

\begin{notation}
    単位的$C^*$-代数$\A$の
    元$T\in\A$について,$T$と$I$とを含む最小の$C^*$-部分代数を$C^*(T)$で表す.
\end{notation}

\begin{proposition}\label{prop-seminal-spectral-theorem}
    $T\in\A$を単位的$C^*$-代数$\A$の正規な元とする.
    等長$*$-同型$\Phi:C(\Sp(T))\iso C^*(T)$であって,$\Phi(1)=I,\Phi(\id)=T$を満たすものが存在する.
\end{proposition}

\begin{corollary}
    $T\in\A$を単位的$C^*$-代数$\A$の正規な元,$\B$を$T,I$を含む$\A$の$C^*$-部分代数とする.
    このとき,$T\in\B$のスペクトルと$T\in\A$のスペクトルとは等しい.
\end{corollary}

\subsection{GNS構成}

\begin{tcolorbox}[colframe=ForestGreen, colback=ForestGreen!10!white,breakable,colbacktitle=ForestGreen!40!white,coltitle=black,fonttitle=\bfseries\sffamily,
title=]
    $C^*$-代数上の正な線型汎関数は\textbf{状態}ともいう.これを,サイクリックな$*$-表現で実現する.
\end{tcolorbox}

\begin{proposition}[positive functional]
    単位的$C^*$-代数$\A$上の汎関数$\varphi:\A\to\C$が\textbf{正}であるとは,$\forall_{A\in\A}\;A\ge0\Rightarrow\varphi(A)\ge0$を満たすことをいう.このとき,次が成り立つ:
    \begin{enumerate}
        \item Cauchy-Schwarzの不等式が成り立つ:$\forall_{A,B\in\A}\;\abs{\varphi(B^*A)}^2\le\varphi(B^*B)\varphi(A^*A)$.
        \item $\varphi$は有界で$\norm{\varphi}=\varphi(I)$.
    \end{enumerate}
    また逆に,連続な汎関数$\varphi:\A\to\C$が$\norm{\varphi}=\varphi(I)$を満たすならば正である.
\end{proposition}

\begin{theorem}
    $\varphi:\A\to\C$を正汎関数とし,$\mathfrak{L}:=\Brace{A\in\A\mid\varphi(A^*A)=0}$とする.
    \begin{enumerate}
        \item $A\in\fL$は$\A A\subset\Ker\varphi$に同値.
        \item $\fL$は$\A$の閉な左イデアルである.
        \item $\A/\fL$は内積$(A+\fL|B+\fL)=\varphi(B^*A)$について前Hilbert空間になる.
        \item $\fH_\varphi$を$\A/\fL$の完備化とする.$\Phi:\A\to\End(\A/\fL)$を$\Phi(A)(B+\fL):=AB+\fL\;(A,B\in\A)$と定めると,ノルム減少的な$*$-準同型を定め,従って$*$-準同型$\Phi:\A\to B(\fH_\varphi)$を定める.
        \item $\Ker\Phi:=\Brace{A\in\A\mid\A A\A\subset\Ker\varphi}$が成り立つ.
    \end{enumerate}
\end{theorem}

\section{スペクトル理論I}

\subsection{スペクトル定理}

\begin{notation}
    $T\in B(H)$に対して,$C^*(T)$で,$T,I$を含む最小のノルム閉かつ$*$-不変な部分代数(すなわち$C^*$-部分代数\ref{exp-C*-algebra})とする.
\end{notation}

\begin{theorem}[spectral theorem (von Neumann)]
    $H$を複素Hilbert空間,$T\in B(H)$を正規作用素とする.
    \begin{enumerate}
        \item 等長$*$-同型$C(\Sp(T))\iso C^*(T);f\mapsto f(T)$であって,$\id(T)=T$かつ$1(T)=I$を満たすものが存在する.
        \item $\forall_{f\in C(\Sp(T))}\;\Sp(f)=f(\Sp(T))$.
    \end{enumerate}
\end{theorem}
\begin{Proof}
    \ref{prop-seminal-spectral-theorem}に同値.
\end{Proof}

\begin{history}
    スペクトル定理は,「無限個の変数を持った有界な二次形式」という形で表現された「自己共役作用素」について,Hilbertによって1906年に初めて証明された.
    今回の証明は,可換な$C^*$-代数のGelfand表現を通じて行った.最短ではないが本質的である.
\end{history}

\subsection{スペクトルと固有値}

\begin{tcolorbox}[colframe=ForestGreen, colback=ForestGreen!10!white,breakable,colbacktitle=ForestGreen!40!white,coltitle=black,fonttitle=\bfseries\sffamily,
title=]
    正規作用素がコンパクトでもあるとき,スペクトル定理\ref{thm-Spectral-Theorem-for-normal-compact-operator}は固有値の言葉で表現された.
    では作用素がコンパクトではないと,スペクトルはもはや固有値とは関係ないのか?

    実は,孤立点を持たないコンパクト集合については,全く固有値と関係ないスペクトルでありえるが,孤立点は必ず固有値である.
\end{tcolorbox}

\begin{lemma}\mbox{}
    \begin{enumerate}
        \item 任意のコンパクト集合$X\subset\C$に対して,ある対角作用素$M\in B(l^2)$が存在して,$\Sp(M)=X$.
        \item コンパクト集合$X\subset\C$が$(X^\circ)^-=X$を満たすならば,ある作用素$M_\id\in B(L^2(X))$が存在して,$\Sp(M_\id)=X$かつ$M_\id$は固有値を持たない(直交射影成分がない).
        \item 実は任意の孤立点を持たないコンパクト集合$X\subset\C$に対して,連続なdiffuse測度\ref{def-diffuse-atomic}が存在して,$\Sp(M_\id)=X$を満たす作用素$M_\id$で固有値を持たないものが存在する.
    \end{enumerate}
\end{lemma}

\begin{lemma}
    $T\in B(H),\lambda\in\Sp(T)$に対して,単位ベクトルの列$\{x_n\}\subset H$が存在して,$\norm{Tx_n-\lambda x_n}\to0$または$\norm{T^*x_n-\o{\lambda}x_n}\to0$が成り立つ.
\end{lemma}

\begin{proposition}
    正規作用素$T\in B(H)$と$\lambda\in\Sp(T)$に対して,
    \begin{enumerate}
        \item 任意の$\ep>0$に対して単位ベクトル$x\in H$が存在して,$\norm{Tx-\lambda x}<\ep$を満たす.
        \item $\lambda\in\Sp(T)$が孤立点ならば,固有値である.
    \end{enumerate}
\end{proposition}

\subsection{スペクトルによる作用素の性質の特徴付け}

\begin{tcolorbox}[colframe=ForestGreen, colback=ForestGreen!10!white,breakable,colbacktitle=ForestGreen!40!white,coltitle=black,fonttitle=\bfseries\sffamily,
title=]
    スペクトル理論による対応$C(\Sp(T))\simeq C^*(T)$を用いることで,
    Hilbert空間の設定を忘れて,$\lambda\in\Sp(T)$が孤立点であるかどうかを判定するだけで,射影$P\in\A$が存在して$PT=TP=\lambda P$を満たすか判定出来る.
    そこで,任意の$C^*$-代数について,$T$の性質は,$\Sp(T)$を見るだけで判定できる.
    なお,任意の$C^*$-代数に対して,あるHilbert空間$H$が存在して,$B(H)$のある$C^*$-部分代数に同型であることに注意.
\end{tcolorbox}

\begin{proposition}
    正規作用素$T\in B(H)$について,次の2条件はそれぞれ同値.
    \begin{enumerate}
        \item $T$は自己共役である.
        \item $\Sp(T)\subset\R$.
    \end{enumerate}
    \begin{enumerate}
        \item $T$は正である.
        \item $\Sp(T)\subset\R_+$.
    \end{enumerate}
\end{proposition}

\begin{proposition}
    任意の正作用素$T\in B(H)$に対して,正作用素$T^{1/2}$がただ一つ存在して,$(T^{1/2})^2=T$を満たす.
    また,この$T^{1/2}$は,$T$と可換な任意の元と可換である.
\end{proposition}

\begin{proposition}
    任意の自己共役作用素$T\in B(H)$に対して,正作用素の組$T_+,T_-$がただ一つ存在して,$T_+T_-=0$かつ$T=T_+-T_-$が成り立つ.
    また,$T_+,T_-$は,$T$と可換な任意の元と可換である.
\end{proposition}

\begin{remark}
    極分解\ref{thm-polar-decomposition}では,$\abs{T}:=(T^*T)^{1/2}$としたのであった.
    $T$が正規ならば,証明抽出より,$\abs{T}\in C^*(T)$が解り,これは$\Sp(T)$上の関数$\abs{\id}:\lambda\to\abs{\lambda}$に対応する.
    特に,$T$が自己共役であるとき,$\abs{T}=T_++T_-$の表示も持つことが分かる.
\end{remark}

\subsection{作用素の可換性}

\begin{discussion}
    正規行列$T\in M_n(\C)$と$S\in M_n(\C)$とが可換であるとは,$S$が任意の$T$の固有空間を不変にすることと同値.
    $T,T^*$は同じ固有空間を持つから,$S$と$T$との可換性と$S$と$T^*$との可換性とは同値.

    これを一般のHilbert空間の作用素に一般化するのは至難の業である(Fuglede's theorem)のは,スペクトルはもはや固有値のみからなるわけではないからである.
\end{discussion}

\begin{proposition}[Fuglede's theorem]\label{prop-Fuglede}
    作用素$S,T\in B(H)$のうち,$T$は正規であるとする.このとき,$ST=TS\Rightarrow ST^*=T^*S$.
\end{proposition}

\begin{proposition}
    2つの作用素$T_1,T_2\in B(H)$が正規同値であるならばユニタリー同値\ref{def-unitary-equivalent}である.
\end{proposition}

\section{スペクトル理論II}

\begin{tcolorbox}[colframe=ForestGreen, colback=ForestGreen!10!white,breakable,colbacktitle=ForestGreen!40!white,coltitle=black,fonttitle=\bfseries\sffamily,
title=]
    前節では$C(\Sp(T))\simeq C^*(T)$によって$T$を調べた.
    連続関数の空間$C(\Sp(T))$をBorel関数の空間に一般化して考察する.
    これはもはや同型にはならず準同型$\B_b(\Sp(T))\to W^*(T)$を定め,これは延長となっている.
    この像はvon Neumann環の言葉で捉える必要が出てくる.
\end{tcolorbox}

\subsection{自己共役作用素の空間の順序完備性}

\begin{proposition}
    $(T_\lambda)_{\lambda\in\Lambda}$を$B(H)_\sa$の有界な広義単調増大ネットとする.
    このとき,
    \begin{enumerate}
        \item $\forall_{\lambda\in\Lambda}\;T_\lambda\le T$を満たす最小の$T\in B(H)$が存在する.これを$\lub T_\lambda$と表す.
        \item $\forall_{x\in H}\;\norm{T_\lambda x-Tx}\to0$.
    \end{enumerate}
\end{proposition}

\subsection{一般化されたスペクトル定理}

\begin{tcolorbox}[colframe=ForestGreen, colback=ForestGreen!10!white,breakable,colbacktitle=ForestGreen!40!white,coltitle=black,fonttitle=\bfseries\sffamily,
title=]
    $C(\Sp(T))\simeq C^*(T)$を,単調収束定理が成り立つように延長する.
    なお,一般に,ノルム減少的な$*$-準同型$C(X)\to B(H)$に対して,単調収束定理が成り立つような延長$\B_b(X)\to B(H)$が一意に存在する.
\end{tcolorbox}

\begin{notation}[有界Borel関数の空間]
    局所コンパクトハウスドルフ空間$X$上の有界な複素Borel関数全体の集合を$\B_b(X)$で表す.
    これは$C^*$-代数で,$X$が第2可算ならば実部$\B_b(X)_\sa$は$C_0(X)_\sa$を含みかつ有界な単調列の極限について閉じている有界実関数の最小の空間である.
\end{notation}

\begin{notation}[可換な元の代数]\label{notation-W*(T)}
    正規作用素$T\in B(H)$に対して,$W^*(T)\subset B(H)$で$T$と可換なすべての元と可換な元からなる集合とする.
    Fugledeの定理\ref{prop-Fuglede}より,$W^*(T)=C^*(T)$.
    $W^*(T)$は$T,I$を含む最小のvon Neumann代数である\ref{lemma-W*(T)}.
\end{notation}

\begin{theorem}
    任意の正規作用素$T\in B(H)$に対して,
    \begin{enumerate}
        \item あるノルム減少的な$*$-準同型$\B_b(\Sp(T))\to W^*(T);f\mapsto f(T)$が存在して,$C(\Sp(T))\iso C^*(T)$の延長である.
        \item $\{f_n\}\subset\B_b(\Sp(T))_\sa$を有界な増大列とし,$f:=\lub f_n$とすると,$f(T)=\lub f_n(T)$が成り立つ.
    \end{enumerate}
\end{theorem}

\subsection{スペクトル測度}

\begin{tcolorbox}[colframe=ForestGreen, colback=ForestGreen!10!white,breakable,colbacktitle=ForestGreen!40!white,coltitle=black,fonttitle=\bfseries\sffamily,
title=]
    積分は測度と見るのが筋がよく,今回のスペクトル写像$\B_b(\Sp(T))\to W^*(T)$も,作用素値の積分と見る.
\end{tcolorbox}

\begin{notation}
    $B(H)_p$で,$B(H)$の射影全体の空間を表す.
    射影族$E(Y_n)$の$B(H)_p$での上限を$\bigvee E(Y_n)$で表し,$\sum E(Y_n)(H)$が生成する閉部分空間が定める射影となる.
\end{notation}

\begin{definition}[spectral measure]
    可測空間$(X,\F)$上の\textbf{スペクトル測度}とは,写像$E:\F\to B(H)_p$であって,次の4条件を満たすものをいう:
    \begin{enumerate}
        \item $E(\emptyset)=0,E(X)=I$.
        \item $E(Y\cap Z)=E(Y)E(Z)$.
        \item $E(Y\sqcup Z)=E(Y)+E(Z)$.これは(4)より従う.
        \item $E(\cup Y_n)=\bigvee E(Y_n)$.
    \end{enumerate}
\end{definition}

\begin{notation}
    任意の$x\in H$に対して,
    $\mu_x(Y):=(E(Y)x|x)\;(Y\in\F)$によって,通常の意味での測度が定まる.
\end{notation}

\begin{proposition}
    $(X,\F)$上のスペクトル測度$E$について,
    \begin{enumerate}
        \item ノルム減少的な$*$-準同型$\B_b(X)\to B(H)_\normal;f\mapsto T_f$が
        \[T_f:=\int_Xf(\lambda)dE(\lambda)\quad\lor\quad (T_fx|y):=\int_Xf(\lambda )d\mu_{xy}\]
        によって定まる.
        \item $\{f_n\}\subset\B_b(X)_\sa$が有界な単調増大列ならば,$f=\lub f_n,T_f=\lub T_{f_n}$を満たす.
    \end{enumerate}
\end{proposition}

\subsection{スペクトル測度によるスペクトル定理の表現}

\begin{proposition}
    任意の正規作用素$T\in B(H)$に対して,$(\Sp(T),\B(\Sp(T)))$上のあるスペクトル測度$E:\B(\Sp(T))\to W^*(T)$が存在して,
    \[T=\int_{\Sp(T)}\lambda dE(\lambda).\]
\end{proposition}

\subsection{スペクトル測度と固有値}

\begin{tcolorbox}[colframe=ForestGreen, colback=ForestGreen!10!white,breakable,colbacktitle=ForestGreen!40!white,coltitle=black,fonttitle=\bfseries\sffamily,
title=]
    スペクトル定理の表現はいささか煩雑になったように見えるが,固有値の位置は明快に表現できる.
\end{tcolorbox}

\begin{proposition}
    $T\in B(H)$を正規作用素,$E$を対応するスペクトル測度とする.
    任意の有界Borel関数$f\in\B_b(\Sp(T))$に対して,次の2条件は同値.
    \begin{enumerate}
        \item $\lambda\in\Sp(f(T))$.
        \item $\forall_{\ep>0}\;E(f^{-1}(B(\lambda,\ep)))\ne0$.
    \end{enumerate}
    また,次の2条件も同値.
    \begin{enumerate}
        \item $\lambda$は$f(T)$の固有値である.
        \item $E(f^{-1}(\lambda))\ne0$.
    \end{enumerate}
    このとき,$E(f^{-1}(\lambda))$は$\lambda$に対応する$f(T)$の固有空間が定める射影である.
\end{proposition}

\begin{corollary}
    $f\in \B_b(\Sp(T))\Rightarrow\Sp(f(T))\subset\o{f(\Sp(T))}$.
\end{corollary}
\begin{corollary}
    $\forall_{\lambda\in\Sp(T)}\;\lambda$は$T$の固有値でないならば$E(\{\lambda\})=0$.
\end{corollary}
\begin{corollary}
    $T\in B(H)$は対角化可能のとき,対応するスペクトル測度$E$は完全に原子的で,$\Sp(T)$の固有値の集合$\Sp(T)_a$上に集中している.
    また,$\forall_{x\in H}\;\Sp(T)=\o{\Sp(T)_a}$で,
    \[Tx=\sum_{\lambda\in\Sp(T)_a}\lambda E(\{\lambda\})x.\]
\end{corollary}

\section{作用素代数}

\begin{tcolorbox}[colframe=ForestGreen, colback=ForestGreen!10!white,breakable,colbacktitle=ForestGreen!40!white,coltitle=black,fonttitle=\bfseries\sffamily,
title=]
    $B(H)$には種々の位相が入る.

    量子力学の「自然数,実数などの古典的な数学的対象を,すべて複素Hilbert空間上の作用素を用いて考えよ」という天啓から始まった
    作用素環の理論は,当然のように,数学全体の基礎づけの役割も果たし得る.
    作用素環の数理構造を解析しているうちに,病理的な対象でさえも場の量子論に自然に現れることが判明するという,逆の結果さえ生まれた.
\end{tcolorbox}

\subsection{2つの作用素位相}

\begin{tcolorbox}[colframe=ForestGreen, colback=ForestGreen!10!white,breakable,colbacktitle=ForestGreen!40!white,coltitle=black,fonttitle=\bfseries\sffamily,
title=]
    強い順に,ノルム位相,強位相,弱位相である.
    なお,作用素ノルム位相は一様作用素位相ともいう.
\end{tcolorbox}

\begin{definition}\mbox{}
    \begin{enumerate}
        \item $B(H)$上の\textbf{強位相}とは,半ノルムの族$(T\mapsto\norm{Tx})_{x\in H}$が誘導する局所凸位相をいう.
        \item $B(H)$上の\textbf{弱位相}とは,半ノルムの族$(T\mapsto\abs{(Tx|y)})_{x,y\in H}$が誘導する局所凸位相をいう.
    \end{enumerate}
    一般のBanach空間$X$上のBanach代数$B(X)$上に,$X$または$X\times X^*$に添字付けられた半ノルムの族を用いて同様の位相を定義できることは明らかである.
\end{definition}
\begin{lemma}\mbox{}
    \begin{enumerate}
        \item 共役作用$*:B(H)\to B(H)$は弱連続であるが,強連続ではない.
        \item 共役作用は$B(H)_\normal$に制限すると強連続であるが,この正規作用素の集合は部分空間にはならない.
        \item 乗法$B(H)\times B(H)\to B(H)$は片方の変数を止めると強連続であるが,$B(H)\times B(H)$上は弱連続でさえない.
        \item 片方の引数の有界集合への制限$B(0,n)\times B(H)\to B(H)$は強連続である.
        \item 単位球$B\subset B(H)$は弱コンパクトである.
    \end{enumerate}
\end{lemma}

\begin{lemma}\mbox{}
    \begin{enumerate}
        \item 強位相と弱位相は$H$が無限次元のとき,距離化可能でないどころか第1可算でさえない.
        \item $H$が可分ならば,単位球$B$は強位相と弱位相の両方で距離化可能である.
        \item $H$が可分ならば,単位球$B$は弱位相についてさらにコンパクトハウスドルフであり,よって第2可算である.よって,$B$は弱可分.
        \item よって,$B(H)=\R_+B$自身の弱可分である.
        \item 任意の$H$の正規直交基底について,$B_f(H)$の元は有限な行列の有理係数和によって,ノルムによって近似出来る.したがって,$B_f(H)$はノルム位相について可分である.
        \item $B_f(H)$は強位相について稠密である.
        \item $B(H)$は強位相についても可分である.
    \end{enumerate}
\end{lemma}

なお,次のような定義もある.

\begin{definition}[uniform operator topology, weak operator topology, strong operator topology]
    $V,W\in\TVS$を位相線形空間,$L(V,W):=\Hom_\TVS(V,W)$をその間の連続線型作用素全体のなす空間とする.
    \begin{enumerate}
        \item $V,W$をノルム空間とする.これが定める作用素ノルム$p(A):=\sup_{v\ne0}\frac{p_W(Av)}{p_V(v)}$が定める位相を,\textbf{一様作用素位相}という.
        \item $U(x,f):=\Brace{A\in L(V,W)\mid\abs{f(A(x))}<1}$として,$(U(x,f))_{x\in V,f\in W^*}$を$0$の開近傍系の基底として生成される位相を,\textbf{弱作用素位相}という.作用素列$(A_n)$が$A$に弱作用素位相で収束することは,任意の$x\in X$について$(A_n(x))$が$(A(x))$に$W$の弱位相に関して収束する\footnote{すなわち任意の有界線型汎関数$f\in W^*$について$f(A_n(x))\to f(A(x))$.}ことに同値.
        \item $N(x,U):=\Brace{A\in L(V,W)\mid Av\in U}$として,$(N(x,U))_{x\in V,0\in U\osub W}$を$0$の開近傍系の基底として生成される位相を,\textbf{作用素強位相}という.作用素列$(A_n)$が$A$に強作用素位相で収束することは,任意の$x\in X$について$(A_n(x))$が$(A(x))$に$W$にてノルム収束することに同値.
    \end{enumerate}
\end{definition}

\subsection{$B(H)$の弱閉集合}

\begin{notation}
    $B(H^n)\simeq M_n(B(H))$の同一視を行う.
    $\Phi:B(H)\to B(H^n)$を$\Phi(T)_{kk}=T,\Phi(T)_{kl}=0\;(k\ne l)$とすると,これは$*$-同型である.
    このスカラーをスカラー行列と同一視する技法はamplificationと呼ばれる.
\end{notation}

\begin{proposition}
    任意の汎関数$\varphi:B(H)\to\bF$について,次の3条件は同値.
    \begin{enumerate}
        \item 元$x_1,\cdots,x_n,y_1,\cdots,y_n\in H$が存在して,$\forall_{T\in B(H)}\;\varphi(T)=\sum(Tx_k|y_k)$.
        \item $\varphi$は弱連続である.
        \item $\varphi$は強連続である.
    \end{enumerate}
\end{proposition}

\begin{corollary}
    $B(H)$の任意の強閉な凸集合は弱閉である.
    特に,任意の強閉部分空間は弱閉である.
\end{corollary}

\subsection{二重可換子代数}

\begin{definition}[commutant]\label{def-commutant}
    任意の部分集合$\A\subset B(H)$に対して,$\A$の\textbf{可換子}とは,弱閉な部分代数
    \[\A':=\Brace{T\in B(H)\mid\forall_{S\in\A}\;TS=ST}\]
    をいう.
\end{definition}

\begin{theorem}[double commutant theorem (von Neumann 1929)]\label{thm-double-commutant-theorem}
    $\A\subset B(H)$を自己共役な単位的部分代数とする.
    このとき,次の3条件は同値.
    \begin{enumerate}
        \item $\A=\A''$.
        \item $\A$は弱閉.
        \item $\A$は強閉.
    \end{enumerate}
    この条件を満たす自己共役な単位的部分代数$\A$を\textbf{von Neumann代数}または\textbf{$W^*$-代数}という.
\end{theorem}

\begin{corollary}
    $B(H)$の自己共役な単位的部分代数$\A$の強閉包と弱閉包は一致し,$\A''$となる.
\end{corollary}

\begin{example}[von Neumann algebra]\mbox{}
    \begin{enumerate}
        \item 自己共役な集合$D\subset B(H)$に対して,$D'$はvon Neumann環である.
        \item $L^\infty(X)$は$W^*$-代数である.
    \end{enumerate}
\end{example}

\subsection{因子}

\begin{tcolorbox}[colframe=ForestGreen, colback=ForestGreen!10!white,breakable,colbacktitle=ForestGreen!40!white,coltitle=black,fonttitle=\bfseries\sffamily,
title=]
    $B(H)$以外の因子が存在するということさえはじめは驚きの事実であった.
    可分なHilbert空間上の同型でない因子は非可算個存在し,その分類は,群論(特に半単純Lie群の分類)や量子力学の基礎づけにおいて基本的な問題となる.
\end{tcolorbox}

\begin{definition}[factor]
    von Neumann環$\A$が\textbf{因子}であるとは,$\A\cap\A'=\C I$を満たすことをいう.
    これは考えうる限り最も可換性の破れた場合をいう.
\end{definition}

\subsection{$\sigma$-弱位相}

\begin{tcolorbox}[colframe=ForestGreen, colback=ForestGreen!10!white,breakable,colbacktitle=ForestGreen!40!white,coltitle=black,fonttitle=\bfseries\sffamily,
title=]
    $H$が無限次元のとき,$\sigma$-弱位相は弱位相よりも真に強いが,有界集合上では一致する.
\end{tcolorbox}

\begin{definition}[$\sigma$-weak topology]
    双線型形式$B(H)\times B^1(H)\to\bF;\brac{S,T}=\Tr(ST)$が引き起こすペアリングによる$w^*$-位相,すなわち,$\sigma(B(H),B^1(H))$-位相,すなわち跡類作用素の双対空間$(B^1(H))^*$としての位相を\textbf{$\sigma$-弱位相}という.
\end{definition}

\begin{proposition}
    汎関数$\varphi:B(H)\to\bF$について,次の4条件は同値:
    \begin{enumerate}
        \item ノルム2乗和が収束する列$\{x_n\},\{y_n\}\subset H,\sum\norm{x_n}^2<\infty,\sum\norm{y_n}^2<\infty$が存在して,$\varphi(S)=\sum(Sx_n|y_n)$と表せる.
        \item (1)を満たし,さらに$(x_n),(y_n)$は組ごとに直交に取れる.
        \item $T\in B^1(H)$が存在して,$\forall_{S\in B(H)}\;\varphi(S)=\Tr(ST)$.
        \item $\varphi$は$\sigma$-弱連続である.
    \end{enumerate}
\end{proposition}
\begin{remark}
    証明から,$B(H)$の弱位相についての双対空間は$B_f(H)$となる.
    よって,$H$が無限次元のとき,弱位相は$\sigma$-弱位相より真に弱い.
\end{remark}

\begin{corollary}
    正な$\sigma$-弱連続汎関数$\varphi:B(H)\to\bF$に対して,直交列$\{x_n\}\subset H$が存在して,$\sum\norm{x_n}^2=\norm{\varphi}$かつ$\forall_{S\in B(H)}\;\varphi(S)=\sum(Sx_n|x_n)$を満たす.
\end{corollary}

\begin{proposition}\mbox{}
    \begin{enumerate}
        \item $B(H)$の有界集合上において,弱位相と$\sigma$-弱位相とは一致する.
        \item 汎関数$\varphi:B(H)\to\bF$が$\sigma$-弱連続であることは,制限$\varphi|_B$が弱連続であることと同値.
    \end{enumerate}
\end{proposition}

\subsection{$\sigma$-弱位相によるvon Neumann代数の特徴付け}

\begin{tcolorbox}[colframe=ForestGreen, colback=ForestGreen!10!white,breakable,colbacktitle=ForestGreen!40!white,coltitle=black,fonttitle=\bfseries\sffamily,
title=]
    von Neumann代数は,$B(H)$の単位的$C^*$-部分代数であって$\sigma$-弱閉であるものとしても特徴付けられる.

    強い順に,ノルム位相,$\sigma$-強位相・強位相,$\sigma$-弱位相・弱位相となる.・でつないだ部分は,有界集合上では一致する.
\end{tcolorbox}

\begin{proposition}
    自己共役な単位的部分代数$\A\subset B(H)$と$S\in\A''$について,任意の$\sum\norm{x_n}^2<\infty$を満たす列$\{x_n\}\subset H$と$\ep>0$について,$T\in\A$が存在して$\sum\norm{(S-T)x_n}^2<\ep^2$を満たす.
    特に,$\A''$は$\A$の$\sigma$-弱閉包である.
\end{proposition}

\begin{definition}[$\sigma$-strong topology]
    $B(H)$上の\textbf{$\sigma$-強位相}とは,半ノルムの族
    \[T\mapsto\paren{\sum^\infty_{n=1}\norm{Tx_n}^2}^{1/2},\quad\{x_n\}\subset H,\sum^\infty_{n=1}\norm{x_n}^2<\infty\]
    が引き起こす局所凸位相をいう.
\end{definition}

\begin{corollary}[命題の再翻訳]
    単位的$*$-部分代数$\A$の$\sigma$-強位相に関する閉包は$\A''$である.
\end{corollary}

\begin{lemma}[$\sigma$-強位相の性質]\mbox{}
    \begin{enumerate}
        \item 有界集合上では,$\sigma$-強位相と強位相とは一致する.
        \item $\sigma$-強位相と$\sigma$-弱位相とは有界集合上でも異なるが,連続な汎関数は同じになる.
    \end{enumerate}
\end{lemma}

\subsection{前双対としての表現}

\begin{tcolorbox}[colframe=ForestGreen, colback=ForestGreen!10!white,breakable,colbacktitle=ForestGreen!40!white,coltitle=black,fonttitle=\bfseries\sffamily,
title=]
    $\A$の前双対$\A_*$は一意的に存在する(S. Sakai).
    また,von Neumann代数は,$C^*$-代数であって,前双対空間を持つものとしても特徴付けられる(3つ目).
\end{tcolorbox}

\begin{theorem}
    任意のvon Neumann代数$\A\subset B(H)$について,Banach空間$\A_*$が存在して,等長同型$\A\simeq(\A_*)^*$が成り立つ.
    $\A$上の$w^*$-位相は$\sigma$-弱位相である.
\end{theorem}

\section{極大可換代数}

\begin{tcolorbox}[colframe=ForestGreen, colback=ForestGreen!10!white,breakable,colbacktitle=ForestGreen!40!white,coltitle=black,fonttitle=\bfseries\sffamily,
title=]
    von Neumann代数の理論のうち,極大可換性に注目して,そのスペクトル定理に対する示唆を議論する.
\end{tcolorbox}

\subsection{極大可換性の定義と特徴付け}

\begin{definition}[maximal commutative / MASA (maximal abelian self-adjoint algebras)]
    可換な自己共役代数$\A\subset B(H)$が\textbf{極大可換}であるとは,任意の可換な真の$*$-部分代数に包含されていないことをいう.
\end{definition}
\begin{lemma}
    $B(H)$の$*$-部分代数$\A$について,次の2条件はそれぞれ同値.
    \begin{enumerate}
        \item 可換である.
        \item $\A\subset\A'$.
    \end{enumerate}
    \begin{enumerate}
        \item 極大可換である.
        \item $\A=\A'$.
    \end{enumerate}
    特に,任意の極大可換部分代数は,弱閉で$I$を含み,したがってvon Neumann代数である.
\end{lemma}

\subsection{サイクリックな元}

\begin{definition}[cyclic, separating]
    $\A\subset B(H)$を自己共役な代数とする.
    \begin{enumerate}
        \item $x\in H$が$\A$について\textbf{サイクリック}であるとは,部分空間$\A x$が$H$上稠密であることをいう.
        \item 射影$P\in B(H)$が$A$に関して\textbf{サイクリック}であるとは,$\exists_{x\in H}\;\A x$が$P(H)$上稠密であることをいう.サイクリックな射影は必ず$\A'$の元である.
        \item $\A$がサイクリックな元$x\in H$を持つことは,$I$がサイクリックな射影であることと同値.
        \item $x\in H$が$\A$に関して\textbf{分離的}であるとは,$\forall_{T\in\A}\;Tx=0\Rightarrow T=0$が成り立つことをいう.
    \end{enumerate}
\end{definition}

\begin{lemma}
    $x\in H$と自己共役な部分代数$\A\subset B(H)$について,
    次の2条件は同値.
    \begin{enumerate}
        \item $x$は$\A$に関してサイクリックである.
        \item $x$は$\A'$に関して分離的である.
    \end{enumerate}
\end{lemma}

\begin{lemma}
    任意の自己共役な部分代数$\A\subset B(H)$について,組ごとに直交な射影の族$\{P_j\}_{j\in J}\subset\A'$が存在して,$\A$に関してサイクリックで,$\sum P_j=I$をみたす.
    $H$が可分ならば,$J$は可算に取れる.
\end{lemma}

\begin{lemma}
    可分なHilbert空間$H$上の任意の可換な自己共役部分代数$\A\subset B(H)$には,分離的な元$x\in H$が存在する.
\end{lemma}

\subsection{極大可環代数の乗算作用素による特徴付け}

\begin{tcolorbox}[colframe=ForestGreen, colback=ForestGreen!10!white,breakable,colbacktitle=ForestGreen!40!white,coltitle=black,fonttitle=\bfseries\sffamily,
title=]
    これが実はスペクトル定理を「対角化」の見方で見つめ直すための大事な枠組みとなる.
\end{tcolorbox}

\begin{proposition}
    $\int:C_c(X)\to\R$を$\sigma$-コンパクトな局所コンパクトハウスドルフ空間$X$上のRadon積分とし,$M:\L^\infty(X)\to B(L^2(X));f\mapsto M_fg=fg\;(g\in\L^2(X))$を乗算作用素を対応付ける対応とする.
    このとき,
    \begin{enumerate}
        \item 写像$f\mapsto M_f$はある極大可換部分代数への等長$*$-同型$L^\infty(X)\mono B(L^2(X))$を定める.
        \item 像である極大可換部分代数は$\Brace{M_f\in B(L^2(X))\mid f\in C_c(X)}$を強稠密な部分代数として含む.
    \end{enumerate}
\end{proposition}

\begin{theorem}
    Hilbert空間$H$上の自己共役な部分代数$\A\subset B(H)$について,次の3条件は同値.
    \begin{enumerate}
        \item $\A$は極大可換である.
        \item $\A$はサイクリックな元を持つ可換なvon Neumann環である.
        \item 第2可算なコンパクトハウスドルフ空間$X$上のRadon積分$\int:C_c(X)\to\bF$と,等長写像$U:L^2(X)\to H$が存在して,対応$L^\infty(X)\iso\A;f\mapsto UM_fU^*$は等長$*$-同型となる.
    \end{enumerate}
\end{theorem}

\begin{proposition}
    可分なHilbert空間$H$上の
    2つの極大可換von Neumann代数$\A_1,\A_2\subset B(H)$は$*$-同型であるとする.
    このとき,あるユニタリー$U\in B(H)$が存在して,$U\A_1U^*=\A_2$.
\end{proposition}

\subsection{空間的スペクトル定理}

\begin{definition}[multiplicity-free]
    正規作用素$T\in B(H)$が\textbf{重複度なし}であるとは,$C^*(T)$に関するサイクリックな元$x\in H$が存在すること,すなわち,部分空間$\Brace{f(T)x\in H\mid f\in C(\Sp(T))}$が$H$上稠密であることをいう.
\end{definition}
\begin{lemma}\mbox{}\label{lemma-W*(T)}
    \begin{enumerate}
        \item $W^*(T)$は$T,I$を含む最小のvon Neumann代数である.
        \item $W^*(T)$は$C^*(T)$の強位相に関する閉包である.
        \item $T$が重複度なしであることと,$W^*(T)\subset B(H)$が極大可換であることとは同値.
    \end{enumerate}
\end{lemma}
\begin{remark}
    任意の$T$の固有値$\lambda$は重複度$1$を持つ.
\end{remark}

\begin{corollary}[spatial version of the spectral theorem]
    $T\in B(H)$を重複度のない正規作用素とする.
    このとき,$\Sp(T)$上のRadon積分と,等長写像$U:L^2(\Sp(T))\to H$が存在して,次は$C(\Sp(T))\to C^*(T)$を延長する等長$*$-同型になる:
    \[\xymatrix@R-2pc{
        L^\infty(\Sp(T))\ar[r]&W^*(T)\\
        \rotatebox[origin=c]{90}{$\in$}&\rotatebox[origin=c]{90}{$\in$}\\
        f\ar@{|->}[r]&f(T):=UM_fU^*
    }\]
\end{corollary}

\subsection{スペクトル対応の帰結}

\begin{proposition}
    可分なHilbert空間$H$上の正規で重複度のない作用素
    $S,T\in B(H)$について,次の3条件は同値:
    \begin{enumerate}
        \item $S,T$はユニタリ同値.
        \item $\Sp(S)=\Sp(T)$かつ$C^*(S),C^*(T)$に関してサイクリックな元$x,y$がそれぞれ存在して,2つの$\Sp(S)$上のRadon積分$f\mapsto(f(S)x|x)$と$f\mapsto(f(T)y|y)$とは同値である.
        \item 等長$*$-同型$\Phi:W^*(S)\to W^*(T)$が存在して$\Phi(S)=T$を満たす.
    \end{enumerate}
\end{proposition}

\subsection{スペクトル定理と対角化再論}

\begin{tcolorbox}[colframe=ForestGreen, colback=ForestGreen!10!white,breakable,colbacktitle=ForestGreen!40!white,coltitle=black,fonttitle=\bfseries\sffamily,
title=]
    $B(H)$の極大可換代数は,$H$の基底の一般化と見れる.
    この見方によれば,スペクトル定理の「作用素を対角化する基底に関する存在定理」という側面が再び見えてくる.
\end{tcolorbox}

\begin{definition}
    極大可換代数$\A\subset B(H)$が\textbf{原子的}であるとは,ある原子的積分\ref{def-diffuse-atomic}が存在して,これに関する本質的に有界な関数の空間$L^\infty(X)$と同型であることをいう.
    $H$が可算のとき,これは$\A\simeq l^\infty$をいう.
\end{definition}

\begin{proposition}[一斉対角化]
    $\{T_j\}_{j\in J}\subset B(H)$を可分Hilbert空間$H$上の正規作用素の可換な族とする.
    このとき,第2可算なコンパクトハウスドルフ空間$X$上のRadon積分$\int:C(X)\to\R$と等長写像$U:L^2(X)\to H$と本質的に有界な関数の族$\{f_j\}_{j\in J}\subset\L^\infty(X)$が存在して,
    $\forall_{j\in J}\; UM_{f_j}U^*=T_j$を満たす.
\end{proposition}

\subsection{一般化?}

\begin{remarks}
    $J$が可算のとき,$\{f_j\}\subset C(X)$に取れる.
    $X=[0,1]$と取れる.
\end{remarks}

\begin{theorem}[spatial spectral theorem 2]
    $H$を可分なHilbert空間,$T\in B(H)$を正規作用素とする.
    $T$と可換で$\sum P_n=I$を満たす組毎に直交な射影の列$\{P_n\}$
    と,$\Sp(T)$上の正規化されたRadon積分の列$(\int_n)$と,等長写像$U_n:L^2_n(\Sp(T))\to P_n(H)$とが存在する.
    また,$L^\infty(\Sp(T))$を,Radon積分$\in:=\sum 2^{-n}\int_n$に関する本質的に有界な関数の空間として,
    次は$C(\Sp(T))\to C^*(T)$を延長する等長$*$-同型である.
    \[\xymatrix@R-2pc{
        L^\infty(\Sp(T))\ar[r]&W^*(T)\\
        \rotatebox[origin=c]{90}{$\in$}&\rotatebox[origin=c]{90}{$\in$}\\
        f\ar@{|->}[r]&f(T):=\sum U_nM_fu_n^*
    }\]
\end{theorem}

\begin{proposition}
    定理のスペクトル写像$L^\infty(\Sp(T))\to W^*(T)$は,$L^\infty(\Sp(T))=(L^1(\Sp(T)))^*$に$w^*$-位相を,$B(H)$に$\sigma$-弱位相を入れたとき,位相同型になる.
    特に,スペクトル写像は$L^\infty(\Sp(T))$の有界集合上において,$w^*$-位相と弱位相について位相同型になる.
\end{proposition}

\chapter{非有界作用素}

\begin{quotation}
    非有界作用素については,稠密部分集合上で一致しても,大域的に一致するとは限らない.
    さらに定義域が全くかぶらないために演算が定義出来ない場合も多い.
\end{quotation}

\section{始域・延長・グラフ}  

\subsection{定義域と値域と演算}

\begin{tcolorbox}[colframe=ForestGreen, colback=ForestGreen!10!white,breakable,colbacktitle=ForestGreen!40!white,coltitle=black,fonttitle=\bfseries\sffamily,
title=]
    一般の作用素$S,T$について,$T=S$とは,グラフが一致することであることを明確に意識する必要がある.
    拡張性$S\subset T$は,グラフの包含関係$\Graph(S)\subset\Graph(T)$の略記と見ると筋が良い.
\end{tcolorbox}

\begin{definition}[operator \textbf{in} a Hilbert space, domain, densely defined, extension]
    Hilbert空間$H$\textbf{内}の作用素$T$とは,$H$の部分空間$D(T)$上の作用素$T:D(T)\to H$をいう.
    $D(T)$を\textbf{定義域}という.
    \begin{enumerate}
        \item $\oo{D(T)}=H$が成り立つとき,$T$は\textbf{稠密に定義されている}という.以降,これを暗黙に仮定することも多い.
        \item $D(S)\subset D(T),S=T\on D(S)$が成り立つとき,$T$は$S$の\textbf{延長}であるといい,$S\subset T$で表す.これは$\cG(S)\subset\cG(T)$とも同値.
        $T=S\Leftrightarrow T\subset S\land T\supset S$が成り立つ.
        \item 値域を$R(T):=T(D(T))$で表す.
    \end{enumerate}
\end{definition}

\begin{definition}[和, 積, 逆]
    作用素$S,T$に対して,$S+T,ST,T^{-1}$の定義域を次のように定める.
    \begin{enumerate}
        \item $D(S+T)=D(S)\cap D(T)$.
        \item $D(ST)=D(T)\cap T^{-1}(D(S))$.
        \item $D(T^{-1})=R(T),R(T^{-1})=D(T)$.
    \end{enumerate}
    続いて,和と積をこの定義域上で定め,逆は$\Graph(T)$の第一成分と第二成分を取り替えて得るグラフが定める作用素とする.
    とすると,和と積は結合的であるが,分配的ではない.
\end{definition}
\begin{remark}
    定義域を暗黙化している関係上,作用素$T$が単射ならば逆$T^{-1}$が定まる.
\end{remark}

\begin{lemma}
    $S,T$が単射であるとき,$(ST)^{-1}=T^{-1}S^{-1}$が成り立つ.
\end{lemma}

\begin{lemma}[グラフの特徴付け]\label{lemma-characterization-of-graph}
    部分空間$\cG\subset X\otimes Y$について,
    \begin{enumerate}
        \item $\cG$はある線型作用素のグラフである.
        \item 任意の$(0,y)\in\G$について,$y=0$である.
    \end{enumerate}
    特に,$\cG$がグラフならば,任意の$\cG$の部分空間はグラフである.
\end{lemma}
\begin{Proof}\cite{Conway} Def. X.1.3.
    (2)$\Rightarrow$(1)を示せば良い.
    部分空間$\cG$に対して定義域を
    \[\D:=\Brace{x\in X\mid \exists_{y\in Y}\;(x,y)\in\G}.\]
    と定める.このとき,任意の$x\in\D$に対して,$(x,y)\in\G$を満たす$y\in Y$は一意的である.
    仮に$y_1,y_2$はいずれも$(x,y_1),(x,y_2)\in\G$を満たすとすると,$\G$は線型空間であるから$(0,y_1-y_2)\in\G$を満たす.
    すると仮定より$y_1=y_2$が従う.
    よって,これについて$T(x)=y$として対応$T:\D\to Y$を定めると,これは線型作用素を定め,グラフは$\G$である.
\end{Proof}

\subsection{共役作用素}

\begin{tcolorbox}[colframe=ForestGreen, colback=ForestGreen!10!white,breakable,colbacktitle=ForestGreen!40!white,coltitle=black,fonttitle=\bfseries\sffamily,
title=]
    $(Tx|y)=(x|T^*y)$を満たす作用素$T^*$のうち最大のものを共役作用素と定義する.
\end{tcolorbox}

\begin{definition}[adjoint operator]
    $H$内の稠密に定義された作用素$T:D(T)\to H$に対して,
    \[D(T^*):=\Brace{x\in H\mid D(T)\to H;y\mapsto(Ty|x)\in B(D(T),H)}\]
    とする.これは,Rieszの表現定理より,$\forall_{y\in D(T)}\;\forall_{x\in D(T^*)}\;\exists!_{T^*x\in H}\;(y|T^*x)=(Ty|x)$と同値.
    すなわち,$D(T^*)$上には$T^*:D(T^*)\to H$が定まる.これを$T$の\textbf{随伴}という.
    $T^*$は稠密に定義されているとは限らない\ref{thm-closable-and-ajoint-operator}.
\end{definition}

\begin{lemma}
    $S,T,S+T,ST$は稠密に定義されているとする.
    \begin{enumerate}
        \item $S\subset T\Rightarrow T^*\subset S^*$である.
        \item $S^*+T^*\subset(S+T)^*$.
        \item $T^*S^*\subset(ST)^*$.
    \end{enumerate}
\end{lemma}

\begin{lemma}
    $T$が稠密に定義されているとき,$\Ker T^*=(\Im T)^\perp$.
    特に,$\Brace{x\in D(T^*)\mid T^*x=\o{\lambda}x}=\Im(T-\lambda I)^\perp$.
\end{lemma}

\begin{lemma}
    任意の作用素の共役作用素は閉作用素である.
\end{lemma}
\begin{Proof}
    \ref{thm-closable-and-ajoint-operator}(1).
\end{Proof}

\subsection{対称作用素}

\begin{tcolorbox}[colframe=ForestGreen, colback=ForestGreen!10!white,breakable,colbacktitle=ForestGreen!40!white,coltitle=black,fonttitle=\bfseries\sffamily,
title=]
    $T\subset T^*$なるとき$T$を対称作用素といい,$T=T^*$となるとき自己共役作用素という.
\end{tcolorbox}

\begin{definition}[symmetric]
    稠密に定義された作用素$S$が\textbf{対称}であるとは,$\forall_{x,y\in D(S)}\;(Sx|y)=(x|Sy)$を満たすことをいう.
\end{definition}

\begin{lemma}[対称作用素の随伴による特徴付け]
    稠密に定義された作用素$S$について,次の2条件は同値.
    $\bF=\C$のとき,(3)とも同値.
    \begin{enumerate}
        \item $S$は対称である.
        \item $S\subset S^*$.
        \item $\forall_{x\in D(S)}\;(Sx|x)\in\R$.
    \end{enumerate}
\end{lemma}

\begin{lemma}
    対称作用素$S$と$\lambda=\al+i\beta\in\C$について,
    \begin{enumerate}
        \item $\forall_{x\in D(S)}\;\norm{(S-\lambda I)x}^2=\norm{(S-\al I)x}^2+\beta^2\norm{x}^2$.
        \item $\beta\ne0$ならば,$S-\lambda I$は単射で,$(S-\lambda I)^{-1}$は$\abs{\beta}^{-1}$によって抑えられる.
    \end{enumerate}
\end{lemma}

\begin{definition}[maximal symmetric, self-adjoint]
    対称作用素$S$であって,任意の対称作用素$T$に対して$S\subset T\Rightarrow S=T$を満たすものを,\textbf{極大対称作用素}という.
    $S=S^*$のとき,$S$を\textbf{自己共役}であるという.
\end{definition}

\begin{lemma}
    任意の自己共役作用素は極大対称的であるが,逆は十分ではない.
\end{lemma}

\subsection{閉作用素の定義と例}

\begin{tcolorbox}[colframe=ForestGreen, colback=ForestGreen!10!white,breakable,colbacktitle=ForestGreen!40!white,coltitle=black,fonttitle=\bfseries\sffamily,
title=]
    作用素の有界性を考えたいが,定義域が全域でないときは,全域であるときの同値概念「閉性」を主役に据える.
    全域で定義された閉作用素は有界である.
\end{tcolorbox}

\begin{definition}[closed operator]
    $H$内の作用素$T$のグラフ$G(T)=\Brace{(x,Tx)\in H\oplus H\mid x\in D(T)}$が$H\oplus H$内で閉部分空間であるとき,これを\textbf{閉作用素}という.
    直和$H\oplus H$は$l^1$-ノルムを入れてあるが,これを\textbf{グラフノルム}ともいう.
\end{definition}

\begin{lemma}[閉作用素の特徴付け]
    $X,Y$をBanach空間,$T:X\to Y$を線型作用素とする.
    \begin{enumerate}
        \item $T$は閉作用素である.
        \item $T$の定義域$\Dom(T)$は$X$の内でグラフノルムについて完備である.
        \item $\Dom(T)$内の$x$に収束する列$(x_n)$について,$(Tx_n)$が$y$に収束するならば,$x\in D(T)$かつ$Tx=y$が成り立つ.
    \end{enumerate}
\end{lemma}
\begin{example}\mbox{}
    \begin{enumerate}
        \item 全域で定義されている有界作用素は閉作用素である.逆に,全域で定義されている閉作用素は有界である(閉グラフ定理).
        \item $D(T)=D(T^*)=H$を満たす作用素は有界である.
        \item 特に,全域で定義されている対称作用素は,自己共役で有界である.
        \item $D(T)=H$または$\Im(T)=H$のいずれか一方が成り立てば,自己共役である.
        \item 任意の作用素の共役作用素は閉作用素である.
    \end{enumerate}
\end{example}

\subsection{可閉作用素}

\begin{tcolorbox}[colframe=ForestGreen, colback=ForestGreen!10!white,breakable,colbacktitle=ForestGreen!40!white,coltitle=black,fonttitle=\bfseries\sffamily,
title=]
    閉作用素に包含される作用素を可閉作用素という.
    閉作用素のグラフの任意の部分空間は閉作用素のグラフになる\ref{lemma-characterization-of-graph}.

    2回共役を取れること(共役も稠密に定義されていること)は可閉性と同値\ref{thm-closable-and-ajoint-operator}.
    $T\subset T^{**}$が成り立ち,$T^{**}$は閉作用素だが,$T$が閉のとき等号が成り立つ.つまり,$\o{T}=T^{**}$となる.
\end{tcolorbox}

\begin{definition}[closable / preclosed, closure, core]\mbox{}
    \begin{enumerate}
        \item $H$内の作用素$T$が\textbf{可閉}または\textbf{前閉}であるとは,グラフ$\cG(T)$のノルム閉包が,ある作用素$\o{T}$のグラフになっていることをいう.
        このとき,$\o{T}$は$T$を含む最小の閉作用素であり,$T$の\textbf{閉包}という.
        \item 閉作用素$T$と$D(T)$の部分空間$D_0$について,$T_0:=T|_{D_0}$の閉包が$T$であるとき,$D_0$を$T$の\textbf{核}という.$\Graph(T)$の稠密部分集合の$\pr_1$による像は核である.
    \end{enumerate}
\end{definition}

\begin{lemma}[可閉性の特徴付け]
    $H$内の作用素$T$について,次の条件は同値.
    \begin{enumerate}
        \item $T$は可閉である.
        \item $\D(T)$内の$0$に収束する列$(x_n)$について,$(Tx_n)$のただ一つの集積点は$0$である.
        \item ある閉作用素$S$が存在して,$T\subset S$を満たす.
    \end{enumerate}
\end{lemma}
\begin{Proof}\mbox{}
    \begin{description}
        \item[(1)$\Leftrightarrow$(3)] (1)$\Rightarrow$(3)は明らかだから逆を示す.
        これは作用素のグラフの特徴付け\ref{lemma-characterization-of-graph}から特に,グラフの部分空間はやはりある線型作用素のグラフであることから従う.
        \item[(1)$\Rightarrow$(2)] $\oo{\Graph T}$がある作用素のグラフであるとき,任意の$(0,v)\in\oo{\Graph T}$に対して$v=0$が成り立つ.$\D(T)$内の任意の$0$に収束する列$\{x_n\}\subset\D(T)$に対して,$Tx_n$が$0$以外の集積点を持つならば,$Tx_{n_k}$が$0$以外に収束する部分列$\{x_{n_k}\}\subset\D(T)$が取れてしまうので(これは一般の第1可算な位相空間で成り立つ.距離化可能な位相線型空間は第1可算である),$\oo{\Graph T}$がグラフであることに矛盾.$\{Tx_n\}$が$0$に集積しないならば,$\oo{\Graph T}$がある線型作用素のグラフであることに矛盾する.
        \item[(2)$\Rightarrow$(1)] 逆に,(2)の条件が満たされるとき,
        任意の$(0,v)\in \oo{\Graph T}$に対して,これに収束する点列が取れるから,(2)の仮定より$v=0$が必要.
        $\oo{\Graph T}$がある作用素のグラフになることが解る.
    \end{description}
\end{Proof}

\subsection{自己共役作用素}

\begin{tcolorbox}[colframe=ForestGreen, colback=ForestGreen!10!white,breakable,colbacktitle=ForestGreen!40!white,coltitle=black,fonttitle=\bfseries\sffamily,
title=]
    $U(z,y)=(-y,z)$によって定まるユニタリ作用素$U:H\otimes H\iso H\otimes H$について,$U(\cG(T^*))=\cG(A)^\perp$の関係が成り立つ.$T$のグラフの直交補空間に,回転$U^{-1}$を施すと$T^*$のグラフを得るのである.

    2回自己共役を取ることは,作用素の閉包を取ることに同値.この操作が出来ることは,可閉であることに同値.
\end{tcolorbox}

\begin{theorem}[可閉性の自己共役作用素による特徴付け]\label{thm-closable-and-ajoint-operator}
    $T$を$H$内の稠密に定義された作用素とする.
    \begin{enumerate}
        \item $T^*$は閉作用素である.
        \item 直交分解$H\oplus H=\oo{\cG(T)}\oplus U(\cG(T^*))$が存在する.ただし,$U\in\Iso(H\oplus H)$は$U(z,y)=(-y,z)$によって定まるユニタリ作用素とした.
        \item $T$が可閉であることと,$T^*$が稠密に定義されていることとは同値.このとき,$\o{T}=T^{**}$すなわち$\oo{\Graph(T)}=\Graph(T^{**})$が成り立つ.特に,閉作用素$T$について,$T^{**}=T$.
    \end{enumerate}
\end{theorem}
\begin{Proof}\mbox{}
    \begin{enumerate}
        \item $T^*$のグラフから分かる.
        \item $U(\cG(T^*))=\cG(A)^\perp$の関係が成り立つが,$T^*$は閉だからこれらは互いに直交補空間である.
        \item $T^*$が稠密に定義されているとき,さらに共役作用素を取れて,$T\subset T^{**}$だから,$T$は前閉である.逆に,
    \end{enumerate}
\end{Proof}

\begin{definition}[essentially self-adjoint]
    対称作用素$S$は$S\subset S^*$を満たすから,$S^*$も稠密に定義されていて,したがって$S$は可閉である.
    $\o{S}=S^*$が成り立つとき,対称作用素$S$は\textbf{本質的に自己共役}であるという.
    一般に$S\subset\o{S}\subset S^*$が成り立つためである.
\end{definition}

\subsection{逆作用素と自己共役}

\begin{proposition}
    作用素$T:H\to H$が次の3条件を満たすならば,$T^{-1},T^*$も次の3条件を満たし,特に単射であり,$(T^*)^{-1}=(T^{-1})^*$が成り立つ.
    \begin{enumerate}
        \item 稠密な定義域を持つ.
        \item 稠密な像を持つ.
        \item 単射である.
        \item 閉作用素である.
    \end{enumerate}
\end{proposition}

\begin{corollary}
    自己共役作用素$T$が単射(稠密な像を持つことと同値)ならば,$T^{-1}$は自己共役作用素である.
\end{corollary}

\subsection{閉作用素の性質}

\begin{theorem}
    稠密に定義された閉作用素$T$について,
    \begin{enumerate}
        \item $T^*T$は自己共役で,$D(T^*T)$は$T$の核である.
        \item $T^*T+I:D(T^*T)\to H$は全単射で,$(T^*T+I)^{-1}\in B(H)$かつ$0\le(T^*T+I)^{-1}\le I$を満たす.
        \item $(TT^*+I)^{-1}T$の閉包は$T(T^*T+I)^{-1}$であり,これは有界作用素で$\norm{T(T^*T+I)^{-1}}\le1$.
        \item ネット$((\ep^2T^*T+I)^{-1})_{\ep>0}$は$I$に強収束する.
    \end{enumerate}
\end{theorem}

\subsection{正規作用素}

\begin{proposition}[正規作用素]\label{prop-normal-operator}
    稠密に定義された閉作用素$T$について,次の3条件は同値.
    \begin{enumerate}
        \item $D(T)=D(T^*)$かつ$\forall_{x\in D(T)}\;\norm{Tx}=\norm{T^*x}$.
        \item $T^*T=TT^*$.
        \item 自己共役な作用素$A,B$が存在して,$T=A+iB,T^*=A-iB$かつ$\forall_{x\in D(T)}\;\norm{Tx}^2=\norm{Ax}^2+\norm{Bx}^2$を満たす.
        このとき,$A,B$はそれぞれ$\frac{1}{2}(T+T^*),\frac{1}{2}i(T^*-T)$の閉包である.
    \end{enumerate}
\end{proposition}

\begin{example}[可換性の取扱に関するNelsonの例]
    
    
\end{example}

\subsection{半有界作用素}

\begin{definition}[semibounded / bounded from below, positive]
    稠密に定義された対称作用素$S$が,$\exists_{\al\in\R}\;\forall_{x\in D(S)}\;(Sx|x)\ge\al\norm{x}^2$を満たすとき,\textbf{半有界}または\textbf{下に有界}であるという.
    この条件を$S\ge\al I$で表す.特に,$\al=0$について$S\ge0$が成り立つとき,正であるという.
\end{definition}

\begin{lemma}\mbox{}
    \begin{enumerate}
        \item 任意の稠密に定義された閉作用素$T$について,$T^*T\ge0$.
        \item 任意の稠密に定義された対称作用素$S$について,有界であること$\o{S}\in B(H)$と$S,-S$のいずれも半有界であることとは同値.
    \end{enumerate}
\end{lemma}

\begin{theorem}[Friedrichs extension, Friedrichs-Freudenthal theorem]
    任意の半有界作用素$S$は,同じ下界を持った自己共役な拡張を持つ.
\end{theorem}

\subsection{非有界作用素の例}

\section{Cayley変換}

\begin{tcolorbox}[colframe=ForestGreen, colback=ForestGreen!10!white,breakable,colbacktitle=ForestGreen!40!white,coltitle=black,fonttitle=\bfseries\sffamily,
title=]
    対称閉作用素を自己共役作用素に延長することを試みる.
    こうして,非有界作用素についてのスペクトル定理を立てることを目指す.
\end{tcolorbox}

\subsection{定義と性質}

\begin{tcolorbox}[colframe=ForestGreen, colback=ForestGreen!10!white,breakable,colbacktitle=ForestGreen!40!white,coltitle=black,fonttitle=\bfseries\sffamily,
title=]
    有界な自己共役作用素をユニタリ作用素に変換するときに$z\mapsto\exp(iz)$を用いたように,非有界な対称閉作用素を等長作用素に変換するには,Cayley変換$\kappa(z)=\frac{z-i}{z+i}$を用いる.
\end{tcolorbox}

\begin{definition}[Cayley transformed]
    $S$を稠密に定義された対称作用素とする.
    $S$のCayley変換とは,
    \[\kappa(S):=\frac{S-iI}{S+iI}\]
    によって定まる作用素とする.定義域と値域は
    \[\Dom(\kappa(S))=(S+iI)\Dom(S),\quad\Im(\kappa(S))=(S-iI)\Dom(S).\]
\end{definition}

\begin{lemma}\mbox{}
    \begin{enumerate}
        \item Cayley変換$\kappa(S):(S+iI)\Dom(S)\to(S-iI)\Dom(S)$は等長写像である.
        \item $I-\kappa(S)$は単射で,$\Im(I-\kappa(S))=\Dom(S)$を満たす.
        \item $i(I+\kappa(S))(I-\kappa(S))^{-1}=S$.
    \end{enumerate}
\end{lemma}

\begin{theorem}\mbox{}
    \begin{enumerate}
        \item Cayley変換は,稠密に定義された対称作用素の全体と,$I-U$が稠密な像を持つような$H$内の等長作用素$U$の全体との間に,順序同型を定める.
        \item 次の4つの数学的対象に関する閉性は一致する:作用素$S$,像$\Im(S+iI)$,$\Im(S-iI)$,作用素$\kappa(S)$.
    \end{enumerate}
\end{theorem}

\subsection{自己共役性の特徴付け}

\begin{proposition}
    稠密に定義された対称作用素$S$について,次の3条件は同値:
    \begin{enumerate}
        \item $S$は自己共役.
        \item $S\pm iI$はいずれも全射.
        \item $\kappa(S)$はユニタリ.
    \end{enumerate}
\end{proposition}

\subsection{不足指数}

\begin{tcolorbox}[colframe=ForestGreen, colback=ForestGreen!10!white,breakable,colbacktitle=ForestGreen!40!white,coltitle=black,fonttitle=\bfseries\sffamily,
title=]
    対称な$S$の自己共役な拡張を見つける問題は,$H$内のユニタリ$U$であって$U|_{\Im(S+iI)}=\kappa(S)$を満たすものを見つけることと同値.
    これは$\Delta_+=\Delta_-$を満たすとき可能である.
\end{tcolorbox}

\begin{definition}[defect indeces]
    $H$を可分とし,$S$をその中の稠密に定義された対称作用素とする.$S$の\textbf{不足指数}とは,
    \[\Delta_+:=\dim(\Im(S+iI)^\perp),\quad\Delta_-:=\dim(\Im(S-iI)^\perp).\]
    $\kappa(S)$の定義域と像の閉包の余次元とみなせる.
\end{definition}

\begin{lemma}
    対称作用素$S$が閉であるとき,自己共役な拡張を持つことと,指数$\Index\kappa(S)=0$となることとは同値.
\end{lemma}

\begin{example}[自己共役な拡張を持たない場合]
    $U$を$H$から真の部分空間への等長作用素とする.例えば,ユニラテラルシフトとすると,$S:=\kappa^{-1}(U)$は極大対称であるが,自己共役ではない.
\end{example}

\begin{lemma}
    対象閉作用素$S$について,
    \begin{enumerate}
        \item 極大対称であることと,$\Delta_+=0\lor\Delta_-=0$とは同値.
        \item 自己共役であることと,$\Delta_+=0\land\Delta_-=0$とは同値.
    \end{enumerate}
\end{lemma}

\subsection{可換性}

\begin{lemma}
    $S$を作用素,$T\in B(H)$とする.
    \begin{enumerate}
        \item $\Dom(S+T)=\Dom(S),\Dom(TS)=\Dom(S)$.
        \item $(S+T)^*=S^*+T^*$かつ$(TS)^*=S^*T^*$.
    \end{enumerate}
\end{lemma}

\begin{definition}[commutative]
    $S$を作用素,$T\in B(H)$とする.
    $TS\subset ST$を満たすとき,$S$と$T$とは\textbf{可換}であるという.
    すなわち,$T(D(S))\subset D(S)$かつ$\forall_{x\in D(S)}\;TSx=STx$であることをいう.
    グラフでは,$(T\oplus T)(\Graph(S))\subset\Graph(S)$を満たすことをいう.
\end{definition}

\begin{lemma}
    $S$を作用素,$T\in B(H)$とする.
    \begin{enumerate}
        \item $S$が閉であるとき,$S$と可換な元のなす代数$\{S\}'$は強閉である.
        \item $T\in\{S\}'\Rightarrow T^*\in\{S^*\}'$.
        \item 稠密に定義された閉作用素$S$について,$\{S\}'\cap\{S^*\}'$は強閉な単位的$*$-部分代数,すなわち,von Neumann代数である.
    \end{enumerate}
\end{lemma}

\subsection{付随するvon Neumann代数}

\begin{definition}[affiliated operator]
    稠密に定義された閉作用素$S$について,
    \begin{enumerate}
        \item $W^*(S):=(\{S\}'\cap\{S^*\}')'$をおく.この記法は$S\in B(H)$の場合の$S$が生成するvon Neumann代数の記法\ref{notation-W*(T)}と整合的である.
        \item von Neumann代数$\A$に対して$W^*(S)\subset\A$が成り立つとき,$S$は$\A$に付随するといい,$S\eta\A$と表される..これは$\A'\subset\{S\}'\cap\{S^*\}'$に同値.
    \end{enumerate}
\end{definition}

\begin{lemma}
    $S$を自己共役作用素,$T\in B(H)$とする.
    \begin{enumerate}
        \item $T\in\{S\}'\Leftrightarrow T\in\{\kappa(S)\}'$.特に,$W^*(S)=W^*(\kappa(S))$.
        \item $S$がvon Neumann$\A$に付随することと,$\kappa(S)\in\A$とは同値.
    \end{enumerate}
\end{lemma}

\begin{lemma}
    正規作用素$T$がvon Neumann代数$\A$に随伴するとする.このとき,$T=A+iB$を満たす自己共役作用素$A,B$\ref{prop-normal-operator}も$\A$に付随する.
\end{lemma}

\subsection{作用素のスペクトル}

\begin{definition}[resolvent set, spectrum, resolvent function]
    $T$を作用素とする.
    \begin{enumerate}
        \item $\lambda I-T:\Dom(T)\to H$が全単射で,かつ,$\exists_{t>0}\;\forall_{x\in\Dom(T)}\;\norm{(\lambda I-T)x}\ge t\norm{x}$を満たす$\lambda$全体の集合を\textbf{解核集合}という.
        \item 解核集合の補集合を\textbf{スペクトル}といい,$\Sp(T)$で表す.
        \item $R(\lambda):=(\lambda I-T)^{-1}:\C\setminus\Sp(T)\to B(H)$を解核関数という.
    \end{enumerate}
\end{definition}

\begin{proposition}
    $T$を作用素,$\lambda\notin\Sp(T)$とする.このとき,任意の$\abs{\zeta}<\norm{R(\lambda)}^{-1}$を満たす複素数$\zeta\in\C$に対して,$\lambda-\zeta\notin\Sp(T)$であり,
    \[R(\lambda-\zeta)=\sum^\infty_{n=0}R(\lambda)^{n+1}\zeta^n.\]
    特に,$\Sp(T)$は$\C$の閉集合である.
\end{proposition}

\begin{example}
    Volterra作用素は空なスペクトルを持つ.
    $M_\id\in B(L^2(\C))$は非有界なスペクトル$\C$を持つ.
\end{example}

\subsection{自己共役作用素のスペクトル}

\begin{proposition}
    自己共役作用素$S$について,
    \begin{enumerate}
        \item スペクトル$\Sp(S)$は$\R$の非空な閉部分集合である.
        \item $\lambda\in\Sp(S)\Leftrightarrow\kappa(\lambda)\in\Sp(\kappa(S))$.
        \item $\lambda$は固有値である$\Leftrightarrow\;\kappa(\lambda)$は$\kappa(S)$の固有値である.
    \end{enumerate}
\end{proposition}

\section{非有界スペクトル理論}

\begin{tcolorbox}[colframe=ForestGreen, colback=ForestGreen!10!white,breakable,colbacktitle=ForestGreen!40!white,coltitle=black,fonttitle=\bfseries\sffamily,
title=]
    有界作用素の理論を通じて,自己共役作用素$S$から,スペクトル写像$f\mapsto f(S)$を$\B_b(\Sp(S))$上に構成することを考える.
    このとき,Cayley変換を用いる.スペクトルの代数化$\{f(S)\}$を,可換von Neumann代数$W^*(S)$と結びついた正規作用素の代数として調べる.
    この理論の応用としては,Stoneの定理や,非有界作用素の極分解を導く.
\end{tcolorbox}

\subsection{乗法作用素}

\begin{notation}[essential homomorphism]
    $X$を$\sigma$-コンパクトな局所コンパクトハウスドルフ空間,
    $\L(X)$をその上の可測関数,$\B(X)$をBorel関数,$\cN(X)$を零関数とする.
    \begin{enumerate}
        \item $\L(X)=\B(X)+\cN(X)$が成り立つ.
        $L(X):=\L(X)/\cN(X)=\B(X)/\cN(X)$は$*$-代数となる.
        \item $\L(X)$から$H$内の正規作用素への斉次写像$\Phi$が\textbf{本質的な準同型}であるとは,$\forall_{f,g\in\L(X)}\;\Phi(f+g),\Phi(fg)$はそれぞれ$\Phi(f)+\Phi(g),\Phi(f)\Phi(g)$の閉包であることをいう.
    \end{enumerate}
    $\L(X)$
\end{notation}

\begin{proposition}
    $L^2(X)$内の乗法作用素への対応$L(X)\ni f\mapsto M_f$は,$L^2(X)$内の正規作用素であってvon Neumann代数$\A:=\Brace{M_f\mid f\in\L^\infty(X)}$に付随する作用素の空間に,本質的な$*$-同型を定める.
\end{proposition}

\subsection{スペクトル定理}

\begin{theorem}
    $H$を可分Hilbert空間,$S$をその中の自己共役作用素とし,$\kappa(S)$を重複度なしとする.
    このとき,ある$\Sp(S)$上の有限Radon積分と同型$U:L^2(\Sp(S))\to H$が存在して,対応
    \[\xymatrix@R-2pc{
        L(\Sp(S))\ar[r]&\{W^*(S)\text{に付随する正規作用素のなす}*\text{-代数}\}\\
        \rotatebox[origin=c]{90}{$\in$}&\rotatebox[origin=c]{90}{$\in$}\\
        f\ar@{|->}[r]&f(S):=UM_fU^*
    }\]
    は本質的$*$-同型である.さらに,$\id(S)=S,1(S)=I$で,$\kappa(S)$の2つの意味は一致する.
\end{theorem}

\begin{definition}[multiplicity-free]
    可分なHilbert空間$H$内の自己共役作用素$S$が\textbf{重複度なし}であるとは,ある元$x\in\cap_{n\ge0}\Dom(S^n)$が存在して,$\{S^nx\in H\mid n\ge0\}$が生成する部分空間は$H$上稠密であることをいう.
\end{definition}

\begin{proposition}
    自己共役作用素$S$が重複度なしであることと,$\kappa(S)$が重複度なしであることとは同値.
\end{proposition}

\begin{lemma}
    略.
\end{lemma}

\begin{theorem}\label{thm-unbounded-spectral-theorem-2}
    $H$を可分Hilbert空間,$S$をその中の自己共役作用素とする.
    このとき,ある$\Sp(S)$上の有限Radon積分が存在して,対応
    \[\xymatrix@R-2pc{
        L(\Sp(S))\ar[r]&\{W^*(S)\text{に付随する正規作用素のなす}*\text{-代数}\}\\
        \rotatebox[origin=c]{90}{$\in$}&\rotatebox[origin=c]{90}{$\in$}\\
        f\ar@{|->}[r]&f(S)
    }\]
    は本質的$*$-同型である.さらに,$\id(S)=S,1(S)=I$で,$\kappa(S)$の2つの意味は一致する.
\end{theorem}

\subsection{連続なスペクトル写像}

\begin{proposition}
    $H$を可分Hilbert空間,$S$をその中の自己共役作用素とする.
    このとき,任意の$x\in H$に対して$\Sp(S)$上の有限なRadon積分$\int_x$が存在し,
    スペクトル定理\ref{thm-unbounded-spectral-theorem-2}のRadon積分に関して絶対連続で,
    \[\forall_{f\in\B(\Sp(S))_+}\quad\int_xf=(f(S)x|x)\]
\end{proposition}

\begin{corollary}
    $\{f_n\}\subset\B(\Sp(S))$は$f\in\B(\Sp(S))$に各点収束し,$\forall_{n\in\N}\;\abs{f_n}\le g\in\B(\Sp(S))$を満たすとする.
    このとき,$\forall_{x\in\Dom(g(S))}\;f_n(S)x\to f(S)x$.
\end{corollary}

\subsection{強連続1-パラメータユニタリー群}

\begin{tcolorbox}[colframe=ForestGreen, colback=ForestGreen!10!white,breakable,colbacktitle=ForestGreen!40!white,coltitle=black,fonttitle=\bfseries\sffamily,
title=]
    量子系の発展過程は,強連続1-パラメータユニタリ群となる.スペクトル理論は,この群と,非有界自己共役作用素とをつなぐ.
\end{tcolorbox}

\begin{definition}
    \textbf{強連続1-パラメータユニタリ群}または\textbf{$C_0$-ユニタリ群}とは,強連続関数$U:\R\to U(H)$であって,$\forall_{s,t\in\R}\;U_{s+t}=U_sU_t$を満たすものをいう.
\end{definition}

\subsection{Stoneの定理}

\begin{theorem}[Stone (1932)]
    $(U_t)_{t\in\R}$を強連続な1-パラメータユニタリ群とする.
    ある自己共役作用素$S$が存在して,$\forall_{t\in\R}\;U_t=\exp(itS)$を満たす.
\end{theorem}

\subsection{極分解}

\begin{proposition}
    任意の稠密に定義された閉作用素$T$の絶対値$\abs{T}$に対して,
    \begin{enumerate}
        \item $\Dom(\abs{T})=\Dom(T)$かつ$\forall_{x\in\Dom(T)}\;\norm{\abs{T}x}=\norm{Tx}$.
        \item 部分等長写像$U$がただ一つ存在して,$\Ker U=\Ker T$かつ$T=U\abs{T}$.
        \item $U^*U\abs{T}=\abs{T},U^*T=\abs{T},UU^*T=T$.
    \end{enumerate}
\end{proposition}

\section{作用素解析}

\begin{tcolorbox}[colframe=ForestGreen, colback=ForestGreen!10!white,breakable,colbacktitle=ForestGreen!40!white,coltitle=black,fonttitle=\bfseries\sffamily,
title=]
    スペクトル写像$L(\Sp(S))\to\{W^*(S)\text{に付随する正規作用素のなす}*\text{-代数}\}$を得た.
    これは,複素関数$f$を,作用素の関数$f(S)$に一般化する手続きとも見れる.

    $X$をBanach空間,$A\in B(X)$とすると,$f$が正則のとき,冪級数によって定義できる\ref{not-extended-exp}.
    このような,関数代数から,Banach代数$B(X)$への準同型を,operational calculusやfunctional calculusという.
\end{tcolorbox}

\subsection{Dunford積分}

\begin{tcolorbox}[colframe=ForestGreen, colback=ForestGreen!10!white,breakable,colbacktitle=ForestGreen!40!white,coltitle=black,fonttitle=\bfseries\sffamily,
title=]
    Dunford積分はBochner積分の例であり,関数空間上のBanach空間値積分$\O(\Sp(A))\to H(X)$である.
\end{tcolorbox}

\begin{notation}
    $\Sp(A)$の近傍で正則な関数の全体を$\F(A)$または$\O(\Sp(A))$で表す.$f\in\F(A)$の定義域を$U_f$で表す.
\end{notation}

\begin{definition}
    $f\in\F(A)$について,$\Sp(A)\subset D\subset U_f$を満たす区分的に滑らかな単純閉曲線による境界を持つ領域$D$を用いて,
    \[f(A):=\frac{1}{2\pi i}\int_{\partial D}f(z)R(z;A)dz\]
    と定める.
\end{definition}

\begin{lemma}[operational calculus]
    $f,g\in\O(\Sp(A))$について,
    \begin{enumerate}
        \item $\al f(T)+\beta g(T)=(\al f+\beta g)(T)$.
        \item $f(T)g(T)=(fg)(T)$.
        \item $f(\Sp(T))=\Sp(f(T))$.
    \end{enumerate}
\end{lemma}

\section{作用素の半群}

\subsection{半群の定義}

\begin{tcolorbox}[colframe=ForestGreen, colback=ForestGreen!10!white,breakable,colbacktitle=ForestGreen!40!white,coltitle=black,fonttitle=\bfseries\sffamily,
title=]
    \ref{not-extended-exp}のように,1階の線形常微分方程式の初期値問題の解は,フロー$u=(e^{tA}u_0)_{t\in\R}$となる.
    $A$は一般の(非有界)作用素に一般化出来る.

    フローは1-パラメータ変換群であったが,熱方程式などの拡散方程式の解は$\R_+$で添字付けられた半群となる.
    これは,解が予測不可能なランダム性を含み,不可逆であることに対応する.
\end{tcolorbox}

\begin{definition}[one parameter semi-group]
    Banach空間$X$内の有界線型作用素の族$\{T_t\}_{t\in\R_+}\subset B(X)$が\textbf{1-パラメータ変換半群}とは,
    \begin{enumerate}
        \item $T_0=I$.
        \item $T_tT_s=T_{t+s}$.
    \end{enumerate}
    を満たすことをいう.
    \begin{enumerate}
        \item $\forall_{t\in\R_+}\;\norm{T_t}\le1$を満たすとき,\textbf{縮小半群}であるという.
        \item $\exists_{M\ge1}\;\forall_{t\in\R_+}\;\norm{T_t}\le M$を満たすとき,\textbf{有界半群}であるという.
        \item $\exists_{M\ge1,\beta\in\R}\;\forall_{t\in\R_+}\;\norm{T_t}\le Me^{\beta t}$が成り立つとき,\textbf{準有界半群}であるという.$\beta\le0$と取れるとき,有界である.
        \item \textbf{強連続半群}または\textbf{$C_0$-半群}とは,族が強連続であることをいう.これは,任意の$x\in X$に対して,そこでの評価が定める$T_t(x):\R_+\to X$が連続であることをいう.\footnote{$B(H)$の作用素強位相は,$H$のノルムが誘導する半ノルムの族が誘導する局所凸位相を言うので.}
    \end{enumerate}
\end{definition}

\begin{example}[熱方程式の解の半群]
    熱方程式の初期値問題
    \[\pp{u}{t}=\frac{1}{2}\Delta u,\quad u(0,x)=f(x)\]
    の解は,$N(x,tI)$の確率密度関数$p(t,x,y)$を用いて
    \[u_f(t,x):=\int_{\R^d}p(t,x,y)f(y)dy\quad f\in C_\infty(\R^d)\]
    とすると,$u_f(t,x)$は解になる.
    すなわち,$p$を積分核とする積分作用素$T_p:C_0(\R^d)\to C^\infty(\R^d)$の像が解を与える.
    \begin{enumerate}
        \item 初期分布$f\in C_0(\R^d)$には連続性しか仮定していないが,$u_f(t,x)$は$C^\infty$-級である(平滑化).
        \item フロー$T_t:\R_+\to B(C_0(\R^d),C^\infty(\R^d))$は半群性を満たす.この条件はChapman-Kolmogorov方程式に同値.
        \item 強連続である:$\lim_{t\to0}\norm{T_tf-f}=0$.
        \item 縮小性を満たす:$\norm{T_tf}\le\norm{f}$.
    \end{enumerate}
    なお,$x$を出発した粒子が時刻$t$に$y$の近傍に居る確率が$p(t,x,y)dy$となるような確率過程をBrown運動という.
\end{example}

\begin{example}
    最も単純なフローは,$X=L^p(0,\infty)$上の平行移動$(T_tu)(x)=u(t+x)$が定める半群$(T_t)_{t\in\R_+}$である.これは強連続である.
    一方で,$L^\infty(0,\infty)$上で考えると,有界ではあるが強連続ではない.
\end{example}

\subsection{強連続性の特徴付け}

\begin{theorem}
    半群$\{T_t\}_{t\in\R_+}\subset B(X)$について,(1)$\Rightarrow$(2)が成り立ち,$t=0$において右強連続ならば(2)$\Rightarrow$(1)も成り立つ.
    \begin{enumerate}
        \item $C_0$-半群である.
        \item 準有界である.
    \end{enumerate}
\end{theorem}

\subsection{生成作用素}

\begin{definition}[generator]
    半群$\{T_t\}_{t\in\R_+}\subset B(X)$について,その\textbf{生成作用素}とは,
    \begin{enumerate}
        \item 定義域を$\D(A):=\Brace{u\in X\mid\lim_{h\searrow0}\frac{T_hu-u}{h}\in X\text{が存在する}}$とし,
        \item $Au:=\lim_{h\searrow0}\frac{T_hu-u}{h}$と対応付けられる作用素をいう.
    \end{enumerate}
\end{definition}

\begin{lemma}
    $A$を$C_0$-半群$\{T_t\}_{t\in\R_+}\subset B(X)$の生成作用素とする.
    \begin{enumerate}
        \item $\forall_{a\in\D(A)}\;[\forall_{t>0}\;T_ta\in\D(A)]\land[\forall_{t\in\R_+}\;T_tAa=At_ta]$.
        \item $T_ta:\R_+\to X$は連続微分可能で,$\dd{}{t}T_ta=T_tAa=AT_ta$.
        \item $A$は閉作用素である.
        \item $A$は稠密に定義されている.また,$\forall_{n\in\N}\;\D(A^n)$も,$\cap_{n\in\N}\D(A^n)$も$X$上稠密である.
    \end{enumerate}
\end{lemma}

\begin{theorem}
    $C_0$-半群は,その生成作用素$A$により一意に定まる.
    その$C_0$-半群を$e^{tA}$で表す.
\end{theorem}

\begin{theorem}[整合性]
    $A\in B(X)$のとき,$T_t:=\sum^\infty_{n=0}\frac{t^n}{n!}A_n$とおけば$C_0$-半群であり,その生成作用素は$A$である.
    またこの対応$B(X)\to X$はノルム減少的である:
    $\forall_{t\in\R_+}\;\norm{T_t}\le e^{t\norm{A}}$.
\end{theorem}

\subsection{生成作用素のレゾルベントの性質}

\begin{tcolorbox}[colframe=ForestGreen, colback=ForestGreen!10!white,breakable,colbacktitle=ForestGreen!40!white,coltitle=black,fonttitle=\bfseries\sffamily,
title=]
    一般に,実数$A<0$について,留数定理より
    \[\frac{1}{\lambda-A}=\int_{\R_+}e^{-\lambda t}e^{tA}dt\]
    が成り立つ.この関係は$C_0$-半群に一般化される.
\end{tcolorbox}

\begin{theorem}
    $C_0$-半群$e^{tA}$は$\exists_{M>0,\beta\in\R}\;\norm{e^{tA}}\le Me^{\beta t}$を満たす準有界半群とする.
    任意の$\lambda>\beta$に対して,
    \begin{enumerate}
        \item $\lambda\in\rho(A)$.
        \item $\forall_{a\in X}\;(\lambda-A)^{-1}a=\int_{\R_+}e^{-\lambda t}e^{tA}adt$.
        \item $\norm{(\lambda-A)^{-1}}\le M(\lambda-\beta)^{-1}$.
    \end{enumerate}
\end{theorem}

\subsection{縮小半群の吉田-Hilleの定理}

\begin{tcolorbox}[colframe=ForestGreen, colback=ForestGreen!10!white,breakable,colbacktitle=ForestGreen!40!white,coltitle=black,fonttitle=\bfseries\sffamily,
title=]
    $X$内の稠密に定義された閉線型作用素$A$がある$C_0$-半群の生成作用素になるための十分条件を探す.
\end{tcolorbox}

\begin{theorem}[Yoshida-Hille]
    Banach空間$X$内の稠密に定義された閉線型作用素$A$について,次の2条件は同値.
    \begin{enumerate}
        \item ある縮小$C_0$-半群の生成作用素となる.
        \item $\forall_{\lambda>0}\;\lambda\in\rho(A)$かつ$\lambda\norm{(\lambda-A)^{-1}}\le 1$.
    \end{enumerate}
\end{theorem}

\begin{theorem}[R. Phillips]
    Banach空間$X$内の稠密に定義された閉線型作用素$A$について,次の2条件は同値.
    \begin{enumerate}
        \item ある準有界$C_0$-半群$\exists_{M\ge1,\beta\in\R}\;\forall_{t\in\R_+}\;\norm{T_t}\le Me^{\beta t}$の生成作用素となる.
        \item $\forall_{\lambda>\beta}\;\lambda\in\rho(A)$かつ$\forall_{n\in\N}\;(\lambda-\beta)^n\norm{(\lambda-A)^{-n}}\le M$.
    \end{enumerate}
\end{theorem}

\subsection{半群の表現定理}

\begin{theorem}
    $C_0$-半群$e^{tA}$について,
    \[\forall_{t\in\R_+}\quad e^{tA}=\text{s-}\lim_{n\to\infty}\paren{I-\frac{t}{n}A}^{-n}\]
\end{theorem}

\subsection{解析的半群}

\begin{definition}[analytic semigroup]
    $C_0$-半群$\{T_t\}_{t\in\R_+}\subset B(X)$が$\gamma\in(0,\pi/2)$について\textbf{$\gamma$-型の解析的半群}であるとは,次の2条件を満たすことをいう.
    \begin{enumerate}
        \item $t\mapsto T_t$が$\Om_\gamma:=\Brace{t\in\C\mid\abs{\arg t}<\gamma,t\ne0}$上の$B(H)$-関数として解析的に延長可能である.
        \item ある$\delta>0$が存在して,$\Om_\gamma^\delta:=\Brace{t\in\C\mid\abs{\arg t}\le\gamma-\delta,t\ne0}$上で$\Om_\gamma^\delta\ni\abs{t}\to0$のとき$T_t\to I$に強収束する.
    \end{enumerate}
\end{definition}

\subsection{初期値問題}

\begin{problem}
    Banach空間$X$における常微分方程式の初期値問題
    \[\dd{u}{t}=Au+f(t)\;(t>0,f\in C(\R_+;X))\qquad u(0)=a\]
    について,正則性条件$u\in C([0,\infty];X)\cap C^1((0,\infty);X)$,$Au\in C((0,\infty);X)$を満たす解$u$を考察する.
\end{problem}

\begin{proposition}
    $A$が$C_0$-半群の生成作用素であるならば,任意の$a\in\D(A)$に対して一意の解$u(t)=e^{tA}a$を持つ.
\end{proposition}

\section{発展方程式}

\subsection{スペクトル定理の特殊化}

\begin{discussion}
    Hilbert空間$X$内の自己共役作用素$H$が下に半有界とする:$\exists_{c\in\R}\;\forall_{x\in\D(H)}\;(Hx,x)\ge-c\norm{x}^2$.
    このとき,$H$のスペクトル分解は
    \[H=\int^\infty_{-c}\lambda dE(\lambda)\]
    となる.
\end{discussion}

\begin{problem}[抽象的発展方程式]
    自己共役作用素$H$に関して,
    \[\dd{u(t)}{t}=-Hu(t)\;(t>0)\qquad\text{s-}\lim_{t\searrow0}u(t)=x\]
    を抽象的発展方程式という.
\end{problem}

\begin{theorem}
    $T_t:=e^{-tH}=\int^\infty_{-c}e^{-\lambda t}dE(\lambda)\;(t>0)$とする.
    \begin{enumerate}
        \item $T_t\;(t>0)$は$X$全域で定義された有界作用素で,$\forall_{t>0}\;\norm{T_t}\le e^{ct}$を満たす.
        \item $\forall_{t,s>0}\;T_tT_s=T_{t+s}$.
        \item $\forall_{x\in X}\;\text{s-}\lim_{t\searrow0}T_tx=x$.
        \item $\forall_{x\in\D(H)}\;\text{s-}\lim_{t\searrow0}\frac{T_tx-x}{t}=-Hx$.
        \item $\forall_{x\in X,t>0}\;T_tx\in\D(H)$で,$t$に関する強微分$\dd{}{t}T_tx:=\text{s-}\lim_{\delta\to0}\frac{T_{t+\delta}x-T_tx}{\delta}$が存在して$\dd{}{t}T_tx=-HT_tx$が成立する.
    \end{enumerate}
\end{theorem}
\begin{remarks}
    これは,抽象的発展方程式の解が$-H$で生成される$C_0$-半群$e^{-tH}$によって,
    $u(t)=T_tx=e^{-tH}x$で与えられることを示している.
\end{remarks}

\subsection{Schrodinger型方程式}

\begin{tcolorbox}[colframe=ForestGreen, colback=ForestGreen!10!white,breakable,colbacktitle=ForestGreen!40!white,coltitle=black,fonttitle=\bfseries\sffamily,
title=]
    Schrodinger型の方程式については,下半有界とは限らない一般の自己共役作用素$H$について,ユニタリ作用素の1-パラメータ変換群が解を与える.
\end{tcolorbox}

\subsection{拡散方程式}

\begin{tcolorbox}[colframe=ForestGreen, colback=ForestGreen!10!white,breakable,colbacktitle=ForestGreen!40!white,coltitle=black,fonttitle=\bfseries\sffamily,
title=]
    発展方程式の原型の1つが,拡散方程式である.
\end{tcolorbox}

\begin{example}
    $\Om\subset\R^m\;(m\ge2)$を領域,$q\in C(\Om;\R)$を下に有界な連続関数とする:$\exists_{c\in\R}\;c\le q<\infty$.
    $X:=L^2(\Om)$をHilbert空間とし,偏微分作用素$C_0^2(\Om)\to L^2(\Om)$を$A:=-\Laplace+q(x)$と定めると,これは下に半有界な対称作用素だから,Friedrichs拡張$\wt{A}$は自己共役である.
    よって,初期値関数$f\in C(\Om)\cap L^2(\Om)$を与えたとき,初期値問題
    \[\pp{u}{t}=\Laplace u-q(x)u\;(t>0)\qquad u(0,x)=f(x)\]
    の解は$u(t)=T_tf=e^{-t\wt{A}}f$によって得られる.
\end{example}


\begin{definition}
    さらに,$q\in C^1(\Om)$ならば,作用素$T_t$は積分核$U(t,x,y)$による積分作用素として表現される.
    この積分核を,拡散方程式の\textbf{基本解}という.
\end{definition}

\begin{theorem}
    半群$e^{-t\wt{A}}$に対して,$(0,\infty)\times\Om\times\Om$上連続な関数$U(t,x,y)$が存在して,任意の$f\in L^2(\Om)$に関して
    \[(T_tf)(x)=\int_\Om U(t,x,y)f(y)dy.\]
    また,$U(t,x,y)$は任意の$t,s>0$と$x,y\in\Om$について,次を満たす:
    \begin{enumerate}
        \item $\int_\Om U(t,x,z)U(s,z,y)dz=U(t+s,x,y)$.
        \item $U(t,x,y)=\o{U(t,y,x)}$.
    \end{enumerate}
\end{theorem}

\begin{corollary}
    $\Om$が有界領域ならば,$T_t$はコンパクト作用素である.
\end{corollary}

\subsection{一般展開定理}

\begin{definition}
    $(\varphi_n(x;\lambda))_{n\in\N,\lambda\in\R}$が偏微分作用素$A$の\textbf{一般化された固有関数系}であるとは,
    \[\forall_{n\in\N,\lambda\in\R}\;A\varphi_n(x,\lambda)=\lambda\varphi_n(x;\lambda)\]
    が成り立つことをいう.
    一般化された固有関数とは,$\varphi_n\in L^2(\Om)$である必要はないことをいう.
\end{definition}

\begin{theorem}[一般展開定理]
    
\end{theorem}

\begin{theorem}
    $T_t=e^{-t\wt{A}}$を与える積分核$U(t,x,y)$は次のように表せる:
    \[U(t,x,y)=\sum_{n\in\N}\int_{-c}^\infty e^{-\lambda t}\varphi_n(x;\lambda)\o{\varphi_n(y;\lambda)}d\rho_n(\lambda).\]
\end{theorem}

\chapter{積分論}

\begin{quotation}
    積分とは,コンパクト台を持つ連続関数の空間上の,正な線型汎関数である.
    これをRudinではpositive linear functionalと呼び続ける.
    この見方をすれば,積分を完全に関数解析の言葉で定義でき,測度とは積分の特別な場合である.
    また積分と測度の双対性も明らかになる.

    理論の完成を知らせる定理は\textbf{表現定理}である.
    積分とは表現であり,2つのBanach空間をつなぐものである.

    \begin{tcolorbox}[colframe=blue, colback=blue!3!,breakable,colbacktitle=ForestGreen!40!white,coltitle=black,fonttitle=\bfseries\sffamily,
    title=]
        \begin{theorem*}[Riesz, Markov]
            $X$を局所コンパクトハウスドルフ空間とする.
            \begin{enumerate}
                \item 積分は正錐と拡張正錐の間に
                位相同型$(C_c(X,\C)^*)_+\simeq_\Top\o{\RM(X)}_+$を定める\ref{thm-Riesz-representation}.
                \item 積分は等長同型$C_0(X,\C)^*\simeq_\Ban\RM(X)$を定める\ref{prop-Riesz-Markov-3}.\footnote{\url{https://ncatlab.org/nlab/show/Riesz+representation+theorem}}
            \end{enumerate}
        \end{theorem*}
    \end{tcolorbox}
    いままでのが局所凸位相線型空間論だとしたら,ここからは錐の構造をよく見るために,
    Riesz空間=束線型空間を対象にした順序線型空間論のような理論展開をする.
\end{quotation}

\begin{definition*}[positive cone, extended positive cone]
    $C^*$-環の内部で,正作用素のなす閉凸錐の描像を思い出す.
    \begin{enumerate}
        \item 点付き順序集合$(V,0)$の正錐とは,$V_+:=\Brace{x\in V\mid x\ge0}$をいう.
        \item $W^*$-代数$V$のpredualを$V_*$とする:$V=\Hom(V_*,\R)$.$V$の拡張正錐$\o{V}_+$とは,正錐$(V_*)_+$から$\o{\R}_+=[0,\infty]$への下半連続な線型写像全体の空間をいう.
    \end{enumerate}
    $\R$の拡張正錐は$\oR_+$で,その一般化と考えられる.
\end{definition*}

\begin{definition*}[positive linear functional, finie Radon measure]
    複素数値関数/測度の空間に言及する言葉を整理する.
    \begin{enumerate}
        \item 正な連続線型汎関数の全体$C_c(X,\C)^*_+$は正錐をなす.
        \item 有限なRadon測度の全体$\RM(X,\C)$は全変動についてBanach空間をなす.
        その正錐は実数値の意味での有限Radon測度$\mu:\M\to\R_+$である.一般のRadon測度にも言及したいから,
        その拡張正錐$\o{\RM(X)}_+$を,正なRadon測度$\M\to\oR_+$全体の集合とする.
    \end{enumerate}
\end{definition*}

\section{表現定理のまとめ}

\begin{tcolorbox}[colframe=ForestGreen, colback=ForestGreen!10!white,breakable,colbacktitle=ForestGreen!40!white,coltitle=black,fonttitle=\bfseries\sffamily,
title=]
    Lebesgue積分論は$P(X)$の位相・代数的構造に注目して編み上げられた理論であるが,埋め込み$P(X)\mono\Map(X,\R)$の先で考えたほうが見通しが良い,というのが今回の理論である.

\end{tcolorbox}

\begin{theorem}[Borel測度の正則性についての結果]\mbox{}
    \begin{enumerate}
        \item 局所コンパクトハウスドルフ空間$X$上のRadon測度は有限ならば外部正則である.特に,正則Borel測度である.
        \item 局所コンパクトハウスドルフ空間$X$の開集合がすべて$\sigma$-コンパクトならば,局所有限なBorel測度は全て正則である.
    \end{enumerate}
    特にEuclid空間$\R^n$上のBorel測度は局所有限ならば正則で,特に確率測度ならば正則である.
\end{theorem}

\subsection{実数上のRadon積分}

\begin{lemma}
    各点完備な位相線型束上の線型汎関数は,正ならば連続である.
\end{lemma}

\begin{theorem}[実数上のRadon積分のLebesgue-Stieltjes積分としての表現\ref{thm-Stieltjes}]
    正な線型汎関数$\int\in (C_c(\R))^*_+$について,Lebesgue-Stieltjes積分が定める対応
    \[F:\Brace{m\in\Map(\R,\R)\mid m\text{は単調増加}}\to(C_c(\R))^*_+\]
    は全射で,$F$が定める同値類は$m$を下半連続に取ることで代表元とすることができ,これによって集合の同型が引き起こされる.
\end{theorem}
\begin{theorem}[実数上のRadon電荷の有界変動分布関数に関する積分としての表現\cite{盛田}定理6.5]
    \[\Psi:\RM([a,b];\R)\iso\Brace{F\in\BV([a,b])\mid F(a)=0,F\text{は右連続}}\]
    を$\Psi(\mu)(x):=\mu([a,x])$で定めると,これは全単射であり,次を満たす.
    \begin{enumerate}
        \item $\Psi$は絶対連続測度を絶対連続関数に写し,$D\Psi(\mu)(x)=\dd{\mu}{m}(x)\;\ae$の関係を保つ.
        \item 絶対値を変動に写す:$\Psi(\abs{\mu})=V_{\psi(\mu)}$が成り立つ.
    \end{enumerate}
\end{theorem}
\begin{theorem}[実数上のRadon測度は有界変動関数によって表現される]
    次の対応は等長同型$(C_0(\R))^*\simeq_\Ban\BV(\R)$を定める:
    \[\xymatrix@R-2pc{
        C_0(\R)\ar[r]&\BV(\R)\\
        \rotatebox[origin=c]{90}{$\in$}&\rotatebox[origin=c]{90}{$\in$}\\
        \Phi\ar@{|->}[r]&\Phi(f)=\int_\R f(u)d\al (u).
    }\]
\end{theorem}

\subsection{Radon積分とこれを定めるRadon測度}

\begin{theorem}[Radon-Riesz\footnote{MalliavinとAlexandorffもRiesz and Radonと呼んでいる.}]
    $X$を局所コンパクトハウスドルフ空間,
    $(C_c(X))_+^*$を正な線型汎関数のなす正錐,
    $\RM(X;[0,\infty])$をその上のRadon測度とする.次の対応は全単射である:
    \[\xymatrix@R-2pc{
        \RM(X)\ar[r]&(C_c(X))_+^*\\
        \rotatebox[origin=c]{90}{$\in$}&\rotatebox[origin=c]{90}{$\in$}\\
        \mu\ar@{|->}[r]&\mu(A)=\int_*[A]
    }\]
    また,位相同型でもある.\footnote{\url{https://ncatlab.org/nlab/show/Riesz+representation+theorem}}
\end{theorem}
\begin{history}
    Rieszの元来の定理は次のものであるが,これと,$[a,b]$上の有界変動関数とその上の正則なBorel測度との間には全単射が存在するために,この定理もRieszの名前で呼ばれる.
\end{history}
\begin{remark}
    $\RM(X;[0,\infty])$の元は,「外部正則かつ開集合上内部正則かつ局所有限なBorel測度$\mu^*$」と一対一対応し,この関係を,「$\mu$は$\mu^*$に付随する本質的測度である」という.\footnote{url{https://ncatlab.org/nlab/show/Radon+measure}}
\end{remark}

\begin{theorem}[Rudin Th'm 2.14\cite{Rudin}]
    $X$を局所コンパクトハウスドルフ空間とする.
    任意の正な線型汎関数$\Lambda\in (C_c(X))^*_+$について,ある$\sigma$-代数$\B(X)\subset\M$が存在して,
    その上の一意な測度$\mu:\M\to\o{\R_+}$は次を満たす:
    \begin{enumerate}
        \item $\Lambda f=\mu f\;\on C_c(X)$.
        \item 局所有限:$\forall_{K\compsub X}\;\mu(K)<\infty$.
        \item 外部正則:$\forall_{E\in\M}\;\mu(E)=\inf\Brace{\mu(V)\in\o{\R_+}\mid E\subset V\osub X}$.
        \item 内部正則:任意の開集合$E\osub X$または有限確定な$E\in\M,\mu(E)<\infty$について,$\mu(E)=\sup\Brace{\mu(K)\in\o{\R_+}\mid K\compsub E}$.\footnote{開でない体積が確定しない集合については,内部正則でないものが存在し得る.}
        \item 完備:$\forall_{E\in\M}\;\forall_{A\subset E}\;\mu(E)=0\Rightarrow A\in\M$.
    \end{enumerate}
    また,$X$を$\sigma$-コンパクトとすれば,$\mu$は正則に取れる.
\end{theorem}

\subsection{$C$の双対空間と符号付き測度}

\begin{theorem}
    次の対応$F$は全射である:
    \[F:\Brace{m\in\Map(\R;\R)\mid m\text{は単調増加}}\to (C(R))_+\*.\]
\end{theorem}

\begin{theorem}[絶対連続関数の双対空間のLebesgue積分による特徴付け (Riesz 1909)\ref{thm-dual-of-C-on-interval}]
    \[C([a,b])^*\simeq_\Set\Brace{\rho\in\BV([a,b])}\]
\end{theorem}


\subsection{$C_0$の双対空間とRadon測度}

\begin{lemma}
    $C_c(X)$は$C_0(X)$上一様ノルムについて稠密である.特に,$C_c(X)^*=C_0(X)^*$.
\end{lemma}

\begin{theorem}[Riesz-Markov (1938)]
    $X$を局所コンパクトハウスドルフ空間,$\RM(X;\R)$をその上の符号付きRadon測度の全変動ノルムに関するBanach空間,$(C_0(X))^*$を$C_0(X)$上の有界線型汎関数のBanach空間とする.
    このとき,次の対応は等長同型$(C_0(X))^*\simeq_\Ban\RM(X;\R)$を定める:
    \[\xymatrix@R-2pc{
        (C_0(X))^*\ar[r]&\RM(X;\R)\\
        \rotatebox[origin=c]{90}{$\in$}&\rotatebox[origin=c]{90}{$\in$}\\
        \Phi\ar@{|->}[r]&\Phi f=\mu[f]\;(f\in C_0(X))
    }\]
\end{theorem}
\begin{remark}
    このときRadon測度は有限だから,実際は正則なBorel測度にもなっている.
    なお,符号付きRadon測度の代わりに複素Radon測度としても成立する\ref{prop-Riesz-Markov-3}.
\end{remark}


\subsection{$C_b$の双対空間とBaire測度}


\begin{definition}[Baire measure]
    位相空間$X$上の
    Baire $\sigma$-代数とは,$C(X)$を可測にする最小の$\sigma$-代数をいう.このとき,
    $C_b(X)$を可測にする最小の$\sigma$-代数としても特徴付けられる.
    Baire $\sigma$-代数上の測度をBaire測度という.
\end{definition}
\begin{lemma}\mbox{}
    \begin{enumerate}
        \item 距離空間上ではBaire集合族とBorel集合族とは一致する.
        \item Tychonoff空間(距離空間はこれを満たす)上のタイトなBaire測度は,Radon測度に一意に延長する.
    \end{enumerate}
\end{lemma}

\begin{theorem}[Riesz-Markov-Kakutani (1941)]
    $X$をコンパクトハウスドルフ空間,$M_B(X)$を符号付きBaire測度が全変動ノルムについてなすBanach空間とする.
    このとき,次の対応は等長同型$(C(X))^*\simeq_\Ban M(X)$を定める:
    \[\xymatrix@R-2pc{
        (C(X))^*\ar[r]&M_B(X)\\
        \rotatebox[origin=c]{90}{$\in$}&\rotatebox[origin=c]{90}{$\in$}\\
        \Phi\ar@{|->}[r]&\Phi f=\mu[f]\;(f\in C(X))
    }\]
\end{theorem}
\begin{remarks}
    したがって,Radon測度の空間へは埋め込みを定める.
\end{remarks}


\begin{theorem}[Markov-Alexandorff (1940) (\cite{Dunford-Schwartz1}, Th'm IV.6.2)]
    $X$を$T_4$-位相空間,$M_f(X)$をその上の有限加法的な符号付きRadon測度が全変動ノルムについてなす空間とする.
    次の対応は等長同型$(C_b(X))^*\simeq_\Ban M_f(X)$を定める:
    \[\xymatrix@R-2pc{
        (C_b(X))^*\ar[r]&M_f(X)\\
        \rotatebox[origin=c]{90}{$\in$}&\rotatebox[origin=c]{90}{$\in$}\\
        \Phi\ar@{|->}[r]&\Phi(f)=\int_Xfd\mu
    }\]
    またこの同型は順序も保つ.
\end{theorem}
\begin{history}
    初めて$C_B(\R)$上の線型汎関数の空間を考察したのはFichtenholz G.とKantorovitch, L. (1934)であった.
    Alexandorff (1940,\cite{Alexandorff})でchargeと呼んでいるものは,有限加法的な符号付き測度で,有限加法的な集合関数である.
    これはMarkov (38)の仕事の拡張だという.
    この論文ではRiesz-Radonの拡張が試みられ,そこでchargeの語を生み出している.\footnote{to call it measure seemed to be inconvenient, therefore, lead by a natural physical analogy, I call it a charge.}
    そしてまさか$T_4$-空間を表す正規性は,このRiesz-Radonの研究から出てきたとは.
\end{history}
\begin{remark}
    \cite{Butzer}で$X=\R$の場合は取り上げられているが,少し差があり,本当に$X$が正規な場合にも成り立つのか疑問が残る.
    どうやら\cite{Dunford-Schwartz1} Theorem IV 6.2には載っているらしい.
    他に参考になりそうなのは,Malliavin, \href{https://www.db-thueringen.de/servlets/MCRFileNodeServlet/dbt_derivate_00023083/bachlor-stilianos_louca.pdf}{このPDF}など.
    さらに,$(C_b(\R))^*$は$\R$のStone-Cechコンパクト化上のRadon測度の空間でもあるらしい.\footnote{\url{https://mathoverflow.net/questions/83593/dual-space-of-continuous-functions}}
\end{remark}

\subsection{$L^\infty$の双対空間}

\begin{theorem}[Hildebrandt, T. H. (1934)]
    次の対応は等長同型$(l^\infty(\R))^*\simeq_\Ban M_f(\R)$を定める.
    \[\xymatrix@R-2pc{
        (l^\infty(\R))^*\ar[r]&M_f(\R)\\
        \rotatebox[origin=c]{90}{$\in$}&\rotatebox[origin=c]{90}{$\in$}\\
        \Phi\ar@{|->}[r]&\Phi(f)=\int_\R fd\mu
    }\]
\end{theorem}

\begin{theorem}[(\cite{Dunford-Schwartz1},Th'm IV.8.16)]
    $(\Om,\F,\mu)$を$\sigma$-有限な測度空間とする.このとき,
    $M_f$を$\F$上の有限加法的な符号付き測度の空間とすると,
    $(L^\infty(\Om))^*\simeq_\Ban M_f(\Om)$が成り立つ.
\end{theorem}

\section{Radon積分}

\begin{tcolorbox}[colframe=ForestGreen, colback=ForestGreen!10!white,breakable,colbacktitle=ForestGreen!40!white,coltitle=black,fonttitle=\bfseries\sffamily,
title=]
    $C_c(X)$上の単調ネットを用いて,$C_c(X)^m,C_c(X)_m$に上積分と下積分とが定義出来る.
    そしてこの上の単調ネットを用いて,一般の関数の上に上積分と下積分とが定義出来る!
    これが位相空間論の強みである.
    そして,2つの値が一致する関数のクラスこそ,Lebesgue可積分関数のクラスとなる.
\end{tcolorbox}

\begin{notation}\mbox{}
    \begin{enumerate}
        \item $X$を局所コンパクトハウスドルフ空間とし,その上のコンパクト台を持つ実数値連続関数のなすノルム代数$C_c(X)$を考える.
        \item $X$のコンパクト部分集合の全体を$\cC\subset P(X)$で表す.
        \item $X$の可測集合全体を$\M$で表し,体積確定集合の全体を$\M^1$で表す\ref{def-measurable-sets}.
        \item $X$上のBorel関数のクラスを$\B(X)$,可測関数の全体を$\L(X)$で表す.$\B\subset\M$より$\B(X)\subset\L(X)$である\ref{def-measurable-function}.$X$上の局所可積分関数の空間を$\L^1_\loc(X)$で表す\ref{def-locally-integrable-function}.
        \item $X$上の有限なRadon電荷$\Phi$全体に全変動ノルム$\norm{\Phi}:=\abs{\Phi}(1)$を考えたBanach空間を$M(X)$で表す.
        \item $X$上の零関数を$\cN(X)$,零集合を$\cN$で表す\ref{def-null-set}.
        \item $C_c(X)$の単調増加ネットの極限として得られる関数全体のなす正錐を
        \[C_c(X)^m:=\Brace{f:X\to\R\cup\{\infty\}\mid C_c(X)\text{上の単調増加ネット}(f_\lambda)_{\lambda\in\Lambda}\text{が存在して}\forall_{x\in X}\;f(x)=\sup f_\lambda(x)\text{を満たす}}\]
        とする.
        また,$C_c(X)_m:=\Brace{f:X\to\R\cup\{-\infty\}\mid\exists_{\{f_\lambda\}\subset C_c(X)}\;f_\lambda\searrow f}$とすると,$C_c(X)_m=-C_c(X)^m$であり,$C_c(X)^m\cap C_c(X)_m=C_c(X)$である.
        \item 関数クラス$M(X)\subset\Map(X,\R)$に対して$M(X)_+$とは,非負関数のなす部分空間$M(X)\cap\Map(X,\R_{\ge 0})$を表す.
        \item 特性関数を$[A]=\chi_A$で表す.
    \end{enumerate}
\end{notation}

\subsection{線型束}

\begin{tcolorbox}[colframe=ForestGreen, colback=ForestGreen!10!white,breakable,colbacktitle=ForestGreen!40!white,coltitle=black,fonttitle=\bfseries\sffamily,
title=]
    まず,議論の対象となる「性質の良い関数と測度のクラス」を定義する.
    積分を定義する関数クラスは,束の構造を持つもの$C_c(X),C_c(X)^m,C_c(X)_m$に注目する.
    $X$上の測度はRadon測度(緊密かつ局所有限なもの)のみを考える.
    Radon測度は,$C_c(X)$上に正な線型汎関数を定める.
    これをRadon積分と呼び,逆にRadon測度を復元すること\ref{prop-recovery-of-Radon-measure}を考える.
    実は,$C_c(X)$が可積分関数$\L^1(\R^n)$の構造を支配する(稠密である\ref{prop-dense-subset-of-Lp}).
\end{tcolorbox}

\begin{lemma}[Riesz空間 / 束線型空間,関数束]\mbox{}
    \begin{enumerate}
        \item $C_c(X)$は束線型空間である:$C_c(X)$は関数の2項演算$\lor,\land$について閉じていて,これについて束をなす.
        \item $C_c(X)^m$も同様で,さらに任意族の上限$\sup_{\lambda}f_\lambda$について閉じている.
        \item $C_c(X)^m$は正錐であり,また,任意の2つの正な元の積について閉じている:$C_c(X)^m_+\cdot C_c(X)^m_+\subset C_c(X)^m$.
        \item $C_c(X)^m$の元は全て下半連続で,非負な下半連続関数の全体を含む:$C^{1/2}(X)_+\subset C_c(X)^m$.
    \end{enumerate}
\end{lemma}

\begin{lemma}[$C_c(X)^m$による底空間$X$の性質の特徴付け]\mbox{}
    \begin{enumerate}
        \item \item $X$がコンパクトであることと,$-1\in C_c(X)^m$は同値.
        \item $X$が$\sigma$-コンパクトであることと,$1\in C_c(X)^m$は同値.
        \item 任意の$f\in C_c(X)^m$について,$\o{\Brace{f<0}}$はコンパクト.任意の$f\in C_c(X)_m$について,$\o{\Brace{f>0}}$はコンパクト.
        \item $X$が特に距離空間であるとき,任意のコンパクト集合$K\subset X$について,$\chi_K\in C_c(X)_m$.
    \end{enumerate}
\end{lemma}

\subsection{単調極限上へのRadon積分の延長}

\begin{tcolorbox}[colframe=ForestGreen, colback=ForestGreen!10!white,breakable,colbacktitle=ForestGreen!40!white,coltitle=black,fonttitle=\bfseries\sffamily,
title=]
    Darbouxの発想は,積分が定まっている関数$f$を用いて,そのネットの極限として得られる関数への積分の値の拡張である.
    こうして
    $C_c(X)$上の正な線型汎関数$\int$を延長してみる.
\end{tcolorbox}

\begin{definition}[Radon integral, monotone limits]\mbox{}
    \begin{enumerate}
        \item \textbf{Radon測度}または\textbf{Radon積分}とは,線型汎函数$\int:C_c(X)\to\R$であって,\textbf{正}であるものをいう:$f\ge 0\Rightarrow\int f\ge 0$.
        \item 上積分$\int^*:C_c(X)^m\to\R\cup\{\infty\}$を,$\int^*f:=\sup\Brace{\int g\in\R\cup\{\infty\}\;\middle|\;g\in C_c(X),g\le f}$と定めると,これは線型汎函数である.
        \item 下積分$\int_*:C_c(X)_m\to\R\cup\{-\infty\}$を,$\int_*f:=\inf\Brace{\int g\in\R\cup\{-\infty\}\;\middle|\;g\in C_c(X),g\ge f}$と定めると,これは線型汎函数である.
    \end{enumerate}
\end{definition}
\begin{remarks}
    局所コンパクトハウスドルフ空間$X$を第2可算とすれば,$(f_n)$は単調列とすれば十分.
    この$C_c(X)$の単調極限なるクラスが大事である理由は,$X=\R$であるとき,区間の定義関数が入るためである:$\chi_{(a,b)}\in C_c(X)^m,\chi_{[a,b]}\in C_c(X)_m$.
    $C_c(X)^m,C_c(X)_m$は錐であったから,上積分と下積分は加法と$\R_{\ge 0}$倍に限っては明らかに「線型」であるが,実際に線型であることは非自明である.
\end{remarks}

\begin{lemma}[well-definedness]\mbox{}
    \begin{enumerate}
        \item $\forall_{f\in C_c(X)_m}\;\int_*f=-\int^*(-f)$.
        \item $\forall_{f\in C_c(X)}\;\int_*f=\int^*f=\int f$.
    \end{enumerate}
\end{lemma}

\begin{lemma}[下積分と上積分の大小関係]\mbox{}
    \begin{enumerate}
        \item コンパクト台を持つ上半連続な正関数の単調減少ネット$\{f_\lambda\}\subset (C_c(X)_{1/2})_+$について,$\forall_{x\in X}\;f_\lambda(x)\searrow 0$ならば,$\norm{f_\lambda}_\infty\searrow0$.
        \item $C_c(X)_m$の単調減少ネット$(f_\lambda)$について,$f_\lambda\searrow0$ならば$\int_*f_\lambda\searrow0$である.
        \item $C_c(X)$の単調増加ネット$(f_\lambda)$について,$\exists_{f\in C_c(X)^m}\;f_\lambda\nearrow f$ならば$\int f_\lambda\nearrow\int^*f$である.
        \item $f,g\in C_c(X)^m$について,$\forall_{t>0}\;\int^*(tf+g)=t\int^*f+\int^*g$.
        \item $C_c(X)^m$の単調増加ネット$(f_\lambda)$について,ある関数$f$について$f_\lambda\nearrow f$ならば$f\in C_c(X)^m$で,$\int^*f_\lambda\nearrow\int^*f$である.
        \item $f\in C_c(X)^m,g\in C_c(X)_m$に対して,$g\le f$ならば,$\int_*g\le\int^*f$である.
    \end{enumerate}
\end{lemma}
\begin{Proof}\mbox{}
    \begin{enumerate}
        \item $f_\lambda$は上半連続だから,$\forall_{\ep>0}\;\{f_\lambda\ge\ep\}$は閉.仮定より台$\o{\Brace{f_\lambda>0}}$はコンパクトだから,$\{f_\lambda\ge\ep\}$はコンパクトでもある.$(f_\lambda)$は$0$に各点収束するから,$\cap_{\lambda\in\Lambda}\{f_\lambda\ge\ep\}=\emptyset$.
        コンパクト性の特徴付け\ref{thm-characterization-of-compactness}より,ある有限個を選び出せばやはり共通部分は空である.
    \end{enumerate}
\end{Proof}

\subsection{Lebesgue可積分性の恢復}

\begin{tcolorbox}[colframe=ForestGreen, colback=ForestGreen!10!white,breakable,colbacktitle=ForestGreen!40!white,coltitle=black,fonttitle=\bfseries\sffamily,
title=]
    同じことをもう一段階繰り返すことで,一般の関数について$C_c(X)^m,C_c(X)_m$を用いて上積分と下積分を定義できる.
    したがって,これらが一致するかどうかによってLebesgue可積分性を復元できる!!
\end{tcolorbox}

\begin{definition}
    任意の実数値関数$f\in\Map(X,\R)$について,
    \begin{enumerate}
        \item 上積分を$\int^{**}f:=\inf\Brace{\int^*g\in\R\cup\{\infty\}\mid g\in C_c(X)^m,g\ge f}$と定める.
        \item 下積分を$\int_{**}f:=\sup\Brace{\int_*g\in\R\cup\{-\infty\}\mid g\in C_c(X)_m,g\le f}$と定める.
        \item 関数$f\in\Map(X,\R)$が\textbf{可積分}であるとは$\int^*f=\int_*f\in\R$が成り立つことをいい,その全体を$\L^1(X)$で表す.このとき,$\int f:=\int^*f=\int_*f$と定める.
    \end{enumerate}
    以降,$\int^*,\int_*$と略記する.
\end{definition}

\begin{lemma}
    実数値関数$f\in\Map(X,\R)$について,次の2条件は同値:
    \begin{enumerate}
        \item $f\in \L^1(X)$.
        \item $\forall_{\ep>0}\;\exists_{g\in C_c(X)^m}\;\exists_{f\in C_c(X)_m}\;h\le f\le g\land\int^*g-\int_*h<\ep$.
    \end{enumerate}
    特に,$f\in C_c(X)^m$については,(2)は$\int^*f<\infty$に同値.
\end{lemma}

\begin{theorem}[Daniell's extension theorem]\label{thm-extension-of-Radon-integral}
    $\int:C_c(X)\to\R$を局所コンパクトハウスドルフ空間$X$上のRadon積分とする.
    \begin{enumerate}
        \item 可積分関数の空間$\L^1(X)$は$C_c(X)$を部分空間にもつ実線型空間である.
        \item $\L^1$は束演算$\land,\lor$について閉じている.
        \item $\int:\L^1(X)\to\R$は正な線型汎函数で,$C_c(X)$上のRadon積分の延長となっている.
    \end{enumerate}
\end{theorem}

\begin{corollary}[三角不等式]
    $f\in\L^1(X)$ならば,$\abs{f}\in\L^1(X)$で,$\Abs{\int f}\le\int\abs{f}$.
\end{corollary}
\begin{Proof}
    束演算で閉じていることより,$\abs{f}=f\lor 0+f\land 0\in\L^1(X)$を満たす.
\end{Proof}

\subsection{Lebesgueの優収束定理}

\begin{tcolorbox}[colframe=ForestGreen, colback=ForestGreen!10!white,breakable,colbacktitle=ForestGreen!40!white,coltitle=black,fonttitle=\bfseries\sffamily,
title=]
    今回の構成により明らかに$\L^1$は積分の定義域として極大であるため,収束定理が自由に成り立つ.
\end{tcolorbox}

\begin{theorem}[monotone convergence theorem]
    関数$f:X\to\R$はある$\L^1(X)$の単調増加列$(f_n)$の各点収束極限であり,$\sup_{n\in\N}\int f_n<\infty$を満たすとする.
    このとき,$f\in\L^1(X)$で,$\int f=\lim_{n\to\infty}\int f_n$である.
\end{theorem}
\begin{Proof}
    差分$g_n:=f_{n+1}-f_n\in\L^1(X)_+$の列を考える.
\end{Proof}

\begin{lemma}[Fatou's lemma]
    $\L^1(X)_+$の列$(f_n)$は$\forall_{x\in X}\;\liminf_{n\to\infty}f_n(x)<\infty$かつ$\liminf_{n\to\infty}\int f_n<\infty$を満たすとする.
    このとき,$\liminf_{n\to\infty}f_n\in\L^1(X)$で,$\int\liminf_{n\to\infty}f_n\le\liminf_{n\to\infty}\int f_n$.
\end{lemma}

\begin{theorem}[Lebesgue convergence theorem]
    $\L^1(X)$の列$(f_n)$がある$g\in\L^1(X)_+$に関して$\forall_{n\in\N}\;\abs{f_n}\le g$を満たしながら
    $f:X\to\R$に各点収束するとする.このとき,$f\in\L^1(X)$で,$\int f=\lim_{n\to\infty}\int f_n$である.
\end{theorem}

\subsection{Stieltjes積分}

\begin{tcolorbox}[colframe=ForestGreen, colback=ForestGreen!10!white,breakable,colbacktitle=ForestGreen!40!white,coltitle=black,fonttitle=\bfseries\sffamily,
title=]
    $\R$上のRadon積分の古典的構成法を議論する.アイデアとしては,ある有界変動関数$\rho:[a,b]\to\R$を定めれば,これを累積分布関数として分布が定まり,それについての積分が定まる.そして,$C[a,b]$の双対空間の元は,このような積分で尽きる.
    これをStieltjes積分という.\footnote{Riemannの方法を一般化したStieltjes積分を,Lebesgue積分の方法で一般化させたLebesgue-Stieltjes積分のことを,歴史的には主な貢献者の名前をとってRadon積分と呼んだ.}
\end{tcolorbox}

\begin{definition}[Stieltjes integral]
    $m:\R\to\R$を単調増加な関数とすると,$m$の不連続点は高々可算個である.
    このとき,$m(x):=\sup\Brace{n(y)\in\R\mid y<x}$と定め直すことで,$m$は左半連続であると仮定しても一般性を失わない.

    任意の$f\in C_c(X)$について,区間$\supp f\subset[a,b]$の分割$\lambda=(x_i)_{i\in n+1}$をとり,
    \begin{align*}
        (Sf)_k&:=\sup\Brace{f(x)\in\R\mid x_{k-1}\le x<x_k},&(If)_k&:=\inf\Brace{f(x)\in\R\mid x_{k-1}\le x<x_k},
    \end{align*}
    と定め,これを用いて
    \begin{align*}
        {\sum_{\lambda}}^*f&=\sum_{k=0}^n(Sf)_k(m(x_k)-m(x_{k-1})),&{\sum_{\lambda}}_*f&=\sum_{k=0}^n(If)_k(m(x_k)-m(x_{k-1})),
    \end{align*}
    と定める.$[a,b]$の有限な分割全体のなす有向集合$\Lambda$について,2つのネット$\paren{{\sum_\lambda}^* f}_{\lambda\in\Lambda},\paren{{\sum_\lambda}_*f}_{\lambda\in\Lambda}$
    が定まる.それぞれ単調増加,単調減少で,$\forall_{\lambda\in\Lambda}\;{\sum_\lambda}_*f\le{\sum_\lambda}^*f$を満たす.
    $f$は特に一様連続であるから,$\forall_{\ep>0}\;\exists_{\delta>0}\;x_k-x_{k-1}<\delta\Rightarrow(Sf)_k-(If)_k<\ep$.
    よって,$\lambda_0$を長さ$(b-a)\delta^{-1}<n$の等分割とすると,$\forall_{\lambda\ge\lambda_0}\;{\sum_\lambda}^*f-{\sum_\lambda}_*f<\ep(m(b)-m(a))$.
    よって,2つのネットは同一の実数に収束する.
    こうして定まる積分$\int:C_c(\R)\to\R$を\textbf{Stieltjes積分}といい,$\int fdm$と表す.
\end{definition}
\begin{remarks}
    Riemann-Stieltjes積分はこの$\int:C_c(\R)\to\R$の延長であり,Lebesgue-Stieltjes積分はその$\L^1(\R)$への更なる延長である(多分).
    しかし,Riemann可積分関数のクラスは,測度の言葉で簡明な特徴付けはあるが,単調列の極限に関して安定でない.また,Riemann-Stieltjes積分は,$\R$の全順序性により過ぎているため,高次元に一般化できない.
\end{remarks}

\begin{lemma}
    Stieltjes積分$\int:C_c(\R)\to\R$はRadon積分,すなわち,正な線型汎関数である.
\end{lemma}

\begin{theorem}\label{thm-Stieltjes}
    $\R$上のRadon積分$\int:C_c(\R)\to\R$について,ある有界変動な右連続な単調増加関数$A:\R\to\R$が存在して,これについてのStieltjes積分と一致する:$A_t=\mu([0,t])$.
\end{theorem}
\begin{Proof}
    任意にRadon積分を取ると,その定義域は$\L^1(\R)$に延長する.
    $\L^1(\R)$は区間の定義関数を含み,これを用いて単調増加関数$m$が復元できる.
\end{Proof}

\section{Daniell積分とBaire測度}

\subsection{有界変動関数}

\begin{tcolorbox}[colframe=ForestGreen, colback=ForestGreen!10!white,breakable,colbacktitle=ForestGreen!40!white,coltitle=black,fonttitle=\bfseries\sffamily,
title=]
    Lebesgue積分において,微積分の基本定理はより一般的な形で成り立つ.

    $C[a,b]$の双対空間とは,有界変動関数の定めるStieltjes積分全体の空間である.
\end{tcolorbox}

\subsubsection{測度の微分}

\begin{tcolorbox}[colframe=ForestGreen, colback=ForestGreen!10!white,breakable,colbacktitle=ForestGreen!40!white,coltitle=black,fonttitle=\bfseries\sffamily,
title=]
    対称微分は微分の一般化となっている.この概念が,$\R^k$上のRadon-Nikodym微分に一致する.
    これを調べるのに,Hardy-Littlewood作用素という有界な非(劣)線形作用素を用いる.
\end{tcolorbox}

\begin{theorem}[測度の分布関数の微分可能性]
    $\mu$を$\R$上の複素Borel測度とし,関数$f:\R\to\C$を$f(x):=\mu((-\infty,x))$と定める.
    このとき,$x\in\R,A\in\C$とLebesgue測度$m$について,次の2条件は同値.
    \begin{enumerate}
        \item $f$は$x$にて微分可能で,微分係数は$A$である:$f'(x)=A$.
        \item $\forall_{\ep>0}\;\exists_{\delta>0}\;\forall_{x\in(a,b)=:I}\;b-a<\delta\Rightarrow\Abs{\frac{\mu(I)}{m(I)}-A}<\ep$.
    \end{enumerate}
\end{theorem}

\begin{definition}[symmetric derivative, maximal function]
    $\mu$を$\R^k\;(k\in\N)$上の複素Borel測度とする.$B(x,r)$を開球として,\textbf{対称差分商}を
    $Q_r\mu(x):=\frac{\mu(B(x,r))}{m(B(x,r))}$と表す.
    \begin{enumerate}
        \item $(D\mu)(x):=\lim_{r\to0}(Q_r\mu)(x)$によって定まる関数$D\mu:\R^k\to\C$を\textbf{対称微分}という.
        \item 有限な正測度$\mu\ge0$について,$M\mu(x):=\sup_{0<r<\infty}(Q_r\mu)(x)$によって定まる関数$M\mu:\R^k\to[0,\infty]$を\textbf{極大関数}という.各点$x\in\R^k$に対して,その周りでの平均値の取り得る上限を表す.
        \item 極大関数は下半連続であり,特に可測である.
    \end{enumerate}
\end{definition}
\begin{remark}
    微分可能ならば対称微分可能であるが,その逆は成り立たない.左右の微分係数が存在するなら対称微分可能で,対称微分係数はそれらの相加平均となる.
\end{remark}

\begin{theorem}[maximal theorem]\mbox{}
    \begin{enumerate}
        \item 任意の複素Borel測度$\mu:\B^k(\R^k)\to\C$について,
        \[\forall_{\lambda>0}\quad m\Brace{M\mu>\lambda}\le \frac{3^k}{\lambda}\norm{\mu}.\]ただし,$\norm{\mu}:=\abs{\mu}(\R^k)$を全変動ノルムとした.
        \item Hardy-Littlewood作用素$M:L^1(\R^k)\to L^{1,m}(\R^k)$は,劣線型作用素として有界である:
        \[\forall_{f\in L^1(\R^k)}\;\forall_{\lambda>0}\quad m\Brace{Mf>\lambda}\le\frac{3^k}{\lambda}\norm{f}_1.\]
    \end{enumerate}
\end{theorem}

\subsubsection{測度の定める分布関数の連続性}

\begin{tcolorbox}[colframe=ForestGreen, colback=ForestGreen!10!white,breakable,colbacktitle=ForestGreen!40!white,coltitle=black,fonttitle=\bfseries\sffamily,
title=]
    Lebesgue点は連続点の一般化になる.
    Riemann可積分関数は殆ど至る所連続であるように,
    Lebesgue可積分関数は殆ど至る所Lebesgue点である.
\end{tcolorbox}

\begin{definition}[weak $L^1$, maximal function, Hardy-Littlewood operator]\mbox{}
    \begin{enumerate}
        \item 可測関数$f$が定める\textbf{分布関数}とは,$\lambda_f:\R_+\to\R_+;t\mapsto m\Brace{\abs{f}>t}$をいう.
        \item 可測関数$f$が\textbf{弱$L^1$}であるとは,その分布関数が次の意味で有界であることをいう:$\lambda\mapsto\lambda\cdot m\Brace{\abs{f}>\lambda}:(0,\infty)\to\R_+$が有界,すなわち$\exists_{C\in\R}\;\lambda_f(t)\le\frac{C}{t}$であることをいう.$L^1$関数は弱$L^1$である:$L^1\subset L^{1,m}$.
        \item 関数$f\in L^1(\R^k)$の\textbf{極大関数}とは,$(Mf)(x):=\sup_{0<r<\infty}\frac{1}{m(B_r)}\int_{B(x,r)}\abs{f}dm$によって定まる関数$Mf:\R^k\to[0,\infty]$をいう.
        \item 作用素$M:L_\loc^1(\R^d;\C)\to L^1_\loc(\R^d;\R)$をHardy-Littlewood作用素という,非線形作用素になる.
    \end{enumerate}
\end{definition}

\begin{definition}[Lebesgue point]
    可積分関数$f\in L^1(\R^k)$に対して,$x\in\R^k$が\textbf{Lebesgue点}であるとは,
    \[\lim_{r\to0}\frac{1}{m(B(x,r))}\int_{B(x,r)}\abs{f(y)-f(x)}dm(y)=0\]
    を満たすことをいう.$f$の連続点はLebesgue点である.
\end{definition}

\begin{theorem}[Lebesgue性は連続性のように扱える]
    可積分関数$f\in L^1(\R^k)$について,殆ど至る所の$x\in\R^k$は$f$-Lebesgue点である.
\end{theorem}

\subsubsection{絶対連続測度の微分}

\begin{tcolorbox}[colframe=ForestGreen, colback=ForestGreen!10!white,breakable,colbacktitle=ForestGreen!40!white,coltitle=black,fonttitle=\bfseries\sffamily,
title=]
    絶対連続測度の対称微分は,そのRadon-Nikodym微分に一致する.
\end{tcolorbox}

\begin{theorem}[Radon-Nikodym微分と対称微分の同値性]
    $\mu:\B(\R^k)\to\C\ll m$をLebesgue測度に対して絶対連続な複素Borel測度とする.
    このとき,
    \begin{enumerate}
        \item $D\mu =\dd{\mu}{m}=:f\;\;m\dae$.
        \item $\forall_{E\in\B(\R^k)}\;\mu(E)=\int_E(D\mu)dm$.
    \end{enumerate}
\end{theorem}

\begin{definition}
    $f:I=[a,b]\to\C$が\textbf{絶対連続}であるとは,$\forall_{\ep>0}\;\exists_{\delta>0}\;\forall_{\sqcup_{i\in[n]}(\al_i,\beta_i)\subset I}\;\sum^n_{i=1}\abs{f(\beta_i)-f(\al_i)}<\ep$が成り立つことをいう.
    $n=1$の場合が,$f$の連続性に同値になる.
\end{definition}

\begin{theorem}[絶対連続性の特徴付け]
    $f:I\to\R$を連続かつ広義単調増加とする.次の3条件は同値.
    \begin{enumerate}
        \item $f$は絶対連続.
        \item $f$は零集合を零集合に写す.
        \item $f$は$I$の殆ど至る所微分可能で,導関数は可積分$f'\in L^1$で,$\forall_{x\in[a,b]}\;f(x)-f(a)=\int^x_af'(t)dt$が成り立つ.
    \end{enumerate}
\end{theorem}

\begin{theorem}[絶対連続関数の全変動]
    $f:I\to\R$を絶対連続とする.
    全変動$F$に対して
    $F,F+f,F-f$は再び絶対連続で,広義単調増加である.
\end{theorem}

\begin{theorem}[微積分の基本定理]\mbox{}
    \begin{enumerate}
        \item $f:I\to\C$を絶対連続とする.このとき,$f$は$I$の殆ど至る所(Lebesgue点において)微分可能で,導関数は可積分になり$f'\in L^1(m)$,
        \[\forall_{x\in I}\quad f(x)-f(a)=\int^x_af'(t)dt\]
        \item $f:I\to\C$を殆ど至る所微分可能で,$f'\in L^1$とする.このとき,$\forall_{x\in I}\;f(x)-f(a)=\int^x_af'(t)dt$が成り立つ.
    \end{enumerate}
\end{theorem}

\subsubsection{有界変動関数}

\begin{definition}[total variation, bounded variation]
    $f:I\to\C$を任意の関数とする.
    \begin{enumerate}
        \item $F(x):=\sup_{N\in\N,(t_i)\in\Delta([a,x])}\sum^N_{i=1}\abs{f(t_i)-f(t_{i-1})}$を\textbf{全変動}という.各区間$[a,x]$上での変動をいう.
        \item 関数$f$が\textbf{有界変動}であるとは,$f$の全変動$F$が有限であることをいう:$F(b)<\infty$.その全体を$\BV[a,b]$で表す.
    \end{enumerate}
\end{definition}

\begin{proposition}[有界変動関数の性質]\mbox{}
    \begin{enumerate}
        \item $[a,b]$上の単調関数や$C^1$-級関数は有界変動である.
        \item $f\in\BV([a,b];\R)$は2つの有界な単調関数$f_1,f_2$を用いて$f=f_1-f_2$と表せる.
        \item $f\in\BV([a,b];\R)$が左連続ならば,$f_1,f_2$も左連続に選べる.
        \item $f\in\BV([a,b];\R)$が左連続ならば,あるBorel測度$\mu:\B([a,b])\to\R_+$が存在して,$f(x)-f(a)=\mu([a,x))$と表せる.このとき,$\mu\ll m\Leftrightarrow f$は絶対連続.
        \item $f\in\BV([a,b])$は$m\dae$微分可能で,$f'\in L^1(m)$.定理より$f(x)=g(x)+\int_a^xf'(t)dt$と表わせ,$g=0$であることと$f$が絶対連続であることとは同値.
    \end{enumerate}
\end{proposition}

\begin{theorem}[有界閉区間上の連続関数の空間の双対空間]\label{thm-dual-of-C-on-interval}
    $a<b\in\R$と$F:[a,b]\to\R$について,次の2条件は同値.
    \begin{enumerate}
        \item $F\in C[a,b]^*$.
        \item 有界変動関数$\rho:[a,b]\to\R$が存在して,$\forall_{u\in C[a,b]}\;F(u)=\int^b_au(x)d\rho(x)$が成り立つ.
    \end{enumerate}
    さらにこのとき,$\norm{F}=V(\rho)$となる.
\end{theorem}
\begin{Proof}
    Hahn-Banachの定理による.
\end{Proof}

\subsection{Daniell積分}

\begin{tcolorbox}[colframe=ForestGreen, colback=ForestGreen!10!white,breakable,colbacktitle=ForestGreen!40!white,coltitle=black,fonttitle=\bfseries\sffamily,
title=]
    Radon積分の延長を調べる前に,Daniell積分の消息を述べる.
    局所コンパクト空間上のRadon測度のPercy Daniell-style\footnote{Daniell, P. J. (1918), “A General Form of Integral”, Annals of Mathematics, Second Series (Annals of Mathematics) 19 (4): 279–294}の定義\ref{thm-extension-of-Radon-integral}はBourbakiのIntegration. Chapter IXでも探求されている.\footnote{\url{http://nlab-pages.s3.us-east-2.amazonaws.com/nlab/show/Radon+measure}}
    積分の公理化の試みのうち,これをDaniell積分という\footnote{\url{https://ja.wikipedia.org/wiki/ダニエル積分}}.
\end{tcolorbox}

\begin{definition}[monotone / positive functional]
    $L\subset\Map(X,\R)$を関数束とする.汎関数$T:L\to\R$が次の3条件を満たすとき,\textbf{単調}である,または\textbf{正}であるという.
    \begin{enumerate}
        \item 正錐の射:$T(f+g)=T(f)+T(g),\forall_{c\in\R_+}\;T(cf)=cT(f)$.
        \item 単調性:$\forall_{f,g\in L}\;f\ge g\Rightarrow T(f)\ge T(g)$.
        \item 単調収束性:任意の広義単調増加列$\{f_i\}\subset L$について,$\lim_{i\to\infty}f_i=:f\in L\Rightarrow T(f)=\lim_{i\to\infty}T(f_i)$.
    \end{enumerate}
    これらは積分の公理として非常に自然である.
\end{definition}

\begin{theorem}
    $L\subset\Map(X,\R)$を関数束,$T:L\to\R$を正な汎関数とする.
    このとき,$X$上の外測度$\mu$が存在して,
    \begin{enumerate}
        \item 任意の$f\in L$について$f$は$\mu$-可測で,$T(f)=\int_Xfd\mu$.
        \item 集合$A\subset X$は$\exists_{f\in L}\;\chi_A\le f$を満たすならば,$\mu(A)=\int_X\chi_Ad\mu=T(\chi_A)$の値は$T$によって一意に定まる.
    \end{enumerate}
\end{theorem}

\begin{definition}[Daniell integral]
    $L\subset\Map(X,\R)$を関数束とする.汎関数$T:L\to\R$が次の3条件を満たすとき,\textbf{Daniell積分}であるという.
    \begin{enumerate}
        \item 正錐の射:$T(f+g)=T(f)+T(g),\forall_{c\in\R_+}\;T(cf)=cT(f)$.
        \item $\forall_{f\in L_+}\;\sup_{0\le k\le f}T(k)<\infty$.
        \item $\{f_i\}\subset L$が広義単調増加ならば,$T(\lim_{i\to\infty}f)=\lim_{i\to\infty}T(f_i)$.
    \end{enumerate}
\end{definition}
\begin{remarks}
    Jordan分解の精神に従い,正な汎関数の差として表せる汎関数のクラスを考える.
    これにより,正汎関数の$\R_+$-線形性の非対称性が解消され,するとこれは積分になる.
\end{remarks}

\begin{theorem}
    $L\subset\Map(X,\R)$を関数束,汎関数$T:L\to\R$をDaniell積分とする.
    このとき,
    \begin{enumerate}
        \item $L_+$上に正な汎関数$T^+,T_-$が存在して,$\forall_{f\in L_+}\;T(f)=T^+(f)-T^-(f)$を満たす.
        \item $X$上に外測度$\mu^+,\mu^-$が存在して,$f\in L$が$\mu^+$-可測かつ$\mu^-$-可測ならば,$T(f)=\int fd\mu^+-\int fd\mu^-$.
    \end{enumerate}
\end{theorem}

\subsection{Baire測度}



\subsection{Rieszの表現定理}

\begin{tcolorbox}[colframe=ForestGreen, colback=ForestGreen!10!white,breakable,colbacktitle=ForestGreen!40!white,coltitle=black,fonttitle=\bfseries\sffamily,
title=]
    さきに,測度の定める積分と,$C_c(X)$上の正な線型汎関数の空間との間の同型対応に関する結果をまとめておく.
    次の章からは,積分の公理化ではなく,Radon測度から議論を始める.
\end{tcolorbox}

\begin{definition}[Baire sets]
    集合族$\{f>t\}_{f\in C_c(X),t\in\R}$が生成する$\sigma$-加法族$\A$を\textbf{Baire集合族}という.
    $\{f>t\}$は必ず開集合であるから,$\A\subset\B$である.なお,部分集合であるだけでなく,$\delta$-環をなす.
    また$\A$はコンパクトな$G_\delta$集合によっても生成される\ref{remark-Baire-function}.
    以降,$\A$がすべて$\mu$-可測であるとき,外測度$\mu$をBaire外測度と呼ぶ.
\end{definition}

\begin{lemma}
    $X$を局所コンパクトハウスドルフ空間とする.
    \begin{enumerate}
        \item (Urysohn) $K\subset U\osub X$をコンパクト集合とする.$\exists_{f\in C_c(X)}\;\chi_K\le f\le\chi_U$.
        \item $K$をコンパクト集合とし,これは開集合$V_1,\cdots,V_n$によって被覆されるとする:$K\subset V_1\cup\cdots\cup V_n$.このとき,各$V_i$内に台を持つ関数$0\le g_i\le1$が存在して,$\forall_{x\in K}\;g_1(x)+\cdots+g_n(x)=1$.
    \end{enumerate}
\end{lemma}


\begin{theorem}[Riesz]\label{thm-Riesz}
    局所コンパクトハウスドルフ空間上の関数束$C_c(X)$と,その上の汎関数$T:C_c(X)\to\R$であって$\forall_{f\in C_c(X)_+}\sup\Brace{T(g)\in\R\mid0\le g\le f}<\infty$を満たすものについて,
    2つのBaire外測度$\mu^+,\mu^-$が存在して,
    \[\forall_{f\in C_c(X)}\quad T(f)=\int_Xfd\mu^+-\int_Xfd\mu^-.\]
\end{theorem}
\begin{remark}
    符号付きBaire外測度$\mu:=\mu^+-\mu^-$と定めれば,
    $T(f)=\int_Xfd\mu$と表示出来るが,これは$\infty-\infty$の場合をのがしているのでステートメントには採用できない.
\end{remark}

\begin{corollary}
    特に$X$がコンパクトならば,ステートメントは次のようになる.
    有界線型汎関数$T:C(X)\to\R$に対して,符号付きBaire外測度$\mu$が一意に存在して,
    \[T(f)=\int_Xfd\mu.\]
    また,この対応は等長線型同型を引き起こす.
\end{corollary}

\begin{theorem}
    Rieszの表現定理\ref{thm-Riesz}の下で,ある符号付きBorel外測度$\o{\mu}$が存在して,
    \[\forall_{f\in C_c(X)}\quad\int fd\o{\mu}=T(f)=\int fd\mu\]
    が成り立つ.
    またとくに$\o{\mu}$はRadon外測度でもあり,Radon測度であることも示せる.
\end{theorem}

\section{可測性}

\begin{tcolorbox}[colframe=ForestGreen, colback=ForestGreen!10!white,breakable,colbacktitle=ForestGreen!40!white,coltitle=black,fonttitle=\bfseries\sffamily,
title=]
    局所的に「可積分関数によって体積が定まる集合」となる集合を可測集合とし,Borel集合族を可測集合に引き戻す関数を可測集合として定義すると,これは点列完備な単位的代数で可算束演算について閉じているBoole代数である(この構造が集合に落とした影が$\sigma$-代数?).
    こうして可測関数の概念を恢復し,可積分性が特徴付けられる.
\end{tcolorbox}

\subsection{関数クラスの点列完備性}

\begin{tcolorbox}[colframe=ForestGreen, colback=ForestGreen!10!white,breakable,colbacktitle=ForestGreen!40!white,coltitle=black,fonttitle=\bfseries\sffamily,
title=]
    これは集合族の単調族定理を,関数空間の言葉でやりなおしているだけか??
\end{tcolorbox}

\begin{definition}[(monotone) sequentially complete]
    クラス$\F\subset\Map(X,\R)$が\textbf{(単調)点列完備}とは,$\F$の(単調)列の各点収束極限が$\F$に属していることをいう.
\end{definition}

\begin{lemma}[単調族定理のような結果]\label{lemma-sequential-completion}
    代数$\A\subset\Map(X,\R)$を$\land,\lor$について閉じているBoole代数とする.
    このとき,単調列完備化$\B(\A)$は再び$\land,\lor$について閉じているBoole代数であって,点列完備である.
\end{lemma}

\subsection{$\sigma$-代数の定める関数の代数}

\begin{tcolorbox}[colframe=ForestGreen, colback=ForestGreen!10!white,breakable,colbacktitle=ForestGreen!40!white,coltitle=black,fonttitle=\bfseries\sffamily,
title=]
    $\sigma$-代数$\M\subset P(X)$は,$\forall_{t\in\R}\;\Brace{f>t}\in\M$なる条件によって,$\Map(X,\R)$上に関数の代数であって点列完備なものを定める.
    これが可測関数の代数的に良い振る舞いの根源となっている.
\end{tcolorbox}

\begin{definition}[$\sigma$-ring, $\sigma$-algebra]
    集合$M\subset P(X)$について,
    \begin{enumerate}
        \item 次の2条件を満たすとき,$M$を$\sigma$-環という:
        \begin{enumerate}[(a)]
            \item 任意の可算族$(A_n)$について,$\cup_{n}A_n\in M$.これは$\emptyset\in M$を含意する.
            \item 任意の$A,B\in M$について,$A\setminus B\in M$.
        \end{enumerate}
        $\sigma$-環は$\delta$-環である.
        有限合併についてのみ閉じているとき,$M$をBoole環という.
        これは,加法を対称差$A+B=(A\cup B)\setminus(A\cap B)=(A\setminus B)\cup(B\setminus A)$,積を共通部分として環をなすためである.
        \item $X\in M$を満たすBoole環をBoole代数といい,$X\in M$を満たす$\sigma$-環を$\sigma$-代数という.
        これは,$\sigma$-環が定める環$(M,+,\cap)$が単位的であることに同値.
    \end{enumerate}
\end{definition}
\begin{remarks}[$\sigma$-環の環構造が単位的であることと$\sigma$-代数であることとは同値]
    $\sigma$-環$M$の演算$+,\cdot$は,対称差$A+B=(A\setminus B)\cup(B\setminus A)$と共通部分$A\cdot B=A\cap B$である.
    環が可換になる(=代数になる)条件は,積の単位元$X$を含むかどうかである.
    「環」の用語はBoole環の略であり,歴史的には環と言っても単位的であることを必要としなかった.
    したがって,単にBoole環$P(X)$の部分環を指す.
\end{remarks}

\begin{lemma}[$\sigma$-代数が定める関数の代数の極めて代数的に良い振る舞い]\label{lemma-function-space-defined-by-measurable-sets}
    $S\subset P(X)$を集合系とする.
    \begin{enumerate}
        \item $\F\subset\Map(X,\R)$を単調点列完備な代数とする.これが定める集合系$S:=\Brace{A\in P(X)\mid [A]\in\F}$は$\sigma$-環である.
        \item $S\subset P(X)$を$\sigma$-代数とする.$\F:=\Brace{f\in\Map(X,\R)\mid\forall_{t\in\R}\{f>t\}\in S}$と定めると,$\F$は点列完備な,$\land,\lor$について閉じている単位的代数であり,$\forall_{f\in\F}\;\forall_{p>0}\;\abs{f}^p\in\F$を満たす.
    \end{enumerate}
\end{lemma}

\subsection{Borel関数はコンパクト台を持つ連続関数の単調列の極限である}

\begin{tcolorbox}[colframe=ForestGreen, colback=ForestGreen!10!white,breakable,colbacktitle=ForestGreen!40!white,coltitle=black,fonttitle=\bfseries\sffamily,
title=単関数各点近似が出来るクラスが可測関数である]
    BaireとLebesgueによるBorel集合論を用いて,先程の一般論から,Borel関数のなす点列完備な単位的代数$\B(X)$を考えると,実はこれは
    $C_c(X)$の各点完備化として得られる:$\B(C_c(X))=\B(X)$.
    これが単関数の理論である.
\end{tcolorbox}

\begin{example}[Euclid空間のBorel集合]
    $\R^n$の開集合は,開矩形の可算積で表せる.開矩形は$2n$個の開半空間の共通部分として表せる.
    したがって,$\R^n$のBorel集合系は,すべての半開空間$\Brace{x\in\R^n\mid x_k>t}_{t\in\R,k\in[n]}$が生成する$\sigma$-環である.
\end{example}

\begin{definition}[Borel map]
    位相空間$X,Y$の間の写像$f:X\to Y$がBorel写像であるとは,$\forall_{B\in\B(Y)}\;f^{-1}(B)\in\B(X)$が成り立つことをいう.
\end{definition}

\begin{corollary}[Borel関数のなす各点完備な単位的代数]
    位相空間$X$上のBorel関数のクラス$\B(X)$は,点列完備な単位的代数で,束演算$\land,\lor$について閉じている.
    また,$\forall_{f\in\B(X)}\;\forall_{p>0}\;\abs{f}^p\in\B(X)$.
\end{corollary}

\begin{lemma}[コンパクト集合の定義関数のコンパクト台を持つ連続関数による近似]
    局所コンパクトハウスドルフ空間$X$について,任意のコンパクト集合$C\subset X$について,$C_c(X)$上の単調減少ネット$(f_\lambda)$が存在して,$f_\lambda\searrow[C]$が成り立つ.
    $X$が第2可算であるとき,数列についての議論で十分.
\end{lemma}

\begin{proposition}[Borel関数の空間の特徴付け]
    $X$を第2可算な局所コンパクトハウスドルフ空間とする.
    \begin{enumerate}
        \item $X$上のBorel関数の空間$\B(X)$は,$C_c(X)$の単調点列完備化である.
        \item $X$上の有界Borel関数の空間$\B_b(X)$は,$C_c(X)$の$l^\infty(X)$上での単調点列完備化である.
    \end{enumerate}
\end{proposition}

\begin{remarks}[Baire function]\label{remark-Baire-function}
    第二可算とは限らない局所コンパクトハウスドルフ空間$X$については,一般に,単調点列完備化について$\B(C_c(X))\subset\B(X)$が成り立つ.
    ただし,$C_c(X)$の単調点列完備化を$\B(C_c(X))$で表した.
    この真に小さいかもしれないクラスを\textbf{Baire関数}という.
    またこのクラスは$\land,\lor$について閉じている(補題\ref{lemma-sequential-completion}).
    $1$がBaire関数であることは,$X$が$\sigma$-コンパクトであることに同値.
    定義関数$[B]$がBaire関数であるとき,$B$を\textbf{Baire集合}という.
    Baire集合はBorel $\sigma$-代数$\B_X$の中の$\delta$-環をなす.
    この$\delta$-環は$X$内のコンパクト$G_\delta$-集合によって生成される.
\end{remarks}

\begin{proposition}[コンパクトな$G_\delta$-集合の関数による特徴付け]
    部分集合$C\subset X$がコンパクト$G_\delta$-集合であることは,$\exists_{f\in C_c(X)}\;\exists_{\ep>0}\;C=\{f\ge \ep\}$に同値.
\end{proposition}

\begin{remark}[一般の位相空間上のBorel関数の空間の得方]
    一般の位相空間$X$について,$\B(X)$を完備化として得たいときは,$C^{1/2}_b(X)$を有界な下半連続関数の空間として,クラス
    \[\F:=C^{1/2}_b(X)-C^{1/2}_b(X)\]
    を考える.これは線型空間で,代数でもある.
    任意の開集合$A$について定義関数$[A]$を含むから,$\F$の単調点列完備化が$\B(X)$である.
\end{remark}

\subsection{可測集合の恢復}

\begin{tcolorbox}[colframe=ForestGreen, colback=ForestGreen!10!white,breakable,colbacktitle=ForestGreen!40!white,coltitle=black,fonttitle=\bfseries\sffamily,
title=]
    現状定義されているのは$\L^1(X)$上の積分のみである.
    これを用いて,体積を求められる集合が体積確定集合,局所的に体積確定な集合を可測集合と復元すれば良い.
    すると,ちゃんとBorel集合を含むクラスとなっているので恢復出来ている.
\end{tcolorbox}

\begin{definition}\label{def-measurable-sets}
    $X$を局所コンパクトハウスドルフ空間とする.
    \begin{enumerate}
        \item $\cC$で$X$のコンパクト部分集合系とする.
        \item $\int:C_c(X)\to\R$をRadon積分(=非負性を保存する線型汎関数)とし,$\L^1(X)$をこれについての可積分関数とする\ref{thm-extension-of-Radon-integral}.
        \[\M^1:=\Brace{B\in P(X)\mid [B]\in\L^1(X)},\qquad\M:=\Brace{A\in P(X)\mid\forall_{C\in\cC}\;A\cap C\in\M^1}\]
        と定め,$\M$の元を\textbf{可測集合}という.
    \end{enumerate}
\end{definition}

\begin{proposition}
    局所コンパクトハウスドルフ空間$X$上のRadon積分$\int$が定める可測集合系$\M$は,Borel集合系$\B$を含む$\sigma$-代数で,$\cC\subset\M^1$を満たす.
\end{proposition}

\subsection{可測関数の恢復}

\begin{tcolorbox}[colframe=ForestGreen, colback=ForestGreen!10!white,breakable,colbacktitle=ForestGreen!40!white,coltitle=black,fonttitle=\bfseries\sffamily,
title=]
    一度$X$のコンパクト集合の言葉を借りることによって,可積分関数の空間$\L^1(X)$から可測関数の空間$\L(X)$を復元できる.
    そして,関数代数としての代数的に理想的な性質は,ずっと引き継がれている.
\end{tcolorbox}

\begin{definition}\label{def-measurable-function}
    Radon積分$\int:C_c(X)\to\R$について,
    \begin{enumerate}
        \item 関数$f:X\to\R$が\textbf{可測}であるとは,$\forall_{t\in\R}\;\{f>t\}\in\M$を満たすことを言う.これは$\forall_{B\in\B_\R}\;f^{-1}(B)\in\M$と$\forall_{g\in\B(\R)}\;f\circ g\in\L(X)$を含意する.
        \item $X$上の可測関数の空間を$\L(X)$で表す.$\B\subset\M$より,$\B(X)\subset\L(X)$である.
    \end{enumerate}
\end{definition}

\begin{corollary}
    局所コンパクトハウスドルフ空間$X$上のRadon積分に関する可測関数の空間$\L(X)$は,
    \begin{enumerate}
        \item 点列完備な単位的代数で,
        \item $\lor,\land$について閉じており,
        \item $\forall_{f\in\L(X)}\;\abs{f}^p\in\L(X)$を満たす.
    \end{enumerate}
\end{corollary}
\begin{Proof}
    補題\ref{lemma-function-space-defined-by-measurable-sets}(2)の具体例である.
\end{Proof}

\subsection{可積分性の特徴付けの恢復}

\begin{lemma}[体積確定集合の特徴付け]\label{lemma-regularity-of-Radon-integral}
    任意の部分集合$B\subset X$について,
    \[\int_*[B]=\sup\Brace{\int[C]\in\R_{\ge0}\;\middle|\;C\subset B,C\in\C},\]
    \[\int^*[B]=\inf\Brace{\int^*[A]\in\R_{\ge0}\;\middle|\;B\subset A,A:\text{open}}.\]
    また,$B\in\M^1$は,$B\in\M$かつ$\int^*[B]<\infty$に同値.
\end{lemma}

\begin{theorem}[可積分関数の特徴付け]\label{thm-Radon-integrability}
    $\int:C_c(X)\to\R$を局所コンパクトハウスドルフ空間$X$上のRadon積分とする.
    \begin{enumerate}
        \item 任意の可積分関数は可測である:$\L^1(X)\subset\L(X)$.
        \item 任意の可測関数について,可積分であることと,$\int^*\abs{f}<\infty$は同値.
    \end{enumerate}
\end{theorem}

\section{測度}

\begin{tcolorbox}[colframe=ForestGreen, colback=ForestGreen!10!white,breakable,colbacktitle=ForestGreen!40!white,coltitle=black,fonttitle=\bfseries\sffamily,
title=]
    定義関数に関する積分によって測度を復元出来る.
    ここで,Radon測度としては,下積分$\int_*$が定めるものを採用するが,$\sigma$-有限な空間については一致する.

\end{tcolorbox}

\subsection{Radon測度の定義}

\begin{tcolorbox}[colframe=ForestGreen, colback=ForestGreen!10!white,breakable,colbacktitle=ForestGreen!40!white,coltitle=black,fonttitle=\bfseries\sffamily,
title=]
    定義関数に対する下積分は,Radon測度を定める(という流儀の定義を採用する).

\end{tcolorbox}

\begin{definition}[measure, Radon measure]\mbox{}\label{def-Radon-measure}
    \begin{enumerate}
        \item $\sigma$-環$S\subset P(X)$上の\textbf{測度}とは,関数$\mu:S\to[0,\infty]$であって,$\sigma$-加法的であるものをいう:$\mu\paren{\cup A_n}=\sum\mu(A_n)$.
        \item $X$が局所コンパクトハウスドルフ空間,$\B_X\subset S$を満たす$(X,S)$について,次の2条件を満たす測度を\textbf{Radon測度}という:
        \begin{enumerate}[(i)]
            \item (locally finite / locally integrable) $\forall_{C\in\cC}\;\mu(C)<\infty$.
            \item (inner regularity) $\forall_{A\in S}\;\mu(A)=\sup\Brace{\mu(C)\in\R_{\ge0}\mid C\subset A,C\in\cC}$.
        \end{enumerate}
    \end{enumerate}
\end{definition}
\begin{remark}
    $\B_X\subset S$であるから,Radon測度はBorel測度である.
    幾何学的に興味のあるほとんどの測度はRadonである.
\end{remark}

\begin{proposition}[Radon積分からRadon測度の復元]\label{prop-recovery-of-Radon-measure}
    局所コンパクトハウスドルフ空間$X$上のRadon積分$\int:C_c(X)\to\R$に,次のRadon積分が定める可測集合$\M$上の2つの測度$\mu^*,\mu_*:\M\to[0,\infty]$が対応する:
    \begin{enumerate}
        \item $\mu^*(A):=\int^*[A]$.
        \item $\mu_*(A):=\int_*[A]$.
    \end{enumerate}
    このとき,$\mu_*$はRadon測度で,$\mu^*$は外正則:
    \[\mu^*(A)=\inf\Brace{\mu^*(B)\in[0,\infty]\mid A\subset B,B:\text{open}}\]
    で$\mu_*\le\mu^*$を満たす.
    また,$A$が測度確定な可測集合$\M^1(\subset\M)$\ref{def-measurable-sets}の可算合併として表せるとき,$\mu_*(A)=\mu^*(A)$.
\end{proposition}
\begin{remarks}
    $\mu^*,\mu_*$が有限のとき,あるいは$X$は$\sigma$-有限のとい,$\mu^*=\mu_*$が成り立ち,Radon測度とは正則Borel測度(外部正則性も満たす)に他ならない.
    しかしRadon測度の定義を上述の2条件とすることで,この命題の主張の逆が成り立つ.すなわち,$\mu^*$とこれに付随する本質的測度$\mu_*$について,命題の条件を満たすならば$\mu_*$はRadon測度になる.
    なお,この外部正則な$\mu^*$をRadon測度と呼ぶことも,組$(\mu_*,\mu^*)$をRadon測度と呼ぶこともある.
    片方からもう片方も復元でき,また有限測度であるとき一致する.
\end{remarks}

\subsection{測度の正則性}

\begin{tcolorbox}[colframe=ForestGreen, colback=ForestGreen!10!white,breakable,colbacktitle=ForestGreen!40!white,coltitle=black,fonttitle=\bfseries\sffamily,
title=]
    局所コンパクトハウスドルフ空間上の有限なRadon測度は外部正則でもある.
    特に第2可算だと任意のBorel確率測度がRadonかつ正則になる.
\end{tcolorbox}

\begin{definition}[Borel meaesure, Baire measure, inner regular / tight, locally finite, outer regular, essential measure]
    $X$を位相空間とし,$\B$をそのBorel $\sigma$-代数とする.
    \begin{enumerate}
        \item Borel測度とは,Borel $\sigma$-代数$\B$を含む$\sigma$-代数上の測度をいう.
        \item Baire測度とは,$X$内のコンパクトな$G_\delta$-集合によって生成される$\sigma$-代数上の測度をいう.
        \item Borel測度が$\mu(A)=\sup\Brace{\mu(K)\in[0,\infty]\;\middle|\; K\overset{\mathrm{cpt}}{\subset}A}$を満たすとき,\textbf{内部正則}または\textbf{緊密}であるという.
        \item 測度が\textbf{局所有限}であるとは,任意の点が測度有限な近傍を持つことを言う.特に,局所コンパクトハウスドルフ空間上では,
        $\forall_{K\overset{\mathrm{cpt}}{\subset}X}\;\mu(K)<\infty$と同値.
        \item 測度が\textbf{外部正則}であるとは,$\forall_{A\in\B}\;\mu(A)=\inf\Brace{\mu(B)\in[0,\infty]\mid A\subset B\osub X}$を満たすことをいう.
        \item 外部正則かつ内部正則なBorel測度を\textbf{正則Borel測度}という.
        \item Borel測度$m$が$M$に付随する\textbf{本質的測度}であるとは,
        \[m(A)=\sup\Brace{m^*(C)\in\R\mid C\subset A,m^*(C)<\infty},\quad\text{ただし}m^*(C)=\inf\Brace{m(B)\in[0,\infty]\mid C\subset B,B\in\B}\]
    \end{enumerate}
    結局,内部正則かつ局所有限なBorel測度をRadon測度といい,いま考えている局所コンパクトハウスドルフ空間$X$上で考えられる素性の良い測度のクラスである.
\end{definition}

\begin{lemma}[Borel測度がRadonになるとき]\mbox{}
    \begin{enumerate}
        \item (Ulam) 完備可分な距離空間上の有限なBorel測度は,内部正則である(このような条件を満たす空間をRadon空間という).特に,完備可分距離空間上のBorel確率測度はRadon測度である.
        \item 他に,第2可算な局所コンパクトハウスドルフ空間上のBorel確率測度は正則になる.
        \item $X$を任意の開集合が$\sigma$-コンパクトな局所コンパクトハウスドルフ空間とする($\R^n$はこれを満たす).任意の局所有限なBorel測度は,正則である.
    \end{enumerate}
\end{lemma}

\begin{lemma}[Radon測度が外部正則になるとき]\mbox{}
    \begin{enumerate}
        \item 局所コンパクト空間上の有限なRadon測度は,外部正則である.特に有限なRadon測度は,有限な正則なBorel測度と同値になる.
    \end{enumerate}
    一方で,局所コンパクトハウスドルフ空間上の$\sigma$-有限なRadon測度で外部正則でないものが存在する(Bourbaki 2004).
\end{lemma}

\begin{theorem}[Radon測度の特徴付け]
    Borel測度$\mu:\B\to\bF$について,
    次の2条件は同値.
    \begin{enumerate}
        \item $\mu$はRadon測度である.
        \item $\mu$は,$X$の任意の開集合上において,内部正則かつ局所有限である上に,外部正則でもある.
        \item $\mu$は,次を満たす2つのBorel測度$m,M$が$\M^1$の元上に取る同一の値を取る:
        $m$は$M$に付随する本質的測度であり,$M$は局所有限かつ$\B$上外部正則かつ内部正則で,$B$が開集合または体積確定$B\in\M^1$ならば$m(B)=M(B)$である.
    \end{enumerate}
    (3)はRadon積分の方を重視する見方であり,今後の議論で同値性を見る\ref{prop-recovery-of-Radon-measure}.
    さらに,台空間$X$が局所コンパクトハウスドルフ空間であるとき,次も同値.
    \begin{enumerate}\setcounter{enumi}{3}
        \item $\mu$は$C_c(X,\R)$上の正な線型汎関数である.
    \end{enumerate}
\end{theorem}

\begin{definition}
    位相空間$X$上の
    Baire $\sigma$-代数とは,$C(X)$を可測にする最小の$\sigma$-代数をいう.このとき,
    $C_b(X)$を可測にする最小の$\sigma$-代数としても特徴付けられる.
    また,$C_c(X)$の点列完備化$\B(C_c(X))$に含まれる関数をBaire関数という,という定義も出来る\ref{remark-Baire-function}.
\end{definition}

\begin{lemma}
    任意の距離空間において,Borel集合とBaire集合とは一致する.
\end{lemma}

\begin{theorem}
    任意のコンパクトハウスドルフ空間において,有限なBaire測度は外部正則である.
\end{theorem}

\subsection{Rieszの表現定理}

\begin{tcolorbox}[colframe=ForestGreen, colback=ForestGreen!10!white,breakable,colbacktitle=ForestGreen!40!white,coltitle=black,fonttitle=\bfseries\sffamily,
title=]
    Radon積分からRadon測度を構成する手順を示したが,きれいに対応が付く.
    こうしてこの2つの対象は数学的には全く等価であることが分かる.
\end{tcolorbox}

\begin{lemma}
    $\mu$を$\sigma$-代数$S\subset P(X)$上の測度,$\F$を$S$-可測関数全体からなる集合とする:$\forall_{t\in\R}\;\{f>t\}\in S$\ref{lemma-sequential-completion}.
    このとき,非負性を保存する斉次な加法的関数$\Phi:\F_+\to[0,\infty]$であって.次の2条件を満たすものが一意的に存在する:
    \begin{enumerate}
        \item $\forall_{A\in S}\;\Phi([A])=\mu(A)$.
        \item $\forall_{f_n,f\in\F_+}\;f_n\nearrow f\Rightarrow\Phi(f)=\lim\Phi(f_n)$.
    \end{enumerate}
\end{lemma}

\begin{theorem}[Riesz' representation theorem 09]\label{thm-Riesz-representation}
    局所コンパクトハウスドルフ空間$X$上のRadon積分と,$X$のBorel集合$\B_X$上のRadon測度との間に,
    \[\mu(A)=\int_*[A]\]
    によって定まる全単射が存在する.
\end{theorem}

\subsection{積分の最大限の拡張}

\begin{tcolorbox}[colframe=ForestGreen, colback=ForestGreen!10!white,breakable,colbacktitle=ForestGreen!40!white,coltitle=black,fonttitle=\bfseries\sffamily,
title=]
    まだ$\L^1(X)\subset\L^1_\ess(X)$に延長出来る.
\end{tcolorbox}

\begin{remark}
    Rieszの表現定理によると,Daniellの延長定理\ref{thm-extension-of-Radon-integral}はまだ最大の延長ではないことがわかる.
\end{remark}

\begin{definition}[essential integral]\mbox{}
    \begin{enumerate}
        \item $\L^1_\ess(X):=\Brace{f\in\L(X)\;\middle|\;\int_*\abs{f}<\infty}$.
        \item $\L^1_\ess(X)$上の積分$\int_\ess$が,
        \[\int_\ess f:=\int_*f\lor 0+\int^*f\land 0\]
        によって定まり,これはLebesgueの優収束定理を満たすようなRadon積分$\int:C_c(X)\to\R$の延長であるような非負性を保存する線型汎関数である.
        $\mu_*\ne\mu^*$のとき,$\L^1(X)\subsetneq\L^1_\ess(X)$である.\footnote{$X$がパラコンパクトであるとき,すなわちコンパクト集合の可算合併であるとき,$\mu_*=\mu^*$であることに注意.}
    \end{enumerate}
\end{definition}

\begin{proposition}\label{prop-Radon-integral-on-paracompact-spaces}
    $X$を$\sigma$-コンパクトな局所コンパクトハウスドルフ空間とする.
    \begin{enumerate}
        \item 任意のRadon測度は外正則である.
        \item $\int:C_c(X)\to\R$をRadon積分とすると,これについての可測関数$f$が可積分であることは$\int_*\abs{f}<\infty$に同値.
    \end{enumerate}
\end{proposition}

\subsection{拡張積分}

\begin{definition}[extended integrable]
    $X$を局所コンパクトハウスドルフ空間,$\int:C_c(X)\to\R$をRadon積分とする.
    \begin{enumerate}
        \item 可測関数$f\ge0$が\textbf{広義積分可能}であるとは,$\int_*f=\int^*f\in\R\cup\{\infty\}$を満たすことをいう.
        \item 一般の可測関数$f$が\textbf{広義積分可能}であるとは,$f\lor0$と$-(f\land0)$のいずれかが可積分で,もう一方が拡張積分可能であることをいう.
        \item $\int_\ext f=\int f\lor0-\int(-f\land 0)\in\R\cup\{\pm\infty\}$と表す.$\infty-\infty$の場合は定義できないので,拡張積分可能な関数全体の集合は線型空間ではない.
    \end{enumerate}
\end{definition}

\begin{lemma}
    正な拡張積分可能な関数は正錐をなし,Radon積分はその上に非負性を保存する斉次な$\sigma$-加法的関数として作用する.
\end{lemma}

\begin{lemma}
    $X$が$\sigma$-コンパクトであるとき,任意の正な可測関数は拡張可積分であり(命題\ref{prop-Radon-integral-on-paracompact-spaces}),
    任意の可測関数は$\int f\lor0=\infty,\int f\land0=-\infty$である場合を除いて拡張可積分である.
\end{lemma}

\section{$L^p$-空間}

\begin{tcolorbox}[colframe=ForestGreen, colback=ForestGreen!10!white,breakable,colbacktitle=ForestGreen!40!white,coltitle=black,fonttitle=\bfseries\sffamily,
title=]
    局所コンパクトハウスドルフ空間$X$とその上のRadon積分$\int:C_c(X)\to\R$を1つ固定し,
    可測関数のなす線型束$\L(X)$の各種部分空間を調べる.
\end{tcolorbox}

\subsection{零集合と完備化}

\begin{tcolorbox}[colframe=ForestGreen, colback=ForestGreen!10!white,breakable,colbacktitle=ForestGreen!40!white,coltitle=black,fonttitle=\bfseries\sffamily,
title=]
    $\L(X),\L^1(X)$はみな関数の代数として極めて良い性質を持っていた.一方で$L(X),L^1(X)$は代数的,そして束としての演算はa.e.の意味でしか成り立たない.
    が,零集合上で無限大の値を取ってしまうような極限関数についても完備になる,という意味では極めて使いやすい空間が出来上がる(Beppo-Leviの定理など).
\end{tcolorbox}

\begin{definition}\mbox{}\label{def-null-set}
    \begin{enumerate}
        \item $\cN(X):=\Brace{f\in\L(X)\;\middle|\;\int\abs{f}=0}$の元を\textbf{零関数}という.
        \item $\cN:=\Brace{N\in\M\mid[N]\in\cN(X)}$の元を\textbf{零集合}という.
    \end{enumerate}
\end{definition}

\begin{lemma}\mbox{}
    \begin{enumerate}
        \item $\cN(X)$は$\L(X)$内の点列完備なイデアルであり,束でもある.
        \item $\cN$は$\sigma$-代数$\M$内のイデアルであり,$\sigma$-環でもある.
    \end{enumerate}
\end{lemma}

\begin{definition}
    今後,$X$上の関数とは,$X\to\R\cup\{\pm\infty\}$を指すとし,$f\in\L(X)\Leftrightarrow\exists_{N\in\cN}\;\forall_{x\notin N}\;f(x)\in\R\land[X\setminus N]f\in\L(X)$であるとする.
\end{definition}
\begin{remarks}
    代数としての,そして束としての構造が「殆ど至る所」しか成り立たなくなり,その構造を回復するには$\L(X)/\cN(X)=:L(X)$なる空間を考える必要があり,この空間はもはや直接の関数空間ではなくなる.
    これは一見不便であるが,次のBeppo-Leviの定理という名の単調収束定理の変種のように,関数の極限はこのように拡張実数値関数を自然に生んでしまう.
    その際には,個別に議論するよりかは,各積分に応じて「殆ど至る所」という言及の仕方が明瞭になりえる.
\end{remarks}

\begin{theorem}[Beppo-Levi's theorem]
    $\L^1(X)$の列$(f_n)$は,$\forall_{n\in\N}\;f_n(x)\le f_{n+1}(x)\;\ae$を満たし,$\lim\int f_n<\infty$であるとする.
    このとき,ある元$f\in\L^1(X)$が存在して,$\int f=\lim\int f_n$かつ$f(x)=\lim f_n(x)\;\ae$を満たす.
\end{theorem}

\subsection{Radon積分のLebesgue分解}

\begin{tcolorbox}[colframe=ForestGreen, colback=ForestGreen!10!white,breakable,colbacktitle=ForestGreen!40!white,coltitle=black,fonttitle=\bfseries\sffamily,
title=]
    任意の完全加法的集合関数はLebesgue分解を持つ.
\end{tcolorbox}

\begin{definition}[continuous / diffuse, atomic]\mbox{}\label{def-diffuse-atomic}
    \begin{enumerate}
        \item Radon積分$\int$が\textbf{連続}であるとは,$\forall_{x\in X}\;\int[\{x\}]=0$を満たすことをいう.
        \item Radon積分が\textbf{原子的}であるとは,ある集合$S\subset X$が存在して,$X\setminus S\in\cN$かつ$\forall_{C\in\cC}\;S\cap C$が可算であることをいう.
    \end{enumerate}
\end{definition}

\begin{lemma}
    実数上のRadon積分$\int:C_c(\R)\to\R$について,次の2条件は同値.
    \begin{enumerate}
        \item $\int$は連続.
        \item $\int$を定める右連続な単調増加関数$\int$は連続.
    \end{enumerate}
\end{lemma}

\begin{example}
    Dirac積分$\delta_x(f)=f(x)$は原子的なRadon積分である.
\end{example}

\begin{lemma}[Radon積分のLebesgue分解]
    任意のRadon積分$\int$について,連続部分$\int_c$と原子的部分$\int_a$とが存在して,$\int=\int_c+\int_a$と表せる.
\end{lemma}
\begin{Proof}
    \[S:=\Brace{x\in X\;\middle|\;\int[\{x\}]>0}\]
    とすると,$S\cap C$は可算集合である.実際,非可算としたら$\int[C]<\infty$に矛盾.よって,可測集合の定義から$S\in\M$.
    よって,
    \[\int_af:=\int[S]f,\quad\int_cf:=\int[X\setminus S]f\]
    はRadon積分を定める.
\end{Proof}

\subsection{Lebesgue空間}

\begin{tcolorbox}[colframe=ForestGreen, colback=ForestGreen!10!white,breakable,colbacktitle=ForestGreen!40!white,coltitle=black,fonttitle=\bfseries\sffamily,
title=]
    線型束と関数代数の構造にとどまらず,
    $\L,\L^p$はセミノルム空間をなす.
\end{tcolorbox}

\begin{definition}[Lebesgue space]
    実数$p\in\cointerval{1,\infty}$について,位数$p$のLebesgue空間とは,線型空間
    \[\L^p(X):=\Brace{f\in\L(X)\;\middle|\;\abs{f}^p\in\L^1(X)}.\]
    をいう.
    また,$\L^p(X)$上の関数を$\norm{f}_p:=\paren{\int\abs{f}^p}^{1/p}$と定める.
    これがセミノルムをなすことを示す.
\end{definition}

\begin{lemma}\mbox{}
    \begin{enumerate}
        \item $\L^p(X)$は線型空間である.
        \item $f\mapsto\norm{f}_p$は非負性を保存し,斉次である.
        \item $f\mapsto\norm{f}_p$はセミノルムである.
    \end{enumerate}
\end{lemma}

\begin{definition}[essential supremum, Lebesgue space]\mbox{}
    \begin{enumerate}
        \item 可測関数$f\in\L(X)$について,
        \[\esssup f:=\inf\Brace{t\in\R\;\middle|\;\int^*[\{f>t\}]=0}=\inf\Brace{t\in\R\;\middle|\;\int^*(f-f\land t)=0}.\]
        \item Lebesgue空間$\L^\infty(X)$で,$X$上の本質的に有界な可測関数のなす線型空間を表す.$\norm{f}_\infty:=\esssup\abs{f}$はその上にセミノルムを定める.
    \end{enumerate}
\end{definition}

\begin{lemma}[H\"{o}lder's inequality]
    共役指数$p,q\in[1,\infty]$について,$f\in\L^p(X),g\in\L^q(X)$とする:$p^{-1}+q^{-1}=1$.
    このとき,$fg\in\L^1(X)$で,$\norm{fg}_1\le\norm{f}_p\norm{g}_q$.
\end{lemma}

\begin{lemma}[Minkowski's inequality]
    $f,g\in\L^p(X),p\in[1,\infty]$について$f+g\in\L^p(X)$で,$\norm{f+g}_p\le\norm{f}_p+\norm{g}_p$.
\end{lemma}

\subsection{完備化}

\begin{tcolorbox}[colframe=ForestGreen, colback=ForestGreen!10!white,breakable,colbacktitle=ForestGreen!40!white,coltitle=black,fonttitle=\bfseries\sffamily,
title=]
    零関数の空間は,ノルムが零の空間として特徴付けられ,Minkowskiの不等式よりノルムは連続である.
    こうして,商空間としてノルム空間を得て,これが完備であることはEgoroffの定理から従う.
    $\L^p$の$\norm{-}_p$に関するCauchy列は殆ど至る所各点収束し,確率的に一様収束する.
\end{tcolorbox}

\begin{lemma}
    任意の$p\in[1,\infty]$について,
    \begin{enumerate}
        \item $\cN(X)\subset\L^p(X)$.
        \item $\cN(X)=\Brace{f\in\L^p(X)\mid\norm{f}_p=0}$.
        \item 商空間$L^p(X):=\L^p(X)/\cN(X)$は$\norm{-}_p$についてノルム空間をなす.
    \end{enumerate}
\end{lemma}

\begin{proposition}[Egoroffの定理]
    $p\in\cointerval{1,\infty}$とする.任意の$\L^p(X)$のCauchy列と任意の$\ep>0$について,部分列$(f_n)$と開集合$A$であって$\int[A]<\ep$を満たすものと零集合$N$が存在して,$(f_n)$は$X\setminus A$上一様に収束し,$X\setminus N$上各点収束する.
\end{proposition}

\begin{theorem}[Riesz-Fischerの定理]\label{thm-Riesz-Fischer}
    任意の$1\le p\le\infty$について,
    $L^p(X)$はBanach空間である.
\end{theorem}

\subsection{稠密部分集合}

\begin{proposition}\label{prop-dense-subset-of-Lp}
    $p\in\cointerval{1,\infty}$とする.$C_c(X)$(の商写像についての像)は$L^p(X)$上稠密である.
\end{proposition}
\begin{remarks}
    $p=\infty$の場合は,双対\ref{thm-duality-of-Lp}により,$(L^1(X))^*$の
    $w^*$-位相について,$C_c(X)\mono L^\infty(X)$は$w^*$-稠密である.従う.
\end{remarks}

\begin{corollary}[Lusinの定理]
    $p\in\cointerval{1,\infty}$とする.任意の$f\in\L^p(X)$と$\ep>0$について,開集合$A$であって$\int[A]<\ep$を満たすものが存在し,$f|_{X\setminus A}\in C_0(X\setminus A)$を満たす.
\end{corollary}

\subsection{Borel関数近似}

\begin{proposition}\label{prop-approximation-by-Borel-function}
    $p\in\cointerval{1,\infty}$とする.
    \begin{enumerate}
        \item 任意の可測関数$f\in\L^p(X)$について,Borel関数$g\in\B(X)$が存在して,$f-g\in\cN(X)$を満たす.
        \item 同様のことが,$X$の$\sigma$-コンパクト部分集合$S$について$X\setminus S$上殆ど至る所$0$な可測関数$f\in\L(X)$についても成り立つ.
    \end{enumerate}
\end{proposition}

\subsection{複素関数について}

\begin{tcolorbox}[colframe=ForestGreen, colback=ForestGreen!10!white,breakable,colbacktitle=ForestGreen!40!white,coltitle=black,fonttitle=\bfseries\sffamily,
title=]
    基本はスムーズに証明されるが,次の命題だけ特別な証明が必要になる.
\end{tcolorbox}

\begin{proposition}
    可測関数$f:X\to\C$が可積分であることと$\int^*\abs{f}<\infty$を満たすことは同値.
    またこの条件下で$\Abs{\int f}\le\int\abs{f}$.
\end{proposition}
\begin{remarks}
    証明抽出により,$f:X\to\C$について,$f\in\L^p(X)\Leftrightarrow\abs{f}\in\L^p(X)$もわかる.
\end{remarks}

\subsection{Lebesgue空間の相互関係}

\begin{tcolorbox}[colframe=ForestGreen, colback=ForestGreen!10!white,breakable,colbacktitle=ForestGreen!40!white,coltitle=black,fonttitle=\bfseries\sffamily,
title=]
    この結果は$\bF=\R,\C$のいずれについても同様の証明によって成り立つ.
\end{tcolorbox}

\begin{proposition}
    $1\le p<r<q\le\infty$のとき,
    \begin{enumerate}
        \item $\L^p(X)\cap\L^q(X)\subset\L^r(X)$.
        \item $\forall_{f\in\L^p(X)\cap\L^q(X)}\;\norm{f}_r\le\norm{f}_p\lor\norm{f}_q$.
    \end{enumerate}
\end{proposition}

\begin{corollary}\mbox{}
    \begin{enumerate}
        \item $\limsup_{r\to\infty}\norm{f}_r\le\norm{f}_\infty$.
        \item $\forall_{p\in[1,\infty]}\;\forall_{f\in\L^p(X)\cap\L^\infty(X)}\;\norm{f}_r\to\norm{f}_\infty$.
    \end{enumerate}
\end{corollary}

\subsection{包含関係}

\begin{tcolorbox}[colframe=ForestGreen, colback=ForestGreen!10!white,breakable,colbacktitle=ForestGreen!40!white,coltitle=black,fonttitle=\bfseries\sffamily,
title=]
    $\int$が有限:$\int 1<\infty$であるときと,$\M$が原始的であるとき$\exists_{\ep>0}\;\forall_{A\in\M}\;A\in\N\lor\int[A]\ge\ep$に,$L^p$空間は次のような包含関係を持つ.
    前者は一般性を失うことなく確率分布とすれば良く,後者は$X$が$\sigma$-コンパクトのとき,積分は$X$のある可算部分集合上に台を持つ状況を考えれば良い.すなわち,離散確率空間である.
\end{tcolorbox}

\begin{corollary}[確率測度に関する$L^p$-空間の包含関係]
    $\int 1=1$かつ$1\le p<q\le\infty$のとき,$\L^q(X)\subset\L^p(X)$かつ$\norm{-}_p\le\norm{-}_q$である.
\end{corollary}
\begin{remarks}[確率空間上のLebesgue空間]
    確率測度については,$p\in[1,\infty]$が小さいほど空間$\L^p(X)$は大きく,ノルム$\norm{f}_p$は小さい.
\end{remarks}

\begin{corollary}[離散確率測度に関する$L^p$-空間の包含関係]
    任意の$1\le p<q\le\infty$について,$l^p\subset l^q$かつ$\norm{-}_q\le\norm{-}_p$が成り立つ.
\end{corollary}

\section{双対理論}

\begin{tcolorbox}[colframe=ForestGreen, colback=ForestGreen!10!white,breakable,colbacktitle=ForestGreen!40!white,coltitle=black,fonttitle=\bfseries\sffamily,
title=完全か法的集合関数論を,Radon積分について展開する]
    Radon-Nikodymの定理を考えるにあたって,
    位相的測度論では,Radon測度の内部正則性と本質的積分を用いて,$\sigma$-有限性の条件を取り除くことが可能だが,特にこれはしないこととする.
    すなわち,$X$を$\sigma$-コンパクトな局所コンパクトハウスドルフ空間とする.
    このとき,Radon測度は相対コンパクト集合に有限な測度を与えるから,$X$は$\sigma$-有限である.

\end{tcolorbox}

\begin{remark}
    一般の局所コンパクトハウスドルフ空間$X$上の$\sigma$-有限なRadon積分を考えれば十分である.
    実際,$\{A_n\}\subset\M^1$かつ$\cup A_n=X$を満たすならば,補題\ref{lemma-regularity-of-Radon-integral}により,少し大きくして$A_n$はすべて開であると仮定して良い.
    次に,再び補題\ref{lemma-regularity-of-Radon-integral}より,各$A_n$に対して,相対コンパクトな開集合の列$(B_{nm})$であって$\forall_{m\in\N}\;\o{B_{nm}}\subset B_{nm+1}$を満たし,$A_n\setminus\cup B_{nm}\in\cN$を満たすものが取れる.
    すると,$Y:=\cup_{n,m\in\N}B_{nm}$は$\sigma$-コンパクトな開集合であり,$X\setminus Y\in\cN$を満たす.こうして$Y$について考えれば,$\sigma$-コンパクト性を仮定して議論していることと等価である.
\end{remark}

\subsection{絶対連続性の特徴付け}

\begin{tcolorbox}[colframe=ForestGreen, colback=ForestGreen!10!white,breakable,colbacktitle=ForestGreen!40!white,coltitle=black,fonttitle=\bfseries\sffamily,
title=]
    絶対連続性は測度の間と同様にして,積分の間にも定義できる.
    同値な積分は同型なLebesgue空間$L^\infty_0(X)=L^\infty(X)$を定める.
\end{tcolorbox}

\begin{definition}[absolutely continuous, equivalent]
    $\int_0,\int:C_c(X)\to\R$をRadon積分とする.
    \begin{enumerate}
        \item $\int_0,\int:C_c(X)\to\R$が次の命題の同値な条件を満たすとき,$\int_0$は$\int$について\textbf{絶対連続}であるといい,$\int_0\ll\int$と表す.
        \item $\int_0\ll\int\land\int\ll\int_0$が成り立つとき,\textbf{同値}であるといい,$\int_0\sim\int$で表す.これは$\cN(X)=\cN_0(X)$と同値で,$L^\infty_0(X)=L^\infty(X)$と同値.
    \end{enumerate}
\end{definition}

\begin{proposition}[絶対連続性の特徴付け]
    $\int_0,\int:C_c(X)\to\R$を局所コンパクトハウスドルフ空間$X$上のRadon積分とする.次の3条件は同値である.
    \begin{enumerate}
        \item 任意の単調減少列$\{f_n\}\subset C_c(X)_+$について,$\lim\int f_n=0\Rightarrow\lim\int_0f_n=0$が成り立つ.
        \item 任意のBorel集合$N\in\B(X)$について,$\int[N]=0\Rightarrow\int_0[N]=0$.
        \item 任意の非負Borel関数$f\in\B(X)_+$について,$\int f=0\Rightarrow\int_0f=0$.
    \end{enumerate}
\end{proposition}

\subsection{Radon-Nikodymの定理}

\begin{definition}[locally integrable]\label{def-locally-integrable-function}
    関数$f$が$\forall_{C\in\cC}\;[C]f\in\L^1(X)$を満たすとき,\textbf{局所可積分}であるという.このとき$f$は可測である.
\end{definition}


\begin{lemma}
    $X$上の局所可積分関数の空間$\L^1_\loc(X)$はRiesz空間である:演算$\land,\lor$について閉じている.
\end{lemma}

\begin{theorem}[Radon-Nikodym]
    $X$を$\sigma$-コンパクトな局所コンパクトハウスドルフ空間,$\int_0,\int:C_c(X)\to\R$をその上のRadon積分とする.
    このとき,次の2条件は同値.
    \begin{enumerate}
        \item $\int_0\ll\int$.
        \item Borel関数$m\ge0$が存在して,$\int$について局所可積分で,$\forall_{f\in\B(X)}\;\int_0f=\int fm$を満たす.
    \end{enumerate}
    なお,$m$は存在すれば零集合の差を除いて一意である.
\end{theorem}
\begin{remarks}
    ここでBorel関数の概念が登場しているのは,$L^1_\loc(X)$の代表元として取っているのみである.
    任意の可測関数はBorel関数を代表元に取れるのであった\ref{prop-approximation-by-Borel-function}.
\end{remarks}

\subsection{複素測度の扱い}


\begin{tcolorbox}[colframe=ForestGreen, colback=ForestGreen!10!white,breakable,colbacktitle=ForestGreen!40!white,coltitle=black,fonttitle=\bfseries\sffamily,
    title=]
    \cite{Rudin-RC}による.複素数が極表示を持ち,複素関数がEulerの表示を持つように,複素測度も極分解を持つ.
\end{tcolorbox}

\begin{theorem}[複素測度は有界変動]
    複素測度$\mu:\M\to\C$の全変動$\abs{\mu}$は測度を定め,また$\abs{\mu}(X)<\infty$を満たす.
\end{theorem}

\begin{definition}[absolutely continuous, concentrated, mutually singular]
    $\mu:\M\to[0,\infty]$を(正)測度,$\lambda,\lambda_i$を任意の測度$\lambda:\M\to\C\cup\{\infty\}$とする.
    \begin{enumerate}
        \item $\lambda\ll\mu:\Leftrightarrow\forall_{E\in\M}\;\lambda(E)\Rightarrow\mu(E)=0$.これは,$\forall_{\ep>0}\;\exists_{\delta>0}\;\forall_{E\in\M}\;\mu(E)<\delta\Rightarrow\abs{\lambda(E)}<\ep$と同値.これは$\mu$がLebesgue測度のとき,明らかに連続性よりも条件が強い一般化となっている.
        \item $\lambda$が$A\in\M$において\textbf{密集している}とは,$\forall_{E\in\M}\;\lambda(E)=\lambda(A\cap E)$を満たすことをいう.
        \item 2つの測度$\lambda_1,\lambda_2$がある互いに素な集合$A,B$について,それぞれ$A,B$に密集しているとき,\textbf{互いに特異}であるといい,$\lambda_1\perp\lambda_2$と表す.
    \end{enumerate}
\end{definition}

\begin{lemma}[$\sigma$-有限測度は本質的には有限]
    $\mu:\M\to[0,\infty]$を$(X,\M)$上の$\sigma$-有限な測度とする.
    このとき,ある可積分関数$w\in L^1(\mu)$が存在して,$0<w<1$を満たす.
\end{lemma}
\begin{remarks}
    これの本質は,$\mu$が$\sigma$-有限ならば,ある有限な測度$\wt{\mu}:=wd\mu:\M\to\R_+$が存在して,$\wt{\mu},\mu$が$0$となる集合は完全に一致する:$\wt{\mu}\ll\mu\land\wt{\mu}\gg\mu$.
\end{remarks}
\begin{theorem}[Lebesgue-Radon-Nikodym]
    $\mu:\M\to[0,\infty]$を$\sigma$-有限測度,$\lambda:\M\to\C$を複素測度とする.
    \begin{enumerate}
        \item 複素測度の組$(\lambda_a,\lambda_s)$がただ一つ存在して,$\lambda=\lambda_a+\lambda_s$かつ$\lambda_a\ll\mu,\lambda_s\perp\mu$.
        \item ただ一つの可積分関数$h\in L^1(\mu)$が存在して,$\forall_{E\in\M}\;\lambda_a(E)=\int_Ehd\mu$が成り立つ.
    \end{enumerate}
\end{theorem}

\begin{corollary}[複素測度の極分解]
    $\mu:\M\to\C$を複素測度とする.このとき,トーラスに値を取る可測関数$h:X\to\bS^1$であって,$d\mu=hd\abs{\mu}$を満たすものが存在する.
\end{corollary}

\subsection{Radon電荷の極分解}

\begin{tcolorbox}[colframe=ForestGreen, colback=ForestGreen!10!white,breakable,colbacktitle=ForestGreen!40!white,coltitle=black,fonttitle=\bfseries\sffamily,
title=]
    $C_c(X)$上のRadon積分が定めるセミノルムの族$\paren{\Abs{\int f}}_{f\in (C_c(X))^*_+}$が誘導する位相を入れると,$C_c(X)^*$はRadon電荷のなす空間と同型のはずである.
    そこで,これを具体的に構成する.
    まず,Radon電荷はトーラス上に値を取る複素可測関数$u$を用いて,これのRadon積分として得られる(そもそも測度として$d\mu=ud\abs{\mu}$が成り立つため).
    これを通じて実数値Radon電荷のJordan分解を得る.これは,2つのRadon積分として実数値Radon電荷が表せてしまう,という理論である.
\end{tcolorbox}

\begin{definition}[Radon charge]
    \textbf{Radon電荷}または\textbf{複素Radon積分}とは,線型汎関数$\Phi:C_c(X)\to\C$であって,
    \[\forall_{f\in C_c(X)_+}\;\sup\Brace{\abs{\Phi(g)}\ge0\mid g\in C_c(X),\abs{g}\le f}<\infty\]
    を満たすものとする.
\end{definition}
\begin{remarks}[複素Radon積分の定義の意味]
    Radon積分は正な線型汎関数$\int:C_c(X)\to\R$とした.正性を複素積分に一般化するのに,少し技術的な困難が伴う.
    これは,Bournologyの定義同様,任意の関数$f\in C_c(X)_+$が指定する$C_c(X)$上の「開円板」を,有界集合に写す,という条件である.
\end{remarks}    

\begin{theorem}[Radon電荷の極分解]
    $X$を$\sigma$-コンパクトな局所コンパクトハウスドルフ空間,$\Phi:C_c(X)\to\C$をRadon電荷とする.
    このとき,Radon積分$\int:C_c(X)\to\R$とBorel関数$u\in\B(X)$が存在して,$\abs{u}=1\;\ae$と$\Phi=\int\cdot u\;\paren{\int\text{-}\ae}$を満たす.
\end{theorem}

\begin{corollary}[Jordan decomposition]
    任意の実数値Radon電荷$\Phi:C_c(X)\to\R$について,2つのRadon積分$\int_+,\int_-$が存在して,互いに素なBorel集合内に台を持ち,$\Phi=\int_+-\int_-$と表せる.
\end{corollary}
\begin{remarks}[Hahn decomposition]
    証明中の分解$X=A_+\cup A_-\cup N$をHahn分解という.
\end{remarks}

\subsection{Radon電荷の空間}

\begin{definition}[total variation]
    $\Phi:C_c(X)\to\C$をRadon電荷とする.
    \begin{enumerate}
        \item $\Phi$の極分解から得られるRadon積分$\int:C_c(X)\to\R$を\textbf{全変動}といい,$\abs{\Phi}$で表す.
        \item Radon電荷$\Phi$が\textbf{有限}であるとは,全変動$\abs{\Phi}$が有限なRadon積分を定めることをいう.有限なRadon電荷全体の空間を$M(X)$で表し,$\norm{\Phi}=\abs{\Phi}(1)<\infty$をノルムとする.
    \end{enumerate}
\end{definition}
\begin{remarks}
    \textbf{任意の複素測度は有界である}\cite{Rudin-RC}:$\exists_{C\in\R}\;\Im\mu\subset C\Delta$.
    が,今回はRadon電荷を特別に定義したので,この事実は自明ではない.
\end{remarks}

\begin{lemma}[全変動の特徴付け]
    全変動$\abs{\Phi}$は,
    \[\forall_{f\in C_c(X)}\;\abs{\Phi(f)}\le\int\abs{f}\]
    を満たすRadon積分$\int$の中で最小のものとして特徴付けられる.
    実電荷$\Phi$について,全変動とは$\abs{\Phi}=\Phi_++\Phi_-$に他ならず,well-definedである.
\end{lemma}

\begin{proposition}[Riesz-Markov representation theorem]\label{prop-Riesz-Markov-3}
    $X$上の有限なRadon電荷の空間$M(X)$は全変動ノルムについてBanach空間をなし,Banach空間$(C_0(X))^*$に等長同型である.
\end{proposition}

\begin{example}\label{exp-Riesz-Markov-theorem-to-sequence-spaces}
    $X=\N$のとき,$M(\N)=l^1$である.すなわち,$(c_0)^*\simeq l^1$.
\end{example}

\subsection{絶対連続な複素Radon積分の空間}

\begin{tcolorbox}[colframe=ForestGreen, colback=ForestGreen!10!white,breakable,colbacktitle=ForestGreen!40!white,coltitle=black,fonttitle=\bfseries\sffamily,
title=]
    $X$上に定まった積分$\int-d\nu:C_c(X)\to\R$に関して,これに絶対連続な全変動測度$\abs{\mu}$を持つ複素測度$\mu$は全て,可積分関数$f\in L^1(X)$を用いた積分$\int$を通じた変換
    \[\int gd\mu=\int gfd\nu\]
\end{tcolorbox}

\begin{proposition}[絶対連続測度の表現]\label{prop-Riesz-4}
    あるRadon積分$\int:C_c(X)\to\R$について,$\abs{\Phi}\ll\int$を満たす有限なRadon電荷$\Phi$全体のなす空間は,次の写像によって$L^1(X)$に等長同型である:
    \[\xymatrix@R-2pc{
        L^1(X)\ar[r]&M(X)\\
        \rotatebox[origin=c]{90}{$\in$}&\rotatebox[origin=c]{90}{$\in$}\\
        f\ar@{|->}[r]&\Phi_f(g):=\int gf\;(f\in\L^1(X),g\in C_c(X))
    }\]
\end{proposition}

\subsection{Lebesgue空間の双対}

\begin{theorem}[双対]\label{thm-duality-of-Lp}
    $X$を$\sigma$-コンパクトな局所コンパクトハウスドルフ空間,$\int:C_c(X)\to\R$をその上のRadon積分とする.
    $p\in\ocinterval{1,\infty},q\in\cointerval{1,\infty},p^{-1}+q^{-1}=1$について,双線型形式
    \[\brac{f,g}=\int fg\;(f\in\L^p(X),g\in\L^q(X))\]
    は等長同型$L^p(X)\simeq (L^q(X))^*$を引き起こす.
\end{theorem}
\begin{remark}
    $p\in(1,\infty)$の場合は$X$が$\sigma$-コンパクトでない場合にも一般化出来る.
\end{remark}

\begin{example}
    $(l^1)^*=l^\infty$である.なお,$l^1=L^1(\N,2^\N,\mu)$で,$\mu$は数え上げ測度である.
\end{example}

\section{積分の積}

\begin{tcolorbox}[colframe=ForestGreen, colback=ForestGreen!10!white,breakable,colbacktitle=ForestGreen!40!white,coltitle=black,fonttitle=\bfseries\sffamily,
title=]
    体$\R$上の各点積を,関数空間上に持ち上げることが出来るのは周知の事実で,これがどのような構造を引き起こすかを考える.
\end{tcolorbox}

\subsection{関数の積の定義と基本性質}

\begin{definition}[product]
    2つの関数$f:X\to\R,g:Y\to\R$について,\textbf{積}$f\otimes g:X\times Y\to\R$を$(f\otimes g)(x,y)=f(x)g(y)$で定める.
\end{definition}

\begin{lemma}\mbox{}
    \begin{enumerate}
        \item $X,Y$を局所コンパクトハウスドルフ空間とする.積空間$X\times Y$も局所コンパクトハウスドルフである.
        \item $X$と$Y$がいずれも$\sigma$-コンパクトであることと,$X\times Y$が$\sigma$-コンパクトであることは同値.
        \item $f\in C_c(X),g\in C_c(Y)\Rightarrow f\otimes g\in C_c(X\times Y)$.
        \item 同様に,$C_0(X)\otimes C_0(Y)\subset C_0(X\times Y)$かつ$C_b(X)\otimes C_b(Y)\subset C_b(X\times Y)$.
    \end{enumerate}
\end{lemma}

\begin{lemma}
    関数$f:X\times Y\to\R$と点$y\in Y$について,
    \begin{enumerate}
        \item $f\in C_c(X\times Y)\Rightarrow f(-,y)\in C_c(X)$.
        \item $f\in C_c(X\times Y)^m\Rightarrow f(-,y)\in C_c(X)^m$.
        \item $f\in\B(X\times Y)\Rightarrow f(-,y)\in\B(X)$.
    \end{enumerate}
\end{lemma}

\subsection{積分の積とFubiniの定理}

\begin{tcolorbox}[colframe=ForestGreen, colback=ForestGreen!10!white,breakable,colbacktitle=ForestGreen!40!white,coltitle=black,fonttitle=\bfseries\sffamily,
title=]
    Fubiniの定理は,積分の積を先に定義すると極めて見通しが良い.
    しかし,逐次積分が計算可能かどうかの判断の前に,積積分についての可積分性の判定が必要であるのが実用性にかけるが,$\sigma$-コンパクトの場合は抜け道がある.
\end{tcolorbox}

\begin{proposition}[product integral]
    $\int_x,\int_y$をそれぞれ局所コンパクトハウスドルフ空間$X,Y$上のRadon積分とする.
    このとき,次の条件を満たすRadon積分$\int_x\otimes\int_y:X\times Y\to\R$がただ一つ存在する:
    \[\forall_{f\in C_c(X),g\in C_c(Y)}\quad\int_x\otimes\int_yf\otimes g=\paren{\int_xf}\paren{\int_yg}.\]
\end{proposition}

\begin{lemma}
    $h\in C_c(X\times Y)$ならば,$x\mapsto\int_yh(x,-)\in C_c(X)$で,$\int_x\int_yh(-,-)=\int_x\otimes\int_yh$.
\end{lemma}

\begin{lemma}
    $h\in C_c(X\times Y)^m$ならば,$x\mapsto\int_y^*h(x,-)\in C_c(X)^m$である.
    また,$h\in\L^1(X\times Y)$ならば,$f\in\L^1(X)$かつ
    \[\int_xf=\int_x\int_y^*h(-,-)=\int_x\otimes\int_yh.\]
\end{lemma}

\begin{theorem}[Fubini]
    $\int_x,\int_y$をそれぞれ局所コンパクトハウスドルフ空間$X,Y$上のRadon積分とする.
    $h\in\B(X\times Y)$で,積$\int_x\otimes\int_y$に関して可積分ならば,Borel関数について$y\mapsto h(x,y)\in\L^1(Y)\;x\text{-}\ae$かつ殆ど至る所定義された関数について$x\mapsto\int_yh(x,-)\in\L^1(X)$かつ
    \[\int_x\int_yh(-,-)=\int_x\otimes\int_yh.\]
\end{theorem}
\begin{remarks}
    特に,$h\in\L^1(X\times Y)$ならば,次の逐次積分は存在しかつ等しい:\[\int_x\int_yh(-,-)=\int_x\otimes\int_yh=\int_y\int_xh(-,-).\]
\end{remarks}

\begin{corollary}[Tonelli]
    Borel関数$h\in\B(X\times Y)$はある$\sigma$-コンパクト集合の外では消えているとする.
    Borel関数が$y\mapsto\abs{h(x,y)}\in\L^1(Y)\;x\text{-}\ae$で,殆ど至る所定義された関数が$x\mapsto\int_y\abs{h(x,-)}\in\L^1(X)$を満たすならば,$h\in\L^1(X\times Y)$で,
    \[\int_x\int_yh(-,-)=\int_x\otimes\int_yh=\int_y\int_xh(-,-).\]
\end{corollary}

\subsection{位相群上の積分}

\begin{tcolorbox}[colframe=ForestGreen, colback=ForestGreen!10!white,breakable,colbacktitle=ForestGreen!40!white,coltitle=black,fonttitle=\bfseries\sffamily,
title=]
    $G$の2つの演算を連続にするような局所コンパクトハウスドルフ位相を備えた群$G$を考える.
\end{tcolorbox}

\begin{definition}[translated function, Haar integarl]
    位相群$G$上の関数$f:G\to\bF(=\R,\C)$について,
    \begin{enumerate}
        \item 左移動$G\times\Map(G,\bF)\to\Map(G,\bF)$を${}_xf(y):=f(x^{-1}y)$,右移動を${}^xf=f(yx)$で定める.このとき,${}_{xy}f={}_x({}_yf)$かつ${}^{xy}f={}^x({}^yf)$が成り立つ.
        \item $G$上の左\textbf{Haar積分}とは,零でないRadon積分であって,左移動不変であるものをいう:$\forall_{x\in G}\;\forall_{f\in C_c(G)}\;\int{}_xf=\int f$.
        \item このとき,移動不変性は実際は任意の$\L^1(G)$について成り立つ.特に,任意の$x\in G$とBorel集合$A\in\B_G$について,$\int[xA]=\int[A]$.
        \item 任意の非空な開集合$A$について,$\int[A]>0$である.
    \end{enumerate}
\end{definition}

\begin{lemma}
    $f\in C_0(G)$ならば,$x\mapsto {}_xf$と$x\mapsto{}^xf$によって定まる2つの写像$G\to C_0(G)$はいずれも一様連続である.
\end{lemma}

\begin{lemma}
    $G$の任意のRadon積分$\int:C_c(G)\to\R$と実数$p\in\cointerval{1,\infty}$について,任意の$f\in C_c(G)$について,$x\mapsto{}_xf,x\mapsto{}^xf$によって定まる写像$G\to\L^p(G)$は一様連続である.
\end{lemma}

\begin{theorem}[Haar]
    局所コンパクトな位相群$G$において,左Haar積分は正のスカラー因数の別を除いて一意に定まる.
\end{theorem}
\begin{remark}
    全く対称的な議論により,右Haar積分も一意的に存在するが,左Haar積分と一致するとは限らない.
    \[G:=\Brace{\begin{pmatrix}a&b\\0&1\end{pmatrix}\in M_2(\R)\;\middle|\;a>0,b\in\R}\]
    と定めると,これは$\R$のaffine同型群で,それぞれの行列は変換$x(t)=at+b$に対応する.
    \[\int_lf=\iint f(a,b)a^{-2}dadb,\quad\int_rf=\iint f(a,b)a^{-1}dadb\]
    と定める.ただし,右辺は$\R_+\times\R$上のLebesgue積分で,$f\in C_c(G)$とした.
\end{remark}

\subsection{調和解析}

\begin{notation}
    位相群$G$上のHaar積分を一つ固定し,これを$\int f(x)dx$で表す.これは,調和解析特有の頻繁な変数変換を視覚的に表すためである.
\end{notation}

\begin{definition}[modular function, unimodular]
    任意の$x\in G$について,Radon積分$C_c(G)\to\R;f\mapsto\int f(yx)dy$は左不変だから,一意な実数$\Delta(x)>0$であって次を満たすものが定まる:$\forall_{f\in\L^1(G)}\;\Delta(x)\int f(yx)dy=\int f(y)dy$.
    \begin{enumerate}
        \item 関数$\Delta:G\to(0,\infty)$を\textbf{モジュラー関数}または\textbf{母数}という.
        \item $\Delta=1$であるとき,群$G$を\textbf{ユニモジュラー}または\textbf{単模}という.これは左右のHaar積分が一致することに同値.
    したがってモジュラー関数は左右のズレを測っているとも考えられる.
    \item $d(yx)=\Delta(x)dy$が成り立つ.
    \end{enumerate}
\end{definition}
\begin{remark}
    有限次元の場合,行列式が単元になるものをユニモジュラー行列という.
\end{remark}

\begin{proposition}
    モジュラー関数$\Delta:G\to(0,\infty)$は,$(0,\infty)=\exp(\R)$を乗法群とみると,連続な群順同型を定める.
    $G$がAbelである,または離散である,またはコンパクトであるならば,$G$は単模である.
\end{proposition}

\subsection{対合代数}

\begin{lemma}
    \[\forall_{f\in C_c(G)}\quad\int f(x^{-1})\Delta(x)^{-1}dx=\int f(x)dx.\]
\end{lemma}

\begin{discussion}
    $\check{f}(x):=f(x^{-1})$とすると,$f\mapsto\int\check{f}(x)dx$は右不変である.
    この間の関係は
    \[\int\check{f}(x)dx=\int f(x^{-1})\Delta(x)\Delta(x)^{-1}dx=\int f(x)\Delta(x)^{-1}dx.\]
    さらに,$f^*(x):=\o{f(x^{-1})}\Delta(x)^{-1}$と定めると,
    \[\forall_{f\in C_c(G)}\quad\norm{f^*}_1=\int\abs{f(x^{-1})}\Delta(x)^{-1}dx=\int\abs{f(x)}dx=\norm{f}\]
    で,結局${}^*$はBanach空間$L^1(G)$上に等長な対合を定める.
\end{discussion}

\begin{proposition}
    $\forall_{1\le p<\infty}$について,$f\in\L^p(G)$ならば,$x\mapsto{}_xf,x\mapsto{}^xf$は一様連続写像$G\to L^p(G)$を定める.
\end{proposition}

\begin{proposition}
    $\forall_{1\le p<\infty}$について,$f\in\L^1(G),g\in\L^p(G)$をそれぞれBorel関数とすると,$y\mapsto f(y)g(y^{-1}x)\in\L^1(G)\;x\text{-}\ae$で,殆ど至る所定義された関数について$x\mapsto\int f(y)g(y^{-1}x)dy\in\L^p(G)$で,
    \[\Norm{\int f(y)g(y^{-1}\cdot)dy}_p\le\norm{f}_1\norm{g}_p.\]
\end{proposition}

\begin{theorem}
    局所コンパクト群$G$とそのHaar積分$\int$について,空間$L^1(G)$は等長対合を備えたBanach代数である.ただし積と対合は次のように定める:
    \[(f\times g)(x)=\int f(y)g(y^{-1}x)dy,\quad f^*(x)=\o{f(x^{-1})}\Delta(x)^{-1}.\]
\end{theorem}

\subsection{対合代数の表現}

\begin{proposition}
    等長同型$\Phi:L^1(X)\mono M(X)$は,$L^1(G)$をある$M(G)$の$*$-不変な閉イデアル上に移す対合代数の$*$-同型である.
\end{proposition}

\section{Hausdorff空間上のRadon測度}

\begin{tcolorbox}[colframe=ForestGreen, colback=ForestGreen!10!white,breakable,colbacktitle=ForestGreen!40!white,coltitle=black,fonttitle=\bfseries\sffamily,
title=]
    確率論的に自然な設定は完備可分距離空間である.というのは,無限次元線型空間は局所コンパクトたり得ないためである.
    実は局所コンパクト性の仮定は本質的でない.Schwartz, L. (1973), Kisynski, Topsoe (1970)によりHausdorff空間上でのRadon積分論が展開された.
    対応するRiesz表現定理は,より一般な積分表現定理となる(Pollard and Topsoe, 1975).
\end{tcolorbox}

\begin{notation}
    $X$を任意のHausdorff空間,$\o{M(X)}_+$をRadon測度の空間,$M(X)$を符号付きHausdorff測度の空間とする.
    有限Radon測度は外部正則であることに注意.
\end{notation}

\subsection{延長定理}

\begin{definition}[Radon content]
    関数$\lambda:\cC\to\R_+$が\textbf{Radon内容}であるとは,次を満たすことをいう:
    \[\forall_{C_1,C_2\in\cC}\;C_1\subset C_2\Rightarrow\lambda(C_2)-\lambda(C_1)=\sup_{\cC\ni C\subset C_2\setminus C_1}\lambda(C)\]
\end{definition}

\begin{theorem}
    Hausdorff空間上のRadon内容は,Radon測度への一意な延長を持つ.
\end{theorem}

\begin{theorem}[Radon測度に関する単調収束定理]
    $X$をHausdorff空間,$\mu\in \o{M(X)}_+$をRadon測度とする.次が成り立つ:
    \begin{enumerate}
        \item $\{G_\al\}\subset\O_X$を単調増加ネット,$G:=\cup_{\al\in A}G_\al$とする.このとき,
        \[\mu(G)=\sup_{\al\in A}\mu(G_\al)=\lim_{\al\in A}\mu(G_\al).\]
        \item $\{f_\al\}\subset C^{1/2}(X;\o{\R_+})$を下半連続関数の単調増加ネット,$f:=\sup_{\al\in A}f_\al$とする.このとき,
        \[\int fd\mu=\sup_{\al\in A}\int f_\al d\mu=\lim_{\al\in A}\int f_\al d\mu.\]
    \end{enumerate}
\end{theorem}

\begin{definition}[$\tau$-smooth meature]
    条件(1)を満たすBorel測度を一般に,$\tau$-なめらかであるという.
    一般の$\tau$-なめらかな測度について(2)が成り立つ.
    $X$が局所コンパクトでBorel測度が有限のとき,Radon測度と$\tau$-滑らかな測度の概念は一致する.
\end{definition}

\subsection{変数変換}

\begin{proposition}
    $\mu\in \o{M(X)}_+$をRadon測度とする.
    \begin{enumerate}
        \item $f:X\to\o{\R_+}$がBorel可測ならば,
        \[\int_Xfd\mu=\sup_{C\in\cC}\int_Cfd\mu.\]
        \item $f:X\to\R_+$が連続ならば,
        \[\nu(B):=\int_Bfd\mu\quad(B\in\B(X))\]
        は再びRadon測度である.これを$d\nu=fd\mu$で表す.
    \end{enumerate}
\end{proposition}

\begin{definition}[molecular measure]
    \[\supp(\mu):=\Brace{x\in X\mid\forall_{U\in\O(x)}\;\mu(U)>0}\]
    を分布の台とし,分布の台が有限集合であるRadon測度を\textbf{分子測度}といい,$\Mol_+(X)\subset\o{M_+(X)}$で表す.
\end{definition}

\subsection{積測度}

\begin{tcolorbox}[colframe=ForestGreen, colback=ForestGreen!10!white,breakable,colbacktitle=ForestGreen!40!white,coltitle=black,fonttitle=\bfseries\sffamily,
title=]
    積$\sigma$-代数も積位相も射影が誘導するが,$\B(X)\otimes\B(Y)\subset\B(X\times Y)$が従ってしまう.
\end{tcolorbox}

\begin{definition}[bimeausre]\mbox{}
    \begin{enumerate}
        \item $\Phi:\A\times\B\to\o{\R_+}$が,各引数毎に測度を定めるとき,\textbf{双測度}であるという.
        \item 双測度$\Phi$がRadonであるとは,コンパクト集合$K\compsub A,L\compsub B$について$\Phi(K,L)<\infty$で,次を満たすことをいう:
        \[\Phi(A,B)=\sup\Brace{\Phi(K,L)\in\R_+\mid K\compsub A,L\compsub B}.\]
    \end{enumerate}
\end{definition}

\begin{theorem}
    $X,Y$をHausdorff空間,$\Phi:\B(X)\otimes\B(Y)\to\o{\R_+}$をRadon双測度とする.このとき,一意なRadon測度$\kappa\in\o{M_+(X\times Y)}$が存在して,$\forall_{K\in\cC(X),L\in\cC(Y)}\;\Phi(K,L)=\kappa(K\times L)$を満たす.
    さらに,この等式は任意のBorel集合$A\in\B(X),B\in\B(Y)$についても成り立つ:$\Phi(A,B)=\kappa(A\times B)$.
\end{theorem}

\begin{corollary}
    $\mu\in\o{M_+(X)},\nu\in\o{M_+(Y)}$について,唯一のRadon測度$\mu\otimes\nu\in\o{M_+(X\times Y)}$が存在して,
    \[\forall_{K\in\cC(X),L\in\cC(Y)}\;\mu\otimes\nu(K\times L)=\mu(K)\nu(L)\]
    を満たす.
\end{corollary}

\begin{theorem}[Fubini]
    $\mu\in\o{M_+(X)},\nu\in\o{M_+(Y)}$,$f:X\times Y\to\o{\R_+}$を下半連続とする.このとき,
    \begin{enumerate}
        \item $x\mapsto\int f(x,y)d\nu(y),y\mapsto\int f(x,y)d\mu(x)$も下半連続である.
        \item 次の等式が成り立つ:
        \[\int_{X\times Y}df(\mu\otimes\nu)=\int_X\int_Yf(x,y)d\nu(y)d\mu(x)=\int_Y\int_Xf(x,y)d\mu(x)d\nu(y).\]
    \end{enumerate}
\end{theorem}

\subsection{像測度と畳み込み}

\begin{proposition}
    $X,Y$をHausdorff空間,$\mu\in\o{M_+(X)}$をRadon測度,$f:X\to Y$を連続とする.
    \begin{enumerate}
        \item $\forall_{K\in\cC(Y)}\;\mu(f^{-1}(K))<\infty$ならば,像測度$\mu^f$は再びRadonである.
        \item $\mu$が有限であるか,$f$が固有写像ならば,(1)の要件は満たされる.
    \end{enumerate}
\end{proposition}

\begin{definition}
    $(S,+)$をHausdorffな位相半群,$\mu,\nu\in M_+(S)$を有限Radon測度とする.このとき,加法$+:S\times S\to S$による像測度$\mu*\nu:=(\mu\otimes\nu)^+$を\textbf{押し出し}という.
\end{definition}

\subsection{測度の貼り合わせ}

\begin{theorem}[貼り合わせの補題]
    $(G_\al)_{\al\in D}$をHausdorff空間$X$の開被覆,$\mu_\al\in\o{M_+(G_\al)}$をRadon測度の族であって各$G_\al\cap G_\beta$上で一致するものとする.このとき,貼り合わせ$\mu\in\o{M_+(X)}$が一意に存在する.
\end{theorem}

\subsection{有限Radon測度の弱収束}

\begin{definition}
    $X$をHausdorff空間,$M_+(X)$を有限Radon測度の空間とする.$M_+(X)$上の\textbf{弱位相}とは,有界な下半連続関数の族$C^{1/2}_b(X)$が定める始位相である.
    \[G_{f,t}:=\Brace{\mu\in M_+(X)\;\middle|\;\int fd\mu>0}\quad(f\in C_b^{1/2}(X),t\in\R)\]
    は準基となる.
\end{definition}

\begin{theorem}[Portmanteau (Topsoe 1970, Th'm 8.1)]
    任意の$mu\in M_+(X)$とネット$\{\mu_\al\}\subset M_+(X)$について,次の5条件は同値:
    \begin{enumerate}
        \item $\mu_\al\to\mu$.
        \item $\forall_{G\in\O_X}\;\liminf\mu_\al(G)\ge\mu(G)$かつ$\mu_\al(X)=\mu(X)$.
        \item $\forall_{F\csub X}\;\limsup\mu_\al(F)\le\mu(F)$かつ$\mu_\al(X)=\mu(X)$.
        \item $\forall_{f\in C^{1/2}_b(X)}\;\liminf\int fd\mu_\al\ge\int fd\mu$.
        \item $\forall_{f\in C^{-1/2}_b(X)}\;\limsup\int fd\mu_\al\le\int fd\mu$.
    \end{enumerate}
    この条件が成り立つとき,$\forall_{f\in C_b(X)}\;\lim\int fd\mu_\al=\int fd\mu$である.また,$X$が完全正則空間ならば,逆も成り立つ.
\end{theorem}

\begin{proposition}
    $M_+(X)$はHausdorff空間である.
\end{proposition}

\begin{theorem}
    $X,Y$をHausdorff空間とする.写像$\otimes:M_+(X)\times M_+(Y)\to M_+(X\times Y);(\mu,\nu)\mapsto\mu\otimes\nu$は弱連続である.
\end{theorem}

\begin{corollary}
    $S$がHausdorff位相半群ならば,$M_+(S)$も畳み込みについてHausdorff位相半群をなす.
\end{corollary}

\subsection{有限Radon測度の分子測度近似}

\begin{tcolorbox}[colframe=ForestGreen, colback=ForestGreen!10!white,breakable,colbacktitle=ForestGreen!40!white,coltitle=black,fonttitle=\bfseries\sffamily,
title=]
    Lebesgue可積分関数は単関数近似する.有限Radon測度は分子測度で近似する.
    これはKrein-Milmanの定理でもある.
\end{tcolorbox}

\begin{proposition}
    任意のHausdorff空間$X$について,その分子測度の空間$\Mol_+(X)$は$M_+(X)$上,$\B(X)$上の各点収束位相について稠密である.
\end{proposition}

\subsection{漠位相}

\begin{definition}
    位相$\sigma(M(X),C_c(X))$を\textbf{漠位相}という.
\end{definition}

\begin{definition}
    任意のネット$\{\mu_\al\}\subset M_+(X)$と$\mu\in M_+(X)$について,次の2条件は同値:
    \begin{enumerate}
        \item $\mu_\al\wto\mu$.
        \item $\mu_\al\vto\mu$かつ$\lim\mu_\al(X)=\mu(X)$.
    \end{enumerate}
    すなわち,弱位相と漠位相は$P(X)$上では一致する.
\end{definition}

\begin{theorem}
    部分集合$M\subset \o{M_+(X)}$について,次の2要件は同値:
    \begin{enumerate}
        \item $M$は漠位相について相対コンパクト.
        \item $\forall_{\varphi\in C_c(X)_+}\;\sup_{\mu\in M}\brac{\mu,\varphi}<\infty$.
    \end{enumerate}
\end{theorem}

\begin{corollary}\mbox{}
    \begin{enumerate}
        \item $\Brace{\mu\in M_+(X)\mid\mu(X)\le c}\;(c>0)$は漠位相についてコンパクトである.
        \item $X$がコンパクトなとき,$P(X)$は漠位相についてコンパクトである.
    \end{enumerate}
\end{corollary}

\begin{proposition}
    $X$が局所コンパクトかつ第1可算ならば,$\o{M_+(X)}$の漠位相は距離化可能である.
\end{proposition}

\subsection{Choquetの積分表現論}

\begin{tcolorbox}[colframe=ForestGreen, colback=ForestGreen!10!white,breakable,colbacktitle=ForestGreen!40!white,coltitle=black,fonttitle=\bfseries\sffamily,
title=]
    確率測度の全体$P(X)$は$M(X)$のコンパクト集合であった.
    Choquet理論は,任意の$C\compsub E$の元を$C$の極点に台を持つ測度で表す理論である.
    これはKrein-Milmanの定理を含意する.
\end{tcolorbox}

\begin{notation}
    $E$を局所凸位相$\R$-線型空間とする.
\end{notation}

\begin{definition}[barycentre / center of gravity / resultant]
    $X\compsub E,\mu\in P(X)$とする.$b\in E$が\textbf{$\mu$-重心}であるとは,
    \[f(b)=\int_Xfd\mu\quad(f\in E^*)\]
    を満たすことをいう.
\end{definition}

\begin{proposition}
    $X\compsub E$は$K=\o{\Conv}(X)$もコンパクトであるとする.このとき,
    \begin{enumerate}
        \item 任意の$\mu\in P(X)$に対して,その重心は存在して$K$内に収まる.
        \item 任意の点$x\in K$はある確率測度$\mu\in P(X)$の重心である.
    \end{enumerate}
\end{proposition}
\begin{remarks}
    $E$が完備ならば,$X$がコンパクトであるとき$K$もコンパクトになる.
\end{remarks}

\begin{theorem}[Krein-Milmanの定理の換言]
    $K\compsub E$をコンパクト凸とする.任意の点$x\in K$は確率測度$\mu\in P(\o{\Ex(K)})$の重心である.
\end{theorem}
\begin{remarks}
    $K$が距離化可能であるとき,さらに$\mu$が台を$\Ex(K)$上に持つように出来ることをいう.
    Choquet-Bishop-de Leeuw定理は,一般の$K$についても実用的な解決を与える.
\end{remarks}

\chapter{Fréchet空間}

\begin{quotation}
    関数解析の主な対象は線型写像であったが,非線形な関数解析に必要な線型化の理論,すなわち微分理論を考える.
    非線型汎函数の最適化理論である変分法は,遥か昔からの基本的問題であった.
    Specifically, it studies the critical points , i.e. the points where the first variational derivative of a functional vanishes, for functionals on spaces of sections of jet bundles. The kinds of equations specifying these critical points are Euler-Lagrange equations.\footnote{\url{https://ncatlab.org/nlab/show/variational+calculus}}
\end{quotation}

\section{Fréchet空間の3つの定義}

\subsection{定義1と特徴付け}

\begin{tcolorbox}[colframe=ForestGreen, colback=ForestGreen!10!white,breakable,colbacktitle=ForestGreen!40!white,coltitle=black,fonttitle=\bfseries\sffamily,
title=局所凸な$F$-空間]
    Fréchet空間とは,斉次性を緩めたノルムを備えた空間($F$-ノルム空間という)で,そのノルム位相について完備であるものをいう.
    なお,セミノルム空間はHausdorffでないから,「完備なセミノルム空間」という概念は筋が悪い.が,セミノルム可算族が定めるセミノルム位相はHausdorff足り得る\ref{def-seminorm-topology}.
    そして,セミノルム可算族が定める位相について完備な空間はFréchet空間の特徴付けとなる.
    さらに,Fréchet空間とは,距離化可能な完備なHausdorff局所凸位相線形空間なるクラスに一致することもわかる(距離がノルムから来ているかは不問).なお,このとき距離関数は平行移動不変性を持つように選べる.
    こうして,Banach空間の一般化として極めて安定な概念が見つかった.
    Banach空間の性質の大部分はFréchet空間に一般化出来る.
\end{tcolorbox}

\begin{definition}[$F$-norm, $F$-space, Frechet space]
    $V$を$K$-線型空間とする.$\norm{-}:V\to K$を関数とする.
    \begin{enumerate}
        \item 次の3条件を満たすとき,$\norm{-}$を\textbf{$G$-[セミ]ノルム}という.\footnote{ノルム空間のunderlying spaceをアーベル群について一般化した時に適切となるノルム概念である.}
        \begin{enumerate}[(a)]
            \item 正:$\norm{0_V}=0$.$[\norm{x}=0\Rightarrow x=0_V]$
            \item \textbf{$-1$の作用}:$\forall_{x\in V}\;\norm{-x}=\norm{x}$.
            \item 三角不等式:$\forall_{x,y\in V}\;\norm{x+y}\le\norm{x}+\norm{y}$.
        \end{enumerate}
        \item $G$-[セミ]ノルムが$V$に定める擬距離が定める位相について,スカラー乗法$K\times V\to V;(a,x)\mapsto ax$が連続であるとき,これを\textbf{$F$-[セミ]ノルム}という.
        \item そのノルム位相が完備な$F$-ノルム空間を\textbf{$F$-空間}という:すなわち,任意のCauchyネットが収束する$F$-ノルム空間を$F$-空間という.
        \item $F$-空間のノルム位相が局所凸であるとき,これを\textbf{Fréchet空間}という.射を連続線型写像とすると,Fréchet空間は圏TVSの充満部分圏をなす.
    \end{enumerate}
\end{definition}

\begin{lemma}[ノルムの$F$-性は距離でいうと平行移動不変性]\mbox{}
    \begin{enumerate}
        \item $F$-ノルムの定める距離$d:V\times V\to \R,d(x,y):=\norm{x-y}$は,平行移動不変性を満たす:$\forall_{x,y,a\in V}\;d(x+a,y+a)=d(x,y)$.
        \item 逆に,平行移動不変な距離$d:V\times V\to \R$について,$\norm{x-y}:=d(x-y,0)$と定めると,これは$F$-ノルムである.
    \end{enumerate}
\end{lemma}
\begin{Proof}\mbox{}
    \begin{enumerate}
        \item 明らか:$d(x+a,y+a)=\norm{(x+a)-(y+a)}=d(x,y)$.
        \item 正であることは明らか.$-1$による作用について,$\norm{-x}=d(-x,0)=d(0,x)=d(x,0)=\norm{x}$.
        三角不等式は
        \[\norm{x+y}=d(x,-y)\le d(x,0)+d(0,-y)=\norm{x}+\norm{y}.\]
    \end{enumerate}
\end{Proof}
\begin{remarks}[距離の言葉による特徴付け]
    $F$-空間を距離の言葉で定義すると,平行移動不変距離$d:V\times V\to \R$について完備な位相線形空間を定めること($d$が定める位相についてスカラー乗法と加法が連続であること)をいう.
    さらにこれが局所凸であるとき,Frechet空間という.
\end{remarks}

\subsection{例}

\begin{example}[連続関数の空間]
    $\Om\osub\R^n$上の連続関数の空間$C(\Om)$はFrechet空間となる.
    $C(\Om)$は局所有界でないので,ノルム付け可能ではない\ref{thm-character-of-TVS}.
\end{example}

\begin{example}[$p<1$の場合のLebesgue空間]\mbox{}
    \begin{enumerate}
        \item Lebesgue空間$l^p\;(0<p<1)$はFrechet空間ではないが,$F$-空間である.なお,この空間の$p$-ノルムは,三角不等式を成り立たせるために$p$乗根を省いて,$\norm{x}_p=\sum_{i}\abs{x_i}^p$とする.
        \item Lebesgue空間$L^p(X)$も,$p<1$のときFréchet空間でない局所凸線型空間となる.
    \end{enumerate}
\end{example}

\begin{example}[滑らかな関数の空間]\mbox{}
    \begin{enumerate}
        \item コンパクトな可微分多様体$X$上の可微分写像の空間$C^\infty(X)$は,半ノルムの族$p^\al_K:C_c^\infty(\R^n)\to\R_+;\Phi\mapsto\sup_{x\in K}\abs{\partial^\al\Phi(x)}\;(K\compsub\R^n,\al\in\N^n)$についてFréchet空間である.Schwartzはこれを$\mathcal{E}(\R^n)$で表した.
        \item コンパクトでない可微分多様体でも,$C^\infty(\R)$などはFréchet空間となる.
    \end{enumerate}
\end{example}

\begin{example}[Hardy space]
    $\Om\osub\C$上の正則関数の空間$H(\Om)\subset C(\Om)$はFrechet空間の閉部分空間をなすから,再びFrechet空間である.
    $H(\Om)$はHeine-Borel性を持つ.局所有界でないので,ノルム付け可能ではない\ref{thm-character-of-TVS}.
\end{example}

\begin{example}[Euclid空間の延長線上としてのFrechet空間]
    射影極限$\R^\infty:=\varprojlim_n\R^n$(位相線型空間としての直積として得られる空間)はFréchet空間となる.つまり,射影$p^n:\R^\infty\to\R^n$と$\R^n$上のノルム$\R^n\to\R$との合成の族$(p^n\circ\norm{-}_n)_{n\in\N}$はセミノルムの族になるが,これが定める位相について完備である.
\end{example}

\begin{example}[the Schwartz space]
    急減少関数の空間$\cS$はFréchet空間である.
\end{example}

\subsection{定義2とBanach空間との関係}

\begin{definition}[ゲージ空間としての捉え方]
    $V$の点を分離するセミノルムの列$\F=(\norm{-}_n)_{n\in\N}$が線型空間$V$に定めるセミノルム位相$\O$を考える.
    \begin{enumerate}
        \item $\O$はHausdorffになり\ref{def-seminorm-topology},
        \item さらに$(V,\O)$は局所凸である\ref{prop-characterization-of-locally-convex-spaces}.
        \item $d(x,y):=\sup_{n\in\N}\norm{x-y}_n=0$とすると,これはこのセミノルム位相$\O$を定める距離であり,平行移動について不変である.すなわち,$(V,\O)$はFrechet空間である.
    \end{enumerate}
\end{definition}
\begin{Proof}
    (3)を示す.
    \begin{enumerate}[(a)]
        \item $d$は非負関数であり,$d(x,y)=0$のとき$\forall_{n\in\N}\;\norm{x-y}=0$である.$\F$は$V$の点を分離することから,これは$x=y$を意味する.
        \item 対称律は明らか.
        \item \begin{align*}
            d(x,y)=\sup_{n\in\N}\norm{x-y}_n&\le\sup_{n\in\N}\paren{\norm{x-z}_n+\norm{z-y}_n}\\
            &\le d(x,z)+d(z,y).
        \end{align*}
        \item 平行移動不変性も明らか.
    \end{enumerate}
\end{Proof}

\begin{proposition}[Banach空間との関係]
    Fréchet空間$X$について,次の2条件は同値.
    \begin{enumerate}
        \item $X^*$はFréchet空間である.
        \item $X$はBanach空間である.
    \end{enumerate}
\end{proposition}

\subsection{定義3と総覧}

\begin{tcolorbox}[colframe=ForestGreen, colback=ForestGreen!10!white,breakable,colbacktitle=ForestGreen!40!white,coltitle=black,fonttitle=\bfseries\sffamily,
title=]
    実は,Frechet空間の定義は次の形にまで簡潔になる.\footnote{\url{http://www2.math.uni-wuppertal.de/~vogt/vorlesungen/fs.pdf}}
    以上の議論をさらに洗練させると,Frechet空間の同値な4つの定義は,次の補題の形にまとまる.
    この消息は,Hahn-Banachの分離定理\ref{sub-Hahn-Banach-separation}で見え隠れした,Minkowski汎関数の言葉を通じて,半ノルムと絶対凸集合との対応\ref{remarks-seminorm-and-absolutely-convex-sets}による.
\end{tcolorbox}

\begin{definition}
    完備な局所凸線型空間であって,距離化可能であるものを\textbf{Frechet空間}という.
\end{definition}

\begin{lemma}[Frechet空間の特徴付け]
    局所凸空間$E$について,次の4条件は同値.
    \begin{enumerate}
        \item $E$は距離化可能である.
        \item $E$には,$0$の開近傍の基底であって,可算なものが存在する.
        \item $E$は可算なセミノルムの基本系が存在する.
        \item $E$の位相は,ある平行移動不変な距離によって与えられる.
    \end{enumerate}
\end{lemma}

\section{Fréchet空間の性質}

\subsection{Banach空間論と並行な議論}

\begin{proposition}
    $E$をFrechet空間,$F\subset E$を閉部分空間とする.このとき,$F,E/F$はいずれもFrechet空間である.
\end{proposition}

\begin{theorem}[開写像定理]
    $E,F$をFrechet空間,$A:E\to F$を全射な連続線型作用素とする.このとき,$A$は開写像である.
\end{theorem}

\begin{corollary}[逆写像定理]
    $E,F$をFrechet空間,$A:E\to F$を全単射な連続線型作用素とする.このとき,線型写像$A^{-1}$も連続である.
\end{corollary}

\begin{theorem}[閉グラフ定理]
    $E,F$をFrechet空間,$A:E\to F$を線型作用素とする.グラフ$G:=\Brace{(x,Ax)\in E\times F\mid x\in E}$が閉集合であるならば,$A$は連続である.
\end{theorem}

\subsection{樽型空間と一様有界性の原理}

\begin{tcolorbox}[colframe=ForestGreen, colback=ForestGreen!10!white,breakable,colbacktitle=ForestGreen!40!white,coltitle=black,fonttitle=\bfseries\sffamily,
title=]
    位相線形空間のうち,樽型空間と呼ばれるクラスについては,一様有界性の原理(Banach-Steinhaus)が成り立つ.
    Frechet空間はこれに当てはまる.
    Bourbaki 1950考案.
\end{tcolorbox}


\begin{lemma}\mbox{}
    \begin{enumerate}
        \item 任意の位相線形空間において,$0$の近傍は併呑である.
        \item 併呑集合の有限共通部分は併呑である.
        \item 樽は任意の有界完備凸部分空間を併呑する.
    \end{enumerate}
\end{lemma}

\begin{lemma}[併呑性の特徴付け]\label{lemma-characterizing-absorbant}
    $A\subset E$を絶対凸集合とする.このとき,次の2条件は同値.
    \begin{enumerate}
        \item $A$の定める計量関数$m_A$について$m_A(x)<\infty$.
        \item $A$は$x$を併呑する:$x\in\Span\{A\}:=\cup_{t>0}tA$.
    \end{enumerate}
\end{lemma}

\begin{notation}
    $A$を絶対凸集合とし,$A=m_A^{-1}([0,1])$と表せるとする.このとき,$m_A$は$X$上のノルムである.
    $A$の定める完備線型空間を
    \[E_A:=\o{\paren{\Span(A)/\Ker m_A,m_A}}\]
    で表す.実際,$\forall_{y\in\Ker m_A}\;m_A(x+y)=m_A(x)$を満たすから,$m_A$は$\Span(A)/\Ker m_A$上にノルムを定める.
\end{notation}

\begin{proposition}
    任意のFrechet空間$E$はbarrelledである.
\end{proposition}
\begin{Proof}
    $M$を絶対凸な吸収的閉集合とする:$E=\cup_{t>0}tM$.このとき,均衡性より$E=\cup_{n\in\N}nM$も成り立つ.
    すると,各$nM$も閉であるから,
    Baireのカテゴリー定理より,ある$n_0$が存在して$n_0M$は内点を持つ.すると均衡性より$M$も内点を持つ.
    $E=\cup_{t>0}tM$と併せると,$0$は$M$の内点である.
\end{Proof}

\section{双対空間}

\subsection{有界型空間}

\begin{tcolorbox}[colframe=ForestGreen, colback=ForestGreen!10!white,breakable,colbacktitle=ForestGreen!40!white,coltitle=black,fonttitle=\bfseries\sffamily,
title=]
    topologyとは開集合系を指定した空間であった.有界となるべき集合を指定することで得られる構造を,bornologyという.
    位相空間の射は$f^*$が開集合系を保つものとしたが,有界型空間は$f_*$が有界系を保つものとした.
    命名はBourbaki (1968)による.
\end{tcolorbox}

\begin{definition}[bornology, bornological set]
    $X$を集合とする.$X$上の\textbf{有界集合系}とは,$\B\subset P(X)$であって次の3条件を満たすものとする:
    \begin{enumerate}
        \item $\cup\B=X$.
        \item 下方閉:$\forall_{B\in\B}\;A\subset B\Rightarrow A\in\B$.
        \item 有限合併閉性:$\forall_{B_1,\cdots,B_n\in\B}\;\cup_{i\in[n]}B_i\in\B$.
    \end{enumerate}
    組$(X,\B)$を\textbf{有界型空間}または\textbf{界相空間}という.
\end{definition}

\begin{example}\mbox{}
    \begin{enumerate}
        \item 任意の$T_1$空間に対して,相対コンパクトな集合の全体は有界集合系をなす.
        \item 任意の測度空間に関して,測度確定な集合の全体は有界集合系をなす.
        \item 任意のBanach空間の帰納極限は有界型な線形空間である.
    \end{enumerate}
\end{example}

\begin{definition}[bounded]\label{def-bounded-function}
    写像$f:X\to Y$が\textbf{有界写像}であるとは,$f_*(\B_X)\subset\B_Y$を満たすことをいう.
\end{definition}

\begin{example}\mbox{}
    \begin{enumerate}
        \item 任意の距離空間において,$\exists_{r\in\R_+}\;A\subset rB$を$A\in\B$たる条件とすると,これは有界集合系をなし,任意のLipschitz写像は有界写像になる.
        \item 任意の有界型空間において,線型作用素が連続であることと有界であることとは同値.
    \end{enumerate}
\end{example}

\begin{proposition}[bornological isomorphism]
    有界写像$f$が全単射ならば逆写像も有界である.これを\textbf{界相同型}という.
\end{proposition}

\begin{theorem}
    局所凸位相線形空間$X,Y$と線形写像$u:X\to Y$に対して,以下の3条件は同値:
    \begin{enumerate}
        \item $u$は有界.
        \item $u$の有界円板を有界円板に写す.
        \item $Y$の任意の併呑円板$D$に対して$u^{-1}(D)$は併呑.
    \end{enumerate}
\end{theorem}

\subsection{Alaoglu-Bourbakiの定理}

\begin{theorem}
    $E$を距離化可能な局所凸空間とする.このとき,双対空間$E^*$は完備である.
\end{theorem}

\begin{corollary}
    $E$を距離化可能な局所凸空間とする.
    \begin{enumerate}
        \item $E^*$は距離化可能である,すなわち再びFrechet空間である.
        \item $E$はノルム空間である,すなわちBanach空間である.
    \end{enumerate}
\end{corollary}

\begin{theorem}[Alaoglu-Bourbaki]
    $E$を局所凸,$U\subset E$を$0$の開近傍とする.極集合$U^\circ$は$\sigma(E^*,E)$-コンパクトである.
\end{theorem}

\subsection{回帰性}

\begin{tcolorbox}[colframe=ForestGreen, colback=ForestGreen!10!white,breakable,colbacktitle=ForestGreen!40!white,coltitle=black,fonttitle=\bfseries\sffamily,
title=]
    樽型同様,有界型空間というのも,Bourbakiによる.
    ある種の位相線形空間は,開集合の系を考えるのが有効であるのと同様に,有界集合の系を考えることも有効になる.
\end{tcolorbox}

\begin{theorem}
    $E$をFrechet空間とする.$M\subset E^*$について,次の4条件は同値.
    \begin{enumerate}
        \item $M$は$w^*$-有界である.
        \item $M$は$w^*$-相対コンパクトである.
        \item $M$は有界である.
        \item $\exists_{k\in\N}\;\exists_{C>0}\;M\subset C^\circ_k$.
    \end{enumerate}
    $M=E^*$がこの同値な条件を満たす時,$E$は\textbf{distinguished}という.
\end{theorem}

\begin{definition}[bornological space]
    $E$を局所凸位相線形空間とする.
    \begin{enumerate}
        \item 部分集合$M\subset E$が\textbf{有界型}であるとは,任意の有界集合$B\subset E$について$t>0$が存在して$B\subset tM$を満たすことをいう.
        \item 任意の絶対凸な有界型集合が$0$の近傍であるとき,$E$を\textbf{有界型空間}または\textbf{界相空間}という.
    \end{enumerate}
\end{definition}

\begin{theorem}
    Frechet空間$E$について,次の2条件は同値.
    \begin{enumerate}
        \item $E$は回帰的である.
        \item 任意の$E$の有界集合は,弱相対コンパクトである.
    \end{enumerate}
\end{theorem}

\subsection{Schwartz空間}

\begin{definition}[precompact]
    $X$を線型空間,$V,U\subset X$を絶対凸とする.
    \begin{enumerate}
        \item $V\prec U:\Leftrightarrow\exists_{t>0}\;V\subset tU$とき,$U$は$V$を併呑するという.
        \item $V\prec U$であるとする.このときさらに,$\forall_{\ep>0}\;\exists_{m\in\N}\;\exists_{x_1,\cdots,x_m\in X}\;V\subset\cup_{j\in[m]}(x_j+\ep U)$が成り立つ時,$V$は\textbf{$U$-プレコンパクト}であるという.
    \end{enumerate}
\end{definition}

\begin{example}
    the Schwartz space $\cS(\R^n)$はSchwartz空間である.
\end{example}

\begin{definition}
    局所凸空間$E$が\textbf{Schwartz空間}であるとは,任意の絶対凸な$0$の開近傍$U$に対して,ある$U$-プレコンパクトな$0$の絶対凸開近傍$V\prec U$が存在することをいう.
\end{definition}

\begin{lemma}
    完備なSchwartz空間$E$において,任意の有界集合は相対コンパクトである.
\end{lemma}

\begin{theorem}
    任意のFrechet-Schwartz空間は回帰的である.
\end{theorem}

\begin{theorem}
    $E$がFrechet-Schwartz空間ならば,$E^*$はSchwartz空間である.
\end{theorem}

\subsection{核型空間}

\begin{tcolorbox}[colframe=ForestGreen, colback=ForestGreen!10!white,breakable,colbacktitle=ForestGreen!40!white,coltitle=black,fonttitle=\bfseries\sffamily,
title=]
    この理論の多くはGrothendieck 55により,テンソル積の言葉で探求された.
    Banach空間が核型ならば,有限次元である.
    A nuclear vector space is a locally convex topological vector space that is as far from being a normed vector space as possible.
\end{tcolorbox}

\begin{definition}[nuclear]
    局所凸空間$E$が\textbf{核型}であるとは,任意の絶対凸な$0$の開近傍$U$に対して,$0$の開近傍$V$と$w^*$-コンパクトな集合$V^\circ$上の有限なRadon測度$\mu$が存在して,$\forall_{x\in E}\;\norm{x}_U\le\int_{V^\circ}\abs{y(x)}d\mu(y)$を満たすことをいう.
\end{definition}

\section{微分論}

\begin{tcolorbox}[colframe=ForestGreen, colback=ForestGreen!10!white,breakable,colbacktitle=ForestGreen!40!white,coltitle=black,fonttitle=\bfseries\sffamily,
title=]
    It is possible to generalize some aspects of analysis (differential calculus) to Fréchet spaces.\footnote{\url{https://ncatlab.org/nlab/show/Fr\%C3\%A9chet+space}}
    全微分に対応する概念をFréchet微分,方向微分に対応する概念をGâteaux微分という.
    定義そのものは全く似ているが,存在性の議論に入ると無限が顔を出す.
    強微分については,連鎖律も基本定理も成り立つ.

    実数値関数の微積分は,可分な場合に限ってよく理解されている.
    主に数学的に研究されているクラスはLipschitz関数である.
\end{tcolorbox}

\subsection{Banach空間間のLipschitz関数}

\begin{tcolorbox}[colframe=ForestGreen, colback=ForestGreen!10!white,breakable,colbacktitle=ForestGreen!40!white,coltitle=black,fonttitle=\bfseries\sffamily,
title=]
    非線形関数解析の道具の多くはこの研究から流入している.
    距離構造のみが目を向けるに足りるBanach空間の構造であるから,その一様構造に注目する.\cite{Lindenstrauss}
\end{tcolorbox}

\begin{theorem}[Mazur and Ulam]
    Banach空間の間の等長同型であって,$0$を$0$に移すものは線型である.
\end{theorem}
\begin{remarks}
    Banach空間の構造は,距離空間の構造だけで分類出来てしまい,線型同型はあとからついて来る.つまり,距離構造の方が豊かとわかる.
\end{remarks}

\begin{theorem}[Kadec]
    任意の可分な無限次元Banach空間は互いに同相である.特に,$\R^{\aleph_0}$は可分なBanach空間である.
\end{theorem}
\begin{remarks}
    Banach空間の位相構造は,線型構造について何も示唆しない.
\end{remarks}

\subsection{Path smoothness}

\begin{tcolorbox}[colframe=ForestGreen, colback=ForestGreen!10!white,breakable,colbacktitle=ForestGreen!40!white,coltitle=black,fonttitle=\bfseries\sffamily,
title=]
    まず,Frechet空間上の連続な線型汎関数$\mu\in V^*$が滑らかであるとはどういうことかを初等的に定義する.
    Frechet空間$V$をある種の多様体だと考えると,その上の任意の曲線$f\in C^\infty(\R,V)$について微分可能であることがまず考えられ,これは連続であることを含意する.
    次に,Gateaux微分を高階にしたものを用いれば,通常のsmoothnessも定義でき,したがって可微分Frechet多様体が定義できる.
\end{tcolorbox}

\begin{definition}
    Fréchet空間$V$上の線型汎関数$\mu\in V^*$が\textbf{path smooth}であるとは,
    任意の可微分写像$f:\R\to V$に対して,合成$\mu\circ f:\R\to\R$が可微分であることをいう.
\end{definition}

\begin{proposition}
    Fréchet空間$V$上のpath smoothな線型汎関数$\mu\in V^*$は連続である.
\end{proposition}

\begin{corollary}
    Schwartz超関数は滑らかである.
\end{corollary}

\begin{definition}[曲線の微分]
    Fréchet空間$V$上の連続道$\gamma:I:=[a,b]\to V;t\mapsto f(t)$について,導関数
    \[f'(t):=\lim_{h\to0}\frac{1}{h}(f(t+h)-f(t))\]
    が存在して連続であるとき,$C^1$級であるという.
\end{definition}

\subsection{偏微分}

\begin{tcolorbox}[colframe=ForestGreen, colback=ForestGreen!10!white,breakable,colbacktitle=ForestGreen!40!white,coltitle=black,fonttitle=\bfseries\sffamily,
title=]
    有限次元の場合と全く同様に微分を定義でき,$C^n(V)$の概念がある.
    すると,滑らかな関数という概念もあり,Fréchet多様体も同様に考えられる.
\end{tcolorbox}

\begin{definition}[directional / Gateaux derivative, Frechet derivative]
    $F,G$をFréchet空間,$P$を$U\osub F$上の連続な非線型写像$P:U\to G$とする.
    \begin{enumerate}
        \item 
    $f\in U$での$h\in F$方向微分を与える写像
    \[\xymatrix@R-2pc{
        DP:U\times F\ar[r]&G\\
        \rotatebox[origin=c]{90}{$\in$}&\rotatebox[origin=c]{90}{$\in$}\\
        (f,h)\ar@{|->}[r]&DP(f)h:=\lim_{t\to0}\frac{1}{t}(P(f+th)-P(f))
    }\]
    が存在して連続であるとき,$P$は$U$上$C^1$級であるという.
    $DP(f)=D_P(f)$を,$f\in F$における\textbf{弱導関数}または\textbf{Gâteaux導関数}という.\footnote{Lagrangeに従えば,$P$の$f$における第一変分という.}
    \item さらに,$DP(f)$が単位球$h\in B^\subset F$上一様に存在するとき,すなわち,
    \[\forall_{h\in F}\quad P(f+h)=P(f)+DP(f)h+o(\norm{h}).\]
    が成り立つとき,$P$は$f\in U$において\textbf{Frechet微分可能}といい,$DP(f)$を特に
    \textbf{強導関数}または\textbf{Fréchet導関数}という.
    \end{enumerate}
\end{definition}

\begin{lemma}
    連続作用素$P:F\to G$は$U\osub F$においてGateaux微分可能であるとする:$DP:U\times F\to G$は連続.
    \begin{enumerate}
        \item Gateaux導関数$DP(f):F\to G$は一意な連続線型作用素である:$DP(f)\in L(F,G)$.なお,連続写像$DP:U\times F\to G$の存在の仮定なくして,ある一点$f\in F$で弱微分可能なだけであるとき,Gateaux微分$DP(f):F\to G$が線型とは限らない.
        \item 連続とは限らない$P$について,$f\in F$においてFrechet微分可能ならば,$f\in F$において連続.なお,Gateaux微分可能性のみでは連続とは限らない.
    \end{enumerate}
\end{lemma}
\begin{Proof}\mbox{}
    \begin{enumerate}
        \item 
        $\forall_{\ep>0}\;\exists_{\delta>0}\;\norm{h}<\delta\Rightarrow\norm{P(f+h)-P(f)-DP(f)h}\le\ep\norm{h}$がGateaux微分可能性の同値な条件である.これは明らかに$DP(f)$の連続性を含意する.
        また,$L_1,L_2$を2つのGateaux微分とすると,$\norm{L_1h-L_2h}=o(h)$であるが,これは$L_1=L_2$を含意する.
    \end{enumerate}
\end{Proof}

\begin{theorem}[$C^1$級]
    写像$f:E\to F$は$x_0$の近傍でGateaux微分可能かつ,Gateaux導関数$Df(x)$は$x_0$にて連続とする:$\lim_{x\to x_0}\norm{Df(x)-Df(x_0)}=0$.このとき,$f$は$x_0$にてFrechet微分可能である.
    これが成り立つ時,写像$f$は$x_0$にて$C^1$級であるという.\footnote{Gateauxの意味か,Frechetの意味かは不問になる.}
\end{theorem}

\subsection{写像論}

\begin{tcolorbox}[colframe=ForestGreen, colback=ForestGreen!10!white,breakable,colbacktitle=ForestGreen!40!white,coltitle=black,fonttitle=\bfseries\sffamily,
title=]
    逆写像定理と陰関数定理がそのまま成り立つ.
\end{tcolorbox}

\begin{theorem}
    $E,F$をBanach空間,$f:E\supset U\to F$を開集合上の$C^1$級関数とする.$Df(x_0)$が可逆であるとき,$f$の$x_0$の近傍$U_0$への制限は$f|_{U_0}$は位相同型で,$f^{-1}$は$f(U_0)$上$C^1$級である.
\end{theorem}

\subsection{高階微分}

\begin{definition}
    $P$の$k$方向の2階微分を
    \[D^2P(f)(h,k)=\lim_{t\to0}\frac{1}{t}(DP(f+tk)h-DP(f)h)\]
    と定める.
\end{definition}

\section{積分論}

\begin{tcolorbox}[colframe=ForestGreen, colback=ForestGreen!10!white,breakable,colbacktitle=ForestGreen!40!white,coltitle=black,fonttitle=\bfseries\sffamily,
title=]
    確率積分と違って,考慮に足りる積分が複数ある.
    Bochner積分\ref{subsection-Pettis-integral}という.
\end{tcolorbox}

\subsection{Riemann積分の一般化}

\begin{notation}
    道
    $f:I:=[a,b]\to F$に沿った積分$\int^b_af(t)dt\in F$を定めたいが,自然な定め方はただ一つに定まる.
\end{notation}

\begin{theorem}
    次の4条件を満たす元$\int^b_af(t)dt\in F$は一意的である:
    \begin{enumerate}
        \item $\forall_{\phi\in F^*}\;\phi\paren{\int^b_af(t)dt}=\int^b_a\phi(f(t))dt$.
        \item $\forall_{\norm{-}:F\to\R:\text{seminorm}}\;\Norm{\int^b_af(t)dt}\le\int^b_a\norm{f(t)}dt$.
        \item 積分作用素$C(I,F)\ni f\mapsto \int^b_af(t)dt\in F$は線型である.
        \item 加法的である:$\int^b_af(t)dt+\int^c_bf(t)dt=\int^c_af(t)dt$.
    \end{enumerate}
\end{theorem}

\begin{theorem}[微積分学の基本定理]
    $P:F\to G$は$C^1$球で,$\forall_{t\in[0,1]}\;f+th\in\Dom(P)$ならば,
    \[P(f+h)-P(f)=\int^1_0DP(f+th)hdt.\]
\end{theorem}

\begin{theorem}[連鎖律]
    $P:F\to G,Q:G\to H$は$C^1$級であるとする.このとき,合成$Q\circ P$も$C^1$級で,
    \[D[Q\circ P](f)h=DQ(P(f))DP(f)h.\]
\end{theorem}

\section{幾何学}

\subsection{連続度}

\begin{definition}
    $X,Y$を距離空間,$f:X\to Y$を関数とする.
    \begin{enumerate}
        \item $\om_f(t):=\sup_{d(f(x),f(y))\ge0\mid d(x,y)\le t}$によって定まる写像$\om_f:\R_+\to\R_+$を\textbf{連続度}という.
        \item $\exists_{t_0>0}\;\forall_{t<t_0}\;\om_f(t)<\infty$かつ$\lim_{t\to0+}\om_f(t)=0$が成り立つ時,$f$は一様連続であるという.
        \item $\exists_{c\in\R}\;\om_f(t)\le ct$が成り立つとき,$f$は\textbf{Lipschitz写像}であるといい,条件を満たす最小の$c=\sup_{x\ne y}\frac{d(f(x),f(y))}{d(x,y)}$をLipschitz定数という.
        \item $\exists_{\al>0}\;\exists_{c\in\R}\;\om_f(t)\le ct^\al$が成り立つとき,$f$は\textbf{Holder$(\al)$-写像}であるという.
    \end{enumerate}
\end{definition}

\begin{lemma}
    $X$が凸集合であるとき,$\om_f$は劣加法的である:$\om_f(t+s)\le\om_f(t)+\om_f(s)$.
\end{lemma}

\subsection{不動点定理}

\begin{tcolorbox}[colframe=ForestGreen, colback=ForestGreen!10!white,breakable,colbacktitle=ForestGreen!40!white,coltitle=black,fonttitle=\bfseries\sffamily,
title=]
    Brouwerの不動点定理は,有限次元についての結果で,いくつかの同値な条件を持ったが,Banach空間ではそのいくつかは失敗する.
\end{tcolorbox}

\begin{theorem}[Schauder-Tychonoff (30)]
    局所凸位相線形空間$E$のコンパクトな凸集合$K$上の全射な連続写像$K\to K$は不動点を持つ.
\end{theorem}

\begin{theorem}[Banach (22)]
    $X$を完備距離空間,
    $f:X\to X$を縮小写像,すなわち$1$より小さいLipschitz定数をもつLipschitz写像とする.
    このとき,$f$はただ一つの不動点を持つ.
\end{theorem}

\chapter{最適化}

\begin{quotation}
    最適化とは極値問題のことであり,変分法とは汎関数の最適化である.
    この問題を関数解析の枠組みから捉え直したい.

    ニューラルネットワークをはじめ,統計モデルとは,関数のクラスに他ならない.
    位相解析の知見が使えないはずがない.
\end{quotation}

\section{凸関数論}

\subsection{下半連続関数}

\begin{tcolorbox}[colframe=ForestGreen, colback=ForestGreen!10!white,breakable,colbacktitle=ForestGreen!40!white,coltitle=black,fonttitle=\bfseries\sffamily,
title=]
    凸関数の特徴付けにおいて,$\Epi F$が凸であるという部分を閉であるとすると,これは下半連続性の特徴付けとなる.
\end{tcolorbox}

\begin{definition}
    関数$F:V\to\o{\R}$が下半連続であるとは,次の2つの同値な条件を満たすことをいう:
    \begin{enumerate}
        \item $\forall_{a\in\R}\;\Brace{u\in V\mid F(u)\le a}$ is clsoed.
        \item $\forall_{\o{u}\in V}\;\liminf_{u\to\o{u}}F(u)\ge F(\o{u})$.
    \end{enumerate}
    $V=\R$のときの特徴付け\ref{prop-characterization-of-lsc}参照.
\end{definition}

\begin{proposition}[下半連続関数の特徴付け]
    関数$F:V\to\o{\R}$について,
    \begin{enumerate}
        \item $F$は下半連続である.
        \item $\Epi F$は閉である.
    \end{enumerate}
\end{proposition}
\begin{remarks}
    特に,閉集合の定義関数は下半連続である.
    すなわち,連続な関数のうち,いくつかの点が下に飛び出していてもエピグラフは閉のままであり,このようなクラスが下半連続関数である.
    そう考えると,$f,-f$がいずれも下半連続であることが$f$の連続性を特徴付ける.
\end{remarks}

\begin{definition}[regularization of mappings]
    任意の下半連続関数族の下限は下半連続であった\ref{prop-subalgebra-of-lsc-function}.
    そこで,任意の関数$F\in\Map(V,\o{\R})$に対して,
    束$C^{1/2}(V)$の中で下限を取ることが考えられる.これを$F$の\textbf{下半連続正則化}といい,$\o{F}$で表す:
    \[\o{F}:=\sup\Brace{f\in C^{1/2}(V)\mid f\le F}.\]
\end{definition}

\begin{corollary}
    $F:V\to\o{\R}$を関数,$\o{F}$をその下半連続正則化とする.このとき,
    \begin{enumerate}
        \item $\Epi\o{F}=\o{\Epi F}$.
        \item $\forall_{u\in V}\;\o{F}(u)=\liminf_{v\to u}F(v)$.
    \end{enumerate}
\end{corollary}

\begin{corollary}
    下半連続な凸関数$F:V\to\o{\R}$は,弱位相$\sigma(V,V^*)$についても下半連続である.
\end{corollary}
\begin{remarks}
    凸集合についてノルム閉であることと弱閉であることとが同値であるという消息は,ここまで換言できる.
\end{remarks}

\subsection{凸関数の連続性}

\begin{tcolorbox}[colframe=ForestGreen, colback=ForestGreen!10!white,breakable,colbacktitle=ForestGreen!40!white,coltitle=black,fonttitle=\bfseries\sffamily,
title=]
    凸関数が連続であることと,局所Lipschitzであることと,ある開集合が存在してその上で有界であることとは同値である.
\end{tcolorbox}

\begin{lemma}
    凸関数$F:V\to\oR$が$u\in V$の近傍において有界である$\exists_{U\in\O(u)}\;\exists_{M\in\R}\;\abs{F|_U}\le M\;\on U$とき,$F$は$u$において連続である.
\end{lemma}

\begin{proposition}
    凸関数$F:V\to\oR$について,次の2条件は同値.
    \begin{enumerate}
        \item 非空な開集合$U$が存在して,$F|_U$は真の関数で\footnote{$-\infty$を取らない},ある定数$M\in\R$より大きくならない.
        \item $F$は真の関数で,$\Dom F$の内部で連続である.
    \end{enumerate}
\end{proposition}

\begin{corollary}[有限次元空間上の凸関数は連続]
    有限次元空間上の真の凸関数は,$\Dom F$の内部で連続である.
\end{corollary}

\begin{corollary}
    $F:V\to\oR$をノルム空間上の真の凸関数とする.
    次の2条件は同値.
    \begin{enumerate}
        \item 非空な開集合$U$が存在して,$F$は$U$上有界である.
        \item $\Dom F$の内部は非空で,$F$はその上で局所Lipschitzである.
    \end{enumerate}
\end{corollary}

\begin{corollary}
    樽型空間(特にBanach空間)上の下半連続な凸関数は,$\Dom F$の内部において連続である.
\end{corollary}

\subsection{連続affine関数の上限としての特徴付け}

\begin{tcolorbox}[colframe=ForestGreen, colback=ForestGreen!10!white,breakable,colbacktitle=ForestGreen!40!white,coltitle=black,fonttitle=\bfseries\sffamily,
title=閉凸関数への注目]
    任意の下半連続関数は,ある連続関数の族の上限として表現できる\ref{prop-lower-semicontinuous-functions}.
    凸な下半連続関数は特に,連続なaffine関数を選べ,これが特徴付けになることを示す.
    以降,凸な下半連続関数を\textbf{閉凸関数}という.これは,定義関数が閉凸関数であることと,そのエピグラフが集合として閉凸であることとが同値であることから.
\end{tcolorbox}

\begin{definition}[continuous affine function]
    ある連続線型汎関数$l\in V^*$と実数$\al$を用いて$l+\al$と表せる関数を\textbf{連続affine関数}という.
    連続affine関数の像の各点上限として得られる関数全体を$\Gamma(V)$で表し,$\Gamma_0(V):=\Gamma(V)\setminus\{\pm\infty\}$とする.
\end{definition}

\begin{proposition}
    $F:V\to\oR$について,次の2条件は同値.
    \begin{enumerate}
        \item $F\in\Gamma(V)$.
        \item $F$は凸な下半連続関数であり,$-\infty$を取るならば恒等関数である.
    \end{enumerate}
\end{proposition}

\subsection{$\Gamma$-正則化}

\begin{tcolorbox}[colframe=ForestGreen, colback=ForestGreen!10!white,breakable,colbacktitle=ForestGreen!40!white,coltitle=black,fonttitle=\bfseries\sffamily,
title=]
    $C^{1/2}(X)$内での下限を正則化というのであった.$\Gamma(X)$はそのうち凸でもある関数のなす部分空間である.
    $\Gamma(X)$での下限は$C^{1/2}(X)$での下限より小さくなる.
    実はこれが双極定理に繋がる.
\end{tcolorbox}

\begin{definition}[$\Gamma$-regularization]
    $F,G:V\to\oR$について,次の同値な条件を満たすとき,$G$を$F$の\textbf{$\Gamma$-正則化}であるという.
    \begin{enumerate}
        \item $G$は,$F$より小さい連続affine関数全体の各点上限関数である.
        \item $G$は$\Gamma(V)$における$F$の下限である.
    \end{enumerate}
\end{definition}

\begin{proposition}[$\Gamma$-正則化のエピグラフ]\label{prop-epigraph-of-gamma-regularization}
    $F:V\to\oR$を関数,$G$をその$\Gamma$-正則化とする.このとき,$F$より小さい連続affine関数が存在するならば,$\Epi G=\o{\Conv\Epi F}$が成り立つ.
\end{proposition}

\begin{proposition}
    $F:V\to\oR$を関数,$G$をその$\Gamma$-正則化とする.
    \begin{enumerate}
        \item $G\le\o{F}\le F$である.
        \item $F$が凸で,連続affine関数によって下から抑えられるとき,$\o{F}=G$.
    \end{enumerate}
\end{proposition}

\section{極関数と微分論}

\begin{notation}
    双線型写像$\brac{-,-}$が定めるペアリングを$(V,V^*)$で表す.それぞれに$\sigma(V,V^*),\sigma(V^*,V)$-位相を考えると,これらはそれぞれ局所凸位相線形空間である.
    ここで,$l:V\to\oR$が連続な線型汎関数であるといったとき,これはある$u^*\in V^*$を用いて$l=\brac{-,u^*}$なる表現を持つ.
\end{notation}

\subsection{極関数}

\begin{tcolorbox}[colframe=ForestGreen, colback=ForestGreen!10!white,breakable,colbacktitle=ForestGreen!40!white,coltitle=black,fonttitle=\bfseries\sffamily,
title=]
    なんだか極めて外積分の議論に似ている.作用素の議論のパターンだろうか.
\end{tcolorbox}

\begin{definition}[polar function / conjugate function]
    関数$F:V\to\oR$の\textbf{極関数}または\textbf{共役関数}とは,
    \[F^*(u^*):=\sup_{u\in V}\Brace{\brac{u,u^*}-F(u)}\]
    によって定まる関数$F^*:V^*\to\oR$をいう.
\end{definition}
\begin{remarks}[極関数は$\Gamma$-正則化の定数項を与える]
    $u^*\in V^*,\al\in\R$に関して,$u\mapsto\brac{u,u^*}-\al$は,連続affine関数で,
    至る所$F$より小さいことと,条件
    \[\forall_{u\in V}\;\al\ge\brac{u,u^*}-F(u)\quad\Leftrightarrow\quad\al\ge F^*(u^*)\]
    とは同値.よって,$u^*\in V^*$が定める連続affine関数で$F$より小さいもののうち最大のものの定数項が$-F^*(u^*)$で,すなわち,関数$F$の連続affine関数による下限は$u\mapsto\brac{u,u^*}-F^*(u^*)$である.
\end{remarks}

\begin{lemma}
    極関数は$F^*\in\Gamma(V^*)$を満たす.特に,$F^*$は凸な下半連続関数である.
\end{lemma}
\begin{Proof}
    $F^*$は連続affine関数族$(\brac{u,-}-F(u))_{u\in\Dom F}$の上限であるため.
\end{Proof}

\begin{corollary}\mbox{}
    \begin{enumerate}
        \item $F^*(0)=-\inf_{u\in V}F(u)$.
        \item $F\le G\Rightarrow F^*\ge G^*$.
        \item $(\inf_{i\in I}F_i)^*=\sup_{i\in I}F^*_i$.
        \item $(\sup_{i\in I}F_i)^*\le\inf_{i\in I}F^*_i$.
        \item $\forall_{\{F_i\}_{i\in I}\subset\Map(V,\R)}\;\forall_{\lambda>0}\;(\lambda F)^*(u^*)=\lambda F^*(u^*/\lambda)$.
        \item $\forall_{\{F_i\}_{i\in I}\subset\Map(V,\R)}\;\forall_{\al\in\R}\;(F+\al)^*=F^*-\al$.
        \item $F_a(v):=F(v-a)\;(a\in V)$を平行移動とする.$(F_a)^*(u^*)=F^*(u^*)+\brac{a,u^*}$.
    \end{enumerate}
\end{corollary}

\subsection{双極定理}

\begin{tcolorbox}[colframe=ForestGreen, colback=ForestGreen!10!white,breakable,colbacktitle=ForestGreen!40!white,coltitle=black,fonttitle=\bfseries\sffamily,
title=]
    極めてきれいな理論が出来上がった.
\end{tcolorbox}

\begin{proposition}[bipolar theorem (Fenchel-Moreau)]
    $F:V\to\oR$を関数とする.このとき,$F^{**}$は$F$の$\Gamma$-正則化である.
    特に,$\forall_{F\in\Gamma(V)}\;F^{**}=F$.
\end{proposition}

\begin{corollary}
    $\forall_{F\in\Map(V,\oR)}\;F^*=F^{***}$.
\end{corollary}

\begin{definition}[duality]
    極化の操作は2つの束$\Gamma(V)$と$\Gamma(V^*)$の全単射を引き起こす.
    この全単射によって対応する元,$F\in\Gamma(V),G\in\Gamma(V^*)$であって$F=G^*,G=F^*$を満たすものを,互いに双対であるという.
\end{definition}
\begin{remark}
    $\pm\infty$は$\pm\infty$に対応するから,この全単射の$\Gamma_0(V)$への制限はそのまま$\Gamma_0(V^*)$への全単射である.
\end{remark}

\subsection{双対理論}

\begin{theorem}[Fenchel-Rockafellar]\label{thm-Fenchel-Rockafellar}
    $\varphi,\psi:V\to\ocinterval{-\infty,\infty}$を真の凸関数とする.
    ある$x_0\in D(\varphi)\cap D(\psi)$上で$\varphi$は連続とする.
    このとき,
    \[\inf_{x\in V}(varphi(x)+\psi(x))=\sup_{f\in V^*}(-\varphi^*(-f)-\psi^*(f))=\max_{f\in V^*}(-\varphi^*(-f)-\psi^*(f)).\]
\end{theorem}

\subsection{極集合への応用}

\begin{tcolorbox}[colframe=ForestGreen, colback=ForestGreen!10!white,breakable,colbacktitle=ForestGreen!40!white,coltitle=black,fonttitle=\bfseries\sffamily,
title=]
    極関数の結果が整備できたいま,特性関数について特殊化させれば,極化集合論である.
    凸解析もマジできれいな理論で,関数解析に応用可能だな.
\end{tcolorbox}

\begin{example}[indicator function, support function]
    $A\subset V$を部分集合とし,$\chi_A$をその標示関数とする.
    その極関数は\textbf{$A$の支持関数}と呼ばれ,
    \[\chi_A^*(u^*)=\sup_{u\in V}\Brace{\brac{u,u^*}-\chi_a(u)}=\sup_{u\in A}\brac{u,u^*}.\]
    これは凸な下半連続関数で,また$V^*$上正斉次性を持つ.
    また,$\Gamma$-正則化の表現より,$\chi_A^{**}=\chi_{\o{\Conv}A}$.
    特に,$A$と$\o{\Conv}A$は同じ支持関数を持つ.
\end{example}

\begin{example}
    ノルム空間$(V,\norm{-})$とその双対空間$(V^*,\norm{-}^*)$について,$\varphi\in\Gamma_0(\R)$を偶関数とし,$\varphi^*\in\Gamma_0(\R)$をその極関数とする.
    このとき,
    \[F(u):=\varphi(\norm{u}),\quad G(u^*):=\varphi^*(\norm{u^*}^*)\]
    と定めると,これらは双対である.
\end{example}

\subsection{劣微分可能性}

\begin{tcolorbox}[colframe=ForestGreen, colback=ForestGreen!10!white,breakable,colbacktitle=ForestGreen!40!white,coltitle=black,fonttitle=\bfseries\sffamily,
title=]
    劣微分の概念は,通常の微分の定義(これはGateaux微分が引き継ぐ)より直観的に「affine関数による近似」として定まる.
    そして通常の微分概念とは違って常に存在するので,応用可能性が高い.
    したがって,極値問題を劣微分の言葉で捉え直すのである:
    \[F(u)=\min_{v\in V}F(v)\quad\Leftrightarrow\quad 0\in\partial F.\]
\end{tcolorbox}

\begin{definition}[exact]
    写像$F:V\to\oR$と,至る所$F$以下な連続affine関数$l:V\to\R,\forall_{u\in V}\;l(u)\le F(u)$に関して,
    \begin{enumerate}
        \item $l$は点$u\in V$において\textbf{完全}であるとは,$l(u)=F(u)$を満たすことをいう.
        \item $u$で完全なaffine関数$l:V\to\oR$は,ある$u^*\in V^*$を用いて,$l(v)=\brac{v-u,u^*}+F(u)=\brac{v,u^*}+F(u)-\brac{u,u^*}$なる表示を持つ必要がある.
        \item この完全な$l$は$F$を超えないものの中で極大である.したがって,定数項は極大であるから極関数$F^*$を用いて:$F(u)-\brac{u,u^*}=-F^*(u^*)$.
        \item $F$が$u\in V$において\textbf{劣微分可能}であるとは,$u$において完全な連続affine下限$l$が存在することをいう.
        \item $l$の傾き$u^*\in V^*$を$F$の$u$における\textbf{劣勾配}といい,劣勾配全体の集合を\textbf{劣微分}といい,$\partial F(u)$で表す.$\partial F(u)=\emptyset$と劣微分不可能であることとは同値.
    \end{enumerate}
\end{definition}

\begin{lemma}[劣勾配の特徴付け]\mbox{}
    \begin{enumerate}
        \item $u^*\in\partial F(u)$.
        \item $F(u)<\infty$かつ$\forall_{v\in V}\;\brac{v-u,u^*}+F(u)\le F(v)$.
        \item $F(u)+F^*(u^*)=\brac{u,u^*}$.
    \end{enumerate}
\end{lemma}

\begin{corollary}[劣微分は弱閉凸集合]
    劣微分$\partial F(u)\subset V^*$は,$\sigma(V^*,V)$-閉な凸集合である.
\end{corollary}

\begin{corollary}
    \[\forall_{F\in\Map(V,\R)}\;u^*\in\partial F(u)\Rightarrow u\in\partial F^*(u^*).\]
    $F\in\Gamma(V)$のとき,$\Leftarrow$も成り立つ.
\end{corollary}

\begin{lemma}[劣微分可能性の必要条件]
    $u\in V$について,
    \begin{enumerate}
        \item $\partial F(u)\ne\emptyset\Rightarrow F(u)=F^{**}(u)=l(u)$.
        \item $F(u)=F^{**}(u)\Rightarrow\partial F(u)=\partial F^{**}(u)$.
    \end{enumerate}
\end{lemma}

\begin{proposition}[凸関数の劣微分可能性の特徴付け]
    $F:V\to\R$を有限な凸関数とする.ある点$u\in V$において連続ならば,$\forall_{v\in(\Dom F)^\circ}\;\partial F(v)\ne\emptyset$.
\end{proposition}
\begin{remark}
    Banach空間上の下半連続な真凸関数は,$(\Dom F)^\circ$の稠密部分集合において劣微分可能である\ref{cor-subdifferentiable-points-of-Banach-operator}.
\end{remark}

\subsection{Gateaux微分可能性}

\begin{tcolorbox}[colframe=ForestGreen, colback=ForestGreen!10!white,breakable,colbacktitle=ForestGreen!40!white,coltitle=black,fonttitle=\bfseries\sffamily,
title=]
    凸関数において,劣微分可能性は通常の微分可能性の自然な一般化にもなっていることを見る.
    Gateaux微分のことばで,初等的に学ぶ極値問題が,無限次元にも通用する一般な形で,凸解析に応用が出来た.
\end{tcolorbox}

\begin{definition}
    ここでは,$DP(f)h\in\o{\R}$が存在する時,方向微分可能といい,これが第二変数について線型になるとき$\exists_{u^*\in V^*}\;\forall_{v\in V}\;DP(u)v=\brac{v,u^*}$,Gateaux微分可能であると言い分けることとする.
    そして,このときの$u^*=:F'(u)$をGateaux微分係数という.
\end{definition}

\begin{proposition}[Gateaux微分と劣微分の関係]
    $F:V\to\oR$を凸関数とする.
    \begin{enumerate}
        \item $F$が$u\in V$にてGateaux微分可能ならば,それは劣微分可能で,$\partial F(u)=\Brace{F'(u)}$が成り立つ.
        \item $F$が$u\in V$にて連続かつ有限で,一点集合の劣微分を持つとき,$F$はGateaux微分可能で,$\partial F(u)=\Brace{F'(u)}$である.
    \end{enumerate}
\end{proposition}

\begin{proposition}[Gateaux微分可能な関数が凸であることの特徴付け]
    $F$を凸関数$A\subset V$上のGateaux微分可能な実関数とする.このとき,次の2条件は同値.
    \begin{enumerate}
        \item $F$は$A$上凸である.
        \item $\forall_{u,v\in A}\;F(v)\ge F(u)+\brac{F'(u),v-u}$.
        \item Gateaux導関数$F':V\to V^*$は単調である:$\forall_{u_1,u_2\in V}\;\brac{u_1-u_2,F'(u_1)-F'(u_2)}\ge0$.\footnote{「変曲点がない」ことを見事に捉えている.}
    \end{enumerate}
\end{proposition}

\subsection{劣微分演算}

\begin{tcolorbox}[colframe=ForestGreen, colback=ForestGreen!10!white,breakable,colbacktitle=ForestGreen!40!white,coltitle=black,fonttitle=\bfseries\sffamily,
title=]
    劣微分とは,導関数を集合値関数として拡張する.
    値が一点集合に退化しているときが,通常の微分概念だと理解するのである.
\end{tcolorbox}

\begin{lemma}
    $F:V\to\oR,\lambda>0$について,
    $\partial:\Map(V,\o{\R})\to\Map(V,P(V^*))$は次を満たす:
    \begin{enumerate}
        \item 線形性:$\forall_{u\in V}\;\partial(\lambda F)(u)=\lambda \partial F(u)$.
        \item 優加法性:$\forall_{u\in V}\;\partial(F_1+F_2)(u)\supset\partial F_1(u)+\partial F_2(u)$.
    \end{enumerate}
\end{lemma}

\begin{proposition}[等号成立十分条件]
    $F_1,F_2\in\Gamma(V)$かつある点$\o{u}\in\Dom F_1\cap\Dom F_2$において$F_1$が連続であるとき,$\forall_{u\in V}\;\partial(F_1+F_2)(u)=\partial F_1(u)+\partial F_2(u)$.
\end{proposition}

\begin{proposition}
    局所凸空間$V,Y$と連続線型作用素$\Lambda:V\to Y$と関数$F\in\Gamma(Y)$について,$F$はある点$\Lambda\o{u}$において連続かつ有限であるとする.
    このとき,$\forall_{u\in V}\;\partial(F\circ\Lambda)(u)=\Lambda^*\partial F(\Lambda u)$.
\end{proposition}

\subsection{}

\begin{notation}
    $V$をBanach空間とする.
    $C(m):=\Brace{(u,a)\in V\times\R\mid a+m\norm{u}\le 0}$と定めると,これは非空な内部を持つ閉凸錐である.
    これが$V\times\R$上に定める関係$(u,a)\le(v,b):\Leftrightarrow(v-u,b-a)\in C(m)$は順序関係を定め,この順序について$C(m)$は正錐となる.
\end{notation}

\begin{proposition}
    部分集合$S\subset V\times\R$が$\inf\Brace{a\in\R\mid(u,a)\in S}>-\infty$を満たすならば,$S$は上述の順序における極大元を持つ.
\end{proposition}

\subsection{非凸関数への応用}

\begin{theorem}
    $F:V\to\oR$を下半連続関数で,$\inf F\in\R$を満たし,$\exists_{u\in V,\ep\in\R}\;F(u)\le\inf F+\ep$を満たすとする.
    このとき,任意の$\lambda>0$に対して,$u_\lambda\in V$が存在して,次の2条件を満たす
    \begin{enumerate}
        \item $\norm{u-u_\lambda}\le\lambda\land F(u_\lambda)\le F(u)$.
        \item $\Epi F\cap\Brace{(u_\lambda,F(u_\lambda))+C(\ep/\lambda)}=(u_\lambda,F(u_\lambda))$.
    \end{enumerate}
\end{theorem}

\subsection{凸関数への応用}

\begin{definition}
    \[\partial_\ep F(u):=\Brace{u^*\in V^*\mid 0\le F(u)+F^*(u^*)-\brac{u,u^*}\le\ep}\]
    を$F$の$u\in V$における\textbf{$\ep$-劣微分}という.
\end{definition}

\begin{theorem}
    
\end{theorem}

\begin{corollary}\label{cor-subdifferentiable-points-of-Banach-operator}
    $V$をBanach空間,$F\in\Gamma_0(V)$とする.$F$が劣微分可能な点の集合は$\Dom F$上稠密である.
\end{corollary}

\section{凸関数の最小化と変分不等式}

\begin{tcolorbox}[colframe=ForestGreen, colback=ForestGreen!10!white,breakable,colbacktitle=ForestGreen!40!white,coltitle=black,fonttitle=\bfseries\sffamily,
title=]
    極値問題のうち,汎関数の最小化の問題を変分問題という.
    古来から物理法則の多くは変分問題の形で定式化出来る.
    一般に最適化問題は有限次元の場合,すなわちパラメトリックな場合を指す.
\end{tcolorbox}

\begin{problem}
    回帰的Banach空間$V$の空でない閉凸集合$C\subset V$上の閉真凸関数(下半連続な真凸関数)$F:C\to\R$を考える.
    $F(u)=\inf_{v\in C}F(v)$を満たす元$u\in C$を\textbf{解}と呼ぶ.
    一般に,$V\setminus C$上では$+\infty$を取るとして延長した閉凸関数\footnote{エピグラフは変わらないので閉凸なままである.}$\wh{F}:V\to\oR$の$V$上での最小化を考えればよい.
\end{problem}

\subsection{最小値の存在}

\begin{proposition}
    開集合は$C$の閉凸集合である(空集合含む).
\end{proposition}
\begin{Proof}
    開集合はある$\al\in\R$を用いて$\Brace{u\in V\mid\wh{F}(u)\le\al}$という等位集合の形で表せるので\ref{def-level-set}.
\end{Proof}

\begin{proposition}
    次の(1)または(2)を満たすとき,解は存在する.(3)も満たすとき,解は一意である.
    \begin{enumerate}
        \item $C$は有界である.
        \item $F$は$C$上強圧的(coercive)である:$\forall_{u\in C}\;\norm{u}\to\infty\Rightarrow\lim F(u)=+\infty$.
        \item $F$は$C$上狭義凸である.
    \end{enumerate}
\end{proposition}

\subsection{解の特徴付け}

\begin{proposition}
    閉真凸関数$F:V\to\R$は連続なGateaux導関数$F'$を持つとする.$u\in C$について,次の3条件は同値.
    \begin{enumerate}
        \item $u$は解である.
        \item $\forall_{v\in C}\;\brac{F'(u),v-u}\ge0$.
        \item $\forall_{v\in C}\;\brac{F'(v),v-u}$.
    \end{enumerate}
\end{proposition}

\begin{proposition}
    閉真凸関数$F_1,F_2:V\to\R$について,$F_1$はGateaux導関数$F'$を持つとし,$F:=F_1+F_2$も閉真凸とする.$u\in C$について,次の3条件は同値.
    \begin{enumerate}
        \item $u$は解である.
        \item $\forall_{v\in C}\;\brac{F'_1(u),v-u}+F_2(v)-F_2(u)\ge0$.
        \item $\forall_{v\in C}\;\brac{F'_1(v),v-u}+F_2(v)-F_2(u)\ge0$.
    \end{enumerate}
\end{proposition}

\subsection{変分不等式}

\begin{problem}
    今度は,回帰的Banach空間$V$において,作用素$A:V\to V^*$と真の閉凸汎関数$\varphi:V\to\oR$を考える.
    任意の$f\in V^*$に対して,
    \[\forall_{v\in V}\quad\brac{Au-f,v-u}+\varphi(v)-\varphi(u)\ge0\]
    を満たす元$u\in V$を求めることを考える.
\end{problem}

\begin{theorem}
    $A,\varphi$はさらに次を満たすとする.
    \begin{enumerate}
        \item $V$の任意の有限次元部分空間上において$A$は弱連続.
        \item $A$は単調:$\forall_{u,v\in V}\;\brac{Au-Av,u-v}\ge0$.
        \item $\exists_{v_0\in\Dom\varphi}\;\norm{v}\to\infty\Rightarrow\frac{\brac{Av,v-v_0}+\varphi(v)}{\norm{v}}\to\infty$.
    \end{enumerate}
    このとき,任意の$f\in V^*$に対して,変分不等式の解$u\in V$は存在する.
\end{theorem}

\section{凸最適化の双対理論}

\begin{tcolorbox}[colframe=ForestGreen, colback=ForestGreen!10!white,breakable,colbacktitle=ForestGreen!40!white,coltitle=black,fonttitle=\bfseries\sffamily,
title=]
    双対理論はRochafellerによる.
    Fenchelに始まり,Rochafellerは共役関数の方法で双対理論を構築した.
    一方で,数理経済学で主流なのはLagrangianによるものである.Hurwicz and 宇沢など.
\end{tcolorbox}

\begin{notation}
    $X$を実ノルム空間とし,連続な双線型写像$\brac{-,-}:X\times X^*\to\R$を評価写像として定める.
    一般の双線型形式が定めるペアリングについて成り立つ.
\end{notation}

\subsection{双対理論一般}

\begin{definition}[primal problem, dual problem]
    $F:V\to\oR$に対して,
    $\inf_{u\in V}F(u)=\inf P\in\R$を主問題$P$とする.これに対して
    \[\sup\brac{u^*,u_0}=\beta,\quad u^*\in L^\perp,\norm{u^*}\le1\]
    を双対問題という.
\end{definition}

\begin{theorem}
    $L$を実ノルム空間$X$の部分空間とする.$u_0\in X$とする.次が成り立つ.
    \begin{enumerate}
        \item $\al=\beta$.
        \item 双対問題は解$u^*$を持つ.
        \item $u\in L$が主問題の解であることと,$\brac{u^*,u_0-u}=\norm{u-u_0}$が成り立つことは同値.
    \end{enumerate}
\end{theorem}

\begin{corollary}
    $L$が有限次元のとき,主問題は解を持つ.
\end{corollary}

\begin{remark}
    $L$は無限次元だが,回帰的Banach空間$X$の閉部分空間であるならば,主問題は解を持つ.
\end{remark}

\subsection{修正理論一般}

\begin{tcolorbox}[colframe=ForestGreen, colback=ForestGreen!10!white,breakable,colbacktitle=ForestGreen!40!white,coltitle=black,fonttitle=\bfseries\sffamily,
title=]
    minimum norm problemを$X^*$上で考える道もある.
\end{tcolorbox}

\begin{definition}[modified primal problem]
    $\inf\norm{u^*-u_0^*}=\al,\;u^*\in L^\perp$を\textbf{修正された主問題}という.
    これに対する双対問題は
    \[\sup\brac{u^*_0,u}=\beta,\quad u\in L,\norm{u}\le 1\]
    となる.
\end{definition}

\begin{theorem}
    $X$を実ノルム空間,$L$をその線型部分空間とする.$u_0^*\in X^*$とすると,次が成り立つ.
    \begin{enumerate}
        \item $\al=\beta$.
        \item 主問題は解$u^*$を持つ.
        \item $u\in L,\norm{u}\le 1$が双対問題の解であることは,$\brac{u_0^*-u^*,u}=\norm{u^*_0-u^*}$を満たすことに同値.
    \end{enumerate}
\end{theorem}

\subsection{応用:Cebysev近似}

\begin{tcolorbox}[colframe=ForestGreen, colback=ForestGreen!10!white,breakable,colbacktitle=ForestGreen!40!white,coltitle=black,fonttitle=\bfseries\sffamily,
title=]
    最適化理論はこのような数学的応用を持つ.
\end{tcolorbox}

\begin{definition}
    連続関数$u_0:[a,b]\to\R$と有限次元部分空間$L:=\Brace{P\in\R[x]\mid\deg P\le N}$
    に対して,近似問題
    \[\max_{a\le x\le b}\abs{u_0(x)-u(x)}=\min!,\quad u\in L\]
    を考える.
\end{definition}

\begin{proposition}
    上述の近似問題は解$u$を持ち,$\abs{u_0(x)-u(x)}$は$[a,b]$上で少なくとも$N+2$個の最大値を取る点を持つ.
\end{proposition}

\section{変分法}

\begin{tcolorbox}[colframe=ForestGreen, colback=ForestGreen!10!white,breakable,colbacktitle=ForestGreen!40!white,coltitle=black,fonttitle=\bfseries\sffamily,
title=]
    ノンパラメトリックな最適化法を議論しよう.
    言うならば,現代の統計学者が向き合っているのは時代の最先端の変分法である.
\end{tcolorbox}

\subsection{変分原理}

\begin{tcolorbox}[colframe=ForestGreen, colback=ForestGreen!10!white,breakable,colbacktitle=ForestGreen!40!white,coltitle=black,fonttitle=\bfseries\sffamily,
title=]
    無限次元空間では,ほとんどの集合がコンパクトにならないことが,変分問題の難しい点である.
    そこで,回帰的Banach空間では(特にHilbert空間では),有界列は弱点列コンパクトである\ref{thm-characterization-of-reflexive-Banach-spaces}ことに注目する.
\end{tcolorbox}

\begin{theorem}[generalized Weierstrass theorem]
    回帰的Banach空間$X$の単位閉球$B$上の
    最小化問題$F(u)=\min!,u\in B$は,汎関数$F:B\to\R$が,弱位相について点列下半連続であるとき,解を持つ.

    このことは,単位閉球$B$上だけでなく,任意の非空な有界閉凸集合$M\subset X$上で成り立つ.
\end{theorem}

\begin{remark}
    変分問題で現れる汎関数は$C^1$の$1$-ノルム$\norm{y}_1=\max_{x\in[a,b]}\abs{y(x)}+\max_{x\in[a,b]}\abs{y'(x)}$では連続だが,$C$の一様ノルムでは連続ではないことが多い.
\end{remark}

\subsection{変分法の基本補題}

\begin{tcolorbox}[colframe=ForestGreen, colback=ForestGreen!10!white,breakable,colbacktitle=ForestGreen!40!white,coltitle=black,fonttitle=\bfseries\sffamily,
title=]
    $C([a,b])$においては,$\Brace{h\in C([a,b])\mid h(a)=h(b)=0}$が「基底」のようなもので,この関数との内積がすべて$0$であるならば,$\al=0$に限る.
\end{tcolorbox}

\begin{lemma}
    $\al\in C([a,b])$について,
    \[\forall_{h\in C([a,b])}\;h(a)=h(b)=0\Rightarrow\int^b_a\al(x)h(x)dx=0\Rightarrow\al=0.\]
\end{lemma}

\begin{lemma}
    $\al\in C([a,b])$について,任意の$h(a)=h(b)=0$を満たす$C^1$級関数$h\in C^1([a,b])$に対して
    $\int^b_a\al(x)h'(x)dx=0$を満たすならば,$\al$は定数関数である.
\end{lemma}

\begin{lemma}[部分積分の公式の逆]
    $\al,\beta\in C([a,b])$が,任意の$h(a)=h(b)=0$を満たす$h\in C^1([a,b])$について
    \[\int^b_a\paren{\al(x)h(x)+\beta(x)h'(x)}=0\]
    を満たすならば,$\beta$は微分可能であり,$\al=\beta'$.
\end{lemma}

\subsection{$n$次変分}

\begin{tcolorbox}[colframe=ForestGreen, colback=ForestGreen!10!white,breakable,colbacktitle=ForestGreen!40!white,coltitle=black,fonttitle=\bfseries\sffamily,
title=]
    変分とは方向微分に他ならない.
\end{tcolorbox}

\begin{definition}[$n$-th variation]
    ノルム空間$X$上の点$u_0\in X$の開近傍$U$上で定義された汎関数$F:U\to\R$について,
    \begin{enumerate}
        \item $h\in X$について,$t=0\in\R$の近傍で定義された関数を$\phi_h(t):=F(u_0+th)$で表す.$\Delta F[h]$とも表す.
        \item $n$次変分$\delta^nF(u_0;h)$を,$\delta^nF(u_0;h):=\phi^{(n)}(0)$で定義する.
    \end{enumerate}
\end{definition}

\begin{lemma}[微分との関係]\mbox{}
    \begin{enumerate}
        \item $F$が$u_0\in X$においてGateaux微分可能であることと,任意の$h\in X$について変分$\delta F(u_0;h)$が存在して対応$\delta F(u_0;h)=F'(u_0)(h)$の$F'$が線型になることと同値.
        \item さらに$F(u_0+h)-F(u_0)=F'(u_0)(h)+o(\norm{h})$も成り立つとき,$F$は$u_0$においてFrechet微分可能であるという.
        \item 任意の$u\in U$においてFrechet微分が存在し,対応$F':U\to X^*$が連続になる時,$F$を$C^1$級という.
    \end{enumerate}
\end{lemma}

\begin{definition}\mbox{}
    \begin{enumerate}
        \item $F$が$u_0\in U$にて極小値を取るとは,ある開近傍$u_0\in V\subset U$が存在して,$\forall_{u\in V}\;F(u)\ge F(u_0)$を満たすことをいう.
        \item $u_0$にて臨界値を取るとは,$\forall_{h\in X}\;\delta F(u_0;h)=0\lor\delta F(u_0;h)$ is undefinedをいう.この条件は,$F$にGateaux導関数が存在するとき,$F'(u_0)=0$に同値.
    \end{enumerate}
\end{definition}

\subsection{極値の存在条件}

\begin{theorem}[極値の特徴付け]\mbox{}
    \begin{description}
        \item[必要条件] $F$が$u_0$にて極値を取るならば,$u_0$は臨界点である:任意の$h\in X$について,1次変分$\delta F(u_0;h)$が存在するならば$0$である.$F$のGateaux導関数$F'(u_0)$が存在するとき,同値な条件$F'(u_0)=0$はEuler方程式と呼ばれる.
        \item[十分条件] $F$が次の3条件を満たすとき,$u_0$にて極小値を取る.
        \begin{enumerate}
            \item (1次の最適性条件) $u_0$は$F$の臨界点である.
            \item (Legendre:2次の最適性条件) ある$u_0$の開近傍$V\subset U$について,任意の$u\in V$で2次変分$\delta^2F(u;h)$が存在し,$\exists_{c>0}\;\forall_{h\in X}\;\delta^2 F(u_0;h)\ge c\norm{h}^2$を満たす.
            \item (Jacobi) $\forall_{\ep>0}\;\exists_{\eta>0}\;\forall_{u,h\in X}\;\norm{u-u_0}<\eta\Rightarrow\abs{\delta^2F(u;h)-\delta^2F(u_0;h)}\le\ep\norm{h}^2$.
        \end{enumerate}
    \end{description}
\end{theorem}

\subsection{変分法の基本問題}

\begin{problem}
    $F\in C^2(\R^3)$に対して,空間
    \[Y(a,b):=\Brace{y\in C^1([a,b])\mid y(a)=A,y(b)=B}\]
    内で汎関数
    \[J[y]=\int^b_aF(x,y(x),y'(x))dx\]
    の$C^1$-極値点となるようなものを求めよ.
    $C$-極値は$C^1$-極値であることに注意.
\end{problem}

\begin{proposition}[基本問題の変分とEuler方程式]\mbox{}
    \begin{enumerate}
        \item \[\delta J=\int^b_a\paren{F_y(x,y,y')h+F_{y'}(x,y,y')h'}dx.\]
        \item $y$が$J[y]$の極値点であるための必要十分条件は,
        \[F_y-\dd{}{x}F_{y'}=0.\]
    \end{enumerate}
    (2)の方程式の積分曲線を\textbf{停留曲線}(extremals)といい,(2)の左辺を$F$の\textbf{変分導関数}という.
\end{proposition}

\begin{remark}
    $y\in C^1([a,b])$は2階導関数をもたないが,$C^1$-極値点になる,という状況がありえる.
\end{remark}

\subsection{双対理論}

\begin{tcolorbox}[colframe=ForestGreen, colback=ForestGreen!10!white,breakable,colbacktitle=ForestGreen!40!white,coltitle=black,fonttitle=\bfseries\sffamily,
title=]
    Friedrichs変換.
\end{tcolorbox}

\subsection{Ritz法}

\begin{definition}
    (連続な)目的汎関数$J(u)$の値域の下限$d:=\inf\Im J$について,許容関数の列$(\wh{u}_i)$で$J$での値が$d$にしゅうそくするものが存在する.
    このような関数列を,変分問題の極小列という.
    Euler方程式を導く解析的な手法ではなく,極小列を構成することで近似的に解く解法を\textbf{直接法}という.
\end{definition}

\begin{example}
    試験関数の空間$D$を決め,$D$の完全系$(\varphi_k)$を1つ取る.
    極小列の$m$番目を$u_m=c_0+c_1\varphi_1+\cdots+c_m\varphi_m$であって,係数$c_1,\cdots,c_m$は$J(u_m)$を最小にするものとすれば良い.これは有限次元の最適化問題に帰着されている.
\end{example}

\subsection{有限要素法}



\section{関数機械}

\begin{tcolorbox}[colframe=ForestGreen, colback=ForestGreen!10!white,breakable,colbacktitle=ForestGreen!40!white,coltitle=black,fonttitle=\bfseries\sffamily,
title=]
A function machine is a generalization of neural networks to potentially infinite dimensional layers, motivated by the study of universal approximation of operators and functionals over abstract Banach spaces.\footnote{\url{https://ncatlab.org/nlab/show/function+machine}}
\end{tcolorbox}

\begin{definition}[function machine, operator layer, functional layer, basis layer, fully-connected layer]
    $K\subset\R^d,K'\subset\R^{d'}$をコンパクト集合,$T$をaffine写像とする.
    \begin{enumerate}
        \item 作用素層とは,affine写像$T^o:\L^1(K,\mu)\to\L^1(K',\mu)$であって,測度の連続族$(W_t\ll\mu)_{t\in K'}$と関数$b\in\L^1(K,\mu)$が存在して,$T^0:f\mapsto\paren{t\mapsto\int_KfdW_t+b(t)}$と表せるものをいう.
    \end{enumerate}
\end{definition}

\section{ニューラルネットワーク}

\begin{tcolorbox}[colframe=ForestGreen, colback=ForestGreen!10!white,breakable,colbacktitle=ForestGreen!40!white,coltitle=black,fonttitle=\bfseries\sffamily,
title=]
A neural network is \textbf{a class of functions} used in both supervised and unsupervised machine learning to approximate a correspondence between samples in a dataset and their associated labels.\footnote{\url{https://ncatlab.org/nlab/show/neural+network}}
\end{tcolorbox}

\section{テンソルネットワーク}

\begin{tcolorbox}[colframe=ForestGreen, colback=ForestGreen!10!white,breakable,colbacktitle=ForestGreen!40!white,coltitle=black,fonttitle=\bfseries\sffamily,
title=]
    モノイド圏論におけるストリング図式と等価な概念(with an attitude)で,はじめは量子物理学で台頭した.
\end{tcolorbox}

\subsection{カーネル法}

\begin{tcolorbox}[colframe=ForestGreen, colback=ForestGreen!10!white,breakable,colbacktitle=ForestGreen!40!white,coltitle=black,fonttitle=\bfseries\sffamily,
title=]
    関数族ではなく,距離関数$d:D\times D\to\R$の指定によってデータ集合$D$を解析することをいう.
    $d$が定める積分核$\exp(-\lambda\cdot d(-,-))$を適切なクラスから選ぶことを考える.
\end{tcolorbox}

\chapter{参考文献}

\bibliography{../StatisticalSciences.bib,../SocialSciences.bib,../mathematics.bib,../statistics.bib}
\begin{thebibliography}{99}
    \item
    Gert K. Pederson "Analysis Now"
    \bibitem{HarmonicAnalysisOnSemigroups}
    Berg, and Christensen, and Ressel. \textit{Harmonic Analysis on Semigroups: Theory of Positive Definite and Related Functions}.
    \item
    Dunford and Schwartz. Linear Operators.
    \bibitem{吉田}
    吉田耕作 (1953) 『ヒルベルト空間論』.共立出版.
    \item
    生西明夫,中神祥臣『作用素環入門I』
    \item
    John B. Conway "A Course in Functional Analysis"
    \item
    Haïm Brezis 『関数解析』
    \bibitem{Murphy}
    Gerard J. Murphy "$C^*$-ALGEBRA AND OPERATOR THEORY"
    \bibitem{増田久弥}
    増田久弥『非線型数学』
    \bibitem{Kolmogorov}
    Kolmogorov, Fomin 『関数解析の基礎』
    \bibitem{藤田}
    藤田宏,黒田成俊,伊藤清三『関数解析』
    \bibitem{Eidelman}
    Eidelman, Yuli, Vitali D. Milman, and Antonis Tsolomitis. 2004. Functional analysis: an introduction. Providence (R.I.): American mathematical Society.
    \bibitem{Lindenstrauss}
    Joram Lindenstrauss, and Yoav Benyamini "Geometric Nonlinear Functional Analysis" Volume 1
    \bibitem{Ekeland and Temam}
    Ekeland and Temam -  Convex Analysis and Variational Problems
    \bibitem{Zeldler}
    Eberhard Zeldler "Applied Functional Analysis"
    \bibitem{Gelfand}
    I. M. Gelfand and S. V. Fomin "Calculus of Variations"
    \bibitem{Hytonen}
    Tuomas Hytonen, Jan van Neerven, Mark Veraar, Lutz Weis "Analysis in Banach Spaces"
\end{thebibliography}

\end{document}