\documentclass[uplatex,dvipdfmx]{jsreport}
\title{偏微分方程式論}\author{司馬博文}\date{\today}
\pagestyle{headings} \setcounter{secnumdepth}{4}
%%%%%%%%%%%%%%% 数理文書の組版 %%%%%%%%%%%%%%%

\usepackage{mathtools} %内部でamsmathを呼び出すことに注意.
%\mathtoolsset{showonlyrefs=true} %labelを附した数式にのみ附番される設定.
\usepackage{amsfonts} %mathfrak, mathcal, mathbbなど.
\usepackage{amsthm} %定理環境.
\usepackage{amssymb} %AMSFontsを使うためのパッケージ.
\usepackage{ascmac} %screen, itembox, shadebox環境.全てLATEX2εの標準機能の範囲で作られたもの.
\usepackage{comment} %comment環境を用いて,複数行をcomment outできるようにするpackage
\usepackage{wrapfig} %図の周りに文字をwrapさせることができる.詳細な制御ができる.
\usepackage[usenames, dvipsnames]{xcolor} %xcolorはcolorの拡張.optionの意味はdvipsnamesはLoad a set of predefined colors. forestgreenなどの色が追加されている.usenamesはobsoleteとだけ書いてあった.
\setcounter{tocdepth}{2} %目次に表示される深さ.2はsubsectionまで
\usepackage{multicol} %\begin{multicols}{2}環境で途中からmulticolumnに出来る.
\usepackage{mathabx}\newcommand{\wc}{\widecheck} %\widecheckなどのフォントパッケージ

%%%%%%%%%%%%%%% フォント %%%%%%%%%%%%%%%

\usepackage{textcomp, mathcomp} %Text Companionとは,T1 encodingに入らなかった文字群.これを使うためのパッケージ.\textsectionでブルバキに!
\usepackage[T1]{fontenc} %8bitエンコーディングにする.comp系拡張数学文字の動作が安定する.

%%%%%%%%%%%%%%% 一般文書の組版 %%%%%%%%%%%%%%%

\definecolor{花緑青}{cmyk}{1,0.07,0.10,0.10}\definecolor{サーモンピンク}{cmyk}{0,0.65,0.65,0.05}\definecolor{暗中模索}{rgb}{0.2,0.2,0.2}
\usepackage{url}\usepackage[dvipdfmx,colorlinks,linkcolor=花緑青,urlcolor=花緑青,citecolor=花緑青]{hyperref} %生成されるPDFファイルにおいて、\tableofcontentsによって書き出された目次をクリックすると該当する見出しへジャンプしたり、さらには、\label{ラベル名}を番号で参照する\ref{ラベル名}やthebibliography環境において\bibitem{ラベル名}を文献番号で参照する\cite{ラベル名}においても番号をクリックすると該当箇所にジャンプする.囲み枠はダサいので,colorlinksで囲み廃止し,リンク自体に色を付けることにした.
\usepackage{pxjahyper} %pxrubrica同様,八登崇之さん.hyperrefは日本語pLaTeXに最適化されていないから,hyperrefとセットで,(u)pLaTeX+hyperref+dvipdfmxの組み合わせで日本語を含む「しおり」をもつPDF文書を作成する場合に必要となる機能を提供する
\usepackage{ulem} %取り消し線を引くためのパッケージ
\usepackage{pxrubrica} %日本語にルビをふる.八登崇之(やとうたかゆき)氏による.

%%%%%%%%%%%%%%% 科学文書の組版 %%%%%%%%%%%%%%%

\usepackage[version=4]{mhchem} %化学式をTikZで簡単に書くためのパッケージ.
\usepackage{chemfig} %化学構造式をTikZで描くためのパッケージ.
\usepackage{siunitx} %IS単位を書くためのパッケージ

%%%%%%%%%%%%%%% 作図 %%%%%%%%%%%%%%%

\usepackage{tikz}\usetikzlibrary{positioning,automata}\usepackage{tikz-cd}\usepackage[all]{xy}
\def\objectstyle{\displaystyle} %デフォルトではxymatrix中の数式が文中数式モードになるので,それを直す.\labelstyleも同様にxy packageの中で定義されており,文中数式モードになっている.

\usepackage{graphicx} %rotatebox, scalebox, reflectbox, resizeboxなどのコマンドや,図表の読み込み\includegraphicsを司る.graphics というパッケージもありますが,graphicx はこれを高機能にしたものと考えて結構です(ただし graphicx は内部で graphics を読み込みます)
\usepackage[top=15truemm,bottom=15truemm,left=10truemm,right=10truemm]{geometry} %足助さんからもらったオプション

%%%%%%%%%%%%%%% 参照 %%%%%%%%%%%%%%%
%参考文献リストを出力したい箇所に\bibliography{../mathematics.bib}を追記すると良い.

%\bibliographystyle{jplain}
%\bibliographystyle{jname}
\bibliographystyle{apalike}

%%%%%%%%%%%%%%% 計算機文書の組版 %%%%%%%%%%%%%%%

\usepackage[breakable]{tcolorbox} %加藤晃史さんがフル活用していたtcolorboxを,途中改ページ可能で.
\tcbuselibrary{theorems} %https://qiita.com/t_kemmochi/items/483b8fcdb5db8d1f5d5e
\usepackage{enumerate} %enumerate環境を凝らせる.

\usepackage{listings} %ソースコードを表示できる環境.多分もっといい方法ある.
\usepackage{jvlisting} %日本語のコメントアウトをする場合jlistingが必要
\lstset{ %ここからソースコードの表示に関する設定.lstlisting環境では,[caption=hoge,label=fuga]などのoptionを付けられる.
%[escapechar=!]とすると,LaTeXコマンドを使える.
  basicstyle={\ttfamily},
  identifierstyle={\small},
  commentstyle={\smallitshape},
  keywordstyle={\small\bfseries},
  ndkeywordstyle={\small},
  stringstyle={\small\ttfamily},
  frame={tb},
  breaklines=true,
  columns=[l]{fullflexible},
  numbers=left,
  xrightmargin=0zw,
  xleftmargin=3zw,
  numberstyle={\scriptsize},
  stepnumber=1,
  numbersep=1zw,
  lineskip=-0.5ex
}
%\makeatletter %caption番号を「[chapter番号].[section番号].[subsection番号]-[そのsubsection内においてn番目]」に変更
%    \AtBeginDocument{
%    \renewcommand*{\thelstlisting}{\arabic{chapter}.\arabic{section}.\arabic{lstlisting}}
%    \@addtoreset{lstlisting}{section}
%    }
%\makeatother
\renewcommand{\lstlistingname}{算譜} %caption名を"program"に変更

\newtcolorbox{tbox}[3][]{%
colframe=#2,colback=#2!10,coltitle=#2!20!black,title={#3},#1}

% 証明内の文字が小さくなる環境.
\newenvironment{Proof}[1][\bf\underline{[証明]}]{\proof[#1]\color{darkgray}}{\endproof}

%%%%%%%%%%%%%%% 数学記号のマクロ %%%%%%%%%%%%%%%

%%% 括弧類
\newcommand{\abs}[1]{\lvert#1\rvert}\newcommand{\Abs}[1]{\left|#1\right|}\newcommand{\norm}[1]{\|#1\|}\newcommand{\Norm}[1]{\left\|#1\right\|}\newcommand{\Brace}[1]{\left\{#1\right\}}\newcommand{\BRace}[1]{\biggl\{#1\biggr\}}\newcommand{\paren}[1]{\left(#1\right)}\newcommand{\Paren}[1]{\biggr(#1\biggl)}\newcommand{\bracket}[1]{\langle#1\rangle}\newcommand{\brac}[1]{\langle#1\rangle}\newcommand{\Bracket}[1]{\left\langle#1\right\rangle}\newcommand{\Brac}[1]{\left\langle#1\right\rangle}\newcommand{\bra}[1]{\left\langle#1\right|}\newcommand{\ket}[1]{\left|#1\right\rangle}\newcommand{\Square}[1]{\left[#1\right]}\newcommand{\SQuare}[1]{\biggl[#1\biggr]}
\renewcommand{\o}[1]{\overline{#1}}\renewcommand{\u}[1]{\underline{#1}}\newcommand{\wt}[1]{\widetilde{#1}}\newcommand{\wh}[1]{\widehat{#1}}
\newcommand{\pp}[2]{\frac{\partial #1}{\partial #2}}\newcommand{\ppp}[3]{\frac{\partial #1}{\partial #2\partial #3}}\newcommand{\dd}[2]{\frac{d #1}{d #2}}
\newcommand{\floor}[1]{\lfloor#1\rfloor}\newcommand{\Floor}[1]{\left\lfloor#1\right\rfloor}\newcommand{\ceil}[1]{\lceil#1\rceil}
\newcommand{\ocinterval}[1]{(#1]}\newcommand{\cointerval}[1]{[#1)}\newcommand{\COinterval}[1]{\left[#1\right)}


%%% 予約語
\renewcommand{\iff}{\;\mathrm{iff}\;}
\newcommand{\False}{\mathrm{False}}\newcommand{\True}{\mathrm{True}}
\newcommand{\otherwise}{\mathrm{otherwise}}
\newcommand{\st}{\;\mathrm{s.t.}\;}

%%% 略記
\newcommand{\M}{\mathcal{M}}\newcommand{\cF}{\mathcal{F}}\newcommand{\cD}{\mathcal{D}}\newcommand{\fX}{\mathfrak{X}}\newcommand{\fY}{\mathfrak{Y}}\newcommand{\fZ}{\mathfrak{Z}}\renewcommand{\H}{\mathcal{H}}\newcommand{\fH}{\mathfrak{H}}\newcommand{\bH}{\mathbb{H}}\newcommand{\id}{\mathrm{id}}\newcommand{\A}{\mathcal{A}}\newcommand{\U}{\mathfrak{U}}
\newcommand{\lmd}{\lambda}
\newcommand{\Lmd}{\Lambda}

%%% 矢印類
\newcommand{\iso}{\xrightarrow{\,\smash{\raisebox{-0.45ex}{\ensuremath{\scriptstyle\sim}}}\,}}
\newcommand{\Lrarrow}{\;\;\Leftrightarrow\;\;}

%%% 注記
\newcommand{\rednote}[1]{\textcolor{red}{#1}}

% ノルム位相についての閉包 https://newbedev.com/how-to-make-double-overline-with-less-vertical-displacement
\makeatletter
\newcommand{\dbloverline}[1]{\overline{\dbl@overline{#1}}}
\newcommand{\dbl@overline}[1]{\mathpalette\dbl@@overline{#1}}
\newcommand{\dbl@@overline}[2]{%
  \begingroup
  \sbox\z@{$\m@th#1\overline{#2}$}%
  \ht\z@=\dimexpr\ht\z@-2\dbl@adjust{#1}\relax
  \box\z@
  \ifx#1\scriptstyle\kern-\scriptspace\else
  \ifx#1\scriptscriptstyle\kern-\scriptspace\fi\fi
  \endgroup
}
\newcommand{\dbl@adjust}[1]{%
  \fontdimen8
  \ifx#1\displaystyle\textfont\else
  \ifx#1\textstyle\textfont\else
  \ifx#1\scriptstyle\scriptfont\else
  \scriptscriptfont\fi\fi\fi 3
}
\makeatother
\newcommand{\oo}[1]{\dbloverline{#1}}

% hslashの他の文字Ver.
\newcommand{\hslashslash}{%
    \scalebox{1.2}{--
    }%
}
\newcommand{\dslash}{%
  {%
    \vphantom{d}%
    \ooalign{\kern.05em\smash{\hslashslash}\hidewidth\cr$d$\cr}%
    \kern.05em
  }%
}
\newcommand{\dint}{%
  {%
    \vphantom{d}%
    \ooalign{\kern.05em\smash{\hslashslash}\hidewidth\cr$\int$\cr}%
    \kern.05em
  }%
}
\newcommand{\dL}{%
  {%
    \vphantom{d}%
    \ooalign{\kern.05em\smash{\hslashslash}\hidewidth\cr$L$\cr}%
    \kern.05em
  }%
}

%%% 演算子
\DeclareMathOperator{\grad}{\mathrm{grad}}\DeclareMathOperator{\rot}{\mathrm{rot}}\DeclareMathOperator{\divergence}{\mathrm{div}}\DeclareMathOperator{\tr}{\mathrm{tr}}\newcommand{\pr}{\mathrm{pr}}
\newcommand{\Map}{\mathrm{Map}}\newcommand{\dom}{\mathrm{Dom}\;}\newcommand{\cod}{\mathrm{Cod}\;}\newcommand{\supp}{\mathrm{supp}\;}


%%% 線型代数学
\newcommand{\vctr}[2]{\begin{pmatrix}#1\\#2\end{pmatrix}}\newcommand{\vctrr}[3]{\begin{pmatrix}#1\\#2\\#3\end{pmatrix}}\newcommand{\mtrx}[4]{\begin{pmatrix}#1&#2\\#3&#4\end{pmatrix}}\newcommand{\smtrx}[4]{\paren{\begin{smallmatrix}#1&#2\\#3&#4\end{smallmatrix}}}\newcommand{\Ker}{\mathrm{Ker}\;}\newcommand{\Coker}{\mathrm{Coker}\;}\newcommand{\Coim}{\mathrm{Coim}\;}\DeclareMathOperator{\rank}{\mathrm{rank}}\newcommand{\lcm}{\mathrm{lcm}}\newcommand{\sgn}{\mathrm{sgn}\,}\newcommand{\GL}{\mathrm{GL}}\newcommand{\SL}{\mathrm{SL}}\newcommand{\alt}{\mathrm{alt}}
%%% 複素解析学
\renewcommand{\Re}{\mathrm{Re}\;}\renewcommand{\Im}{\mathrm{Im}\;}\newcommand{\Gal}{\mathrm{Gal}}\newcommand{\PGL}{\mathrm{PGL}}\newcommand{\PSL}{\mathrm{PSL}}\newcommand{\Log}{\mathrm{Log}\,}\newcommand{\Res}{\mathrm{Res}\,}\newcommand{\on}{\mathrm{on}\;}\newcommand{\hatC}{\widehat{\C}}\newcommand{\hatR}{\hat{\R}}\newcommand{\PV}{\mathrm{P.V.}}\newcommand{\diam}{\mathrm{diam}}\newcommand{\Area}{\mathrm{Area}}\newcommand{\Lap}{\Laplace}\newcommand{\f}{\mathbf{f}}\newcommand{\cR}{\mathcal{R}}\newcommand{\const}{\mathrm{const.}}\newcommand{\Om}{\Omega}\newcommand{\Cinf}{C^\infty}\newcommand{\ep}{\epsilon}\newcommand{\dist}{\mathrm{dist}}\newcommand{\opart}{\o{\partial}}\newcommand{\Length}{\mathrm{Length}}
%%% 集合と位相
\renewcommand{\O}{\mathcal{O}}\renewcommand{\S}{\mathcal{S}}\renewcommand{\U}{\mathcal{U}}\newcommand{\V}{\mathcal{V}}\renewcommand{\P}{\mathcal{P}}\newcommand{\R}{\mathbb{R}}\newcommand{\N}{\mathbb{N}}\newcommand{\C}{\mathbb{C}}\newcommand{\Z}{\mathbb{Z}}\newcommand{\Q}{\mathbb{Q}}\newcommand{\TV}{\mathrm{TV}}\newcommand{\ORD}{\mathrm{ORD}}\newcommand{\Tr}{\mathrm{Tr}}\newcommand{\Card}{\mathrm{Card}\;}\newcommand{\Top}{\mathrm{Top}}\newcommand{\Disc}{\mathrm{Disc}}\newcommand{\Codisc}{\mathrm{Codisc}}\newcommand{\CoDisc}{\mathrm{CoDisc}}\newcommand{\Ult}{\mathrm{Ult}}\newcommand{\ord}{\mathrm{ord}}\newcommand{\maj}{\mathrm{maj}}\newcommand{\bS}{\mathbb{S}}\newcommand{\PConn}{\mathrm{PConn}}

%%% 形式言語理論
\newcommand{\REGEX}{\mathrm{REGEX}}\newcommand{\RE}{\mathbf{RE}}
%%% Graph Theory
\newcommand{\SimpGph}{\mathrm{SimpGph}}\newcommand{\Gph}{\mathrm{Gph}}\newcommand{\mult}{\mathrm{mult}}\newcommand{\inv}{\mathrm{inv}}

%%% 多様体
\newcommand{\Der}{\mathrm{Der}}\newcommand{\osub}{\overset{\mathrm{open}}{\subset}}\newcommand{\osup}{\overset{\mathrm{open}}{\supset}}\newcommand{\al}{\alpha}\newcommand{\K}{\mathbb{K}}\newcommand{\Sp}{\mathrm{Sp}}\newcommand{\g}{\mathfrak{g}}\newcommand{\h}{\mathfrak{h}}\newcommand{\Exp}{\mathrm{Exp}\;}\newcommand{\Imm}{\mathrm{Imm}}\newcommand{\Imb}{\mathrm{Imb}}\newcommand{\codim}{\mathrm{codim}\;}\newcommand{\Gr}{\mathrm{Gr}}
%%% 代数
\newcommand{\Ad}{\mathrm{Ad}}\newcommand{\finsupp}{\mathrm{fin\;supp}}\newcommand{\SO}{\mathrm{SO}}\newcommand{\SU}{\mathrm{SU}}\newcommand{\acts}{\curvearrowright}\newcommand{\mono}{\hookrightarrow}\newcommand{\epi}{\twoheadrightarrow}\newcommand{\Stab}{\mathrm{Stab}}\newcommand{\nor}{\mathrm{nor}}\newcommand{\T}{\mathbb{T}}\newcommand{\Aff}{\mathrm{Aff}}\newcommand{\rsub}{\triangleleft}\newcommand{\rsup}{\triangleright}\newcommand{\subgrp}{\overset{\mathrm{subgrp}}{\subset}}\newcommand{\Ext}{\mathrm{Ext}}\newcommand{\sbs}{\subset}\newcommand{\sps}{\supset}\newcommand{\In}{\mathrm{in}\;}\newcommand{\Tor}{\mathrm{Tor}}\newcommand{\p}{\b{p}}\newcommand{\q}{\mathfrak{q}}\newcommand{\m}{\mathfrak{m}}\newcommand{\cS}{\mathcal{S}}\newcommand{\Frac}{\mathrm{Frac}\,}\newcommand{\Spec}{\mathrm{Spec}\,}\newcommand{\bA}{\mathbb{A}}\newcommand{\Sym}{\mathrm{Sym}}\newcommand{\Ann}{\mathrm{Ann}}\newcommand{\Her}{\mathrm{Her}}\newcommand{\Bil}{\mathrm{Bil}}\newcommand{\Ses}{\mathrm{Ses}}\newcommand{\FVS}{\mathrm{FVS}}
%%% 代数的位相幾何学
\newcommand{\Ho}{\mathrm{Ho}}\newcommand{\CW}{\mathrm{CW}}\newcommand{\lc}{\mathrm{lc}}\newcommand{\cg}{\mathrm{cg}}\newcommand{\Fib}{\mathrm{Fib}}\newcommand{\Cyl}{\mathrm{Cyl}}\newcommand{\Ch}{\mathrm{Ch}}
%%% 微分幾何学
\newcommand{\rE}{\mathrm{E}}\newcommand{\e}{\b{e}}\renewcommand{\k}{\b{k}}\newcommand{\Christ}[2]{\begin{Bmatrix}#1\\#2\end{Bmatrix}}\renewcommand{\Vec}[1]{\overrightarrow{\mathrm{#1}}}\newcommand{\hen}[1]{\mathrm{#1}}\renewcommand{\b}[1]{\boldsymbol{#1}}

%%% 函数解析
\newcommand{\HS}{\mathrm{HS}}\newcommand{\loc}{\mathrm{loc}}\newcommand{\Lh}{\mathrm{L.h.}}\newcommand{\Epi}{\mathrm{Epi}\;}\newcommand{\slim}{\mathrm{slim}}\newcommand{\Ban}{\mathrm{Ban}}\newcommand{\Hilb}{\mathrm{Hilb}}\newcommand{\Ex}{\mathrm{Ex}}\newcommand{\Co}{\mathrm{Co}}\newcommand{\sa}{\mathrm{sa}}\newcommand{\nnorm}[1]{{\left\vert\kern-0.25ex\left\vert\kern-0.25ex\left\vert #1 \right\vert\kern-0.25ex\right\vert\kern-0.25ex\right\vert}}\newcommand{\dvol}{\mathrm{dvol}}\newcommand{\Sconv}{\mathrm{Sconv}}\newcommand{\I}{\mathcal{I}}\newcommand{\nonunital}{\mathrm{nu}}\newcommand{\cpt}{\mathrm{cpt}}\newcommand{\lcpt}{\mathrm{lcpt}}\newcommand{\com}{\mathrm{com}}\newcommand{\Haus}{\mathrm{Haus}}\newcommand{\proper}{\mathrm{proper}}\newcommand{\infinity}{\mathrm{inf}}\newcommand{\TVS}{\mathrm{TVS}}\newcommand{\ess}{\mathrm{ess}}\newcommand{\ext}{\mathrm{ext}}\newcommand{\Index}{\mathrm{Index}\;}\newcommand{\SSR}{\mathrm{SSR}}\newcommand{\vs}{\mathrm{vs.}}\newcommand{\fM}{\mathfrak{M}}\newcommand{\EDM}{\mathrm{EDM}}\newcommand{\Tw}{\mathrm{Tw}}\newcommand{\fC}{\mathfrak{C}}\newcommand{\bn}{\boldsymbol{n}}\newcommand{\br}{\boldsymbol{r}}\newcommand{\Lam}{\Lambda}\newcommand{\lam}{\lambda}\newcommand{\one}{\mathbf{1}}\newcommand{\dae}{\text{-a.e.}}\newcommand{\das}{\text{-a.s.}}\newcommand{\td}{\text{-}}\newcommand{\RM}{\mathrm{RM}}\newcommand{\BV}{\mathrm{BV}}\newcommand{\normal}{\mathrm{normal}}\newcommand{\lub}{\mathrm{lub}\;}\newcommand{\Graph}{\mathrm{Graph}}\newcommand{\Ascent}{\mathrm{Ascent}}\newcommand{\Descent}{\mathrm{Descent}}\newcommand{\BIL}{\mathrm{BIL}}\newcommand{\fL}{\mathfrak{L}}\newcommand{\De}{\Delta}
%%% 積分論
\newcommand{\calA}{\mathcal{A}}\newcommand{\calB}{\mathcal{B}}\newcommand{\D}{\mathcal{D}}\newcommand{\Y}{\mathcal{Y}}\newcommand{\calC}{\mathcal{C}}\renewcommand{\ae}{\mathrm{a.e.}\;}\newcommand{\cZ}{\mathcal{Z}}\newcommand{\fF}{\mathfrak{F}}\newcommand{\fI}{\mathfrak{I}}\newcommand{\E}{\mathcal{E}}\newcommand{\sMap}{\sigma\textrm{-}\mathrm{Map}}\DeclareMathOperator*{\argmax}{arg\,max}\DeclareMathOperator*{\argmin}{arg\,min}\newcommand{\cC}{\mathcal{C}}\newcommand{\comp}{\complement}\newcommand{\J}{\mathcal{J}}\newcommand{\sumN}[1]{\sum_{#1\in\N}}\newcommand{\cupN}[1]{\cup_{#1\in\N}}\newcommand{\capN}[1]{\cap_{#1\in\N}}\newcommand{\Sum}[1]{\sum_{#1=1}^\infty}\newcommand{\sumn}{\sum_{n=1}^\infty}\newcommand{\summ}{\sum_{m=1}^\infty}\newcommand{\sumk}{\sum_{k=1}^\infty}\newcommand{\sumi}{\sum_{i=1}^\infty}\newcommand{\sumj}{\sum_{j=1}^\infty}\newcommand{\cupn}{\cup_{n=1}^\infty}\newcommand{\capn}{\cap_{n=1}^\infty}\newcommand{\cupk}{\cup_{k=1}^\infty}\newcommand{\cupi}{\cup_{i=1}^\infty}\newcommand{\cupj}{\cup_{j=1}^\infty}\newcommand{\limn}{\lim_{n\to\infty}}\renewcommand{\l}{\mathcal{l}}\renewcommand{\L}{\mathcal{L}}\newcommand{\Cl}{\mathrm{Cl}}\newcommand{\cN}{\mathcal{N}}\newcommand{\Ae}{\textrm{-a.e.}\;}\newcommand{\csub}{\overset{\textrm{closed}}{\subset}}\newcommand{\csup}{\overset{\textrm{closed}}{\supset}}\newcommand{\wB}{\wt{B}}\newcommand{\cG}{\mathcal{G}}\newcommand{\Lip}{\mathrm{Lip}}\DeclareMathOperator{\Dom}{\mathrm{Dom}}\newcommand{\AC}{\mathrm{AC}}\newcommand{\Mol}{\mathrm{Mol}}
%%% Fourier解析
\newcommand{\Pe}{\mathrm{Pe}}\newcommand{\wR}{\wh{\mathbb{\R}}}\newcommand*{\Laplace}{\mathop{}\!\mathbin\bigtriangleup}\newcommand*{\DAlambert}{\mathop{}\!\mathbin\Box}\newcommand{\bT}{\mathbb{T}}\newcommand{\dx}{\dslash x}\newcommand{\dt}{\dslash t}\newcommand{\ds}{\dslash s}
%%% 数値解析
\newcommand{\round}{\mathrm{round}}\newcommand{\cond}{\mathrm{cond}}\newcommand{\diag}{\mathrm{diag}}
\newcommand{\Adj}{\mathrm{Adj}}\newcommand{\Pf}{\mathrm{Pf}}\newcommand{\Sg}{\mathrm{Sg}}

%%% 確率論
\newcommand{\Prob}{\mathrm{Prob}}\newcommand{\X}{\mathcal{X}}\newcommand{\Meas}{\mathrm{Meas}}\newcommand{\as}{\;\mathrm{a.s.}}\newcommand{\io}{\;\mathrm{i.o.}}\newcommand{\fe}{\;\mathrm{f.e.}}\newcommand{\F}{\mathcal{F}}\newcommand{\bF}{\mathbb{F}}\newcommand{\W}{\mathcal{W}}\newcommand{\Pois}{\mathrm{Pois}}\newcommand{\iid}{\mathrm{i.i.d.}}\newcommand{\wconv}{\rightsquigarrow}\newcommand{\Var}{\mathrm{Var}}\newcommand{\xrightarrown}{\xrightarrow{n\to\infty}}\newcommand{\au}{\mathrm{au}}\newcommand{\cT}{\mathcal{T}}\newcommand{\wto}{\overset{w}{\to}}\newcommand{\dto}{\overset{d}{\to}}\newcommand{\pto}{\overset{p}{\to}}\newcommand{\vto}{\overset{v}{\to}}\newcommand{\Cont}{\mathrm{Cont}}\newcommand{\stably}{\mathrm{stably}}\newcommand{\Np}{\mathbb{N}^+}\newcommand{\oM}{\overline{\mathcal{M}}}\newcommand{\fP}{\mathfrak{P}}\newcommand{\sign}{\mathrm{sign}}\DeclareMathOperator{\Div}{Div}
\newcommand{\bD}{\mathbb{D}}\newcommand{\fW}{\mathfrak{W}}\newcommand{\DL}{\mathcal{D}\mathcal{L}}\renewcommand{\r}[1]{\mathrm{#1}}\newcommand{\rC}{\mathrm{C}}
%%% 情報理論
\newcommand{\bit}{\mathrm{bit}}\DeclareMathOperator{\sinc}{sinc}
%%% 量子論
\newcommand{\err}{\mathrm{err}}
%%% 最適化
\newcommand{\varparallel}{\mathbin{\!/\mkern-5mu/\!}}\newcommand{\Minimize}{\text{Minimize}}\newcommand{\subjectto}{\text{subject to}}\newcommand{\Ri}{\mathrm{Ri}}\newcommand{\Cone}{\mathrm{Cone}}\newcommand{\Int}{\mathrm{Int}}
%%% 数理ファイナンス
\newcommand{\pre}{\mathrm{pre}}\newcommand{\om}{\omega}

%%% 偏微分方程式
\let\div\relax
\DeclareMathOperator{\div}{div}\newcommand{\del}{\partial}
\newcommand{\LHS}{\mathrm{LHS}}\newcommand{\RHS}{\mathrm{RHS}}\newcommand{\bnu}{\boldsymbol{\nu}}\newcommand{\interior}{\mathrm{in}\;}\newcommand{\SH}{\mathrm{SH}}\renewcommand{\v}{\boldsymbol{\nu}}\newcommand{\n}{\mathbf{n}}\newcommand{\ssub}{\Subset}\newcommand{\curl}{\mathrm{curl}}
%%% 常微分方程式
\newcommand{\Ei}{\mathrm{Ei}}\newcommand{\sn}{\mathrm{sn}}\newcommand{\wgamma}{\widetilde{\gamma}}
%%% 統計力学
\newcommand{\Ens}{\mathrm{Ens}}
%%% 解析力学
\newcommand{\cl}{\mathrm{cl}}\newcommand{\x}{\boldsymbol{x}}

%%% 統計的因果推論
\newcommand{\Do}{\mathrm{Do}}
%%% 応用統計学
\newcommand{\mrl}{\mathrm{mrl}}
%%% 数理統計
\newcommand{\comb}[2]{\begin{pmatrix}#1\\#2\end{pmatrix}}\newcommand{\bP}{\mathbb{P}}\newcommand{\compsub}{\overset{\textrm{cpt}}{\subset}}\newcommand{\lip}{\textrm{lip}}\newcommand{\BL}{\mathrm{BL}}\newcommand{\G}{\mathbb{G}}\newcommand{\NB}{\mathrm{NB}}\newcommand{\oR}{\o{\R}}\newcommand{\liminfn}{\liminf_{n\to\infty}}\newcommand{\limsupn}{\limsup_{n\to\infty}}\newcommand{\esssup}{\mathrm{ess.sup}}\newcommand{\asto}{\xrightarrow{\as}}\newcommand{\Cov}{\mathrm{Cov}}\newcommand{\cQ}{\mathcal{Q}}\newcommand{\VC}{\mathrm{VC}}\newcommand{\mb}{\mathrm{mb}}\newcommand{\Avar}{\mathrm{Avar}}\newcommand{\bB}{\mathbb{B}}\newcommand{\bW}{\mathbb{W}}\newcommand{\sd}{\mathrm{sd}}\newcommand{\w}[1]{\widehat{#1}}\newcommand{\bZ}{\boldsymbol{Z}}\newcommand{\Bernoulli}{\mathrm{Ber}}\newcommand{\Ber}{\mathrm{Ber}}\newcommand{\Mult}{\mathrm{Mult}}\newcommand{\BPois}{\mathrm{BPois}}\newcommand{\fraks}{\mathfrak{s}}\newcommand{\frakk}{\mathfrak{k}}\newcommand{\IF}{\mathrm{IF}}\newcommand{\bX}{\mathbf{X}}\newcommand{\bx}{\boldsymbol{x}}\newcommand{\indep}{\raisebox{0.05em}{\rotatebox[origin=c]{90}{$\models$}}}\newcommand{\IG}{\mathrm{IG}}\newcommand{\Levy}{\mathrm{Levy}}\newcommand{\MP}{\mathrm{MP}}\newcommand{\Hermite}{\mathrm{Hermite}}\newcommand{\Skellam}{\mathrm{Skellam}}\newcommand{\Dirichlet}{\mathrm{Dirichlet}}\newcommand{\Beta}{\mathrm{Beta}}\newcommand{\bE}{\mathbb{E}}\newcommand{\bG}{\mathbb{G}}\newcommand{\MISE}{\mathrm{MISE}}\newcommand{\logit}{\mathtt{logit}}\newcommand{\expit}{\mathtt{expit}}\newcommand{\cK}{\mathcal{K}}\newcommand{\dl}{\dot{l}}\newcommand{\dotp}{\dot{p}}\newcommand{\wl}{\wt{l}}\newcommand{\Gauss}{\mathrm{Gauss}}\newcommand{\fA}{\mathfrak{A}}\newcommand{\under}{\mathrm{under}\;}\newcommand{\whtheta}{\wh{\theta}}\newcommand{\Em}{\mathrm{Em}}\newcommand{\ztheta}{{\theta_0}}
\newcommand{\rO}{\mathrm{O}}\newcommand{\Bin}{\mathrm{Bin}}\newcommand{\rW}{\mathrm{W}}\newcommand{\rG}{\mathrm{G}}\newcommand{\rB}{\mathrm{B}}\newcommand{\rN}{\mathrm{N}}\newcommand{\rU}{\mathrm{U}}\newcommand{\HG}{\mathrm{HG}}\newcommand{\GAMMA}{\mathrm{Gamma}}\newcommand{\Cauchy}{\mathrm{Cauchy}}\newcommand{\rt}{\mathrm{t}}
\DeclareMathOperator{\erf}{erf}

%%% 圏
\newcommand{\varlim}{\varprojlim}\newcommand{\Hom}{\mathrm{Hom}}\newcommand{\Iso}{\mathrm{Iso}}\newcommand{\Mor}{\mathrm{Mor}}\newcommand{\Isom}{\mathrm{Isom}}\newcommand{\Aut}{\mathrm{Aut}}\newcommand{\End}{\mathrm{End}}\newcommand{\op}{\mathrm{op}}\newcommand{\ev}{\mathrm{ev}}\newcommand{\Ob}{\mathrm{Ob}}\newcommand{\Ar}{\mathrm{Ar}}\newcommand{\Arr}{\mathrm{Arr}}\newcommand{\Set}{\mathrm{Set}}\newcommand{\Grp}{\mathrm{Grp}}\newcommand{\Cat}{\mathrm{Cat}}\newcommand{\Mon}{\mathrm{Mon}}\newcommand{\Ring}{\mathrm{Ring}}\newcommand{\CRing}{\mathrm{CRing}}\newcommand{\Ab}{\mathrm{Ab}}\newcommand{\Pos}{\mathrm{Pos}}\newcommand{\Vect}{\mathrm{Vect}}\newcommand{\FinVect}{\mathrm{FinVect}}\newcommand{\FinSet}{\mathrm{FinSet}}\newcommand{\FinMeas}{\mathrm{FinMeas}}\newcommand{\OmegaAlg}{\Omega\text{-}\mathrm{Alg}}\newcommand{\OmegaEAlg}{(\Omega,E)\text{-}\mathrm{Alg}}\newcommand{\Fun}{\mathrm{Fun}}\newcommand{\Func}{\mathrm{Func}}\newcommand{\Alg}{\mathrm{Alg}} %代数の圏
\newcommand{\CAlg}{\mathrm{CAlg}} %可換代数の圏
\newcommand{\Met}{\mathrm{Met}} %Metric space & Contraction maps
\newcommand{\Rel}{\mathrm{Rel}} %Sets & relation
\newcommand{\Bool}{\mathrm{Bool}}\newcommand{\CABool}{\mathrm{CABool}}\newcommand{\CompBoolAlg}{\mathrm{CompBoolAlg}}\newcommand{\BoolAlg}{\mathrm{BoolAlg}}\newcommand{\BoolRng}{\mathrm{BoolRng}}\newcommand{\HeytAlg}{\mathrm{HeytAlg}}\newcommand{\CompHeytAlg}{\mathrm{CompHeytAlg}}\newcommand{\Lat}{\mathrm{Lat}}\newcommand{\CompLat}{\mathrm{CompLat}}\newcommand{\SemiLat}{\mathrm{SemiLat}}\newcommand{\Stone}{\mathrm{Stone}}\newcommand{\Mfd}{\mathrm{Mfd}}\newcommand{\LieAlg}{\mathrm{LieAlg}}
\newcommand{\Sob}{\mathrm{Sob}} %Sober space & continuous map
\newcommand{\Op}{\mathrm{Op}} %Category of open subsets
\newcommand{\Sh}{\mathrm{Sh}} %Category of sheave
\newcommand{\PSh}{\mathrm{PSh}} %Category of presheave, PSh(C)=[C^op,set]のこと
\newcommand{\Conv}{\mathrm{Conv}} %Convergence spaceの圏
\newcommand{\Unif}{\mathrm{Unif}} %一様空間と一様連続写像の圏
\newcommand{\Frm}{\mathrm{Frm}} %フレームとフレームの射
\newcommand{\Locale}{\mathrm{Locale}} %その反対圏
\newcommand{\Diff}{\mathrm{Diff}} %滑らかな多様体の圏
\newcommand{\Quiv}{\mathrm{Quiv}} %Quiverの圏
\newcommand{\B}{\mathcal{B}}\newcommand{\Span}{\mathrm{Span}}\newcommand{\Corr}{\mathrm{Corr}}\newcommand{\Decat}{\mathrm{Decat}}\newcommand{\Rep}{\mathrm{Rep}}\newcommand{\Grpd}{\mathrm{Grpd}}\newcommand{\sSet}{\mathrm{sSet}}\newcommand{\Mod}{\mathrm{Mod}}\newcommand{\SmoothMnf}{\mathrm{SmoothMnf}}\newcommand{\coker}{\mathrm{coker}}\newcommand{\Ord}{\mathrm{Ord}}\newcommand{\eq}{\mathrm{eq}}\newcommand{\coeq}{\mathrm{coeq}}\newcommand{\act}{\mathrm{act}}

%%%%%%%%%%%%%%% 定理環境(足助先生ありがとうございます) %%%%%%%%%%%%%%%

\everymath{\displaystyle}
\renewcommand{\proofname}{\bf\underline{[証明]}}
\renewcommand{\thefootnote}{\dag\arabic{footnote}} %足助さんからもらった.どうなるんだ?
\renewcommand{\qedsymbol}{$\blacksquare$}

\renewcommand{\labelenumi}{(\arabic{enumi})} %(1),(2),...がデフォルトであって欲しい
\renewcommand{\labelenumii}{(\alph{enumii})}
\renewcommand{\labelenumiii}{(\roman{enumiii})}

\newtheoremstyle{StatementsWithUnderline}% ?name?
{3pt}% ?Space above? 1
{3pt}% ?Space below? 1
{}% ?Body font?
{}% ?Indent amount? 2
{\bfseries}% ?Theorem head font?
{\textbf{.}}% ?Punctuation after theorem head?
{.5em}% ?Space after theorem head? 3
{\textbf{\underline{\textup{#1~\thetheorem{}}}}\;\thmnote{(#3)}}% ?Theorem head spec (can be left empty, meaning ‘normal’)?

\usepackage{etoolbox}
\AtEndEnvironment{example}{\hfill\ensuremath{\Box}}
\AtEndEnvironment{observation}{\hfill\ensuremath{\Box}}

\theoremstyle{StatementsWithUnderline}
    \newtheorem{theorem}{定理}[section]
    \newtheorem{axiom}[theorem]{公理}
    \newtheorem{corollary}[theorem]{系}
    \newtheorem{proposition}[theorem]{命題}
    \newtheorem{lemma}[theorem]{補題}
    \newtheorem{definition}[theorem]{定義}
    \newtheorem{problem}[theorem]{問題}
    \newtheorem{exercise}[theorem]{Exercise}
\theoremstyle{definition}
    \newtheorem{issue}{論点}
    \newtheorem*{proposition*}{命題}
    \newtheorem*{lemma*}{補題}
    \newtheorem*{consideration*}{考察}
    \newtheorem*{theorem*}{定理}
    \newtheorem*{remarks*}{要諦}
    \newtheorem{example}[theorem]{例}
    \newtheorem{notation}[theorem]{記法}
    \newtheorem*{notation*}{記法}
    \newtheorem{assumption}[theorem]{仮定}
    \newtheorem{question}[theorem]{問}
    \newtheorem{counterexample}[theorem]{反例}
    \newtheorem{reidai}[theorem]{例題}
    \newtheorem{ruidai}[theorem]{類題}
    \newtheorem{algorithm}[theorem]{算譜}
    \newtheorem*{feels*}{所感}
    \newtheorem*{solution*}{\bf{[解]}}
    \newtheorem{discussion}[theorem]{議論}
    \newtheorem{synopsis}[theorem]{要約}
    \newtheorem{cited}[theorem]{引用}
    \newtheorem{remark}[theorem]{注}
    \newtheorem{remarks}[theorem]{要諦}
    \newtheorem{memo}[theorem]{メモ}
    \newtheorem{image}[theorem]{描像}
    \newtheorem{observation}[theorem]{観察}
    \newtheorem{universality}[theorem]{普遍性} %非自明な例外がない.
    \newtheorem{universal tendency}[theorem]{普遍傾向} %例外が有意に少ない.
    \newtheorem{hypothesis}[theorem]{仮説} %実験で説明されていない理論.
    \newtheorem{theory}[theorem]{理論} %実験事実とその(さしあたり)整合的な説明.
    \newtheorem{fact}[theorem]{実験事実}
    \newtheorem{model}[theorem]{模型}
    \newtheorem{explanation}[theorem]{説明} %理論による実験事実の説明
    \newtheorem{anomaly}[theorem]{理論の限界}
    \newtheorem{application}[theorem]{応用例}
    \newtheorem{method}[theorem]{手法} %実験手法など,技術的問題.
    \newtheorem{test}[theorem]{検定}
    \newtheorem{terms}[theorem]{用語}
    \newtheorem{solution}[theorem]{解法}
    \newtheorem{history}[theorem]{歴史}
    \newtheorem{usage}[theorem]{用語法}
    \newtheorem{research}[theorem]{研究}
    \newtheorem{shishin}[theorem]{指針}
    \newtheorem{yodan}[theorem]{余談}
    \newtheorem{construction}[theorem]{構成}
    \newtheorem{motivation}[theorem]{動機}
    \newtheorem{context}[theorem]{背景}
    \newtheorem{advantage}[theorem]{利点}
    \newtheorem*{definition*}{定義}
    \newtheorem*{remark*}{注意}
    \newtheorem*{question*}{問}
    \newtheorem*{problem*}{問題}
    \newtheorem*{axiom*}{公理}
    \newtheorem*{example*}{例}
    \newtheorem*{corollary*}{系}
    \newtheorem*{shishin*}{指針}
    \newtheorem*{yodan*}{余談}
    \newtheorem*{kadai*}{課題}

\raggedbottom
\allowdisplaybreaks
%%%%%%%%%%%%%%%% 数理文書の組版 %%%%%%%%%%%%%%%

\usepackage{mathtools} %内部でamsmathを呼び出すことに注意.
%\mathtoolsset{showonlyrefs=true} %labelを附した数式にのみ附番される設定.
\usepackage{amsfonts} %mathfrak, mathcal, mathbbなど.
\usepackage{amsthm} %定理環境.
\usepackage{amssymb} %AMSFontsを使うためのパッケージ.
\usepackage{ascmac} %screen, itembox, shadebox環境.全てLATEX2εの標準機能の範囲で作られたもの.
\usepackage{comment} %comment環境を用いて,複数行をcomment outできるようにするpackage
\usepackage{wrapfig} %図の周りに文字をwrapさせることができる.詳細な制御ができる.
\usepackage[usenames, dvipsnames]{xcolor} %xcolorはcolorの拡張.optionの意味はdvipsnamesはLoad a set of predefined colors. forestgreenなどの色が追加されている.usenamesはobsoleteとだけ書いてあった.
\setcounter{tocdepth}{2} %目次に表示される深さ.2はsubsectionまで
\usepackage{multicol} %\begin{multicols}{2}環境で途中からmulticolumnに出来る.
\usepackage{mathabx}\newcommand{\wc}{\widecheck} %\widecheckなどのフォントパッケージ

%%%%%%%%%%%%%%% フォント %%%%%%%%%%%%%%%

\usepackage{textcomp, mathcomp} %Text Companionとは,T1 encodingに入らなかった文字群.これを使うためのパッケージ.\textsectionでブルバキに!
\usepackage[T1]{fontenc} %8bitエンコーディングにする.comp系拡張数学文字の動作が安定する.

%%%%%%%%%%%%%%% 一般文書の組版 %%%%%%%%%%%%%%%

\definecolor{花緑青}{cmyk}{1,0.07,0.10,0.10}\definecolor{サーモンピンク}{cmyk}{0,0.65,0.65,0.05}\definecolor{暗中模索}{rgb}{0.2,0.2,0.2}
\usepackage{url}\usepackage[dvipdfmx,colorlinks,linkcolor=花緑青,urlcolor=花緑青,citecolor=花緑青]{hyperref} %生成されるPDFファイルにおいて、\tableofcontentsによって書き出された目次をクリックすると該当する見出しへジャンプしたり、さらには、\label{ラベル名}を番号で参照する\ref{ラベル名}やthebibliography環境において\bibitem{ラベル名}を文献番号で参照する\cite{ラベル名}においても番号をクリックすると該当箇所にジャンプする.囲み枠はダサいので,colorlinksで囲み廃止し,リンク自体に色を付けることにした.
\usepackage{pxjahyper} %pxrubrica同様,八登崇之さん.hyperrefは日本語pLaTeXに最適化されていないから,hyperrefとセットで,(u)pLaTeX+hyperref+dvipdfmxの組み合わせで日本語を含む「しおり」をもつPDF文書を作成する場合に必要となる機能を提供する
\usepackage{ulem} %取り消し線を引くためのパッケージ
\usepackage{pxrubrica} %日本語にルビをふる.八登崇之(やとうたかゆき)氏による.

%%%%%%%%%%%%%%% 科学文書の組版 %%%%%%%%%%%%%%%

\usepackage[version=4]{mhchem} %化学式をTikZで簡単に書くためのパッケージ.
\usepackage{chemfig} %化学構造式をTikZで描くためのパッケージ.
\usepackage{siunitx} %IS単位を書くためのパッケージ

%%%%%%%%%%%%%%% 作図 %%%%%%%%%%%%%%%

\usepackage{tikz}\usetikzlibrary{positioning,automata}\usepackage{tikz-cd}\usepackage[all]{xy}
\def\objectstyle{\displaystyle} %デフォルトではxymatrix中の数式が文中数式モードになるので,それを直す.\labelstyleも同様にxy packageの中で定義されており,文中数式モードになっている.

\usepackage{graphicx} %rotatebox, scalebox, reflectbox, resizeboxなどのコマンドや,図表の読み込み\includegraphicsを司る.graphics というパッケージもありますが,graphicx はこれを高機能にしたものと考えて結構です(ただし graphicx は内部で graphics を読み込みます)
\usepackage[top=15truemm,bottom=15truemm,left=10truemm,right=10truemm]{geometry} %足助さんからもらったオプション

%%%%%%%%%%%%%%% 参照 %%%%%%%%%%%%%%%
%参考文献リストを出力したい箇所に\bibliography{../mathematics.bib}を追記すると良い.

%\bibliographystyle{jplain}
%\bibliographystyle{jname}
\bibliographystyle{apalike}

%%%%%%%%%%%%%%% 計算機文書の組版 %%%%%%%%%%%%%%%

\usepackage[breakable]{tcolorbox} %加藤晃史さんがフル活用していたtcolorboxを,途中改ページ可能で.
\tcbuselibrary{theorems} %https://qiita.com/t_kemmochi/items/483b8fcdb5db8d1f5d5e
\usepackage{enumerate} %enumerate環境を凝らせる.

\usepackage{listings} %ソースコードを表示できる環境.多分もっといい方法ある.
\usepackage{jvlisting} %日本語のコメントアウトをする場合jlistingが必要
\lstset{ %ここからソースコードの表示に関する設定.lstlisting環境では,[caption=hoge,label=fuga]などのoptionを付けられる.
%[escapechar=!]とすると,LaTeXコマンドを使える.
  basicstyle={\ttfamily},
  identifierstyle={\small},
  commentstyle={\smallitshape},
  keywordstyle={\small\bfseries},
  ndkeywordstyle={\small},
  stringstyle={\small\ttfamily},
  frame={tb},
  breaklines=true,
  columns=[l]{fullflexible},
  numbers=left,
  xrightmargin=0zw,
  xleftmargin=3zw,
  numberstyle={\scriptsize},
  stepnumber=1,
  numbersep=1zw,
  lineskip=-0.5ex
}
%\makeatletter %caption番号を「[chapter番号].[section番号].[subsection番号]-[そのsubsection内においてn番目]」に変更
%    \AtBeginDocument{
%    \renewcommand*{\thelstlisting}{\arabic{chapter}.\arabic{section}.\arabic{lstlisting}}
%    \@addtoreset{lstlisting}{section}
%    }
%\makeatother
\renewcommand{\lstlistingname}{算譜} %caption名を"program"に変更

\newtcolorbox{tbox}[3][]{%
colframe=#2,colback=#2!10,coltitle=#2!20!black,title={#3},#1}

% 証明内の文字が小さくなる環境.
\newenvironment{Proof}[1][\bf\underline{[証明]}]{\proof[#1]\color{darkgray}}{\endproof}

%%%%%%%%%%%%%%% 数学記号のマクロ %%%%%%%%%%%%%%%

%%% 括弧類
\newcommand{\abs}[1]{\lvert#1\rvert}\newcommand{\Abs}[1]{\left|#1\right|}\newcommand{\norm}[1]{\|#1\|}\newcommand{\Norm}[1]{\left\|#1\right\|}\newcommand{\Brace}[1]{\left\{#1\right\}}\newcommand{\BRace}[1]{\biggl\{#1\biggr\}}\newcommand{\paren}[1]{\left(#1\right)}\newcommand{\Paren}[1]{\biggr(#1\biggl)}\newcommand{\bracket}[1]{\langle#1\rangle}\newcommand{\brac}[1]{\langle#1\rangle}\newcommand{\Bracket}[1]{\left\langle#1\right\rangle}\newcommand{\Brac}[1]{\left\langle#1\right\rangle}\newcommand{\bra}[1]{\left\langle#1\right|}\newcommand{\ket}[1]{\left|#1\right\rangle}\newcommand{\Square}[1]{\left[#1\right]}\newcommand{\SQuare}[1]{\biggl[#1\biggr]}
\renewcommand{\o}[1]{\overline{#1}}\renewcommand{\u}[1]{\underline{#1}}\newcommand{\wt}[1]{\widetilde{#1}}\newcommand{\wh}[1]{\widehat{#1}}
\newcommand{\pp}[2]{\frac{\partial #1}{\partial #2}}\newcommand{\ppp}[3]{\frac{\partial #1}{\partial #2\partial #3}}\newcommand{\dd}[2]{\frac{d #1}{d #2}}
\newcommand{\floor}[1]{\lfloor#1\rfloor}\newcommand{\Floor}[1]{\left\lfloor#1\right\rfloor}\newcommand{\ceil}[1]{\lceil#1\rceil}
\newcommand{\ocinterval}[1]{(#1]}\newcommand{\cointerval}[1]{[#1)}\newcommand{\COinterval}[1]{\left[#1\right)}


%%% 予約語
\renewcommand{\iff}{\;\mathrm{iff}\;}
\newcommand{\False}{\mathrm{False}}\newcommand{\True}{\mathrm{True}}
\newcommand{\otherwise}{\mathrm{otherwise}}
\newcommand{\st}{\;\mathrm{s.t.}\;}

%%% 略記
\newcommand{\M}{\mathcal{M}}\newcommand{\cF}{\mathcal{F}}\newcommand{\cD}{\mathcal{D}}\newcommand{\fX}{\mathfrak{X}}\newcommand{\fY}{\mathfrak{Y}}\newcommand{\fZ}{\mathfrak{Z}}\renewcommand{\H}{\mathcal{H}}\newcommand{\fH}{\mathfrak{H}}\newcommand{\bH}{\mathbb{H}}\newcommand{\id}{\mathrm{id}}\newcommand{\A}{\mathcal{A}}\newcommand{\U}{\mathfrak{U}}
\newcommand{\lmd}{\lambda}
\newcommand{\Lmd}{\Lambda}

%%% 矢印類
\newcommand{\iso}{\xrightarrow{\,\smash{\raisebox{-0.45ex}{\ensuremath{\scriptstyle\sim}}}\,}}
\newcommand{\Lrarrow}{\;\;\Leftrightarrow\;\;}

%%% 注記
\newcommand{\rednote}[1]{\textcolor{red}{#1}}

% ノルム位相についての閉包 https://newbedev.com/how-to-make-double-overline-with-less-vertical-displacement
\makeatletter
\newcommand{\dbloverline}[1]{\overline{\dbl@overline{#1}}}
\newcommand{\dbl@overline}[1]{\mathpalette\dbl@@overline{#1}}
\newcommand{\dbl@@overline}[2]{%
  \begingroup
  \sbox\z@{$\m@th#1\overline{#2}$}%
  \ht\z@=\dimexpr\ht\z@-2\dbl@adjust{#1}\relax
  \box\z@
  \ifx#1\scriptstyle\kern-\scriptspace\else
  \ifx#1\scriptscriptstyle\kern-\scriptspace\fi\fi
  \endgroup
}
\newcommand{\dbl@adjust}[1]{%
  \fontdimen8
  \ifx#1\displaystyle\textfont\else
  \ifx#1\textstyle\textfont\else
  \ifx#1\scriptstyle\scriptfont\else
  \scriptscriptfont\fi\fi\fi 3
}
\makeatother
\newcommand{\oo}[1]{\dbloverline{#1}}

% hslashの他の文字Ver.
\newcommand{\hslashslash}{%
    \scalebox{1.2}{--
    }%
}
\newcommand{\dslash}{%
  {%
    \vphantom{d}%
    \ooalign{\kern.05em\smash{\hslashslash}\hidewidth\cr$d$\cr}%
    \kern.05em
  }%
}
\newcommand{\dint}{%
  {%
    \vphantom{d}%
    \ooalign{\kern.05em\smash{\hslashslash}\hidewidth\cr$\int$\cr}%
    \kern.05em
  }%
}
\newcommand{\dL}{%
  {%
    \vphantom{d}%
    \ooalign{\kern.05em\smash{\hslashslash}\hidewidth\cr$L$\cr}%
    \kern.05em
  }%
}

%%% 演算子
\DeclareMathOperator{\grad}{\mathrm{grad}}\DeclareMathOperator{\rot}{\mathrm{rot}}\DeclareMathOperator{\divergence}{\mathrm{div}}\DeclareMathOperator{\tr}{\mathrm{tr}}\newcommand{\pr}{\mathrm{pr}}
\newcommand{\Map}{\mathrm{Map}}\newcommand{\dom}{\mathrm{Dom}\;}\newcommand{\cod}{\mathrm{Cod}\;}\newcommand{\supp}{\mathrm{supp}\;}


%%% 線型代数学
\newcommand{\vctr}[2]{\begin{pmatrix}#1\\#2\end{pmatrix}}\newcommand{\vctrr}[3]{\begin{pmatrix}#1\\#2\\#3\end{pmatrix}}\newcommand{\mtrx}[4]{\begin{pmatrix}#1&#2\\#3&#4\end{pmatrix}}\newcommand{\smtrx}[4]{\paren{\begin{smallmatrix}#1&#2\\#3&#4\end{smallmatrix}}}\newcommand{\Ker}{\mathrm{Ker}\;}\newcommand{\Coker}{\mathrm{Coker}\;}\newcommand{\Coim}{\mathrm{Coim}\;}\DeclareMathOperator{\rank}{\mathrm{rank}}\newcommand{\lcm}{\mathrm{lcm}}\newcommand{\sgn}{\mathrm{sgn}\,}\newcommand{\GL}{\mathrm{GL}}\newcommand{\SL}{\mathrm{SL}}\newcommand{\alt}{\mathrm{alt}}
%%% 複素解析学
\renewcommand{\Re}{\mathrm{Re}\;}\renewcommand{\Im}{\mathrm{Im}\;}\newcommand{\Gal}{\mathrm{Gal}}\newcommand{\PGL}{\mathrm{PGL}}\newcommand{\PSL}{\mathrm{PSL}}\newcommand{\Log}{\mathrm{Log}\,}\newcommand{\Res}{\mathrm{Res}\,}\newcommand{\on}{\mathrm{on}\;}\newcommand{\hatC}{\widehat{\C}}\newcommand{\hatR}{\hat{\R}}\newcommand{\PV}{\mathrm{P.V.}}\newcommand{\diam}{\mathrm{diam}}\newcommand{\Area}{\mathrm{Area}}\newcommand{\Lap}{\Laplace}\newcommand{\f}{\mathbf{f}}\newcommand{\cR}{\mathcal{R}}\newcommand{\const}{\mathrm{const.}}\newcommand{\Om}{\Omega}\newcommand{\Cinf}{C^\infty}\newcommand{\ep}{\epsilon}\newcommand{\dist}{\mathrm{dist}}\newcommand{\opart}{\o{\partial}}\newcommand{\Length}{\mathrm{Length}}
%%% 集合と位相
\renewcommand{\O}{\mathcal{O}}\renewcommand{\S}{\mathcal{S}}\renewcommand{\U}{\mathcal{U}}\newcommand{\V}{\mathcal{V}}\renewcommand{\P}{\mathcal{P}}\newcommand{\R}{\mathbb{R}}\newcommand{\N}{\mathbb{N}}\newcommand{\C}{\mathbb{C}}\newcommand{\Z}{\mathbb{Z}}\newcommand{\Q}{\mathbb{Q}}\newcommand{\TV}{\mathrm{TV}}\newcommand{\ORD}{\mathrm{ORD}}\newcommand{\Tr}{\mathrm{Tr}}\newcommand{\Card}{\mathrm{Card}\;}\newcommand{\Top}{\mathrm{Top}}\newcommand{\Disc}{\mathrm{Disc}}\newcommand{\Codisc}{\mathrm{Codisc}}\newcommand{\CoDisc}{\mathrm{CoDisc}}\newcommand{\Ult}{\mathrm{Ult}}\newcommand{\ord}{\mathrm{ord}}\newcommand{\maj}{\mathrm{maj}}\newcommand{\bS}{\mathbb{S}}\newcommand{\PConn}{\mathrm{PConn}}

%%% 形式言語理論
\newcommand{\REGEX}{\mathrm{REGEX}}\newcommand{\RE}{\mathbf{RE}}
%%% Graph Theory
\newcommand{\SimpGph}{\mathrm{SimpGph}}\newcommand{\Gph}{\mathrm{Gph}}\newcommand{\mult}{\mathrm{mult}}\newcommand{\inv}{\mathrm{inv}}

%%% 多様体
\newcommand{\Der}{\mathrm{Der}}\newcommand{\osub}{\overset{\mathrm{open}}{\subset}}\newcommand{\osup}{\overset{\mathrm{open}}{\supset}}\newcommand{\al}{\alpha}\newcommand{\K}{\mathbb{K}}\newcommand{\Sp}{\mathrm{Sp}}\newcommand{\g}{\mathfrak{g}}\newcommand{\h}{\mathfrak{h}}\newcommand{\Exp}{\mathrm{Exp}\;}\newcommand{\Imm}{\mathrm{Imm}}\newcommand{\Imb}{\mathrm{Imb}}\newcommand{\codim}{\mathrm{codim}\;}\newcommand{\Gr}{\mathrm{Gr}}
%%% 代数
\newcommand{\Ad}{\mathrm{Ad}}\newcommand{\finsupp}{\mathrm{fin\;supp}}\newcommand{\SO}{\mathrm{SO}}\newcommand{\SU}{\mathrm{SU}}\newcommand{\acts}{\curvearrowright}\newcommand{\mono}{\hookrightarrow}\newcommand{\epi}{\twoheadrightarrow}\newcommand{\Stab}{\mathrm{Stab}}\newcommand{\nor}{\mathrm{nor}}\newcommand{\T}{\mathbb{T}}\newcommand{\Aff}{\mathrm{Aff}}\newcommand{\rsub}{\triangleleft}\newcommand{\rsup}{\triangleright}\newcommand{\subgrp}{\overset{\mathrm{subgrp}}{\subset}}\newcommand{\Ext}{\mathrm{Ext}}\newcommand{\sbs}{\subset}\newcommand{\sps}{\supset}\newcommand{\In}{\mathrm{in}\;}\newcommand{\Tor}{\mathrm{Tor}}\newcommand{\p}{\b{p}}\newcommand{\q}{\mathfrak{q}}\newcommand{\m}{\mathfrak{m}}\newcommand{\cS}{\mathcal{S}}\newcommand{\Frac}{\mathrm{Frac}\,}\newcommand{\Spec}{\mathrm{Spec}\,}\newcommand{\bA}{\mathbb{A}}\newcommand{\Sym}{\mathrm{Sym}}\newcommand{\Ann}{\mathrm{Ann}}\newcommand{\Her}{\mathrm{Her}}\newcommand{\Bil}{\mathrm{Bil}}\newcommand{\Ses}{\mathrm{Ses}}\newcommand{\FVS}{\mathrm{FVS}}
%%% 代数的位相幾何学
\newcommand{\Ho}{\mathrm{Ho}}\newcommand{\CW}{\mathrm{CW}}\newcommand{\lc}{\mathrm{lc}}\newcommand{\cg}{\mathrm{cg}}\newcommand{\Fib}{\mathrm{Fib}}\newcommand{\Cyl}{\mathrm{Cyl}}\newcommand{\Ch}{\mathrm{Ch}}
%%% 微分幾何学
\newcommand{\rE}{\mathrm{E}}\newcommand{\e}{\b{e}}\renewcommand{\k}{\b{k}}\newcommand{\Christ}[2]{\begin{Bmatrix}#1\\#2\end{Bmatrix}}\renewcommand{\Vec}[1]{\overrightarrow{\mathrm{#1}}}\newcommand{\hen}[1]{\mathrm{#1}}\renewcommand{\b}[1]{\boldsymbol{#1}}

%%% 函数解析
\newcommand{\HS}{\mathrm{HS}}\newcommand{\loc}{\mathrm{loc}}\newcommand{\Lh}{\mathrm{L.h.}}\newcommand{\Epi}{\mathrm{Epi}\;}\newcommand{\slim}{\mathrm{slim}}\newcommand{\Ban}{\mathrm{Ban}}\newcommand{\Hilb}{\mathrm{Hilb}}\newcommand{\Ex}{\mathrm{Ex}}\newcommand{\Co}{\mathrm{Co}}\newcommand{\sa}{\mathrm{sa}}\newcommand{\nnorm}[1]{{\left\vert\kern-0.25ex\left\vert\kern-0.25ex\left\vert #1 \right\vert\kern-0.25ex\right\vert\kern-0.25ex\right\vert}}\newcommand{\dvol}{\mathrm{dvol}}\newcommand{\Sconv}{\mathrm{Sconv}}\newcommand{\I}{\mathcal{I}}\newcommand{\nonunital}{\mathrm{nu}}\newcommand{\cpt}{\mathrm{cpt}}\newcommand{\lcpt}{\mathrm{lcpt}}\newcommand{\com}{\mathrm{com}}\newcommand{\Haus}{\mathrm{Haus}}\newcommand{\proper}{\mathrm{proper}}\newcommand{\infinity}{\mathrm{inf}}\newcommand{\TVS}{\mathrm{TVS}}\newcommand{\ess}{\mathrm{ess}}\newcommand{\ext}{\mathrm{ext}}\newcommand{\Index}{\mathrm{Index}\;}\newcommand{\SSR}{\mathrm{SSR}}\newcommand{\vs}{\mathrm{vs.}}\newcommand{\fM}{\mathfrak{M}}\newcommand{\EDM}{\mathrm{EDM}}\newcommand{\Tw}{\mathrm{Tw}}\newcommand{\fC}{\mathfrak{C}}\newcommand{\bn}{\boldsymbol{n}}\newcommand{\br}{\boldsymbol{r}}\newcommand{\Lam}{\Lambda}\newcommand{\lam}{\lambda}\newcommand{\one}{\mathbf{1}}\newcommand{\dae}{\text{-a.e.}}\newcommand{\das}{\text{-a.s.}}\newcommand{\td}{\text{-}}\newcommand{\RM}{\mathrm{RM}}\newcommand{\BV}{\mathrm{BV}}\newcommand{\normal}{\mathrm{normal}}\newcommand{\lub}{\mathrm{lub}\;}\newcommand{\Graph}{\mathrm{Graph}}\newcommand{\Ascent}{\mathrm{Ascent}}\newcommand{\Descent}{\mathrm{Descent}}\newcommand{\BIL}{\mathrm{BIL}}\newcommand{\fL}{\mathfrak{L}}\newcommand{\De}{\Delta}
%%% 積分論
\newcommand{\calA}{\mathcal{A}}\newcommand{\calB}{\mathcal{B}}\newcommand{\D}{\mathcal{D}}\newcommand{\Y}{\mathcal{Y}}\newcommand{\calC}{\mathcal{C}}\renewcommand{\ae}{\mathrm{a.e.}\;}\newcommand{\cZ}{\mathcal{Z}}\newcommand{\fF}{\mathfrak{F}}\newcommand{\fI}{\mathfrak{I}}\newcommand{\E}{\mathcal{E}}\newcommand{\sMap}{\sigma\textrm{-}\mathrm{Map}}\DeclareMathOperator*{\argmax}{arg\,max}\DeclareMathOperator*{\argmin}{arg\,min}\newcommand{\cC}{\mathcal{C}}\newcommand{\comp}{\complement}\newcommand{\J}{\mathcal{J}}\newcommand{\sumN}[1]{\sum_{#1\in\N}}\newcommand{\cupN}[1]{\cup_{#1\in\N}}\newcommand{\capN}[1]{\cap_{#1\in\N}}\newcommand{\Sum}[1]{\sum_{#1=1}^\infty}\newcommand{\sumn}{\sum_{n=1}^\infty}\newcommand{\summ}{\sum_{m=1}^\infty}\newcommand{\sumk}{\sum_{k=1}^\infty}\newcommand{\sumi}{\sum_{i=1}^\infty}\newcommand{\sumj}{\sum_{j=1}^\infty}\newcommand{\cupn}{\cup_{n=1}^\infty}\newcommand{\capn}{\cap_{n=1}^\infty}\newcommand{\cupk}{\cup_{k=1}^\infty}\newcommand{\cupi}{\cup_{i=1}^\infty}\newcommand{\cupj}{\cup_{j=1}^\infty}\newcommand{\limn}{\lim_{n\to\infty}}\renewcommand{\l}{\mathcal{l}}\renewcommand{\L}{\mathcal{L}}\newcommand{\Cl}{\mathrm{Cl}}\newcommand{\cN}{\mathcal{N}}\newcommand{\Ae}{\textrm{-a.e.}\;}\newcommand{\csub}{\overset{\textrm{closed}}{\subset}}\newcommand{\csup}{\overset{\textrm{closed}}{\supset}}\newcommand{\wB}{\wt{B}}\newcommand{\cG}{\mathcal{G}}\newcommand{\Lip}{\mathrm{Lip}}\DeclareMathOperator{\Dom}{\mathrm{Dom}}\newcommand{\AC}{\mathrm{AC}}\newcommand{\Mol}{\mathrm{Mol}}
%%% Fourier解析
\newcommand{\Pe}{\mathrm{Pe}}\newcommand{\wR}{\wh{\mathbb{\R}}}\newcommand*{\Laplace}{\mathop{}\!\mathbin\bigtriangleup}\newcommand*{\DAlambert}{\mathop{}\!\mathbin\Box}\newcommand{\bT}{\mathbb{T}}\newcommand{\dx}{\dslash x}\newcommand{\dt}{\dslash t}\newcommand{\ds}{\dslash s}
%%% 数値解析
\newcommand{\round}{\mathrm{round}}\newcommand{\cond}{\mathrm{cond}}\newcommand{\diag}{\mathrm{diag}}
\newcommand{\Adj}{\mathrm{Adj}}\newcommand{\Pf}{\mathrm{Pf}}\newcommand{\Sg}{\mathrm{Sg}}

%%% 確率論
\newcommand{\Prob}{\mathrm{Prob}}\newcommand{\X}{\mathcal{X}}\newcommand{\Meas}{\mathrm{Meas}}\newcommand{\as}{\;\mathrm{a.s.}}\newcommand{\io}{\;\mathrm{i.o.}}\newcommand{\fe}{\;\mathrm{f.e.}}\newcommand{\F}{\mathcal{F}}\newcommand{\bF}{\mathbb{F}}\newcommand{\W}{\mathcal{W}}\newcommand{\Pois}{\mathrm{Pois}}\newcommand{\iid}{\mathrm{i.i.d.}}\newcommand{\wconv}{\rightsquigarrow}\newcommand{\Var}{\mathrm{Var}}\newcommand{\xrightarrown}{\xrightarrow{n\to\infty}}\newcommand{\au}{\mathrm{au}}\newcommand{\cT}{\mathcal{T}}\newcommand{\wto}{\overset{w}{\to}}\newcommand{\dto}{\overset{d}{\to}}\newcommand{\pto}{\overset{p}{\to}}\newcommand{\vto}{\overset{v}{\to}}\newcommand{\Cont}{\mathrm{Cont}}\newcommand{\stably}{\mathrm{stably}}\newcommand{\Np}{\mathbb{N}^+}\newcommand{\oM}{\overline{\mathcal{M}}}\newcommand{\fP}{\mathfrak{P}}\newcommand{\sign}{\mathrm{sign}}\DeclareMathOperator{\Div}{Div}
\newcommand{\bD}{\mathbb{D}}\newcommand{\fW}{\mathfrak{W}}\newcommand{\DL}{\mathcal{D}\mathcal{L}}\renewcommand{\r}[1]{\mathrm{#1}}\newcommand{\rC}{\mathrm{C}}
%%% 情報理論
\newcommand{\bit}{\mathrm{bit}}\DeclareMathOperator{\sinc}{sinc}
%%% 量子論
\newcommand{\err}{\mathrm{err}}
%%% 最適化
\newcommand{\varparallel}{\mathbin{\!/\mkern-5mu/\!}}\newcommand{\Minimize}{\text{Minimize}}\newcommand{\subjectto}{\text{subject to}}\newcommand{\Ri}{\mathrm{Ri}}\newcommand{\Cone}{\mathrm{Cone}}\newcommand{\Int}{\mathrm{Int}}
%%% 数理ファイナンス
\newcommand{\pre}{\mathrm{pre}}\newcommand{\om}{\omega}

%%% 偏微分方程式
\let\div\relax
\DeclareMathOperator{\div}{div}\newcommand{\del}{\partial}
\newcommand{\LHS}{\mathrm{LHS}}\newcommand{\RHS}{\mathrm{RHS}}\newcommand{\bnu}{\boldsymbol{\nu}}\newcommand{\interior}{\mathrm{in}\;}\newcommand{\SH}{\mathrm{SH}}\renewcommand{\v}{\boldsymbol{\nu}}\newcommand{\n}{\mathbf{n}}\newcommand{\ssub}{\Subset}\newcommand{\curl}{\mathrm{curl}}
%%% 常微分方程式
\newcommand{\Ei}{\mathrm{Ei}}\newcommand{\sn}{\mathrm{sn}}\newcommand{\wgamma}{\widetilde{\gamma}}
%%% 統計力学
\newcommand{\Ens}{\mathrm{Ens}}
%%% 解析力学
\newcommand{\cl}{\mathrm{cl}}\newcommand{\x}{\boldsymbol{x}}

%%% 統計的因果推論
\newcommand{\Do}{\mathrm{Do}}
%%% 応用統計学
\newcommand{\mrl}{\mathrm{mrl}}
%%% 数理統計
\newcommand{\comb}[2]{\begin{pmatrix}#1\\#2\end{pmatrix}}\newcommand{\bP}{\mathbb{P}}\newcommand{\compsub}{\overset{\textrm{cpt}}{\subset}}\newcommand{\lip}{\textrm{lip}}\newcommand{\BL}{\mathrm{BL}}\newcommand{\G}{\mathbb{G}}\newcommand{\NB}{\mathrm{NB}}\newcommand{\oR}{\o{\R}}\newcommand{\liminfn}{\liminf_{n\to\infty}}\newcommand{\limsupn}{\limsup_{n\to\infty}}\newcommand{\esssup}{\mathrm{ess.sup}}\newcommand{\asto}{\xrightarrow{\as}}\newcommand{\Cov}{\mathrm{Cov}}\newcommand{\cQ}{\mathcal{Q}}\newcommand{\VC}{\mathrm{VC}}\newcommand{\mb}{\mathrm{mb}}\newcommand{\Avar}{\mathrm{Avar}}\newcommand{\bB}{\mathbb{B}}\newcommand{\bW}{\mathbb{W}}\newcommand{\sd}{\mathrm{sd}}\newcommand{\w}[1]{\widehat{#1}}\newcommand{\bZ}{\boldsymbol{Z}}\newcommand{\Bernoulli}{\mathrm{Ber}}\newcommand{\Ber}{\mathrm{Ber}}\newcommand{\Mult}{\mathrm{Mult}}\newcommand{\BPois}{\mathrm{BPois}}\newcommand{\fraks}{\mathfrak{s}}\newcommand{\frakk}{\mathfrak{k}}\newcommand{\IF}{\mathrm{IF}}\newcommand{\bX}{\mathbf{X}}\newcommand{\bx}{\boldsymbol{x}}\newcommand{\indep}{\raisebox{0.05em}{\rotatebox[origin=c]{90}{$\models$}}}\newcommand{\IG}{\mathrm{IG}}\newcommand{\Levy}{\mathrm{Levy}}\newcommand{\MP}{\mathrm{MP}}\newcommand{\Hermite}{\mathrm{Hermite}}\newcommand{\Skellam}{\mathrm{Skellam}}\newcommand{\Dirichlet}{\mathrm{Dirichlet}}\newcommand{\Beta}{\mathrm{Beta}}\newcommand{\bE}{\mathbb{E}}\newcommand{\bG}{\mathbb{G}}\newcommand{\MISE}{\mathrm{MISE}}\newcommand{\logit}{\mathtt{logit}}\newcommand{\expit}{\mathtt{expit}}\newcommand{\cK}{\mathcal{K}}\newcommand{\dl}{\dot{l}}\newcommand{\dotp}{\dot{p}}\newcommand{\wl}{\wt{l}}\newcommand{\Gauss}{\mathrm{Gauss}}\newcommand{\fA}{\mathfrak{A}}\newcommand{\under}{\mathrm{under}\;}\newcommand{\whtheta}{\wh{\theta}}\newcommand{\Em}{\mathrm{Em}}\newcommand{\ztheta}{{\theta_0}}
\newcommand{\rO}{\mathrm{O}}\newcommand{\Bin}{\mathrm{Bin}}\newcommand{\rW}{\mathrm{W}}\newcommand{\rG}{\mathrm{G}}\newcommand{\rB}{\mathrm{B}}\newcommand{\rN}{\mathrm{N}}\newcommand{\rU}{\mathrm{U}}\newcommand{\HG}{\mathrm{HG}}\newcommand{\GAMMA}{\mathrm{Gamma}}\newcommand{\Cauchy}{\mathrm{Cauchy}}\newcommand{\rt}{\mathrm{t}}
\DeclareMathOperator{\erf}{erf}

%%% 圏
\newcommand{\varlim}{\varprojlim}\newcommand{\Hom}{\mathrm{Hom}}\newcommand{\Iso}{\mathrm{Iso}}\newcommand{\Mor}{\mathrm{Mor}}\newcommand{\Isom}{\mathrm{Isom}}\newcommand{\Aut}{\mathrm{Aut}}\newcommand{\End}{\mathrm{End}}\newcommand{\op}{\mathrm{op}}\newcommand{\ev}{\mathrm{ev}}\newcommand{\Ob}{\mathrm{Ob}}\newcommand{\Ar}{\mathrm{Ar}}\newcommand{\Arr}{\mathrm{Arr}}\newcommand{\Set}{\mathrm{Set}}\newcommand{\Grp}{\mathrm{Grp}}\newcommand{\Cat}{\mathrm{Cat}}\newcommand{\Mon}{\mathrm{Mon}}\newcommand{\Ring}{\mathrm{Ring}}\newcommand{\CRing}{\mathrm{CRing}}\newcommand{\Ab}{\mathrm{Ab}}\newcommand{\Pos}{\mathrm{Pos}}\newcommand{\Vect}{\mathrm{Vect}}\newcommand{\FinVect}{\mathrm{FinVect}}\newcommand{\FinSet}{\mathrm{FinSet}}\newcommand{\FinMeas}{\mathrm{FinMeas}}\newcommand{\OmegaAlg}{\Omega\text{-}\mathrm{Alg}}\newcommand{\OmegaEAlg}{(\Omega,E)\text{-}\mathrm{Alg}}\newcommand{\Fun}{\mathrm{Fun}}\newcommand{\Func}{\mathrm{Func}}\newcommand{\Alg}{\mathrm{Alg}} %代数の圏
\newcommand{\CAlg}{\mathrm{CAlg}} %可換代数の圏
\newcommand{\Met}{\mathrm{Met}} %Metric space & Contraction maps
\newcommand{\Rel}{\mathrm{Rel}} %Sets & relation
\newcommand{\Bool}{\mathrm{Bool}}\newcommand{\CABool}{\mathrm{CABool}}\newcommand{\CompBoolAlg}{\mathrm{CompBoolAlg}}\newcommand{\BoolAlg}{\mathrm{BoolAlg}}\newcommand{\BoolRng}{\mathrm{BoolRng}}\newcommand{\HeytAlg}{\mathrm{HeytAlg}}\newcommand{\CompHeytAlg}{\mathrm{CompHeytAlg}}\newcommand{\Lat}{\mathrm{Lat}}\newcommand{\CompLat}{\mathrm{CompLat}}\newcommand{\SemiLat}{\mathrm{SemiLat}}\newcommand{\Stone}{\mathrm{Stone}}\newcommand{\Mfd}{\mathrm{Mfd}}\newcommand{\LieAlg}{\mathrm{LieAlg}}
\newcommand{\Sob}{\mathrm{Sob}} %Sober space & continuous map
\newcommand{\Op}{\mathrm{Op}} %Category of open subsets
\newcommand{\Sh}{\mathrm{Sh}} %Category of sheave
\newcommand{\PSh}{\mathrm{PSh}} %Category of presheave, PSh(C)=[C^op,set]のこと
\newcommand{\Conv}{\mathrm{Conv}} %Convergence spaceの圏
\newcommand{\Unif}{\mathrm{Unif}} %一様空間と一様連続写像の圏
\newcommand{\Frm}{\mathrm{Frm}} %フレームとフレームの射
\newcommand{\Locale}{\mathrm{Locale}} %その反対圏
\newcommand{\Diff}{\mathrm{Diff}} %滑らかな多様体の圏
\newcommand{\Quiv}{\mathrm{Quiv}} %Quiverの圏
\newcommand{\B}{\mathcal{B}}\newcommand{\Span}{\mathrm{Span}}\newcommand{\Corr}{\mathrm{Corr}}\newcommand{\Decat}{\mathrm{Decat}}\newcommand{\Rep}{\mathrm{Rep}}\newcommand{\Grpd}{\mathrm{Grpd}}\newcommand{\sSet}{\mathrm{sSet}}\newcommand{\Mod}{\mathrm{Mod}}\newcommand{\SmoothMnf}{\mathrm{SmoothMnf}}\newcommand{\coker}{\mathrm{coker}}\newcommand{\Ord}{\mathrm{Ord}}\newcommand{\eq}{\mathrm{eq}}\newcommand{\coeq}{\mathrm{coeq}}\newcommand{\act}{\mathrm{act}}

%%%%%%%%%%%%%%% 定理環境(足助先生ありがとうございます) %%%%%%%%%%%%%%%

\everymath{\displaystyle}
\renewcommand{\proofname}{\bf\underline{[証明]}}
\renewcommand{\thefootnote}{\dag\arabic{footnote}} %足助さんからもらった.どうなるんだ?
\renewcommand{\qedsymbol}{$\blacksquare$}

\renewcommand{\labelenumi}{(\arabic{enumi})} %(1),(2),...がデフォルトであって欲しい
\renewcommand{\labelenumii}{(\alph{enumii})}
\renewcommand{\labelenumiii}{(\roman{enumiii})}

\newtheoremstyle{StatementsWithUnderline}% ?name?
{3pt}% ?Space above? 1
{3pt}% ?Space below? 1
{}% ?Body font?
{}% ?Indent amount? 2
{\bfseries}% ?Theorem head font?
{\textbf{.}}% ?Punctuation after theorem head?
{.5em}% ?Space after theorem head? 3
{\textbf{\underline{\textup{#1~\thetheorem{}}}}\;\thmnote{(#3)}}% ?Theorem head spec (can be left empty, meaning ‘normal’)?

\usepackage{etoolbox}
\AtEndEnvironment{example}{\hfill\ensuremath{\Box}}
\AtEndEnvironment{observation}{\hfill\ensuremath{\Box}}

\theoremstyle{StatementsWithUnderline}
    \newtheorem{theorem}{定理}[section]
    \newtheorem{axiom}[theorem]{公理}
    \newtheorem{corollary}[theorem]{系}
    \newtheorem{proposition}[theorem]{命題}
    \newtheorem{lemma}[theorem]{補題}
    \newtheorem{definition}[theorem]{定義}
    \newtheorem{problem}[theorem]{問題}
    \newtheorem{exercise}[theorem]{Exercise}
\theoremstyle{definition}
    \newtheorem{issue}{論点}
    \newtheorem*{proposition*}{命題}
    \newtheorem*{lemma*}{補題}
    \newtheorem*{consideration*}{考察}
    \newtheorem*{theorem*}{定理}
    \newtheorem*{remarks*}{要諦}
    \newtheorem{example}[theorem]{例}
    \newtheorem{notation}[theorem]{記法}
    \newtheorem*{notation*}{記法}
    \newtheorem{assumption}[theorem]{仮定}
    \newtheorem{question}[theorem]{問}
    \newtheorem{counterexample}[theorem]{反例}
    \newtheorem{reidai}[theorem]{例題}
    \newtheorem{ruidai}[theorem]{類題}
    \newtheorem{algorithm}[theorem]{算譜}
    \newtheorem*{feels*}{所感}
    \newtheorem*{solution*}{\bf{[解]}}
    \newtheorem{discussion}[theorem]{議論}
    \newtheorem{synopsis}[theorem]{要約}
    \newtheorem{cited}[theorem]{引用}
    \newtheorem{remark}[theorem]{注}
    \newtheorem{remarks}[theorem]{要諦}
    \newtheorem{memo}[theorem]{メモ}
    \newtheorem{image}[theorem]{描像}
    \newtheorem{observation}[theorem]{観察}
    \newtheorem{universality}[theorem]{普遍性} %非自明な例外がない.
    \newtheorem{universal tendency}[theorem]{普遍傾向} %例外が有意に少ない.
    \newtheorem{hypothesis}[theorem]{仮説} %実験で説明されていない理論.
    \newtheorem{theory}[theorem]{理論} %実験事実とその(さしあたり)整合的な説明.
    \newtheorem{fact}[theorem]{実験事実}
    \newtheorem{model}[theorem]{模型}
    \newtheorem{explanation}[theorem]{説明} %理論による実験事実の説明
    \newtheorem{anomaly}[theorem]{理論の限界}
    \newtheorem{application}[theorem]{応用例}
    \newtheorem{method}[theorem]{手法} %実験手法など,技術的問題.
    \newtheorem{test}[theorem]{検定}
    \newtheorem{terms}[theorem]{用語}
    \newtheorem{solution}[theorem]{解法}
    \newtheorem{history}[theorem]{歴史}
    \newtheorem{usage}[theorem]{用語法}
    \newtheorem{research}[theorem]{研究}
    \newtheorem{shishin}[theorem]{指針}
    \newtheorem{yodan}[theorem]{余談}
    \newtheorem{construction}[theorem]{構成}
    \newtheorem{motivation}[theorem]{動機}
    \newtheorem{context}[theorem]{背景}
    \newtheorem{advantage}[theorem]{利点}
    \newtheorem*{definition*}{定義}
    \newtheorem*{remark*}{注意}
    \newtheorem*{question*}{問}
    \newtheorem*{problem*}{問題}
    \newtheorem*{axiom*}{公理}
    \newtheorem*{example*}{例}
    \newtheorem*{corollary*}{系}
    \newtheorem*{shishin*}{指針}
    \newtheorem*{yodan*}{余談}
    \newtheorem*{kadai*}{課題}

\raggedbottom
\allowdisplaybreaks
%%%%%%%%%%%%%%%% 数理文書の組版 %%%%%%%%%%%%%%%

\usepackage{mathtools} %内部でamsmathを呼び出すことに注意.
%\mathtoolsset{showonlyrefs=true} %labelを附した数式にのみ附番される設定.
\usepackage{amsfonts} %mathfrak, mathcal, mathbbなど.
\usepackage{amsthm} %定理環境.
\usepackage{amssymb} %AMSFontsを使うためのパッケージ.
\usepackage{ascmac} %screen, itembox, shadebox環境.全てLATEX2εの標準機能の範囲で作られたもの.
\usepackage{comment} %comment環境を用いて,複数行をcomment outできるようにするpackage
\usepackage{wrapfig} %図の周りに文字をwrapさせることができる.詳細な制御ができる.
\usepackage[usenames, dvipsnames]{xcolor} %xcolorはcolorの拡張.optionの意味はdvipsnamesはLoad a set of predefined colors. forestgreenなどの色が追加されている.usenamesはobsoleteとだけ書いてあった.
\setcounter{tocdepth}{2} %目次に表示される深さ.2はsubsectionまで
\usepackage{multicol} %\begin{multicols}{2}環境で途中からmulticolumnに出来る.
\usepackage{mathabx}\newcommand{\wc}{\widecheck} %\widecheckなどのフォントパッケージ

%%%%%%%%%%%%%%% フォント %%%%%%%%%%%%%%%

\usepackage{textcomp, mathcomp} %Text Companionとは,T1 encodingに入らなかった文字群.これを使うためのパッケージ.\textsectionでブルバキに!
\usepackage[T1]{fontenc} %8bitエンコーディングにする.comp系拡張数学文字の動作が安定する.

%%%%%%%%%%%%%%% 一般文書の組版 %%%%%%%%%%%%%%%

\definecolor{花緑青}{cmyk}{1,0.07,0.10,0.10}\definecolor{サーモンピンク}{cmyk}{0,0.65,0.65,0.05}\definecolor{暗中模索}{rgb}{0.2,0.2,0.2}
\usepackage{url}\usepackage[dvipdfmx,colorlinks,linkcolor=花緑青,urlcolor=花緑青,citecolor=花緑青]{hyperref} %生成されるPDFファイルにおいて、\tableofcontentsによって書き出された目次をクリックすると該当する見出しへジャンプしたり、さらには、\label{ラベル名}を番号で参照する\ref{ラベル名}やthebibliography環境において\bibitem{ラベル名}を文献番号で参照する\cite{ラベル名}においても番号をクリックすると該当箇所にジャンプする.囲み枠はダサいので,colorlinksで囲み廃止し,リンク自体に色を付けることにした.
\usepackage{pxjahyper} %pxrubrica同様,八登崇之さん.hyperrefは日本語pLaTeXに最適化されていないから,hyperrefとセットで,(u)pLaTeX+hyperref+dvipdfmxの組み合わせで日本語を含む「しおり」をもつPDF文書を作成する場合に必要となる機能を提供する
\usepackage{ulem} %取り消し線を引くためのパッケージ
\usepackage{pxrubrica} %日本語にルビをふる.八登崇之(やとうたかゆき)氏による.

%%%%%%%%%%%%%%% 科学文書の組版 %%%%%%%%%%%%%%%

\usepackage[version=4]{mhchem} %化学式をTikZで簡単に書くためのパッケージ.
\usepackage{chemfig} %化学構造式をTikZで描くためのパッケージ.
\usepackage{siunitx} %IS単位を書くためのパッケージ

%%%%%%%%%%%%%%% 作図 %%%%%%%%%%%%%%%

\usepackage{tikz}\usetikzlibrary{positioning,automata}\usepackage{tikz-cd}\usepackage[all]{xy}
\def\objectstyle{\displaystyle} %デフォルトではxymatrix中の数式が文中数式モードになるので,それを直す.\labelstyleも同様にxy packageの中で定義されており,文中数式モードになっている.

\usepackage{graphicx} %rotatebox, scalebox, reflectbox, resizeboxなどのコマンドや,図表の読み込み\includegraphicsを司る.graphics というパッケージもありますが,graphicx はこれを高機能にしたものと考えて結構です(ただし graphicx は内部で graphics を読み込みます)
\usepackage[top=15truemm,bottom=15truemm,left=10truemm,right=10truemm]{geometry} %足助さんからもらったオプション

%%%%%%%%%%%%%%% 参照 %%%%%%%%%%%%%%%
%参考文献リストを出力したい箇所に\bibliography{../mathematics.bib}を追記すると良い.

%\bibliographystyle{jplain}
%\bibliographystyle{jname}
\bibliographystyle{apalike}

%%%%%%%%%%%%%%% 計算機文書の組版 %%%%%%%%%%%%%%%

\usepackage[breakable]{tcolorbox} %加藤晃史さんがフル活用していたtcolorboxを,途中改ページ可能で.
\tcbuselibrary{theorems} %https://qiita.com/t_kemmochi/items/483b8fcdb5db8d1f5d5e
\usepackage{enumerate} %enumerate環境を凝らせる.

\usepackage{listings} %ソースコードを表示できる環境.多分もっといい方法ある.
\usepackage{jvlisting} %日本語のコメントアウトをする場合jlistingが必要
\lstset{ %ここからソースコードの表示に関する設定.lstlisting環境では,[caption=hoge,label=fuga]などのoptionを付けられる.
%[escapechar=!]とすると,LaTeXコマンドを使える.
  basicstyle={\ttfamily},
  identifierstyle={\small},
  commentstyle={\smallitshape},
  keywordstyle={\small\bfseries},
  ndkeywordstyle={\small},
  stringstyle={\small\ttfamily},
  frame={tb},
  breaklines=true,
  columns=[l]{fullflexible},
  numbers=left,
  xrightmargin=0zw,
  xleftmargin=3zw,
  numberstyle={\scriptsize},
  stepnumber=1,
  numbersep=1zw,
  lineskip=-0.5ex
}
%\makeatletter %caption番号を「[chapter番号].[section番号].[subsection番号]-[そのsubsection内においてn番目]」に変更
%    \AtBeginDocument{
%    \renewcommand*{\thelstlisting}{\arabic{chapter}.\arabic{section}.\arabic{lstlisting}}
%    \@addtoreset{lstlisting}{section}
%    }
%\makeatother
\renewcommand{\lstlistingname}{算譜} %caption名を"program"に変更

\newtcolorbox{tbox}[3][]{%
colframe=#2,colback=#2!10,coltitle=#2!20!black,title={#3},#1}

% 証明内の文字が小さくなる環境.
\newenvironment{Proof}[1][\bf\underline{[証明]}]{\proof[#1]\color{darkgray}}{\endproof}

%%%%%%%%%%%%%%% 数学記号のマクロ %%%%%%%%%%%%%%%

%%% 括弧類
\newcommand{\abs}[1]{\lvert#1\rvert}\newcommand{\Abs}[1]{\left|#1\right|}\newcommand{\norm}[1]{\|#1\|}\newcommand{\Norm}[1]{\left\|#1\right\|}\newcommand{\Brace}[1]{\left\{#1\right\}}\newcommand{\BRace}[1]{\biggl\{#1\biggr\}}\newcommand{\paren}[1]{\left(#1\right)}\newcommand{\Paren}[1]{\biggr(#1\biggl)}\newcommand{\bracket}[1]{\langle#1\rangle}\newcommand{\brac}[1]{\langle#1\rangle}\newcommand{\Bracket}[1]{\left\langle#1\right\rangle}\newcommand{\Brac}[1]{\left\langle#1\right\rangle}\newcommand{\bra}[1]{\left\langle#1\right|}\newcommand{\ket}[1]{\left|#1\right\rangle}\newcommand{\Square}[1]{\left[#1\right]}\newcommand{\SQuare}[1]{\biggl[#1\biggr]}
\renewcommand{\o}[1]{\overline{#1}}\renewcommand{\u}[1]{\underline{#1}}\newcommand{\wt}[1]{\widetilde{#1}}\newcommand{\wh}[1]{\widehat{#1}}
\newcommand{\pp}[2]{\frac{\partial #1}{\partial #2}}\newcommand{\ppp}[3]{\frac{\partial #1}{\partial #2\partial #3}}\newcommand{\dd}[2]{\frac{d #1}{d #2}}
\newcommand{\floor}[1]{\lfloor#1\rfloor}\newcommand{\Floor}[1]{\left\lfloor#1\right\rfloor}\newcommand{\ceil}[1]{\lceil#1\rceil}
\newcommand{\ocinterval}[1]{(#1]}\newcommand{\cointerval}[1]{[#1)}\newcommand{\COinterval}[1]{\left[#1\right)}


%%% 予約語
\renewcommand{\iff}{\;\mathrm{iff}\;}
\newcommand{\False}{\mathrm{False}}\newcommand{\True}{\mathrm{True}}
\newcommand{\otherwise}{\mathrm{otherwise}}
\newcommand{\st}{\;\mathrm{s.t.}\;}

%%% 略記
\newcommand{\M}{\mathcal{M}}\newcommand{\cF}{\mathcal{F}}\newcommand{\cD}{\mathcal{D}}\newcommand{\fX}{\mathfrak{X}}\newcommand{\fY}{\mathfrak{Y}}\newcommand{\fZ}{\mathfrak{Z}}\renewcommand{\H}{\mathcal{H}}\newcommand{\fH}{\mathfrak{H}}\newcommand{\bH}{\mathbb{H}}\newcommand{\id}{\mathrm{id}}\newcommand{\A}{\mathcal{A}}\newcommand{\U}{\mathfrak{U}}
\newcommand{\lmd}{\lambda}
\newcommand{\Lmd}{\Lambda}

%%% 矢印類
\newcommand{\iso}{\xrightarrow{\,\smash{\raisebox{-0.45ex}{\ensuremath{\scriptstyle\sim}}}\,}}
\newcommand{\Lrarrow}{\;\;\Leftrightarrow\;\;}

%%% 注記
\newcommand{\rednote}[1]{\textcolor{red}{#1}}

% ノルム位相についての閉包 https://newbedev.com/how-to-make-double-overline-with-less-vertical-displacement
\makeatletter
\newcommand{\dbloverline}[1]{\overline{\dbl@overline{#1}}}
\newcommand{\dbl@overline}[1]{\mathpalette\dbl@@overline{#1}}
\newcommand{\dbl@@overline}[2]{%
  \begingroup
  \sbox\z@{$\m@th#1\overline{#2}$}%
  \ht\z@=\dimexpr\ht\z@-2\dbl@adjust{#1}\relax
  \box\z@
  \ifx#1\scriptstyle\kern-\scriptspace\else
  \ifx#1\scriptscriptstyle\kern-\scriptspace\fi\fi
  \endgroup
}
\newcommand{\dbl@adjust}[1]{%
  \fontdimen8
  \ifx#1\displaystyle\textfont\else
  \ifx#1\textstyle\textfont\else
  \ifx#1\scriptstyle\scriptfont\else
  \scriptscriptfont\fi\fi\fi 3
}
\makeatother
\newcommand{\oo}[1]{\dbloverline{#1}}

% hslashの他の文字Ver.
\newcommand{\hslashslash}{%
    \scalebox{1.2}{--
    }%
}
\newcommand{\dslash}{%
  {%
    \vphantom{d}%
    \ooalign{\kern.05em\smash{\hslashslash}\hidewidth\cr$d$\cr}%
    \kern.05em
  }%
}
\newcommand{\dint}{%
  {%
    \vphantom{d}%
    \ooalign{\kern.05em\smash{\hslashslash}\hidewidth\cr$\int$\cr}%
    \kern.05em
  }%
}
\newcommand{\dL}{%
  {%
    \vphantom{d}%
    \ooalign{\kern.05em\smash{\hslashslash}\hidewidth\cr$L$\cr}%
    \kern.05em
  }%
}

%%% 演算子
\DeclareMathOperator{\grad}{\mathrm{grad}}\DeclareMathOperator{\rot}{\mathrm{rot}}\DeclareMathOperator{\divergence}{\mathrm{div}}\DeclareMathOperator{\tr}{\mathrm{tr}}\newcommand{\pr}{\mathrm{pr}}
\newcommand{\Map}{\mathrm{Map}}\newcommand{\dom}{\mathrm{Dom}\;}\newcommand{\cod}{\mathrm{Cod}\;}\newcommand{\supp}{\mathrm{supp}\;}


%%% 線型代数学
\newcommand{\vctr}[2]{\begin{pmatrix}#1\\#2\end{pmatrix}}\newcommand{\vctrr}[3]{\begin{pmatrix}#1\\#2\\#3\end{pmatrix}}\newcommand{\mtrx}[4]{\begin{pmatrix}#1&#2\\#3&#4\end{pmatrix}}\newcommand{\smtrx}[4]{\paren{\begin{smallmatrix}#1&#2\\#3&#4\end{smallmatrix}}}\newcommand{\Ker}{\mathrm{Ker}\;}\newcommand{\Coker}{\mathrm{Coker}\;}\newcommand{\Coim}{\mathrm{Coim}\;}\DeclareMathOperator{\rank}{\mathrm{rank}}\newcommand{\lcm}{\mathrm{lcm}}\newcommand{\sgn}{\mathrm{sgn}\,}\newcommand{\GL}{\mathrm{GL}}\newcommand{\SL}{\mathrm{SL}}\newcommand{\alt}{\mathrm{alt}}
%%% 複素解析学
\renewcommand{\Re}{\mathrm{Re}\;}\renewcommand{\Im}{\mathrm{Im}\;}\newcommand{\Gal}{\mathrm{Gal}}\newcommand{\PGL}{\mathrm{PGL}}\newcommand{\PSL}{\mathrm{PSL}}\newcommand{\Log}{\mathrm{Log}\,}\newcommand{\Res}{\mathrm{Res}\,}\newcommand{\on}{\mathrm{on}\;}\newcommand{\hatC}{\widehat{\C}}\newcommand{\hatR}{\hat{\R}}\newcommand{\PV}{\mathrm{P.V.}}\newcommand{\diam}{\mathrm{diam}}\newcommand{\Area}{\mathrm{Area}}\newcommand{\Lap}{\Laplace}\newcommand{\f}{\mathbf{f}}\newcommand{\cR}{\mathcal{R}}\newcommand{\const}{\mathrm{const.}}\newcommand{\Om}{\Omega}\newcommand{\Cinf}{C^\infty}\newcommand{\ep}{\epsilon}\newcommand{\dist}{\mathrm{dist}}\newcommand{\opart}{\o{\partial}}\newcommand{\Length}{\mathrm{Length}}
%%% 集合と位相
\renewcommand{\O}{\mathcal{O}}\renewcommand{\S}{\mathcal{S}}\renewcommand{\U}{\mathcal{U}}\newcommand{\V}{\mathcal{V}}\renewcommand{\P}{\mathcal{P}}\newcommand{\R}{\mathbb{R}}\newcommand{\N}{\mathbb{N}}\newcommand{\C}{\mathbb{C}}\newcommand{\Z}{\mathbb{Z}}\newcommand{\Q}{\mathbb{Q}}\newcommand{\TV}{\mathrm{TV}}\newcommand{\ORD}{\mathrm{ORD}}\newcommand{\Tr}{\mathrm{Tr}}\newcommand{\Card}{\mathrm{Card}\;}\newcommand{\Top}{\mathrm{Top}}\newcommand{\Disc}{\mathrm{Disc}}\newcommand{\Codisc}{\mathrm{Codisc}}\newcommand{\CoDisc}{\mathrm{CoDisc}}\newcommand{\Ult}{\mathrm{Ult}}\newcommand{\ord}{\mathrm{ord}}\newcommand{\maj}{\mathrm{maj}}\newcommand{\bS}{\mathbb{S}}\newcommand{\PConn}{\mathrm{PConn}}

%%% 形式言語理論
\newcommand{\REGEX}{\mathrm{REGEX}}\newcommand{\RE}{\mathbf{RE}}
%%% Graph Theory
\newcommand{\SimpGph}{\mathrm{SimpGph}}\newcommand{\Gph}{\mathrm{Gph}}\newcommand{\mult}{\mathrm{mult}}\newcommand{\inv}{\mathrm{inv}}

%%% 多様体
\newcommand{\Der}{\mathrm{Der}}\newcommand{\osub}{\overset{\mathrm{open}}{\subset}}\newcommand{\osup}{\overset{\mathrm{open}}{\supset}}\newcommand{\al}{\alpha}\newcommand{\K}{\mathbb{K}}\newcommand{\Sp}{\mathrm{Sp}}\newcommand{\g}{\mathfrak{g}}\newcommand{\h}{\mathfrak{h}}\newcommand{\Exp}{\mathrm{Exp}\;}\newcommand{\Imm}{\mathrm{Imm}}\newcommand{\Imb}{\mathrm{Imb}}\newcommand{\codim}{\mathrm{codim}\;}\newcommand{\Gr}{\mathrm{Gr}}
%%% 代数
\newcommand{\Ad}{\mathrm{Ad}}\newcommand{\finsupp}{\mathrm{fin\;supp}}\newcommand{\SO}{\mathrm{SO}}\newcommand{\SU}{\mathrm{SU}}\newcommand{\acts}{\curvearrowright}\newcommand{\mono}{\hookrightarrow}\newcommand{\epi}{\twoheadrightarrow}\newcommand{\Stab}{\mathrm{Stab}}\newcommand{\nor}{\mathrm{nor}}\newcommand{\T}{\mathbb{T}}\newcommand{\Aff}{\mathrm{Aff}}\newcommand{\rsub}{\triangleleft}\newcommand{\rsup}{\triangleright}\newcommand{\subgrp}{\overset{\mathrm{subgrp}}{\subset}}\newcommand{\Ext}{\mathrm{Ext}}\newcommand{\sbs}{\subset}\newcommand{\sps}{\supset}\newcommand{\In}{\mathrm{in}\;}\newcommand{\Tor}{\mathrm{Tor}}\newcommand{\p}{\b{p}}\newcommand{\q}{\mathfrak{q}}\newcommand{\m}{\mathfrak{m}}\newcommand{\cS}{\mathcal{S}}\newcommand{\Frac}{\mathrm{Frac}\,}\newcommand{\Spec}{\mathrm{Spec}\,}\newcommand{\bA}{\mathbb{A}}\newcommand{\Sym}{\mathrm{Sym}}\newcommand{\Ann}{\mathrm{Ann}}\newcommand{\Her}{\mathrm{Her}}\newcommand{\Bil}{\mathrm{Bil}}\newcommand{\Ses}{\mathrm{Ses}}\newcommand{\FVS}{\mathrm{FVS}}
%%% 代数的位相幾何学
\newcommand{\Ho}{\mathrm{Ho}}\newcommand{\CW}{\mathrm{CW}}\newcommand{\lc}{\mathrm{lc}}\newcommand{\cg}{\mathrm{cg}}\newcommand{\Fib}{\mathrm{Fib}}\newcommand{\Cyl}{\mathrm{Cyl}}\newcommand{\Ch}{\mathrm{Ch}}
%%% 微分幾何学
\newcommand{\rE}{\mathrm{E}}\newcommand{\e}{\b{e}}\renewcommand{\k}{\b{k}}\newcommand{\Christ}[2]{\begin{Bmatrix}#1\\#2\end{Bmatrix}}\renewcommand{\Vec}[1]{\overrightarrow{\mathrm{#1}}}\newcommand{\hen}[1]{\mathrm{#1}}\renewcommand{\b}[1]{\boldsymbol{#1}}

%%% 函数解析
\newcommand{\HS}{\mathrm{HS}}\newcommand{\loc}{\mathrm{loc}}\newcommand{\Lh}{\mathrm{L.h.}}\newcommand{\Epi}{\mathrm{Epi}\;}\newcommand{\slim}{\mathrm{slim}}\newcommand{\Ban}{\mathrm{Ban}}\newcommand{\Hilb}{\mathrm{Hilb}}\newcommand{\Ex}{\mathrm{Ex}}\newcommand{\Co}{\mathrm{Co}}\newcommand{\sa}{\mathrm{sa}}\newcommand{\nnorm}[1]{{\left\vert\kern-0.25ex\left\vert\kern-0.25ex\left\vert #1 \right\vert\kern-0.25ex\right\vert\kern-0.25ex\right\vert}}\newcommand{\dvol}{\mathrm{dvol}}\newcommand{\Sconv}{\mathrm{Sconv}}\newcommand{\I}{\mathcal{I}}\newcommand{\nonunital}{\mathrm{nu}}\newcommand{\cpt}{\mathrm{cpt}}\newcommand{\lcpt}{\mathrm{lcpt}}\newcommand{\com}{\mathrm{com}}\newcommand{\Haus}{\mathrm{Haus}}\newcommand{\proper}{\mathrm{proper}}\newcommand{\infinity}{\mathrm{inf}}\newcommand{\TVS}{\mathrm{TVS}}\newcommand{\ess}{\mathrm{ess}}\newcommand{\ext}{\mathrm{ext}}\newcommand{\Index}{\mathrm{Index}\;}\newcommand{\SSR}{\mathrm{SSR}}\newcommand{\vs}{\mathrm{vs.}}\newcommand{\fM}{\mathfrak{M}}\newcommand{\EDM}{\mathrm{EDM}}\newcommand{\Tw}{\mathrm{Tw}}\newcommand{\fC}{\mathfrak{C}}\newcommand{\bn}{\boldsymbol{n}}\newcommand{\br}{\boldsymbol{r}}\newcommand{\Lam}{\Lambda}\newcommand{\lam}{\lambda}\newcommand{\one}{\mathbf{1}}\newcommand{\dae}{\text{-a.e.}}\newcommand{\das}{\text{-a.s.}}\newcommand{\td}{\text{-}}\newcommand{\RM}{\mathrm{RM}}\newcommand{\BV}{\mathrm{BV}}\newcommand{\normal}{\mathrm{normal}}\newcommand{\lub}{\mathrm{lub}\;}\newcommand{\Graph}{\mathrm{Graph}}\newcommand{\Ascent}{\mathrm{Ascent}}\newcommand{\Descent}{\mathrm{Descent}}\newcommand{\BIL}{\mathrm{BIL}}\newcommand{\fL}{\mathfrak{L}}\newcommand{\De}{\Delta}
%%% 積分論
\newcommand{\calA}{\mathcal{A}}\newcommand{\calB}{\mathcal{B}}\newcommand{\D}{\mathcal{D}}\newcommand{\Y}{\mathcal{Y}}\newcommand{\calC}{\mathcal{C}}\renewcommand{\ae}{\mathrm{a.e.}\;}\newcommand{\cZ}{\mathcal{Z}}\newcommand{\fF}{\mathfrak{F}}\newcommand{\fI}{\mathfrak{I}}\newcommand{\E}{\mathcal{E}}\newcommand{\sMap}{\sigma\textrm{-}\mathrm{Map}}\DeclareMathOperator*{\argmax}{arg\,max}\DeclareMathOperator*{\argmin}{arg\,min}\newcommand{\cC}{\mathcal{C}}\newcommand{\comp}{\complement}\newcommand{\J}{\mathcal{J}}\newcommand{\sumN}[1]{\sum_{#1\in\N}}\newcommand{\cupN}[1]{\cup_{#1\in\N}}\newcommand{\capN}[1]{\cap_{#1\in\N}}\newcommand{\Sum}[1]{\sum_{#1=1}^\infty}\newcommand{\sumn}{\sum_{n=1}^\infty}\newcommand{\summ}{\sum_{m=1}^\infty}\newcommand{\sumk}{\sum_{k=1}^\infty}\newcommand{\sumi}{\sum_{i=1}^\infty}\newcommand{\sumj}{\sum_{j=1}^\infty}\newcommand{\cupn}{\cup_{n=1}^\infty}\newcommand{\capn}{\cap_{n=1}^\infty}\newcommand{\cupk}{\cup_{k=1}^\infty}\newcommand{\cupi}{\cup_{i=1}^\infty}\newcommand{\cupj}{\cup_{j=1}^\infty}\newcommand{\limn}{\lim_{n\to\infty}}\renewcommand{\l}{\mathcal{l}}\renewcommand{\L}{\mathcal{L}}\newcommand{\Cl}{\mathrm{Cl}}\newcommand{\cN}{\mathcal{N}}\newcommand{\Ae}{\textrm{-a.e.}\;}\newcommand{\csub}{\overset{\textrm{closed}}{\subset}}\newcommand{\csup}{\overset{\textrm{closed}}{\supset}}\newcommand{\wB}{\wt{B}}\newcommand{\cG}{\mathcal{G}}\newcommand{\Lip}{\mathrm{Lip}}\DeclareMathOperator{\Dom}{\mathrm{Dom}}\newcommand{\AC}{\mathrm{AC}}\newcommand{\Mol}{\mathrm{Mol}}
%%% Fourier解析
\newcommand{\Pe}{\mathrm{Pe}}\newcommand{\wR}{\wh{\mathbb{\R}}}\newcommand*{\Laplace}{\mathop{}\!\mathbin\bigtriangleup}\newcommand*{\DAlambert}{\mathop{}\!\mathbin\Box}\newcommand{\bT}{\mathbb{T}}\newcommand{\dx}{\dslash x}\newcommand{\dt}{\dslash t}\newcommand{\ds}{\dslash s}
%%% 数値解析
\newcommand{\round}{\mathrm{round}}\newcommand{\cond}{\mathrm{cond}}\newcommand{\diag}{\mathrm{diag}}
\newcommand{\Adj}{\mathrm{Adj}}\newcommand{\Pf}{\mathrm{Pf}}\newcommand{\Sg}{\mathrm{Sg}}

%%% 確率論
\newcommand{\Prob}{\mathrm{Prob}}\newcommand{\X}{\mathcal{X}}\newcommand{\Meas}{\mathrm{Meas}}\newcommand{\as}{\;\mathrm{a.s.}}\newcommand{\io}{\;\mathrm{i.o.}}\newcommand{\fe}{\;\mathrm{f.e.}}\newcommand{\F}{\mathcal{F}}\newcommand{\bF}{\mathbb{F}}\newcommand{\W}{\mathcal{W}}\newcommand{\Pois}{\mathrm{Pois}}\newcommand{\iid}{\mathrm{i.i.d.}}\newcommand{\wconv}{\rightsquigarrow}\newcommand{\Var}{\mathrm{Var}}\newcommand{\xrightarrown}{\xrightarrow{n\to\infty}}\newcommand{\au}{\mathrm{au}}\newcommand{\cT}{\mathcal{T}}\newcommand{\wto}{\overset{w}{\to}}\newcommand{\dto}{\overset{d}{\to}}\newcommand{\pto}{\overset{p}{\to}}\newcommand{\vto}{\overset{v}{\to}}\newcommand{\Cont}{\mathrm{Cont}}\newcommand{\stably}{\mathrm{stably}}\newcommand{\Np}{\mathbb{N}^+}\newcommand{\oM}{\overline{\mathcal{M}}}\newcommand{\fP}{\mathfrak{P}}\newcommand{\sign}{\mathrm{sign}}\DeclareMathOperator{\Div}{Div}
\newcommand{\bD}{\mathbb{D}}\newcommand{\fW}{\mathfrak{W}}\newcommand{\DL}{\mathcal{D}\mathcal{L}}\renewcommand{\r}[1]{\mathrm{#1}}\newcommand{\rC}{\mathrm{C}}
%%% 情報理論
\newcommand{\bit}{\mathrm{bit}}\DeclareMathOperator{\sinc}{sinc}
%%% 量子論
\newcommand{\err}{\mathrm{err}}
%%% 最適化
\newcommand{\varparallel}{\mathbin{\!/\mkern-5mu/\!}}\newcommand{\Minimize}{\text{Minimize}}\newcommand{\subjectto}{\text{subject to}}\newcommand{\Ri}{\mathrm{Ri}}\newcommand{\Cone}{\mathrm{Cone}}\newcommand{\Int}{\mathrm{Int}}
%%% 数理ファイナンス
\newcommand{\pre}{\mathrm{pre}}\newcommand{\om}{\omega}

%%% 偏微分方程式
\let\div\relax
\DeclareMathOperator{\div}{div}\newcommand{\del}{\partial}
\newcommand{\LHS}{\mathrm{LHS}}\newcommand{\RHS}{\mathrm{RHS}}\newcommand{\bnu}{\boldsymbol{\nu}}\newcommand{\interior}{\mathrm{in}\;}\newcommand{\SH}{\mathrm{SH}}\renewcommand{\v}{\boldsymbol{\nu}}\newcommand{\n}{\mathbf{n}}\newcommand{\ssub}{\Subset}\newcommand{\curl}{\mathrm{curl}}
%%% 常微分方程式
\newcommand{\Ei}{\mathrm{Ei}}\newcommand{\sn}{\mathrm{sn}}\newcommand{\wgamma}{\widetilde{\gamma}}
%%% 統計力学
\newcommand{\Ens}{\mathrm{Ens}}
%%% 解析力学
\newcommand{\cl}{\mathrm{cl}}\newcommand{\x}{\boldsymbol{x}}

%%% 統計的因果推論
\newcommand{\Do}{\mathrm{Do}}
%%% 応用統計学
\newcommand{\mrl}{\mathrm{mrl}}
%%% 数理統計
\newcommand{\comb}[2]{\begin{pmatrix}#1\\#2\end{pmatrix}}\newcommand{\bP}{\mathbb{P}}\newcommand{\compsub}{\overset{\textrm{cpt}}{\subset}}\newcommand{\lip}{\textrm{lip}}\newcommand{\BL}{\mathrm{BL}}\newcommand{\G}{\mathbb{G}}\newcommand{\NB}{\mathrm{NB}}\newcommand{\oR}{\o{\R}}\newcommand{\liminfn}{\liminf_{n\to\infty}}\newcommand{\limsupn}{\limsup_{n\to\infty}}\newcommand{\esssup}{\mathrm{ess.sup}}\newcommand{\asto}{\xrightarrow{\as}}\newcommand{\Cov}{\mathrm{Cov}}\newcommand{\cQ}{\mathcal{Q}}\newcommand{\VC}{\mathrm{VC}}\newcommand{\mb}{\mathrm{mb}}\newcommand{\Avar}{\mathrm{Avar}}\newcommand{\bB}{\mathbb{B}}\newcommand{\bW}{\mathbb{W}}\newcommand{\sd}{\mathrm{sd}}\newcommand{\w}[1]{\widehat{#1}}\newcommand{\bZ}{\boldsymbol{Z}}\newcommand{\Bernoulli}{\mathrm{Ber}}\newcommand{\Ber}{\mathrm{Ber}}\newcommand{\Mult}{\mathrm{Mult}}\newcommand{\BPois}{\mathrm{BPois}}\newcommand{\fraks}{\mathfrak{s}}\newcommand{\frakk}{\mathfrak{k}}\newcommand{\IF}{\mathrm{IF}}\newcommand{\bX}{\mathbf{X}}\newcommand{\bx}{\boldsymbol{x}}\newcommand{\indep}{\raisebox{0.05em}{\rotatebox[origin=c]{90}{$\models$}}}\newcommand{\IG}{\mathrm{IG}}\newcommand{\Levy}{\mathrm{Levy}}\newcommand{\MP}{\mathrm{MP}}\newcommand{\Hermite}{\mathrm{Hermite}}\newcommand{\Skellam}{\mathrm{Skellam}}\newcommand{\Dirichlet}{\mathrm{Dirichlet}}\newcommand{\Beta}{\mathrm{Beta}}\newcommand{\bE}{\mathbb{E}}\newcommand{\bG}{\mathbb{G}}\newcommand{\MISE}{\mathrm{MISE}}\newcommand{\logit}{\mathtt{logit}}\newcommand{\expit}{\mathtt{expit}}\newcommand{\cK}{\mathcal{K}}\newcommand{\dl}{\dot{l}}\newcommand{\dotp}{\dot{p}}\newcommand{\wl}{\wt{l}}\newcommand{\Gauss}{\mathrm{Gauss}}\newcommand{\fA}{\mathfrak{A}}\newcommand{\under}{\mathrm{under}\;}\newcommand{\whtheta}{\wh{\theta}}\newcommand{\Em}{\mathrm{Em}}\newcommand{\ztheta}{{\theta_0}}
\newcommand{\rO}{\mathrm{O}}\newcommand{\Bin}{\mathrm{Bin}}\newcommand{\rW}{\mathrm{W}}\newcommand{\rG}{\mathrm{G}}\newcommand{\rB}{\mathrm{B}}\newcommand{\rN}{\mathrm{N}}\newcommand{\rU}{\mathrm{U}}\newcommand{\HG}{\mathrm{HG}}\newcommand{\GAMMA}{\mathrm{Gamma}}\newcommand{\Cauchy}{\mathrm{Cauchy}}\newcommand{\rt}{\mathrm{t}}
\DeclareMathOperator{\erf}{erf}

%%% 圏
\newcommand{\varlim}{\varprojlim}\newcommand{\Hom}{\mathrm{Hom}}\newcommand{\Iso}{\mathrm{Iso}}\newcommand{\Mor}{\mathrm{Mor}}\newcommand{\Isom}{\mathrm{Isom}}\newcommand{\Aut}{\mathrm{Aut}}\newcommand{\End}{\mathrm{End}}\newcommand{\op}{\mathrm{op}}\newcommand{\ev}{\mathrm{ev}}\newcommand{\Ob}{\mathrm{Ob}}\newcommand{\Ar}{\mathrm{Ar}}\newcommand{\Arr}{\mathrm{Arr}}\newcommand{\Set}{\mathrm{Set}}\newcommand{\Grp}{\mathrm{Grp}}\newcommand{\Cat}{\mathrm{Cat}}\newcommand{\Mon}{\mathrm{Mon}}\newcommand{\Ring}{\mathrm{Ring}}\newcommand{\CRing}{\mathrm{CRing}}\newcommand{\Ab}{\mathrm{Ab}}\newcommand{\Pos}{\mathrm{Pos}}\newcommand{\Vect}{\mathrm{Vect}}\newcommand{\FinVect}{\mathrm{FinVect}}\newcommand{\FinSet}{\mathrm{FinSet}}\newcommand{\FinMeas}{\mathrm{FinMeas}}\newcommand{\OmegaAlg}{\Omega\text{-}\mathrm{Alg}}\newcommand{\OmegaEAlg}{(\Omega,E)\text{-}\mathrm{Alg}}\newcommand{\Fun}{\mathrm{Fun}}\newcommand{\Func}{\mathrm{Func}}\newcommand{\Alg}{\mathrm{Alg}} %代数の圏
\newcommand{\CAlg}{\mathrm{CAlg}} %可換代数の圏
\newcommand{\Met}{\mathrm{Met}} %Metric space & Contraction maps
\newcommand{\Rel}{\mathrm{Rel}} %Sets & relation
\newcommand{\Bool}{\mathrm{Bool}}\newcommand{\CABool}{\mathrm{CABool}}\newcommand{\CompBoolAlg}{\mathrm{CompBoolAlg}}\newcommand{\BoolAlg}{\mathrm{BoolAlg}}\newcommand{\BoolRng}{\mathrm{BoolRng}}\newcommand{\HeytAlg}{\mathrm{HeytAlg}}\newcommand{\CompHeytAlg}{\mathrm{CompHeytAlg}}\newcommand{\Lat}{\mathrm{Lat}}\newcommand{\CompLat}{\mathrm{CompLat}}\newcommand{\SemiLat}{\mathrm{SemiLat}}\newcommand{\Stone}{\mathrm{Stone}}\newcommand{\Mfd}{\mathrm{Mfd}}\newcommand{\LieAlg}{\mathrm{LieAlg}}
\newcommand{\Sob}{\mathrm{Sob}} %Sober space & continuous map
\newcommand{\Op}{\mathrm{Op}} %Category of open subsets
\newcommand{\Sh}{\mathrm{Sh}} %Category of sheave
\newcommand{\PSh}{\mathrm{PSh}} %Category of presheave, PSh(C)=[C^op,set]のこと
\newcommand{\Conv}{\mathrm{Conv}} %Convergence spaceの圏
\newcommand{\Unif}{\mathrm{Unif}} %一様空間と一様連続写像の圏
\newcommand{\Frm}{\mathrm{Frm}} %フレームとフレームの射
\newcommand{\Locale}{\mathrm{Locale}} %その反対圏
\newcommand{\Diff}{\mathrm{Diff}} %滑らかな多様体の圏
\newcommand{\Quiv}{\mathrm{Quiv}} %Quiverの圏
\newcommand{\B}{\mathcal{B}}\newcommand{\Span}{\mathrm{Span}}\newcommand{\Corr}{\mathrm{Corr}}\newcommand{\Decat}{\mathrm{Decat}}\newcommand{\Rep}{\mathrm{Rep}}\newcommand{\Grpd}{\mathrm{Grpd}}\newcommand{\sSet}{\mathrm{sSet}}\newcommand{\Mod}{\mathrm{Mod}}\newcommand{\SmoothMnf}{\mathrm{SmoothMnf}}\newcommand{\coker}{\mathrm{coker}}\newcommand{\Ord}{\mathrm{Ord}}\newcommand{\eq}{\mathrm{eq}}\newcommand{\coeq}{\mathrm{coeq}}\newcommand{\act}{\mathrm{act}}

%%%%%%%%%%%%%%% 定理環境(足助先生ありがとうございます) %%%%%%%%%%%%%%%

\everymath{\displaystyle}
\renewcommand{\proofname}{\bf\underline{[証明]}}
\renewcommand{\thefootnote}{\dag\arabic{footnote}} %足助さんからもらった.どうなるんだ?
\renewcommand{\qedsymbol}{$\blacksquare$}

\renewcommand{\labelenumi}{(\arabic{enumi})} %(1),(2),...がデフォルトであって欲しい
\renewcommand{\labelenumii}{(\alph{enumii})}
\renewcommand{\labelenumiii}{(\roman{enumiii})}

\newtheoremstyle{StatementsWithUnderline}% ?name?
{3pt}% ?Space above? 1
{3pt}% ?Space below? 1
{}% ?Body font?
{}% ?Indent amount? 2
{\bfseries}% ?Theorem head font?
{\textbf{.}}% ?Punctuation after theorem head?
{.5em}% ?Space after theorem head? 3
{\textbf{\underline{\textup{#1~\thetheorem{}}}}\;\thmnote{(#3)}}% ?Theorem head spec (can be left empty, meaning ‘normal’)?

\usepackage{etoolbox}
\AtEndEnvironment{example}{\hfill\ensuremath{\Box}}
\AtEndEnvironment{observation}{\hfill\ensuremath{\Box}}

\theoremstyle{StatementsWithUnderline}
    \newtheorem{theorem}{定理}[section]
    \newtheorem{axiom}[theorem]{公理}
    \newtheorem{corollary}[theorem]{系}
    \newtheorem{proposition}[theorem]{命題}
    \newtheorem{lemma}[theorem]{補題}
    \newtheorem{definition}[theorem]{定義}
    \newtheorem{problem}[theorem]{問題}
    \newtheorem{exercise}[theorem]{Exercise}
\theoremstyle{definition}
    \newtheorem{issue}{論点}
    \newtheorem*{proposition*}{命題}
    \newtheorem*{lemma*}{補題}
    \newtheorem*{consideration*}{考察}
    \newtheorem*{theorem*}{定理}
    \newtheorem*{remarks*}{要諦}
    \newtheorem{example}[theorem]{例}
    \newtheorem{notation}[theorem]{記法}
    \newtheorem*{notation*}{記法}
    \newtheorem{assumption}[theorem]{仮定}
    \newtheorem{question}[theorem]{問}
    \newtheorem{counterexample}[theorem]{反例}
    \newtheorem{reidai}[theorem]{例題}
    \newtheorem{ruidai}[theorem]{類題}
    \newtheorem{algorithm}[theorem]{算譜}
    \newtheorem*{feels*}{所感}
    \newtheorem*{solution*}{\bf{[解]}}
    \newtheorem{discussion}[theorem]{議論}
    \newtheorem{synopsis}[theorem]{要約}
    \newtheorem{cited}[theorem]{引用}
    \newtheorem{remark}[theorem]{注}
    \newtheorem{remarks}[theorem]{要諦}
    \newtheorem{memo}[theorem]{メモ}
    \newtheorem{image}[theorem]{描像}
    \newtheorem{observation}[theorem]{観察}
    \newtheorem{universality}[theorem]{普遍性} %非自明な例外がない.
    \newtheorem{universal tendency}[theorem]{普遍傾向} %例外が有意に少ない.
    \newtheorem{hypothesis}[theorem]{仮説} %実験で説明されていない理論.
    \newtheorem{theory}[theorem]{理論} %実験事実とその(さしあたり)整合的な説明.
    \newtheorem{fact}[theorem]{実験事実}
    \newtheorem{model}[theorem]{模型}
    \newtheorem{explanation}[theorem]{説明} %理論による実験事実の説明
    \newtheorem{anomaly}[theorem]{理論の限界}
    \newtheorem{application}[theorem]{応用例}
    \newtheorem{method}[theorem]{手法} %実験手法など,技術的問題.
    \newtheorem{test}[theorem]{検定}
    \newtheorem{terms}[theorem]{用語}
    \newtheorem{solution}[theorem]{解法}
    \newtheorem{history}[theorem]{歴史}
    \newtheorem{usage}[theorem]{用語法}
    \newtheorem{research}[theorem]{研究}
    \newtheorem{shishin}[theorem]{指針}
    \newtheorem{yodan}[theorem]{余談}
    \newtheorem{construction}[theorem]{構成}
    \newtheorem{motivation}[theorem]{動機}
    \newtheorem{context}[theorem]{背景}
    \newtheorem{advantage}[theorem]{利点}
    \newtheorem*{definition*}{定義}
    \newtheorem*{remark*}{注意}
    \newtheorem*{question*}{問}
    \newtheorem*{problem*}{問題}
    \newtheorem*{axiom*}{公理}
    \newtheorem*{example*}{例}
    \newtheorem*{corollary*}{系}
    \newtheorem*{shishin*}{指針}
    \newtheorem*{yodan*}{余談}
    \newtheorem*{kadai*}{課題}

\raggedbottom
\allowdisplaybreaks
\usepackage[math]{anttor}
\begin{document}
\tableofcontents

\chapter{基本概念}

\begin{quotation}
    偏微分方程式論は,一般の多様体上の微分作用素の逆を,積分変換の形で書き表そうとする理論である.
\end{quotation}

\begin{notation}
    $U\osub\R^n$とする.
    \begin{description}
        \item[関数の微分について] $u\in C^2(U)$に対して,
        \begin{enumerate}
            \item 微分作用素を$\partial_{x_i}:=\pp{}{x_i}$とも表す.
            \item 勾配$D:C^1(\Om)\to C(\Om;\R^n)$を
            \[Du=\paren{\pp{u}{x_1},\cdots,\pp{u}{x_n}}\]
            で表す.特に値をベクトル場と見做す時,$\nabla,\grad$とも表す.
            \item Hesse行列を
            \[D^2u(x)=\begin{pmatrix}\pp{^2u}{x_1^2}&\cdots&\pp{^2u}{x_1\partial x_n}\\\vdots&\ddots&\vdots\\\pp{^2u}{x_n\partial x_1}&\cdots&\pp{^2u}{x_n^2}\end{pmatrix}\]
            で表す.$H(u),(\nabla\otimes\nabla)(u)$とも表す.
            \item d'Alembertianを$\Box u:=u_{tt}-\Laplace u$で表す.ただし$\Lap$は$\R^{n-1}$のものとした.
            \item 点$\b{a}\in U$での$\bnu\in\R^n$方向の微分を
            \[D_{\bnu}u(\b{a}):=\left.\dd{u(\b{a}+t\bnu)}{t}\right|_{t=0}=\v\cdot Du(\b{a})\]
            で表す.方向微分作用素は$\pp{}{\v},\v\cdot\nabla$とも表す.
            \item $\partial U$に沿った,外向き単位法線ベクトルを$\nu=(\nu^1,\cdots,\nu^n)$で表す.
        \end{enumerate}
        \item[ベクトル場の微分について] $u\in C^2(U;\R^n)$に対して,
        \begin{enumerate}
            \item $Du$または$\pp{u}{x}$で\textbf{Jacobi行列}を表す.勾配行列ともいう.
            \item $Ju:=\abs{\det Du}$でJacobianを表す.
            \item ベクトル場$u$からみた\textbf{Lagrange微分}または物質微分を
            \[\frac{Du}{Dt}:=u_t+\bnu\cdot Du\]
            で表す.
            \item 発散$\div:\X(U)\to C(U)$を
            \[\div u=\pp{u^1}{x^1}+\cdots+\pp{u^n}{x^n}=\Tr(Du)\]
            で表す.$\nabla\cdot u$とも表す.
        \end{enumerate}
    \end{description}
\end{notation}

\begin{notation}
    多重指数について,
    \begin{enumerate}
        \item $\al\in\N^n$を多重指数とし,$\abs{\al}:=\al_1+\cdots+\al_n$を次数とする.
        \item 多重指数$\al\in\N^n$について,$D^\al u:=\pp{^{\abs{\al}}u}{x_1^{\al_1}\cdots\partial x^{\al_n}_n}$.
        \item 自然数$k\in\N$について,$D^ku:=\paren{D^\al u}_{\abs{\al}=k}$を$k$階微分の族とする.例えば$D^2u\in M_n(\R)$はHesse行列である.
        \item $(x_1+\cdots+x_n)^k=\sum_{\abs{\al}=k}\frac{\abs{\al}!}{\al_1!\cdots\al_n!}x_1^{\al_1}\cdots x_n^{\al_n}$を$\sum_{\abs{\al}=k}\frac{\abs{\al}!}{\al!}x^\al$と表すと,Taylor展開は次のように表せる:
        \[u(x)=\sum_{k=0}^\infty\sum_{\abs{\al}=k}\frac{D^\al u(0)}{\al!}x^\al.\]
    \end{enumerate}
\end{notation}

\begin{proposition}\label{prop-chain-rule}
    $U\osub\R^n$について,
    \begin{enumerate}
        \item 連鎖律:$v\in C^1(U)$について,
        \[\dd{v(x_1(t),\cdots,x_n(t))}{t}=\pp{v}{x_1}\pp{x_1}{t}+\cdots+\pp{v}{x_n}\pp{v_n}{t}=Dv(x_1(t),\cdots,x_n(t))\cdot Dx(t).\]
        \item Leibniz則:$u,v\in C^1(U)$について,
        \[D^\al(uv)=\sum_{\beta\le\al}\comb{\al}{\beta}D^\beta uD^{\al-\beta}v.\]
        \item 多項定理:
        \[(x_1+\cdots+x_n)^k=\sum_{\abs{\al}=k}\comb{\abs{\al}}{\al}x^\al.\]
    \end{enumerate}
\end{proposition}

\begin{notation}
    積分について,
    \begin{enumerate}
        \item $\dx=\frac{dx}{2\pi}$を平均測度とする.
        \item $\dint dx:=\int\dx$を平均積分とする.
        \item 平均測度に関するLebesgue空間を$\dL^p(\bT):= L^p(\bT,\dx)$とする.
    \end{enumerate}
\end{notation}

\section{偏微分方程式の定義}

\subsection{偏微分方程式の属性}

\begin{definition}[partial differential equation]
    写像$F:\R^{n^k}\times\R^{n^{k-1}}\times\cdots\times\R^n\times\R\times U\to\R$が定める関数$u:U\to\R$についての条件
    \[F(D^ku(x),D^{k-1}u(x),\cdots,Du(x),u(x),x)=0\quad(n\ge 2)\]
    を\textbf{$k$階の偏微分方程式}という.
\end{definition}

\begin{definition}[linear]
    $F$を未知関数$u$とその導関数$D^ku,D^{k-1}u,\cdots,Du,u$に依存する部分$L(D^ku,D^{k-1}u,\cdots,Du,u)$と,
    $x$のみに依存した既知の形を持つ部分$f$とに分解して考える:$F(D^ku(x),\cdots,u(x),x)=L(D^ku(x),\cdots,u(x))-f(x)$.
    \begin{enumerate}
        \item $F$が$D^ku(x),D^{k-1}u(x),\cdots,Du(x),u(x)$に関して線型であるとき,すなわち,
        \[Lu:=\sum_{\abs{\al}\le k}a_\al(x)D^\al u(x)=f(x),\qquad L:=\sum_{\abs{\al}\le k}a_\al(x)D^\al,\;f:U\to\R\;:u\text{に依らない既知関数}\]
        という表示を持つとき,これを\textbf{線型PDE}という.
        \item さらに$f=0$を満たすとき,$F$は\textbf{斉次}(homogeneous)であるという.
        \item $L$は$C^k(U)$上の線型作用素である.$L$の最高階の部分を\textbf{主要部}という.
    \end{enumerate}
\end{definition}

\subsection{1階線型PDEの分類}

\begin{example}[1階線型PDEの例]\mbox{}
    \begin{enumerate}
        \item 次の1階の線型偏微分方程式を\textbf{移流/輸送方程式(advection / transport equation)}という:
        \[u_t+\sum_{i=1}^nb^iu_{x_i}=0\;\on\;\R^n\times\R^+\quad(b\in\R^n).\]
        \item 次の1階の線型偏微分方程式を\textbf{Liouvilleの方程式}という:
        \[u_t-\sum_{i=1}^n(b^iu)_{x_i}=0.\]
    \end{enumerate}
\end{example}

\begin{example}[Hans Lewy 57の例]
    次の1階の線型PDEは解を持たない:
    \[\pp{u}{x_1}+i\pp{u}{x_2}-2i(x_1+ix_2)\pp{u}{x_3}=\dot{f}(x_3),\quad f\in C^1(\R).\]
    一般に定数係数の線型PDEは(ある意味で)解を持つ(Malgrange-Ehrenpreisの定理)が,多項式係数になった時点でこれは成り立たない.
    また,$F$に解析性を課すと,一般のPDEは一意な局所解を持つ(Cauchy-Kovalevskayaの定理)が,$C^\infty$-級なだけでは成り立たない.
    さらにMizohata (62)によるより単純な例で
    \[\pp{u}{x}+ix\pp{u}{y}=F(x,y)\]
    は解を持たない.
\end{example}

\subsection{2階線型PDEの分類}

\begin{definition}[2階線型PDEのPetrowskiの分類]
    実係数の2階線型PDE
    \[\sum_{i,j\in[n]}a_{ij}(Du,u,x)\pp{^2u}{x_i\partial x_j}+F(Du,x)=0\quad x\in U\]
    について,$u\in C^2(U)$ならば,$A:=(a_{ij})_{i,j\in[n]}$は対称になるように取れる.
    \begin{enumerate}
        \item $A$の固有値$\lambda_1,\cdots,\lambda_k>0,\lambda_{k+1},\cdots,\lambda_{k+l}<0,\lambda_{k+l+1},\cdots\lambda_n=0$について,
        \[\sum_{i\in[k]}u_{\eta_i\eta_j}-\sum_{i\in[l]}u_{\eta_i\eta_j}+F(Du,x)=0,\quad\eta_j:=\frac{\ep_j}{\sqrt{\abs{\lambda_j}}}.\]
        という形の表示を\textbf{標準形}という.
        \item $A$は可逆とする:$k+l=n$.
        \begin{enumerate}
            \item $A$の固有値が同符号のとき,\textbf{楕円型}という.
            \item $A$の固有値が一つだけ異符号のものを持つとき,\textbf{双曲型}という.
            \item $A$の符号数$(k,l)$が$k,l\ge 2$を満たすとき,\textbf{超双曲型}(ultra hyperbolic)という.
        \end{enumerate}
        \item $A$が$0$を固有値として持つとき,\textbf{(広義の)放物型}という.
        \item さらに,固有値$0$の重複度は$1$で,
        他の固有値は同符号,
        さらに対応する変数の1階の項が標準形において消えないとき,\textbf{狭義の放物型}という.
    \end{enumerate}
\end{definition}

\begin{definition}\mbox{}
    \begin{enumerate}
        \item 楕円型方程式$-\Laplace u=0$を\textbf{Laplace方程式}またはpotential方程式という.
        \item この非斉次化$-\Laplace u=f$を\textbf{Poisson方程式}という.ただし$f$は$u$に依らない未知関数とした.
        \item $-\Lap$に未知定数$\lambda\in\R$に関する線型項$(V(x)-\lambda)$を追加した楕円型方程式$(-\Lap+V-\lambda)u=0$を,\textbf{固有方程式},または\textbf{Helmholtz方程式}という.
        \item この非斉次化$(-\Lap+V-\lambda)u=f$を\textbf{resolvent方程式}という.
        \item $-\Lap_x$に新変数の一階微分$D_t$を追加して得る方程式$u_t-\Lap u=0$は放物型で,\textbf{熱・拡散方程式}という.
        \item この複素化$-iD_t$を追加して得る$iu_t+\Lap u=0$をSchrodinger方程式という.
        \item $-\Lap_x$に新変数の二階微分$D_{tt}$を追加して得る方程式$u_{tt}-\Lap u=0$は双曲型で,\textbf{波動方程式}という.
        \item この2階の線型微分作用素$\Box u:= -\Laplace u+\pp{^2u}{t^2}$を\textbf{d'Alembertian}という.
        これを用いると,波動方程式は$\Box u=0$と表せる.
    \end{enumerate}
\end{definition}

\begin{example}\mbox{}
    \begin{enumerate}
        \item $u_{xx}+yu_{yy}=0$は,$\Brace{(x,y)\in\R^2\mid y>0}$で楕円形,$\Brace{(x,y)\in\R^2\mid y<0}$で双曲型,$\Brace{(x,y)\in\R^2\mid y=0}$で広義放物型となる.このような場合,解の性質を体系的に解析することは非常に難しい.
        \item 次の2次元の2階PDEは,$D=ac-b^2$が正ならば楕円型,負ならば双曲型,零ならば広義放物型となる:
        \[au_{xx}+2bu_{xy}+cu_{yy}+du_x+eu_y+fu=0.\]
        \item $A=(a_{ij})\in M_n(\C)$のとき,どの型にも当てはまらない場合が多く,Schrodinger方程式はその例である.
    \end{enumerate}
\end{example}

\subsection{非線型PDEの分類}

\begin{tcolorbox}[colframe=ForestGreen, colback=ForestGreen!10!white,breakable,colbacktitle=ForestGreen!40!white,coltitle=black,fonttitle=\bfseries\sffamily,
title=]
    非線型PDEも,最高階については線型であるとき,救いようがある.
    線型$\subset$半線形$\subset$準線型$\subset$完全非線型の包含関係があり,右に行くほど「非線型性が強い」という.
\end{tcolorbox}

\begin{definition}[quasilinear, semilinear, fully nonlinear]
    非線型PDEについて,
    \begin{enumerate}
        \item 最高階の項が線型であるとき,すなわち,次の表示を持つとき,\textbf{準線型PDE}(quasilinear)という:
        \[\sum_{\abs{\al}=k}a_\al(D^{k-1}u,\cdots,Du,u,x)D^\al u+F(D^{k-1}u,D^{k-2}u,\cdots,Du,u,x)=0.\]
        \item さらに,最高階の項の係数が低階項$u,Du,\cdots,D^{k-1}u$に依存せず,$a_\al(x)$と表せるとき,\textbf{半線型PDE}(semilinear)という.
        \item $F$が最高階の項に対しても非線型のとき,\textbf{完全非線型}(fully nonlinear)という.
    \end{enumerate}
\end{definition}
\begin{remark}
    準線型方程式はLagrangeの偏微分方程式ともいう.
\end{remark}

\begin{example}
    次の方程式は準線型であるが,半線型ではない.これを\textbf{極小曲面(Plateau problem)の方程式}\ref{model-plateau-problem}という:
    \[\paren{1+\paren{\pp{u}{y}}^2}\pp{^2u}{x^2}-2\pp{u}{x}\pp{u}{y}\pp{^2u}{x\partial y}+\paren{1+\paren{\pp{u}{x}}^2}\pp{^2u}{y^2}=0.\]
\end{example}

\subsection{1階線型PDEの解法}

\begin{tcolorbox}[colframe=ForestGreen, colback=ForestGreen!10!white,breakable,colbacktitle=ForestGreen!40!white,coltitle=black,fonttitle=\bfseries\sffamily,
title=]
    1階の単独偏微分方程式については,Cauchyの定理から古典解の存在が保証される.
    (実は,ある種の2変数の高階方程式もGreen関数が存在するという意味で同様に扱える).またその形は,Hamilton-Jacobi方程式に帰着できる.
    1階の単独PDEは,線型・非線型の別に拘らず,1階の常微分方程式系(これを特性方程式といい,その解を特性曲線という)と幾何学的に繋がっている.
    その最初の発見がHamilton-Jacobi理論であった.
\end{tcolorbox}

\begin{theorem}[Cauchy]
    $f_i\in C^l(U)\;((t^0,u^0)\in U\osub\R^{n+1},i\in[n],l\ge1)$について,次の常微分方程式系と初期条件を満たす解がただ一つ存在し,これは$C^{l+1}$-級である:
    \[\forall_{i\in[n]}\;\dd{u_i}{t}=f_i(t,u)\;\on U,\quad u_i(t^0)=u_i^0.\]
\end{theorem}

\begin{theorem}[線型1階PDE]
    方程式
    \[\sum_{j\in[n]}a_j(x)\pp{u}{x_j}=0\]
    と$u\in  C^1(U)$について,次は同値.
    \begin{enumerate}
        \item $u$は解である.
        \item 任意の$x^0\in U$を通る特性曲線$X_t$上について,$u(x)$の値は一定になる.
    \end{enumerate}
\end{theorem}

\begin{theorem}[斉次な輸送方程式の初期値問題]
    \[\begin{cases}
        u_t+b\cdot Du=0\quad\mathrm{in}\R^n\times(0,\infty)\\
        u=g\quad\on\R^n\times\{0\}.
    \end{cases}\]
    が十分なめらかな解$u$を持つならば,
    \[u(x,t):=g(x-tb)\quad(x\in\R^n,t\in\R_+)\]
    がその解である.特に,
    \begin{enumerate}
        \item $g\in C^1$のとき,これは解である.
        \item $g$が$C^1$-級のとき,$C^1$-級の解を持たない.
    \end{enumerate}
\end{theorem}

\begin{theorem}[非斉次な輸送方程式の初期値問題]
    \[\begin{cases}
        u_t+b\cdot Du=f\quad\mathrm{in}\R^n\times(0,\infty)\\
        u=g\quad\on\R^n\times\{0\}.
    \end{cases}\]
    の解は次のように表せる:
    \[u(x,t)=g(x-tb)+\int^t_0f(x+(s-t)b,s)ds\quad(x\in\R^n,t\in\R_+).\]
\end{theorem}

\subsection{1階非線形PDEの解法}

\begin{tcolorbox}[colframe=ForestGreen, colback=ForestGreen!10!white,breakable,colbacktitle=ForestGreen!40!white,coltitle=black,fonttitle=\bfseries\sffamily,
title=]
    前項の議論で,PDEをODEに変換した手法を抽出して,「特性曲線の手法」としてまとめることで,1解PDEの一般理論が構築出来る.
\end{tcolorbox}

\begin{definition}[準線型1階PDE]
    次の1階の単独な準線型PDEを考える:
    \[\sum_{i\in[n]}a_i(x_1,\cdots,x_n,u)\pp{u}{x_i}=a(x_1,\cdots,x_n,u).\]
    これは次の常微分方程式系に同値であり,これを\textbf{特性方程式}という:
    \[\forall_{i\in[n]}\;\dd{x_i(t)}{t}=a_i(x_1,\cdots,x_n,u),\quad \dd{u(t)}{t}=a(x_1,\cdots,x_n,u).\]
    実際,PDEは$(x_1,\cdots,x_n,u)\in\R^{n+1}$の空間における超曲面$S:u=u(x_1,\cdots,x_n)$がベクトル$(a_1,\cdots,a_n,a)$に直交することを意味している.
    \begin{enumerate}
        \item ベクトル$(a_1,\cdots,a_n,a)$を\textbf{Mongeのベクトル}という.
        \item これがなすベクトル場$(x_1,\cdots,x_n,u)\mapsto(a_1,\cdots,a_n,a)$を\textbf{特性ベクトル場}という.
        \item 超曲面$S$は,特性ベクトル場の積分曲線の合併として得られる.これらの積分曲線$(x_1(t),\cdots,x_n(t),u(t))$を\textbf{特性曲線}(characteristic curve)という.
        \item $(x_1(t),\cdots,x_n(t))$を特性基礎曲線という.
        \item 陰関数$\phi(x_1,\cdots,x_n,u)=c$が$c\in\R$に依らずにPDEの解を与えるとき,これを\textbf{(第一)積分}という.$\phi$が実際に積分で表せることと,PDEが求積可能であることは同値.
    \end{enumerate}
\end{definition}

\begin{definition}[一般の1階PDE]
    次の1階単独PDEを考える:
    \[F\paren{x_1,\cdots,x_n,u,\pp{u}{x_1},\cdots,\pp{u}{x_n}}=0.\]
    $p_i:=\pp{u}{x_i}$も独立変数と見做すと,次に同値:
    \[\begin{cases}
        \dd{x_i}{t}=\pp{F}{p_i},&i\in[n],\\
        \dd{u}{t}=\sum_{i\in[n]}p_i\pp{F}{p_i},\\
        \dd{p_i}{t}=-\paren{p_i\pp{F}{u}+\pp{F}{x_i}}&i\in[n].
    \end{cases}\]
    これを\textbf{特性方程式}といい,この解曲線$(x(t),u(t),p(t))$を\textbf{特性帯}といい,$(x(t),u(t))$を\textbf{特性曲線}という.
\end{definition}
\begin{history}
    この特性方程式を解くことによって,元のPDEを解くことができる,という理論はLagrangeとCauchyによって証明された.
    一方でJacobiは,十分多くのパラメータを持つPDEの解が得られれば,微分と消去法によって特性方程式の一般解が求まることを示した.
    このことの意味を深く問うたのはElie Cartanであり,S. Lieの偏微分方程式の幾何学的取り扱いと,Poincaréによる力学系の積分不変式の理論との上に,「特性系の理論」を打ち立てた.
    この研究から,外微分形式の理論が生まれ育っていった.
\end{history}

\begin{example}
    次の形をした1階単独PDE(のクラス)を\textbf{Hamilton-Jacobi方程式}という:
    \[\pp{u}{t}+H\paren{t,x,\pp{u}{x}}=0.\]
    対応する特性方程式系の一部に,次の1階の常備分方程式系が含まれており,これを座標$x$と運動量$p$に関する\textbf{正準方程式系}という:
    \[\begin{cases}
        \dd{x_i}{t}=\pp{H}{p_i}\\
        \dd{p_i}{t}=-\pp{H}{x_i}
    \end{cases}\quad i\in[n].\]
    次の方程式系も併せれば,特性方程式系をなす:
    \[\begin{cases}
        \dd{u}{t}=\sum_{i\in[n]}p_i\pp{H}{p_i}-H,\\
        \dd{h}{t}=-\pp{H}{t}
    \end{cases},\quad h:=\pp{u}{t}.\]
    正準方程式系は正準変換=シンプレクティック変換という座標変換のクラスについて不変である.
    これに対応する1階単独PDEの座標変換を接触変換という.Poissonの括弧式に対応するのが角括弧式またはLagrangeの括弧式である.
\end{example}

\subsection{連立系}

\begin{definition}
    $F_1,\cdots,F_M$に関する偏微分方程式の集合を\textbf{方程式系}といい,$M=1$の場合を\textbf{単独方程式}という.
\end{definition}

\begin{example}
    Cauchy-Riemann方程式は2元の連立系である.
    Maxwell方程式は6元に関する方程式で,次の定数係数線形偏微分作用素に関する:
    \[\rot= \begin{pmatrix}0&-\pp{}{x_3}&\pp{}{x_2}\\\pp{}{x_3}&0&-\pp{}{x_1}\\-\pp{}{x_2}&\pp{}{x_1}&0\end{pmatrix}\]
\end{example}

\subsection{擬微分作用素}

従来,求積法,Cauchyの存在定理を基礎に置いたHamilton-Jacobi理論,Cauchy-Kovalevskajaの定理に基づく構成の3つのみが一般理論として得られ,後は各論しか存在しないものと目されてきた.
\begin{quotation}
    しかし,最近のL. HörmanderらのFourier積分作用素の研究(71,72),及び佐藤幹夫-河合隆裕-柏原正樹の量子化強いた接触変換の理論(73)により,線型擬微分方程式について新たな一般理論が建設された.
    というのも,
    \begin{enumerate}
        \item 解の特異性のみに注目するならば,解は方程式の主要部から定まる1階PDEである陪特性方程式によってほとんど決まってしまう.
        \item 陪特性方程式の同次正準変換あるいは接触変換に応じる方程式及び解の変換が存在する.
    \end{enumerate}
    ことが明らかにされた.(1)は,音や光が波動であるにも拘らず,あたかも粒子であるかのように伝わる事実の数学的裏付けである.
\end{quotation}

\section{例と歴史}

\begin{remarks}[偏微分方程式の逆問題]
    偏微分方程式による数理モデルは,重回帰による数理モデルと同様に,観測から逆問題を考え得る.
\end{remarks}

\subsection{工学への応用}

\begin{definition}
    Dirichlet境界条件にあたる「固定端の問題」を\textbf{本質的境界条件}という.
    一方で,Neumann境界条件(=境界での法線微分を与える)にあたる「自由端の問題」を\textbf{自然境界条件}という.
\end{definition}

\subsection{熱・拡散方程式}

\begin{tcolorbox}[colframe=ForestGreen, colback=ForestGreen!10!white,breakable,colbacktitle=ForestGreen!40!white,coltitle=black,fonttitle=\bfseries\sffamily,
title=]
    単位時間あたりにやりとりする熱の量は温度勾配に等しいことが,各点での$2$階の偏微分方程式を要請する.
    これは物質の濃度変化=拡散が濃度勾配に比例するという現象に相似である.
\end{tcolorbox}

\begin{model}[熱伝導の模型]
    綿密度$\rho$と比熱$\sigma$を持つ一次元直線$\R$上の温度分布$u(t,x)$の経時変化を考えると,まず熱伝導の法則(Newtonの冷却法則)が,時区間$[t,t+\De t]$における流入量$\paren{\pp{u}{t}\De t}(\rho\De x)\sigma$として
    \[\paren{k\pp{u}{t}(t,x+\Delta x)-k\pp{u}{x}(t,x)}\Delta t=k\pp{^2u}{x^2}\Delta x\Delta t\]
    を要請する.ただし,熱のやりとりは区間$[x,x+\De x]$の右端と左端で符号の取り方を変えていること,等号は書かれている以上の微小な違いを除いて成り立つことを意味する.
    この式の$\De x,\De t\to0$での極限を考えると,
    \[\rho\sigma\pp{u}{t}=k\pp{^2u}{x^2}\]
    となる.これを1次元の\textbf{熱伝導方程式}という.
\end{model}
\begin{remarks}[境界条件について]
    (斉次な)Dirichlet境界条件は両端を熱浴に接していること(物質拡散の模型では吸収壁という),(斉次な)Neumann条件は両端で断熱されている(または反射壁という)ことを表現しているとみなせる.
    一方この2つの和として表せる境界条件を\textbf{Robinの境界条件},数学では第3種境界条件という.
\end{remarks}
\begin{history}
    熱方程式は求積可能ではないが,変数分離法が使え,この方面の研究がFourier解析の起源となる.
\end{history}

\begin{example}[移流拡散の模型]
    溶媒が一定・一様の速度$\beta$を持っている場合,次の方程式が得られる:
    \[\pp{u}{t}=\nu\pp{^2u}{x^2}-\beta\pp{u}{x}.\]
    新たな項を\textbf{移流項}ともいう.これは物質の総増加量が
    \[(u(t+\De t,x+\beta\De t)-u(t,x))\De x=\paren{\pp{u}{t}+\beta\pp{u}{x}}\De t\De x\]
    と表せることによる.
\end{example}

\subsection{化学:反応拡散方程式}

\begin{tcolorbox}[colframe=ForestGreen, colback=ForestGreen!10!white,breakable,colbacktitle=ForestGreen!40!white,coltitle=black,fonttitle=\bfseries\sffamily,
title=]
    物質の生成される速度が濃度の自乗に比例$\dd{u}{t}=ku^2$し,さらにこの反応が伝播していくとき,
\end{tcolorbox}

\begin{model}[scalar reaction-diffution equation]
    一般に,
    \[u_t-\Lap u=f(u)\]
    という形の方程式を\textbf{反応拡散方程式}という.
    例えば,
    物質の生成される速度が濃度の自乗に比例$\dd{u}{t}=ku^2$し,さらにこの反応が伝播していく現象を考えると,
    \[\pp{u}{t}=\nu\pp{^2u}{x^2}+ku^2.\]
    という非線形方程式を得る.
\end{model}

\begin{model}[反応と繁殖の拡散模型]\mbox{}
    \begin{enumerate}
        \item 生物が過密になると濃度の自乗で減り,しかし濃度に比例して個体数が増える,という現象は,1次の項も登場する
        \[\dd{u}{t}=au-bu^2\]
        と表せる.これを\textbf{ロジスティック方程式}という.
        \item さらに拡散も考える場合,次の方程式を得る:
        \[\pp{u}{t}=\nu\pp{^2u}{x^2}+au-bu^2.\]
        これはFisher (1930)が遺伝学的優生種の密度変化を記述する模型として提案され,数学的にはKolmogorov-Petrovskii-Piskunov (KPP)により進行波解が研究された.
        他に燃焼面の移動の模型としても用いられる.
    \end{enumerate}
\end{model}

\begin{example}
    数学では次の形で研究される:
    \[\pp{u}{t}=\Laplace u+f(u),\quad u:\R^n\times\R\to\R^n\]
    これは$\R^n$上にフラクタル的なパターン形成をする.
    KPP方程式の他に,次のような場合は名前がついている.
    \begin{enumerate}
        \item 藤田型方程式:$f(u)=ku^\al$.
        \item Allen-Cahn方程式:$f(u)=u(1-u)(u-a)$.
    \end{enumerate}
\end{example}

\subsection{流体力学:多孔質媒体方程式}

\begin{model}
    多孔性(porous)の媒質では,見えなかった変数$m\ge 1$が動く:
    \[u_t-\Lap(u^m)=0.\]
    $m>1$のとき,媒質自体が物質を吸収する現象を考慮しており,これを\textbf{多孔質媒体方程式}という.
    特に$mu^{m-1}$を拡散係数とする拡散方程式と見ることができ,このような場合非線形拡散方程式という.
    次のように導出される:
    \begin{enumerate}
        \item 物質の密度$u$は圧力$p$の関数である:$u=\gamma p^\al$.これは個体の状態方程式とみれる.
        \item 物質の移動速度を$v$とすれば,物質の量の収支は,$\nu$を多孔性の係数として,
        \[\pp{(uv)}{x}=-\nu\pp{u}{t}\]
        を要請する.
        \item 最後に,Darcyの法則(物質の運動速度は圧力勾配に比例する)が要請する関係式$v=-k\pp{p}{x}$によって$p,v$を消去し,$m:=1+1/\al$とおくと得る.
    \end{enumerate}
\end{model}

\subsection{流体力学:連続方程式}

\begin{tcolorbox}[colframe=ForestGreen, colback=ForestGreen!10!white,breakable,colbacktitle=ForestGreen!40!white,coltitle=black,fonttitle=\bfseries\sffamily,
title=]
    連続体の運動方程式$\pp{\rho}{t}+\div(\rho\v)=0$は非線形になるという点で本質的に難しい系になる.
\end{tcolorbox}

\begin{model}[質量保存の法則]
    一般に,ベクトル値関数$F$に対して,
    \[u_t+\div F(u)=0.\]
    をスカラー保存則の方程式という.
    特に,
    密度$\rho:\R_+\times\R^3\to\R_+$と流速$v:\R_+\times\R^3\to\R^3$について,
    \[\pp{\rho}{t}(x,y,z)+\div(\rho(x,y,z)v(x,y,z))=0\]
    を\textbf{流体力学の連続方程式}という.
    これは,密度の時間変化が,連続体の流出量に等しい,という質量保存の法則を表している.
    \begin{enumerate}
        \item 特に液体など,密度が一定・一様な流体を考えると,$\rho_t=0,\nabla\rho=0$より,方程式は$0=\div(v)$となる.
        これを非圧縮性流体という.
        \item 方程式の右辺を$\frac{D\rho}{Dt}$とおいて,\textbf{流れに沿っての密度の時間微分}という.
    \end{enumerate}
\end{model}

\subsection{流体力学:Eulerの運動方程式}

\begin{tcolorbox}[colframe=ForestGreen, colback=ForestGreen!10!white,breakable,colbacktitle=ForestGreen!40!white,coltitle=black,fonttitle=\bfseries\sffamily,
title=非圧縮性非粘性流体]
    完全流体の運動方程式を特にEulerの方程式ともいう.
    密度の時空間分布については「圧縮」というが,流体の変形に対する抵抗を「粘性」と言い,粘性がない流体は完全流体という.
\end{tcolorbox}

\begin{model}
    $u$をベクトル場として,
    \[u_t+u\cdot Du=-Dp,\quad\div u=0.\]
    この左辺の微分作用素$\partial_t+\v\cdot\nabla$を\textbf{Lagrange微分}または\textbf{物質微分}という.
    流体の流れに沿っての微分を意味している.
\end{model}

\subsection{流体力学:Navier-Stokes方程式}

\begin{tcolorbox}[colframe=ForestGreen, colback=ForestGreen!10!white,breakable,colbacktitle=ForestGreen!40!white,coltitle=black,fonttitle=\bfseries\sffamily,
title=非圧縮性粘性流体]
    粘性を考慮して得られる,\textbf{非圧縮性粘性流体の運動方程式}をNavier-Stokes方程式という.
\end{tcolorbox}

\begin{model}
    速度ベクトルに直角な方向の速度変化は剪断摩擦力を生じ,速度ベクトルに沿った速度変化は体積弾性応力を生じる.
    この模型として,粘性力は流体内の任意の微小平面片を通して,その点での$u$の一階の偏微分に線型に依存するベクトルとして作用すると仮定する,この条件を満たす流体を\textbf{Newton流体}という.
    応力テンソルに関する煩雑な考慮を経ると,次を得る:
    \[u_t+u\cdot Du-\Lap u=-Dp,\quad\div u=0.\]
\end{model}

\subsection{構造力学:弾性曲線方程式}

\begin{model}[beam equation]
    構造力学において,棒状の直線部材であって,引張・圧縮などの軸力に加え,剪断と曲げモーメントも作用する部材のこと\textbf{はり部材}(beam)という.
    はり部材のたわみ$u$についてのEuler-Lagrange方程式は次のようになる:
    \[u_{tt}+u_{xxxx}=0.\]
\end{model}

\subsection{解析力学:Hamilton-Jacobi方程式}

\begin{model}
    幾何光学において,
    \[\abs{Du}=1\]
    をEikonal方程式という.
    形式的には,Hamilton-Jacobi方程式
    \[u_t+H(Du,x)=0\]
    の一種である.
\end{model}

\subsection{統計力学:Brown運動}

\begin{tcolorbox}[colframe=ForestGreen, colback=ForestGreen!10!white,breakable,colbacktitle=ForestGreen!40!white,coltitle=black,fonttitle=\bfseries\sffamily,
title=]
    形式上は移流拡散方程式と同じ支配方程式に従う.
\end{tcolorbox}

\begin{model}
    ある粒子の存在確率を$u(t,x)$とすると,これは熱や濃度と違う法則に従う.
    粒子の単位時間の運動が$N(c,\nu)$に従うとき,
    \[\pp{u}{t}=\frac{\nu}{2}\pp{^2}{x^2}-c\pp{u}{x}.\]
    これはドリフトを移流項とする拡散方程式とみれる.
\end{model}
\begin{history}
    さらに,Einsteinの研究を受けて,統計物理学の分野でLangevin方程式なども提案されている.
    が,見本道の可微分性も加味したモデルは確率微分方程式が与える.
\end{history}

\subsection{統計力学:Fokker-Planck方程式}

\begin{model}
    \[u_t-\sum_{i,j=1}^n(a^{ij}u)_{x_ix_j}-\sum_{i=1}^n(b^iu)_{x_i}=0\]
    という2階の偏微分方程式を\textbf{Fokker-Planck方程式}という.
    \begin{enumerate}
        \item さらなる高階項を持つ方程式Kramers-Moyal展開の2次で切ったものと見れる.
        \item 物理量$x(t)$の揺らぎが確率微分方程式$\dot{x}=a(x,t)+b(x,t)W_t$に従うとき,$x(t)$の確率分布$P(x,t)$がFokker-Planck方程式に従う.
        \item 特に線形ブラウン運動(オルンシュタイン=ウーレンベック過程)に対する方程式を線形フォッカー・プランク方程式といい,このときはLangevin方程式となる.
    \end{enumerate}
\end{model}

\subsection{拡散過程:Kolmogorov後退方程式}

\begin{tcolorbox}[colframe=ForestGreen, colback=ForestGreen!10!white,breakable,colbacktitle=ForestGreen!40!white,coltitle=black,fonttitle=\bfseries\sffamily,
    title=]
    Fokker-Planck方程式の符号が違う場合である.

    放物型のPDEに,Schrödinger方程式$iu_t+\Lap u=0$と熱・拡散方程式$u_t-\Lap u=0$とがある.
    KacはFeynmanの経路積分による量子化の発見に触発されて,その複素化されていないバージョンでもある熱拡散方程式に対して,
    確率過程の言葉で厳密な定式化を行った.
    一方で,Schrödinger方程式に対しては,測度論の方法に頼ることはできない.
\end{tcolorbox}

\begin{model}[KBE (Kolmogorov backward equations) 1931]
    拡散過程理論で現れる
    \[u_t-\sum_{i,j=1}^na^{ij}u_{x_ix_j}+\sum_{i=1}^nb^iu_{x_i}=0\]
    をKolmogorov方程式という.
    このうち,前進方程式もあったが,これは物理学でのFokker-Planck方程式としてすでに知られているものであった.
\end{model}

\begin{problem}[Kolmogorovの後退方程式に関するCauchy問題]
    初期値$f\in C_b(\R^d),V,g\in C_b(\R^d)$を持ち,$\A$の係数は1次の増大条件
    \[\norm{\al(t,x)}+\abs{b(t,x)}\le K(1+\abs{x})\quad(\abs{x}\to\infty)\]
    を満たすとする.次のKolmogorovの後退方程式のCauchy問題の解$u\in C^{1,2}(\R^+\times\R^d)\cap C(\R_+\times\R^d)$を考える:
    \[\mathrm{(C)}\quad\begin{cases}
        \pp{u}{t}=\A u+Vu+g\quad t>0,x\in\R^d,\\
        u(0,x)=f(x),\quad x\in\R^d.
    \end{cases}\]
\end{problem}


\begin{theorem}[Feynman-Kac formula]
    (C)の解$u$が存在して緩増加$\forall_{T>0}\;\exists_{C,p>0}\;\forall_{t\in[0,T]}\;\forall_{x\in\R^d}\;\abs{u(t,x)}\le C(1+\abs{x}^p)$ならば,解は一意で,次のように表示できる:
    \[u(t,x)=E_x\Square{f(X_t)\exp\paren{\int^t_0V(X_s)ds}+\int^t_0g(X_s)\exp\paren{\int^s_0V(X_r)dr}ds}.\]
\end{theorem}

\subsection{波動方程式}

\begin{model}[弦の振動の模型]
    一様な綿密度$\rho$を持つ弦が2次元空間内で第二成分方向に振動するとし,その変異$u:\R_+\times\R\to\R$を記述することを考える.
    点$x\in\R$におけるNewtonの運動方程式の微小近似は,その点での張力の鉛直成分$T\sin\theta$として微分係数$T\sin\theta\simeq T\pp{u}{x}$を用いることより,$T\in\R$を張力として,
    \[\rho dx\pp{^2u}{t^2}(t,x)=T\pp{u}{x}(t,x+dx)-T\pp{u}{x}(t,x)\]
    を要請する.これは$c:=\sqrt{T/\rho}$として,
    \[\frac{1}{c^2}\pp{^2u}{t^2}-\pp{^2u}{x^2}=0\]
    と表せる.
\end{model}
\begin{remarks}
    重要な観察として,横波も縦波(疎密派)も同じ方程式に従うことが挙げられる.
    波動方程式は求積可能である.
\end{remarks}

\subsection{非斉次波動:電信方程式}

\begin{model}
    \[u_{tt}+2du_t-u_{xx}=0.\]
\end{model}

\begin{example}[電信模型]
    外力項のある波動方程式
    \[\pp{^2u}{t^2}-c^2\pp{^2u}{x^2}=f(t,x)\]
    のうち,$f$を速度抵抗$f(t,x)=-k\pp{u}{t}(t,x)$とした場合を\textbf{電信方程式}という.
    Heavisideにより直線状のケーブルを伝わる電磁波の減衰を記述する際に用いられたモデルである.
    $k$が定数ならば$u$の変換により1階項は消せて,次のようになる:
    \[\pp{^2v}{t^2}-c^2\pp{^2v}{x^2}-\kappa^2v=0\quad v:=ue^{kt/2},k=2\kappa.\]
    これは波動方程式と違って求積不可能だという.
\end{example}

\subsection{微分幾何:極小曲面}

\begin{tcolorbox}[colframe=ForestGreen, colback=ForestGreen!10!white,breakable,colbacktitle=ForestGreen!40!white,coltitle=black,fonttitle=\bfseries\sffamily,
title=]
    次の方程式を極小曲面方程式という:
    \[\div\paren{\frac{Du}{(1+\abs{Du}^2)^{1/2}}}=0.\]
\end{tcolorbox}

\begin{model}[変分原理による模型]\label{model-plateau-problem}
    変分原理(最小エネルギー原理)による振動の模型化はより強力な処理ができる.
    \begin{enumerate}
        \item 外力$f$が保存力であるとき,釣り合いの方程式は$-T\Laplace u=f$とまとめられる.
        特に$T=1,f=0$のときにこれを導出した,膜全体のエネルギーを表す汎関数は
        \[D[u]=\frac{1}{2}\iint_\Om\paren{\paren{\pp{u}{x}}^2+\paren{\pp{u}{y}}^2}dxdy\]
        と表せ,\textbf{Dirichlet積分}という.
        \item 一方で,微小近似を行わずに解析すると,$-T\Laplace u=f$の代わりに次を得る:
        \[\paren{1+\paren{\pp{u}{y}}^2}\pp{^2u}{x^2}-2\pp{u}{x}\pp{u}{y}\pp{^2u}{x\partial y}+\paren{1+\paren{\pp{u}{x}}^2}\pp{^2u}{y^2}=0.\]
        これを\textbf{極小曲面の方程式}という.
        これは膜の振動において,膜の面積を極小にするという条件に値するためである.
        \item 極小曲面の方程式の左辺の量を$(1+\abs{\nabla u}^2)^{3/2}$で正規化したものは,曲面$z=u(x,y)$の\textbf{平均曲率}という.
        極小曲面は平均曲率が$0$になるような曲面であるといえる.
        \item 一方で非保存力であるときは,仮想仕事の原理から$\rho\pp{^2u}{t^2}=T\Laplace u+f$を得る.
        Neumann境界条件を自然境界条件というのは,変分法から自然に導かれることによる.
    \end{enumerate}
\end{model}

\subsection{微分幾何:Monge-Ampere方程式}

\begin{model}
    \[\det(D^2u)=f.\]
\end{model}

\subsection{非線形波動:Burgers方程式とKdV方程式}

\begin{tcolorbox}[colframe=ForestGreen, colback=ForestGreen!10!white,breakable,colbacktitle=ForestGreen!40!white,coltitle=black,fonttitle=\bfseries\sffamily,
title=]
    
\end{tcolorbox}

\begin{model}
    \[u_{tt}-\Lap u+f(u)=0\]
    を非線形波動方程式という.
\end{model}

\begin{model}[Korteweg-De Vries方程式]
    \[\pp{u}{t}+c\pp{u}{x}=0\]
    を1階の波動方程式という.
    熱・拡散方程式と符号を除いて同じ形をしている.
    \begin{enumerate}
        \item この非線形化
        \[\pp{u}{t}+cu\pp{u}{x}=0\]
        は波が有限の時間で崩れてしまう(解析解は多価になる).
        これを粘性のないBurgers方程式(inviscid Burgers)という.
        \item 消散項を追加することでこれは防げる:
        \[\pp{u}{t}+cu\pp{u}{x}-v\pp{^2u}{x^2}=0\]
        これを\textbf{Burgersの方程式}と言い,1次元のNavier-Stokes方程式から圧力項を省略したものとも思える.
        \item 一方で3階の消散項を採用するとKdV方程式を得る:
        \[u_t+uu_x+u_{xxx}=0.\]
        \item この線形化を\textbf{Airy方程式}という:
        \[u_t+u_{xxx}=0.\]
    \end{enumerate}
\end{model}

\begin{model}
    砕けるようなことのない穏やかな波は偏微分方程式でモデル化可能である.
    これは最終的に4つの方程式にまとまる.
    これらをうまく変換して,種々の方程式が作り出される.
    \begin{enumerate}
        \item 次を\textbf{KdV方程式}という.Korteweg-de Vries方程式の略称である.
        \[\pp{u}{t}+6u\pp{u}{x}+\pp{^3u}{x^3}=0.\]
        これには1-ソリトン解(遠くまで崩れずに進行する孤立波)
        \[u(t,x)=\frac{2\kappa^2}{\cosh^2\kappa(x-4\kappa^2t-\delta)}\]
        が存在する.
        \item 
    \end{enumerate}
\end{model}
\begin{history}
    KdV方程式の研究の契機は,1834年に英国の造船技師かつ流体力学者J. Scott Russellがエディンバラ郊外の運河で孤立波を見て,数マイルも馬に乗って追いかけることができたことが始まりである.
    1895には定式化された.
\end{history}

\subsection{電磁気:Maxwellの波動方程式}

\begin{model}
    次の4式を言う.
    \begin{enumerate}
        \item $\div E=\frac{\rho}{\ep_0}$.
        \item $\div B=0$.
        \item $\rot E+\pp{B}{t}=0$.
        \item $\frac{\rot B}{\mu_0}=i+\ep_0\pp{E}{t}$.
    \end{enumerate}
    これらから,電磁波の支配方程式
    \[\frac{1}{c^2}\pp{^2H}{t^2}=\Laplace H\]
    を得る.
\end{model}

\subsection{相対論:Klein-Gordon方程式}

\begin{model}
    スピン0の相対論的な自由粒子を表す場(クライン–ゴルドン場)が満たす方程式
    \[u_{tt}-\Lap u+m^2u=0.\]
\end{model}

\subsection{量子論:Schrödingerの波動方程式}

\begin{tcolorbox}[colframe=ForestGreen, colback=ForestGreen!10!white,breakable,colbacktitle=ForestGreen!40!white,coltitle=black,fonttitle=\bfseries\sffamily,
title=]
    複素係数の偏微分方程式であることが,他の例と比べて特異的である.
\end{tcolorbox}

\begin{model}
    Hamilton系の作用$S$が満たす偏微分方程式
    \[\pp{S}{t}+H\paren{x,\pp{S}{x}}=0.\]
    を特性方程式に持つような偏微分方程式
    \[\frac{h}{i}\pp{u}{t}+H\paren{x,\frac{h}{i}\pp{}{x}}u=0\]
    を\textbf{Schrödingerの波動方程式}という.
\end{model}

\begin{example}
    特にポテンシャル関数$V$が与える場で質量$m$の一粒子が運動するとき,$H(x,p)=p^2/2m+V(x)$で,
    \[-\frac{h}{i}\pp{u}{t}=-\frac{h^2}{2m}\Laplace u+V(x)u\]
    となる.Schrödingerは変分原理の類比的適用から得た.
    Feynmanは経路積分により,この基本解を先に得る形式を立てた.
\end{example}

\begin{model}
    次を非線形Schrodinger方程式という:
    \[iu_t+\Lap u=f(\abs{u}^2)u.\]
\end{model}

\section{解の定義}

\begin{tcolorbox}[colframe=ForestGreen, colback=ForestGreen!10!white,breakable,colbacktitle=ForestGreen!40!white,coltitle=black,fonttitle=\bfseries\sffamily,
title=]
    偏微分方程式は統計模型同様,模型だと思った方がいい.
\end{tcolorbox}

\begin{history}[well-posedness]
    近代の偏微分方程式論は,方程式の物理学的な性質の説明を数学者の自前にしようとしたHadamardの自覚に源を発し,Petrovskiiが集大成した.
    Hadamardは数理物理の問題として偏微分方程式に当たるとき,
    \begin{enumerate}
        \item 付加条件を課した問題の解が一意に存在しする.
        \item 解が安定である,すなわち,データに連続に依存する.
    \end{enumerate}
    ときに,\textbf{well-posed}であると呼んだ.
    方程式に含まれる偏微分記号を形式的文字で置き換えて得られる特性多項式の代数幾何学的性質によって,
    その方程式の解の性質を統制するというHadamardの考えは今後も指導理念であり続けることだろう.
\end{history}

\subsection{PDEの積分方程式への変換}

\begin{tcolorbox}[colframe=ForestGreen, colback=ForestGreen!10!white,breakable,colbacktitle=ForestGreen!40!white,coltitle=black,fonttitle=\bfseries\sffamily,
title=]
    偏微分方程式を1つの線形作用素と見る.
\end{tcolorbox}

\begin{definition}\mbox{}
    \begin{enumerate}
        \item $f\in C(I)$を外力などのデータとして,
        \[(Sf)(x):=\int^1_0G(x,y)f(y)dy\]
        として定まる写像$S:C(I)\to C^2(I)$は線型作用素である.これを\textbf{解作用素}または\textbf{Green作用素}という.
        \item データに対して解作用素が定まり(一価性),それが何らかのノルムについて連続であるとき,元のPDEを\textbf{well-posed}であるという.
    \end{enumerate}
\end{definition}
\begin{remark}
    一般にGreen関数は見つからない.
    そこで,変分問題による定式化を通じて,最小化解を捉え,それが元の境界値問題の解であることを確かめる手法がよく使われる.
\end{remark}

\begin{definition}
    積分方程式について,
    \begin{enumerate}
        \item 積分区間に変数が現れるものをVolterra型,現れないものをFredholm型という.
        \item 未知関数が被積分関数を通じてのみ現れる場合を第1種,そうでない場合を第2種という.
        \item 既知関数$f$が零であるとき,斉次であるという.
    \end{enumerate}
\end{definition}

\begin{example}
    初期値問題
    \[-\dd{^2u}{x^2}+q(x)u=f(x)\;\on[0,1],\quad u(0)=u(1)=0,q\in C([0,1]).\]
    はGreen関数法によって,積分方程式に変換できる.すなわち,$u$が上の初期値問題の解であることは,次の積分方程式に同値:
    \[u(x)-\int^1_0K(x,y)u(y)dy=g(x).\]
    このように,連続関数$K$を核とする積分作用素を$T$として,$u-Au=g$の形を持つ積分方程式を\textbf{Fredholm型の第二種積分方程式}という.
    この型の積分方程式の研究が,HilbertによるHilbert空間の概念を芽生えさせた.
\end{example}

\subsection{微分作用素の逆としての積分方程式}

\begin{tcolorbox}[colframe=ForestGreen, colback=ForestGreen!10!white,breakable,colbacktitle=ForestGreen!40!white,coltitle=black,fonttitle=\bfseries\sffamily,
title=]
    次に第一種積分方程式を考えると,これは固有値問題になる.
    特に微分作用素の固有関数が基底をなすならば,解作用素がGreen作用素によるものと別の形で与えられる.
    さらに,固有関数がわかっているのならば,$e^{-tA}$に当たるものを考えれば,
    \[\dd{u}{t}=-Au\quad t\in\R_+\]
    という形の$u$の出現にも対応できる.これは
    熱方程式である.しかし$e^{-tA}$が曲者で,一般論は行列のようにはいかず,Hille-Yoshida理論が必要になる.
\end{tcolorbox}

\begin{example}[Laplacianの逆]
    $A:=-\Lap$を微分作用素とすると,Poisson方程式は$Au=f$と表せる.いわば,この逆作用素$A^{-1}$を求めれば良い.
    実は,Green関数$G$が存在して
    \[A^{-1}f=\int_\Om G(-,y)f(y)dy\]
    と表せるが,この$G$は$x=y$にて特異性を持ち,もはや連続核ではない.
\end{example}



\subsection{安定性}

\begin{definition}
    あるノルム$N$について,
    解が与えられたデータの定数倍で抑えられるとき,\textbf{$N$-安定である}という.
    これを一般化すると,ある種の連続性がデータに対して成り立つとき,安定であるという.
\end{definition}

\subsection{正値性}

\begin{definition}\mbox{}
    \begin{enumerate}
        \item 調和関数の最大値の定理から明らかであるが,$\partial\Om$上で$u\ge0$を満たすならば$\Om$上でも$u\ge0$であるとき,\textbf{正値性を保存する}という.
        \item 同値な事実であるが,$\partial\Om$上の順序関係が$\Om$上でも成り立つことを,\textbf{順序を保存する}という.
    \end{enumerate}
\end{definition}

\subsection{古典解と弱解}

\begin{definition}
    2階の偏微分作用素
    \[L:=\sum_{i,j\in[n]}a_{ij}(x,t)\pp{^2}{x_i\partial x_j}+\sum_{i\in[n]}b_i(x,t)\pp{}{x_i}+c(x,t)\]
    について,
    \begin{enumerate}
        \item $u\in C^2(U)$が$\forall_{x\in U}\;Lu(x)=0$を満たすとき,\textbf{古典解}という.
        \item $u\in C^{2,1}(U)$が$\forall_{x\in U}\;u_t(x)=Lu(x)$を満たすとき,\textbf{古典解}という.
        \item $C^{2,1}(U)$よりも大きな空間の元で,より一般的な意味で方程式を満たすものを\textbf{弱解}という.
    \end{enumerate}
\end{definition}

\begin{example}\mbox{}
    \begin{enumerate}
        \item 最も標準的な弱解はSchwartz超関数によるものである.
        \item また,楕円型,放物型方程式では,最大値原理に注目して粘性解と呼ばれる弱解が導入される(Pierre-Louis Lions).
        \item 線型PDEについては,2つの弱解は同値である(石井仁司 1995).
    \end{enumerate}
\end{example}

\subsection{付加条件とデータ}

\begin{tcolorbox}[colframe=ForestGreen, colback=ForestGreen!10!white,breakable,colbacktitle=ForestGreen!40!white,coltitle=black,fonttitle=\bfseries\sffamily,
title=]
    PDEでは常微分方程式に比べて「自由度」が高すぎるため,初期条件や境界条件という形で条件を強くして問題を精緻に考える.
\end{tcolorbox}

\section{解析解の存在}

\begin{tcolorbox}[colframe=ForestGreen, colback=ForestGreen!10!white,breakable,colbacktitle=ForestGreen!40!white,coltitle=black,fonttitle=\bfseries\sffamily,
title=]
    PDEのCauchy問題の解法に,冪級数の方法があり得る.
    これはCauchy-Kovalevskayaの定理に基礎をおく.
    これは一般には完全非線型な方程式にも成り立つ.
\end{tcolorbox}

\subsection{1階系への還元}

\begin{problem}
    一般の$k$階の準線型PDEに関する,非特性的な境界上の解析的なデータに関するCauchy問題は,
    一般性を失うことなく次の形であるとみなしてよい:
    \[\begin{cases}
        \sum_{\abs{\al}=k}a_\al(D^{k-1}u,\cdots,u,x)D^\al u+a_0(D^{k-1}u,\cdots,u,x)=0'&\abs{x}<r,\\
        u=\pp{u}{x_n}=\cdots=\pp{^{k-1}u}{x_n^{k-1}}=0&\abs{x'}<r,x_n=0.
    \end{cases},\qquad x':=(x_1,\cdots,x_{n-1}).\]
    さらに,1階系に直すために次を定義する:
    \begin{enumerate}
        \item $u$の$k$階以下の偏導関数からなるベクトルを
        \[\b{u}:=\paren{u,\pp{u}{x_1},\cdots,\pp{u}{x_n},\pp{^2u}{x_1^2},\cdots,\pp{^{k-1}u}{x_n^{k-1}}}.\]
        で表す.次元を$m$とおき,$u=(u^1,\cdots,u^m):\R^n\to\R^m$とみる.
        \item すると境界条件は$\b{u}=0$とみなせる.
    \end{enumerate}
    すると,問題は次に等価になる:
    \[\begin{cases}
        \b{u}_{x_n}=\sum_{j=1}^{n-1}\b{B}_j(\b{u},x')\b{u}_{x_j}+\b{c}(\b{u},x')&\abs{x}<r,\\
        \b{u}=0&\abs{(x_1,\cdots,x_{n-1})}<r,x_n=0.
    \end{cases}\]
    ただし,
    \[\b{B_j}=(b^{kl}_j):\R^m\times\R^{n-1}\to M_m(\R),\b{c}=(c^j):\R^m\times\R^{n-1}\to\R^m\;\text{は実解析的}\]
    はいずれも$x_n$に依らないとした.
    必要ならば$\b{u}$に余分な成分を追加して$x_n$とすればよいから,この仮定も一般性を失わない.
\end{problem}

\begin{theorem}[Cauchy-Kovalevskaja]
    $\{\b{B}_j\}_{j=1}^{n-1}$と$\{\b{c}\}$は実解析的とする.このとき,$r>0$と実解析関数
    \[\b{u}=\sum_{\al\in\N^n}\b{u}_\al x^\al.\]
    が存在して,Cauchy問題
    \[\begin{cases}
        u^k_{x_n}=\sum_{k=1}^{n-1}\sum_{l=1}^mb_j^{kl}(\b{u},x')u^l_{x_j}+c^k(\b{u},x')&k\in[m],\\
        \b{u}=0&\abs{(x_1,\cdots,x_{n-1})}<r,x_n=0.
    \end{cases}\]
    の解となる.
\end{theorem}

\subsection{線型方程式の場合}

\begin{lemma}
    冪級数
    \[\varphi(z)=\sum_{\al\in\N^n}a_\al z^\al\]
    について,次は同値:
    \begin{enumerate}
        \item $\Om_C:=\Brace{z\in\C^n\mid\abs{z_1}+\cdots+\abs{z_n}<C}$上で収束する.
        \item 任意の$0<C'<C$に対して,$M_{C'}>0$が存在して,
        \[\varphi(z)\ll M_{C'}\paren{1-\frac{z_1+\cdots+z_n}{C'}}^{-1}.\]
    \end{enumerate}
\end{lemma}

\begin{theorem}[\cite{大島利雄-PDE} 定理4.3]
    $\C^N$-値関数$u(t,z_1,\cdots,z_n)$に対する線型偏微分作用素
    \[P_ku:=\pp{^{m_k}u_k}{t^{m_k}}\sum_{i\in[N]}\sum_{j+\abs{\al}\le m_i,j\le m_i-1}a_{j\al}^{ki}(t,z)\pp{^{j+\abs{\al}}u_i}{t^j\partial z_1^{\al_1}\cdots z_n^{\al_n}},\qquad k\in[N].\]
    について,
    \begin{enumerate}[{[C}1{]}]
        \item 係数$a_{j\al}^{ki}$は
        \[\o{U_L}:=\Brace{(t,z)\in\C^{n+1}\mid\abs{t}+\abs{z_1}+\cdots+\abs{z_n}\le L^{-1}}\]
        の近傍で正則とする.
        \item このとき,任意の$L'>L$に対して,$K\ge1$が存在して,
        \[V_{L'}:=\Brace{z\in\C^n\mid\abs{z_1}+\cdots+\abs{z_n}<L'^{-1}},\quad U_{K,L'}:=\Brace{(t,z)\in\C^{n+1}\mid K\abs{t}+\abs{z_1}+\cdots+\abs{z_n}<L'^{-1}}.\]
        と定める.
    \end{enumerate}
    Cauchy問題
    \[\begin{cases}
        P_ku=f_k&k\in[N],\\
        \del^j_tu_k|_{t_0}=h_j^k&k\in[N],j\in m_k.
    \end{cases}\]
    のデータが$f_k\in\O(U_{K,L'}),h_j^k\in\O(V_{L'})$を満たすとき,任意の$L>L'$に対してある$K\ge1$が存在して,原点で正則な解がただ一つ存在し,$U_{K,L'}$上で正則である.
\end{theorem}

\begin{theorem}[局所解の正則延長]
    線型常微分方程式系のCauchy問題
    \[\M:\dd{^{m_k}u_k}{z^{m_k}}(z)-\sum_{i=1}^N\sum_{j=0}^{m_i-1}a_j^{ki}(z)\dd{^ju_i}{z^j}(z)=f_k(z),\qquad k\in[N]\]
    \[\dd{^ju_k}{z^j}(0)=h^k_j,\qquad k\in[N],j\in m_k.\]
    において,関数$a_j^{ki},f_k$はいずれも$\Om\osub\C$上で正則とする.
    このとき,任意の解$u$は,原点を出発する$\Om$内の任意の曲線に沿って解析接続される.
\end{theorem}

\section{ベクトル解析}

\begin{tcolorbox}[colframe=ForestGreen, colback=ForestGreen!10!white,breakable,colbacktitle=ForestGreen!40!white,coltitle=black,fonttitle=\bfseries\sffamily,
title=]
    微分作用素$\nabla:=\paren{\pp{}{x^1},\cdots,\pp{}{x^n}}:C^1(\Om)\to\X^0(\Om)$について,$\nabla f\in\X^0(\Om),\nabla\times X\in\X^0(\Om),\nabla\cdot X\in C(\Om)$という3種類の線型微分作用素についての結果の$n=3$の場合をベクトル解析という.
\end{tcolorbox}

\subsection{ベクトル場}

\begin{definition}[flow, complete]\mbox{}
    \begin{enumerate}
        \item $C^1$-級ベクトル場が与えられたとき,局所的には\textbf{流線}が存在する.これが$\R_+$上大域的に存在するとき\textbf{正に完備},$\R$上大域的に存在するとき\textbf{完備}という.
        \item 正に完備なベクトル場は\textbf{1-径数変換半群}$(T_s)_{s\in\R_+}$を定める.ベクトル場が完備のとき,1-径数変換群という.
        \item 逆に,変換群$(T_t)_{t\in\R}$から見て,対応する速度のベクトル場$X$を\textbf{無限小変換}という.
    \end{enumerate}
\end{definition}

\begin{definition}
    $f\in C^1(\Om)\;(\Om\osub E^3)$について,
    \begin{enumerate}
        \item \textbf{勾配}とは,次で定めるベクトル場をいう:
        \[\grad f:=\pp{f}{x}\b{i}+\pp{f}{y}\b{j}+\pp{f}{z}\b{k}.\]
        \item 微分作用素として$\nabla:C^1(\Om)\to\X(\Om)$を
        \[\nabla:=\pp{}{x}\b{i}+\pp{}{y}\b{j}+\pp{}{z}\b{k}\]
        で定める.$D$とも書く.
        \item 任意の$v\in T_p(\Om)$に対して,この方向の方向微分を
        \[D_vf:=v\cdot\grad f(p)\]
        で定める.
        \item 速度の場$\v\in\X(\Om)$に対して,関数の左作用$\rho\v$を\textbf{流束の場}という.
    \end{enumerate}
\end{definition}

\subsection{微分形式}

\begin{tcolorbox}[colframe=ForestGreen, colback=ForestGreen!10!white,breakable,colbacktitle=ForestGreen!40!white,coltitle=black,fonttitle=\bfseries\sffamily,
title=]
    線積分,面積分は,それぞれ1-形式と2-形式の積分をいう.
    微分形式は,一般の位相多様体上に定義できて,部分多様体(とその形式線形和)上に測度とそれによる積分を引き起こす.
    大域的に定義できるのは$k$-次元多様体上の$k$-形式で,それ以外の形式は適切な部分多様体の埋め込み上で定義され,微分形式の引き戻しによる.
\end{tcolorbox}

\begin{definition}[volume element, surface measure]\mbox{}
    \begin{enumerate}
        \item $\R^k$における$k$-形式を一般に体積要素といい,$\dvol$で表す.
        \item $\R^k$における$k-1$-形式を一般に曲面速度といい,$dS$で表す.
    \end{enumerate}
\end{definition}

\begin{definition}[微分形式の積分]
    $X$を$n$次元位相多様体,$\om\in\Om^n(X)$を連続な$n$-形式とする.$X$がパラコンパクトかつHausdorffならば1の分割$(w_U)$が取れる.
    $\om_U:=\om|_{U}$を制限とすると,$\om=\sum_Uw_U\om_U$が成り立つから,
    \[\int_X\om:=\sum_U\int_Uw_U(x^1,\cdots,x^n)\om_U(x^1,\cdots,x^n)dx^1\cdots dx^n\]
    と定めれば良い.
\end{definition}

\begin{theorem}
    この定義は座標系・1の分割の定義に依らないが,多様体$X$の向きに依る.
    また,微分形式から測度への対応は線型である.
\end{theorem}
\begin{remark}
    擬形式(pseudoform)という概念を使えば向きにも依らない.
    すると逆に,任意の絶対連続なRadon測度に対して,擬形式が対応する.
\end{remark}

\subsection{ベクトル値関数}

\begin{theorem}[逆関数定理]
    開近傍$x_0\in U\osub\R^n$上で
    $f\in C^1(U;\R^n)$かつ$Jf(x_0)\ne0$ならば,ある開近傍$x_0\in V\subset U$と$z_0\in W$が存在して,次を満たす:
    \begin{enumerate}
        \item $f|_V:V\to W$は$C^1$-級の可微分同相写像である.
        \item $f\in C^k(U;\R^n)$ならば,$f|_V$は$C^k$-級の可微分同相写像でもある.
    \end{enumerate}
\end{theorem}

\begin{notation}
    $(x,y)\in\R^{n+m}$で表す.
    $z_0=f(x_0,y_0)$に注目する.
    $Df=(D_xf,D_yf)\in M_{m,n+m}(C(\R^{n+m}))$と分解して,
    $J_yf:=\abs{\det D_yf}$と表す.
\end{notation}

\begin{theorem}[陰関数定理]\label{thm-Implicit-Function-Theorem}
    $U\osub\R^{n+m}$上の関数$f\in C^1(U;\R^m)$が$J_yf(x_0,y_0)\ne0$を満たすならば,
    開近傍$(x_0,y_0)\in V\subset U$と$x_0\in W\osub\R^n$と陽関数$g\in C^1(W;\R^m)$が存在して,次を満たす:
    \begin{enumerate}
        \item $g(x_0)=y_0$.
        \item $\forall_{x\in W}\;f(x,g(x))=z_0$.
        \item $\forall_{(x,y)\in V}\;f(x,y)=z_0\Rightarrow y=g(x)$.
        \item $\forall_{f\in C^k(U;\R^n)}\;g\in C^k(W;\R^m)$.
    \end{enumerate}
    この$g$は,方程式$f(x,y)=z_0$によって$x_0$の近傍で陰に定義されている,という.
\end{theorem}
\begin{remarks}
    $\R^{n+1}$上の曲線が,
    陰関数$f(x,y)=z_0\in\R^1$によって定義されているとする.
    そのうちのパラメータ$y\in\R^{1}$に注目したとき,$F_{y}(x_0,y_0)\ne0$ならば,
    $\R^{n}$上の関数$g(x)=y$のグラフとして局所的に理解することが出来る.
\end{remarks}

\section{場の積分}

\subsection{線積分}

\begin{tcolorbox}[colframe=ForestGreen, colback=ForestGreen!10!white,breakable,colbacktitle=ForestGreen!40!white,coltitle=black,fonttitle=\bfseries\sffamily,
title=]
    曲線上に押し出された測度を$ds:=\abs{Dx(t)}dt$または$\abs{ds}$,向き付きの測度を$\cdot ds:=\cdot Dx(t)dt$または$\cdot d\vec{s}$で表す.
    共通点は小文字ということである.
\end{tcolorbox}

\begin{definition}[関数とベクトル場の線積分]
    $x:I\to\R^n$を区間上の正則な$C^1$-曲線,$\gamma:=\Im x$とする.
    \begin{enumerate}
        \item 関数$f\in C(\gamma)$の線積分を
        \[\int_\gamma fds:=\int_If(x(t))\abs{Dx(t)}dt.\]
        と定める.$ds:=\abs{x'(t)}dt$を\textbf{線素}という.
        \item ベクトル場$X\in\X(\gamma)$の\textbf{接方向に関する線積分}を
        \[\int_\gamma X\cdot ds:=\int_I(X(\gamma(t))|D\gamma(t))dt.\]
        と定める.
        \item ベクトル場$X\in\X(\gamma)$の\textbf{法線方向に関する線積分}を
        \[\int_\gamma X\cdot\b{n}ds:=\int_I(X(\gamma(t))|\b{n}(\gamma(t)))\abs{Dx(t)}dt.\]
        と定める.
    \end{enumerate}
\end{definition}

\begin{proposition}[線素の陽関数による表示]
    $x:I\to\R^2$が正則であるとき,Jacobianは零でないから
    曲線は局所的な陽な表示を持ち,これを$x_2=f(x_1)$とする.
    このとき,線素は
    \[ds=\sqrt{1+f'(t)^2}dt\]
    に等しい.
\end{proposition}

\begin{proposition}[線素の単位法線ベクトルによる表示]
    $\nu:\Im(x)\to\R^2$を各点における上向きの単位法線ベクトルとすると,曲線$(t,f(t))$の方向ベクトルは$(1,f'(t))$であるから,各点で$\nu\perp(1,f')$となる.
    \begin{enumerate}
        \item $\nu=\paren{-\frac{f'}{\sqrt{1+f'^2}},\frac{1}{\sqrt{1+f'^2}}}$が成り立つ.
        \item $dx_1=\nu_2ds=\nu\cdot e_2ds$とも表せる.
        \item $F(x_1,x_2)=x_2-f(x_1)$とするとグラフの陰関数表示$F(x)=0$を得る.これについて,$\nu=\frac{\nabla F}{\abs{\nabla F}}$である.
    \end{enumerate}
\end{proposition}

\subsection{面積分}

\begin{tcolorbox}[colframe=ForestGreen, colback=ForestGreen!10!white,breakable,colbacktitle=ForestGreen!40!white,coltitle=black,fonttitle=\bfseries\sffamily,
title=]
    曲面上に押し出された測度を$dS:=\Abs{\pp{\psi}{x}(x,y)\times\pp{\psi}{y}(x,y)}dxdy$または$\abs{dS}$,向き付きの測度を$\cdot dS:=\cdot \paren{\pp{\psi}{x}(x,y)\times\pp{\psi}{y}(x,y)}dxdy=\cdot\b{n}dS$または$\cdot d\Vec{S}$とする.
\end{tcolorbox}

\begin{definition}[面積分]
    $R\subset\R^{n-1}$を領域,$\psi:R\to\Om\subset\R^3$を曲面とし,$S:=\Im\psi$とする.
    \begin{enumerate}
        \item $\psi:R\to\R^n$が曲面であるとは,$C^1$-級かつ各点でJacobianが消えないことをいう:$\rank\paren{\pp{\psi}{u}}=n-1$.
        \item 関数$f\in C(\Om)$の面積分を
        \[\int_SfdS:=\iint_R f(\psi(u))\Norm{\pp{\psi}{u^1}(u)\times\pp{\psi}{u^2}(u)}du^1du^2\]
        で定める.
        \item ベクトル場$X\in\X(\Om)$の面積分を
        \[\int_SX\cdot dS:=\iint_R\paren{X(\psi(u))\middle|\pp{\psi}{u^1}(u)\times\pp{\psi}{u^2}(u)}du^1du^2\]
        で定める.
    \end{enumerate}
\end{definition}

\begin{theorem}[面素の陽関数による表示]
    曲面$f:R\to\R^n,\Im f=:S$の面積は
    \[\int_R\paren{1+\paren{\pp{f}{x_1}}^2+\cdots+\paren{\pp{f}{x_{n-1}}}^2}^{1/2}dx_1\cdots dx_{n-1}\]
    と表せる.一般の場合はこれを面素とする.
\end{theorem}

\section{場の微分}

\begin{tcolorbox}[colframe=ForestGreen, colback=ForestGreen!10!white,breakable,colbacktitle=ForestGreen!40!white,coltitle=black,fonttitle=\bfseries\sffamily,
title=]
    Jacobi行列$Du$は,$u$が実数値のとき勾配に,$u$がベクトル値のときJacobi行列になる.
    $u$がベクトル値のとき,Jacobi行列の跡を発散という.
    勾配は単に微分であるため,微積分学の基本定理が成り立つ.
    実は,$D,\div,\rot$はいずれも外微分に対応する.
\end{tcolorbox}

\subsection{発散とGauss-Greenの定理}

\begin{tcolorbox}[colframe=ForestGreen, colback=ForestGreen!10!white,breakable,colbacktitle=ForestGreen!40!white,coltitle=black,fonttitle=\bfseries\sffamily,
title=]
    発散定理は,「微分の積分」の計算法を単位法線ベクトルの言葉で与えているという意味で,$\R^n$上への微積分学の基本定理の一般化である.
\end{tcolorbox}

\begin{theorem}[(Ostrogradsky-)Gauss-Green]\label{thm-Gauss-Green}
    $\Om\osub\R^n$を有界領域,$\partial\Om$を区分的$C^1$-級とする.
    \begin{enumerate}
        \item 関数$u\in C^1(\o{U})$について,
        \[\forall_{i\in[n]}\;\int_Uu_{x_i}dx=\int_{\partial U}u\nu^idS.\]
        \item ベクトル値関数$u\in C^1(\o{U};\R^n)$について,
        \[\int_U\div (u)dx=\int_{\partial U}u\cdot \nu\;dS.\]
    \end{enumerate}
\end{theorem}
\begin{history}
    Gauss (1839)以前にGreen (1828), Ostrogradskii (1831)も発表していた.
\end{history}

\subsection{Gauss-Greenの定理の系}

\begin{tcolorbox}[colframe=ForestGreen, colback=ForestGreen!10!white,breakable,colbacktitle=ForestGreen!40!white,coltitle=black,fonttitle=\bfseries\sffamily,
title=]
    $\R^n$上の部分積分は,その座標方向$i\in[n]$の境界$\partial U$の単位法線ベクトル$\b{\nu}$の成分$\nu^i$を考慮に入れねばならないが,そこだけが相違点である.
\end{tcolorbox}

\begin{corollary}[成分毎の部分積分公式]\label{cor-partial-integral}
    $u,v\in C^1(\o{U})$について,
    \[\tcboxmath{\forall_{i\in[n]}\quad\int_Uu_{x_i}vdx=-\int_Uuv_{x_i}dx+\int_{\partial U}uv\nu^idS.}\]
\end{corollary}
\begin{Proof}
    積$uv\in C^1(\o{U})$についてGauss-Greenの定理を用いると,
    \begin{align*}
        \forall_{i\in[n]}\quad\int_U(uv)_{x_i}dx=\int_{\partial U}uv\nu^idS.
    \end{align*}
\end{Proof}

\begin{corollary}[Greenの恒等式]\label{cor-Green-identity}
    $u,v\in C^2(\o{U})$について,次が成り立つ:
    \begin{enumerate}\setcounter{enumi}{-1}
        \item Laplacianに対する部分積分公式:
        \[\tcboxmath{\int_U\Lap udx=\int_{\partial U}\pp{u}{\nu}dS.}\]
        \item 勾配の内積に対する部分積分公式:
        \[\tcboxmath{\int_U(Dv|Du)dx=-\int_Uu\Lap vdx+\int_{\partial U}\pp{v}{\nu}udS.}\]
        \item 対称化された部分積分公式
        \[\tcboxmath{\int_U(u\Lap v-v\Lap u)dx=\int_{\partial U}\paren{u\pp{v}{\nu}-v\pp{u}{\nu}}dS.}\]
    \end{enumerate}
\end{corollary}
\begin{Proof}\mbox{}
    \begin{enumerate}
        \item 
        $u=u_{x_i}$に対して,関数に対するGauss-Greenの定理\ref{thm-Gauss-Green}(1)を用いる:
        \[\int_U(u_{x_i})_{x_i}dx=\int_{\partial U}u_{x_i}\nu^idS.\]
        これの$i\in[n]$に対する和を取れば主張を得る.
        また,$\div(Du)=\Lap u$に注意すれば,
        ベクトル値関数$Du$に関するGauss-Greenの定理\ref{thm-Gauss-Green}(2)から直接:
        \[\int_U\div(Du)dx=\int_{\partial U}Du\cdot\b{\nu}dS=\int_{\partial U}\pp{u}{\b{\nu}}dS.\]
        \item $u_{x_i}v_{x_i}$に関して部分積分を考えると,
        \[\int_{U}u_{x_i}v_{x_i}dx=-\int_Uuv_{x_ix_i}dx+\int_{\partial U}uv_{x_i}\nu^idS.\]
        これを$i\in[n]$について足し合わせたものである.
        \item (2)の$u,v$を入れ替えると,
        \[\int_UDu\cdot Dvdx=-\int_Uv\Lap udx+\int_{\partial U}\pp{u}{\nu}vdS\]
        を得る.(2)の辺々から引くと,
        \[0=\int_U(v\Lap u-u\Lap v)dx+\int_{\partial U}\paren{\pp{v}{\nu}u-\pp{u}{\nu}v}dS.\]
    \end{enumerate}
\end{Proof}
\begin{remarks}[法線方向微分とは]
    基本的に,$U$上の$D$は$\partial U$上の$\pp{}{\nu}$に対応する.
    これが(0)の精神であり,成分に拠らない形で理解されるGauss-Greenの定理である.
    これが劣調和函数の平均定理\ref{thm-mean-value-theorem-of-subharmonic-function}を導く.
    \begin{enumerate}
        \item 第一恒等式は部分積分公式である.
        \item 第二恒等式は対称な形にした部分積分公式に過ぎないが,
        $U$上の二階微分を$\partial U$上の法線方向の一階微分に還元する点で,Green関数の解法の中心になる\ref{lemma-decomposition-of-C2-function-through-Green-function}.
        これを第三恒等式とも言うらしい.
    \end{enumerate}
    \[u\xrightarrow{\text{微分}}Du\xrightarrow{\text{微分}}\Lap u\]
    とし,積分領域に合せて$dx,\cdot\b{\nu}dS$とを合せ,$Du\cdot\b{\nu}=\pp{u}{\b{\nu}}$の規則を忘れなければ間違えない.
\end{remarks}

\subsection{積分の微分}

\begin{tcolorbox}[colframe=ForestGreen, colback=ForestGreen!10!white,breakable,colbacktitle=ForestGreen!40!white,coltitle=black,fonttitle=\bfseries\sffamily,
title=]
    
\end{tcolorbox}

\begin{theorem}[積分領域が動く積分に対する微分\footnote{\cite{Evans}}]\label{thm-differentiation-of-integral-on-moving-region}
    $U_\tau\subset\R^n$を滑らかな境界を持つ領域の滑らかな族,
    $\partial U_\tau$の各点の速度ベクトルを$\b{v}$で表す.
    このとき,任意の滑らかな関数$f\in C^1(\R^n\times\R)$に対して,
    \[\dd{}{\tau}\int_{U_\tau}fdx=\int_{\partial U_\tau}f\b{v}\cdot\nu dS+\int_{U_\tau}f_\tau dx.\]
\end{theorem}
\begin{history}
    $n=2$ではLeibnizの微分則,$n=3$ではReynolds transport theorem
    と呼ばれ,連続体力学の分野で知られていた.
\end{history}
\begin{remarks}[物理的直感の先行]
    微積分学の基本定理の一般化ともみれる.
    領域が動く場合は,微分と積分の交換だけでなく,
    余分の項が生じ,それは動いた境界上での吸収量を表す項である.
\end{remarks}

\begin{example}\label{exp-moving-region-appeared-in-N-C-WE}
    非斉次な波動方程式の零化された初期値問題のDuhamelの原理による解公式\ref{thm-N-C-WE}において,次の計算を行なった,
    \[u(x,t):=\int^t_0u(x,t;s)ds\paren{=\int^1_0u(x,t;tr)tdr}.\]
    このとき,上の定理によると,
    \[u_t(x,t)=u(x,t;t)+\int^t_0u_t(x,t;s)ds.\]
    実際,微分の定義に戻ってみると,
    \begin{align*}
        u_t(x,t)&=\lim_{h\to0}\frac{u(x,t+h)-u(x,t)}{h}\\
        &=\lim_{h\to0}\frac{1}{h}\paren{\int^{t+h}_0u(x,t+h;s)ds-\int^t_0u(x,t;s)ds}\\
        &=\lim_{h\to0}\frac{1}{h}\int^{t+h}_hu(x,t+h;s)ds+\lim_{h\to0}\int^t_0\frac{u(x,t+h;s)-u(x,t;s)}{h}ds\\
        &=\lim_{h\to0}\frac{1}{h}\int^{t+h}_t\Paren{u(x,t;s)+hu_t(x,t;s)+O(h^2)}ds+\int^t_0u_t(x,t;s)ds.
    \end{align*}
\end{example}

\subsection{勾配}

\begin{definition}[勾配]
    $\nabla:C^\infty(\Om)\to\X(\Om)$を
    \[\nabla f:=f^1\pp{}{x^1}+\cdots+f^n\pp{}{x^n}\]
    で定める.これは$\grad f$とも表す.
\end{definition}

\begin{theorem}[勾配ベクトル場の積分定理(1次元のStokesの定理)]
    $f\in C^1(\Om),\gamma:I\to\Om\subset\R^n$を曲線とすると,
    \[\int_\gamma\grad(f)\cdot dx=\int_{\partial\gamma}f=\int_{\partial I}f\circ\gamma=f(\gamma(b))-f(\gamma(a)).\]
\end{theorem}
\begin{remarks}
    この定理は2次元のGaussの発散定理を含む?
\end{remarks}

\subsection{微積分学の基本定理の幾何的構造}

\begin{definition}
    $\om\in\Om^p(\Om)\;(p\in\N^+)$について,
    \begin{enumerate}
        \item 外微分が消えることを\textbf{閉}であるという.
        \item ポテンシャルを持つことを\textbf{完全}であるという:$\exists_{\eta\in\Om^{p-1}(\Om)}\;\om=d\eta$.
        \item $p=1$のときのポテンシャルを第一積分という.
    \end{enumerate}
\end{definition}

\begin{theorem}[Poincaré]
    単連結領域$\Om\osub\R^n$について,閉形式は完全である.
\end{theorem}
\begin{remarks}
    閉形式と完全形式とはいずれも線型空間をなし,その商空間をde Rhamコホモロジーという.
\end{remarks}



\subsection{2次元の回転}

\begin{tcolorbox}[colframe=ForestGreen, colback=ForestGreen!10!white,breakable,colbacktitle=ForestGreen!40!white,coltitle=black,fonttitle=\bfseries\sffamily,
title=]
    2次元では回転微分作用素$r:\X\to C^\infty$が登場する.
    これを通じて定まるベクトル場$X\in\X$のスカラーポテンシャル$r(X)= H$をHamiltonianという?
\end{tcolorbox}

\begin{definition}[2次元多様体の微分作用素]
    $\Om\subset\R^2$を領域とする.
    \begin{enumerate}
        \item $C^1$-級ベクトル場に対して,2次元の回転を次のように定める:
        \[\xymatrix@R-2pc{
            r:\X^1(\Om)\ar[r]&C(\Om)\\
            \rotatebox[origin=c]{90}{$\in$}&\rotatebox[origin=c]{90}{$\in$}\\
            f^1\pp{}{x^1}+f^2\pp{}{x^2}\ar@{|->}[r]&\pp{f^2}{x^1}-\pp{f^1}{x^2}.
        }\]
        \item $C^\infty(\Om)$-上の代数の同型$G_1:\X(\Om)\iso\Om^1(\Om),G_2:C^\infty(\Om)\iso\Om^2(\Om)$を次で定める:
        \[\xymatrix@R-2pc{
            G_1:\X(\Om)\ar[r]_-{\sim}&\Om^1(\Om)&G_2:C^\infty(\Om)\ar[r]_-{\sim}&\Om^2(\Om)\\
            \rotatebox[origin=c]{90}{$\in$}&\rotatebox[origin=c]{90}{$\in$}&\rotatebox[origin=c]{90}{$\in$}&\rotatebox[origin=c]{90}{$\in$}\\
            f^1\pp{}{x^1}+f^2\pp{}{x^2}\ar@{|->}[r]&f_1dx^1+f_2dx^2&f\ar@{|->}[r]&fdx^1\wedge dx^2
        }\]
    \end{enumerate}
\end{definition}

\begin{theorem}[Greenの定理]
    $\Om\subset\R^2$を区分的に$C^1$-級な境界を持つコンパクト領域とする.このとき,
    \[\int_{\partial\Om}X\cdot ds=\int_\Om r(X)\dvol.\]
\end{theorem}

\begin{theorem}[2次元のde Rham cohomology]
    領域$\Om\subset\R^2$について,次の図式は可換である:
    \[\xymatrix{
        C^\infty(\Om)\ar[r]^-{\grad}\ar@{=}[d]&\X(\Om)\ar[d]_-{G_1}\ar[r]^-r&C^\infty(\Om)\ar[d]_-{G_2}\\
        \Om^0(\Om)\ar[r]_-{d^0}&\Om^1(\Om)\ar[r]_-{d^1}&\Om^2(\Om)
    }\]
    特に,勾配ベクトル場は回転が消える:$r\circ\grad=0$.
\end{theorem}

\subsection{3次元の回転}

\begin{history}
    Heavisideは$\curl$の記法を用いた.
    ベクトル場が流速の場であるとき,回転を\textbf{渦度}といい,$\int_{\partial S}X\cdot ds$は循環という.
\end{history}

\begin{definition}[3次元多様体の微分作用素]
    $\Om\osub\R^n$を領域とする.
    \begin{enumerate}
        \item $C^1$-級ベクトル場に対する発散を$\Div(X):=\nabla\cdot X$と定める.
        \item $C^1$-級ベクトル場に対する回転を次のように定める:
        \[\xymatrix@R-2pc{
            \nabla\times:\X(\Om)\ar[r]&\X(\Om)\\
            \rotatebox[origin=c]{90}{$\in$}&\rotatebox[origin=c]{90}{$\in$}\\
            {\begin{pmatrix}f^1\\f^2\\f^3\end{pmatrix}}\ar@{|->}[r]&{\begin{pmatrix}\pp{f^3}{x^2}-\pp{f^2}{x^3}\\\pp{f^1}{x^3}-\pp{f^3}{x^1}\\\pp{f^2}{x^1}-\pp{f^1}{x^2}\end{pmatrix}}
        }\]
        特に$\rot(X)=\curl(X):=\nabla\times(X)$とも表す.
        \item $C^\infty(\Om)$-上の代数の同型$G_1:\X\iso\Om^1,G_2:\X\to\Om^2,G_3:C^\infty\to\Om^3$を次のように定める:
        \[\begin{cases}
            G_1\paren{f^1\pp{}{x^1}+f^2\pp{}{x^2}+f^3\pp{}{x^3}}=f_1dx^1+f_2dx^2+f_3dx^3\\
            G_2\paren{f^1\pp{}{x^1}+f^2\pp{}{x^2}+f^3\pp{}{x^3}}=f_1dx^2\wedge dx^3+f_2dx^3\wedge dx^1+f_3dx^1\wedge dx^2\\
            G_3(f)=fdx^1\wedge dx^2\wedge dx^3.
        \end{cases}\]
    \end{enumerate}
\end{definition}
\begin{remarks}
    3次元ベクトル場$X\in\X(\Om)\;(\Om\osub\R^3)$について,$X=\grad f$を満たす関数$f\in C^\infty(\Om)$をスカラーポテンシャル,$X=\rot(Y)$を満たす$Y\in\X(\Om)$をベクトルポテンシャルという.
\end{remarks}

\begin{theorem}[(Kelvin-)Stokes]
    $\psi:R\to\R^3$を曲面,$S:=\Im\psi$とする.このとき,
    \[\int_{\partial S}X\cdot ds=\int_S(\rot X)\cdot dS.\]
\end{theorem}

\begin{theorem}[3次元のde Rham cohomology]\label{thm-3dimension-deRham}
    $\Om\osub\R^3$を領域とする.次の図式は可換になる:
    \[\xymatrix{
        C^\infty\ar[r]^-\grad\ar@{=}[d]&\X\ar[r]^-\rot\ar[d]_-{G_1}&\X\ar[r]^-\div\ar[d]_-{G_2}&C^\infty\ar[d]_-{G_3}\\
        \Om^0\ar[r]^-{d^0}&\Om^1\ar[r]^-{d^1}&\Om^2\ar[r]^-{d^2}&\Om^3
    }\]
    特に,
    \begin{enumerate}
        \item 勾配ベクトル場の回転は零:$\rot\circ\grad=0$.
        \item 回転ベクトル場の発散は零:$\div\circ\rot=0$.
    \end{enumerate}
    また,$\Om$が単連結ならば,逆$\grad\circ\rot=0,\rot\circ\div=0$も成り立つ.
\end{theorem}

\subsection{ベクトルポテンシャル}

\begin{tcolorbox}[colframe=ForestGreen, colback=ForestGreen!10!white,breakable,colbacktitle=ForestGreen!40!white,coltitle=black,fonttitle=\bfseries\sffamily,
title=]
    スカラーポテンシャルは定数分の不定性しかなかったが,ベクトルポテンシャルは渦なしのベクトル場の不定性がある.
    この議論を通じて,自然に2階の微分$\Lap$に目が行く.
\end{tcolorbox}

\begin{observation}
    $\R^3$のde Rham構造\ref{thm-3dimension-deRham}より,ベクトルポテンシャルが存在するためには$\div X=0$が必要.これで十分かは$V$の連結性による.
\end{observation}

\begin{theorem}
    領域$V$上の$C^\infty$-級ベクトル場$X$について,次は同値:
    \begin{enumerate}
        \item $\div X=0$ならばベクトルポテンシャルが存在する.
        \item $\R^3\setminus V$はコンパクトな連結成分を持たない.
    \end{enumerate}
\end{theorem}

\begin{proposition}[ベクトルポテンシャルの不定性]
    $Y$を$X$のベクトルポテンシャルとする.第三のベクトル場$Z$が$\rot Z=0$ならば,$Y+Z$も$X$のベクトルポテンシャル.
\end{proposition}

\begin{theorem}[一般領域上のベクトルポテンシャルの存在条件 de Rham (1931)]
    一般の領域$V\subset\R^3$上の$C^\infty$-級ベクトル場$X\in\X(V)$について,次は同値:
    \begin{enumerate}
        \item ベクトルポテンシャルを持つ:$\exists_{Y\in\X(V)}\;X=\rot(Y)$.
        \item $V$内の任意の向きのある閉曲面$S$について,
        \[\int_SX\cdot dS=0.\]
    \end{enumerate}
\end{theorem}

\begin{example}
    Coulomb力ないし重力場ポテンシャルの勾配$f(r)=\frac{r}{\abs{r}^3}$は大域的なベクトルポテンシャルを持たない.
    これは,$f$が原点で定義されておらず,これを含む球面を取ると循環が消えない.
\end{example}

\subsection{ベクトル場のHelmholtz分解}

\begin{tcolorbox}[colframe=ForestGreen, colback=ForestGreen!10!white,breakable,colbacktitle=ForestGreen!40!white,coltitle=black,fonttitle=\bfseries\sffamily,
title=]
    任意のベクトル場は回転なしの場と発散なしの場の和に分解できる.
    一般の多様体におけるHodge分解の$n=3$での消息である.
\end{tcolorbox}

\begin{theorem}[Helmholtz]
    任意の領域$V\subset\R^3$上の任意の$C^\infty$-級ベクトル場$X$について,
    \begin{enumerate}
        \item $\exists_{Y\in \X(V)}\;\Lap Y=X$.
        \item $X=\grad(\div Y)-\rot(\rot Y)$.
    \end{enumerate}
    特に,任意のベクトル場は渦なしのベクトル場と湧き出しなしのベクトル場との重ね合わせで表せる.
\end{theorem}
\begin{remarks}
    $-\div Y,-\rot Y$がそれぞれ$X$のポテンシャルとベクトルポテンシャルと見れば,これは常に存在することになる.
    特に,Poisson方程式の解を解くことで得られることになる.
\end{remarks}

\section{積分変換}

\begin{tcolorbox}[colframe=ForestGreen, colback=ForestGreen!10!white,breakable,colbacktitle=ForestGreen!40!white,coltitle=black,fonttitle=\bfseries\sffamily,
title=]
    Fourier, Radon変換は$x\in\R^n$についての変換で,平面波に分解する.
    Laplace変換は時間軸$t\in\R_+$についての変換である.
\end{tcolorbox}

\subsection{Fourier変換の理論}

\begin{definition}
    $u\in L^1(\R^n;\C)$に対して,
    \begin{enumerate}
        \item Fourier変換を次のように定める:
        \[\F[u](y)=\wh{u}(y):=\frac{1}{(2\pi)^{n/2}}\int_{\R^n}e^{-ix\cdot y}u(x)dx.\]
        \item \textbf{Fourier逆変換}を次のように定める:
        \[\wc{u}(y):=\frac{1}{(2\pi)^{n/2}}\int_{\R^n}e^{ix\cdot y}u(x)dx.\]
    \end{enumerate}
\end{definition}

\begin{theorem}[Plancherel]
    Fourier変換の$L^1(\R^n;\C)\cap L^2(\R^n;\C)$への制限の像は$L^2(\R^n;\C)$に入っており,等長同型を定める.
\end{theorem}

\begin{proposition}\label{prop-inversion-formula-for-Fourier-transform}
    $u,v\in L^2(\R^n;\C)$について,
    \begin{enumerate}
        \item Fourier変換は内積を保つ:$(u|v)=(\wh{u}|\wh{v})$.
        \item 任意の$\al\in\N^n$について,$D^\al u\in L^2(\R^n;\C)$ならば,$\wh{(D^\al u)}=(iy)^\al\wh{u}$.
        \item $u,v\in L^1(\R^n;\C)$でもあるならば,$\wh{(u*v)}=(2\pi)^{n/2}\wh{u}\wh{v}$.
        \item $u,v\in L^1(\R^n;\C)$でもあるならば,$u=\wc{\wh{u}}$.
    \end{enumerate}
\end{proposition}

\subsection{Radon変換}

\begin{notation}\mbox{}
    \begin{enumerate}
        \item $S^{n-1}:=\partial B(0,1)\subset\R^n$とし,その点を$\om=(\om_1,\cdots,\om_n)$で表す.
        \item 超平面を$\Pi(s,\om):=\Brace{y\in\R^n\mid y\cdot\om=s}\;(s\in\R)$で表す.
    \end{enumerate}
\end{notation}

\begin{definition}[Radon transform]
    $u\in C^\infty_c(\R^n)$の\textbf{Radon変換}を次のように定める:
    \[\cR[u](s,\om)=\wt{u}(s,\om):=\int_{\Pi(s,\om)}udS,\qquad(s,\om)\in\R\times S^{n-1}.\]
\end{definition}

\begin{theorem}[Radon変換とFourier変換の関係]
    $u\in C^\infty_c(\R^n)$について,
    \[\o{u}(r,\om):=\int_\R \wt{u}(s,\om)e^{-irs}ds=(2\pi)^{n/2}\wh{u}(r\om),\quad(r\in\R,\om\in S^{n-1}).\]
\end{theorem}

\begin{theorem}[Radon変換の反転公式]\mbox{}
    \begin{enumerate}
        \item \[u(x)=\frac{1}{2(2\pi)^n}\int_\R\int_{S^{n-1}}\o{u}(r,\om)r^{n-1}e^{ir\om\cdot x}dSdr.\]
        \item $n=2k+1\;(k\in\N)$次元のとき,さらに
        \[u(x)=\int_{S^{n-1}}r(x\cdot \om,\om)dS,\quad r(s,\om):=\frac{(-1)^k}{2(2\pi)^{2k}}\pp{^{2k}}{s^{2k}}\wt{u}(s,\om).\]
    \end{enumerate}
\end{theorem}

\subsection{Laplace変換}

\begin{definition}[Laplace transform]
    $u\in L^1(\R^+)$について,\textbf{Laplace変換}を次のように定める:
    \[\L[u](s)=u^\#(s):=\int^\infty_0e^{-st}u(t)dt,\quad (s\in\R_+).\]
\end{definition}

\begin{theorem}\label{thm-uniqueness-of-Laplace-transform}
    $f,g\in C(\R_+)$は増大条件
    \[\exists_{b>0}\;\sup_{t>0}\paren{\frac{\abs{f(t)}}{e^{bt}}+\frac{\abs{g(t)}}{e^{bt}}}<\infty\]
    を満たすとする.このとき,ある$c>0$が存在して$\L[f]=\L[g]\;\on{[c,\infty)}$ならば,$f=g\;\In\R_+$である.
\end{theorem}

\section{不等式}

\begin{proposition}[Cauchyの不等式]\mbox{}
    \begin{enumerate}
        \item 任意の$a,b\in\R$について,
        \[ab\le\frac{a^2+b^2}{2}.\]
        \item 任意の$a,b>0$と$\ep>0$について,
        \[ab\le\ep a^2+\frac{b^2}{4\ep}.\]
    \end{enumerate}
\end{proposition}
\begin{Proof}\mbox{}
    \begin{enumerate}
        \item $(a-b)^2=a^2+b^2-2ab\ge0$による.
        \item $ab=(\sqrt{2\ep}a)\paren{\frac{b}{\sqrt{2\ep}}}$について(1)を用いる.
    \end{enumerate}
\end{Proof}

\begin{proposition}[Youngの不等式]
    共役指数$p,q\in(1,\infty)$に関して,
    \begin{enumerate}
        \item 任意の$a,b>0$について,
        \[ab\le\frac{a^p}{p}+\frac{b^q}{q}.\]
        \item $C(\ep):=(\ep p)^{-q/p}q^{-1}$とすると,任意の$a,b>0$と$\ep>0$について,
        \[ab\le\ep a^p+C(\ep)b^q.\]
    \end{enumerate}
\end{proposition}
\begin{Proof}\mbox{}
    \begin{enumerate}
        \item $x\mapsto e^x$が凸関数であることを用いると,
        \[ab=e^{\frac{1}{p}\log a^p+\frac{1}{q}\log b^q}\le\frac{1}{p}e^{\log a^p}+\frac{1}{q}e^{\log b^q}=\frac{a^p}{p}+\frac{b^q}{q}.\]
        \item $ab=((\ep p)^{1/p}a)\paren{\frac{b}{(\ep p)^{1/p}}}$に対して(1)を用いる.
    \end{enumerate}
\end{Proof}

\begin{corollary}[Hölderの不等式]
    共役指数$p,q\in[1,\infty]$に関して,任意の$u\in L^p(U),v\in L^q(U)$について,
    \[\int_U\abs{uv}dx\le\norm{u}_{L^p(U)}\norm{v}_{L^q(U)}.\]
\end{corollary}
\begin{Proof}
    $\norm{u}_{L^p(U)}=\norm{v}_{L^p(U)}=1$に規格化して考える.
    Youngの不等式より,
    \[\int_U\abs{uv}dx\le\int_U\Abs{\frac{u^p}{p}+\frac{v^q}{q}}dx\le \frac{\norm{u}_{L^p(U)}^p}{p}+\frac{\norm{v}_{L^q(U)}^q}{q}=1=\norm{u}_{L^p(U)}\norm{v}_{L^q(U)}.\]
\end{Proof}

\section{$n$次元球の体積と極座標}

\begin{tcolorbox}[colframe=ForestGreen, colback=ForestGreen!10!white,breakable,colbacktitle=ForestGreen!40!white,coltitle=black,fonttitle=\bfseries\sffamily,
title=]
    $n$次元球の体積を
    $\om_n:=\abs{B^n(0,1)}$,$n$次元球面の面積を$n\om_n=\abs{S^n}$と表す.
    \cite{Evans}だと$\al(n)$である.
    $A_n(r)=\dd{V_{n+1(r)}}{r}$なる関係がある.
    $A(r),V(r)$の記法は,$dA,\dvol$に合致する.
\end{tcolorbox}

\subsection{$n$次元球の体積と表面積}

\begin{theorem}[球の体積]
    $n$次元球$B(0,r)\subset\R^n$について,
    \begin{enumerate}
        \item 体積は
        \[V_n(r)=\frac{\Gamma(1/2)^n}{\Gamma((n+2)/2)}r^n=\om_nr^n.\]
        \item 特に,$r=1$のとき,体積とその表面積は
        \[\om_n=\frac{\Gamma(1/2)^n}{\Gamma((n+2)/2)}=\frac{\pi^{n/2}}{(n/2)\Gamma(n/2)},\quad\sigma_n=n\om_n.\]
    \end{enumerate}
    特に,関係
    \[\frac{2}{n\om_n}=\frac{\Gamma\paren{\frac{n}{2}}}{\pi^{n/2}}\]
    はPoisson核の議論\ref{remark-Poisson-kernel}で重宝する.
\end{theorem}
\begin{remark}[$\sigma_1$の解釈]
    $n=1$次元球$[-1,1]$の体積は
    \[\om_1=\frac{\Gamma(1/2)}{\Gamma(3/2)}=2.\]
    であるが,その表面積は
    \[\sigma_1=1\cdot\om_1=2.\]
    とカウントされる.2点からなる集合であるためだろうか.
\end{remark}

\begin{lemma}[Gamma関数の性質]\mbox{}
    \begin{enumerate}
        \item 任意の$z\in\C\setminus\Z_{\le0}$について,$\Gamma(z+1)=z\Gamma(z)$.
        \item $\Gamma(1)=\Gamma(2)=1$.$\Gamma\paren{\frac{1}{2}}=\sqrt{\pi}$.
        \item $\Gamma(k)=(k-1)!$.$\Gamma\Paren{k-\frac{1}{2}}=\paren{k-\frac{3}{2}}\cdots\frac{1}{2}\sqrt{\pi}$.
    \end{enumerate}
\end{lemma}


\begin{corollary}\mbox{}\label{cor-volume-of-ball-in-odd-and-even-dimension}
    \begin{enumerate}
        \item 偶数次元$n=2k\;(k\in\N^+)$のとき,$\om_n=\frac{\pi^k}{k!}$.
        \item 奇数次元$n=2k-1\;(k\in\N^+)$のとき,
        \[\om_n=\frac{\pi^{k-1}}{2\paren{k-\frac{1}{2}}!}=\frac{\pi^{k-1}}{2\paren{k-\frac{1}{2}}\paren{k-\frac{3}{2}}\cdots\paren{\frac{1}{2}}}=\frac{\pi^{k-1}}{2}\frac{2^n}{(2k-1)(2k-3)\cdots 1}=\frac{\pi^{k-1}2^{n-1}}{n!!}\]
        \item $n=2k\;(k\in\N^+)$が偶数次元のとき,
        \[\frac{\om_{2k}}{\om_{2k+1}}=\frac{\paren{k+\frac{1}{2}}!!}{k!}=\frac{1}{2}\frac{(2k+1)!!}{2k!!}=\frac{1}{2}\frac{(n+1)(n-1)\cdots5\cdot3\cdot1}{n(n-2)\cdots4\cdot2\cdot1}.\]
    \end{enumerate}
\end{corollary}

\subsection{極座標積分}

\begin{tcolorbox}[colframe=ForestGreen, colback=ForestGreen!10!white,breakable,colbacktitle=ForestGreen!40!white,coltitle=black,fonttitle=\bfseries\sffamily,
title=]
    $\R^n$上の積分を,任意の点$x_0\in\R^n$を中心とした球面$\partial B(x_0,r)$上の積分に分割出来るが,
    このときJacobianの$r$などは出現しない.これは面積分の妙技による.間違いやすいので注意.
\end{tcolorbox}

\begin{theorem}[Co-area formula]
    $u:\R^n\to\R$をLipschitz連続関数とし,殆ど至る所の$r\in\R$について,等位集合$\Brace{x\in\R^n\mid u(x)=r}$は滑らかな超曲面をなすとする.
    $f:\R^n\to\R$を連続な可積分関数とする.このとき,
    \[\int_{\R^n}f\abs{Du}dx=\int^\infty_{-\infty}\paren{\int_{\Brace{u=r}}fdS}dr.\]
\end{theorem}

\begin{corollary}[球面積分の性質]\mbox{}\label{cor-property-of-ball-surface-integral}
    \begin{enumerate}
        \item $f:\R^n\to\R$は連続で可積分であるとする.このとき,
        \[\forall_{x_0\in\R^n}\quad\int_{\R^n}fdx=\int^\infty_0\paren{\int_{\partial B(x_0,r)}fdS}dr.\]
        \item 特に,次が成り立つ:
        \[\forall_{r>0}\quad\dd{}{r}\paren{\int_{B(x_0,r)}fdx}=\int_{\partial B(x_0,r)}fdS.\]
    \end{enumerate}
\end{corollary}
\begin{Proof}\mbox{}
    \begin{enumerate}
        \item 余面積公式を$u(x):=\abs{x}$と取った場合の結果である.
        \item (1)より,
        \[\int_{B(x_0,r)}fdx=\int_{l=0}^{l=r}\int_{\partial B(x_0,l)}fdSdl\]
        と表せる.これを微分すれば従う.
    \end{enumerate}
\end{Proof}

\begin{corollary}[球面上の積分と球体上の積分との関係]\label{thm-sphere-and-ball}
    $B(0,R)\subset\R^n, f\in L^1(B(0,R))$について,
    \[\int_{B(0,R)}f(x)dx=\int^R_0\int_{\partial B(0,r)}f(x)d\sigma(x)dr=\int^R_0r^{n-1}dr\int_{\partial B(0,1)}f(rx)d\sigma(x).\]
\end{corollary}

\chapter{Poisson方程式とポテンシャル}

\begin{quotation}
    任意の開集合$D\osub\R^n$について,
    \[-\Laplace u=f\quad\on\quad D\osub\R^n\]
    をPoisson方程式というが,$f=0$のときをLaplace方程式と言い,これを満たす$u$を\textbf{調和関数}という.
    
    Poisson方程式は,
    $u:D\to\R$をスカラー場(電荷・濃度・ポテンシャル)として,
    熱方程式から考えるとこれは$t$を含まない方程式で,
    いわば熱方程式の$t\to\infty$の極限に当たる,\textbf{定常状態の分布$u$}を記述する方程式とみれる.
    このときのスカラー場$u:\Om\to\R$は極めて滑らかで,実解析性を持つ.これは一般の,滑らかな係数を持つ楕円型微分方程式の解の正則性定理の系ともみれる.
\end{quotation}

\begin{notation}\mbox{}
    \begin{enumerate}
        \item 面積測度を$dS$で表すとする.
        \item Borel可測関数$u:\Om\to\R$について,球面$\partial B(x,r)$上の平均値を
        \[\M_r[u(x)]=\dint_{\partial B(x,r)}udS:=\frac{1}{\abs{\partial B(x,r)}}\int_{\partial B(x,r)}udS=\frac{1}{n\om_nr^{n-1}}\int_{\partial B(x,r)}fdS.\]
        \item 局所可積分関数$u\in L^1_\loc(\Om)$について,球$B(x,r)$上の平均値を
        \[\A_r[u(x)]=\dint_{B(x,r)}u\dvol:=\frac{1}{\abs{B(x,r)}}\int_{B(x,r)}u\dvol=\frac{1}{\om_nr^n}\int_{B(x,r)}f\dvol.\]
        \item $\om_n:=\abs{B^n(0,1)}=\frac{2\pi^{n/2}}{n\Gamma(n/2)}$をGilbarg-Trudinger\cite{Gilbarg}に倣う.Evans\cite{Evans}では$\al(n)$である.
        \item 複素関数に対する微分作用素を
        \[\partial_z:=\frac{1}{2}(\partial_x-i\partial_y),\quad\partial_{\o{z}}\frac{1}{2}(\partial_x+i\partial_y).\]
        \item 領域$D$上の境界への連続延長を持つ調和関数の全体を,
        \[\H(D):=\Brace{u\in C^0([D])\mid u\text{は}D\text{上調和}}.\]
    \end{enumerate}
\end{notation}

\section{調和関数と正則関数}

\begin{tcolorbox}[colframe=ForestGreen, colback=ForestGreen!10!white,breakable,colbacktitle=ForestGreen!40!white,coltitle=black,fonttitle=\bfseries\sffamily,
title=]
    $\partial_{\o{z}}f=0$が正則関数の$C^1(\C)$上の特徴付けであった.
    より条件をゆるくした$\Lap:=4\partial_z\partial_{\o{z}}$について$\Lap f=0$を満たす$C^1(\C)$の関数クラスを\textbf{調和関数}という.
\end{tcolorbox}

\subsection{共役調和関数}

\begin{definition}[Hodge star operator]
    $D\osub\C$を領域とする.
    次の1-形式の間の対応$*:\Om^1(D)\to\Om^{2-1}(D)$を\textbf{Hodge双対}という:
    \[*(u_xdx+u_ydy)=-u_ydx+u_xdy.\]
    特に,$*dx=dy,*dy=-dx$で定まる線型同型であり,$**=-I$を満たす周期4である.
\end{definition}
\begin{lemma}[$\C$上のHodge双対の意味]\label{lemma-Hodge-star}
    $\gamma$を正則な曲線,
    $\gamma$の外側単位法線方向ベクトルを$n$とする.
    \begin{enumerate}
        \item この方向への方向微分を$\pp{}{n}$で表すと,
        \[\int_\gamma *du=\int_{\gamma}\pp{u}{n}\abs{dz}.\]
        \item $d(*du)=\Lap udx\land dy$.一般にはこちらをHodge双対の定義とする.
        \item $D$を$C^1$-級境界を持つ有界領域とすると,調和関数$u:D\to\R$に対して,
        \[\int_{\partial D}\partial_Nu\abs{dz}=\int_{\partial D}*du=\int_Dd(*du)=\int_D\Lap udx\wedge dy=0.\]
    \end{enumerate}
\end{lemma}
\begin{remarks}
    $*$-作用素が形式上時計回りに90度回転する変換に対応していることの定式化である.
\end{remarks}

\begin{theorem}[単連結領域上の調和関数には共役が存在する]
    単連結領域$D$上の実調和関数$u:D\to\R$について,
    \begin{enumerate}
        \item 次の正則関数$f\in\O(D)$の実部である:
        \[f(x)=2\int^z_{z_0}\partial_zu(\zeta)d\zeta+u(z_0),\quad z_0\in D.\]
        \item 共役調和関数$v:D\to\R$が次のように表せる:
        \[v(z)=\int^z_{z_0}*du+C,\quad z_0\in D.\]
    \end{enumerate}
\end{theorem}
\begin{Proof}\mbox{}
    \begin{enumerate}
        \item $u$は実調和だから,$\Lap u=4\partial_{\o{z}}\partial_{z}u=0$より,$\partial_zu$は正則.
        1-形式の空間において,次のように計算できる:
        \begin{align*}
            2\partial_zudz&=(u_x-iu_y)(dx+idy)\\
            &=(u_xdx+u_ydy)+i(-u_ydx+u_xdy)
            =du+i*du.
        \end{align*}
        よって,積分の実部は
        \[\Re\int^z_{z_0}2\partial_zudz=\int^z_{z_0}du=u(z)-u(z_0).\]
        \item $f$の虚部は,
        \[\Im\int^z_{z_0}2\partial_zudz=\int^z_{z_0}*du.\]
    \end{enumerate}
\end{Proof}

\begin{example}\mbox{}
    \begin{enumerate}
        \item $u=x^2-y^2$の共役調和関数は,
        \[v(z)=\int^z_0*du=\int^1_0(2tyz+2txy)dt=2xy.\]
        \item $u=\log\abs{z}$の共役調和関数は,
        \[v(z)=\int^z_{z_0}*d\log\abs{z}=\int^z_{z_0}d\theta=\arg z+C.\]
    \end{enumerate}
\end{example}

\subsection{円板上の調和関数の積分表示}

\begin{tcolorbox}[colframe=ForestGreen, colback=ForestGreen!10!white,breakable,colbacktitle=ForestGreen!40!white,coltitle=black,fonttitle=\bfseries\sffamily,
title=]
    制限$|_{\partial D}:\H(D)\to C^0(\partial D)$の逆を,Poisson積分$P:C^0(\partial D)\to\H(D)$が与える.
    すなわち,$P$が解作用素である.
    これは円板上のGreen作用素\ref{thm-Green-function-on-disk}として得られるためである.
    複素関数論ではこの消息が先に到達される.
\end{tcolorbox}

\begin{definition}[Poisson integral]
    境界上の連続関数$u\in C^0(\partial\Delta(z_0,R))$に対して,\textbf{Poisson積分}を次で定める:
    \[P[f](z):=\frac{1}{2\pi}\int^{2\pi}_0\frac{R^2-\abs{z-z_0}^2}{\abs{\zeta-z}^2}f(\zeta)d\theta,\quad\zeta=z_0+Re^{i\theta}.\]
\end{definition}

\begin{theorem}
    $u\in\H(\Delta(0,R))$を調和関数とする.このとき,
    \[u(z)=P[u](z)\quad\on\Delta(0,R).\]
\end{theorem}

\begin{theorem}
    Laplace方程式のDirichlet問題
    \[\mathrm{(D)}\begin{cases}
        \Lap U=0\quad\on\Delta(0,R),\\
        U=u\quad\on\partial\Delta(0,R)
    \end{cases},\quad u\in C^0(\partial\Delta(0,R))\]
    の解は$U=P[u]$が与える.
\end{theorem}

\subsection{実解析の観点からの再論}

\begin{tcolorbox}[colframe=ForestGreen, colback=ForestGreen!10!white,breakable,colbacktitle=ForestGreen!40!white,coltitle=black,fonttitle=\bfseries\sffamily,
title=]
    さらなる詳しい結果が議論されている\cite{Rudin-RealandComplex}.
\end{tcolorbox}

前節では円板$\Delta(z_0,R)$上のPoisson核
\[P(z,Re^{i\theta})=\frac{R^2-\abs{z-z_0}^2}{\abs{Re^{i\theta}-(z-z_0)}^2}\]
に関する積分$P[u](z)=\frac{1}{2\pi}\int^{2\pi}_0P(z,Re^{i\theta})f(Re^{i\theta})d\theta$
を議論した.仮に$z_0=0,R=1$とすると
\[P(z,e^{i\theta})=\frac{1-\abs{z}^2}{\abs{e^{i\theta}-z^2}^2}\]
である.

\begin{proposition}\label{prop-Poisson-kernel-on-disk-of-C}
    $z_0=0,R=1$のとき,Poisson核は次のようにも表せる:
    \[P(z,e^{i\theta})=P_r(\arg(z)-\theta)=\Re\frac{e^{i\theta}+z}{e^{i\theta}-z}=\frac{1-\abs{z}^2}{1-2\abs{z}\cos(\arg(z)-\theta)+r^2}.\]
    特に,$P(z,e^{i\theta})$自体も$e^{i\theta},z$の調和関数である.
\end{proposition}

\subsection{Harnackの不等式}

\begin{tcolorbox}[colframe=ForestGreen, colback=ForestGreen!10!white,breakable,colbacktitle=ForestGreen!40!white,coltitle=black,fonttitle=\bfseries\sffamily,
title=]
    球の上ではPoisson積分表示を持つことによる調和関数の性質である.
    ここからもHarnackの定理($L^\infty$-安定性)の結果を得る.
\end{tcolorbox}

\begin{proposition}[Harnack inequality]
    $u\in\H(\Delta(z_0,R))$は非負であるとする.このとき,
    \[\forall_{r<R}\;\forall_{z\in[\Delta(z_0,r)]}\quad\frac{R-r}{R+r}u(z_0)\le u(z)\le\frac{R+r}{R-r}u(z_0).\]
\end{proposition}
\begin{Proof}
    Poisson積分による表示
\end{Proof}

\section{複素領域上のD型境界値問題の解の構成}

\subsection{Perron族による調和関数の構成}

\begin{definition}
    $D\subset\C$を領域とする.
    \begin{enumerate}
        \item 円板$B:=\Delta(z_0,R)\ssub D$に対して,次で定める写像$P_B:\SH(D)\to\SH(D)$を\textbf{$B$上のPoisson修正}という:
        \[\wt{u}:=P_B[u](z)=\begin{cases}
            u(z),&z\in D\setminus B,\\
            P[u|_{\partial B}](z),&z\in B.
        \end{cases}\]
        \item 族$\F\subset\SH(D)$が\textbf{Perron族}であるとは,次を満たすことをいう:
        \begin{enumerate}
            \item $\forall_{u,v\in\F}\;u\lor v\in\F$.
            \item $\forall_{u\in\F}\;\forall_{\Delta(z_0,R)\ssub D}\;P_B[u]\in\F$.
        \end{enumerate}
        \item $u(z):=\sup_{v\in\F}v(z)$で定まる関数$u:D\to\ocinterval{-\infty,\infty}$を$\sup\F$と表す.
    \end{enumerate}
\end{definition}

\begin{lemma}\mbox{}
    \begin{enumerate}
        \item $P_B[u]\in\SH(D)$.
        \item $\SH(D)$はPerron族である.
        \item $D\osub\C$を有界とし,$f\in C^0(\partial D)$に対して
        \[\SH_f(D):=\Brace{u\in\SH(D)\cap C^0([D])\mid u|_{\partial D}\le f}\]
        は再びPerron族である.
    \end{enumerate}
\end{lemma}

\begin{theorem}[一般化されたHarnackの定理]
    $\F\subset\SH(D)$をPerron族とする.$\sup\F$は$D$上で調和または$u=\infty$である.
\end{theorem}
\begin{remarks}
    劣調和関数の単調増加列はPerron族である.
\end{remarks}

\subsection{バリア}

\begin{definition}[barrier]
    $D\osub\C$を領域$\zeta\in\partial D$を境界点とする.
    $\om\in C([D])$が次を満たすとき,\textbf{バリア}という:
    \begin{enumerate}
        \item 優調和関数である:$-\om\in\SH(D)$.
        \item $\om(\zeta)=0$かつ$\forall_{z\in D\setminus\{\zeta\}}\;\om(z)>0$.
    \end{enumerate}
\end{definition}

\begin{example}\mbox{}
    \begin{enumerate}
        \item $\Delta(0,1)$は全ての境界点$\zeta\in\partial\Delta(0,1)$でバリア$\om(z)=\Re\paren{1-\frac{z}{\zeta}}$を持つ.
        \item $D:=\Delta(0,1)\setminus\{0\}$は$0$でのバリアを持たない.
    \end{enumerate}
\end{example}

\begin{proposition}
    $D\osub\C$を領域とする.
    \begin{enumerate}
        \item $p\in\partial D$が$\exists_{q\in\C}\;[D]\cap[p,q]=\{p\}$を満たせば,$p$でのバリアを持つ.
        \item $D$がJordan領域であれば,バリアを持つ.
    \end{enumerate}
\end{proposition}

\begin{theorem}
    $D\osub\C$を有界領域,$\zeta_0\in\partial D$を境界点とする.
    \begin{enumerate}
        \item $\zeta_0$がバリア$\om$を持てば,$U(z):=\sup\SH_f(D)$は$\lim_{D\ni z\to\zeta_0}U(z)=f(\zeta_0)$を満たす.
        \item 次の2条件は同値である:
        \begin{enumerate}
            \item $D$の全ての境界点はバリアを持つ.
            \item $U(z):=\sup\SH_f(D)$は斉次Dirichlet問題の解を与える:$U\in\H(D)$かつ$U|_{\partial D}=f$.
        \end{enumerate}
    \end{enumerate}
\end{theorem}

\subsection{Green関数の定義と特徴付け}

\begin{tcolorbox}[colframe=ForestGreen, colback=ForestGreen!10!white,breakable,colbacktitle=ForestGreen!40!white,coltitle=black,fonttitle=\bfseries\sffamily,
title=]
    一般の領域$D\osub\C$のGreen関数は,対角集合$\Delta\subset[D]\times[D]$上で特異性(対数の分岐点になる)を持つ.
\end{tcolorbox}

\begin{definition}
    $D\osub\C$を任意の境界点がバリアを持つ有界領域とする.斉次Dirichlet問題
    \[\begin{cases}
        \Lap u=0\quad\on D\\
        u(\zeta)=\log\abs{\zeta-z_0}\quad\on\partial D
    \end{cases}\]
    の解を$G_{z_0}(z)$とする.
    これについて,次のように定義出来る関数$g:[D]\setminus\{z_0\}\to\R$を\textbf{$D$のGreen関数}という:
    \[g(z,z_0):=G_{z_0}(z)-\log\abs{z-z_0}.\]
\end{definition}

\begin{proposition}
    連続関数$h:[D]\setminus\{z_0\}\to\R$が次の2条件を満たすならば,$h(z)=g(z,z_0)$である:
    \begin{enumerate}
        \item $h|_{\partial D}=0$.
        \item $h(z)+\log\abs{z-z_0}$は$D$上の調和関数に延長出来る.
    \end{enumerate}
\end{proposition}

\begin{example}
    $D=\Delta(0,1)$のときのGreen関数は,$z_0=0$のとき$g(z,0)=-\log\abs{z}$で,一般には
    \[g(z,z_0)=-\log\Abs{\frac{z-z_0}{1-\o{z_0}z}}=\log\abs{1-\o{z_0}z}-\log\abs{z-z_0}.\]
    であるが,一般の単連結Jordan領域$D\osub\C$について,双正則写像$f:D\to\Delta(0,1),f(z_0)=0$を取れば,$D$のGreen関数は
    \[g(z,z_0)=-\log\abs{f(z)}\]
    と表せ,理論的に怖いものではない.
    実は,Green関数とその共役調和関数を求めることと,Riemann写像$f$を求めることとは同値であることが判る.
\end{example}
\begin{remark}[基本解との関連について]
    $n=2$次元の場合は,
    一般の単連結Jordan領域$D\osub\C$について,双正則写像$f:D\to\Delta(0,1),f(z_0)=0$を取れば,$D$のGreen関数は
    \[g(z,z_0)=-\log\abs{f(z)}\]
    と表せるのであった.
    $\Phi$にはこれがそのまま出ているが,$f$の部分は畳み込み演算が担っているのであろうか?
\end{remark}

\subsection{Green関数の性質}

\begin{tcolorbox}[colframe=ForestGreen, colback=ForestGreen!10!white,breakable,colbacktitle=ForestGreen!40!white,coltitle=black,fonttitle=\bfseries\sffamily,
title=]
    Green関数とその共役調和関数を求めることと,Riemann写像$f$を求めることとは同値であることが判る.
    Riemannはこの議論をひっくり返すことで,Dirichlet問題の解としてRiemann写像$D\iso\Delta(0,1)$を構成しようとした.
\end{tcolorbox}

\begin{proposition}
    \[\forall_{[D]\times[D]\setminus\Delta}\quad g(z,w)=g(w,z).\]
\end{proposition}

\begin{theorem}
    領域$D,D'\osub\C$はGreen関数$g,g'$を持ち,
    $f:D\to D'$を双正則写像で,$[D]\to[D']$に拡張可能なものとする.
    このとき,
    \[\forall_{z,w\in D}\;g(z,w)=g'(f(z),f(w))\]
\end{theorem}

\begin{theorem}[Green関数を定める正則関数が定めるRiemann写像]
    $D$を$C^1$-境界を持つ単連結領域とし,$D$のGreen関数$g(z,z_0)$に対して$\Re G(z)=g(z,z_0)$を満たす$D\setminus\{z_0\}$上の多価正則関数$G(z)$を選ぶと,
    $f(z):=e^{-G(z)}$は$D$を$\Delta(0,1)$へ写す双正則写像で,$f(z_0)=0$を満たす.
\end{theorem}

\section{Laplacianの物理的考察}

\begin{tcolorbox}[colframe=ForestGreen, colback=ForestGreen!10!white,breakable,colbacktitle=ForestGreen!40!white,coltitle=black,fonttitle=\bfseries\sffamily,
title=]
    空間の一様性・等方性に対応する性質を持つため,Laplace方程式$\Lap u=0$の基本解は$r$のみに依存する調和関数だと当たりがつく.
\end{tcolorbox}

\subsection{Laplace作用素と他の微分作用素}

\begin{tcolorbox}[colframe=ForestGreen, colback=ForestGreen!10!white,breakable,colbacktitle=ForestGreen!40!white,coltitle=black,fonttitle=\bfseries\sffamily,
    title=]
    形式的には$\nabla\cdot\nabla$と見れるが,$\Lap=\div\circ\grad$で定義される.
    この内積の見方は実はHodge作用素と深く結びついている.
\end{tcolorbox}

\begin{proposition}\mbox{}\label{prop-Laplace-operator}
    \begin{enumerate}
        \item $\div\circ\grad=\Lap$.
        \item $\rot\circ\rot=\grad\circ\div-\Lap$.
    \end{enumerate}
\end{proposition}

\subsection{対称性}

\begin{tcolorbox}[colframe=ForestGreen, colback=ForestGreen!10!white,breakable,colbacktitle=ForestGreen!40!white,coltitle=black,fonttitle=\bfseries\sffamily,
title=]
    次の2つの対称性を満たす最も簡単な微分作用素として特徴づけることができる.
\end{tcolorbox}

\begin{lemma}
    微分作用素$\Lap$について,
    \begin{enumerate}
        \item 空間の一様性:座標系の平行移動に依らない.
        \item 空間の等方性:座標系の回転と反転に依らない.
        すなわち,任意の直交行列$O\in\mathrm{O}_n(\R)$と$u\in C^2(\R^n)$に対して,$\mu^O v(x):=v(Ox)$とすると,次の可換性が成り立つ:
        \[\Lap(\mu^Ou(x))=\mu^O((\Lap u(x)))=\Lap u(Ox).\]
    \end{enumerate}
\end{lemma}

\subsection{極表示}

\begin{theorem}[Laplacianの極座標表示]\mbox{}\label{thm-Laplacian-in-polar-coordinate}
    \begin{enumerate}
        \item Laplace作用素$\Lap:C^2(\R^2)\to C(\R^2)$について,
        \[\Lap=\pp{^2}{r^2}+\frac{1}{r}\pp{}{r}+\frac{1}{r^2}\pp{^2}{\theta^2}=\frac{1}{r}\pp{}{r}\paren{r\pp{}{r}}+\frac{1}{r^2}\pp{^2}{\theta^2}.\]
        \item Laplace作用素$\Lap:C^2(\R^3)\to C(\R^3)$について,
        \[\Lap=\frac{1}{r^2}\pp{}{r}\paren{r^2\pp{}{r}}+\frac{1}{r^2}\Lambda,\quad\Lambda=\frac{1}{\sin\theta}\pp{}{\theta}\paren{\sin\theta\pp{}{\theta}}+\frac{1}{\sin^2\theta}\pp{^2}{\varphi^2}.\]
        \item 以下同様にして,Laplace作用素が$r$のみの関数$u\in C^2(\R^n)$に作用するときは,次が成り立つ:
        \[\Lap u=\paren{\dd{^2}{r^2}+\frac{n-1}{r}\dd{}{r}}u.\]
    \end{enumerate}
\end{theorem}
\begin{remarks}
    $\Lambda_\theta:=\pp{^2}{\theta^2}$は単位円周$S^1$上のLaplace-Beltrami作用素$\Lap_0:\Om^0(S^1)\to\Om^0(S^1)$になっている.
    $\Lambda_{\theta,\varphi}$は単位球面$S^2$上のLaplace-Beltrami作用素になっている.
\end{remarks}

\subsection{定常状態の模型としての調和関数}

\begin{tcolorbox}[colframe=ForestGreen, colback=ForestGreen!10!white,breakable,colbacktitle=ForestGreen!40!white,coltitle=black,fonttitle=\bfseries\sffamily,
title=]
    調和関数は,定常状態に至っているベクトル場$F$のポテンシャル$u$として現れる.定常状態の定義を
    \[\int_{\partial D}F\cdot ndS=0\]
    とすると,Gaussの発散定理より,これは$D$上至る所で$\div F=0$を意味する.
    $F=-Du$を勘案すると,$\div(Du)=\Lap u=0$が従う.
    するとこの逆の消息が命題として得られる.
    この消息が劣調和函数の平均定理\ref{thm-mean-value-theorem-of-subharmonic-function}に直ちに繋がる.
\end{tcolorbox}

\begin{proposition}
    $\Om\osub\R^n$を有界領域,$u$をその近傍で$C^2$-級の調和関数とする.
    \[\int_{\partial\Om}\pp{u}{\b{n}}dS=0.\]
\end{proposition}
\begin{Proof}
    Greenの積分恒等式(0)\ref{cor-Green-identity}により,
    \[\int_{\partial\Om}\pp{u}{\nu}dS=\int_\Om\Lap udx=0.\]
\end{Proof}
\begin{remarks}
    実は$\partial_NvdA$というのがHodge双対$*dv$となっており,
    \ref{lemma-Hodge-star}の一般化である.
\end{remarks}


\begin{definition}[法線微分]
    関数$f\in C^1(\R^3)$の領域$D\subset\R^3$の単位法線ベクトル$n$に関する方向微分を
    \[\pp{f}{n}:=\nabla f\cdot n\]
    で表す.
\end{definition}

\subsection{調和関数の例}

\begin{tcolorbox}[colframe=ForestGreen, colback=ForestGreen!10!white,breakable,colbacktitle=ForestGreen!40!white,coltitle=black,fonttitle=\bfseries\sffamily,
    title=]
    球面対称である調和関数に狙いを定めて例を探す.
    物理的に単純な設定は多くの対称性を満たし,複雑な場合もその「重ね合わせ」と見れることが多いからである.
    こうしてLaplace方程式の基本解の当たりがつく.
\end{tcolorbox}

\begin{example}[静電場のポテンシャル]\mbox{}
    \begin{enumerate}
        \item 次の関数\[\frac{1}{\abs{x}}=\frac{1}{\sqrt{x_1^2+x_2^2+x_3^2}}\]
        は$\R^3\setminus\{0\}$上調和である.
        静電場のポテンシャルは,この関数の定数倍$\varphi_0\in C^\infty(\R^3\setminus\{0\})$で与えられる.
        \item 曲面$S\subset\R^3$上に綿密度$\rho_1$で電荷が分布した場合の静電ポテンシャルは
        \[\varphi_1(x)=\int_S\frac{\rho_1(\xi)}{\abs{x-\xi}}dS_\xi\]
        で与えられ,$\R^3\setminus S$上調和である.これを\textbf{1重層ポテンシャル}という.
        \item 導関数
        \[\psi_1(x)=-\pp{}{x_1}\frac{1}{\abs{x}}\]
        も$\R^3\setminus\{0\}$上調和である.これは
        \[\varphi_\ep=\frac{1}{2\ep}\paren{\frac{1}{\abs{x-\ep\b{e}_1}}-\frac{1}{\abs{x+\ep\b{e}_1}}}\]
        の$\ep\searrow0$の極限とみなせるため,\textbf{2重極ポテンシャル}という.
        \item 最後に,曲面$S$上に双極子が分布した場合のポテンシャル
        \[\psi_2(x)=\int_S\pp{}{n_\xi}\paren{\frac{1}{\abs{x-\xi}}}\rho_2(\xi)dS_\xi\]
        を\textbf{2重層ポテンシャル}という.
    \end{enumerate}
\end{example}

\begin{proposition}[球対称な調和関数の必要条件]\label{prop-necessary-condition-for-sphere-symmetric-harmonic-functions}
    $r(x)=\sqrt{x^2_1+\cdots+x_n^2}$の関数の形をした調和関数$u(x)=v(r)$は,$v'$が消えないとき,$v'(r)=\frac{a}{r^{n-1}}\;(a\in\R)$を満たす必要がある.
\end{proposition}
\begin{Proof}
    $r$は$x\ne0$上可微分であるから,
    \[\pp{r}{x_i}=\pp{}{x_i}(\sqrt{x_1^2+\cdots+x_n^2})=\frac{x_i}{\sqrt{x_1^2+\cdots+x_n^2}}=\frac{x_i}{r}\quad(x\ne0).\]
    よって,
    \[u_{x_i}=v'(r)\frac{x_i}{r},\quad u_{x_ix_i}=v''(r)\paren{\frac{x_i}{r}}^2+v'(r)\paren{\frac{1}{r}-\frac{x_i}{r^2}\frac{x_i}{r}}.\]
    だから,
    \[\Lap u=v''(r)+v'(r)\paren{\frac{n}{r}-\frac{1}{r}}=v''(r)+\frac{n-1}{r}v'(r)\]
    が必要.よって,$v$は微分方程式
    \[v''+\frac{n-1}{r}v'=0\]
    を満たすことが必要である.

    そしてこの微分方程式は$v'\ne0$のとき
    \[\frac{v''}{v'}=\frac{1-n}{r}=\log(\abs{v'})',\qquad\therefore\quad v'(r)=\frac{a}{r^{n-1}}\;(a\in\R)\]
    を含意する.
\end{Proof}

\begin{example}[基本調和関数]\mbox{}
    \begin{enumerate}
        \item $n=1$のとき,劣調和性は凸関数を意味し,調和関数は凸かつ凹関数であるから,1次関数に限る.
        \item $n=2$のとき,
        \[u(x)=\begin{cases}
            \log\abs{x}&x\ne0,\\
            -\infty&x=0
        \end{cases}\]
        は調和になる.
        これを\textbf{基本調和関数}という.
        Poisson方程式の基本解はこの定数倍になる.
        \item $n\ge3$のとき,次を\textbf{基本調和関数}という:
        \[u(x)=\begin{cases}
            -\frac{1}{\abs{x}^{n-2}}&x\ne 0\\
            -\infty&x=0.
        \end{cases}\]
    \end{enumerate}
\end{example}

\begin{example}
    多項式
    \[u(x,y)=a+bx+cy+d(x^2-y^2)+exy\quad(a,b,c,d,e\in\R)\]
    は$\R^2$上の調和関数である.
\end{example}

\begin{remarks}
    $\R^3$の静電場のポテンシャル$e_a(x)=\frac{1}{\abs{x-a}}$を,$a\in\R^3$に関して適当な重みで重ね合わせることにより,任意の静電ポテンシャルは実現できる.
    これを表現するのが畳み込みである.
    重ね合わせの原理の無限化である.
    超関数の言葉で,この静電場のポテンシャルの最も重要な性質は
    \[\Lap_x\frac{1}{\abs{x-a}}=-4\pi\delta(x-a)\]
    であることである.
\end{remarks}

\subsection{基本解:Cauchyの積分表示を与えるもの}

\begin{tcolorbox}[colframe=ForestGreen, colback=ForestGreen!10!white,breakable,colbacktitle=ForestGreen!40!white,coltitle=black,fonttitle=\bfseries\sffamily,
title=]
    Greenの第二恒等式\ref{cor-Green-identity}が導く積分恒等式
    \[u(x)=\int_{\Om}\delta_0(y-x)udx=\int_{\partial\Om}\paren{u\pp{\Phi}{\nu}(y-x)-\Phi(y-x)\pp{u}{\nu}}dS+\int_\Om\Phi(y-x)\Lap udx,\quad x\in\Om.\]
    はCauchyの積分表示の実解析における一般化となっている.
\end{tcolorbox}

\begin{remarks}
    $n=3$の場合の基本解は$-\frac{1}{4\pi}\frac{1}{\abs{x}}$が与えるが,
    これは重力や静電気力のポテンシャルの関数形である.
    さて,一般の調和関数を,ある状態における静電ポテンシャルとみるならば,それは境界または外部に分布した電荷の重ね合わせによって得られるはずである.
    この「重ね合わせ」が畳み込みによって表現される.
\end{remarks}

\begin{definition}[fundamental solution]
    一般に,変数$x$に関する偏微分作用素$L_x$に対して,
    \[L_x\Phi(x,y)=\delta(x-y)\]
    を満たす関数$\Phi$を,\textbf{点$x$を特異点にもつ基本解}という.
\end{definition}
\begin{remarks}[Green's representation formula]\label{remarks-Green}
    これを$-\Lap\Phi=\delta_0\;\on\R^n$とも表す.
    すると,$u$と$\Phi$に対するGreenの第二恒等式\ref{cor-Green-identity}より,
    \[u(x)=\int_{\Om}\delta_0(y-x)udx=\int_{\partial\Om}\paren{u\pp{\Phi}{\nu}(y-x)-\Phi(y-x)\pp{u}{\nu}}dS+\int_\Om\Phi(y-x)\Lap udx,\quad x\in\Om.\]
    すると,もし$u$がコンパクト台を持ち,$\Om=\R^n$ならば,第一項が消えて,
    \begin{align*}
        -\Lap u(x)&=\int_{\R^n}-\Lap\Phi(x-y)f(y)dy\\
        &=\int_{\R^n}\delta_0(x-y)f(y)dy=f(x).
    \end{align*}
    を得ることになる.
\end{remarks}

\section{劣調和函数と球面平均性}

\begin{tcolorbox}[colframe=ForestGreen, colback=ForestGreen!10!white,breakable,colbacktitle=ForestGreen!40!white,coltitle=black,fonttitle=\bfseries\sffamily,
title=]
    広義劣調和関数は半連続性から上に有界で,実は局所可積分でもある.
    正則関数の絶対値は劣調和である.
\end{tcolorbox}

\subsection{定義と球面平均性}

\begin{definition}[subharmonic, harmonic]
    $\Om\osub\R^n$上の関数$u\in C^2(\Om)$について,
    \begin{enumerate}
        \item \textbf{$\Om$上劣調和}であるとは,次を満たすことをいう:
        \[\forall_{x\in\Om}\;-\Lap u(x)\le 0.\]
        \item \textbf{$\Om$上調和}であるとは,$u,-u$がいずれも劣調和であることをいう.
    \end{enumerate}
\end{definition}

\begin{lemma}[球面平均の微分の表示]\label{lemma-derivative-of-surface-avarage-of-arbitrary-function}
    $\Om\osub\R^n$を領域,$u\in C^2(\Om)$,$B(x,r)\ssub\Om\;(x\in\Om)$について,
    球面平均
    \[\phi(R):=\dint_{\partial B(x,R)}u(y)dS(y)\]
    の微分は
    \[\phi'(R)=\frac{1}{\abs{\partial B(x,R)}}\int_{B(x,R)}\Lap u\;dz=\frac{R}{n}\dint_{B(x,R)}\Lap udz.\]
    と球平均の$R/n$倍で表せる.
\end{lemma}
\begin{Proof}
    変数変換$y=x+Rz$のJacobianは$\Abs{\pp{y}{z}}=R^{n-1}$であることに注意して,
    \begin{align*}
        \phi(R)&=\dint_{\partial B(x,R)}u(y)dS(y)\\
        &=\int_{\partial B(0,1)}u(x+Rz)\frac{dS(z)}{\abs{\partial B(x,R)}/R^{n-1}}\\
        &=\dint_{\partial B(0,1)}u(x+Rz)dS(z).
    \end{align*}
    と計算出来る.
    すると,$\phi\in C^2(\Om)$であることと$\partial B(0,1)$のコンパクト性から,$\phi,\phi'$はいずれも可積分であるため,
    微分と積分の交換が可能であることより,$\phi'$が次のように計算できる:
    \begin{align*}
        \phi'(R)&=\dint_{\partial B(0,1)}Du(x+Rz)\cdot zdS(z)\\
        &=\dint_{\partial B(x,R)}Du(y)\frac{y-x}{R}dS(y)=\dint_{\partial B(x,R)}\pp{u}{\nu}dS(y)\\
        &=\frac{r}{n}\dint_{B(x,r)}\Lap u(y)dy
    \end{align*}
    $\frac{y-x}{R}$とは$y\in\partial B(x,R)$における外側単位法線ベクトルに他ならないことに注意.
    最後の2行はGreenの恒等式(1)\ref{cor-Green-identity}による.
\end{Proof}

\begin{theorem}[劣調和関数の平均定理]\label{thm-mean-value-theorem-of-subharmonic-function}
    $u\in C^2(\Om)$を劣調和函数とする.
    このとき,任意の$B(x,R)\ssub\Om$について,次が成り立つ:
    \begin{enumerate}
        \item $u(x)\le\dint_{\partial B(x,R)}u(y)dS_y$.
        \item $u(x)\le\dint_{B(x,R)}u(y)dy$.
    \end{enumerate}
\end{theorem}
\begin{Proof}
    任意の$B(x,R)\ssub\Om\;(x\in\Om,R>0)$を取る.
    \begin{enumerate}
        \item 補題と条件$\Lap u\ge0$を併せると,この球面での平均の微分は$\varphi'(r)\ge0$より,単調増加である.
        よって,Lebesgueの優収束定理から,
        \[\phi(R)\ge\lim_{r\to0}\phi(r)=\lim_{r\to0}\dint_{\partial B(x,r)}u(y)dS_y=u(x).\]
        \item 球面上の積分と球体上の積分との関係\ref{thm-sphere-and-ball}から,(1)の結果より
        \begin{align*}
            \dint_{B(x,R)}u(y)\dvol&=\frac{1}{\abs{B(x,R)}}\int^R_0\int_{\partial B(0,r)}u(x)dS_xdr\\
            &\ge\frac{1}{\abs{B(x,R)}}\int^R_0u(x)\abs{\partial B(x,r)}dr=u(x).
        \end{align*}
        が従う.
    \end{enumerate}
\end{Proof}

\subsection{調和関数の球面平均性による特徴付け}

\begin{tcolorbox}[colframe=ForestGreen, colback=ForestGreen!10!white,breakable,colbacktitle=ForestGreen!40!white,coltitle=black,fonttitle=\bfseries\sffamily,
title=]
    調和関数は球面平均性によって特徴づけられる.
    より一般に,関数の劣調和性と広義劣調和性(上半連続で球面平均性を持つ関数)とは等価である.
\end{tcolorbox}

\begin{corollary}[調和関数の球面平均定理]\label{cor-surface-avarage-of-harmonic-functions}
    $\Om\osub\R^n$を領域,$u\in C^2(\Om)$は$\Om$上調和とする.
    このとき,任意の$B(x,R)\ssub\Om$について,
    \[u(x)=\dint_{\partial B(x,r)}udS=\dint_{B(x,r)}udy.\]
\end{corollary}
\begin{Proof}
    補題より,
    球面平均$\dint_{\partial B(x,r)}udS$は$r\ge0$について定値であることによる.
    球上の平均は,積分の計算\ref{thm-sphere-and-ball}による.
\end{Proof}

\begin{proposition}[\cite{Evans} Th'm I.2.2.3]
    $u\in C^2(U)$が任意の開球$B(x,r)\subset U$について
    \[u(x)=\dint_{\partial B(x,r)}udS.\]
    を満たすならば,$u$は調和関数である.
\end{proposition}
\begin{Proof}
    仮に調和関数でないとして矛盾を導く:$\Lap u\not\equiv0$.
    このとき,ある開球$B(x,r)$が存在して,その上で$\Lap u>0$を満たす.
    このとき,$x$を中心とした球面上の平均の値は変わらないから,$\phi'=0$が必要であるが,
    \[0=\phi'(r)=\frac{r}{n}\dint_{B(x,r)}\Lap u(y)dy>0\]
    より,これは矛盾.
\end{Proof}

\subsection{劣調和関数の球面平均性による特徴付け}

\begin{tcolorbox}[colframe=ForestGreen, colback=ForestGreen!10!white,breakable,colbacktitle=ForestGreen!40!white,coltitle=black,fonttitle=\bfseries\sffamily,
title=]
    %積分の言葉による特徴付けの美点の一つに,極限定理が使いやすいということがある.
    %この論理の流れは複素解析におけるCauchy流の手法と相似形である.
    %微分方程式に関するこの性質を「安定性」という.
    %Laplace方程式に関する$L^\infty$-安定性とは,調和関数の広義一様収束極限を指す非正式の用語である.
\end{tcolorbox}

\begin{definition}
    領域$\Om\osub\R^n$上の
    関数$u:\Om\to\R\cup\{-\infty\}$が\textbf{広義劣調和}であるとは,次の2条件を満たすことをいう:
    \begin{enumerate}
        \item $u$は上半連続:$u\in USC(\Om)$.
        \item $u$は球面平均値定理を満たす:$\forall_{0<r<\dist(x,\partial\Om)}\;u(x)\le\dint_{B(x,r)}udS$.
    \end{enumerate}
\end{definition}

\begin{theorem}[2つの定義の同値性]
    $u\in C^2(\Om)$について,次の2条件は同値:
    \begin{enumerate}
        \item $u$は広義劣調和.
        \item $u$は劣調和:$-\Lap u\le0\;\on\Om$.
    \end{enumerate}
\end{theorem}
\begin{Proof}
    (1)$\Rightarrow$(2)を示せば良い.
    $u\in C^2(\Om)$と仮定したから,2次までのTaylorの定理が使える.よって,
    任意の$B(x_0,r)\ssub\Om\;(x_0\in\Om,r>0)$について,広義劣調和性より,
    \begin{align*}
        u(x_0)&\le\M_r[u(x_0)]\\
        &=\dint_{\partial B(0,1)}u(x_0+rz)dS_z\\
        &=\dint_{\partial B(0,1)}\paren{u(x_0)+rDu(x_0)z+\frac{r^2}{2}z^\top D^2u(x_0)z+o(r^2)}dS_z\\
        &=u(x_0)+\frac{1}{\abs{\partial B(0,1)}}r\int_{\partial B(0,1)}Du(x_0)\cdot dS_z+\frac{1}{\abs{\partial B(0,1)}}\frac{r^2}{2}\int_{\partial B(0,1)}D^2u(x_0)z\cdot dS_z+o(r^2)\\
        &=u(x_0)+\frac{1}{\abs{\partial B(0,1)}}r\int_{B(0,1)}\div(Du(x_0))\dvol(z)+\frac{1}{\abs{\partial  B(0,1)}}\frac{r^2}{2}\int_{B(0,1)}\div(D^2u(x_0)z)\dvol(z)+o(r^2)
    \end{align*}
    と変形出来るが,$\div(Du(x_0))=0$は定数ベクトル場の微分なので零,同様に$z$の微分であることを忘れなければ,$\div(D^2u(x_0)z)=\Tr(D^2u(x_0))=\Lap u(x_0)$より,
    両辺の$u(x_0)$を相殺して$r^2/2$で割ることで,
    \[0\le\dint_{\partial B(0,1)}\Lap u(x_0)dz+O(r)=\Lap u(x_0)+O(r).\]
    $r\to0$とすると,$x_0\in\Om$は任意だったから,$\Lap u\ge0$.
\end{Proof}

\subsection{Harnackの定理:Poisson方程式の解作用素の$L^\infty$-安定性}

\begin{tcolorbox}[colframe=ForestGreen, colback=ForestGreen!10!white,breakable,colbacktitle=ForestGreen!40!white,coltitle=black,fonttitle=\bfseries\sffamily,
title=]
    正則関数の場合と同様,調和関数の広義一様収束極限は再び正則である.
\end{tcolorbox}

\begin{corollary}[$L^\infty$-安定性 (Harnack)]\label{cor-Harnack-L-infty-stability}
    任意の領域$\Om\subset\R^n$に於ける
    広義調和関数の族$\{u_n\}\subset C(\Om)$が$u$に局所一様収束するならば,$u$も広義調和である.
\end{corollary}
\begin{Proof}
    任意の$B(x,r)\subset\Om$について,$u_n$の調和性より,
    \[u_n(x)=\A_r[u_n(x)]=\dint_{B(x,r)}u_n(y)\dvol.\]
    この極限$n\to\infty$を取ると,$u(x)=\A_r[u(x)]$より,体積平均値の原理を満たす.
    コンパクト一様収束極限は連続であることに注意すると,$u$もたしかに広義調和.
\end{Proof}

\subsection{広義調和関数の実解析性}

\begin{tcolorbox}[colframe=ForestGreen, colback=ForestGreen!10!white,breakable,colbacktitle=ForestGreen!40!white,coltitle=black,fonttitle=\bfseries\sffamily,
title=]
    Weylの定理はこれを一般化し,Laplace方程式の全ての弱解は滑らかであることを保証する.
    さらに一般化すると楕円型方程式には正則性が成り立つ.
\end{tcolorbox}

\begin{lemma}[軟化子の性質]
    次の3条件を満たす関数$\rho\in C^\infty(\R^n)$は存在する.
    \begin{enumerate}
        \item $\supp\rho\subset\o{B(0,1)}$.
        \item $\rho\ge0$かつ$\int_{\R^n}\rho dx=1$.
        \item $\rho$は$x=0$を中心として球対称.
    \end{enumerate}
    これに対して,$\rho^\ep(x):=\frac{1}{\ep^n}\rho\paren{\frac{x}{\ep}}$とおくと,線型作用素$\rho^\ep*:C(\Om)\to C^\infty(\Om)$を定める.
\end{lemma}
\begin{Proof}
    \[\wh{\rho}(x):=e^{-\frac{1}{1-\abs{x}^2}}1_{\Brace{\abs{x}<1}}\]
    を考えると,$\abs{x}=1$上でも任意階微分可能である.
\end{Proof}

\begin{theorem}[Laplace方程式の解の正則性]
    $\Om\osub\R^n$を領域,$u\in C(\Om)$を広義調和関数とする.
    このとき,
    \begin{enumerate}
        \item $u\in C^2(\Om)$でもあり,$\Lap u(x)=0\;\on\Om$.
        \item 実は$u\in C^\infty(\Om)$である.
    \end{enumerate}
\end{theorem}
\begin{Proof}\mbox{}
    \begin{enumerate}
        \item 前節の定理より.
        \item 任意の$B(x_0,\ep)\ssub\Om\;(x_0\in\Om,\ep>0)$を取る.
        これに対して,
        \begin{align*}
            u^\ep(x_0)&=\int_{B(x_0,\ep)}\rho^\ep(x-y)u(y)dy\\
            &=\int^\ep_0\int_{\partial B(0,1)}\rho^\ep(r\sigma)u(x_0-r\sigma)r^{n-1}drd\sigma\\
            &=u(x_0)\abs{\partial B(0,1)}\int^\ep_0h^\ep(r)r^{n-1}dr=u(x_0)
        \end{align*}
        が成り立つから,$u\in C^\infty(\Om)$.
    \end{enumerate}
\end{Proof}

\begin{theorem}[調和関数は解析的である]
    任意の領域$\Om\subset\R^n$における調和関数は,$\Om$で実解析的である.
\end{theorem}
\begin{Proof}
    Greenの積分表示\ref{thm-mean-value-theorem-of-subharmonic-function}によると,調和関数$u\in C^2(\Om)$は
    \[u(x)=\int_{\partial\Om}\paren{u(y)\pp{\Phi}{\nu}(y-x)-\Phi(y-x)\pp{u}{\nu}(y)}dS,\quad x\in\Om.\]
    と表せる.被積分関数は$x$に関して実解析的であることから従う.
\end{Proof}

\subsection{展望:Laplace方程式の弱解の正則性}

\begin{tcolorbox}[colframe=ForestGreen, colback=ForestGreen!10!white,breakable,colbacktitle=ForestGreen!40!white,coltitle=black,fonttitle=\bfseries\sffamily,
title=]
    超関数の意味で調和ならば,原義調和な修正が存在する.
    一方で,波動方程式は滑らかでない弱解を持つ.
\end{tcolorbox}

\begin{definition}
    $u\in L^1_\loc(D)$が\textbf{超関数の意味で調和}であるとは,
    \[\forall_{\varphi\in C_c^\infty(D)}\quad\int_Du\Lap\varphi dx=0\]
    を満たすことをいう.
\end{definition}

\begin{theorem}[Weylの補題]
    $u\in L^1_\loc(D)$が超関数の意味で調和であるとする.このとき,ある調和関数$\wt{u}$が存在して,$u=\wt{u}\;\ae D$.
\end{theorem}

\section{最大値原理と初期値問題の一意性}

\begin{tcolorbox}[colframe=ForestGreen, colback=ForestGreen!10!white,breakable,colbacktitle=ForestGreen!40!white,coltitle=black,fonttitle=\bfseries\sffamily,
title=]
    波動は中心から伝播し,最大点・最小点は境界に限るので,境界を見ればいい(弱原理).
    もし内点で最大点・最小点が見つかったならば,その調和関数は定数である(強原理).
    この性質は,全て調和関数の球面平均性から導出される.
\end{tcolorbox}

\begin{remark}
    弱最大値原理は有界開集合,強最大値原理は領域で成り立つ.
\end{remark}

\subsection{弱最大値原理}

\begin{tcolorbox}[colframe=ForestGreen, colback=ForestGreen!10!white,breakable,colbacktitle=ForestGreen!40!white,coltitle=black,fonttitle=\bfseries\sffamily,
title=]
    調和関数の最大値と最小値を知るには,境界値だけを見れば良い.
    これによりDirichlet問題の解の一意性,一様ノルムに関する安定性,正値保存性,順序保存性も導かれる.
\end{tcolorbox}

\begin{theorem}
    $\Om\osub\R^n$を有界開集合,$u\in C(\o{\Om})\cap C^2(\Om)$について,
    \begin{enumerate}
        \item $\Om$上劣調和ならば,$\max_{x\in\o{\Om}}u(x)=\max_{x\in\partial\Om}u(x)$.
        \item $\Om$上調和ならば,$u$は必ず$\partial\Om$上で最大値と最小値を取る.
    \end{enumerate}
\end{theorem}
\begin{Proof}\mbox{}
    \begin{enumerate}
        \item 劣調和関数$u$は$x_0\in\Om$で最大値を取るとし,矛盾を導く.
        $u$の変形
        \[u^\ep(x):=u(x)+\ep\abs{x}^2\quad(\ep>0)\]
        を考えると,$\ep\to0$のときの$u^\ep\searrow u$は$\Om$上の一様収束.
        これより,$\max_{x\in\o{\Om}}u^\ep(x)\to\max_{x\in\o{\Om}}u(x)$と$\max_{x\in\partial\Om}u^\ep(x)\to\max_{x\in\partial\Om}u(x)$とが従う.
        これは
        \[\abs{\max_{x\in\o{\Om}}u(x)-\max_{x\in\o{\Om}}u^\ep(x)}\le\max_{x\in\o{\Om}}\abs{u(x)-u^\ep(x)}\to0\]
        であるためである,
        
        これと$\max_{x\in\o{\Om}}u(x)>\max_{x\in\partial\Om}u(x)$とを併せると,十分小さい$\ep>0$について$\exists_{x^\ep\in\Om}\;\max_{x\in\o{\Om}}u^\ep(x)=u^\ep(x^\ep)$.
        したがって,$x^\ep$での$u^\ep$のHesse行列は半負定値になる:$D^2u^\ep(x^\ep)\le0$.
        よって特に跡が負:
        \[\Tr(D^2u^\ep(x^\ep))=\Lap u^\ep(x^\ep)=\Lap u(x^\ep)+2\ep n\le0\]
        よって,$\Lap u(x^\ep)<0$で,これは$u$の劣調和性に矛盾.
        \item 調和関数は劣調和かつ優調和であるため.
    \end{enumerate}
\end{Proof}

\subsection{Dirichlet問題の解の一意性}

\begin{corollary}[Dirichlet問題の解の一意性]\label{cor-uniqueness-of-Dirichlet-problem-of-Laplace-eq}
    有界領域$\Om\osub\R^n$上の,連続なデータ$g\in C(\partial\Om),f\in C(\o{\Om})$が定める次のDirichlet問題の古典解$u\in C(\o{\Om})\cap C^2(\Om)$は高々1つである:
    \[\mathrm{(D)}\quad\begin{cases}
        \Lap u=f&\on \Om,\\
        u=g&\on \partial\Om.
    \end{cases}\]
\end{corollary}
\begin{Proof}
    $u_1,u_2$をいずれも古典解とすると,$\om:=u_1-u_2$は,$f=g=0$とした場合のDirichlet問題を満たすが,これは弱最大値原理より,最大値も最小値も$0$である.よって,$\om=0$.
\end{Proof}
\begin{remarks}
    この議論は最大値原理を用いなくとも,次の議論を正当化するSobolevの方法が考え得る.まず$\Lap\om=0$より,$\om$と内積を取っても例だから,Greenの定理\ref{cor-Green-identity}より,
    \[(\Lap\om,\om)_{L^2(\Om)}=-\int_\Om(\nabla\om)^2\dvol+\int_{\partial\Om}\underbrace{\om}_{=0\;\on\partial\Om}\pp{\om}{\b{n}}dS=0.\]
    よって$\nabla\om=0$が必要より,$\om$は定数.境界条件と併せて,$\om=0$.
\end{remarks}

\begin{remark}[領域の有界性は本質的な制約である]
    \[\begin{cases}
        -\Lap u=0&\on\R^{n-1}\times\R^+,\\
        u=0&\on\R^{n-1}\times\{0\}
    \end{cases}\]
    の解は$u=0$と$u=\pr_n$の2つある.
    これについては,無限遠での増大性に制限を加えることで,一意性が回復される(Phragmen-Lindelof).
\end{remark}

\subsection{強最大値定理}

\begin{tcolorbox}[colframe=ForestGreen, colback=ForestGreen!10!white,breakable,colbacktitle=ForestGreen!40!white,coltitle=black,fonttitle=\bfseries\sffamily,
title=]
    調和関数の
    最大値・最小値が内点でも到達されるならば,それは定数関数である.
    これは,内点でもあるような最大値点全体の集合が$\Om$内で連結であることから従う.
\end{tcolorbox}

\begin{theorem}
    $\Om\osub\R^n$を領域,$u\in C(\Om)$を広義劣調和関数とする.
    次が成り立つならば,$u$は定数である:
    \[\exists_{x_0\in\Om}\;u(x_0)=\max_{\o{\Om}} u=:C.\]
\end{theorem}
\begin{Proof}
    最大点の集合$A:=u^{-1}(C)\subset\Om$は仮定により空でない.
    \begin{enumerate}
        \item $u$の上半連続性より,$A$は$\Om$-相対閉集合である.
        \item 任意の$\o{x}\in A$について,$u$の広義劣調和性から,任意の$B(x_0,r)\ssub\Om$について,
        \begin{align*}
            C=u(\o{x})&\le\int_{\partial B(\o{x},r)}u\frac{dS_y}{\abs{\partial B(\o{x},r)}}.
        \end{align*}
        よって,
        \[\int_{\partial B(\o{x},r)}(u(y)-C)\frac{dS_y}{\abs{\partial B(\o{x},r)}}.\]
        一方で$C$は最大値だったから$u(y)-C\le0$より,
        $u$は$\partial B(\o{x},r)$上で一定値$C$を取ることが必要.
        $r>0$は任意だったから,$A$は開集合でもある.
    \end{enumerate}
\end{Proof}

\subsection{解の無限伝播速度}

\begin{tcolorbox}[colframe=ForestGreen, colback=ForestGreen!10!white,breakable,colbacktitle=ForestGreen!40!white,coltitle=black,fonttitle=\bfseries\sffamily,
    title=比較原理:解作用素は正作用素である]
    調和関数は,境界上で正な値を取れば,内点では常に正である.
\end{tcolorbox}

\begin{theorem}[比較原理]\label{thm-comparison-principle}
    $\Om\osub\R^n$を有界領域,$u\in C^2(\Om)\cap C(\o{\Om})$は次を満たすとする:
    \[\begin{cases}
        \Lap u=0\quad\on \Om,\\
        u=g\ge0\quad\on\partial \Om.
    \end{cases}\]
    このとき,
    \begin{enumerate}
        \item $u\ge0\;\on\o{\Om}$であり,
        \item $\exists_{x\in\partial\Om}\;g(x)>0$である
    \end{enumerate}
    ならば,$u>0\;\on\Om$.
\end{theorem}
\begin{Proof}
    仮に$u_0:=u(x)\le 0$を満たす点$x\in\Om$が存在するならば,その点が$\o{\Om}$上の最小値を達成してしまうから,$u\equiv u_0$が必要だが,これは$\partial\Om$上のある点では値が正であることに矛盾.
\end{Proof}

\subsection{Neumann問題の解の一意性}

\begin{corollary}[Neumann問題の一意性]
    次を満たす有界領域$\Om\osub\R^n$については,Neumann問題(N)の解が定数の差を除いて一意に定まる:
    \begin{quote}
        (内部球条件) 任意の$x_0\in\partial\Om$について,開球$B\subset\Om$が存在して$\partial\Om\cap B=\{x_0\}$を満たす.
    \end{quote}
    \[\mathrm{(N)}\quad\begin{cases}
        -\Lap u=f&\on\Om\\
        Du\cdot n=g&\on\Om.
    \end{cases}\]
    ただし,$n$は$\partial\Om$上の単位外側法線ベクトル場とする.
    \footnote{$n$として傾けたものを採用した境界条件をreflection境界条件という.}
\end{corollary}
\begin{Proof}
    $\om:=u_1-u_2$とすると,$f=g=0$としたNeumann境界問題を満たす.ここでもし$\om$が定数でないならば,強最大値原理より$\exists_{x_0\in\Om}\;\om(x_0)=\max_{\o{\Om}}\om$が必要.
    さらに弱最大値原理より,内部では最大値に届かないから,$\forall_{x\in\Om}\;\om(x_0)>\om(x)$が必要.
    これはHopfの補題より$(D\om(x_0)|n(x_0))>0$を含意し,Neumann境界条件に矛盾.
\end{Proof}

\begin{observation}
    $C^2$-級有界領域については内部球が一様に取れる.
    一方で,$C^1$-球領域は$y=x^{3/2}$の頂点が反例となる.
\end{observation}

\begin{lemma}[内部球条件の使い方 (Hopf)]\mbox{}
    \begin{enumerate}
        \item $y\in\R^n$を中心とする開球$B(y,R)\;(R>0)$上の劣調和関数
        $u\in C^2(B(y,R))\cap C^1(\o{B(y,R)})$が
        境界上の点$x_0\in\partial B(y,R)$において$\forall_{x\in B(y,R)}\;u(x_0)>u(x)$を満たすとする.このとき,$(Du(x_0)|\b{n}(x_0))>0$.
        \item 有界領域$\Om\osub\R^n$の境界上の点$x_0\in\partial\Om$において内部球条件が成り立つとする.
        このとき,任意の調和関数$u\in C^2(\Om)\cap C^1(\o{\Om})$について,
        \[\forall_{x\in\Om}\;u(x_0)>u(x)\Rightarrow (Du(x_0)|n(x_0))>0.\]
    \end{enumerate}
\end{lemma}

\subsection{展望:一般の楕円型方程式の最大値の原理}

\begin{tcolorbox}[colframe=ForestGreen, colback=ForestGreen!10!white,breakable,colbacktitle=ForestGreen!40!white,coltitle=black,fonttitle=\bfseries\sffamily,
title=]
    最大値の原理は,より弱い形で,より一般の楕円型の微分作用素について成り立つ.
\end{tcolorbox}

\begin{theorem}[楕円型方程式の解の正値保存性]
    有界領域$\Om$上の
    2階の楕円型方程式
    \[Lu=\sum_{i,j\in[m]}a_{ij}(x)\pp{^2u}{x_i\partial x_j}+\sum_{j\in[m]}b_j(x)\pp{u}{x_j}+c(x)u\quad\on\Om,\quad (a_{ij}):\text{正定値対称}\]
    は,連続な係数$a,b,c\in C(\o{\Om})$を持つとする.
    これがさらに$c<0\;\on\Om$を満たすとき,解$u\in C^2(\Om)\cap C^1(\o{\Om})$は,$u|_{\partial\Om}\ge0\Rightarrow u\ge0\;\on\o{\Om}$を満たす.
\end{theorem}
\begin{remark}
    $c\le0$としても成り立つが,証明は煩雑になる.$c=0$の場合はさらに弱最大値の原理が成り立つ.
    また最大値の原理は,さらに高階な方程式については,いかなる形でも成り立たない.
\end{remark}

\section{調和関数の局所評価}

\begin{tcolorbox}[colframe=ForestGreen, colback=ForestGreen!10!white,breakable,colbacktitle=ForestGreen!40!white,coltitle=black,fonttitle=\bfseries\sffamily,
title=]
    Cauchyの評価
    $\abs{f^{(n)}(a)}\le\frac{n!}{r^n}\norm{f}_{\partial\Delta(a,r)}$
    と同様な消息が成り立つ.
    \begin{enumerate}
        \item $L^\infty$-評価:任意の$[B(a,r)]\subset U\osub\R^n$について,
        \[\abs{D^\al u(a)}\le \frac{C}{r^{\abs{a}}}\norm{u}_{L^\infty(B(a,r))}\le \frac{C}{r^{\abs{a}}}\norm{u}_{\partial B(a,r)}.\]
        \item 結果として,調和関数列が一様ノルムについて有界ならば,一様ノルムについて相対コンパクトである.これは正則関数の場合は「正規族」と呼んでいた性質である.
        \item $L^1$-評価:任意の$[B(a,r)]\subset U\osub\R^n$について,
        \[\abs{D^\al u(a)}\le\frac{C}{r^{n+\abs{k}}}\norm{u}_{L^1(B(a,r))}.\]
        $L^\infty$-の場合よりもさらに次元の数$n$だけ厳しい収束レートである.
        \item Liouvilleの定理:$\R^n$上で有界ならば定数.
        \item Harnackの不等式:任意の非負の調和関数は,有界領域$D$上で
        \[\sup_{x\in D}u\le C\inf_{x\in D}u\]
        を満たす.ある種のDoobの不等式か!?
    \end{enumerate}
\end{tcolorbox}

\subsection{導関数の$L^1$-局所評価}

\begin{tcolorbox}[colframe=ForestGreen, colback=ForestGreen!10!white,breakable,colbacktitle=ForestGreen!40!white,coltitle=black,fonttitle=\bfseries\sffamily,
title=]
    調和関数の$k$階微分係数は,$O\paren{\norm{u}_{L^1(B(x_0,r))}/r^{n+k}}$で抑えられる.
\end{tcolorbox}

\begin{theorem}\label{thm-L1-evaluation-of-derivarive}
    $\Om\osub\R^n$を領域,$u$をその上の調和関数,$\al\in\N^n$を多重指数,$k:=\abs{\al}$とする.
    任意の内部球$B(x_0,r)\subset\Om$について,
    \[\abs{D^\al u(x_0)}\le\frac{C_k}{r^{n+k}}\norm{u}_{L^1(B(x_0,r))},\quad C_0:=\frac{1}{\om_n},C_k:=\frac{(2^{n+1}nk)^k}{\om_n}.\]
\end{theorem}
\begin{Proof}\mbox{}
    \begin{enumerate}
        \item $k=0$のとき,球面平均定理\ref{cor-surface-avarage-of-harmonic-functions}と三角不等式より,
        \[u(x_0)=\dint_{B(x_0,r)}udy\le\frac{1}{\om_nr^n}\int_{B(x_0,r)}\abs{u(y)}dy.\]
        \item $k=1$のとき,$u$が調和ならば,その導関数$u_{x_i}$も調和であることに注意すると,$\nu$を球面の外側単位法線ベクトル,$\nu^i$をその各成分とすると,発散定理から,まず一階の微分が
        \begin{align*}
            \forall_{x_0\in\R^n}\quad\abs{u_{x_i}(x_0)}&=\Abs{\dint_{B(x_0,r/2)}u_{x_i}dy}\\
            &=\frac{1}{\abs{B(x_0,r/2)}}\Abs{\int_{\partial B(x_0,r/2)}u\nu^idS_y}\\
            &\le\frac{\abs{\partial B(x_0,r/2)}}{\abs{B(x_0,r/2)}}\norm{u}_{L^\infty(B(x_0,r/2))}\\
            &=\frac{n\om_n(r/2)^{n-1}}{\om_n(r/2)^n}=\frac{2n}{r}.
        \end{align*}
        と$L^\infty(B(x_0,r/2))$-ノルムによって評価できる.引き続き,これを評価するために$u$の$\partial B(x_0,r/2)$での最大値を評価すると,(1)より,
        \begin{align*}
            \forall_{x\in\partial B(x_0,r/2)}\quad\abs{u(x)}&=\Abs{\dint_{B(x,r/2)}udy}\le\frac{1}{\abs{B(x,r/2)}}\int_{B(x,r/2)}\abs{u}dy\\
            &\le\frac{1}{\abs{B(x,r/2)}}\int_{B(x_0,r)}\abs{u}dy=\frac{1}{\om_n}\paren{\frac{2}{r}}^n\norm{u}_{L^1(B(x_0,r))}.
        \end{align*}
        を得る.2つを併せると,
        \[\abs{Du^\al(x_0)}\le\frac{2n}{r}\cdot\frac{1}{\om_n}\paren{\frac{2}{r}}^n\norm{u}_{L^1(B(x_0,r))}=\frac{2^{n+1}n}{\om_nr^{n+1}}\]
        と評価できている.
        \item $k>2$のときも同様.
    \end{enumerate}
\end{Proof}

\begin{lemma}[成分毎のGauss-Greenの定理の復習]
    領域$\Om\osub\R^n$は区分的$C^1$-境界を持つとする.
    $\nu$を$\partial\Om$の外向き単位法線ベクトルとすると,
    \[\forall_{u\in C^1(\Om)}\quad\int_\Om u_{x_i}dy=\int_{\partial\Om}u\nu^idS_y.\]
\end{lemma}
\begin{Proof}
    ベクトル値関数
    $\om(x):=u(x)e_i$に対する発散定理によると,
    \begin{align*}
        \int_\Om u_{x_i}dy&=\int_\Om\div(\om(y)) dy\\
        &=\int_{\partial\Om}\om(y)\cdot\nu dS_y
        =\int_{\partial\Om}u(x)\nu^idS_y.
    \end{align*}
    というのもこれは,\ref{thm-Gauss-Green}の再生産である.
\end{Proof}

\subsection{Liouvilleの定理}

\begin{corollary}\label{cor-Liouville}
    $\R^n$上の調和関数が有界ならば定数である.
\end{corollary}
\begin{Proof}
    導関数の$L^1$-評価\ref{thm-L1-evaluation-of-derivarive}より,任意の$B(x_0,r)\subset\R^n$について,
    \begin{align*}
        \abs{Du(x_0)}&\le\frac{C_1}{r^{n+1}}\norm{u}_{L^1(B(x_0,r))}\\
        &=\frac{C_1}{r^{n+1}}\int_{B(x_0,r)}\abs{u(y)}dy\\
        &=\frac{C_1}{r^{n+1}}\om_nr^n\sup_{y\in\R^n}\abs{u(y)}\xrightarrow{r\to\infty}0.
    \end{align*}
\end{Proof}

\begin{proposition}
    $\R^n$上の下に有界な調和関数は定数に限る.
\end{proposition}
\begin{remark}
    $\R^2$の下に有界な優調和関数は定数に限るが,$\R^n\;(n\ge3)$では正しくない.
\end{remark}

\subsection{導関数の$L^\infty$-評価}

\begin{tcolorbox}[colframe=ForestGreen, colback=ForestGreen!10!white,breakable,colbacktitle=ForestGreen!40!white,coltitle=black,fonttitle=\bfseries\sffamily,
title=]
    調和関数の$k$階微分係数の$\Om'\ssub\Om$上の$L^\infty$-ノルムは,$\Om$上の$L^\infty$-ノルムによって,
    $O(d^{-k})\;(d:=\dist(\partial\Om,\Om'))$で抑えられる.
\end{tcolorbox}

\begin{theorem}[Harnackの評価]\label{thm-Harnack-evaluation-of-derivative}
    $u$は$\Om$上調和とする.任意のコンパクト部分集合$\Om'\ssub\Om$と任意の多重指数$\al\in\N^n$について,
    \[\sup_{x\in\Om'}\abs{D^\al u(x)}\le\frac{C}{d^{\abs{\al}}}\sup_{x\in\Om}\abs{u},\quad  C>0,d:=\dist(\partial\Om,\Om').\]
    さらに,$C=(n\abs{\al})^{\abs{\al}}$と取れる.
\end{theorem}

\begin{corollary}[一様有界列は相対コンパクト (Harnack)]
    $\{u_n\}\subset C(\Om)$を調和関数の一様有界列とする.
    このとき,$\Om$上広義一様収束する部分列が存在する.
\end{corollary}
\begin{Proof}
    Harnackの評価により,調和関数の一様有界列は,任意の有界部分集合で同程度連続であることが従う.
    よって,調和関数列は,一様有界ならばすぐにAscoli-Arzelaの定理の要件を満たす正規族になる.
\end{Proof}

\subsection{Harnackの不等式}

\begin{theorem}
    $U\osub\R^n$を有界領域,$V\ssub U$を領域とする.
    このとき,$V$のみに依存する定数$C>0$が存在して,任意の非負調和関数$u:U\to\R_+$について次が成り立つ:
    \[\forall_{u\in\H(U)_+}\quad\sup_{V}u\le C\inf_Vu.\]
    特に,
    \[\forall_{x,y\in V}\quad\frac{1}{C}u(y)\le u(x)\le Cu(y).\]
\end{theorem}
\begin{Proof}
    実は球面平均定理\ref{thm-mean-value-theorem-of-subharmonic-function}から出る.

\end{Proof}

\begin{corollary}[Harnackの収束定理]
    $\{u_n\}\subset H(\Om)$を調和関数の単調増加列で,ある$y\in\Om$について$\{u_n(y)\}\subset\R$は有界になるとする.
    このとき,$\{u_n\}$は広義一様収束する.
\end{corollary}
\begin{Proof}
    点$u_n(y)$にHarnackの不等式を使うと,$\Om$の任意の有界部分集合上での一様評価が得られる.
\end{Proof}

\begin{corollary}[Poisson核が与えるHarnackの不等式]
    $u\in C^2(B(0,R))$を非負な調和関数とする.次が成り立つ:
    \[\frac{R^{n-2}(R-\abs{x})}{(R+\abs{x})^{n-1}}u(0)\le u(x)\le \frac{R^{n-2}(R+\abs{x})}{(R-\abs{x})^{n-1}}u(0)\]
\end{corollary}

\section{演習:外部問題への取り組み}

\begin{tcolorbox}[colframe=ForestGreen, colback=ForestGreen!10!white,breakable,colbacktitle=ForestGreen!40!white,coltitle=black,fonttitle=\bfseries\sffamily,
title=]
    非有界領域上のDirichlet問題,Neumann問題を考えたい.
    そこで,単純閉曲線$S$に関して,これが定める有界領域上の境界値問題を内部問題,非有界領域上の境界値問題を外部問題というが,この外部問題について考える.
\end{tcolorbox}

\subsection{Kelvin変換}

\begin{tcolorbox}[colframe=ForestGreen, colback=ForestGreen!10!white,breakable,colbacktitle=ForestGreen!40!white,coltitle=black,fonttitle=\bfseries\sffamily,
title=]
    Kelvin変換は調和性を保つ.
\end{tcolorbox}

\begin{definition}
    $\Om\subset\R^n\setminus\{0\}$を開集合,$\wt{\Om}:=\Brace{x\in\Om\;\middle|\;\frac{x}{\abs{x}^2}\in\Om}$とする.
    \[\xymatrix@R-2pc{
        K:C^2(\Om)\ar[r]&C^2(\wt{\Om})\\
        \rotatebox[origin=c]{90}{$\in$}&\rotatebox[origin=c]{90}{$\in$}\\
        u\ar@{|->}[r]&\wt{u}(x):=\abs{x}^{2-n}u\paren{\frac{x}{\abs{x}^2}}=\abs{\o{x}}^{2-n}u(\o{x}).
    }\]
    を\textbf{Kelvin変換}という.ただし,$\o{x}:=\frac{x}{\abs{x}^2}$とした.
    これは$\partial B(0,1)$に対する鏡像変換となっている.
\end{definition}

\begin{proposition}\mbox{}
    \begin{enumerate}
        \item $D_x\o{x}(D_x\o{x})^\top=\abs{x}^{-4}I_n$が成り立つ.すなわち,変換$x\mapsto\o{x}$は等角である.
        \item 次が成り立つ:
        \[\Lap\wt{u}(x)=\abs{x}^{-2-n}\Lap u\paren{\frac{x}{\abs{x}^2}},\quad\on\wt{\Om}.\]
        \item $u$が調和ならば,$\wt{u}$も調和である.
    \end{enumerate}
\end{proposition}
\begin{Proof}\mbox{}
    \begin{enumerate}
        \item \[\frac{x}{\abs{X}^2}=\frac{1}{x^2_1+\cdots+x_n^2}\vctrr{x_1}{\vdots}{x_n}\]
        の第$j$成分の$i$に関する偏微分は
        \[\pp{}{x_i}\paren{\frac{x_j}{x_1^2+\cdots+x_n^2}}=-\frac{2x_ix_j}{(x_1^2+\cdots+x_n^2)^2}.\]
        \[\pp{}{x_i}\paren{\frac{x_i}{x_1^2+\cdots+x_n^2}}=\frac{1}{(x_1^2+\cdots+x_n^2)^2}-\frac{2x_i^2}{(x_1^2+\cdots+x_n^2)^2}.\]
        より,
        \[D\paren{\frac{x}{\abs{x}^2}}=\frac{I_n}{\abs{x}^2}-\frac{2}{\abs{x}^4}\begin{pmatrix}x_1x_1&\cdots&x_1x_n\\\vdots&\ddots&\vdots\\x_nx_1&\cdots&x_nx_n\end{pmatrix}.\]
        よって,
        \begin{align*}
            D\paren{\frac{x}{\abs{x}^2}}D\paren{\frac{x}{\abs{x}^2}}^\top&=\frac{I}{\abs{x}^4}-\frac{4}{\abs{x}^4}\begin{pmatrix}x_1x_1&\cdots&x_1x_n\\\vdots&\ddots&\vdots\\x_nx_1&\cdots&x_nx_n\end{pmatrix}+\frac{4}{\abs{x}^8}\Paren{x_ix_j\abs{x}^2}_{i,j\in[n]}\\
            &=\abs{x}^{-4}I_n.
        \end{align*}
        \item 
    \end{enumerate}
\end{Proof}

\subsection{球面外の外部問題}

\begin{problem}
    $\R^2$上のLaplace方程式の外部問題
    \[\begin{cases}
        -\Lap u=0&\abs{x}>1,\\
        u(x)=x_1&\abs{x}=1,\\
        \lim_{\abs{x}}\to\infty u(x)=0.
    \end{cases}\]
    を解け.
\end{problem}
\begin{Proof}
    $u$のKelvin変換
    \[v(x):=u\paren{\frac{x}{\abs{x}^2}},\qquad x\ne0.\]
    を考えると,$v(0):=0$と定めれば,次が必要:
    \[\begin{cases}
        -\Lap v=0&\abs{x}>1,\\
        v(x)=x_1&\abs{x}=1,\\
        v(0)=0.
    \end{cases}\]
    $v(x)=x_1$はこれを満たす調和関数であるが,有界領域におけるDirichlet問題の解の一意性\ref{cor-uniqueness-of-Dirichlet-problem-of-Laplace-eq}より,
    これが解のすべてである.
    以上より,Kelvin変換の対合性を用いて,
    \[u(x)=v\paren{\frac{x}{\abs{x}^2}}=\frac{x_1}{\abs{x}^2}.\]
\end{Proof}

\section{全空間上の基本解}

\begin{tcolorbox}[colframe=ForestGreen, colback=ForestGreen!10!white,breakable,colbacktitle=ForestGreen!40!white,coltitle=black,fonttitle=\bfseries\sffamily,
title=]
    $\R^n$上のPoisson方程式は,境界条件がないので,外力条件$f\in C_c^1(\R^n)$のみがデータである.
    基本解$\Phi$との畳み込み$\Phi*:C_c^1(\R^n)\to C(\R^n)$は$-\Lap u=f\;\on\R^n$の解作用素である.
    解作用素は負のLaplace作用素$-\Lap:u\mapsto f$の逆である.
\end{tcolorbox}

\subsection{Poisson方程式の基本解の定義と性質}

\begin{tcolorbox}[colframe=ForestGreen, colback=ForestGreen!10!white,breakable,colbacktitle=ForestGreen!40!white,coltitle=black,fonttitle=\bfseries\sffamily,
title=]
    $D^2\Phi(x-y)$が局所可積分でさえないため,議論が込み入る.
\end{tcolorbox}

\begin{definition}[fundamental solution / Newton Kernel]
    Poisson方程式の\textbf{基本解}$\Phi:\R^n\to\R\cup\{-\infty\}$を
    \[\tcboxmath{\Phi(x):=\begin{cases}
        -\frac{1}{2\pi}\log\abs{x},&n=2,\\
        \frac{1}{n(n-2)\om_n}\frac{1}{\abs{x}^{n-2}},&n\ge3.
    \end{cases}}\]
    と,基本調和関数の定数倍で定める.
\end{definition}

\begin{proposition}[微分に関する性質]\label{prop-derivative-of-fundamental-solution}
    任意の$n\ge2$次元の基本解$\Phi:\R^n\to\R\cup\{-\infty\}$について,
    \begin{enumerate}
        \item 原点を極とし,$\R^n\setminus\{0\}$上で調和な関数である.
        \item 基本調和関数の定数倍であるため,$\R^n\setminus\{0\}$上球対称である.特に,$\Phi'(y)$は$y\ne0$において$\frac{1}{\abs{y}^{n-1}}$の定数倍であることが必要なのであった\ref{prop-necessary-condition-for-sphere-symmetric-harmonic-functions}.
        \item 特に,第$i$変数に関する微分は原点を除いて定義され,
        \[\tcboxmath{\Phi_{x_i}(y)=-\frac{1}{n\om_n}\frac{y_i}{\abs{y}^n}\quad y\ne0.}\]
        と表せる.超関数の意味ではこれが微分である.
        \item 勾配
        \[D\Phi(y)=-\frac{1}{n\om_n}\frac{y}{\abs{y}^{n}}\quad y\ne0.\]
        とHesse行列
        \[\tcboxmath{D_{ij}\Phi(y)=-\frac{1}{n\om_n}\frac{\abs{y}^2\delta_{ij}-ny_iy_j}{\abs{y}^{n+2}}.}\]
        とは,次のように評価できる:
        \[\abs{D\Phi(x)}\le\frac{C_1}{\abs{x}^{n-1}},\quad\abs{D^2\Phi(x)}\le\frac{C_2}{\abs{x}^n},\quad \paren{x\ne0,C_1=\frac{1}{n\om_n},C_2=\frac{1}{\om_n}}.\]
        特に,Hesse行列は原点の近傍で可積分ではなく,局所可積分でさえない!
        \item 一般階数の微分については,
        \[\abs{D^\al\Phi(x)}\le\frac{C}{\abs{x}^{n-2+\abs{\al}}},\quad(C>0).\]
    \end{enumerate}
\end{proposition}

\begin{corollary}[基本解の係数の設計意図]\label{lemma-design-of-fundamental-solution-of-Poisson-equation}
    内向き単位法線方向の微分係数が,その点を通る原点中心の球面の面積の逆数に等しくなる:
    \[\pp{\Phi}{\nu}(x)=\frac{1}{n\om_n}\frac{1}{\abs{x}^{n-1}}=\frac{1}{\abs{\partial B(0,\abs{x})}}.\]
\end{corollary}
\begin{Proof}
    点$y\in\R^n$における外向き法線とは,ベクトル$-\frac{y}{\abs{y}}$である.よって,
    \[\pp{\Phi}{\nu}(x)=-D\Phi(x)\cdot\frac{y}{\abs{y}}=\frac{1}{n\om_n}\frac{\abs{y}^2}{\abs{y}^{n+1}}=\frac{1}{n\om_n}\frac{1}{\abs{y}^{n-1}}.\]
    ここで,$n\om_n$が$n$次元球の表面積であるから,その$\abs{y}^{n-1}$倍は$B(0,\abs{y})$の表面積である.
\end{Proof}

\begin{proposition}[原点の近傍での積分の評価]\mbox{}
    \begin{enumerate}
        \item $n=2$のとき,
        \[\int_{B(0,\ep)}\abs{\Phi(x)}dx=\frac{\ep^2\abs{\log\ep}}{2}-\frac{\ep^2}{4}\le C\ep^2\abs{\log\ep}\]
        \[\int_{\partial B(0,\ep)}\abs{\Phi(x)}dx=\ep\abs{\log\ep}.\]
        \item $n\ge3$のとき,
        \[\int_{B(0,\ep)}\Phi(x)dx=\frac{n-1}{n-2}\frac{1}{2\sqrt{\pi}}\frac{\ep^2}{2}\le C\ep^2.\]
        \[\int_{\partial B(0,\ep)}\Phi(x)dx=\frac{n-1}{n-2}\frac{1}{2\sqrt{\pi}}\ep.\]
    \end{enumerate}
\end{proposition}
\begin{Proof}\mbox{}
    \begin{enumerate}
        \item \begin{align*}
            \int_{B(0,\ep)}\frac{1}{2\pi}\abs{\log\abs{x}}dx&=\int^\ep_0r\abs{\log r}dr\\
            &=\Square{\frac{r^2\log r}{2}}^\ep_0-\int^\ep_0\frac{r}{2}dr=\frac{\ep^2\abs{\log\ep}}{2}-\frac{\ep^2}{4}.
        \end{align*}
        \item \begin{align*}
            \int_{B(0,\ep)}\frac{1}{n(n-2)\om_n}\frac{1}{\abs{x}^{n-2}}dx&=\frac{1}{n(n-2)\om_n}\int_0^\ep\frac{(n-1)\om_{n-1}r^{n-1}}{r^{n-2}}dr\\
            &=\frac{n-1}{n(n-2)}\frac{\om_{n-1}}{\om_n}\Square{\frac{r^{2}}{2}}^\ep_0.
            \end{align*}
    \end{enumerate}
\end{Proof}

\subsection{大域的Poisson方程式の解公式}

\begin{tcolorbox}[colframe=ForestGreen, colback=ForestGreen!10!white,breakable,colbacktitle=ForestGreen!40!white,coltitle=black,fonttitle=\bfseries\sffamily,
title=]
    基本解が$x=0$の周りで特異性を持つために,
    ここの近傍だけ分離して積分を評価する必要がある.
    このときのために,$f\in C_c^2(\R^n)$を仮定すると評価が本質的に基本解$\Phi$の評価に依拠し,証明が簡単になる.
    原点から離れた部分での積分の評価は,2回の部分積分\ref{cor-partial-integral}を通じて$\Lap\Phi=0$の項を作り出すことによる,部分積分によるもう一項は積分領域がコンパクトであるために,簡単に評価出来るのである.
\end{tcolorbox}

\begin{problem}
    Poisson方程式
    \[\Lap u=f,\qquad \In\R^n,f\in C_c^2(\R^n)\]
    を考える.
\end{problem}

\begin{theorem}[基本解は畳み込みによって解を与える]\label{thm-solution-to-Poisson-equation}
    非斉次項$f\in C^2_c(\R^n)$に対して,
    \[u:=\Phi*f=\int_{\R^n}\Phi(x-y)f(y)dy\]
    とする.このとき,
    \begin{enumerate}
        \item $u\in C^2(\R^n)$.
        \item $u$は$n\ge3$ならば$u\in C_0(\R^n)$,特に有界でもある.
        \item $u$は大域的なPoisson方程式$-\Lap u=f\;\on\R^n$を満たす.
    \end{enumerate}
\end{theorem}
\begin{Proof}\mbox{}
    \begin{enumerate}
        \item Lebesgueの優収束定理は用いず,直接考えることとする.
        \[u(x)=\int_{\R^n}\Phi(x-y)f(y)dy=\int_{\R^n}\Phi(y)f(x-y)dy\]
        の微分であるが,$\Phi$の微分は原点では定義できないので,少なくとも通常の関数の意味では微分を考えたくない.
        そこで,最右辺の表示に注目する.
        $i$番目の標準基底を$e_i\in\R^n$として,
        \[\frac{u(x+he_i)-u(x)}{h}=\int_{\R^n}\Phi(y)\frac{f(x+he_i-y)-f(x-y)}{h}dy\]
        と表せ,この右辺の被積分関数の$\frac{f(x+he_i-y)-f(x-y)}{h}$が$h\to0$について$\R^n$上で一様に$f_{x_i}(x-y)$に収束するためである($f\in C_c$より).
        \item $f$の台を含む任意のコンパクト集合$\supp f\subset K\compsub\R^n$について,
        \[\abs{u(x)}=\Abs{\int_K\Phi(x-y)f(y)dy}\]
        である.このとき被積分関数は,任意の$x\in\R^n$について,
        \[\sup_{y\in\R^n}\abs{\Phi(x-y)f(y)}=\sup_{y\in  K}\abs{\Phi(x-y)f(y)}\]
        が成り立っているが,$\Phi$は$\abs{x}$のみの関数だから,この右辺は$\abs{x}$を十分大きく取れば,十分小さく出来る.すなわち,
        一様に$\Phi(x-y)f(y)\xrightarrow{\abs{x}\to\infty}0$.
        よって,Lebesgueの優収束定理より,$\abs{u(x)}\xrightarrow{\abs{x}\to\infty}0$でもある.
        \item 再び$f$の台のコンパクト性より,微分と積分を交換して
        \[\Lap u(x)=\paren{\int_{B(0,\ep)}+\int_{\R^n\setminus B(0,\ep)}}\Phi(z)\Lap_x f(x-z)dz=:I_\ep(x)+J_\ep(x).\]
        と表せる.原点の近傍$I_\ep$については,$\Phi$自体だと可積分であり,それに有界な連続関数$\Lap f$を乗じて居るからHolderから可積分性は大丈夫であるが,$\ep\to0$にしたがって減少することをあとは確認すればよい.
        原点を除いた項$J_\ep$が$f$に収束するはずであるが,$\Lap f$は捉えられないから,部分積分によって微分を$\Phi$に移して,基本解の設計意図\ref{lemma-design-of-fundamental-solution-of-Poisson-equation}を用いる.
        
        実際にそれぞれを評価してみる.
        \begin{description}
            \item[$I_\ep(x)$の評価] 
            この項は$\ep$を十分小さく取ると収束する.評価は,基本解の積分の評価を通じて,
            \[\abs{I_\ep}\le\norm{\Lap_xf(x-y)}_{L^\infty(\R^n)}\int_{B(0,\ep)}\abs{\Phi(y)}dy\le\begin{cases}
                C\ep^2\abs{\log\ep},&n=2,\\
                C\ep^2,&N\ge3.
            \end{cases}\]
            となる.
            \item[$J_\ep(x)$の評価] 
            まず,部分積分\ref{cor-partial-integral}により,
            \begin{align*}
                J_\ep&=\int_{\R^n\setminus B(0,\ep)}\Phi(y)\Lap_yf(x-y)dy\\
                &=-\int_{\R^n\setminus B(0,\ep)}D\Phi(y)\cdot D_yf(x-y)dy+\int_{\partial B(0,\ep)}\Phi(y)D_yf(x-y)\cdot\nu dS(y)=:K_\ep+L_\ep.
            \end{align*}
            の2つに分解する.
            ただしこのとき,$\R^n\setminus B(0,\ep)$の境界としての$\partial B(0,\ep)$の向きを考えれば,$\nu$は内側法線ベクトル$\nu=-\frac{y}{\abs{y}}$であることに注意.
            \begin{enumerate}
                \item するとまず,$L_\ep$は積分領域がコンパクトであるから,$I_\ep$と同様にして基本解の積分の評価を通じて
                \[\abs{L_\ep}\le\norm{Df}_{L^\infty(\R^n)}\int_{\partial B(0,\ep)}\abs{\Phi(y)}dS(y)\le\begin{cases}
                    C\ep\abs{\log\ep},&n=2,\\
                    C\ep,&n\ge3.
                \end{cases}\]
                と評価出来る.
                \item 次に,残った$K_\ep$は再び部分積分を施して$\Lap\Phi$の項を作ると,$\Phi$の$\R^n\setminus\{0\}$上での調和性よりこれ零であるから,
                \begin{align*}
                    K_\ep&=-\int_{\R^n\setminus B(0,\ep)}D\Phi(y)\cdot D_yf(x-y)dy\\
                    &=\int_{\R^n\setminus B(0,\ep)}\Lap\Phi(y)f(x-y)dy-\int_{\partial B(0,\ep)}f(x-y)D\Phi(y)\cdot\nu dS(y)\\
                    &=-\int_{\partial B(0,\ep)}f(x-y)D\Phi(y)\cdot\nu dS(y).
                \end{align*}
                いま,$D\Phi(y)=-\frac{1}{n\om_n}\frac{y}{\abs{y}^n}$より,
                \[D\Phi(y)\cdot\nu=\frac{1}{n\om_n}\frac{1}{\abs{y}^{n-1}}.\]
                これは$\abs{\partial B(0,\ep)}$に等しいから,$f$の連続性の仮定から,
                \[K_\ep=-\dint_{\partial B(x,\ep)}f(y)dS(y)\xrightarrow{\ep\to0}-f(x).\]
            \end{enumerate}
            \item[結論] 以上より,$\ep>0$は任意であったから,$-\Lap u(x)=f(x)$が判る.
        \end{description}
    \end{enumerate}
\end{Proof}
\begin{remark}
    $f\in C^1_c(\R^n)$にも弱められるが,証明は一気に煩雑になる\cite{Gilbarg}.
\end{remark}

\begin{corollary}[有界な解の一意性 (representation formula)]
    $f\in C_c^2(\R^n)\;(n\ge3)$が定める大域的なPoisson方程式$-\Lap u=f\;\on\R^n$の有界な解は
    \[u(x)=\Phi*f(x)+C\quad(C\in\R,x\in\R^n)\]
    という形のものに限る.
\end{corollary}
\begin{Proof}
    $u$を任意の有界な解とすると,Laplacianの線型性より$u-\Phi*f(x)$も有界な調和関数である.
    よってLiouvilleの定理\ref{cor-Liouville}より,$\R^n$上定数である.
\end{Proof}

\begin{remarks}
    結局,基本解を定数だけずらしたものはみな解であり,有界な解はこれに限る.
    なお,$n=2$の場合の基本解は非有界である.
\end{remarks}

\subsection{解公式の一般化}

\subsection{展望:Newtonポテンシャル論の到達点}

\begin{tcolorbox}[colframe=ForestGreen, colback=ForestGreen!10!white,breakable,colbacktitle=ForestGreen!40!white,coltitle=black,fonttitle=\bfseries\sffamily,
title=]
    Dirichlet問題のデータ$f$を連続関数のクラスとした場合の最も精密な結果である.

\end{tcolorbox}

\begin{theorem}[Holder]
    $\Om$を有界領域,$f:\Om\to\R$をHolder連続とする.
    このとき,$f$を密度とするNewtonポテンシャル$u(x):=\int_\Om\Phi(x-y)\rho(y)dy$は$C^2$-級で,Poisson方程式$-\Lap u=\rho$を満たす.
\end{theorem}

\section{Green関数:D-境界値問題の解法}

\begin{tcolorbox}[colframe=ForestGreen, colback=ForestGreen!10!white,breakable,colbacktitle=ForestGreen!40!white,coltitle=black,fonttitle=\bfseries\sffamily,
title=]
    有界領域$U$上のDirichlet問題
    \[\begin{cases}
        -\Lap u=f&\In U,\\
        u=g&\on\partial U.
    \end{cases}\]
    の解は,基本解$\Phi$を修正した関数
    $G(x,y):=\Phi(y-x)-\phi^x(y)$とその導関数$\pp{G}{\nu}$が定める積分変換の和
    \[u(x)=-\int_{\partial U}g(y)\pp{G}{\nu}(x,y)dS(y)+\int_Uf(y)G(x,y)dy,\qquad x\in U\]
    が与える.ここで,畳み込みを,2変数関数を核とした積分変換とに,見方を変えている.
    解作用素が複雑になったためである.
    $G$は対称であるが,境界上の関数との畳み込みを考えるために具体的な形の積分核$K,P$には2変数の定義域は重ならないことになり,ここに畳み込みの残り香がある.
\end{tcolorbox}

\subsection{問題の所在}

\begin{tcolorbox}[colframe=ForestGreen, colback=ForestGreen!10!white,breakable,colbacktitle=ForestGreen!40!white,coltitle=black,fonttitle=\bfseries\sffamily,
title=]
    任意の$u\in C^2(\R^n)$についての大域的な恒等式
    \[u(x)=\int_U\Phi(y-x)(-\Lap u(y))dy\]
    を,有界開集合$U\osub\R^n$上ではどう変形すべきかを考える.
    実は,$\Phi$が無限遠で消えるために
    $\partial U$上の積分の項が消えていただけであるから,この項が本来は存在する.
    この積分変換を陽に記述するのがこの有界領域の場合における真の問題である.
\end{tcolorbox}

\begin{lemma}
    任意の$u\in C^2(\o{U})$の$x\in U$での値は次のように分解できる:
    \[u(x)=\int_{\partial U}\paren{\Phi(y-x)\pp{u}{\nu}(y)-u(y)\pp{\Phi}{\nu}(y-x)}dS(y)-\int_U\Phi(y-x)\Lap u(y)dy.\]
\end{lemma}
\begin{Proof}\mbox{}
    \begin{enumerate}[{Step}1]
        \item 次の関係に注目する:
        \[\int_{U\setminus B(x,\ep)}\Paren{u(y)\Lap\Phi(y-x)-\Phi(y-x)\Lap u(y)}dy=-\int_{U\setminus B(x,\ep)}\Phi(y-x)\Lap u(y)dy.\]
        これは$\Phi(y-x)$の導関数は$y=x$上では定義されないためにこれを積分範囲に入れることは出来ず,しかしこの点を避ければ$\Lap\Phi=0$であるためである.
        \item 左辺をGreenの第二恒等式\ref{cor-Green-identity}を用いて変形する:
        \begin{align*}
            \int_{U\setminus B(x,\ep)}\Paren{u(y)\Lap\Phi(y-x)-\Phi(y-x)\Lap u(y)}dy&=\int_{\partial U-\partial B(x,\ep)}\paren{u(y)\pp{\Phi}{\nu}(y-x)-\Phi(y-x)\pp{u}{\nu}(y)}dS(y).
        \end{align*}
        最後に$\ep\to0$を考えるのであるが,$\partial B(0,\ep)$上での積分である2項は片方は消えて片方は$u(x)$に収束することを議論する.
        \begin{enumerate}
            \item 右辺第二項は,
            \[\Abs{\int_{\partial B(x,\ep)}\Phi(y-x)\pp{u}{\nu}(y)dS(y)}\le\Norm{Du}_{L^\infty(\o{U})}\int_{\partial B(x,\ep)}\abs{\Phi(y-x)}dS(y)=O(\ep)\quad(\ep\to0).\]
            最後の収束は,積分領域の面積が$\abs{\partial B(x,\ep)}=O(\ep^{n-1})$かつ基本解のその上での最大値が$\norm{\Phi(y-x)}_{L^\infty(\partial B(x,\ep))}=O(\ep^{-(n-2)})$であるためである.
            \item 右辺第一項は,基本解の設計意図\ref{lemma-design-of-fundamental-solution-of-Poisson-equation}より,
            \[\int_{\partial B(x,\ep)}u(y)\pp{\Phi}{\nu}(y-x)dS(y)=\dint_{\partial B(x,\ep)}u(y)dS(y)\to u(x)\quad(\ep\to0).\]
        \end{enumerate}
        \item 以上より,Step1の右辺とStep2の右辺を$\ep\to0$の先で繋ぎ合わせると結論を得る.
    \end{enumerate}
\end{Proof}

\begin{observation}[調和関数の和として解作用素を得る]
    上式に登場する値のうち,$\Lap u$の$U$上の値,$u$の$\partial U$上の値はPoisson方程式の初期値問題で与えられる.
    一方で,$\pp{u}{\nu}$の$\partial U$上の値は不明であるため,一般領域上のPoisson方程式を解くには,この項を除去する必要がある.
    まず一般の調和関数$h\in C^2(U)$に対して,
    \[0=\int_{\partial U}\paren{h(y)\pp{u}{\nu}(y)-u(y)\pp{h}{\nu}(y)}dy-\int_Uh\Lap udy.\]
    これを上の補題で得た$u$の表示
    \[u(x)=\int_{\partial U}\paren{\Phi(y-x)\pp{u}{\nu}(y)-u(y)\pp{\Phi}{\nu}(y-x)}dS(y)-\int_U\Phi(y-x)\Lap u(y)dy.\]
    に足し合わせることで,
    $G:=\Phi+h$について
    \[u(x)=\int_{\partial\Om}\paren{u\pp{G}{\nu}-G\pp{u}{\nu}}dS+\int_\Om G\Lap udx.\]
    を得る.
    したがって,$G$を$\partial\Om$上で消えるように$h\in C^2(\Om)$を選べれば,$\pp{u}{\nu}$を消去することに成功することになる.
    この$G$を\textbf{第1種Green関数}ともいう.
\end{observation}

\begin{tbox}{red}{}
    境界上で$\Phi(y-x)$を打ち消す調和関数$\phi^x(y)$を探せ!
    すなわち,$G(x,y)=\Phi(y-x)+\phi^x(y)$は,任意の$x\in U$に対して,任意の$y\in\partial U$で$0$になるようにしたい.
\end{tbox}

\subsection{$U$上のGreen関数}

\begin{tcolorbox}[colframe=ForestGreen, colback=ForestGreen!10!white,breakable,colbacktitle=ForestGreen!40!white,coltitle=black,fonttitle=\bfseries\sffamily,
title=]
    $\partial U$上の積分の項の積分核は$\pp{G}{\nu}$が与える.よって,
    $U$上のPoisson方程式のDirichlet問題を解くことは,Green関数を$U$について構成し,$\pp{G}{\nu}$を求める問題に還元される.
\end{tcolorbox}

\begin{definition}[corrector function, (Dirichlet's) Green function]
    有界領域$U\osub\R^n$について,
    \begin{enumerate}
        \item $x\in U$に対して,次のDirichlet問題(C)の解$\phi^x\in C^2(\o{U})$を\textbf{修正関数}という:
        \[\text{(C)}\quad\begin{cases}
            \Lap\phi^x(y)=0\quad y\in U\\
            \phi^x(y)=\Phi(y-x)\quad y\in\partial U.
        \end{cases}\]
        \item 領域$U$上の\textbf{Green関数}とは,$U^2\setminus\Delta$上の関数
        \[G(x,y):=\Phi(y-x)-\phi^x(y)\quad(x,y\in U,x\ne y)\]
        をいう.これはたしかに,任意の$x\in U$に対して,$y\in\partial U$上全域で消えており,$\Phi$に対して調和関数の分しか変更していない.
    \end{enumerate}
\end{definition}
\begin{remarks}
    これを,次のようにも表す:
    \[\begin{cases}
        -\Lap_y G(x,y)=\delta_x& y\in U,\\
        G(x,y)=0& y\in \partial U.
    \end{cases}\]
\end{remarks}

\begin{lemma}[Green関数による関数の積分表示]\label{lemma-decomposition-of-C2-function-through-Green-function}
    有界領域$U\osub\R^n$上の任意の関数$u\in C^2(\o{U})$について,
    \[\forall_{x\in U}\quad u(x)=-\int_{\partial U}u(y)\pp{G}{\nu}(x,y)dS(y)-\int_UG(x,y)\Lap u(y)dy.\]
    ただし,$\pp{G}{\nu}(x,y)=D_yG(x,y)\cdot\nu(y)$は$G$の第二変数$y$に関する外側単位法線ベクトルに沿った微分とする.
\end{lemma}
\begin{Proof}
    $\phi^x$は$y\in U$上調和であるから,
    Greenの第二恒等式\ref{cor-Green-identity}より,
    \[0=\int_{\partial U}\paren{\phi^x(y)\pp{u}{\nu}(y)-u(y)\pp{\phi^x(y)}{\nu}(y)}dy-\int_U\phi^x(y)\Lap udy.\]
    これと,前の補題の結果
    \[u(x)=\int_{\partial U}\paren{\Phi(y-x)\pp{u}{\nu}(y)-u(y)\pp{\Phi}{\nu}(y-x)}dS(y)-\int_U\Phi(y-x)\Lap u(y)dy.\]
    の辺々を引くことで,$G(x,y)=\Phi(y-x)-\phi^x(y)$について
    \[u(x)=\int_{\partial U}\paren{G(x,y)\pp{u}{\nu}(y)-u(y)\pp{G(x,y)}{\nu}}dy-\int_UG(x,y)\Lap u(y)dy.\]
    であるが,$G(x,y)$は$\partial U$上で零なので,この項は消える.
\end{Proof}

\subsection{Green関数による解公式}

\begin{tcolorbox}[colframe=ForestGreen, colback=ForestGreen!10!white,breakable,colbacktitle=ForestGreen!40!white,coltitle=black,fonttitle=\bfseries\sffamily,
title=]
    特に,任意の調和関数は,その境界値に$\pp{G}{\nu}$を核とする積分変換を施すことで内部での値を復元することができる.
\end{tcolorbox}

\begin{problem}[境界を持つ有界領域上のPoisson方程式のDirichlet問題の解法]
    $U\osub\R^n$を有界開集合,$\partial U$を$C^1$-級であるとし,次のPoisson方程式$-\Lap u=f$のDirichlet問題を考える:
    \[\text{(D)}\quad\begin{cases}
        -\Lap u=f\quad\mathrm{in}\;U\\
        u=g\quad\on\;\partial U.
    \end{cases}\]
\end{problem}

\begin{corollary}[Green関数によるDirichlet問題の解の積分表示]\label{cor-solution-via-Green-function}
    $u\in C^2(\o{U})$がDirichlet問題(D)の解ならば,
    \[\forall_{x\in U}\quad u(x)=-\int_{\partial U}g(y)\pp{G}{\nu}(x,y)dS(y)+\int_Uf(y)G(x,y)dy.\]
\end{corollary}
\begin{Proof}
    $U$上で$\Lap u=f$であり,$\partial U$上で$u=g$であるため.
\end{Proof}

\begin{remarks}
    熱方程式の基本解と解釈しているものを,積分核$\pp{G}{\nu}$として畳み込みでないことばで与えている.
    これを熱核と同様に名前を与えて\textbf{Poisson核}という.
    非斉次項はもっと簡単で,Green関数$G$と畳み込むだけである.
    熱核は境界条件と畳み込むが,Laplace方程式の基本解は非斉次項と畳み込む,ここが違う.
\end{remarks}

\begin{proposition}[解の存在の十分条件]
    $\Om$を有界とする.
    有界関数$f\in L^\infty(\Om)$に対して,
    \begin{enumerate}
        \item 次の$v$は$C^1(\Om)$の元になる:
        \[v(x):=\int_\Om G(x,y)f(y)dy.\]
        \item 任意の$\xi\in\Om$に対して,
        \[\lim_{\Om\ni x\to\xi}\int_\Om G(x,y)f(y)dy=0.\]
        \item さらに$f\in C^1(\Om)$も満たすとき,$v\in C^2(\Om)$で,$-\Lap v=f\;\In\Om$である.
    \end{enumerate}
\end{proposition}

\subsection{有界領域上のGreen関数の対称性}

\begin{theorem}[Green関数は対称である]
    $U\osub\R^n$を有界領域とする.
    任意の$(x,y)\in U^2\setminus\Delta$について,$G(y,x)=G(x,y)$.
\end{theorem}
\begin{Proof}\mbox{}
    \begin{description}
        \item[証明の方針] 任意に$x,y\in U,x\ne y$を取り,
        \[v(z):=G(x,z)=\Phi(z-x)-\phi^x(z),\quad w(z):=G(y,z)=\Phi(z-y)-\phi^y(z)\]
        とおいて,$v(y)=w(x)$を示せば良い.
        \item[問題の所在] いま,$\Lap v=0\;\on U\setminus\{x\},\Lap w=0\;\on U\setminus\{y\},w=v=0\;\on\partial U$であるから,$V:=U\setminus(B(x,\ep)\cup B(y,\ep))$上のGreenの恒等式\ref{cor-Green-identity}(2)より,
        \[\int_{\partial V}\paren{v\pp{w}{\nu}-w\pp{v}{\nu}}dS=0\]
        \[\therefore\quad\int_{\partial B(x,\ep)}\paren{w\pp{v}{\nu}-v\pp{w}{\nu}}dS(z)=\int_{\partial B(y,\ep)}\paren{v\pp{w}{\nu}-w\pp{v}{\nu}}dS.\]
        この左辺が$w(x)$に,右辺が$v(y)$に収束することを示す.
        \item[評価] 左辺が$w(x)$に収束することを示す.
        \begin{enumerate}
            \item 左辺第2項は消える.実際,$w(z)=\Phi(z-y)-\phi^y(z)$は$x$の近傍で滑らかであるからその近傍で有界で,$v(z)=G(z-x)-\phi^x(z)$は$G(z-x)$の部分が$x$を特異点に持つが,これは$x$からの距離$\ep$の$\ep^{-(n-2)}$の関数であるから,
            \[\Abs{\int_{\partial B(x,\ep)}\pp{w}{\nu}vdS}\le C\ep^{n-1}\sup_{\partial B(x,\ep)}\abs{v}=O(\ep)\quad(\ep\to0).\]
            \item 左辺第1項は$w(z)$に収束する.
            実際,$\Phi$の設計意図\ref{lemma-design-of-fundamental-solution-of-Poisson-equation}より,
            \begin{align*}
                \int_{\partial B(x,\ep)}\pp{v}{\nu}wdS&=\int_{\partial B(x,\ep)}\pp{\Phi}{\nu}(x-z)w(z)dS(z)-\int_{\partial B(x,\ep)}\pp{\phi^x}{\nu}(z)w(z)dS(z)\\
                &=\dint_{\partial B(x,\ep)}w(z)dS(z)+O(\ep)\xrightarrow{\ep\to0}w(x).
            \end{align*}
        \end{enumerate}
    \end{description}
\end{Proof}

\begin{proposition}[Green関数は正値である]
    $U\osub\R^n$を有界領域である.
    任意の$(x,y)\in U^2\setminus\Delta$について,
    $G(x,y)>0$.
\end{proposition}

\begin{proposition}[一般の有界領域上のPoisson方程式の解]
    $\Om$を有界とする.
    有界関数$f\in L^\infty(\Om)$に対して,
    \begin{enumerate}
        \item 次の$v$は$C^1(\Om)$の元になる:
        \[v(x):=\int_\Om G(x,y)f(y)dy.\]
        \item 任意の$\xi\in\Om$に対して,
        \[\lim_{\Om\ni x\to\xi}\int_\Om G(x,y)f(y)dy=0.\]
        \item さらに$f\in C^1(\Om)$も満たすとき,$v\in C^2(\Om)$で,$-\Lap v=f\;\In\Om$である.
    \end{enumerate}
\end{proposition}

\section{Green関数の構成とLaplace方程式のD-境界値問題の解公式}

\begin{tcolorbox}[colframe=ForestGreen, colback=ForestGreen!10!white,breakable,colbacktitle=ForestGreen!40!white,coltitle=black,fonttitle=\bfseries\sffamily,
title=]
    \begin{enumerate}
        \item 一般領域$U$上のDirichlet問題(D)の解決は,
        修正問題
        \[\text{(C)}\quad\begin{cases}
            \Lap\phi^x=0\quad\mathrm{in}\; U\\
            \phi^x=\Phi(y-x)\quad\on\;\partial U.
        \end{cases}\]
        を解けば,あとは$G(x,y):=\Phi(y-x)-\phi^x(y)$を核として非斉次項$f$を積分し,$\pp{G}{\nu}$を核として境界条件$g$を積分
        して得る解の畳み込みとして,一般解が得られる.
        \item これは幾何学的な問題である.次のような鏡映変換$x\mapsto\wt{x}$に対して,$\phi^x(y):=\Phi(y-\wt{x})$とすればよい
        \begin{enumerate}
            \item $\Phi$の特異点は$\o{U}$の外にあり,$\phi^x$は$y\in U$上で調和である.
            \item 特異点の移動先$\wt{x}$は,$\partial U$からみて$x$と等距離にある.
        \end{enumerate}
        すると,任意の$y\in\partial U$について
        \[\phi^x(y)=\Phi(y-\wt{x})=\Phi(y-x).\]
        となる.基本解は距離の関数であることに注意.
        \item こうして得たGreen関数$G(x,y):=\Phi(y-x)-\Phi(r\abs{y-\wt{x}})$の境界$\partial U$上での法線方向微分
        \[K(x,y):=-\pp{G}{\nu}(x,y)\qquad(x,y)\in U\times\partial U\]
        を\textbf{Poisson核}といい,この核が定める積分変換が境界条件に対する解作用素$C_b(\partial U)\to C^\infty(U)$を与える.
    \end{enumerate}
\end{tcolorbox}

\subsection{上半平面のPoisson核}

\begin{tcolorbox}[colframe=ForestGreen, colback=ForestGreen!10!white,breakable,colbacktitle=ForestGreen!40!white,coltitle=black,fonttitle=\bfseries\sffamily,
title=]
    前節の一般論と異なり,$\R_+^n$は非有界であるが,全く同様の論法が通り,
    $-\pp{G}{\nu}(x,y)=K(x,y)$の関係を得るので,
    境界情報$g$をPoisson核$K$で畳み込むことでLaplace方程式のDirichlet問題の解を得る.
\end{tcolorbox}

\begin{notation}
    $\R^n_+:=\R^{n-1}\times\R^+$とする.
\end{notation}

\begin{definition}[上半平面上のGreen関数]\mbox{}
    \begin{enumerate}
        \item $x\in\R^n_+$の$\partial\R^n_+$に関する\textbf{鏡映}とは,$\wt{x}=(x_1,\cdots,x_{n-1},-x_n)$をいう.
        \item \textbf{上半平面上の修正関数}を鏡映によって
        \[\phi^x(y):=\Phi(y-\wt{x})=\Phi(y_1-x_1,\cdots,y_{n-1}-x_{n-1},y_n+x_n)\quad(x,y\in\R^n_+)\]
        で定める.これはたしかに次の問題(C)の答えである:
        \[\text{(C)}\quad\begin{cases}
            \Lap\phi^x(y)=0\quad\In\;\R^n_+,\\
            \phi^x(y)=\Phi(y-x)\quad\on\partial\R^n_+.
        \end{cases}\]
        \item \textbf{上半平面上のGreen関数}を
        \[G(x,y):=\Phi(y-x)-\Phi(y-\wt{x})\quad(x,y\in\R^n_+,x\ne y)\]
        で定める.
    \end{enumerate}
\end{definition}
\begin{Proof}
    $\phi^x(y):=\Phi(y-\wt{x})$は,任意の$y\in\R^n_+$については$y-\wt{x}\ne0$より$\Phi$の特異点を通らないから$\Lap_y\Phi(y-\wt{x})=0$で,
    任意の$y\in\partial\R^n_+$については,$\Phi$の球対称性より,
    \[\Phi(y-\wt{x})=\Phi\paren{\begin{pmatrix}y_1-x_1\\\vdots\\y_{n-1}-x_{n-1}\\x_n\end{pmatrix}}=\Phi\paren{\begin{pmatrix}y_1-x_1\\\vdots\\y_{n-1}-x_{n-1}\\-x_n\end{pmatrix}}=\Phi(y-x)\]
    が成り立つから,確かに問題(C)の解である.
\end{Proof}

\begin{lemma}[Green関数の境界上での外側法線ベクトルに関する微分]\mbox{}
    \begin{enumerate}
        \item Green関数の$y_n$に関する微分は
        \[G_{y_n}(x,y)=\frac{1}{n\om_n}\frac{2x_n}{\abs{y-x}^n}.\]
        \item 境界上での外側法線ベクトルに関する微分は
        \[\forall_{y\in\partial\R^n_+}\quad\pp{G}{\nu}(x,y)=-G_{y_n}(x,y)=-\frac{2x_n}{n\om_n}\frac{1}{\abs{x-y}^n}=:-K(x,y).\]
    \end{enumerate}
\end{lemma}
\begin{Proof}\mbox{}
    \begin{enumerate}
        \item 基本解の微分\ref{prop-derivative-of-fundamental-solution}に代入すれば,
        \[G_{y_n}(x,y)=\Phi_{y_n}(y-x)-\Phi_{y_n}(y-\wt{x})=-\frac{1}{n\om_n}\paren{\frac{y_n-x_n}{\abs{y-x}^n}-\frac{y_n+x_n}{\abs{y-\wt{x}}^n}}.\]
        あとは,$y,x,\wt{x}$は原点を中心とした同一球面上に存在しているために,$\abs{y-x}^n=\abs{y-\wt{x}}^n$に注意すれば良い.
        \item 境界$y\in\R^n_+$での外側法線とは,$n$番目の標準基底$-e_n$であるから,$\pp{G}{\nu}=(-e_n)\cdot DG=-G_{y_n}$より.
    \end{enumerate}
\end{Proof}

\begin{definition}[上半平面に対するPoisson核が与えるPoissonの公式]
    これで積分核$K$を
    \[K(x,y):=-\pp{G}{\nu}(x,y)=G_{y_n}(x,y)=\frac{2}{n\om_n}\frac{x_n}{\abs{x-y}^n}\]
    と定めたことになる.
    そして,上半平面上の境界値$g\in C_b(\R^{n-1})=C_b(\partial\R^n_+)$に対して,$K$を核とした積分変換を施せば,これがLaplace方程式の解$\Lap u=0\;\In\R^n_+$を定める.
    これによる調和関数の内部での値の復元公式をPoisson公式という.
\end{definition}

\begin{lemma}[上半平面のPoisson核は調和な総和核]\label{lemma-Poisson-Kernel-on-upper-half-plane}
    Poisson核
    \[K(x,y)=\frac{2}{n\om_n}\frac{x_n}{\abs{x-y}^n}=\frac{2}{n\om_n}\frac{x_n}{\Paren{(x_1-y_1)^2+\cdots+(x_{n-1}-y_{n-1})^2+x_n^2}^{n/2}},\qquad(x,y)\in\R^n_+\times\partial\R^n_+.\]
    について,
    \begin{enumerate}
        \item $\R^n_+\times\partial\R^n_+$上の二変数関数と見れば,特異点を持たず,いずれの変数に関しても調和である.
        \item 任意の$x\in\R^n_+$に対して,$y\mapsto K(x,y)$は$\partial\R^n_+$上の確率密度である:
        \[1=\int_{\partial\R^n_+}K(x,y)dy,\quad x\in\R^n_+.\]
        実は$x_n>0$を添え字とする$x_n\to0$に関する総和核となる.
    \end{enumerate}
\end{lemma}
\begin{Proof}\mbox{}
    \begin{enumerate}
        \item 明らか.
        \item direct computationらしい.
        総和核をなすことは,$P_1(x):=\frac{1}{(1+\abs{x}^2)^{n/2}}$に対して関係
        \[P_{x_n}(y)=x_n^{-(n-1)}P_1(y/x_n)\]
        という関係が成り立つことによる.
    \end{enumerate}
\end{Proof}
\begin{remarks}[Poisson総和核としての表式]\mbox{}\label{remark-Poisson-kernel}
    \begin{enumerate}
        \item $K(x,y)\;((x,y)\in\R^n_+\times\partial\R^n_+)$は$z:={}^t\!(x_1-y_1,\cdots,x_{n-1}-y_{n-1}),t:=x_n$とおくと$x-y=(z,t)\in\R^n$と見れて,
        $K$はこの$(z,t)\in\R^n_+$上の関数とも見れるから,
        \[K(x,y)=\frac{2}{n\om_n}\frac{x_n}{\abs{x-y}^n}=\frac{2}{n\om_n}\frac{t}{(t^2+\norm{z}^2)^{n/2}}=:P(t,z)\]
        とも表示できる.
        \item 特に$n=2$の場合は,
        \[P(t,z)=P_t(x)=\frac{1}{\pi}\frac{t}{t^2+z^2}\quad(z\in\R,t>0).\]
        \item $P_t$は$e^{-\abs{\xi}}$という形の関数のFourier変換像である.任意の$t>0$に対して,
        \[P(t,z)=\F[e^{-2\pi  t\abs{\xi}}](x)=\int_{\R^n}e^{-2\pi t\abs{\xi}}e^{-2\pi i\xi\cdot x}d\xi.\]
        が成り立つ.よって,任意の$t>0$について
        \[\wh{P}_t(z)=e^{-t\abs{z}}\]
        を満たす.
    \end{enumerate}
\end{remarks}

\subsection{非有界性に対する証明}

\begin{tcolorbox}[colframe=ForestGreen, colback=ForestGreen!10!white,breakable,colbacktitle=ForestGreen!40!white,coltitle=black,fonttitle=\bfseries\sffamily,
title=]
    Poisson核$K(x,y)$は$P_{x_n}({}^t\!(x_1-y_1,\cdots,x_{n-1}-y_{n-1}))$とみると総和核になるという性質\ref{lemma-Poisson-Kernel-on-upper-half-plane}により,解作用素を与えることになる.
\end{tcolorbox}

\begin{theorem}[上半平面上のLaplace方程式のDirichlet問題の解]
    有界連続な境界条件$g\in C(\R^{n-1})\cap L^\infty(\R^{n-1})$に対して,
    \[u(x):=\int_{\partial\R^n_+}K(x,y)g(y)dy=\frac{2x_n}{n\om_n}\int_{\partial\R^n_+}\frac{g(y)}{\abs{x-y}^n}dy\quad(x\in\R^n_+)\]
    と定めると,次が成り立つ:
    \begin{enumerate}
        \item $u\in C^\infty(\R^n_+)\cap L^\infty(\R^n_+)$.
        \item $\Lap u=0\;\on\R^n_+$.
        \item $\forall_{x^0\in\partial\R^n_+}\;\lim_{\R^n_+\ni x\to x^0}u(x)=g(x^0)$.
    \end{enumerate}
\end{theorem}
\begin{Proof}\mbox{}
    \begin{enumerate}
        \item $g$は有界とした.任意の$x\in\R^n_+$に対して$y\mapsto K(x,y)$は調和だから,$u$は再び有界で$C^\infty(\R^n_+)$-級である.
        \item 次の積分と微分の交換
        \[\Lap u(x)=\int_{\partial\R^n_+}\Lap_xK(x,y)g(y)dy=0,\quad x\in \R^n_+.\]
        は$K$の調和性と$g$の有界性より成功する.
        \item 任意の境界点$x^0\in\partial\R^n_+$と$\ep>0$を取る.
        まず$g$の連続性より,$\exists_{\delta>0}\;\forall_{y\in\partial\R^n_+}\;\abs{y-x^0}<\delta\Rightarrow\abs{g(y)-g(x^0)}<\ep$.
        \begin{description}
            \item[問題の所在] 
            このとき,$K(x,-)$は$\partial\R^n_+$上の確率密度だから,任意の$x\in B(x^0,\delta/2)$に対して,
            \begin{align*}
                \abs{u(x)-g(x^0)}&=\Abs{\int_{\partial\R^n_+}K(x,y)(g(y)-g(x^0))dy}\\
                &\le\int_{\partial\R^n_+\cap B(x^0,\delta)}K(x,y)\abs{g(y)-g(x^0)}dy+\int_{\partial\R^n_+\setminus B(x^0,\delta)}K(x,y)\abs{g(y)-g(x^0)}dy\\
                &=:I+J
            \end{align*}
            と分解できる.すると$I$は$y\in B(x^0,\delta)$を満たして居るから,
            \[I\le\ep\int_{\partial\R^n_+}K(x,y)dy=\ep.\]
            問題は$J$である.
            \begin{align*}
                J&\le2\norm{g}_\infty\int_{\partial\R^n_+\setminus B(x^0,\delta)}K(x,y)dy\\
                &=\frac{2^2\norm{g}_\infty x_n}{n\om_n}\int_{\partial\R^n_+\setminus B(x^0,\delta)}\frac{1}{\abs{x-y}^n}dy
            \end{align*}
            までは進む.
            いま,$x\in B(x^0,\delta/2)$であるが,$y\notin B(x^0,\delta)$と取ってある.
            \item[非有界領域上の積分項$J$の評価] いま少し考慮を要するが,$\abs{y-x}\ge\frac{1}{2}\abs{y-x^0}$なる位置関係を満たしている.
            よって,
            \[J\le\frac{2^{2+n}\norm{g}_\infty x_n}{n\om_n}\int_{\partial\R^n_+\setminus B(x^0,\delta)}\frac{1}{\abs{y-x^0}^n}dy\]
            と有界係数による評価が得られ,この右辺は$x_n\to0$のとき$0$に収束する.
        \end{description}
    \end{enumerate}
\end{Proof}

\subsection{単位球上のPoisson核}

\begin{definition}[単位球上のGreen関数]\mbox{}
    \begin{enumerate}
        \item $x\in\R^n\setminus\{0\}$の球面$\partial B(0,R)$に関する\textbf{反転}とは,$\wt{x}:=\frac{R^2}{\abs{x}^2}x$をいう.
        ただし$0\mapsto\infty$とする.
        変換$x\mapsto\wt{x}$を\textbf{反転}という.
        \item 単位球上の修正関数を反転によって$\phi^x(y):=\Phi(\abs{x}(y-\wt{x}))\;(x\ne0)$と定める.ただし,$\phi^0(y)=\Phi(1)$とする.
        \item \textbf{単位球上のGreen関数}を\[G(x,y):=\Phi(y-x)-\Phi(\abs{x}(y-\wt{x}))\quad(x,y\in B(0,1),x\ne y).\]
        とする.
    \end{enumerate}
\end{definition}
\begin{lemma}
    単位球面$B(0,1)$について,
    \begin{enumerate}
        \item 修正関数
        \[\phi^x(y)=\Phi(\abs{x}(y-\wt{x}))=\begin{cases}
            -\frac{1}{2\pi}\log\abs{x}\abs{y-\wt{x}}&n=2,\\
            \frac{1}{n(n-2)\om_n}\frac{1}{\abs{x}^{n-2}}\frac{1}{\abs{y-\wt{x}}^{n-2}}&n\ge3.
        \end{cases}\quad(x\ne0)\]
        はたしかに修正問題
        \[\text{(C)}\quad\begin{cases}
            \Lap\phi^x(y)=0\quad\mathrm{\in}\;B^\circ(0,1),\\
            \phi^x(y)=\Phi(y-x)\quad\on\partial B(0,1).
        \end{cases}\]
        の解である.
        $\phi^0(y)=\Phi(1)$は明らかに解である.
        \item Green関数$G(x,y)=\Phi(y-x)-\Phi(\abs{x}(y-\wt{x}))$の$\partial B(0,1)$上での第二変数の第$i$変数に関する微分は
        \[G_{y_i}(x,y)=-\frac{1}{n\om_n}\frac{1}{\abs{x-y}^n}\paren{(y_i-x_i)-(y_i\abs{x}^2-x_i)}=\frac{(1-\abs{x}^2)}{n\om_n}\frac{y_i}{\abs{x-y}^n}.\]
        \item Green関数の境界上での外側法線ベクトルに関する微分は
        \[\forall_{y\in\partial B(0,1)}\quad-\pp{G}{\nu}(x,y)=\frac{1}{n\om_n}\frac{1-\abs{x}^2}{\abs{x-y}^n}=:K(x,y).\]
    \end{enumerate}
\end{lemma}
\begin{Proof}\mbox{}
    \begin{enumerate}
        \item $x\ne0$のとき,$\wt{x}\notin B^\circ(0,1)$なので,$\phi^x$は明らかに$B^\circ(0,1)$上調和である.
        さらに境界点$y\in\partial B(0,1)$に関しては,$\norm{x}^{n-2}\norm{y-\wt{x}}^{n-2}=\norm{x-y}^{n-2}$であるために,$\phi^x(y)=\Phi(y-x)\;\on \partial B(0,1)$が成り立つ.
        というのも,
        \begin{align*}
            \norm{x}^2\norm{y-\wt{x}}^2&=\norm{x}^2\paren{\norm{y}^2-\frac{2y\cdot x}{\norm{x}^2}+\frac{1}{\norm{x}^2}}\\
            &=\norm{x}^2-2y\cdot x+1=\norm{x-y}^2
        \end{align*}
        が成り立つためである.
        \item 基本解の微分\ref{prop-derivative-of-fundamental-solution}より,
        \[\pp{\Phi}{y_i}(y-x)=-\frac{1}{n\om_n}\frac{y_i-x_i}{\abs{x-y}^n}\]
        さらに,$\phi^x$の微分は合成関数の微分則より,
        \[\pp{\phi^x}{y_i}(y)=\Phi_{y_i}(\abs{x}(y-\wt{x}))\abs{x}=-\frac{1}{n\om_n}\frac{\abs{x}(y_i-x_i/\abs{x}^2)}{\abs{x}^n\abs{y-\wt{x}}^n}\abs{x}=-\frac{1}{n\om_n}\frac{y_i\abs{x}^2-x_i}{\abs{x}^n\abs{y-\wt{x}}^n}.\]
        さらに,$y\in\partial B(0,1)$上では$\abs{x}^n\abs{y-\wt{x}}^n=\abs{x-y}^n$に注意.
        \item (2)より,
        \begin{align*}
            \pp{G}{\nu}(x,y)&=\sum_{i\in[n]}y_iG_{y_i}(x,y)\\
            &=-\frac{1}{n\om_n}\frac{1-\abs{x}^2}{\abs{x-y}^n}\sum_{i\in[n]}y_i^2=-\frac{1}{n\om_n}\frac{1-\abs{x}^2}{\abs{x-y}^n}
        \end{align*}
    \end{enumerate}
\end{Proof}
\begin{remark}
    一般の$r$の場合は,$u$を調和関数としたときのGreen関数による解公式\ref{cor-solution-via-Green-function}を通じて,$\xi=rx$とすると
    \begin{align*}
        u(\xi)&=\frac{1-\abs{\xi/r}^2}{n\om_n}\int_{\partial B(0,1)}\frac{u(y)}{\abs{\xi/r-y}^n}dS(y)\\
        &=\frac{1}{r^2}\frac{r^2-\abs{\xi}^2}{n\om_n}r^n\int_{\partial B(0,r)}\frac{u(\eta/r)}{\abs{\xi-\eta}^n}\frac{1}{r^{n-1}}dS(\eta)\\
        &=\frac{r^2-\abs{\xi}^2}{n\om_nr}\int_{\partial B(0,r)}\frac{g(\eta/r)}{\abs{\xi-\eta}^n}dS(\eta).
    \end{align*}
    を満たす必要があるところから,Poisson核は次のように定めれば良い:
\end{remark}

\begin{definition}[球上のPoisson核が与えるPoissonの公式]
    \textbf{球$B(0,r)$上のPoisson核}を
    \[K(x,y):=\frac{r^2-\abs{x}^2}{n\om_nr}\frac{1}{\abs{x-y}^n}\quad(x\in B^\circ(0,r),y\in\partial B(0,r))\]
    で定める.
    球面上の境界値$g\in C_b(\partial B(0,R))$に対して,$K$を核とした積分変換を施せば,これがLaplace方程式の解$\Lap u=0\In B(0,R)$を定める.
    これによる調和関数の内部での値の復元公式をPoisson公式という.
\end{definition}
\begin{remarks}[Poisson総和核としての表示]\mbox{}
    \begin{enumerate}
        \item $K(x,y)\;((x,y)\in B(0,r)\times\partial B(0,r))$は$P(z,\zeta)$とも表し,これが定める積分変換
        \[P[u](z)=\int_{\partial B(0,r)}u(\xi)P(z,\zeta)dS(\zeta)\]
        は\textbf{Poisson積分}と呼ばれる.
        \item 特に$n=2$の場合は
        \[P(z,\zeta)=\frac{R^2-\abs{z}^2}{2\pi R}\frac{1}{\abs{z-\zeta}^2}\qquad(z\in B(0,R),\zeta\in\partial B(0,R))\]
        となる.これが複素平面上のPoisson積分の形である.
        \item さらに$R=1$とすると,共役調和関数が命題\ref{prop-Poisson-kernel-on-disk-of-C}から,
        \[P(z,\zeta)=\Re\paren{\frac{\zeta+z}{\zeta-z}}=\Re\paren{\frac{1+(z/\zeta)}{1-(z/\zeta)}}\]
        と見つかる.$\abs{\zeta}=1$であるために,これは回転変換を$z$に施していることになる.
        さらに,みやすくするために$\zeta=e^{-i\varphi}$,$r:=\abs{z}$と定めると
        \[P(z,e^{-i\varphi})=\Re\paren{\frac{e^{-i\varphi}+z}{e^{-i\varphi}-z}}=\frac{1-r^2}{1-2r\cos(\arg(z)+\varphi)+r^2}=:P_r(\arg(z)+\varphi)\]
        となる.よって,$r:=\abs{z},\theta:=\arg(z)+\varphi=\arg(z/\zeta)$の二変数関数
        \[P_r(\theta)=\sum_{n\in\Z}r^{\abs{n}}e^{in\theta}=\Re\paren{\frac{1+re^{i\theta}}{1-re^{i\theta}}}\qquad0\le r<1,\theta\in\cointerval{0,2\pi}\]
        と見れることが解る.$(P_r)$は$r\to1$に関して$\bT\simeq\cointerval{0,2\pi}$上の総和核である.
    \end{enumerate}
\end{remarks}

\begin{theorem}\label{thm-Green-function-on-disk}
    $g\in C(\partial B(0,r))$ならば,
    \[u(x):=\int_{\partial B(0,r)}K(x,y)=\frac{r-\abs{x}^2}{n\om_nr}\int_{\partial B(0,r)}\frac{g(y)}{\abs{x-y}^n}dS(y)\quad(x\in B^\circ(0,r))\]
    について,次が成り立つ:
    \begin{enumerate}
        \item $u\in C^\infty(B^\circ(0,r))$.
        \item $\Lap u=0\;\text{in}\;B^\circ(0,r)$.
        \item $\forall_{x^0\in \partial B(0,r)}\;\lim_{B^0(0,r)\ni x\to x^0}u(x)=g(x^0)$.
    \end{enumerate}
\end{theorem}

\section{PerronによるD-境界値問題の解の構成}

\begin{tcolorbox}[colframe=ForestGreen, colback=ForestGreen!10!white,breakable,colbacktitle=ForestGreen!40!white,coltitle=black,fonttitle=\bfseries\sffamily,
title=]
    前節で得たPoissonの積分表示を,Cauchyの積分表示の代わりとして,球を中心とした理論構成が可能になる.
    この手法の一般化により,非線形方程式も含む広い範囲の偏微分方程式に対して,粘性解の存在と一意性が得られる.
\end{tcolorbox}

\subsection{弱劣調和関数}

\begin{definition}[generalized subharmonic function, weakly subharmonic function]
    $\Om\osub\R^n$を領域とする.
    \begin{enumerate}
        \item 連続関数$u\in C(\Om)$が\textbf{弱劣調和}であるとは,
        任意の開球$B\subset\Om$とその上の調和関数$h\in H(B)$について,$u\le h\;\on\partial B$ならば$u\le h\;\In B$が成り立つことをいう.
        \item 可積分関数$u\in L^1(\Om)$が\textbf{超関数の意味で劣調和}であるとは,任意の非負な$\varphi\in C^2_c(\Om)_+$に対して,
        \[\int_\Om u\Lap\varphi dx\ge0\]
        を満たすことをいう.
    \end{enumerate}
\end{definition}

\begin{proposition}
    $\Om\osub\R^n$を領域,$u\in C^2(\Om)$とする.次は同値:
    \begin{enumerate}
        \item $u$は劣調和である.
        \item $u$は弱劣調和である.
        \item $u$は超関数の意味で劣調和である.
    \end{enumerate}
\end{proposition}
\begin{Proof}\mbox{}
    \begin{description}
        \item[(3)$\Rightarrow$(1)] $u\in C^2(\Om)$の軟化列$\{u^\ep:=\rho_\ep*u\}\subset C^\infty(\Om)$を考えると,十分小さい$\ep>0$に対して,$u_\ep$は調和である.$u$はその広義一様収束極限である.
    \end{description}
\end{Proof}

\subsection{弱劣調和関数に対する比較原理}

\begin{proposition}[弱劣調和関数と弱優調和関数の大小関係の消息]
    $\Om\osub\R^n$を有界領域,
    $u,v\in C(\o{\Om})$をそれぞれ弱劣調和,弱優調和とする.
    このとき,$u\le v\;\on\partial\Om$ならば,次のいずれかが成り立つ:
    \begin{enumerate}
        \item $u<v\;\on\Om$である.
        \item $u\equiv v\;\on\o{\Om}$である.
    \end{enumerate}
\end{proposition}
\begin{Proof}\mbox{}
    \begin{description}
        \item[方針] (2)ではないとき,(1)でもないと仮定して矛盾を導く.すなわち,ある内点$x_0\in\Om$において
        \[(u-v)(x_0)=\max_{\o{\Om}}(u-v)=:M\ge0\]
        が成り立ち((2)の否定),さらにある$B(x_0,r)\subset\Om$が存在して,$u-v\not\equiv M\;\on\partial B$である((1)の否定)と仮定する.
        \item[弱劣調和函数の含意] すると,$\partial B$上で$u,v$と同じ値を持つ調和関数をそれぞれ$\o{u},\o{v}$とすると,$u,v$は連続としたから定理\ref{thm-Green-function-on-disk}よりこれは存在し,
        $u\le\o{u}\land\o{v}\le v\;\on B(x_0,r)$を満たす.
        するといま,$u-v\le\o{u}-\o{v}$は調和だから,最大値の原理より,
        \begin{align*}
            M&=\max_{\o{\Om}}(u-v)\ge\max_{\partial\Om}(u-v)=\max_{\partial\Om}(\o{u}-\o{v})\\
            &\ge\max_{\Om}(\o{u}-\o{v})\qquad\because\text{弱最大値原理}\\
            &\ge(\o{u}-\o{v})(x_0)\ge(u-v)(x_0)=M.
        \end{align*}
        このとき,調和関数$\o{u}-\o{v}$はその最大値を内点$x_0\in B$で達成しているので,$\o{B}$上定数になってしまう.
        これは$u,v$が$\partial B$上定値でないとした仮定に矛盾する.
    \end{description}
\end{Proof}

\subsection{弱劣調和関数の調和関数による持ち上げ}

\begin{proposition}[harmonic lifting]
    $\Om\osub\R^n$を有界領域,
    $u\in C(\Om)$を弱劣調和,$\o{u}\in C^2(\o{B})$を開球$B\subset\Om$上の$\partial B$上で$u$と同じ値を取る調和関数とする:$u\le\o{u}\;\on\o{B}$.
    このとき,$u$の$B$上での値を$\o{u}$に修正した関数
    \[U(x):=\begin{cases}
        \o{u}(x)&x\in B,\\
        u(x)&x\in\Om\setminus B.
    \end{cases}\]
    は再び$\Om$上弱劣調和である.
\end{proposition}
\begin{Proof}
    任意の開球$B'\subset\Om$とその上の調和関数$h\in H(B')$であって$U\le h\;\on\partial B'$を満たすものを取る.
    これについて,$U\le h\;\on B'$を示せば良い.
    $B\cap B'=\emptyset$ならば,この$B'$上で$U=u$であるから明らかに弱劣調和.よって$B\cap B'=:V\ne\emptyset$とする.
    \begin{enumerate}
        \item まず$u\le U\le h\;\on\o{B'}$と$u$の劣調和性から$u\le h\;\on B'$である.
        よっていま,$B'\setminus B$で$U=u\le h$,$B'\cap B$で$U=\o{u}$が成り立っている.
        \item 上の叙述は条件
        \[\begin{cases}
            U=\o{u}&\In V,\\
            U\le h&\on\partial V.
        \end{cases}\]
        を含んでいるから,比較原理\ref{thm-comparison-principle}より,$U\le h\;\on V$も得る.
    \end{enumerate}
    以上で,$U\le h\;\on B'$.
\end{Proof}

\begin{corollary}
    $\Om\osub\R^n$を有界領域,
    $\{u_i\}_{i\in I}\subset C(\Om)$を弱劣調和関数の族とする.このとき,
    $u(x):=\sup_{i\in I}u_i(x)$も弱劣調和である.
\end{corollary}
\begin{Proof}
    任意の開球$B\subset\Om$とその上の調和関数$h\in H(B)$で$u\le h\;\on\partial B$を満たすものを取る.
    このとき,$\forall_{i\in I}\;u_i\le h\;\on\o{B}$が成り立つから,結局$u\le h\;\on\o{B}$.
\end{Proof}

\subsection{Perronによる解の存在定理}

\begin{tcolorbox}[colframe=ForestGreen, colback=ForestGreen!10!white,breakable,colbacktitle=ForestGreen!40!white,coltitle=black,fonttitle=\bfseries\sffamily,
title=]
    境界条件$g\in C(\partial\Om)$に関する
    Dirichlet問題
    \[\mathrm{(D)}\quad\begin{cases}
        \Lap u=0&\In\Om,\\
        u=g&\on\partial\Om.
    \end{cases}\]
    の解の候補を弱劣調和関数族$\S_g$の上限$\sup\S_g$として構成する.
\end{tcolorbox}

\begin{theorem}[Perron (1923)]\label{thm-Perron}
    $\Om\osub\R^n$を有界領域,境界条件$g\in C(\partial\Om)$を連続とする.
    境界上で$g$に優越される弱劣調和函数の全体
    \[\S_g:=\Brace{v\in C(\o{\Om})\mid v\text{は弱劣調和で}v\le g\;\on\partial\Om}\ne\emptyset\]
    の上限$u:=\sup_{v\in\S_g}v$は$\Om$上調和である.
    これを\textbf{Perron解}という.
\end{theorem}

\subsection{境界条件の達成要件}

\begin{tcolorbox}[colframe=ForestGreen, colback=ForestGreen!10!white,breakable,colbacktitle=ForestGreen!40!white,coltitle=black,fonttitle=\bfseries\sffamily,
title=]
    Perronの方法の美点の一つは,調和関数の構成と境界条件の達成とを明確に区別して構成できる点である.
\end{tcolorbox}

\begin{definition}[local barrier, regular point]
    $\Om\osub\R^n$を有界領域とする.
    \begin{enumerate}
        \item $\xi\in\partial\Om$の\textbf{バリア}とは,次の2条件を満たす連続関数$\om_\xi\in C(\o{\Om})$をいう:
        \begin{enumerate}
            \item $\om_\xi$は$\Om$上弱優調和である.
            \item $\om_\xi(\xi)=0$を除いて,$\om_\xi>0\;\on\o{\Om}\setminus\{\xi\}$.
        \end{enumerate}
        \item $\xi\in\partial\Om$が(Laplace作用素に対する)\textbf{正則点}であるとは,そこでバリアが存在することをいう.
    \end{enumerate}
\end{definition}

\begin{proposition}[正則点における境界条件の達成]
    $u:=\sup\S_g$をPerron解\ref{thm-Perron},
    $\xi\in\partial\Om$を正則点とする.
    このとき,$u(x)\xrightarrow{x\to\xi}g(\xi)$.
\end{proposition}

\begin{corollary}
    Dirichlet問題(D)に対して,
    次の2条件は同値:
    \begin{enumerate}
        \item 古典解$u\in C^2(\Om)\cap C(\o{\Om})$を持つ.
        \item $\Om$は正則な境界を持つ.
    \end{enumerate}
\end{corollary}
\begin{Proof}
    (1)$\Rightarrow$(2)を示せば良いが,境界条件を$g:=\abs{x-\xi}$を置いたときの解$u\in C^2(\o{\Om})$は$\xi\in\partial\Om$のバリアになる.
\end{Proof}

\subsection{正則境界を持つ領域の例}

\begin{tcolorbox}[colframe=ForestGreen, colback=ForestGreen!10!white,breakable,colbacktitle=ForestGreen!40!white,coltitle=black,fonttitle=\bfseries\sffamily,
title=]
    2次元の場合は精密に議論できても,$n\ge3$から話が変わる.
\end{tcolorbox}

\begin{proposition}[外部球条件]
    $\Om\osub\R^n$を有界領域,$\xi\in\partial\Om$を境界点とする.
    \begin{enumerate}
        \item ある開球$B(y,R)$であって$\o{B}\cap\o{\Om}=\{\xi\}$を満たすものが取れるとき,
        これは正則点である.
        \item $\partial\Om$が$C^2$-級ならば,任意の点で外部球条件を満たし,球の半径は一様に取れる.
    \end{enumerate}
\end{proposition}
\begin{Proof}
    \[\om_\xi(x):=\begin{cases}
        \log\frac{\abs{x-y\xi}}{R}&n=2,\\
        \frac{1}{R^{n-2}}-\frac{1}{\abs{x-y\xi}^{n-2}}&n\ge3.
    \end{cases}\]
\end{Proof}

\begin{notation}
    $\Om\osub\R^n\;(u\ge3)$を有界領域とする.
    外部D-境界値問題
    \[\text{(OD)}\quad\begin{cases}
        \Lap u=0&\In\R^n\setminus\Om,\\
        u=1&\on\partial\Om,\\
        u=0&x\to\infty.
    \end{cases}\]
    の解を$u$とする.
\end{notation}

\begin{definition}[(harmonic / Newtonian) capacity]
    $\Om\osub\R^n$を有界領域,$\Sigma\subset\R^n\setminus\Om$を$\Om$を囲む曲面とする.このとき,
    \[\mathrm{Cap}\, \Om:=-\int_{\Sigma}\pp{u}{\nu}dS=\int_{\R^n\setminus\Om}\abs{Du}^2dx,\quad \nu\text{:外側単位法線ベクトル}\]
    を\textbf{容量}と言う.
    $u$を導体電位(conductor potential)とも言う.
\end{definition}
\begin{remarks}
    $u$が定める電場について,電位の基準点を$\infty$に取れば,$\mathrm{Cap}\,\Om$は$\Om$が導体であった場合の電位,すなわち,$\partial\Om$に単位電荷が分布した場合の電位に,定数倍を除いて等しい.
    この妙は,任意の領域$\Om$に対して値が定義できていることにある.
\end{remarks}

\begin{theorem}[Wiener's criterion]
    有界領域$\Om\osub\R^n$の境界点$x_0\in\partial\Om$に対して,
    \[C_j:=\mathrm{Cap}\,\Brace{x\notin\Om\mid\abs{x-x_0}\le\lambda^j},\qquad\lambda\in(0,1),j\in\N.\]
    と定める.このとき,次は同値:
    \begin{enumerate}
        \item $x_0$は正則点である.
        \item 級数$\sum_{j\in\N}\frac{C_j}{\lambda^{j(n-2)}}$は発散する.
    \end{enumerate}
\end{theorem}

\section{エネルギー法と変分原理によるD-問題の解の特徴付け}

\begin{tcolorbox}[colframe=ForestGreen, colback=ForestGreen!10!white,breakable,colbacktitle=ForestGreen!40!white,coltitle=black,fonttitle=\bfseries\sffamily,
title=]
    $L^2$-ノルムによる方法である.
\end{tcolorbox}

\subsection{エネルギー積分による解の一意性の証明}

\begin{proposition}
    $U\osub\R^n$を有界領域,$\partial U$を$C^1$-級とする.次のPoisson方程式のD-境界値問題の古典解$u\in C^2(\o{U})$は一意的である:
    \[\text{(D)}\quad\begin{cases}
        -\Lap u=f&\interior U,\\
        u=g&\on\partial U.
    \end{cases}\]
\end{proposition}
\begin{Proof}
    $u,v$を2つの解とすると,差$w:=u-v$は$f=g=0$に関する問題(D)の解である.
    部分積分によると,
    \[0=\int_\Om w\Lap wdx=\int_{\partial\Om} w\pp{w}{\nu}dS-\int_\Om(D w|Dw)dx=\int_\Om\abs{Dw}^2dx\]
    より,$Dw=0\;\ae$が必要.境界条件と併せて,$w=0\;\on\Om$.
\end{Proof}

\begin{definition}[energy integral]
    方程式$\Lap u=f$の両辺に$u$を乗じて部分積分をする手続きを\textbf{エネルギー積分}といい,この値に注目する手法をエネルギー法という.
\end{definition}
\begin{remarks}
    「両辺に$u$を乗じて部分積分をする」という行為自体が,超関数微分の発想に他ならないのである.
    そこで,弱解の理論はエネルギー法に深く関連する.
\end{remarks}

\subsection{Dirichletの原理}

\begin{definition}[energy functional, admissible set]\mbox{}
    \begin{enumerate}
        \item 次を\textbf{エネルギー汎関数}という:
        \[I[w]:=\int_U\paren{\frac{1}{2}\abs{Dw}^2-wf}dx.\]
        \item 次を\textbf{許容集合}という:
        \[\A:=\Brace{w\in C^2(\o{U})\mid w=g\;\on\partial U}.\]
    \end{enumerate}
\end{definition}

\begin{theorem}[Dirichlet's principle]
    $u\in C^2(\o{U})$について,
    \begin{enumerate}
        \item Poisson方程式のD-境界値問題(D)の解であるならば,$I[u]=\min_{w\in\A}I[w]$.
        \item $u\in\A$が変分問題$\min_{w\in\A}I[w]$を解くならば,これは解である.
    \end{enumerate}
\end{theorem}
\begin{Proof}\mbox{}
    \begin{description}
        \item[(1)$\Rightarrow$(2)] 任意に許容集合の元$w\in\A$をとり,$I[u]\le I[w]$を示せば良い.
        $-\Lap u-f=0$に$(u-w)$を乗じたエネルギー積分は部分積分により
        \[0=\int_\Om(-\Lap u-f)(u-w)dx=\int_\Om\paren{(Du|D(u-w))-f(u-w)}dx\]
        と変形できる.よって,Cauchy-Schwarzの不等式と恒等式$ab\le\frac{1}{2}(a^2+b^2)$より,
        \begin{align*}
            \int_\Om(\abs{Du}^2-fu)dx&=\int_\Om((Du|Dw)-fw)dx\\
            &\le\int_\Om\frac{1}{2}\abs{Du}^2+\frac{1}{2}\abs{Dw}^2-fwdx
        \end{align*}
        となるが,これは$I[u]\le I[w]$の変形になっている.
        \item[(2)$\Rightarrow$(1)] 任意の変位方向$\varphi\in C^\infty_c(\Om)$に対して,$f(t):=I[u+t\varphi]$の$t=0$での微分係数は零になることが必要であるから,
        \[f(t)=\int_\Om\frac{1}{2}\paren{\abs{Du}^2+2t(Du|D\varphi)+t^2\abs{D\varphi}^2-2(u+t\varphi)f}dx\]
        \[f'(t)=\int_\Om((Du|D\varphi)+t\abs{D\varphi}^2-f\varphi)dx.\]
        よって,部分積分と$\varphi$の台がコンパクトであることより,
        \[\int_\Om((Du|D\varphi)-f\varphi)dx=\int_\Om(-\Lap u-f)\varphi dx=0\]
        が必要.変分法の基本補題より,$-\Lap u-f=0$が必要.
    \end{description}
\end{Proof}
\begin{remarks}\mbox{}
    \begin{enumerate}
        \item $I[u]$がすごくLagrangianっぽい.
        \item $\A$の導関数との1-ノルム
        \[\norm{w}^2:=\int_\Om(\abs{w}^2+\abs{Dw}^2)dx\]
        に関する完備化はSobolev空間になり,特に$g=0$の場合は$W_0^{1,2}(\Om)=H_0^1(\Om)$と表される.
        すると(2)はLax-Mlgramの定理を通じて,弱解の存在を保証している.
    \end{enumerate}
\end{remarks}


\section{Dirichlet固有値問題とHelmholtz方程式}

\begin{problem}
    有界領域$\Om\osub\R^n$と$V\in C(\o{\Om}),\lambda\in\R$について,
    次のDirichlet型の固有値問題
    \[\text{(E-D)}\quad\begin{cases}
        -\Lap u+V(x)u=\lambda u&\In\Om,\\
        u=0&\on\partial\Om.
    \end{cases}\]
    の非自明な解$(u,\lambda),u\not\equiv0$を考える.
    また,Neumann型の境界条件を与えた場合を
    \[\text{(E-N)}\quad\begin{cases}
        -\Lap u+V(x)u=\lambda u&\In\Om,\\
        \pp{u}{\nu}=0&\on\partial\Om.
    \end{cases},\qquad\text{(E-E)}\quad\begin{cases}
        -\Lap u+V(x)u=\lambda u\qquad\In\Om,\\
        \int_{\partial\Om}\paren{u_1\pp{u_2}{\nu}-u_2\pp{u_1}{\nu}}dS(x)=0.
    \end{cases}\]
    とする.
\end{problem}

\subsection{固有関数の直交性}

\begin{proposition}[固有関数の直交性]
    $(u_1,\lambda_1),(u_2,\lambda_2)$を(E-E)の解とする((E-D),(E-N)の解はこれを満たす).
    \begin{enumerate}
        \item $\lambda_1\ne\lambda_2$のとき,$(u_1|u_2)=0$である.
        \item $V>0\;\on\Om$のとき,全ての固有値は正である.
    \end{enumerate}
\end{proposition}
\begin{Proof}\mbox{}
    \begin{enumerate}
        \item 実は任意の2解$\lambda_1,\lambda_2\in\R$について,
        \[(\lambda_1-\lambda_2)\int_\Om u_1u_2dx=0\]
        が成り立つためである.これは,部分積分(Greenの第二恒等式\ref{cor-Green-identity})を2回行うことによる:
        \begin{align*}
            \lambda_1\int_\Om u_1u_2dx&=\int_\Om(-\Lap u_1+V(x)u_1)u_2dx\\
            &=-\int_{\partial\Om}u_2\pp{u_1}{\nu}dS(x)+\int_\Om(Du_1|Du_2)dx+\int_\Om Vu_1u_2dx\\
            &=-\int_{\partial\Om}u_2\pp{u_1}{\nu}dS(x)-\int_\Om u_1\Lap u_2dx+\int_{\partial\Om}u_1\pp{u_2}{\nu}dS(x)+\int_\Om Vu_1u_2dx\\
            &=\int_{\partial\Om}\paren{u_1\pp{u_2}{\nu}-u_2\pp{u_1}{\nu}}dS(x)+\int_\Om(-\Lap u_2+Vu_2)u_1dx\\
            &=\lambda_2\int_\Om u_1u_2dx.
        \end{align*}
        \item 同様の発想で,まずは次の式を考えると,
        \begin{align*}
            \lambda\int_{\Om}u^2dx&=\int_{\Om}u(-\Lap u+V(x)u)dx\\
            &=\int_\Om V(x)u^2dx-\int_\Om u\Lap udx\\
            &=\int_\Om(V(x)u^2+\abs{Du}^2)dx\ge0
        \end{align*}
        より,$\lambda\ge0$が必要.さらに,$\lambda=0$とすると,この右辺は$0$である必要があり,そのとき$V(x)>0$より$u\equiv0$が要請されるが,これは$u$の非自明性に矛盾.
    \end{enumerate}
\end{Proof}

\subsection{レゾルベント方程式と熱方程式}

\begin{problem}
    熱方程式の初期値問題
    \[\begin{cases}
        v_t-\Lap v=0&\In U\times\R^+,\\
        v=f&\on U\times\{0\}
    \end{cases}\]
    の古典解$v\in C^2(U\times\R_+)$を考える.
\end{problem}

\begin{proposition}
    古典解$v\in C^2(U\times\R_+)$のLaplace変換
    \[v^\#(x,s)=\int^\infty_0e^{-st}v(x,t)dt,\qquad s>0.\]
    は任意の$s>0$に対して,$u(x):=v^\#(x,s)$はレゾルベント方程式
    \[-\Lap u+su=f\quad\In U\]
    を満たす.
\end{proposition}

\chapter{熱方程式と拡散現象}

\begin{quotation}
    データを
    \[U\osub\R^n,\quad\Lap:=\Lap_x,\quad\begin{cases}
        u:\o{U}\times\R_+\to\R&\text{未知}\\
        f:U\times\R_+\to\R&\text{既知}
    \end{cases}\]
    として,非斉次熱方程式
    \begin{equation}\label{N-HE}\tag{N-HE}
        u_t(x,t)-\Lap u(x,t)=f(x,t)\;\on (x,t)\in U\times\R^+
    \end{equation}
    を考える.
    Laplace方程式に関する知識は,すべて熱方程式に関するバージョンを持つ.
    すなわち,アイデアが全く等しく,ただ結論のみが違う.
    その際,放物項$u_t$がどう影響を与えるかが焦点である.
    ただ唯一違うのは,解のクラスに「調和関数」のような名前がついているわけではなく,平均値の定理や最大値の原理などは,解の性質として議論する.
\end{quotation}

\section{熱方程式の物理的考察}

\begin{tcolorbox}[colframe=ForestGreen, colback=ForestGreen!10!white,breakable,colbacktitle=ForestGreen!40!white,coltitle=black,fonttitle=\bfseries\sffamily,
title=]
    熱方程式は,熱や化学物質の濃度を$u$として,その拡散過程を記述する.
\end{tcolorbox}

\begin{observation}[熱方程式の発生]
    $u$の流れ(flux)を$\b{F}$とする.殆どの場合,線型関係
    \[\b{F}=-aDu\quad(a>0)\]
    が成り立つ(Newtonの冷却法則など).
    これを,$V\subset U$を滑らかな部分多様体として,流入・出関係
    \[\dd{}{t}\int_Vudx=-\int_{\partial V}\b{F}\cdot\b{\nu}\;dS,\qquad\therefore\quad u_t=-\div\b{F}\]
    に代入すると,偏微分方程式
    \[u_t=a\div(Du)=a\Lap u\]
    を得る.特に比例定数を$a=1$
\end{observation}

\section{上半空間上の基本解}

\begin{problem}
    全空間$\R^n$上の斉次な熱方程式
    \begin{equation}\label{HE}\tag{HE}
        u_t-\Lap u=0\quad\on\R^n\times\R^+.
    \end{equation}
    を考える.
    \begin{enumerate}
        \item その$\R^n\times\R^+$上のDirichlet境界値問題,または$\R^n$上のCauchy初期値問題
        \begin{equation}\label{HE-C}\tag{HE-C}
            \begin{cases}
                u_t-\Lap u=0&\In\R^n\times\R^+,\\
                u=g&\on\R^n\times\{0\}
            \end{cases}
        \end{equation}
        を考える.
        \item さらに,非斉次熱方程式に関する
        $\R^n\times\R^+$上のDiriclet境界値問題,すなわち$\R^n$上のCauchy初期値問題
        \begin{equation}\label{N-HE-C}\tag{N-HE-C0}
            \begin{cases}
                u_t-\Lap u=f&\In\R^n\times\R^+,\\
                u=0&\on\R^n\times\{0\}.
            \end{cases}
        \end{equation}
        を考える.
    \end{enumerate}
\end{problem}

\subsection{方程式の相似変換対称性の観察:Gauss核の発見}

\begin{tcolorbox}[colframe=ForestGreen, colback=ForestGreen!10!white,breakable,colbacktitle=ForestGreen!40!white,coltitle=black,fonttitle=\bfseries\sffamily,
title=]
    Laplace方程式は回転変換不変な解として基本解を得た.
    熱方程式では,相似変換不変な解として探す.すなわち,解は,ある1変数関数$v$について,
    \[u(x,t):=\frac{1}{\sqrt{t}^n}v\paren{\frac{x}{\sqrt{t}}}\]
    と表せる.Gauss核$v(\xi)=be^{-\xi^2/4}$はその例である.
\end{tcolorbox}

\begin{observation}[相似変換不変性への注目]
    $u$が\ref{HE}の解ならば,1-径数族$u(\lambda x,\lambda^2t)\;(\lambda\in\R)$も解である.
    特に,$\frac{r^2}{t}$が保存量になる.
    そこで,Laplace方程式の時に$r$の関数を考えたように,
    \[u(x,t)=v\paren{\frac{r^2}{t}},\quad r:=\abs{x}\]
    の関数を考えてみることが指針として考え得る.
\end{observation}

\begin{definition}[dilation scaling]
    変換$u(x,t)\mapsto\lambda^\al u(\lambda^\beta x,\lambda t)$を\textbf{相似拡大}という.
\end{definition}

\begin{lemma}[相似変換不変で回転対称で有界な解としてのGauss核]
    \ref{HE}の解$u:\R^n\times\R_+\to\R$はある$\al,\beta\in\R$に関して,
    任意の$\lambda>0,x\in\R^n,t>0$について,
    \[u(x,t)=\lambda^\al u(\lambda^\beta x,\lambda t)\]
    を満たすとする.
    \begin{enumerate}
        \item $v(y):=u(y,1)\;(y\in\R^n)$とすると,次の表示を持つ:
        \[u(x,t)=\frac{1}{t^\al}v\paren{\frac{x}{t^\beta}},\qquad x\in\R^n,t>0.\]
        \item $v\in C^2(\R^n)$は次を満たす:
        \[av(y)+\frac{1}{2}y\cdot Dv(y)+\Lap v(y)=0.\]
        \item 回転対称$w(\abs{y})=v(y)$かつ$\lim_{r\to\infty}w,w'=0$を満たす解$w\in C^1_0(\R)$は
        \[w=be^{-\frac{r^2}{4}},\qquad b\in\R,r:=\abs{y}.\]
        と表せる.
    \end{enumerate}
\end{lemma}
\begin{Proof}\mbox{}
    \begin{enumerate}
        \item $\lambda:=1/t$と取ると明らか.
        \item 
        \begin{description}
            \item[$v$の方程式への帰着] $y:=xt^{-\beta}$という変数を準備しておくと,連鎖律\ref{prop-chain-rule}に注意して,
            \begin{align*}
                u_t&=-\al t^{-(\al+1)}v(xt^{-\beta})+t^{-\al}Dv(xt^{-\beta})\cdot(-\beta)xt^{-(\beta+1)}\\
                &=-\al t^{-(\al+1)}v(y)-\beta t^{-(\al+1)}(Dv(y)|y).
            \end{align*}
            \[u_{x_i}=t^{-\al}v_{x_i}(xt^{-\beta})t^{-\beta},\quad u_{x_ix_i}=t^{-(\al+2\beta)}v_{x_ix_i}(y).\]
            より,$\Lap u=t^{-(\al+2\beta)}\Lap v(y)$.以上より,方程式\ref{HE}は
            \[\al t^{-(\al+1)}v(y)+\beta t^{-(\al+1)}(y|Dv(y))+t^{-(\al+2\beta)}\Lap v(y)=0\]
            に等しい.特に$\beta=1/2$の場合を考えると,
            \[t^{-(\al+1)}\paren{\al v(y)+\frac{1}{2}(y|Dv(y))+\Lap v(y)}=0.\]
            \item[$v$に関して解く] こうして$\R^n$上の関数$v$についての方程式
            \[\al v+\frac{1}{2}(y|Dv)+\Lap v=0\]
            にまで落ちた.$v$は回転対称である$v(y)=w(\abs{y})$とすると,まず第二項は
            \[Dv(y)=\paren{w'(\abs{y})\frac{y_1}{\abs{y}},\cdots,w'(\abs{y})\frac{y_n}{\abs{y}}}\]
            より,
            \[(y|Dv)=\sum_{i\in[n]}w'(\abs{y})\frac{y_i^2}{\abs{y}}=\abs{y}w'.\]
            次に第三項は
            \begin{align*}
                v_{y_i}(y)&=w'\frac{y_i}{\abs{y}},\\
                v_{y_iy_i}(y)&=w''\frac{y^2_i}{\abs{y}^2}+w'\paren{\frac{1}{\abs{y}}-\frac{y_i^2}{\abs{y}^3}}.
            \end{align*}
            より,
            \[\Lap v(y)=w''+w'\frac{n-1}{\abs{y}}.\]
            以上より,$w(r)\in C^1(\R)$に関する方程式は
            \[\al w+\frac{1}{2}rw'+w''+\frac{n-1}{r}w'=0,\qquad r:=\abs{y}.\]
            さらに$\al=\frac{n}{2}$の場合を考え,両辺に$r^{n-1}$を乗じると,
            \[\frac{n}{2}r^{n-1}w+\frac{1}{2}r^nw'+r^{n-1}w''+(n-1)r^{n-2}w'=\frac{1}{2}(r^nw)'+(r^{n-1}w')'=0.\]
            すなわち,ある$a\in\R$について,
            \[r^{n-1}w'+\frac{1}{2}r^nw=a\]
            が必要.仮定$\lim_{r\to\infty}w,w'=0$より,$a=0$.
            これは1階線型方程式$w'=-\frac{1}{2}rw$であるから,変数係数$-\frac{1}{2}r$の原始関数$-\frac{1}{4}r^2$を取れば,一般解は
            \[w=be^{-\frac{r^2}{4}},\qquad b\in\R.\]
        \end{description}
    \end{enumerate}
\end{Proof}

\begin{proposition}[相似変換不変な解の発見]\label{prop-fundamental-solution-of-HE-is-a-solution}
    \[u(x,t)=\frac{b}{t^{n/2}}e^{-\frac{\abs{x}^2}{4t}},\qquad(b\in\R,(x,t)\in\R^n\times\R^+)\]
    は,熱方程式\ref{HE}の相似変換不変な解である:$u(x,t)=\lambda^{n/2} u(\lambda^{1/2} x,\lambda t)$.
\end{proposition}
\begin{Proof}
    $\al=n/2,\beta=1/2$と取ったことを思い出せば,
    \[u(x,y)=\frac{1}{t^{n/2}}v\paren{\frac{x}{t^{1/2}}}=\frac{b}{t^{n/2}}e^{-\frac{\abs{x}^2}{4t}}.\]
\end{Proof}

\subsection{基本解の定義}

\begin{tcolorbox}[colframe=ForestGreen, colback=ForestGreen!10!white,breakable,colbacktitle=ForestGreen!40!white,coltitle=black,fonttitle=\bfseries\sffamily,
title=]
    \begin{definition}[fundamental solution of the heat equation]
        熱方程式の\textbf{基本解}または\textbf{熱核}とは,
        \[\tcboxmath{\Phi(x,t):=\frac{1}{(4\pi t)^{n/2}}e^{-\frac{\abs{x}^2}{4t}}1_{\R^n\times(0,\infty)}(x,t)\qquad(x\in\R^n,t>0).}\]
        をいう.
        すなわち,$n$次元正規分布$\rN_n(0,2t)$の密度関数をいう.
    \end{definition}
\end{tcolorbox}

\begin{proposition}[基本解の性質]\label{prop-property-of-fundamental-solution-to-EH}
    $n\in\N^+$次元空間上の熱方程式の基本解$\Phi$について,
    \begin{enumerate}
        \item 原点での特異性:$t=0$を除いた点$\R^n\times\R^+\setminus\{(0_n,0)\}$上滑らかかつ$\Phi>0$を満たし,変数$x$については球対称である(すなわち,$\abs{x}$の関数である).
        原点での特異性が総和核としての性質をもたらす.形式的には,
        \[\begin{cases}
            \Phi_t-\Lap\Phi=0&\In\R^n\times\R^+,\\
            \Phi=\delta_0&\on\R^n\times\{0\}.
        \end{cases}\]
        \item $\Phi$自身も,$\R^n\times\R^+$上で超関数の意味で,各点毎には原点を除いた領域$\R^n\times\R\setminus\{(0_n,0)\}$上で斉次熱方程式\ref{HE}の解である\ref{prop-fundamental-solution-of-HE-is-a-solution}:
        \[\Phi_t=\Lap_x\Phi.\]
        $b=\frac{1}{(4\pi)^{n/2}}$とした理由は次の補題による.
        \item 平行移動しても解である:さらに,任意の$y\in\R^n,s\in\R^+$に対して,平行移動$(x,t)\mapsto\Phi(x-y,t-s)$も$\R^n\times\R\setminus\{(y,s)\}$上で\ref{HE}の解である.
        もちろん重ね合わせも解になる.
        \item 畳み込みによって関数を近似できる:$\R^n$上の総和核である.
        すなわち,
        \[\forall_{\delta>0}\quad\lim_{t\to0}\int_{\abs{x}\ge\delta}\abs{\Phi(x,t)}dx=0.\]
        これより,次の定理\ref{thm-solution-for-HHE} (2)を満たす.
        \item 導関数の$\R^n\times\cointerval{\delta,\infty}$上の一様有界性:$\Phi$の任意階の導関数は,任意の$\R^n\times\cointerval{\delta,\infty}\;(\delta>0)$上で一様に有界である.
        これにより,\textbf{$t=0$を避ければ微分と積分の交換が可能}で,$\Phi$との畳み込みは可微分である(定理\ref{thm-solution-for-HHE}).
    \end{enumerate}
\end{proposition}
\begin{Proof}
    (5)を示す.
    \begin{description}
        \item[有界性] $\Phi\in\S$より,任意階の$x_i$に関する導関数は急減少である.
        \[\partial_{x_i}\Phi(x,t)=-\frac{x_i}{\sqrt{4\pi t}}e^{-\frac{\abs{x}^2}{4t}}.\]
        \[\partial_t\Phi(x,t)=-\frac{1}{4\sqrt{\pi}}t^{-\frac{3}{2}}e^{-\frac{\abs{x}^2}{4t}}+\frac{\abs{x}^2}{4t^2}\frac{1}{\sqrt{4\pi t}}e^{-\frac{\abs{x}^2}{4t}}.\]
        この形をした関数は$\R^n\times\cointerval{\delta,\infty}$上有界である.
        \item[一様性] 
    \end{description}
\end{Proof}

\begin{lemma}[基本解の設計意図:基本解の積分は1]
    任意の$t>0$について,
    \[\int_{\R^n}\Phi(x,t)dx=1.\]
\end{lemma}
\begin{Proof}
    変数変換$z:=\frac{x}{2\sqrt{t}}$を考えると,$dz=\paren{\frac{1}{2\sqrt{t}}}^ndx$で,
    \begin{align*}
        \int_{\R^n}\Phi(x,t)dx&=\frac{1}{(4\pi t)^{n/2}}\int_{\R^n}e^{-\frac{\abs{x}^2}{4t}}dx\\
        &=\frac{1}{\pi^{n/2}}\int_{\R^n}e^{-\abs{z}^2}dz\\
        &=\frac{1}{\pi^{n/2}}\prod_{i=1}^n\int_\R e^{-z^2_i}dz_i=1.
    \end{align*}
\end{Proof}

\subsection{基本解のFourier変換による特徴付け}

\begin{tcolorbox}[colframe=ForestGreen, colback=ForestGreen!10!white,breakable,colbacktitle=ForestGreen!40!white,coltitle=black,fonttitle=\bfseries\sffamily,
title=]
    熱方程式をFourier変換すると,$\wh{H_t}(\xi)=e^{-t\abs{\xi}^2}$のFourier逆変換が基本解であることが分かり,実際その通りになっている.
\end{tcolorbox}

\begin{theorem}[\cite{Strichartz03-Distribution} 3.5節]\mbox{}\label{thm-Fourier-transform-of-Gaussian-density}
    \begin{enumerate}
        \item $\R$上のGauss核について,\[\F[e^{-tx^2}](\xi)=\sqrt{\frac{\pi}{t}}e^{-\frac{\xi^2}{4t}}.\]
        \item $\R^d$上のGauss核について,
        \[\F[e^{-t\abs{x}^2}]=\paren{\frac{\pi}{t}}^{d/2}e^{-\frac{\abs{\xi}^2}{4t}}.\]
    \end{enumerate}
\end{theorem}
\begin{Proof}\mbox{}
    \begin{enumerate}[{Step}1]
        \item 平方完成により,
        \begin{align*}
            \int^\infty_{-\infty}e^{-tx^2}e^{ix\xi}dx=e^{-\frac{\xi^2}{4t}}\int^\infty_{-\infty}e^{-t\paren{x-\frac{i\xi}{2t}}^2}dx
        \end{align*}
        を得る.
        \item これは$z=x-\frac{i\xi}{2t}$の変数変換により計算可能である.
        まず$\xi>0$の場合には,
        $\C$上の積分路$\partial([-R,R]\times[0,i\xi/2t])$を考えることにより,これを計算できる.
        左端と右端の積分は,
        \[\abs{e^{-tz^2}}=\abs{e^{-t(x^2-y^2)-2itxy}}\le e^{-t(x^2-y^2)}\]
        より,$R\to\infty$の極限で$0$に収束する.
        よって,被積分関数が$\C$上の正則関数であることより,
        \[\sqrt{\frac{\pi}{t}}=\int_\R e^{-tx^2}dx=\int_\R \exp\paren{-t\paren{x-\frac{i\xi}{2t}}^2}dx.\]
        \item $\xi<0$の場合と,元の式と併せて,
        \[\F[e^{-tx^2}](\xi)=\sqrt{\frac{\pi}{t}}e^{-\frac{\xi^2}{4t}}.\]
    \end{enumerate}
\end{Proof}

\begin{theorem}[上半平面上の熱核]
    $\wh{H_t}(\xi)=e^{-t\abs{\xi}^2}$によって$\R^d$上の関数$H_t$を定めると,次のように表示できる:
    \[H_t(x)=\frac{1}{(4\pi t)^{d/2}}e^{-\frac{\abs{x}^2}{4t}}.\]
    さらに,$(H_t)_{t>0}$は$t\to0$の極限において総和核をなす.
    なお,これは$\rN_d(0,2tI_d)$の密度関数でもある.
\end{theorem}

\begin{observation}
    上半平面$\R^{d+1}_+:=\Brace{(x_1,\cdots,x_d,t)\in\R^{d+1}\mid t\ge0}$における熱方程式の境界値問題
    \[\begin{cases}
        u_t=\Lap u&\In \R^{d+1}_+,\\
        u=f&\on\partial\R^{d+1}_+.
    \end{cases}\]
    を考える.$t\in\R_+$を固定して$x\in\R^d$についてFourier変換すると,$t>0$についてのODE
    \[\begin{cases}
        \pp{\wh{u_t}}{t}(\xi)=-\abs{\xi}^2\wh{u_t}(\xi),\\
        \wh{u_0}(\xi)=\wh{f}(\xi).
    \end{cases}\]
    を得る.これは1階線型ODEの初期値問題であるから,ただ一つの解$\wh{u_t}(\xi)=e^{-t\abs{\xi}^2}\wh{f}(\xi)$を持つ.
    すなわち,$u$が一般解であるための必要条件は,
    $\wh{H_t}(\xi)=e^{-t\abs{\xi}^2}$を満たす関数$H_t$について,
    $u_t=H_t*f$である.
\end{observation}

\begin{observation}
    空間変数$x\in\R^n$のみについてFourier変換を考えると,
    \[\begin{cases}
        \wh{u}_t+\abs{y}^2\wh{u}=0&\R^n\times\R^+,\\
        \wh{u}=\wh{g}&\R^n\times\{0\}.
    \end{cases}\]
    $t$と$x$について変数分離型だと見做し,$t$について解くと,
    \[\wh{u}=e^{-t\abs{y}^2}\wh{g}\]
    が必要.ここで,$\wh{F}=e^{-t\abs{y}^2}$が成り立つとしよう.
    実は,$F=\frac{1}{(2t)^{n/2}}e^{-\frac{\abs{x}^2}{4t}}$である.
    $\wh{u}\in L^2(\R^n)$に注意すると,反転公式\ref{prop-inversion-formula-for-Fourier-transform}より,
    \[u=\check{\wh{u}}=\check{\wh{F}\wh{g}}=\frac{g*F}{(2\pi)^{n/2}}=\frac{1}{(4\pi t)^{n/2}}\int_{\R^n}e^{-\frac{\abs{x-y}^2}{4t}}g(y)dy,\quad(x\in\R^n,t>0).\]
\end{observation}

\subsection{斉次D-境界値問題の解公式}

\begin{tcolorbox}[colframe=ForestGreen, colback=ForestGreen!10!white,breakable,colbacktitle=ForestGreen!40!white,coltitle=black,fonttitle=\bfseries\sffamily,
title=]
    熱方程式が斉次である場合,初期値/境界値$g$に対して基本解$\Phi$を畳み込むことで,これを満たす解を得る.
    \begin{enumerate}
        \item データが$g\in C_b(\R^n)$ならば,各点で境界条件を満たす.
        \item さらに$g\in UC_b(\R^n)$ならば,$u(t,-)\to g(t)$は一様.
        \item $g\in L^p(\R^n)\;(1\le p<\infty)$のとき,平均の意味でも,$u(t,-)\to g\in L^p(\R^n)$
    \end{enumerate}
\end{tcolorbox}

\begin{problem*}
    斉次熱方程式の$\R^n\times\R^+$上のDirichlet境界値問題,または$\R^n$上のCauchy初期値問題\ref{HE-C}
    \[\text{(HE-C)}\quad\begin{cases}
        u_t-\Lap u=0&\In\R^n\times\R^+,\\
        u=g&\on\R^n\times\{0\}
    \end{cases}\]
    を考える.
\end{problem*}

\begin{theorem}\label{thm-solution-for-HHE}
    $p\in[1,\infty]$について,$g\in L^p(\R^n)$とし,これとの基本解の畳み込み
    \[u(t,x):=\int_{\R^n}\Phi(x-y,t)g(y)dy=\frac{1}{(4\pi t)^{n/2}}\int_{\R^n}e^{-\frac{\abs{x-y}^2}{4t}}g(y)dy,\qquad(x\in\R^n,t>0).\]
    を考える.
    \begin{enumerate}
        \item $\R^n\times\R^+$上滑らか$u\in C^\infty(\R^n\times\R^+)$.
        \item 方程式\ref{HE}を満たす:$u_t-\Lap u=0$.
        \item 任意の$t>0$について,畳み込み$\Phi*:L^p(\R^n)\to L^p(\R^n)$はノルム減少的.すなわち,
        $u(t,-)\in L^p(\R^n)$で,$\norm{u(t,-)}_{L^p(\R^n)}\le\norm{g}_{L^p(\R^n)}$.
        \item $p<\infty$ならば,$u(t,-)\xrightarrow[t\to0]{L^p(\R^n)}g$.
        \item $p=\infty$かつ$g\in C(\R^n)$のとき,各点$x^0\in\R^n$で,$\lim_{\R^n\times\R^+\ni(x,t)\to(x^0,0)}u(x,t)=g(x^0)$.
        \item 特に$g\in UC(\R^n)$ならば,一様に$u(t,-)\xrightarrow[t\to0]{L^\infty(\R^n)}g$.
    \end{enumerate}
\end{theorem}
\begin{Proof}\mbox{}
    \begin{enumerate}
        \item 
        $\Phi$は原点$(0_n,0)$を除いて無限回微分可能であり,
        導関数は任意の$\R^n\times\cointerval{\delta,\infty}\;(\delta>0)$上で一様に有界である
        \ref{prop-property-of-fundamental-solution-to-EH}(5)ことを認めれば,すぐに従う.
        
        実際,任意の$(x,t)\in\R^n\times\R^+$に対して,十分小さく$\delta\in(0,t)$と取ることで,この上では
        微分と積分の交換は可能であり,$(0,\delta)$上での積分も無視できるほど小さいことから,$u$の可微分性が従う.
        \item 同様の議論により,\ref{HE}を満たすことは,
        \[u_t(x,t)-\Lap u(x,t)=\int_{\R^n}((\Phi_t-\Lap_x\Phi)(x-y,t))g(y)dy,\quad(x\in\R^n,t>0)\]
        より,
        \[\Phi_t=\Lap_x\Phi\]
        に同値.
        \item Youngの不等式と,任意の$t>0$について$\norm{\Phi(-,t)}_{L^1(\R^n)}=1$であることによる.
        \item $(\Phi(-,t))_{t>0}$は総和核であるため\ref{prop-property-of-fundamental-solution-to-EH}.
        \item ノルム収束ではなく各点収束も示せることは,次のように緻密に議論する必要がある.
        任意の$x^0\in\R^n,\ep>0$を取ると,ある$\delta>0$が存在して,任意の$y\in B(x^0,\delta)$について$\abs{g(y)-g(x^0)}<\ep$が成り立つ.
        \begin{enumerate}[(i)]
            \item 任意の$x\in B(x^0,\delta/2)$に対して,$\Phi(-,t)$の$\R^n$上の積分は$1$だから,評価したい対象は
            \begin{align*}
                \abs{u(x,t)-g(x^0)}&=\Abs{\int_{\R^n}\Phi(x-y,t)(g(y)-g(x^0))dy}\\
                &\le\paren{\int_{B(x^0,\delta)}+\int_{\R^n\setminus B(x^0,\delta)}}\Phi(x-y,t)\abs{g(y)-g(x^0)}dy=:I+J.
            \end{align*}
            と分解出来る.
            $I$については即座に,評価
            \[I\le\int_{B(x^0,\delta)}\Phi(x-y,t)\ep dy=\ep\]
            が成り立つ.
            \item $J$については,$x\in B(x^0,\delta/2),\abs{y-x^0}\ge\delta$とすると
            \[\abs{y-x^0}\le\abs{y-x}+\frac{\delta}{2}\le\abs{y-x}+\frac{1}{2}\abs{y-x^0}.\]
            の評価が成り立つから,特に$\abs{y-x}\ge\frac{1}{2}\abs{y-x^0}$に注意すると,
            \begin{align*}
                J&\le2\norm{g}_{L^\infty}\int_{\R^n\setminus B(x^0,\delta)}\Phi(x-y,t)dy\\
                &\le\frac{C}{t^{n/2}}\int_{\R^n\setminus B(x^0,\delta)}e^{-\frac{\abs{x-y}^2}{4t}}dy\\
                &\le\frac{C}{t^{n/2}}\int_{\R^n\setminus B(x^0,\delta)}e^{-\frac{\abs{y-x^0}^2}{16t}}dy\\
                &=C\int_{\R^n\setminus B(x^0,\delta/\sqrt{t})}e^{-\frac{\abs{z}^2}{16}}dz\xrightarrow{t\to+0}0.
            \end{align*}
            ただし,最後はやはり$z:=\frac{y-x^0}{\sqrt{t}}$の変数変換を用いた.
        \end{enumerate}
        \item これは,$(\Phi(-,t))_{t>0}$は総和核であるため\ref{prop-property-of-fundamental-solution-to-EH}.
    \end{enumerate}
\end{Proof}

\begin{observation}[熱方程式は無限伝播速度を許す]
    $g\in C_b(\R^n)_+$を初期値とする.$g\not\equiv 0$ならば,解$u(-,t):=g*\Phi_t$は$\R^n\times\R^+$上で正である.
    これは初期温度がたとえ1点でも正ならば,次の瞬間$t>0$では至る所温度は正になることを言っている.
    これは波動方程式とは対照的な性質である.
\end{observation}

\begin{theorem}[\cite{黒田成俊-関数解析} 5.2節]
    上述の4条件を満たす核$\Phi$は基本解に限る.
\end{theorem}

\subsection{非斉次D-境界値問題の解公式}

\begin{tcolorbox}[colframe=ForestGreen, colback=ForestGreen!10!white,breakable,colbacktitle=ForestGreen!40!white,coltitle=black,fonttitle=\bfseries\sffamily,
title=]
    境界条件を簡単のため$g=0$としても,手のつけようがないように思われるが,
    非斉次項を境界条件と解釈した問題の解を積分したものによって,非斉次方程式の$D$-境界値問題の解が得られる(Duhamelの原則).
\end{tcolorbox}

\begin{problem*}
    非斉次熱方程式に関する
    $\R^n\times\R^+$上のDiriclet境界値問題,すなわち$\R^n$上のCauchy初期値問題
    \[\text{(C-N-HE)}_0\quad\begin{cases}
        u_t-\Lap u=f&\In\R^n\times\R^+,\\
        u=0&\on\R^n\times\{0\}.
    \end{cases}\]
    を考える.ただし,境界条件は簡単のため$g=0$とした.
\end{problem*}

\begin{observation}[Duhamelの発想]
    時点$s\in\R^+$を定め,
    \[u(x,t;s):=\Phi_{t-s}*f_s(x)=\int_{\R^n}\Phi(x-y,t-s)f(y,s)dy\]
    とすると,これは
    \[\begin{cases}
        u_t(-;s)-\Lap u(-;s)=0&\In\R^n\times(s,\infty),\\
        u(-;s)=f(-,s)&\on\R^n\times\{s\}
    \end{cases}\]
    を満たす.つまり,時点$s\in\R^n$での強制項$f(-;s)$を,その時点での初期値と解釈してしまったわけだ.
    無限伝播速度を仮定するならば,これは物理的に等価で,解を与えるはずであり,後は$s\in(0,\infty)$で積分すれば良い.すなわち,
    \begin{align*}
        u(x,t)&:=\int^t_0u(x,t;s)ds\\
        &=\int^t_0\int_{\R^n}\Phi(x-y,t-s)f(y,s)dy\\
        &=\int^t_0\frac{1}{(4\pi (t-s))^{n/2}}\int_{\R^n}e^{-\frac{\abs{x-y}^2}{4(t-s)}}f(y,s)dyds,\qquad(x,t)\in\R^n\times\R^+.
    \end{align*}
    が解の候補である.
\end{observation}
\begin{remark}
    Duhamelの発想は,熱方程式だけでなく,輸送方程式と波動方程式にも通用する.
\end{remark}

\begin{theorem}[非斉次方程式の境界値問題の解公式]\label{thm-solution-for-NHE}
    外力項は$f\in C^{2,1}_c(\R\times\R^+)$を満たすとし,これに対して$u:\R^n\times\R^+\to\R$を
    \[u(x,t):=\int^t_0\frac{1}{(4\pi (t-s))^{n/2}}\int_{\R^n}e^{-\frac{\abs{x-y}^2}{4(t-s)}}f(y,s)dyds\]
    で定める.
    \begin{enumerate}
        \item 解の滑らかさ:$u\in C^{2,1}(\R\times\R^+)$.
        \item 非斉次方程式の解である:$u_t(x,t)-\Lap u(x,t)=f(x,t)\;\In\R^n\times\R^+$.
        \item 各点で境界条件を満たす:$\forall_{x^0\in\R^n}\;\lim_{\R^n\times\R^+\ni(x,t)\to(x^0,0)}u(x,t)=0$.
        実は一様にも満たすが.
    \end{enumerate}
\end{theorem}
\begin{Proof}\mbox{}
    \begin{enumerate}
        \item おそらくどちらでもよいが,$f$の方に変数を移して微分する.すなわち,
        \[u(x,t)=\int^t_0\int_{\R^n}\Phi(x-y,t-s)f(y,s)dyds=\int^t_0\int_{\R^n}\Phi(y,s)f(x-y,t-s)dyds\]
        の最右辺の表示について$t$で微分すると,実は
        \[u_t(x,t)=\int^t_0\int_{\R^n}\Phi(y,s)f_t(x-y,t-s)dyds+\int_{\R^n}\Phi(y,t)f(x-y,0)dyds\]
        となる.これは,動く積分領域上の微分\ref{thm-differentiation-of-integral-on-moving-region}または次のように直接計算することもできる:
        \begin{align*}
            \frac{u(x,t+h)-u(x,t)}{h}&=\frac{1}{h}\paren{\int^{t+h}_0\int_{\R^n}\Phi(y,s)f(x-y,t+h-s)dyds-\int^t_0\int_{\R^n}\Phi(y,s)f(x-y,t-s)dyds}\\
            &=\int^t_0\int_{\R^n}\Phi(y,s)\frac{f(x-y,t+h-s)-f(x-y,t-s)}{h}dyds\\
            &\qquad+\frac{1}{h}\int^{t+h}_t\int_{\R^n}\Phi(y,s)f(x-y,t+h-s)dyds.
        \end{align*}
        であるが,第一項は$f$がコンパクト台を持つために$\frac{f(x-y,t+h-s)-f(x-y,t-s)}{h}\to f_t(x-y,t-s)$は一様収束し,
        第二項は$f$のTaylor展開を考えることで,
        \[\frac{1}{h}\int^{t+h}_t\int_{\R^n}\Phi(y,s)(f(x-y,t-s)+hf_t(x-y,t-s)+O(h^2))dyds\]
        となり,第一項のみ残ることがわかり,それが被積分関数に$s=t$を代入したものになることは
        微積分学の基本定理による.
        $x_i$に関する微分は,$x_i$は積分区間に登場しないため,もっと簡単に
        \[u_{x_ix_j}(x,t)=\int^t_0\int_{\R^n}\Phi(y,s)f_{x_ix_j}(x-y,t-s)dyds,\qquad i,j\in[n].\]
        以上より,$u_t,D_x^2u$も存在して連続である.
        \item (1)での計算より,$u_t-\Lap u$の値は積分の形で次のように求まっているから,これが$f$に等しいことを示せば良い:
        \begin{align*}
            u_t(x,t)-\Lap_xu(x,t)&=\int^t_0\int_{\R^n}\Phi(y,s)\paren{\pp{}{t}-\Lap_x}f(x-y,t-s)dyds\\
            &\qquad+\int_{\R^n}\Phi(y,s)f(x-y,0)dy\\
            &=\int^t_\ep\int_{\R^n}\Phi(y,s)\paren{-\pp{}{s}-\Lap_y}f(x-y,t-s)dyds\\
            &\qquad+\int^\ep_0\int_{\R^n}\Phi(y,s)\paren{-\pp{}{s}-\Lap_y}f(x-y,t-s)dyds\\
            &\qquad+\int_{\R^n}\Phi(y,s)f(x-y,0)dy\\
            &=:I_\ep+J_\ep+K
        \end{align*}
        とする.ただし,途中で$\partial_tf(x-y,t-s)=-\partial_sf(x-y,t-s)$と$\partial_{x_ix_i}f(x-y,t-s)=\partial_{y_iy_i}f(x-y,t-s)$を用いた.
        \begin{description}
            \item[第二項$J_\ep$] 特異点の近傍での積分$J_\ep$は,積分区間が小さいので,被積分関数の評価は殆どせずとも
            \[\abs{J_\ep}\le\paren{\norm{f_t}_{L^\infty}+\norm{D^2f}_{L^\infty}}\int^\ep_0\int_{\R^n}\Phi(y,s)dyds=\paren{\norm{f_t}_{L^\infty}+\norm{D^2f}_{L^\infty}}\ep.\]
            と評価できる.
            \item[第一項$I_\ep$] 特異点を除いた積分$I_\ep$には部分積分を用いる.
            まず,$\partial_s$についてのみ部分積分を用いると,
            \begin{align*}
                I_\ep&=\int^t_\ep\int_{\R^n}\Phi(y,s)\paren{-\pp{}{s}-\Lap_y}f(x-y,t-s)dyds\\
                &=-\int_{\partial(\R^n\times(\ep,t))}\Phi(y,s)f(x-y,t-s)dyds\\
                &\qquad+\int_{\R^n\times(\ep,t)}\underbrace{\paren{\pp{}{s}-\Lap y}\Phi(y,s)}_{=0}f(x-y,t-s)dyds\\
                &=\int_{\R^n}\Phi(y,\ep)f(x-y,t-\ep)dy-\int_{\R^n}\Phi(y,t)f(x-y,0)dy\\
                &=\Phi_\ep*f_{t-\ep}(x)-K.
            \end{align*}
        \end{description}
        以上の考察より,次のようにして結論が導かれる:
        \begin{align*}
            u_t(x,t)-\Lap_xu(x,t)&=\lim_{\ep\to0}\Phi_\ep*f_{t-\ep}(x)\\
            &=f(x,t),\qquad(x,t)\in\R^n\times\R^+.
        \end{align*}
        この収束を厳密に議論するには,例えば斉次な場合の境界値の議論\ref{thm-solution-for-HHE}(4)と同様にできる.
        \item まず,畳み込みのノルム減少性より,
        \begin{align*}
            \abs{u(-,t)}&\le\int^t_0\abs{u(-,t;s)}ds\\
            &\le\int^t_0\norm{f(-,t;s)}_{L^\infty(\R^n)}ds\\
            &\le t\norm{f}_{L^\infty(\R^n\times\R^+)}
        \end{align*}
        であるから,$\norm{u(-,t)}_{L^\infty(\R^n)}\le t\norm{f}_{L^\infty(\R^n\times\R^+)}$.よって,$x^0$に依らず,
        $t\searrow0$について,$u(x,t)\to0$.
    \end{enumerate}
\end{Proof}
\begin{remarks}
    $f$がコンパクト台を持つという条件は,Poisson方程式(Laplace方程式の非斉次化)の解公式\ref{thm-solution-to-Poisson-equation}における
    場合と同様,$\Phi$の導関数との積が可積分でないために,$f$に微分と積分が交換できる根拠を転嫁するためである.
\end{remarks}

\subsection{一般のD-境界値問題の解公式}

\begin{tcolorbox}[colframe=ForestGreen, colback=ForestGreen!10!white,breakable,colbacktitle=ForestGreen!40!white,coltitle=black,fonttitle=\bfseries\sffamily,
title=]
一般のDirichlet問題は,$f=0$とした場合と$g=0$とした場合の解の和で与えられる.
\end{tcolorbox}

\begin{corollary}[重ね合わせの原理]
    一般の非斉次熱方程式の境界値問題
    \[\begin{cases}
        u_t-\Lap u=f&\In\R^n\times\R^+,\\
        u=g&\on\R^n\times\{0\}.
    \end{cases}\qquad f\in C^{2,1}_c(\R^n\times\R^+),g\in C_b(\R^n).\]
    の解は,線型和
    \[u(x,t):=\int_{\R^n}\Phi(x-y,t)g(y)dy+\int^t_0\int_{\R^n}\Phi(x-y,t-s)f(y,s)dyds\]
    が与える.
\end{corollary}
\begin{Proof}
    第一項は定理\ref{thm-solution-for-HHE},第二項は定理\ref{thm-solution-for-NHE}が与え,$\R^n\times\{0\}$上では第二項が消えて$g$のみが残り,$\R^n\times\R^+$上での$u_t-\Lap u$では第一項が消えて第二項のみが残る.
\end{Proof}

\section{平均値の公式}

\begin{tcolorbox}[colframe=ForestGreen, colback=ForestGreen!10!white,breakable,colbacktitle=ForestGreen!40!white,coltitle=black,fonttitle=\bfseries\sffamily,
title=]
    調和関数のような平均値の性質は持たず,公式は複雑になる.
    しかし,基本解$\Phi$はLaplace方程式の場合と同様,熱球面上で一定の値を取り(動径$\abs{x}$の関数),熱球面は等位集合になっている.
\end{tcolorbox}

\begin{notation}[parabolic cylinder, parabolic boundary,  heat ball]
    $U\osub\R^n$を有界開集合,
    $T>0$を時点とし,次の記法を定める:
    \begin{enumerate}
        \item ${U_T:=U\times(0,T]}$を\textbf{放物円筒}という.
        \item これは,$\o{U}\times[0,T]$の\textbf{放物内部}という.
        位相的な内部$U\times(0,T)$より$\o{U}\times\{T\}$だけ大きい点に注意.
        \item $U_T$の\textbf{放物境界}とは,$\Gamma_T:=\o{U}_T\setminus U_T=\partial U\times[0,T]\cup\o{U}\times\{0\}$とする.位相的な境界とは$\o{U}\times\{T\}$だけ小さい点に注意.
        \item $x\in\R^n,t\in\R,r>0$に対して,
        開球にあたる概念・\textbf{熱球}を,基本解の等位集合の合併として
        \[E(x,t;r):=\Brace{(y,s)\in\R^{n+1}\;\middle|\;s\le t,\Phi(x-y,t-s)\ge\frac{1}{r^n}}.\]
        と定める.$\Phi$は$\partial E(x,t;r)$上で一定値$1/r^n$を取ることに注意.また,$(x,t)$は$E(x,t;r)$の位相的な境界点にあり,$t$が一番大きいものとして得られることに注意.
        実際,平均値の公式が$t$よりも先の時点の情報を用いていることは物理的な論理に反することになる.
    \end{enumerate}
\end{notation}

\begin{theorem}[熱方程式に対する平均値の公式]
    $U$を有界開集合,$u\in C^{2,1}(U_T)$を$U_T:=U\times\ocinterval{0,T}$上の熱方程式の解とする.
    すると,任意の熱球$E(x,t;r)\subset U_T$に対して,
    \[u(x,t)=\frac{1}{4r^n}\iint_{E(x,t;r)}u(y,s)\frac{\abs{x-y}^2}{(t-s)^2}dyds.\]
\end{theorem}
\begin{Proof}
    熱球$E(x,t;r)\subset U_T$を任意にとる.すると,平行移動した関数$v(y,s):=u(y+x,s+t)$を考えると,
    \[(y+x,s+t)\in E(x,t;r)\quad\Leftrightarrow\quad(y,s)\in E(0,0;r)\]
    に注意すれば,
    \[v(x,t)=\frac{1}{4r^n}\iint_{E(0,0;r)}v(y,s)\frac{\abs{x-y}^2}{(t-s)^2}dyds\]
    を示せばよい.以降,$v$を$u$と書き,$E(0,0;r)$を$E(r)$と書く.
    \begin{enumerate}[{Step}1]
        \item 関数
        \[\phi(r):=\frac{1}{r^n}\iint_{E(r)}u(y,s)\frac{\abs{y}^2}{s^2}dyds\]
        の微分を考える.まず,
        \[r^n\Phi(y,s)=\frac{1}{\sqrt{4\pi(s/r^2)}^n}e^{-\frac{\abs{y/r}^2}{s/r^2}}=\Phi(y/r,s/r^2)\]
        に注意すれば,変数変換$x:=y/r,t:=s/r^2$により
        \begin{align*}
            \phi(r)&=\frac{1}{r^n}\iint_{E(1)}u(rx,r^2t)\frac{r^2\abs{x}^2}{r^4t^2}r^ndxr^2dt=\iint_{E(1)}u(rx,r^2t)\frac{\abs{x}^2}{t^2}dxdt.
        \end{align*}
        と書き直せる.するとこの微分は
        \begin{align*}
            \phi'(r)&=\iint_{E(1)}\paren{\sum_{i=1}^nu_{x_i}x_i\frac{\abs{x}^2}{t^2}+2ru_t\frac{\abs{x}^2}{t}}dxdt\\
            &=\frac{1}{r^{n+2}}\iint_{E(r)}\paren{\sum_{i=1}^nu_{y_i}\frac{y_i}{r}\frac{r^2\abs{y}^2}{s^2}+2ru_s\frac{\abs{y}^2}{s}}dyds\\
            &=\frac{1}{r^{n+1}}\iint_{E(r)}\paren{\sum_{i=1}^nu_{y_i}y_i\frac{\abs{y}^2}{s^2}+2u_s\frac{\abs{y}^2}{s}}dyds=:A+B.
        \end{align*}
        と計算できる.
        \item まず$B$を評価することを考える.いま,
        \[\partial E(r)=\Brace{(y,s)\in\R^{n+1}\;\middle|\;s\le0,\Phi(-y,-s)=\frac{1}{r^n}}\]
        上では
        \[\Phi(-y,-s)=\frac{1}{(-4\pi s)^{n/2}}e^{\frac{\abs{y}^2}{4s}}=\frac{1}{r^n}\]
        より,特に
        \[\psi(y,s):=\log\Phi(-y,-s)-\log r^{-n}=-\frac{n}{2}\log(-4\pi s)+\frac{\abs{y}^2}{4s}+n\log r=0,\qquad(y,s)\in\partial E(r)\]
        に注目する.
        以降,
        \[\psi_s(y,s)=-\frac{n}{2}\frac{1}{s}-\frac{\abs{y}^2}{4s^2},\quad\psi_{y_i}(y,s)=\frac{y_i}{2s}.\]
        を用いる.
        $y_i$に関する
        部分積分により,$B$は
        \begin{align*}
            B&=\frac{1}{r^{n+1}}\iint_{E(r)}\paren{\sum_{i=1}^n2u_s\frac{y_i^2}{s}}dyds\\
            &=\frac{1}{r^{n+1}}\iint_{E(r)}\paren{4\sum_{i=1}^nu_sy_i\psi_{y_i}}dyds\\
            &=-\frac{1}{r^{n+1}}\iint_{E(r)}4\sum_{i=1}^n\paren{u_{sy_i}y_i\psi+_s\psi}dyds\\
            &=-\frac{1}{r^{n+1}}\iint_{E(r)}\paren{4nu_s\psi+4\sum_{i=1}^nu_{sy_i}y_i\psi}dyds.
        \end{align*}
        と変形できる.引き続き第二項に,$s$に関する部分積分を考えることで,
        \begin{align*}
            B&=\frac{1}{r^{n+1}}\iint_{E(r)}\paren{-4nu_s\psi+4\sum_{i=1}^nu_{y_i}y_i\psi_s}dyds\\
            &=\frac{1}{r^{n+1}}\iint_{E(r)}\paren{-4nu_s\psi+4\sum_{i=1}^nu_{y_i}y_i\paren{-\frac{n}{2s}-\frac{\abs{y}^2}{4s^2}}}dyds\\
            &=\frac{1}{r^{n+1}}\iint_{E(r)}\paren{-4nu_s\psi-\frac{2n}{s}\sum_{i=1}^nu_{y_i}y_i}dyds-\underbrace{\frac{1}{r^{n+1}}\iint_{E(r)}\frac{\abs{y}^2}{s^2}\sum_{i=1}^nu_{y_i}y_idyds}_{=A}
        \end{align*}
        \item 以上の計算と,$u$が熱方程式の解であることを併せれば,部分積分から
        \begin{align*}
            \psi'(r)&=A+B=\frac{1}{r^{n+1}}\iint_{E(r)}\paren{-4nu_s\psi-\frac{2n}{s}\sum_{i=1}^nu_{y_i}y_i}dyds\\
            &=\frac{1}{r^{n+1}}\iint_{E(r)}\paren{-4n\underbrace{\Lap u\psi}_{=\sum_{i=1}^nu_{y_iy_i}\psi}-\frac{2n}{s}\sum_{i=1}^nu_{y_i}y_i}dyds\\
            &=\frac{1}{r^{n+1}}\iint_{E(r)}\sum_{i=1}^n\paren{4nu_{y_i}\psi_{y_i}-\frac{2n}{s}u_{y_i}y_i}dyds\\
            &=\frac{1}{r^{n+1}}\sum_{i=1}^n\iint_{E(r)}u_{y_i}\paren{4n\psi_{y_i}-\frac{2n}{s}y_i}dyds=0.
        \end{align*}
        が解る.よって関数$\psi$は定数であるから,次の補題より,
        \begin{align*}
            \phi(r)&=\lim_{r\to0}\phi(r)=\lim_{r\to0}\frac{1}{r^n}\iint_{E(r)}u(y,s)\frac{\abs{y}^2}{s^2}dyds\\
            &=\lim_{r\to0}\iint_{E(1)}u(rx,r^2t)\frac{\abs{x}^2}{t^2}dxdt\\
            &=u(0,0)\iint_{E(1)}\frac{\abs{x}^2}{t^2}dxdt=4u(0,0).
        \end{align*}
        を得る.途中の収束は,$E(1)$がコンパクトであることに注意すれば,
        $\{u(rx,r^2t)\}_{0\le r\le 1}$がすべて可積分であるため,Lebesgueの優収束定理による.
    \end{enumerate}
\end{Proof}

\begin{lemma}
    \[\iint_{E(1)}\frac{\abs{x}^2}{t^2}dxdt=4.\]
\end{lemma}
\begin{Proof}
    \[E(1)=\Brace{(y,s)\in\R^{n+1}\;\middle|\;s\le0,\Phi(-y,-s)\ge1}\]
    であるが,
    \begin{align*}
        \Phi(-y,-s)&=(-4\pi s)^{-\frac{n}{2}}e^{\frac{\abs{y}^2}{4s}}\ge1\\
        &\Leftrightarrow e^{\frac{\abs{y}^2}{4s}}\ge(-4\pi s)^{\frac{n}{2}}\\
        &\Leftrightarrow \frac{\abs{y}^2}{4s}\ge\frac{n}{2}\log(-4\pi s)\\
        &\Leftrightarrow\abs{y}^2\le 2ns\log(-4\pi s).
    \end{align*}
    と同値変形できる.$e^{\frac{\abs{y}^2}{4s}}\le1$に注意すれば,$s\in\Square{-\frac{1}{4\pi},0}$が必要であるため,特に$E(1)$はコンパクトで,次のように表せる:
    \[E(1)=\Brace{(y,s)\in\R^{n+1}\;\middle|\;-\frac{1}{4\pi}\le s\le0,\abs{y}^2\le 2ns\log(-4\pi s)}.\]
    $R:=\sqrt{2ns\log(-4\pi s)}$とおく.
    よって,
    積分は次のように計算できる:
    \begin{align*}
        \iint_{E(1)}\frac{\abs{y}^2}{s^2}dyds&=\int^0_{-1/4\pi}\int^R_0\int_{\partial B(0,r)}\frac{r^2}{s^2}dSdrds\\
        &=\int^0_{-1/4\pi}\int^R_0\frac{n\om_nr^{n+1}}{s^2}drds&\because\abs{\partial B(0,r)}=r^{n-1}n\om_n\\
        &=\int^0_{-1/4\pi}\frac{n\om_n}{s^2}\frac{R^{n+2}}{n+2}ds\\
        &=\frac{2n}{n+2}(2n)^{n/2}n\om_n\int^0_{-1/4\pi}s^{\frac{n-2}{2}}(\log(-4\pi s))^{\frac{n+2}{2}}ds.
    \end{align*}
    ここで,変数変換$z:=-\log(-4\pi s)$より,
    \begin{align*}
        \int^0_{-1/4\pi}s^{\frac{n-2}{2}}(\log(-4\pi s))^{\frac{n+2}{2}}ds&=\int^\infty_0\paren{\frac{e^{-z}}{4\pi}}^{\frac{n}{2}}z^{\frac{n+2}{2}}dz\\
        &=\frac{1}{(4\pi)^{\frac{n}{2}}}\int^\infty_0e^{-w}w^{\frac{n}{2}+2-1}\paren{\frac{2}{n}}^{\frac{n}{2}+2}dw\\
        &=\frac{1}{(2n\pi)^{n/2}}\frac{4}{n^2}\Gamma\paren{\frac{n}{2}+2}.
    \end{align*}
    と計算できるから,
    $n$次元単位球の表面積の関係
    \[n\om_n=\frac{2\pi^{n/2}}{\Gamma\paren{\frac{n}{2}}}\]
    に注意すれば,総じて,
    \begin{align*}
        \iint_{E(1)}\frac{\abs{y}^2}{s^2}dyds&=\frac{2n}{n+2}(2n)^{n/2}n\om_n\frac{1}{(2n\pi)^{n/2}}\frac{4}{n^2}\Gamma\paren{\frac{n}{2}+2}\\
        &=\frac{2n}{n+2}2\frac{4}{n^2}\frac{n+2}{2}\frac{n}{2}=4.
    \end{align*}
\end{Proof}

\section{最大値原理と解の一意性}

\begin{tcolorbox}[colframe=ForestGreen, colback=ForestGreen!10!white,breakable,colbacktitle=ForestGreen!40!white,coltitle=black,fonttitle=\bfseries\sffamily,
title=]
    平均値の公式から,最大値の原理が従う論理の流れはLaplace方程式に類似する.ここから解の一意性も得られる.
    が,上半平面上での一意性は,「解が爆発しないものの中で」という制限が必要になる.
\end{tcolorbox}

\subsection{最大値の原理}

\begin{theorem}[熱方程式に対する最大値原理]
    $U$を有界開集合,
    $u\in C^{2,1}(U_T)\cap C(\o{U_T})$は$U_T$上の熱方程式の解とする.
    \begin{enumerate}
        \item (弱) $\max_{\o{U_T}}u=\max_{\Gamma_T}u$.
        \item $U$は連結とする.ある$(x_0,t_0)\in U_T$において$u(x_0,t_0)=\max_{\o{U_T}}u$を達成するならば,$u$は$\o{U_{t_0}}$上定数である.
    \end{enumerate}
\end{theorem}
\begin{Proof}
    (2)を示せば(1)は従う.
    点$(x_0,t_0)\in U_T$にて$u(x_0,t_0)=\max_{\o{U_T}}u=:M$を達成するとする.
    \begin{enumerate}[{Step}1]
        \item 任意の$E(x_0,t_0;r)\subset U_T$について,平均値の定理から
        \[M=u(x_0,t_0)=\frac{1}{4r^n}\iint_{E(x_0,t_0;r)}u(y,s)\frac{\abs{x_0-y}^2}{(t_0-s)^2}dyds\]
        が成り立つが,
        補題より,
        \[\frac{1}{4r^n}\iint_{E(x_0,t_0;r)}\frac{\abs{x_0-y}^2}{(t_0-s)^2}dyds=1\]
        であることから,
        \[\iint_{E(x_0,t_0;r)}\Paren{u(y,s)-M}\frac{\abs{x_0-y}^2}{(t_0-s)^2}dyds=0.\]
        $u$が連続であることと併せれば,$u\equiv M\;\on E(x_0,t_0;r)$を得る.
        \item $(y_0,s_0)\in U_T$で$s_0<t_0$を満たす点と$(x_0,t_0)$とを結ぶ線分$L$は$U_T$に含まれるとすると,$u\equiv M\;\on L$である.
        
        実際,
        \[r_0:=\min\Brace{s\ge s_0\mid \forall_{(x,t)\in L}\;s\le t\le t_0\Rightarrow u(x,t)=M}.\]
        が$r_0>s_0$を満たすとすると矛盾が導ける.$u$は連続であるから,$r_0$の定義でminを取っている所の集合は閉集合である.特に,$r_0$について,
        $(z_0,r_0)\in L$を満たす任意の$z_0\in U$について,$u(z_0,r_0)=M$を満たす.
        よってStep1から,任意の$E(z_0,r_0;r)\subset U_T$について,$u(z_0,r_0)\equiv M\;\on E(z_0,r_0;r)$を満たす.
        このとき,$E(z_0,r_0;r)$はある$\ep>0$について$L\cap\Brace{r_0-\ep\le t\le r_0}$という形の集合を含むから,$r_0$の最小性に矛盾する.
        \item あとは,任意の点$(x,t)\in U\times\cointerval{0,T}$が$(x_0,t_0)$と線分の有限個のつなぎ合わせによって結べることを示せばよい.
        
        まず,$U$が連結であるために,ある有限個の点$x_0,x_1,\cdots,x_m=x$であって,これらを結ぶ折れ線は$U$内に存在するように出来る.
        これは,この性質を持つ$U$の点$x\in U$の全体$V\subset U$は,非空の,$U$の開かつ閉集合であるためである.
        $V$が開集合であることは明らか.閉集合であることも,$V$の任意の収束列$\{x_n\}\subset V$に対して,$U$内の開球$B\osub U$であって$x_n$を無限個含むものが取れるから,$n_0:=\min\Brace{n\in\N\mid x_n\in B}$とすれば,$[x_0,x_{n_0}],[x_{n_0},x_\infty]$は$x_0$と$x_\infty:=\lim_{n\to\infty}x_n$を結ぶから,$x_\infty\in V$である.

        これに対して,対応するだけの$t_0>t_1>\cdots>t_m=t$を任意に取れば,$(x_0,t_0),(x_1,t_1),\cdots,(x_m,t_m)$を結ぶ折れ線は$U_T$に含まれる.
    \end{enumerate}
\end{Proof}
\begin{remarks}
    $u$が全運命$U_T$上での最大値をある時点$t_0$のある地点$x_0\in U$にて取るならば,
    $t_0$以前は$u$は定数であることが必要.
\end{remarks}

\begin{corollary}
    $U$を有界領域とする.$u\in C^{2,1}(U_T)\cap C(\o{U_T})$が,非負なデータ$g\ge0$に関する
    斉次熱方程式の初期値問題
    \[\begin{cases}
        u_t-\Lap u=0&\In U_T,\\
        u=0&\on \partial U\times[0,T],\\
        u=g&\on U\times\{0\}
    \end{cases}\]
    の解であるとする.
    このとき,$g\not\equiv0$ならば,$u$は$U_T$上正である.
\end{corollary}
\begin{Proof}
    $U_T=U\times\ocinterval{0,T}$で非正な値をとる点が存在するならば,
    それは$\Gamma_T=\partial U\times[0,T]\cup U\times\{0\}$上の最小値より小さい値だから,
    弱最小値定理に矛盾する.
\end{Proof}
\begin{remarks}
    これは無限の伝播速度を持つという消息のもう一つの現出である.
\end{remarks}

\subsection{有界領域上の解の一意性}

\begin{theorem}[有界領域上の初期値問題の解の一意性]
    $U$を有界開集合とする.データ$g\in C(\Gamma_T),f\in C(U_T)$が連続ならば,これらが定める非斉次熱方程式の境界値問題
    \[\begin{cases}
        u_t-\Lap u=f&\In U_T,\\
        u=g&\on \Gamma_T
    \end{cases}\]
    の解$u\in C^{2,1}(U_T)\cap C(\o{U_T})$は高々1つである.
\end{theorem}
\begin{Proof}
    $u,\wt{u}$はいずれも解であるとすると,$w:=u-\wt{u}$は
    \[\begin{cases}
        u_t-\Lap u=0&\In U_T\\
        u=0&\on\Gamma_T
    \end{cases}\]
    を満たすが,強い意味での最大値最小値原理より,$w=0$が必要.
\end{Proof}

\subsection{上半平面上の弱最大値の原理}

\begin{theorem}[非有界領域上の最大値の原理]
    斉次熱方程式の初期値問題
    \[\begin{cases}
        u_t-\Lap u=0&\In\R^n\times(0,T),\\
        u=g&\on\R^n\times\{0\}.
    \end{cases}\]
    の解$u\in C^{2,1}(\R^n\times\ocinterval{0,T})\cap C(\R^n\times[0,T])$は,増加速度についての条件
    \[\exists_{a,A>0}\;u(x,t)\le Ae^{a\abs{x}^2}\quad(x\in\R^n,t\in[0,T])\]
    を満たすとする.このとき,$\sup_{\R^n\times[0,T]}u=\sup_{\R^n}g$が成り立つ.
\end{theorem}

\subsection{上半平面上の解の一意性}

\begin{tcolorbox}[colframe=ForestGreen, colback=ForestGreen!10!white,breakable,colbacktitle=ForestGreen!40!white,coltitle=black,fonttitle=\bfseries\sffamily,
title=]
    熱方程式に関する大域的なCauchy問題も,$O(e^{a\abs{x}^2})$を満たす解は一意的である.
\end{tcolorbox}

\begin{theorem}[Cauchy問題の一意性]\label{thm-uniqueness-of-upper-half-plane}
    $g\in C(\R^n),f\in C(\R^n\times[0,T])$を連続なデータとする.このとき,非斉次熱方程式の初期値問題
    \[\begin{cases}
        u_t-\Lap u=f&\In\R^n\times(0,T),\\
        u=g&\on\R^n\times\{0\}.
    \end{cases}\]
    の解$u\in C^{2,1}(\R^n\times\ocinterval{0,T})\cap C(\R^n\times[0,T])$であって,
    増加速度についての条件
    \[\exists_{a,A>0}\;u(x,t)\le Ae^{a\abs{x}^2}\quad(x\in\R^n,t\in[0,T])\]
    を満たすものは高々1つである.
\end{theorem}

\begin{remarks}
    実は,
    斉次熱方程式の零な初期値問題
    \[\begin{cases}
        u_t-\Lap u=f&\In\R^n\times(0,T),\\
        u=0&\on\R^n\times\{0\}.
    \end{cases}\]
    の解は無限個ある.
    自明な解$u\equiv0$を除いて,それらの解は$\abs{x}\to\infty$の極限について極めて速く増加し,その意味でもはや「非物理的」である.
    同様の現象はHamilton-Jacobi方程式と保存量の議論でも起こる.
\end{remarks}

\section{解の可微分性}

\begin{tcolorbox}[colframe=ForestGreen, colback=ForestGreen!10!white,breakable,colbacktitle=ForestGreen!40!white,coltitle=black,fonttitle=\bfseries\sffamily,
title=]
    $t$についての微分と$x$についての微分とでふるまいがちがく,$x\mapsto u(x,t)$は任意の$t\ocinterval{0,T}$について実解析的であるが,$t\mapsto u(x,t)$はそうではないことに関係する\cite{Evans}.
\end{tcolorbox}

\begin{notation}
    \[C(x,t;r):=\Brace{(y,s)\in\R^{n+1}\mid \abs{x-y}\le r,t-r^2\le s\le t}\]
    を円筒とする.$(x_0,t_0)$を上面の半径$r$の円の中心とし,高さを$r^2$とする円筒となっている.
\end{notation}

\begin{theorem}[熱方程式の解の可微分性]\label{thm-smoothness-of-solution-to-HE}
    $U$を開集合とする.
    $u\in C^{2,1}(U_T)$は$U_T$上の熱方程式の解であるとする.
    このとき,$u\in C^\infty(U_T)$が成り立つ.
\end{theorem}
\begin{Proof}
    任意の$(x_0,t_0)\in U_T$と$C:=C(x_0,t_0;r)\subset U_T$をとる.
    $(\eta_\ep)_{\ep>0}$を$(x,t)\in\R^{n+1}$上の軟化子とし,
    $u^\ep:=\eta_\ep*u$と表す.
    $n$次元閉球を$B(x,r):=\Brace{y\in\R^n\mid\abs{y-x}\le r}$と表す.
    $C':=C(x_0,t_0;3r/4)$上で$1$で,$C$の放物境界$\partial B(x_0,r)\times[t_0-r^2,t_0]\cup B(x_0,r)\times\{t_0-r^2\}$の近傍で$0$な,$C$上に台を持つ可微分関数$\zeta$を取り,
    これを用いてカットオフをかけたものを
    \[v^\ep(x,t):=\zeta(x,t)u^\ep(x,t),\qquad x\in\R^n,t\in[0,t_0]\]
    と表す.
    \begin{enumerate}[{Step}1]
        \item いま,
        \[v_t^\ep=\zeta u^\ep_t+\zeta_tu^\ep,\quad\Lap v^\ep=\zeta\Lap u^\ep+2(D\zeta|Du^\ep)+u^\ep\Lap\zeta\]
        であるから,
        \[\begin{cases}
            v^\ep_t-\Lap v^\ep=\zeta_tu^\ep-2(D\zeta|Du^\ep)-u^\ep\Lap\zeta=:\wt{f}&\In\R^n\times(0,t_0),\\
            v^\ep=0&\on\R^n\times\{0\}.
        \end{cases}\]
        を満たす.
        $u^\ep$は$U_T$の近傍で可微分であるために,解公式を用いるための$f$の可微分性の条件はみたされており,また$\zeta$の存在よりコンパクトな台を持つ.
        さらに$v^\ep$はこの有界な解であるから,一意性\ref{thm-uniqueness-of-upper-half-plane}から
        \[v^\ep(x,t)=\int^t_0\int_{\R^n}\Phi(x-y,t-s)\wt{f}(y,s)dyds\]
        を結論付けられる.
        \item 任意の$(x,t)\in C'':=C(x_0,t_0;r/2)$について,十分小さい$\ep>0$についてこの上で$u^\ep=v^\ep$であるから,
        \begin{align*}
            u^\ep(x,t)&=\iint_C\Phi(x-y,t-s)\wt{f}(y,s)dyds\\
            &=\iint_C\Phi(x-y,t-s)\Paren{(\zeta_s(y,s)-\Lap\zeta(y,s))u^\ep(y,s)-2(D\zeta(y,s)|Du^\ep(y,s))}dyds.
        \end{align*}
        $\Phi(x-y,t-s)$は$(y,s)=(x,t)$で特異性を持つが,
        $(x,t)\in C''$の近傍で$\wh{f}=v_t^\ep-\Lap v^\ep=u_t^\ep-\Lap u^\ep=0$より,部分積分によって次のように計算を進めることができる:
        \begin{align*}
            u^\ep(x,t)&=\iint_C\Phi(x-y,t-s)(\zeta_s-\Lap\zeta)u^\ep dyds+2\iint_C\Paren{(D_y\Phi(x-y,t-s)|D\zeta)u^\ep+\Phi(x-y,t-s) u^\ep\Lap\zeta} dyds\\
            &=\iint_C\Paren{\Phi(x-y,t-s)(\zeta_s+\Lap\zeta)+2(D_y\Phi(x-y,t-s)|D\zeta)}u^\ep dyds.
        \end{align*}
        $\ep\to0$の極限を取ることで,
        \[u(x,t)=\iint_C\Paren{\Phi(x-y,t-s)(\zeta_s+\Lap\zeta)+2(D_y\Phi(x-y,t-s)|D\zeta)}u dyds.\]
        を得る.
        \item Step2で得た$C''$上での$u$の表示の積分核
        \[K(x,t,y,s):=\Phi(x-y,t-s)(\zeta_s+\Lap\zeta)+2(D_y\Phi(x-y,t-s)|D\zeta)\]
        は$C'$上で$0$で,$C$上で可微分である.よって,$u$は$C''$上で可微分である.
    \end{enumerate}
\end{Proof}
\begin{remark}
    初期値のデータに依存して,$\Gamma_T$上では滑らかではない可能性はあるが,その後すぐに軟化する.
\end{remark}

\section{解の局所評価}

\begin{theorem}
    任意の$k,l\in\N$に対して,定数$C>0$が存在して,
    任意の円筒$C(x,t;r/2)\subset C(x,t;r)\subset U_T$と任意の$U_T$上の熱方程式の解$u$に対して,
    \[\max_{C(x,t;r/2)}\abs{D^k_xD^l_tu}\le\frac{C}{r^{k+2l+n+2}}\norm{u}_{L^1(C(x,t;r))}.\]
\end{theorem}
\begin{Proof}
    $(x,t)=(0,0)$と仮定して議論する.一般の$(x,t)\in U_T$については,$v(y,s):=u(y-x,s-t)$についての議論の帰着させることが出来る.
    \begin{enumerate}[{Step}1]
        \item $C(1):=C(0,0,1)\subset U_T$が成り立つとする.このとき,定理\ref{thm-smoothness-of-solution-to-HE}と同様にして
        \[u(x,t)=\iint_{C(1)}K(x,t,y,s)u(y,s)dyds\qquad(x,t)\in C(1/2).\]
        という表示を得る.$K\in C_c^\infty$より,この微分は$C(1/2)$上で
        \[\abs{D_x^kD_t^lu(x,t)}\le\iint_{C(1)}\abs{D_x^kD_t^lK(x,t,y,s)}\abs{u(y,s)}dyds\le C\norm{u}_{L^1(C(1))}.\]
        と評価できる.
        \item 任意の$C(r)\subset U_T$を取ると,
        \[v(x,t):=u(rx,r^2t)\]
        は$C(1)$の近傍で定義されており,$C(1)$上の熱方程式を満たす.
        よって,Step1での議論から,
        \[\abs{D_x^kD_t^lv(x,t)}\le C\norm{v}_{L^1(C(1))},\qquad (x,t)\in C(1/2).\]
        という評価を得る.
        関係
        \[D_x^kD_t^lv(x,t)=r^{2l+k}D_x^kD_t^lu(rx,r^2t).\]
        \[\norm{v}_{L^1(C(1))}=\iint_{C(1)}\abs{u(rx,r^2t)}dxdt=\frac{1}{r^{n+2}}\iint_{C(r)}\abs{u(y,s)}dyds=\frac{1}{r^{n+2}}\norm{u}_{L^1(C(r))}.\]
        に注意して,結論を得る.
    \end{enumerate}
\end{Proof}


\section{エネルギー法}

\begin{tcolorbox}[colframe=ForestGreen, colback=ForestGreen!10!white,breakable,colbacktitle=ForestGreen!40!white,coltitle=black,fonttitle=\bfseries\sffamily,
title=]
    熱方程式の解の一意性も,最大値の原理とは別に,部分積分を用いて証明できる.
    Laplace方程式との違いは,時間依存性も存在することだけである.
\end{tcolorbox}

\subsection{一意性}

\begin{theorem}
    $U\osub\R^n$を有界開集合,$\partial U$を$C^1$-級,$f,g$を関数とする.
    このとき,次の初期値問題
    \[\begin{cases}
        u_t-\Lap u=f&\In U_T,\\
        u=g&\on \Gamma_T
    \end{cases}\]
    の解$u\in C^{2,1}(\o{U_T})$は高々1つである.
\end{theorem}
\begin{Proof}
    $u,\wt{u}\in C^{2,1}(\o{U}_T)$をいずれも解とする.
    $w:=u-\wt{u}$は
    \[\begin{cases}
        w_t-\Lap w=0&\In U_T,\\
        w=0&\on \Gamma_T.
    \end{cases}\]
    の解である.
    ここで,エネルギー
    \[E(t):=\int_Uw^2(x,t)dx,\qquad t\in[0,T].\]
    を考える.この微分は,部分積分より
    \begin{align*}
        \dot{E}(t)&=2\int_Uww_tdx\\
        &=2\int_Uw\Lap wdx=-2\int_U\abs{Dw}^2dx\le0.
    \end{align*}
    を満たすから,$e$は単調減少.よって,
    \[(0\le)e(t)\le e(0)=0,\qquad t\in[0,T].\]
    これは$w=0\;\In U_T$を意味する.
\end{Proof}

\subsection{後向き一意性}

\begin{tcolorbox}[colframe=ForestGreen, colback=ForestGreen!10!white,breakable,colbacktitle=ForestGreen!40!white,coltitle=black,fonttitle=\bfseries\sffamily,
title=]
    初期値は違うかもしれないが,同じ境界値$g:\partial U\to\R$を$[0,T]$間に経験してきたことはわかっているものとする.
    さらに,時点$T$で$u(-,T)=\wt{u}(-,T)\;\In U$ならば,$U_T$上で等しい.
\end{tcolorbox}

\begin{theorem}
    $U\osub\R^n$を有界開集合,$\partial U$を$C^1$-級,$f,g$を関数とする.$u,\wt{u}\in C^{2,1}(\o{U_T})$はいずれも
    \[\begin{cases}
        u_t-\Lap u=0&\In U_T,\\
        u=g&\on\partial U\times[0,T]
    \end{cases}\]
    の解であるとする.このとき,$u(-,T)=\wt{u}(-,T)\;\In U$ならば,$u=\wt{u}\;\In U_T$.
\end{theorem}
\begin{Proof}\mbox{}
    \begin{enumerate}
        \item 
        \begin{description}
            \item[2階微分の計算] \[E(t):=\int_Uw^2(x,t)dx,\qquad t\in[0,T].\]
            の微分は,部分積分より
            \begin{align*}
                \dot{E}(t)&=2\int_Uww_tdx\\
                &=2\int_Uw\Lap wdx=-2\int_U\abs{Dw}^2dx.
            \end{align*}
            さらにもう一度微分すると,
            \begin{align*}
                \ddot{E}(t)&=-4\int_U(Dw|Dw_t)dx\\
                &=4\int_U\Lap ww_tdx=4\int_U(\Lap w)^2dx.
            \end{align*}
            \item[2階微分の評価] 任意の$t\in[0,T]$について$w(-,t)=0\;\on\partial U$より,部分積分とCauchy-Schwarz不等式から
            \begin{align*}
                (\dot{E}(t))^2&=4\paren{\int_U\abs{Dw}^2dx}^2=4\paren{\int_Uw\Lap wdx}^2\\
                &\le\paren{\int_Uw^2dx}\paren{4\int_U(\Lap w)^2dx}=E(t)\ddot{E}(t).
            \end{align*}
        \end{description}
        \item $E\equiv0$を示せば良いから,ある$[t_1,t_2]\subset[0,T]\;(t_1<t_2)$が存在して$\forall_{t\in\cointerval{t_1,t_2}}\;E(t)>0$かつ$E(t_2)=0$を満たすと仮定して矛盾を導く.
        \[f(t):=\log E(t),\qquad t\in\cointerval{t_1,t_2}.\]
        を考えると,(1)での議論より,
        \[f''(t)=\frac{\ddot{E}(t)}{E(t)}-\frac{\dot{E}(t)^2}{E(t)^2}\ge0.\]
        よって,$f$は$(t_1,t_2)$上凸である:
        \[f((1-\tau)t_1+\tau t)\le(1-\tau)f(t_1)+\tau f(t),\qquad t\in(t_1,t_2),\tau\in(0,1).\]
        すなわち,
        \[E((1-\tau)t_1+\tau t)\le E(t_1)^{1-\tau}E(t)^{\tau},\qquad t\in(t_1,t_2),\tau\in(0,1).\]
        であるが,$t=t_2$と取ると,特に
        \[(0\le)E((1-\tau)t_1+\tau t_2)\le E(t_1)^{1-\tau}E(t_2)^\tau=0,\qquad\tau\in(0,1).\]
        より,$E=0\;\on[t_1,t_2]$.
    \end{enumerate}
\end{Proof}

\chapter{波動方程式}

\begin{quotation}
    データを
    \[U\osub\R^n,\Lap:=\Lap_x,\;\begin{cases}
        u:\o{U}\times\R_+\to\R&\text{未知},\\
        f:U\times\R_+\to\R&\text{既知}.
    \end{cases}\]
    として,非斉次方程式
    \[\text{(N-WE)}\quad u_{tt}-\Lap u=f,\quad\on U\times\R^+\]
    を考える.この解は,Laplace方程式と熱方程式の解と大きく振る舞いが違う.
    例えば,$C^\infty$-級ではなく,また有限の伝播速度を持つ.
\end{quotation}

\section{波動方程式の物理的考察}

\begin{tcolorbox}[colframe=ForestGreen, colback=ForestGreen!10!white,breakable,colbacktitle=ForestGreen!40!white,coltitle=black,fonttitle=\bfseries\sffamily,
title=]
    $u(x,t)$は,ある座標軸をあらかじめとっておいて,元々の粒子のその方向への変位を表すスカラー量である.
    また,物理的考察からも,初期条件は変位$u$とその方向への速度$u_t$の2つが必要であることがわかる.
\end{tcolorbox}

\subsection{波動方程式の発生}

\begin{observation}
    $U$を領域とし,
    任意の滑らかな部分領域$V\subset U$にかかる変位正方向への力の大きさは,力の場を$\b{F}:U\to\R^n$とすると,
    \[\dd{^2}{t^2}\int_Vudx=\int_Vu_{tt}dx=-\int_{\partial V}\b{F}\cdot\b{\nu}dS\]
    となる必要がある.よって,これを微分形で
    \[u_{tt}=-\div\b{F}\]
    と得る.考えている物体が弾性体ならば,$\b{F}$はやはり勾配$Du$の関数であり,これに対する線形近似$\b{F}(Du)\approx -a Du$
    \[0=u_{tt}+\div\b{F}(Du)\approx u_{tt}-a\Lap u.\]
    $a=1$としたものが波動方程式である.
\end{observation}

\subsection{Bessel関数}

\begin{tcolorbox}[colframe=ForestGreen, colback=ForestGreen!10!white,breakable,colbacktitle=ForestGreen!40!white,coltitle=black,fonttitle=\bfseries\sffamily,
title=]
    円筒対称性が見られる問題設定において自然に出現する方程式をBessel方程式といい,これを満たす関数をBessel関数または円筒関数という.
    実際,Besselがこの方程式を導入したのは,Kepler方程式を解析的に解く研究の過程においてである.
\end{tcolorbox}

\begin{observation}[分離解の動径方向成分はBessel方程式を満たす]
    極形式\ref{thm-Laplacian-in-polar-coordinate}の2次の波動方程式
    \[u_{tt}=c^2(u_{rr}+r^{-1}u_r+r^{-2}u_{\theta\theta}).\]
    を考える.
    $u=R(r)\Theta(\theta)T(t)$と表せる解が存在するならば,
    $R(r)=:f(\mu r)$で定まる関数$f$は
    次の常微分方程式を満たす必要がある:
    \[x^2f''(x)+xf'(x)+(x^2-\nu^2)f(x)=0,\quad(\nu\in\R).\]
    上の方程式は,\textbf{$\nu$位のBessel方程式}という.
\end{observation}
\begin{Proof}
    波動方程式に代入して
    \[\frac{T''}{c^2T}=\frac{R''}{R}+\frac{R'}{rR}+\frac{\Theta''}{r^2\Theta}\]
    を得る.左辺は$r,\theta$に依らず,右辺は$t$に依らないから,いずれも定数に等しい必要がある.これを$-\mu^2$で表すと,右辺が$-\mu^2$で一定であるための条件は
    \[\frac{r^2R''}{R}+\frac{rR'}{R}+\mu^2r^2=-\frac{\Theta''}{\Theta}\]
    となる.両辺は再び定数である必要がある.これを$\nu^2$で表す.以上より,残る必要条件は
    \[\begin{cases}
        T''+c^2\mu^2T=0\\
        \Theta''+\nu^2\Theta=0\\
        r^2R''(r)+rR'(r)+(\mu^2r^2-\nu^2)R(r)=0.
    \end{cases}\]
    となる.前二者は簡単な線型二階ODEである.
    後者は,変数変換$x=\mu r$により,$R(r)=f(\mu r)$とすると,
    \[R'(r)=\mu f'(\mu r),\quad R''(r)=\mu^2 f''(\mu r)\]
    で,方程式は
    \[\paren{\frac{x}{\mu}}^2\mu^2f''(x)+\frac{x}{\mu}\mu f'(x)+(x^2-\nu^2)f(x)=0\]
    と簡略化される.
\end{Proof}

\begin{definition}
    $\nu$位のBesselの微分方程式の解であって,$J_\nu$を\textbf{第一種Bessel関数}という.
\end{definition}

\section{定数係数輸送方程式}

\begin{tcolorbox}[colframe=ForestGreen, colback=ForestGreen!10!white,breakable,colbacktitle=ForestGreen!40!white,coltitle=black,fonttitle=\bfseries\sffamily,
title=]
    波動方程式の対称性を考察する代わりに,輸送方程式を考察しよう.
    波動は状態の輸送であり,この形に落ちるのである.
\end{tcolorbox}

\begin{problem}
    \[u_t+b\cdot Du=0\;\In\R^n\times\R^+\qquad b\in\R^n.\]
    を考える.
\end{problem}

\begin{observation}[輸送方程式の幾何学的意味]
    $u$を解とする.
    ある点$(x,t)\in\R^n\times\R^+$に対して,$z(s):=u(x+sb,t+s)\;(s\in\R)$とすると,導関数は
    \[\dd{z(s)}{s}=Du(x+bs,t+s)\cdot b+u_t(x+bs,t+s)=0.\]
    となる必要がある.これは輸送方程式を方向$(b,1)\in\R^{n+1}$に関する微分の言葉で翻訳したことになり,
    解$u$は方向$(b,1)\in\R^{n+1}$については定値である必要があることがわかった.よって,ある初期値を方向ベクトル$(b,1)$に横断する形で与えれば,解がわかりそうだ.
\end{observation}
\begin{remarks}[method of characteristics]
    このように,偏微分方程式を特定の部分多様体上で常微分方程式(これを\textbf{特性方程式}という)に帰着させる方法は\textbf{特性曲線法}(method of characteristics)という.
\end{remarks}

\subsection{初期値問題}

\begin{tcolorbox}[colframe=ForestGreen, colback=ForestGreen!10!white,breakable,colbacktitle=ForestGreen!40!white,coltitle=black,fonttitle=\bfseries\sffamily,
title=]
    輸送方程式の解は,初期の波形$g:\R^n\to\R$が経時的に速度$b$で伝播していく現象を記述する.1次元の場合を
    \[u_t=-bu_x\]
    と見れば,$b>0$かつ$u_x(x,t)<0$のとき,時刻$t$の$x\in\R$での波形は下がりトレンドで,時間が経つと大きくなる$u_t=-bu_x>0$.
    接線の傾きの絶対値の$b$倍が$u_t$となるので,$b$は速さである.
\end{tcolorbox}

\begin{problem}
    \[\text{(C)}\quad\begin{cases}
        u_t+b\cdot Du=0&\In \R^n\times\R^+,\\
        u=g&\R^n\times\{0\}.
    \end{cases}\]
    の形で初期値を与える.
\end{problem}

\begin{observation}\label{observation-homogeneous-transport-eq}
    こうすれば,$(x,t)$を通り,方向ベクトル$(b,1)$を持つ直線は点$(x-tb,0)$で$\R^n\times\{0\}$に交わるから,
    \[u(x,t):=g(x-tb)\qquad(x,t)\in\R^n\times\R_+\]
    が何らかの意味で解を与えるであろう.例えば,$g\in C^1(\R^n)$ならば,$u$は解である.
\end{observation}
\begin{remarks}
    $g\notin C^1(\R^n)$の場合を考えよう.
    その場合は$u$は弱解として復活させる理論が要請されることになる.
\end{remarks}

\subsection{非斉次問題}

\begin{tcolorbox}[colframe=ForestGreen, colback=ForestGreen!10!white,breakable,colbacktitle=ForestGreen!40!white,coltitle=black,fonttitle=\bfseries\sffamily,
title=]
    $f(x,t)$を人為的に加えられる変位と見れば,伝播してくる部分$g(x-bt)$と波のその部分に加えられた変位の総計
    \[\int^t_0f(x+(s-t)b,s)ds\]
    との和になる.
\end{tcolorbox}

\begin{problem}
    \[\text{(NC)}\quad\begin{cases}
        u_t+b\cdot Du=f&\In \R^n\times\R^+,\\
        u=g&\on\R^n\times\{0\}.
    \end{cases}\]
    はどうなるか.
\end{problem}

\begin{observation}\label{observation-nonhomogeneous-transport-eq}
    同様にして$z(s):=u(x+sb,t+s)\;(s\in\R)$と定めると,
    \[\dd{z(s)}{s}=Du(x+sb,t+s)\cdot b+u_t(x+sb,t+s)=f(x+sb,t+s)\]
    となる.解$u$は,$g(x-tb)$に非斉次性による補正項を加えた形であると予想すると,$g(x-tb)=u(x-tb,0)=z(-t)$と表せるから,
    \begin{align*}
        u(x,t)-g(x-tb)&=z(0)-z(-t)\\
        &=\int^0_{-t}\dd{z(s)}{s}ds\\
        &=\int^0_{-t}f(x+sb,t+s)ds\\
        &=\int^t_0f(x+(u-t)b,u)du,\qquad u:=s+t
    \end{align*}
    すなわち,
    \[u(x,t):=g(x-tb)+\int^t_0f(x+(s-t)b,s)ds,\qquad (x,t)\in\R^n\times\R^+\]
    が適切な枠組みで解公式を与えるであろう.
\end{observation}

\begin{remarks}[Duhamelの原理]
    \[\begin{cases}
        u_t+b\cdot Du=f&\In\R^n\times\R^+,\\
        u=0&\on\R^n\times\{0\}.
    \end{cases}\]
    の解は,
    \[\begin{cases}
        u_t+b\cdot Du=0&\In\R^n\times\R^+,\\
        u=f&\on\R^n\times\{s\}.
    \end{cases}\]
    の解
    \[u(x,t;s)=u(x-(t-s)b,s)=f(x-(t-s)b,s)\]
    の積分
    \[u(x,t)=\int^t_0u(x,t;s)ds=\int^t_0f(x-(t-s)b,s)ds.\]
    が与えて,一般の場合はこれらの重ね合わせが解を与える,というだけである.
\end{remarks}

\section{球面平均による大域的な解}

\begin{tcolorbox}[colframe=ForestGreen, colback=ForestGreen!10!white,breakable,colbacktitle=ForestGreen!40!white,coltitle=black,fonttitle=\bfseries\sffamily,
title=]
    Laplace方程式や熱方程式のような方程式の対称性を考察せずとも,$n=1$の場合は直接,$n\ge2$の場合も球面平均によって解を得ることができる.
\end{tcolorbox}

\subsection{1次元のd'Alembertの公式}

\begin{tcolorbox}[colframe=ForestGreen, colback=ForestGreen!10!white,breakable,colbacktitle=ForestGreen!40!white,coltitle=black,fonttitle=\bfseries\sffamily,
title=]
    \begin{enumerate}
        \item 1次元の波動方程式は\textbf{2階の輸送方程式},あるいは輸送方程式の連立系
        \[\begin{cases}
            (\partial_t-\partial_x)u=v,\\
            (\partial_t+\partial_x)v=0.
        \end{cases}\]
        の解になる.よって,2解線型ODEと同様,解空間は2次元になり,2つの初期値によって定まる.特に,
        $x+t$の関数と$x-t$の関数との重ね合わせである.
        逆に,初期値さえ考えなければ,任意の$F,G\in C^2(\R)$に対して,
        \[u:=F(x+t)+G(x-t)\]
        は波動方程式の解になる.これは,
        これは,連立系
        \[\begin{cases}
            u_t-u_x=0\\
            u_t+u_x=0
        \end{cases}\]
        の解ということである.
        \item 具体的に,初期値問題の解は,輸送の輸送であるから,双方向を考える必要があり,$b=\pm1$についての初期値$g$,外力$h$の解
        \[g(x\pm t)+\int_{\mp x}^0h(x\pm t)dt\]
        の重ね合わせ
        \[\frac{1}{2}\Paren{g(x+t)+\int^{x+t}_0h(y)dy}+\frac{1}{2}\Paren{g(x-t)-\int^{x-t}_0h(y)dy}.\]
        になる.
    \end{enumerate}
    これらの考察から,進行方向という概念を持ち,奇数次元と偶数次元で違う消息の始まりを見出せる.
\end{tcolorbox}

\begin{problem}
    1次元の波動方程式の初期値問題
    \[\text{(C-WE)}^1\quad\begin{cases}
        u_{tt}-u_{xx}=0&\In \R\times\R^+,\\
        u=g,\quad u_t=h&\on \R\times\{0\}.
    \end{cases}\]
    を考える.
\end{problem}

\begin{observation}[1次元の波動方程式は輸送方程式の2回適用の構造を持つ]
    1次元の場合,
    \[u_{tt}-u_{xx}=\paren{\pp{}{t}+\pp{}{x}}\paren{\pp{}{t}-\pp{}{x}}u=0\]
    の形を持っている!
    \begin{description}
        \item[第一段階] そこで,
        \[v:=\paren{\pp{}{t}-\pp{}{x}}u\]
        とおくと,これは
        \[\paren{\pp{}{t}+\pp{}{x}}v=v_t+v_x=0\quad\In\R\times\R^+\]
        を満たす必要がある.
        これは斉次な定数係数の輸送方程式\ref{observation-homogeneous-transport-eq}で,
        $v$の値は$\vctr{1}{1}$方向で不変であることに注目すると,
        \[(u_t-u_x=)v(x,t)=v(x-t,0)=:a(x-t)\]
        と表せる.
        \item[第二段階] 上式は再び定数係数の輸送方程式で,ただし非斉次項$a$を持つ\ref{observation-nonhomogeneous-transport-eq}.
        この解は,外力項の寄与のみを考えた場合
        \[\begin{cases}
            \begin{cases}
                u_t-u_x=a(x-t)&\In\R^n\times\R_+,\\
                u=0&\on\R^n\times\{0\}.
            \end{cases}
        \end{cases}\]
        の解
        \[u(x,t)=\int^t_0a(x+t-2s)ds=\frac{1}{2}\int^{x+t}_{x-t}a(u)du.\]
        を用いて,初期条件を$u(x,0)=:b(x)$と定めれば,
        \[u(x,t)=b(x+t)+\frac{1}{2}\int^{x+t}_{x-t}a(u)du.\]
        と解ける.
        \item[初期条件について] まず,$b(x)=u(x,0)=g(x)$である.次に,
        \[a(x)=v(x,0)=u_t(x,0)-u_x(x,0)=h(x)-g'(x).\]
        以上より,
        \begin{align*}
            u(x,t)&=\frac{1}{2}\int^{x+t}_{x-t}a(y)dy+b(x+t)\\
            &=\frac{1}{2}\int^{x+t}_{x-t}\Paren{h(y)-g'(y)}dy+g(x+t)\\
            &=\frac{1}{2}\Paren{g(x+t)+g(x-t)}+\frac{1}{2}\int^{x+t}_{x-t}h(y)dy.
        \end{align*}
    \end{description}
\end{observation}
\begin{remark}
    2つ目の非斉次輸送方程式
    \[\text{(NC)}\quad\begin{cases}
        u_t+b\cdot Du=f&\In \R^n\times\R^+,\\
        u=g&\R^n\times\{0\}.
    \end{cases}\]
    の公式\ref{observation-nonhomogeneous-transport-eq}
    \[u(x,t)=g(x-tb)+\int^t_0f(x+(s-t)b,s)ds,\qquad (x,t)\in\R^n\times\R^+\]
    は$f(x,t):=a(x-t),b=-1$で用いた.
\end{remark}

\begin{theorem}[d'Alembert's formula]
    データ$g\in C^2(\R),h\in C^1(\R)$が定める関数
    \[u(x,t):=\frac{1}{2}\Paren{g(x+t)+g(x-t)}+\frac{1}{2}\int^{x+t}_{x-t}h(y)dy\qquad(x,t)\in\R\times\R^+\]
    について,
    \begin{enumerate}
        \item 可微分である:$u\in C^2(\R\times\R_+)$.
        \item 波動方程式の解である:$u_{tt}-u_{xx}=0\;\In\R\times\R^+$.
        \item 任意の$x^0\in\R$に対して,
        \[\lim_{\R^n\times\R^+\ni(x,t)\to (x^0,0)}u(x,t)=g(x^0),\quad\lim_{\R^n\times\R^+\ni(x,t)\to (x^0,0)}u_t(x,t)=h(x^0).\]
    \end{enumerate}
\end{theorem}
\begin{Proof}\mbox{}
    \begin{enumerate}
        \item 可微分性は$g\in C^2(\R),h\in C^1(\R)$の仮定から明らか.
        \item $u(x,t)=:A(x,t)+B(x,t)$の第一項は
        \[A_{tt}(x,t)=\frac{1}{2}\Paren{g''(x+t)+g''(x-t)}=A_{xx}(x,t).\]
        第二項は,
        \begin{align*}
            B(x,t)&=\frac{1}{2}\paren{\int^{x+t}_0h(y)dy-\int^{x-t}_{0}h(y)dy}\\
            &=\frac{1}{2}\paren{\int^{x}_{-t}h(y+t)dy-\int^{x}_th(y-t)dy}\\
            &=\frac{1}{2}\paren{\int^t_{-x}h(y+x)dy+\int^{t}_{x}h(x-z)dy},\qquad z:=-y
        \end{align*}
        と表せるから,
        \[B_x=\frac{1}{2}\Paren{h(x+t)-h(x-t)},\quad B_t=\frac{1}{2}\Paren{h(t+x)+h(x-t)}.\]
        \[B_{xx}=\frac{1}{2}\Paren{h'(x+t)-h'(x-t)}=B_{tt}.\]
        \item 前者は$g$の連続性と積分の区間に対する絶対連続性から明らか.
        後者は,
        \[u_t=\frac{1}{2}\Paren{g'(x+t)-g'(x-t)}+\frac{1}{2}\Paren{h(t+x)+h(x-t)}.\]
        より,$g',h$の連続性から明らか.
    \end{enumerate}
\end{Proof}

\subsection{半直線上の解公式:奇関数鏡像の方法}

\begin{problem}[固定端波動の問題]
    $x=0$で変位$0$に固定された弦についての,
    $\R^+$上の波動方程式の初期値・境界値問題
    \[\begin{cases}
        u_{tt}-u_{xx}=0&\In\R^+\times\R^+,\\
        u=g,\quad u_t=h&\on\R^+\times\{0\},\\
        u=0&\{0\}\times\R^+
    \end{cases}\]
    を考える.特に,$g(0)=h(0)=0$に注意.
    まず,$u,g,h$の奇関数鏡映(odd reflection)を考え,$\R$上の問題に延長する:
    \[\wt{u}(x,t):=\begin{cases}
        u(x,t)&\on\R_+\times\R_+,\\
        -u(-x,t)&\on-\R_+\times\R_+.
    \end{cases},\quad\wt{g}(x,t):=\begin{cases}
        g(x)&\on\R_+,\\
        -g(-x)&\on-\R_+.
    \end{cases},\quad\wt{h}(x,t):=\begin{cases}
        h(x)&\on\R_+,\\
        -h(-x)&\on-\R_+.
    \end{cases}\]
\end{problem}
\begin{observation}\label{observation-fixed-edge-wave-equation}
    すると,拡張された問題の解は,d'Alembertの公式より,
    \[\wt{u}(x,t)=\frac{1}{2}\Paren{\wt{g}(x+t)+\wt{g}(x-t)}+\frac{1}{2}\int^{x+t}_{x-t}\wt{h}(y)dy,\qquad\In\R\times\R_+.\]
    よって,これの$\R_+\times\R_+$上への制限を考えると,
    $\wt{g},\wt{h}$の引数が負になり得る$x-t<0$のとき,
    \[\frac{1}{2}\int^{x+t}_{x-t}\wt{h}(y)dy=\frac{1}{2}\int^{x+t}_0h(y)dy-\frac{1}{2}\int^0_{x-t}-h(-y)dy=\frac{1}{2}\paren{\int^{x+t}_0+\int^0_{t-x}}h(y)dy\]
    より,
    \[u(x,t)=\begin{cases}
        \frac{1}{2}\Paren{g(x+t)+g(x-t)}+\frac{1}{2}\int^{x+t}_{x-t}h(y)dy&x\ge t,\\
        \frac{1}{2}\Paren{g(x+t)-g(t-x)}+\frac{1}{2}\int^{x+t}_{-x+t}h(y)dy&x<t.
    \end{cases}\qquad(x,t)\in\R_+\times\R_+.\]
    これは,$g,h$について,壁$x=0$での反射を経たかどうかで場合分けしてるのみで,物理的な本質は変わっていないことがわかる.
\end{observation}
\begin{remarks}
    この拡張された問題に対してd'Alembertの公式を使うことが考え得るが,$g''(0)=0$でない限り$g\in C^2(\R)$でないので,この解$u$は$C^2$-級ではない.
\end{remarks}

\subsection{球面平均の方法}

\begin{tcolorbox}[colframe=ForestGreen, colback=ForestGreen!10!white,breakable,colbacktitle=ForestGreen!40!white,coltitle=black,fonttitle=\bfseries\sffamily,
title=]
    $n\ge2$の場合も,球対称性を期待して,動径$r$にどのような方程式が必要かを考察する.
    すると,極座標系に関する1次元の波動方程式に落ちる!
    が,その手続きは煩雑極まる.
\end{tcolorbox}

\begin{problem}
    \[\text{(C-WH)}^{\ge2}\quad\begin{cases}
        u_{tt}-\Lap u=0&\In\R^n\times\R^+,\\
        u=g,\quad u_t=h&\on\R^n\times\{0\}.
    \end{cases}\]
    の古典解$u\in C^m(\R^n\times\R^+)$を考えるにあたって,
    球面平均
    \begin{align*}
        U(x;r,t)&:=\dint_{\partial B(x,r)}u(y,t)dS(y),\\
        G(x;r)&:=\dint_{\partial B(x,r)}g(y)dS(y),\\
        H(x;r)&:=\dint_{\partial B(x,r)}h(y)dS(y).
    \end{align*}
    を考える.
\end{problem}

\begin{lemma}[Euler-Poisson-Darboux]\label{lemma-Euler-Poisson-Darboux}
    任意の$\text{(C-WH)}^{\ge2}$の古典解$u\in C^m(\R^n\times\R^+)\;(m\ge2)$と
    点$x\in\R^n$について,
    \begin{enumerate}
        \item $U\in C^m(\R_+\times\R_+)$.
        \item $U(x;r,t)$は次を満たす:
        \[\begin{cases}
            U_{tt}-U_{rr}-\frac{n-1}{r}U_{r}=0&\R^+\times\R^+,\\
            U=G,\quad U_t=H&\on\R_+\times\{0\}.
        \end{cases}\]
    \end{enumerate}
\end{lemma}
\begin{Proof}\mbox{}
    \begin{enumerate}
        \item $U$が$t$に関して$C^m$-級であることは,$u$が$C^m$-級であることより,導関数はコンパクト集合$\partial B(x,r)$上で可積分なので明らか.
        次に,$r$については,
        \begin{description}
            \item[$r$の一階導関数$U_r$] $U$を
            \[U=\int_{\partial B(0,1)}u(x+rz,t)\frac{rdS(z)}{\abs{\partial B(0,1)}}\]
            という表示にして,
            $r$について微分してみると,Gaussの発散定理から,
            \begin{align*}
                U_r&=\int_{\partial B(0,1)}z\cdot D_xu(x+rz,t)\frac{rdS(z)}{\abs{\partial B(0,1)}}\\
                &=\int_{B(0,1)}\Lap_xu(x+rz,t)\frac{rdz}{\abs{\partial B(0,1)}}\\
                &=\int_{B(x,r)}\Lap_xu(y,t)\frac{rdy}{r^n\abs{\partial B(0,1)}}=\frac{r}{n}\dint_{B(0,r)}\Lap_xu(y,t)dy.
            \end{align*}
            ただし,$\frac{1}{r^n\abs{\partial B(0,1)}}=\frac{1}{r^nn\om_n}=\frac{1}{n\abs{B(0,1)}}$に注意.
            この議論は補題\ref{lemma-derivative-of-surface-avarage-of-arbitrary-function}の繰り返しである.
            また,ここで$t$に依らずに$\lim_{r\to0}U_r=0$であることを観察しておく.
            \item[$r$の二階導関数$U_rr$] $U_r$を
            \[U_r=\frac{1}{nr^{n-1}}\int^{l=r}_{l=0}\int_{\partial B(0,l)}\Lap_xu(y,t)\frac{dS(y)dl}{\abs{B(0,1)}}\]
            と観てもう一度微分すると,
            \begin{align*}
                U_{rr}&=\paren{\frac{1}{n}-1}\frac{1}{r^n\abs{B(0,1)}}\int_{B(0,r)}\Lap_xudy+\frac{r}{n\abs{B(0,r)}}\int_{\partial B(0,r)}\Lap udS(y)\\
                &=\paren{\frac{1}{n}-1}\dint_{B(0,r)}\Lap_xudy+\dint_{\partial B(0,r)}\Lap udS(y).
            \end{align*}
            この議論は余面積公式\ref{cor-property-of-ball-surface-integral}の議論の繰り返しである.
            \item[一般化] この手続はいくらでも繰り返せる.よって,少なくとも$C^2(\R^n\times\R_+)$であるから,$U\in C^m(\R_+\times\R_+)$を得る.
            \item[$U,U_r,U_{rr}$の$\R_+\times\R_+$上の連続性] $\lim_{r\to+0}U_r=0,\lim_{r\to+0}U_{rr}=\frac{1}{n}\Lap u$より,いずれも$\{0\}\times\R_+$上連続.
        \end{description}
        \item 実は$(r^{n-1}U_r)_r=r^{n-1}U_{rr}$を示せば良い.
        $u$は波動方程式の解だから,
        \begin{align*}
            r^{n-1}U_r&=r^{n-1}\frac{r}{n}\dint_{B(x,r)}u_{tt}dy=\frac{1}{n\om_n}\int_{B(x,r)}u_{tt}dy
        \end{align*}
        よって,余面積公式\ref{cor-property-of-ball-surface-integral}より,
        \[(r^{n-1}U_r)_r=\frac{1}{n\om_n}\int_{\partial B(x,r)}u_{tt}dS=r^{n-1}\dint_{\partial B(x,r)}u_{tt}dS=r^{n-1}U_{tt}.\]
    \end{enumerate}
\end{Proof}
\begin{remark}
    Euler-Poisson-Darbouxの式は
    \[(r^{n-1}U_r)_r=(n-1)r^{n-2}U_r+r^{n-1}U_{rr}=r^{n-1}U_{tt}\]
    に等価.
\end{remark}

\subsection{3次元のKirchhoffの公式}

\begin{tcolorbox}[colframe=ForestGreen, colback=ForestGreen!10!white,breakable,colbacktitle=ForestGreen!40!white,coltitle=black,fonttitle=\bfseries\sffamily,
title=]
    3次元の場合は簡単な変換により,Euler-Poisson-Darbouxの方程式が波動方程式に直るので,
    これを通じて解公式がみつかる.
\end{tcolorbox}

\begin{problem}
    初期値問題
    \[\text{(C-WH)}^{3}\quad\begin{cases}
        u_{tt}-\Lap u=0&\In\R^3\times\R^+,\\
        u=g,\quad u_t=h&\on\R^3\times\{0\}.
    \end{cases}\]
    の古典解$u\in C^2(\R^3\times\R^+)$を考える.
\end{problem}

\begin{theorem}[Kirchhoff's formula]\mbox{}
    \begin{enumerate}
        \item \[\wt{U}:=rU,\quad\wt{G}:=rG,\quad\wt{H}:=rH\]
        は,初期値-境界値問題
        \[\begin{cases}
            \wt{U}_{tt}-\wt{U}_{rr}=0&\In\R^+\times\R^+,\\
            \wt{U}=\wt{G},\quad\wt{U}_t=\wt{H}&\on\R^+\times\{0\},\\
            \wt{U}=0&\on\{0\}\times\R^+.
        \end{cases}\]
        の解である.
        \item 初期値問題$\text{(C-WH)}^3$の古典解$u\in C^2(\R^3\times\R^+)$は
        \[u(x,t)=\dint_{\partial B(x,t)}\Paren{th(y)+g(y)+Dg(y)\cdot(y-x)}dS(y),\qquad (x,t)\in\R^3\times\R^+.\]
        という形が必要.
    \end{enumerate}
\end{theorem}
\begin{Proof}\mbox{}
    \begin{enumerate}
        \item 第一式は,Euler-Poisson-Darbouxの補題\ref{lemma-Euler-Poisson-Darboux}から,
        \begin{align*}
            \wt{U}_{tt}&=rU_{tt}\\
            &=r\paren{U_{rr}+\frac{2}{r}U_r}\\
            &=rU_{rr}+2U_r=(U+rU_r)_r=\wt{U}_{rr}.
        \end{align*}
        さらに,$\wt{G}_{rr}(0)=\wt{U}_{rr}(0,0)=0\cdot U_{rr}+2U_r(0,0)=0$も分かる.

        第二式は$U=G,U_t=H$の両辺に$r$を乗じたものである.
        第三式は明らか.
        \item (1)の問題は半直線上の波動方程式\ref{observation-fixed-edge-wave-equation}より,
        \[\wt{U}(x;r,t)=\frac{1}{2}\Paren{\wt{G}(r+t)-\wt{G}(r-t)}+\frac{1}{2}\int^{r+t}_{-r+t}\wt{H}(y)dy,\qquad 0\le r\le t.\]
        これから$u$に対する必要条件が次のように分かる:
        \begin{align*}
            u(x,t)&=\lim_{r\searrow0}\frac{\wt{U}(x;r,t)}{r}\\
            &=\lim_{r\searrow0}\paren{\frac{\wt{G}(t+r)-\wt{G}(t-r)}{2r}+\frac{1}{2r}\int^{t+r}_{t-r}\wt{H}(y)dy}\\
            &=\wt{G}'(t)+\wt{H}(t).
        \end{align*}
        これら2つの関数の導関数を計算すると,
        \begin{align*}
            \wt{G}(x;r)&=rG(x;r)=r\dint_{\partial B(x,r)}g(y)dS(y)\\
            \wt{G}'(x;r)&=\dint_{\partial B(x,r)}g(y)dS(y)+r\pp{}{r}\dint_{\partial B(0,1)}g(x+rz)dS(z)\\
            &=\dint_{\partial B(x,r)}g(y)dS(y)+r\dint_{\partial B(x,r)}Dg(y)\cdot\paren{\frac{y-x}{r}}dS(y).
        \end{align*}
        以上より,
        \[u(x,t)=\dint_{\partial B(x,t)}\Paren{th(y)+g(y)+Dg(y)\cdot (y-x)}dS(y).\]
    \end{enumerate}
\end{Proof}
\begin{remarks}
    球面平均法でもう一つ肝心な点は,$\wt{U}$は$\R^+\times\R^+$上だけでなく,閉集合$\R_+\times\R_+$上で連続な点である.
\end{remarks}

\begin{theorem}[Kirchhoff's formula]\label{thm-Kirchhoff-formula}
    データを$g\in C^3(\R^3),h\in C^2(\R^3)$に対して,
    \begin{align*}
        u(x,t)&:=\pp{}{t}\paren{t\dint_{\partial B(x,t)}gdS}+t\dint_{\partial B(x,t)}hdS\\
        &=\dint_{\partial B(x,t)}\Paren{th(y)+g(y)+Dg(y)\cdot(y-x)}dS(y),\qquad (x,t)\in\R^3\times\R^+.
    \end{align*}
    を考える.
    \begin{enumerate}
        \item $u\in C^2(\R^3\times\R^+)$.
        \item $u_{tt}-\Lap u=0\;\In\R^3\times\R^+$.
        \item \[\lim_{(y,s)\to(x^0,0)}u(y,s)=g(x^0),\quad\lim_{(y,s)\to(x^0,0)}u_t(y,s)=h(x^0).\]
    \end{enumerate}
    すなわち,初期値問題$\text{(C-WH)}^3$の解である.
\end{theorem}
\begin{Proof}
    あとは十分性を確認すれば良い.
\end{Proof}
\begin{remarks}
    $Dg(y)$の項が解の正則性を落としている.
\end{remarks}

\subsection{2次元のPoissonの公式}

\begin{tcolorbox}[colframe=ForestGreen, colback=ForestGreen!10!white,breakable,colbacktitle=ForestGreen!40!white,coltitle=black,fonttitle=\bfseries\sffamily,
title=]
    3次元の場合は$\wt{U}:=rU$というような1次元化の変換は見つからない.
    そこで,次元降下を考える.
\end{tcolorbox}

\begin{problem}
    初期値問題
    \[\text{(C-WH)}^{2}\quad\begin{cases}
        u_{tt}-\Lap u=0&\In\R^2\times\R^+,\\
        u=g,\quad u_t=h&\on\R^2\times\{0\}.
    \end{cases}\]
    の古典解$u\in C^2(\R^2\times\R^+)$を考えるにあたって,単純な埋め込み$\R^2\ni x=(x_1,x_2)\mapsto\o{x}=(x_1,x_2,0)\in\R^3$によって,
    \[\o{u}(x_1,x_2,x_3,t):=u(x_1,x_2,t),\quad\o{g}(x_1,x_2,x_3):=g(x_1,x_2),\quad\o{h}(x_1,x_2,x_3):=h(x_1,x_2).\]
    と定めることにより,
    \[\begin{cases}
        \o{u}_{tt}-\Lap\o{u}=0&\In\R^3\times\R^+,\\
        \o{u}=\o{g},\quad\o{u}_t=\o{h}&\on\R^3\times\{0\}.
    \end{cases}\]
    を解くことに同値であることに注目する.
\end{problem}

\begin{observation}[Hadamardの降下法]
    まずKirchhoffの公式より,
    \[u(x,t)=\o{u}(\o{x},\o{t})=\pp{}{t}\paren{t\dint_{\partial\o{B}(\o{x},t)}\o{g}d\o{S}}+t\dint_{\partial\o{B}(\o{x},t)}\o{h}d\o{S}.\]
    \begin{description}
        \item[球面平均についての考察] $\partial\o{B}(\o{x},t)$とは,円板$B(x,t)\subset\R^2$を大円とする球であるから,半球面のパラメータ付けを$\gamma(y):=\sqrt{t^2-\abs{y-x}^2}\;(y\in B(x,t))$とすると,
        $(1+\abs{D\gamma}^2)^{1/2}=t(t^2-\abs{y-x}^2)^{-1/2}$に注意すれば,
        \begin{align*}
            \dint_{\partial\o{B}(\o{x},t)}\o{g}d\o{S}&=\frac{1}{\abs{\partial\o{B}(\o{x},t)}}\int_{\partial\o{B}(\o{x},t)}\o{g}d\o{S}\\
            &=\frac{2}{4\pi t^2}\int_{B(x,t)}g(y)\sqrt{1+\abs{D\gamma(y)}^2}dy\\
            &=\frac{1}{2\pi t}\int_{B(x,t)}\frac{g(y)}{(t^2-\abs{y-x}^2)^{1/2}}dy\\
            &=\frac{t}{2}\dint_{B(x,t)}\frac{g(y)}{(t^2-\abs{y-x}^2)^{1/2}}dy.
        \end{align*}
        以上より,
        \[u(x,t)=\frac{1}{2}\pp{}{t}\paren{t^2\dint_{B(x,t)}\frac{g(y)}{\sqrt{t^2-\abs{y-x}^2}}dy}+\frac{t^2}{2}\dint_{B(x,t)}\frac{h(y)}{\sqrt{t^2-\abs{y-x}^2}}dy.\]
        \item[微分の計算] まず,積分区間の$t$を被積分関数にくりこみ,
        \[t^2\dint_{B(x,t)}\frac{g(y)}{\sqrt{t^2-\abs{y-x}^2}}dy=t\dint_{B(0,1)}\frac{g(x+tz)}{\sqrt{1-\abs{z}^2}}dz.\]
        とし,これを微分すると,
        \begin{align*}
            &\pp{}{t}\paren{t^2\dint_{B(x,t)}\frac{g(y)}{\sqrt{t^2-\abs{y-x}^2}}dy}\\
            &=\dint_{B(0,1)}\frac{g(x+tz)}{\sqrt{1-\abs{z}^2}}dz+t\dint_{B(0,1)}\frac{Dg(x+tz)\cdot z}{\sqrt{1-\abs{z}^2}}dz\\
            &=t\dint_{B(x,t)}\frac{g(y)}{\sqrt{t^2-\abs{y-x}^2}}dy+t\dint_{B(x,t)}\frac{Dg(y)\cdot(y-x)}{\sqrt{t^2-\abs{y-x}^2}}dy.
        \end{align*}
        以上より,
        \[u(x,t)=\frac{1}{2}\dint_{B(x,t)}\frac{tg(y)+t^2h(y)+tDg(y)\cdot(y-x)}{\sqrt{t^2-\abs{y-x}^2}}dy,\qquad(x,t)\in\R^2\times\R^+.\]
    \end{description}
\end{observation}

\begin{theorem}[Poisson's formula]
    データを$g\in C^3(\R^2),h\in C^2(\R^2)$に対して,
    \[u(x,t):=\frac{1}{2}\dint_{B(x,t)}\frac{tg(y)+t^2h(y)+tDg(y)\cdot(y-x)}{\sqrt{t^2-\abs{y-x}^2}}dy,\qquad(x,t)\in\R^2\times\R^+.\]
    を考える.
    \begin{enumerate}
        \item $u\in C^2(\R^2\times\R^+)$.
        \item $u_{tt}-\Lap u=0\;\In\R^2\times\R^+$.
        \item \[\lim_{(y,s)\to(x^0,0)}u(y,s)=g(x^0),\quad\lim_{(y,s)\to(x^0,0)}u_t(y,s)=h(x^0).\]
    \end{enumerate}
    すなわち,初期値問題$\text{(C-WH)}^2$の解である.
\end{theorem}

\subsection{奇数次元の解公式}

\begin{tcolorbox}[colframe=ForestGreen, colback=ForestGreen!10!white,breakable,colbacktitle=ForestGreen!40!white,coltitle=black,fonttitle=\bfseries\sffamily,
title=]
    次の補題で表されるような性質を持つ構造
    \[\paren{\frac{1}{r}\pp{}{r}}^{k-1}\Paren{r^{2k-1}U(x;r,t)}=(2k-1)!!\cdot rU+\beta_1r^2\pp{U}{r}(r)+\beta_2r^3\pp{^2U}{r^2}(r)+\cdots+\beta_{k-1}r^{k}\pp{^{k-1}U}{r^{k-1}}(r).\]
    に注目する.$\partial_{rr}$で微分すると冪$k-1,2k-1$が1つずつ増える.
\end{tcolorbox}

\begin{lemma}\label{lemma-structure-for-odd-dimension-WE}
    $\phi\in C^{k+1}(\R)$と$k\in\N^+$に対して,
    \begin{enumerate}
        \item \[\paren{\dd{^2}{r^2}}\paren{\frac{1}{r}\dd{}{r}}^{k-1}\Paren{r^{2k-1}\phi(r)}=\paren{\frac{1}{r}\dd{}{r}}^k\paren{r^{2k}\dd{\phi}{r}(r)}.\]
        \item $\beta^k_0:=(2k-1)!!=1\cdot3\cdot5\cdots(2k-1)$とある$\beta_j^k$が存在して,任意の$\phi\in C^{k-1}(\R)$について,
        \[\paren{\frac{1}{r}\dd{}{r}}^{k-1}\Paren{r^{2k-1}\phi(r)}=\sum_{j=0}^{k-1}\beta_j^kr^{j+1}\dd{^j\phi}{r^j}(r).\]
    \end{enumerate}
    以降,$\gamma_n=(n-2)!!=1\cdot3\cdot5\cdots(n-2)$と定めると,$n=2k+1$のとき,$\beta^k_0=\gamma_n$.
\end{lemma}

\begin{problem}
    $n=:2k+1\;(k\ge1)$-次元の初期値問題
    \[\text{(C-WH)}^{2k+1}\quad\begin{cases}
        u_{tt}-\Lap u=0&\In\R^n\times\R^+,\\
        u=g,\quad u_t=h&\on\R^n\times\{0\}.
    \end{cases}\]
    の古典解$u\in C^{k+1}(\R^n\times\R^+)$を考える.
    すると,球面平均$U\in C^{k+1}$も成り立つ.さらに,
    \[\begin{cases}
        \wt{U}(r,t):=\paren{\frac{1}{r}\pp{}{r}}^{k-1}\Paren{r^{2k-1}U(x;r,t)},\\
        \wt{G}(r):=\paren{\frac{1}{r}\pp{}{r}}^{k-1}\Paren{r^{2k-1}G(x;r)},\\
        \wt{H}(r):=\paren{\frac{1}{r}\pp{}{r}}^{k-1}\Paren{r^{2k-1}H(x;r)}.
    \end{cases},\qquad r>0,t\ge0.\]
    と定めると,$\wt{U}(r,0)=\wt{G}(r),\wt{U}_t(r,0)=\wt{H}(r)$が成り立つ:
    \[U(x;r,0)=\dint_{\partial B(x,r)}u(y,0)dS(y)=\dint_{\partial B(x,r)}g(y)dS(y)=G(x;r).\]
\end{problem}

\begin{lemma}[Euler-Poisson-Darboux公式の奇数次元での変換]
    $\wt{U}$は次の半直線上の波動方程式を満たす:
    \[\begin{cases}
        \wt{U}_{tt}-\wt{U}_{rr}=0&\In\R^+\times\R^+,\\
        \wt{U}=\wt{G},\quad\wt{U}_t=\wt{H}&\on\R^+\times\{0\},\\
        \wt{U}=0&\on\{0\}\times\R^+.
    \end{cases}\]
    すなわち,
    \[u(x,t)=\frac{1}{\gamma_n}\paren{\paren{\pp{}{t}}\paren{\frac{1}{t}\pp{}{t}}^{\frac{n-3}{2}}\paren{t^{n-2}\dint_{\partial B(x,t)}gdS}+\paren{\frac{1}{t}\pp{}{t}}^{\frac{n-3}{2}}\paren{t^{n-2}\dint_{\partial B(x,t)}hdS}}.\]
    と表せる必要がある.
\end{lemma}
\begin{Proof}\mbox{}
    \begin{description}
        \item[波動方程式を満たす] 第二式の成立は論じた.第三式も補題の(2)式から
        \[\wt{U}(r,t)=\paren{\frac{1}{r}\pp{}{r}}^{k-1}\Paren{r^{2k-1}U(x;r,t)}=\sum_{j=0}^{k-1}\beta^k_jr^{j+1}\dd{^jU}{r^j}.\]
        より明らかだから,第一式を確認すれば良い.
        補題(1)の式変形と,$U$についてのEuler-Poisson-Darbouxの式\ref{lemma-Euler-Poisson-Darboux}より,
        \begin{align*}
            \wt{U}_{rr}&=\paren{\pp{^2}{r^2}}\paren{\frac{1}{r}\pp{}{r}}^{k-1}\Paren{r^{2k-1}U}\\
            &=\paren{\frac{1}{r}\pp{}{r}}^k(r^{2k}U_r)\\
            &=\paren{\frac{1}{r}\pp{}{r}}^{k-1}\Paren{r^{2k-1}U_{rr}+2kr^{2k-2}U_r}\\
            &=\paren{\frac{1}{r}\pp{}{r}}^{k-1}\Paren{r^{2k-1}\paren{U_{rr}+\frac{n-1}{r}U_r}}\\
            &=\paren{\frac{1}{r}\pp{}{r}}^{k-1}\Paren{r^{2k-1}U_{tt}}=\wt{U}_{tt}.
        \end{align*}
        \item[$u$の形への示唆] 
        半直線上のd'Alembertの公式\ref{observation-fixed-edge-wave-equation}から,
        \[\wt{U}(r,t)\frac{1}{2}\Paren{\wt{G}(r+t)-\wt{G}(t-r)}+\frac{1}{2}\int^{t+r}_{t-r}\wt{H}(y)dy\qquad(r,t)\in\R\times\R_+.\]
        が必要.よって,
        \[\wt{U}(r,t)=\beta^k_0rU+r^2\sum_{j=1}^{k-1}\beta^k_jr^{j-1}\dd{^jU}{r^j}\]
        から,
        \[\lim_{r\to0}\frac{\wt{U}(r,t)}{\beta^k_0r}=\lim_{r\to0}U(x;r,t)=u(x,t).\]
        が成り立つ.これより,$u$について,
        \begin{align*}
            u(x,t)&=\frac{1}{\beta^k_0}\lim_{r\to0}\paren{\frac{\wt{G}(t+r)-\wt{G}(r-t)}{2r}+\frac{1}{2r}\int^{t+r}_{t-r}\wt{H}(y)dy}\\
            &=\frac{1}{\beta^k_0}\Paren{\wt{G}'(t)+\wt{H}(t)}.
        \end{align*}
        が必要である.$n=2k+1$と定めたので,$k=\frac{n-1}{2},2k-1=n-2$に注意すれば,与式を得る.
    \end{description}
\end{Proof}

\begin{theorem}\label{thm-C-WE-in-odd-dimension}
    $n=2m-1\ge3$を奇数とする.
    $g\in C^{m+1}(\R^n),h\in C^m(\R^n)$について,
    \[u(x,t):=\frac{1}{\gamma_n}\paren{\paren{\pp{}{t}}\paren{\frac{1}{t}\pp{}{t}}^{\frac{n-3}{2}}\paren{t^{n-2}\dint_{\partial B(x,t)}gdS}+\paren{\frac{1}{t}\pp{}{t}}^{\frac{n-3}{2}}\paren{t^{n-2}\dint_{\partial B(x,t)}hdS}},\qquad(x,t)\in\R^n\times\R_+\]
    について,
    \begin{enumerate}
        \item $u\in C^2(\R^n\times\R_+)$.
        \item $u_{tt}-\Lap u=0\;\In\R^n\times\R^+$.
        \item 任意の$x^0\in\R^n$について,
        \[\lim_{\R^n\times\R^+\ni(x,t)\to(x^0,0)}u(x,t)=g(x^0),\quad\lim_{\R^n\times\R^+\ni(x,t)\to(x^0,0)}u_t(x,t)=h(x^0).\]
    \end{enumerate}
    すなわち,初期値問題$\text{(C-WH)}^{2m-1}$の解である.
\end{theorem}
\begin{Proof}
    十分性を確認すれば良い.
    \begin{enumerate}
        \item 微分作用素の線形性より,$g\equiv0,h\equiv0$のそれぞれの場合について示せば良い.
        \begin{description}
            \item[$g\equiv0$の場合] まず,
            \[u(x,t)=\frac{1}{\gamma_n}\paren{\frac{1}{t}\pp{}{t}}^{\frac{n-3}{2}}\Paren{t^{n-2}H(x;t)}\]
            と表せる際に$u_{tt}-\Lap u=0$を示す.
            実際,補題\ref{lemma-structure-for-odd-dimension-WE}より,
            \[u_{tt}=\frac{1}{\gamma_n}\paren{\frac{1}{t}\pp{}{t}}^{\frac{n-1}{2}}(t^{n-1}H_t).\]
            ここで,
            \[H_t=\frac{t}{n}\dint_{B(x,t)}\Lap hdy\]
            である\ref{lemma-derivative-of-surface-avarage-of-arbitrary-function}から,
            \begin{align*}
                u_{tt}&=\frac{1}{\gamma_n}\paren{\frac{1}{t}\pp{}{t}}^{\frac{n-1}{2}}t^{n-1}\frac{t}{n\om_nt^n}\int_{B(x,t)}\Lap hdy\\
                &=\frac{1}{n\om_n\gamma_n}\paren{\frac{1}{t}\pp{}{t}}^{\frac{n-1}{2}}\int_{B(x,t)}\Lap hdy\\
                &=\frac{1}{n\om_n\gamma_n}\paren{\frac{1}{t}\pp{}{t}}^{\frac{n-3}{2}}\paren{\frac{1}{t}\int_{\partial B(x,t)}\Lap hdS(y)}.
            \end{align*}
            一方で,$\Lap u$について,
            \begin{align*}
                \Lap_xH(x;t)&=\Lap_x\dint_{\partial B(0,t)}h(x+y)dS(y)=\dint_{\partial B(x,t)}\Lap hdS.
            \end{align*}
            以上より,$u_{tt}=\Lap u\;\In\R^n\times\R^+$.
            \item[$h\equiv0$の場合] 全く同様の計算が可能である.
        \end{description}
        \item 補題\ref{lemma-structure-for-odd-dimension-WE}(2)の展開によると,
        \begin{description}
            \item[$u$の展開] $t$を係数に含まない項は第一項が生み出す項
            \[u(x,t)=\frac{1}{\gamma_n}\gamma_n\dint_{\partial B(x,t)}gdS+O(t)\]
            のみであるから,$(x,t)\to(x^0,0)$のとき,$g(x^0)$に収束する.
            \item[$u_t$の展開] $u(x,t)$の$t$の一次の項に注目すれば,
            \[u_t(x,t)=\frac{1}{\gamma_n}2\beta_1\pp{}{t}\dint_{B(x,t)}gdS+\frac{1}{\gamma_n}\gamma_n\dint_{B(x,t)}hdS+O(t).\]
            第一項は実は$t$の一次の項で,$0$に収束するから,$(x,t)\to(x^0,0)$のとき,$h(x^0)$に収束する.
        \end{description}
    \end{enumerate}
\end{Proof}
\begin{remarks}\mbox{}
    \begin{enumerate}
        \item 解$u$を定めるためには,コンパクト領域$\partial B(x,t)$のみの値に依存している.
        \item $t$についての正則性は初期値$g$よりも落ちている(d'Alembertの公式では観られない現象).一方でエネルギーノルムは落ちない.
        \item ある地点$x$の初期値$g(x)$の影響が別の$t>0$に及ぶには,ある程度の時間経過が必要である.
    \end{enumerate}
\end{remarks}

\subsection{熱方程式による導出}

\begin{problem}
    零化された$n=:2k+1\;(k\ge1)$-次元の初期値問題
    \[\text{(C-WH)}^{2k+1}_0\quad\begin{cases}
        u_{tt}-\Lap u=0&\In\R^n\times\R^+,\\
        u=g,\quad u_t=0&\on\R^n\times\{0\}.
    \end{cases}\]
    の古典解$u\in C^{k+1}(\R^n\times\R^+)$を,さらなる制約$u\in C_b(\R^n\times\R^+)$と$g\in C_c^\infty(\R^n)$の下で考える.
    さらに,$u(x,t):=u(x,-t)\;(x\in\R^n,t<0)$と偶関数に延長すれば,大域上で$u_{tt}-\Lap u=0\;\In\R^n\times\R$を満たす.
\end{problem}

\begin{proposition}[波動方程式の解のLaplace変換は熱方程式を解く]
    \[v(x,t):=\frac{1}{\sqrt{4\pi t}}\int_{\R}e^{-\frac{s^2}{4t}}u(x,s)ds=\frac{1}{\sqrt{4\pi t}}\int_\R e^{-\lambda^2}u(x,\sqrt{4t}\lambda)\sqrt{4t}d\lambda\quad(x,t)\in\R^n\times\R^+\]
    について,次の熱方程式の初期値問題を解く:
    \[\begin{cases}
        v_t-\Lap v=0&\In\R^n\times\R^+,\\
        v=g&\on\R^n\times\{0\}.
    \end{cases}\]
    よって,次のようにも表せる:
    \[v(x,t)=\frac{1}{(4\pi t)^{n/2}}\int_{\R^n}e^{-\frac{\abs{x-y}^2}{4t}}g(y)dy.\]
\end{proposition}
\begin{Proof}\mbox{}
    \begin{description}
        \item[初期値について] $e^{-\frac{s^2}{4t}}$は$t\searrow0$に関する総和核であるから,$\lim_{t\to0}v=g\;\on\R^n$が一様位相についても成り立つ.
        \item[方程式について] $u$が波動方程式を満たすことと部分積分により,
        \begin{align*}
            \Lap v(x,t)&=\frac{1}{\sqrt{4\pi t}}\int_\R e^{-\frac{s^2}{4t}}\Lap u(x,s)ds=\frac{1}{\sqrt{4\pi t}}\int_\R e^{-\frac{s^2}{4t}}u_{ss}(x,s)ds\\
            &=\frac{1}{\sqrt{4\pi t}}\SQuare{e^{-\frac{s^2}{4t}}u_s(x,s)}^\infty_{-\infty}+\frac{1}{\sqrt{4\pi t}}\int_\R \frac{s}{2t}e^{-\frac{s^2}{4t}}u_s(x,s)ds\\
            &=\frac{1}{\sqrt{4\pi t}}\SQuare{\frac{s}{2t}e^{-\frac{s^2}{4t}}u_s}^\infty_{-\infty}-\frac{1}{\sqrt{4\pi t}}\int_\R\paren{\frac{1}{2t}-\frac{s^2}{4t^2}}e^{-\frac{s^2}{4t}}uds\\
            &=\frac{1}{\sqrt{4\pi t}}\int_\R \paren{\frac{s^2}{4t^2}-\frac{1}{2t}}e^{-\frac{s^2}{4t}}u(x,s)ds=v_t(x,t).
        \end{align*}
        実際,
        \[v_t=\frac{1}{\sqrt{4\pi t}}\paren{-\frac{1}{2t}}\int_\R e^{-\frac{s^2}{4t}}u(x,s)ds+\frac{1}{\sqrt{4\pi t}}\int_\R\frac{s^2}{4t^2}e^{-\frac{s^2}{4t}}u(x,s)ds.\]
    \end{description}
\end{Proof}

\begin{lemma}
    $n=2k+1$のとき,
    \begin{enumerate}
        \item \[\paren{-\frac{1}{2r}\dd{}{r}}^k(e^{-\lambda r^2})=\lambda^ke^{-\lambda r^2}.\]
        \item \[\lambda^{\frac{n-1}{2}}\int^\infty_0e^{-\lambda r^2}r^{n-1}G(x;r)dr=\frac{1}{2^k}\int^\infty_0r\paren{\paren{\frac{1}{r}\pp{}{r}}^k(r^{2k-1}G(x;r))}e^{-\lambda r^2}dr.\]
    \end{enumerate}
\end{lemma}
\begin{Proof}\mbox{}
    \begin{enumerate}
        \item $\dd{}{r}e^{-\lambda r^2}=-2r\lambda e^{-\lambda r^2}$より.
        \[-\frac{1}{2r}\dd{}{r}e^{-\lambda r^2}=\lambda e^{-\lambda r^2}.\]
        これを繰り返すことによる.
        \item (1)の等式を用いたのちに,$k$回部分積分を繰り返すことにより,
        \begin{align*}
            \text{(LHS)}&=\int^\infty_0\lambda^ke^{-\lambda r^2}r^{2k}G(x;r)dr\\
            &=\frac{(-1)^k}{2^k}\int^\infty_0\paren{\paren{\frac{1}{r}\dd{}{r}}^k(e^{-\lambda r^2})}r^{2k}G(x;r)dr\\
            &=\frac{1}{2^k}\int^\infty_0r\paren{\paren{\frac{1}{r}\pp{}{r}}^k(r^{2k-1}G(x;r))}e^{-\lambda r^2}dr.
        \end{align*}
    \end{enumerate}
\end{Proof}

\begin{proposition}
    問題$\text{(C-WH)}^{2k+1}_0$の古典解$u\in C^{k+1}(\R^n\times\R^+)$について,
    \begin{enumerate}
        \item \[\int_0^\infty e^{-\lambda s^2}u(x,s)ds=\frac{n\om_n}{2}\paren{\frac{\lambda}{\pi}}^{\frac{n-1}{2}}\int^\infty_0e^{-\lambda r^2}r^{n-1}\underbrace{\dint_{\partial B(x,r)}gdS(y)}_{=G(x;r)}dr\]
        \item \[\int^\infty_0e^{-\lambda r^2}u(x,s)ds=\frac{n\om_n}{\pi^{\frac{n-1}{2}}2^{k+1}}\int^\infty_0\paren{\paren{\frac{1}{r}\pp{}{r}}^k\Paren{r^{2k-1}G(x;r)}}e^{-\lambda r^2}dr.\]
        \item \[u(x,t)=\frac{1}{\gamma_n}\pp{}{t}\paren{\frac{1}{t}\pp{}{t}}^{\frac{n-3}{2}}\paren{t^{n-2}\dint_{\partial B(x,t)}gdS}.\]
    \end{enumerate}
\end{proposition}
\begin{Proof}\mbox{}
    \begin{enumerate}
        \item 命題の結論より,
        \[v=\frac{1}{\sqrt{4\pi t}}\int_{\R}e^{-\frac{s^2}{4t}}u(x,s)ds=\frac{1}{(4\pi t)^{n/2}}\int_{\R^n}e^{-\frac{\abs{x-y}^2}{4t}}g(y)dy.\]
        左辺は$\lambda:=\frac{1}{4t}$と定めると,$u$は$\R$上に偶関数として拡張したことに注意して,
        \[\text{(LHS)}=\sqrt{\frac{\lambda}{\pi}}\int_\R e^{-\lambda s^2}u(x,s)ds=2\sqrt{\frac{\lambda}{\pi}}\int^\infty_0e^{-\lambda s^2}u(x,s)ds.\]
        右辺は極座標変換により,
        \[\text{(RHS)}=\paren{\frac{\lambda}{\pi}}^{n/2}\int_{\R^n}e^{-\lambda\abs{x-y}^2}g(y)dy=\frac{n\om_n}{2}\paren{\frac{\lambda}{\pi}}^{n/2}\int^\infty_0e^{-\lambda r^2}r^{n-1}G(x;r)dr.\]
        \item 補題より,
        \begin{align*}
            \int_0^\infty e^{-\lambda s^2}u(x,s)ds&=\frac{n\om_n}{2}\paren{\frac{\lambda}{\pi}}^{\frac{n-1}{2}}\int^\infty_0e^{-\lambda r^2}r^{n-1}G(x;r)dr\\
            &=\frac{n\om_n}{\pi^{\frac{n-1}{2}}2^{k+1}}\int^\infty_0\paren{\paren{\frac{1}{r}\pp{}{r}}^k\Paren{r^{2k-1}G(x;r)}}e^{-\lambda r^2}dr.
        \end{align*}
        \item $\tau:=r^2$とみると,$u$の$\tau$に関するLaplace変換とみなせるから,
        \[u(x,t)=\frac{n\om_n}{\pi^k2^{k+1}}t\paren{\frac{1}{t}\pp{}{t}}^k(t^{2k-1}G(x;t)).\]
        ここで,
        \[\frac{n\om_n}{\pi^k2^{k+1}}\frac{n\pi^{1/2}}{2^{k+1}\Gamma\paren{\frac{n}{2}+1}}=\frac{1}{(n-2)(n-4)\cdots5\cdot 3}=\frac{1}{\gamma_n}.\]
        より,
        \[u(x,t)=\frac{1}{\gamma_n}\pp{}{t}\paren{\frac{1}{t}\pp{}{t}}^{\frac{n-3}{2}}\paren{t^{n-2}\dint_{\partial B(x,t)}gdS}.\]
        と書き直せる.
    \end{enumerate}
\end{Proof}

\subsection{偶数次元の解公式}

\begin{problem}
    偶数次元$n\ge2$の初期値問題
    \[\text{(C-WH)}^{n}\quad\begin{cases}
        u_{tt}-\Lap u=0&\In\R^n\times\R^+,\\
        u=g,\quad u_t=h&\on\R^n\times\{0\}.
    \end{cases}\]
    の古典解$u\in C^m(\R^n\times\R^+)$を考えるにあたって,
    埋め込み
    \[\o{u}(x_1,\cdots,x_{n+1},t):=u(x_1,\cdots,x_n,t),\quad \o{u}=\o{g},\quad\o{u}_t=\o{h}\;\on\R^{n+1}\times\{0\}.\]
    と$\R^{n+1}\times\R^+$上の波動方程式を考える.
\end{problem}

\begin{observation}
    $x\in\R^n,t>0$を任意に取り,$\o{x}:=(x_1,\cdots,x_n,0)\in\R^{n+1}$について,
    \[u(x,t)=u(\o{x},t)=\frac{1}{\gamma_{n+1}}\paren{\pp{}{t}\paren{\frac{1}{t}\pp{}{t}}^{\frac{n-2}{2}}\paren{t^{n-1}\dint_{\partial\o{B}(\o{x},t)}\o{g}d\o{S}}+\paren{\frac{1}{t}\pp{}{t}}^{\frac{n-2}{2}}\paren{t^{n-1}\dint_{\partial\o{B}(\o{x},t)}\o{h}d\o{S}}}.\]
    が必要.
    \begin{description}
        \item[球上の積分の計算] まず$g$の方から,$\partial \o{B}(\o{x},t)$上での積分について考える.半球$\partial\o{B}(x,t)\cap\Brace{y_{n+1}\ge0}$のパラメータ付けは
        \[\gamma(y):=\sqrt{t^2-\abs{y-x}^2},\qquad(y\in B(x,t)\subset\R^n)\]
        と取れ,その線素は$\sqrt{1+\abs{D\gamma(y)}^2}=t\sqrt{t^2-\abs{y-x}^2}$であるから,
        \begin{align*}
            \dint_{\partial\o{B}(\o{x},t)}\o{g}d\o{S}&=\frac{2}{(n+1)\om_{n+1}t^n}\int_{B(x,t)}g(y)\sqrt{1+\abs{D\gamma(y)}^2}dy\\
            &=\frac{2}{(n+1)\om_{n+1}t^{n-1}}\int_{B(x,t)}\frac{g(y)}{(t^2-\abs{y-x}^2)^{1/2}}dy\\
            &=\frac{2t\om_n}{(n+1)\om_{n+1}}\dint_{B(x,t)}\frac{g(y)}{(t^2-\abs{y-x}^2)^{1/2}}dy.
        \end{align*}
        \item[総合] 同様の計算を$h$についても行うと,
        \begin{align*}
            u(x,t)&=\frac{1}{\gamma_{n+1}}\frac{2\om_n}{(n+1)\om_{n+1}}\Paren{\pp{}{t}\paren{\frac{1}{t}\pp{}{t}}^{\frac{n-2}{2}}\paren{t^n\dint_{B(x,t)}\frac{g(y)}{(t^2-\abs{y-x}^2)^{1/2}}dy}\\
            &\hspace{3cm}+\paren{\frac{1}{t}\pp{}{t}}^{\frac{n-2}{2}}\paren{t^n\dint_{B(x,t)}\frac{h(y)}{(t^2-\abs{y-x}^2)^{1/2}}dy}}.
        \end{align*}
        実はさらに,この係数は,$n$が偶数であるために,球の体積比\ref{cor-volume-of-ball-in-odd-and-even-dimension}から,
        \[\frac{1}{\gamma_{n+1}}\frac{2\om_n}{(n+1)\om_{n+1}}=\frac{1}{(n-1)!!}\frac{2}{n+1}\frac{2(n+1)!!}{n!!}=\frac{1}{n!!}\]
        よって,$\gamma_n:=n!!=2\cdot4\cdots(n-2)\cdot n$と定めると,$1/\gamma_n$が係数.
    \end{description}
\end{observation}

\begin{theorem}\label{thm-C-WE-in-even-dimension}
    $n=2m-2\ge2$を偶数,$g\in C^{m+1}(\R^n),h\in C^m(\R^n)\;(m\ge2)$とし,
    \begin{align*}
        u(x,t)&:=\frac{1}{\gamma_n}\Paren{\pp{}{t}\paren{\frac{1}{t}\pp{}{t}}^{\frac{n-2}{2}}\paren{t^n\dint_{B(x,t)}\frac{g(y)}{(t^2-\abs{y-x}^2)^{1/2}}dy}\\
        &\hspace{3cm}+\paren{\frac{1}{t}\pp{}{t}}^{\frac{n-2}{2}}\paren{t^n\dint_{B(x,t)}\frac{h(y)}{(t^2-\abs{y-x}^2)^{1/2}}dy}}.
    \end{align*}
    を考える.
    \begin{enumerate}
        \item $u\in C^2(\R^n\times\R_+)$.
        \item $u_{tt}-\Lap u=0\;\In\R^n\times\R^+$.
        \item 任意の$x^0\in\R^n$について,
        \[\lim_{\R^n\times\R^+\ni(x,t)\to(x^0,0)}u(x,t)=g(x^0),\quad\lim_{\R^n\times\R^+\ni(x,t)\to(x^0,0)}u_t(x,t)=h(x^0).\]
    \end{enumerate}
\end{theorem}
\begin{Proof}
    十分性も,奇数次元の場合の公式から従う.
\end{Proof}
\begin{remarks}[次数の違いとHuygensの原理]\mbox{}
    \begin{enumerate}
        \item 奇数次元の場合と違って,依存領域が$\partial B(x,t)$だけでなく,$B(x,t)$全体が要る.
        \item $n\ge3$の奇数次元では,wavefrontのみが影響を与える,ソニックブームのように.一方で,
        $n\ge2$の偶数次元では,wavefrontが通過した後も波は影響を持ち,干渉などの現象を持続敵に起こす,水面の定常的な模様のように.
    \end{enumerate}
\end{remarks}

\subsection{滑らかでないデータの場合の例}

\begin{example}[滑らかでない孤立波の発展はLaplace変換で解ける]
    初期値問題
    \[\begin{cases}
        u_{tt}=c^2u_{xx}&\In\R\times\R^+,\\
        u=g,\quad u_t=h&\on\R\times\{0\},
    \end{cases}\quad g(x):=1_{(-\pi/2,\pi/2)}\cos x\in C_c(\R).\]
    を考えると,$g$は$x=\pm\pi/2$にて可微分でさえない.
    したがって,d'Alembertの公式による解
    \[u(x,t)=\frac{1}{2}\Paren{g(x+ct)+g(x-ct)}\]
    は確かに解を与えていない.
    しかし実は,$g$が偶関数,$g'$が奇関数であることに注意すると,
    \[u(x,t):=\frac{1}{2}\Paren{g(x+ct)+g(ct-x)}\]
    は解を与えている.
\end{example}

\begin{example}
    $A,B,C\in\C\setminus\{0\}$が定めるPDE
    \[Au_{xx}+2Bu_{xt}+Cu_{tt}=0\]
    について,次は同値である:
    \begin{enumerate}
        \item ある実数$c_1<c_2$が存在して,任意の関数$f,g\in C^2(\R)$に対して,
        \[u(x,t):=f(x-c_1t)+g(x-c_2t)\]
        は$Au_{xx}+2Bu_{xt}+Cu_{tt}=0$の解になる.
        \item 方程式は双曲型である:$B^2-AC>0$である.
    \end{enumerate}
    \begin{description}
        \item[(1)$\Rightarrow$(2)] 必要条件をまず考えると,各微分を計算することより,任意の$f,g\in C^2(\R)$に対して
        \[Au_{xx}+2B_{xt}+Cu_{tt}=(A-2Bc_1+Cc_1^2)f''(x-c_1t)+(A-2Bc_2+Cc_2^2)g''(x-c_2t)=0\]
        が必要である.このためには,例えば$f(x)=x^31_{(0,\infty)},g(x)=x^31_{(-\infty,0)}\in C^2(\R)$を考えると,2階微分はそれぞれ$(0,\infty),(-\infty,0)$に台と持つから,
        \[\begin{cases}
            A-2Bc_1+Cc_1^2=0\\
            A-2Bc_2+Cc_2^2=0
        \end{cases}\]
        が必要であるが,これは$B^2-AC>0$と同値.
        \item[(2)$\Rightarrow$(1)] 実際にこれで十分であることは,上の連立方程式を満たす$c_1<c_2$を取れば,任意の$f,g\in C^2(\R)$に対して構成した$u$は常に方程式を満たすようになる.
    \end{description}
\end{example}

\section{非斉次波動方程式の解公式}

\subsection{Duhamelの原理による解公式}

\begin{problem}
    非斉次波動方程式の零化された初期値問題
    \[\text{(N-C-WE)}_s\quad\begin{cases}
        u_{tt}-\Lap u=f&\In\R^n\times\R^+,\\
        u=0\quad u_t=0&\on\R^n\times\{s\}.
    \end{cases}\]
    を考えるにあたって,次の斉次な初期値問題の解を$u(x,t;s)$とする:
    \[\begin{cases}
        u_{tt}(-;s)-\Lap u(-;s)=0&\In\R^n\times(s,\infty),\\
        u(-;s)=0,\quad u_t(-;s)=f(-,s)&\on\R^n\times\{s\}.
    \end{cases}\]
    そして,解の候補をその$s\in\R_+$についての積分とする.
\end{problem}
\begin{remarks}[Duhamelの原理の物理的直感]
    追加した斉次な初期値問題は,波動のない静寂の中で,外力$f(-,s)$を作用させた次の瞬間を表している.
    これを時間で積分すれば,経時的な外力$f$を取り込むことが可能である.
\end{remarks}

\begin{theorem}[非斉次波動方程式の解]\label{thm-N-C-WE}
    $n\ge2,f\in C^{\floor{n/2}+1}(\R^n\times\R_+)$とする.このとき,
    \[u(x,t):=\int_0^tu(x,t;s)ds\qquad(x\in\R^n,t\in\R_+)\]
    に対して,
    \begin{enumerate}
        \item $u\in C^2(\R^n\times\R_+)$.
        \item $u_{tt}-\Lap u=f\;\In\R^n\times\R^+$.
        \item 任意の$x^0\in\R^n$について,
        \[\lim_{\R^n\times\R^+\ni(x,t)\to(x^0,0)}u(x,t)=0,\quad\lim_{\R^n\times\R^+\ni(x,t)\to(x^0,0)}u_t(x,t)=0.\]
    \end{enumerate}
\end{theorem}
\begin{Proof}\mbox{}
    \begin{enumerate}
        \item $n$が奇数$n=2m-1$のとき,$f\in C^m(\R^n\times\R_+)$であるから,奇数次元の解公式\ref{thm-C-WE-in-odd-dimension}から$u(-;s)\in C^2(\R^n\times[s,\infty))$がわかる.
        よって,$u$も$C^2(\R^n\times\R_+)$.
        $n$が偶数$n=2m-2$の場合も定理\ref{thm-C-WE-in-even-dimension}より同様.
        \item 動く領域についての微分則\ref{thm-differentiation-of-integral-on-moving-region}によって,
        \[u_t(x,t)=\underbrace{u(x,t;t)}_{=0}+\int^t_0u_t(x,t;s)ds.\]
        \[u_{tt}(x,t)=u_t(x,t;t)+\int^t_0u_{tt}(x,t;s)ds=f(x,t)+\int^t_0u_{tt}(x,t;s)ds.\]
        また,\ref{exp-moving-region-appeared-in-N-C-WE}のように微分の定義に戻って考えることもできる.
        一方で,空間での微分は簡単で
        \[\Lap u(x,t)=\int_0^t\Lap u(x,t;s)ds=\int_0^t u_{tt}(x,t;s)ds.\]
        これより,$u_{tt}-\Lap u=f$がわかった.
        \item 積分を始めていないのだから,連続性から$u(x,0)=u_t(x,0)=\lim_{t\to0}u_t(x,t)=0$.
    \end{enumerate}
\end{Proof}

\begin{corollary}
    一般の非斉次波動方程式の初期値問題の解は,
    零化した初期値問題の解と,斉次化した初期値問題の解との
    和が与える.
\end{corollary}


\begin{example}[$n=1$の場合の解]
    斉次な初期値問題の解$u(x,t;s)$は,$n=1$の場合は次の形をしている:
    \[u(x,t;s)=\frac{1}{2}\int^{x+t-s}_{x-t+s}f(y,s)dy.\]
    積分区間の違いは,$t=s$での速度$f(y,s)$は地点$x$時刻$t$には$x\mp(t-s)$に到達するためである.
    よって,解は
    \[u(x,t)=\frac{1}{2}\int^t_0\int^{x+t-s}_{x-t+s}f(y,s)dyds=\frac{1}{2}\int^t_0\int^{x+u}_{x-u}f(y,t-u)dydu,\quad u:=t-s.\]
\end{example}

\begin{example}[$n=3$の場合の解]
    $n=3$の場合の修正問題の解は,初期時刻が$0$ではなく$s$であるので$t$の代わりに$t-s$を用いることに注意すれば,Kirchhoffの公式\ref{thm-Kirchhoff-formula}より,
    \[u(x,t;s)=\dint_{\partial B(x,t-s)}(t-s)f(y,s)dS(y).\]
    これをさらに積分すれば,
    \begin{align*}
        u(x,t)&=\int^t_0(t-s)\frac{1}{4\pi(t-s)^2}\int_{\partial B(0,t-s)}f(y,s)dS(y)ds\\
        &=\frac{1}{4\pi}\int^t_0\int_{\partial B(x,r)}\frac{f(y,t-r)}{r}dS(y)dr\\
        &=\frac{1}{4\pi}\int_{B(x,t)}\frac{f(y,t-\abs{y-x})}{\abs{y-x}}dy.
    \end{align*}
    この被積分関数を\textbf{遅延ポテンシャル}(retarded potential)という.
\end{example}

\subsection{滑らかでないデータの場合の例}

\begin{example}
    \[\begin{cases}
        u_{tt}=c^2u_{xx}+f(x,t)&\In\R\times\R^+,\\
        u=u_t=0&\on\R\times\{0\}.
    \end{cases}\qquad f(x,t)=F'''(x)t\;(F\in C^3(\R)),c>0\]
    を考えると,$f(x,t)$は$x$については可微分とは限らないため,Duhamelの原理を用いて解を求めることはできない.
    しかしこの場合,
    \[v(x,t):=u(x,t/c)+F'(x)\frac{t}{c^3}\]
    とおくと,これは
    \[\begin{cases}
        v_{tt}=v_{xx}&\In\R\times\R^+,\\
        v=0\quad v_t=\frac{1}{c^3}F'(x)&\on\R\times\{0\}.
    \end{cases}\]
    を満たす.$F'\in C^2(\R)$に注意すれば,これはd'Alembertの公式を適用することができる.
    実際,
    \[v_{xx}=u_{xx}(x,t/c)+F'''(x)\frac{t}{c^3},\quad v_{tt}=c^{-2}u_{tt}(x,t/c)\]
    であり,2つは
    \[c^2u_{xx}(x,t/c)+F'''(x)\frac{t}{c}=c^2u_{xx}(x,t/c)+f(x,t/c)=u_{tt}(x,t/c)\]
    と,確かに等号で結ばれている.
    よって,d'Alembertの公式より,
    \[v(x,t)=\frac{1}{2c^3}\int^{x+t}_{x-t}F'(y)dy.\]
    \[\therefore\qquad u(x,t)=\frac{1}{2c^3}\int^{x+tc}_{x-tc}F'(y)dy-F'(x)\frac{t}{c^2}.\]
\end{example}

\section{エネルギー法}

\begin{tcolorbox}[colframe=ForestGreen, colback=ForestGreen!10!white,breakable,colbacktitle=ForestGreen!40!white,coltitle=black,fonttitle=\bfseries\sffamily,
title=]
    波動方程式の解の一意性は,Laplace方程式や熱方程式の最大値原理のような比較原理からは出ない.そして滑らかさも落ちる/高次元では古典解の存在のためにデータに更なる滑らかさを必要とするのであった.
    もしかしたら,今までのアプローチは筋が悪いのかもしれない.
    実は,エネルギー法の考え方に沿ったノルムを通じた考察がとても筋が良いことが見えてくる.
\end{tcolorbox}

\begin{problem}
    $U\osub\R^n$を有界開集合,$\partial U$を$C^1$-級,
    $U_T:=U\times\ocinterval{0,T},\Gamma_T:=U_T\setminus U$を放物円筒・境界とする.
    極めて一般的な初期値・境界値問題
    \[\text{(C-WH)}\quad\begin{cases}
        u_{tt}-\Lap u=f&\In U_T,\\
        u=g&\on \Gamma_T,\\
        u_t=h&\on U\times\{0\}.
    \end{cases}\]
    を考える.
\end{problem}

\subsection{解の一意性}

\begin{definition}[energy]
    滑らかな関数$w\in C^2(U_T)$に対して,
    \[E(t):=\frac{1}{2}\int_U\Paren{w_t^2(x,t)+\abs{D_xw(x,t)}^2}dx,\qquad t\in[0,T]\]
    をエネルギーとする.
\end{definition}

\begin{lemma}[エネルギー保存則]
    斉次な波動方程式$u_{tt}-\Lap u=0$の古典解$u\in C^2(\R^n\times\R^+)$について,エネルギー$E(t)$は一定である.
\end{lemma}
\begin{Proof}
    $w\in C^2$であるから,微分と積分の交換が可能で,また
        \begin{align*}
            \pp{}{t}\abs{Dw}^2&=\pp{}{t}\paren{\paren{\pp{w}{x^1}}^2+\cdots+\paren{\pp{w}{x^n}}^2}\\
            &=2\paren{\pp{w}{x^1}\pp{^2w}{t\partial x^1}+\cdots+\pp{w}{x^n}\pp{^2w}{t\partial x^n}}=2D_xw\cdot D_xw_t.
        \end{align*}
        に注意すれば,
        \[\int_U(Dw|Dw_t)dx=\int_{\partial U}w_t\pp{w}{\nu}dS-\int_Uw_t\Lap wdx\]
        であり,第一項は境界条件$w=0\;\on\partial U\times[0,T]\subset\Gamma_T$より,$w_t=0$,
        すなわち零だから,
        \begin{align*}
            \dd{}{t}E(t)&=\int_U\Paren{w_tw_{tt}+(Dw|Dw_t)}dx\\
            &=\int_Uw_t\Paren{w_{tt}-\Lap w}dx=0.
        \end{align*}
        がわかる.すなわち,$E(t)$は$[0,T]$上定数である.
\end{Proof}

\begin{theorem}[波動方程式の解の一意性]
    初期値・境界値問題(C-WH)の古典解$u\in C^2(U_T)$は高々一つである.
\end{theorem}
\begin{Proof}\mbox{}
    \begin{description}
        \item[方針] $u,\wt{u}$を2つの解とし,差を$w:=u-\wt{u}$とすると,線形性から$f=g=h=0$とした問題の解である.
        \item[エネルギーは保存する] 補題より,$w$について$E(t)$は一定値$E(0)=0$を取る.
        \item[結論] よって,$w_t,Dw$も$U_T$上零である.境界条件$w=0\;\on U\times\{0\}$と併せて,$w=0\;\In U_T$.
    \end{description}
\end{Proof}

\subsection{依存領域}

\begin{tcolorbox}[colframe=ForestGreen, colback=ForestGreen!10!white,breakable,colbacktitle=ForestGreen!40!white,coltitle=black,fonttitle=\bfseries\sffamily,
title=]
    依存領域の経時変化の全体は,ちょうどソニックブームのような,
    現在一地点$(x_0,t_0)$をapexとし,次第に拡大していく円板の軌跡として得る円錐になる.
\end{tcolorbox}

\begin{notation}[backward wave cone]
    $u\in C^2$を$u_{tt}-\Lap u=0\;\In \R^n\times\R^+$の古典解とする.
    任意に$(x_0,t_0)\in\R^n\times\R^+$を取り,これを頂点とする\textbf{後向き波動円錐}
    \[K(x_0,t_0):=\Brace{(x,t)\in\R^n\times\R^+\mid t\in[0,t_0],\abs{x-x_0}\le t_0-t}\]
    を考える.
\end{notation}

\begin{definition}[local energy]
    時刻$t\in[0,t_0]$の周りの\textbf{局所エネルギー}を次のように定める:
    \[e(t):=\int_{B(x,t_0-t)}\Paren{u_t^2(x,t)+\abs{Du(x,t)}^2}dx,\qquad t\in[0,t_0].\]
\end{definition}

\begin{theorem}[波動方程式の解の有限伝播速度]
    $u\in C^2$を$u_{tt}-\Lap u=0\;\In \R^n\times\R^+$の古典解とする.
    時刻$0$のある球$B(x_0,t_0)\times\{0\}$上で$u\equiv u_t\equiv 0$とする.このとき,
    $u$は$K(x_0,t_0)$上で零である.
\end{theorem}
\begin{Proof}\mbox{}
    \begin{description}
        \item[局所エネルギーの微分] 積分領域$B(x_0,t_0-t)$は$t$が増加するにつれて小さくなっていくことと,Greenの第一恒等式\ref{cor-Green-identity}より,
        \begin{align*}
            \dd{e}{t}(t)&=\int_{B(x_0,t_0-t)}\Paren{u_tu_{tt}+(Du|Du_t)}dx-\frac{1}{2}\int_{\partial B(x_0,t_0-t)}\Paren{u_t^2+\abs{Du}^2}dS\\
            &=\int_{B(x_0,t_0-t)}u_t(\underbrace{u_{tt}-\Lap u}_{=0})dx+\int_{\partial B(x_0,t_0-t)}\pp{u}{\nu}u_tdS-\frac{1}{2}\int_{\partial B(x_0,t_0-t)}\Paren{u_t^2+\abs{Du}^2}dS\\
            &=\int_{\partial B(x_0,t_0-t)}\paren{\pp{u}{\nu}u_t-\frac{1}{2}u_t^2-\frac{1}{2}\abs{Du}^2}dS.
        \end{align*}
        ここで,Cauchy-Schwarzの不等式より,
        \[\Abs{\pp{u}{\nu}}=\abs{Du\cdot \nu}\le\abs{Du}.\]
        だから,
        \[\Abs{\pp{u}{\nu}u_t}\le\abs{u_t}\abs{Du}\le\frac{1}{2}u_t^2+\frac{1}{2}\abs{Du}^2.\]
        よって,$e'(t)\le0\;\on[0,t_0]$.これより,$\forall_{t\in[0,t_0]}\;e(t)\le e(0)=0$が仮定より成り立つ.
        実のところ,$e(t)\ge0$でもあるから,$e=0\;\on[0,t_0]$.
        \item[結論] これは,$u_t=Du=0\;\on K(x_0,t_0)$を要請する.そして初期条件$u=0\;\on B(x_0,t_0)\times\{0\}$より,$u=0\;\on K(x_0,t_0)$でもある.
    \end{description}
\end{Proof}
\begin{remarks}
    この結果は,境界条件$g=u,h=u_t\;\on\R^n\times\{0\}$に次元に応じた適切な滑らかさを課したならば,
    一般次元の公式\ref{thm-C-WE-in-odd-dimension}, \ref{thm-C-WE-in-even-dimension}の依存領域から知っている.
    が,エネルギー法によると,境界条件の滑らかさも要らず,証明も簡明である.
\end{remarks}

\subsection{エネルギー法のその他の例}

\begin{example}[外力がある場合]
    境界値問題
    \[\begin{cases}
        u_{tt}-c^2u_{xx}+f(u)=0&\In\R^+\times\R^+\\
        u(0,t)=u_x(0,t)=0&t\in\R^+.
    \end{cases}\]
    を満たす古典解$u\in C^2_c(\R_+\times\R_+)$について,
    \[E(t):=\frac{1}{2}\int^\infty_0(u_t^2+c^2u_x^2+2F(u))dx,\qquad F(x):=\int^x_0f(y)dy\]
    とすると,無限遠での減衰条件と$x=0$での境界条件から$u_x(\infty,t)u_t(\infty,t)-u_x(0,t)u_t(0,t)=0$で,
    \[\int^\infty_0u_xu_{tx}dx=\SQuare{u_xu_t}^\infty_0-\int^\infty_0u_{xx}u_tdx=-\int^\infty_0u_{xx}u_tdx.\]
    だから,
    \begin{align*}
        \dd{}{t}E(t)&=\int^\infty_0(u_tu_{tt}+c^2u_xu_{tx}+f(u)u_t)dx\\
        &=\int^\infty_0u_t(u_{tt}-c^2u_{xx}+f(u))dx=0.
    \end{align*}
\end{example}

\begin{example}[Neumann型の境界条件の場合]
    境界値問題
    \[\begin{cases}
        u_{tt}-c^2u_{xx}=0&\R^+\times\R^+,\\
        u_x(0,t)+Au(0,t)=0&t>0.
    \end{cases}\qquad c>0,A\in\R\]
    を満たす古典解$u\in C^2_c(\R_+\times\R_+)$について,
    \[E(t):=\frac{1}{2}\int^\infty_0(u_t^2+c^2u_x^2)dx+\frac{1}{2}au(0,t)^2,\qquad a:=-\rednote{c^2}A\]
    と定めると,
    \[\int^\infty_0u_xu_{tx}dx=\SQuare{u_xu_t}^\infty_0-\int^\infty_0u_{xx}u_tdx=-u_x(0,t)u_t(0,t)-\int^\infty_0u_{xx}u_tdx\]
    で,境界条件より$-u_{x}(0,t)=Au(0,t)$であるから,
    \begin{align*}
        \dd{E(t)}{t}&=\int^\infty_0(u_tu_{tt}+c^2u_xu_{tx})dx+au(0,t)u_t(0,t)\\
        &=\int^\infty_0u_t(u_{tt}-c^2u_{xx})dx+A\rednote{c^2}u(0,t)u_t(0,t)+au(0,t)u_t(0,t)=0.
    \end{align*}
\end{example}

\chapter{1階の非線型方程式}

\begin{quotation}
    完全積分・特性曲線という局所的な方法と,Hamilton-Jacobiの粘性解・Burgers方程式のエントロピー解などの大域的な方法がある.
\end{quotation}

\begin{problem}
    $U\osub\R^n$上の$C^1$-級関数$F:\R^n\times\R\times\o{U}\to\R$が与える道関数$u:\o{U}\to\R$についての方程式
    \[F(Du,u,x)=0\quad\In U\]
    と境界条件
    \[u=g\quad\on\Gamma,\qquad(\Gamma\subset\partial U,g:\Gamma\to\R).\]
    を考える.
\end{problem}

\begin{notation}\mbox{}
    \begin{enumerate}
        \item $F=F(p,z,x)$とし,$p,z,x$はそれぞれ$Du,u,x$を代入するところの変数とする.
    \end{enumerate}
\end{notation}



\section{完全積分とHamilton-Jacobi方程式}

\begin{tcolorbox}[colframe=ForestGreen, colback=ForestGreen!10!white,breakable,colbacktitle=ForestGreen!40!white,coltitle=black,fonttitle=\bfseries\sffamily,
title=]
    解の族が見つかった場合,その包絡線も解である.
    前者を完全積分,後者を特異積分ともいう.
\end{tcolorbox}

\begin{remark}
    特異積分が見つかれば解の全てというわけではない.反例は,
    \[F(Du,u,x)=F_1(Du,u,x)F_2(Du,u,x)=0\]
    という形のPDEが与える.
\end{remark}

\subsection{完全積分の定義と例}

\begin{tcolorbox}[colframe=ForestGreen, colback=ForestGreen!10!white,breakable,colbacktitle=ForestGreen!40!white,coltitle=black,fonttitle=\bfseries\sffamily,
title=]
    大雑把に言って,次元$\R^n$と同数の$n$個の任意定数を含んでる解を完全積分という.
    完全解が得られれば,他の解も導く算譜があるが,それで全てとは限らない.
\end{tcolorbox}

\begin{problem}
    \begin{equation}\label{eq-5-1}
        F(Du,u,x)=0
    \end{equation}
    について,ある開集合$A\osub\R^n$に添字付けられた$C^2$-級解の族$(u=u(x;a))_{a\in A}$を得ているとする.
    \[(D_au,D_{xa}^2u):=\begin{pmatrix}u_{a1}&u_{x_1a_1}&\cdots&u_{x_na_1}\\\vdots&\vdots&\ddots&\vdots\\u_{a_n}&u_{x_1a_n}&\cdots&u_{x_na_n}\end{pmatrix} \in M_{n,n+1}(C(U))\]
    と表す.
\end{problem}

\begin{definition}[complete integral]
    $C^2$-級関数$u=u(x;a)$が$U\times A$上の\textbf{完全積分}であるとは,次の2条件を満たすことをいう:
    \begin{enumerate}
        \item 任意の$a\in A$に対して$u(x;a)$は\ref{eq-5-1}の解である.
        \item $U\times A$上で$\rank(D_au,D_{xa}^2u)=n$.
    \end{enumerate}
\end{definition}
\begin{remarks}
    条件(2)は族$(u(x;a))$が$a_1,\cdots,a_n$の全てに本質的に依存していることを要請している.
    実際,ある$B\osub\R^{n-1}$に添字付けられた$C^2$-級解の族$v=v(x,b)\;(x\in U,b\in B)$について,ある$C^1$-可微分写像$\psi=(\psi^1,\cdots,\psi^{n-1}):A\to B$が存在して,
    \[u(x;a)=v(x;\psi(a)),\qquad(x\in U,a\in A)\]
    を満たすならば,(2)は満たさない.実際,このとき,
    \[u_{x_ia_j}(x;a)=\sum_{k=1}^{n-1}v_{x_ib_k}(x,\psi(a))\psi^k_{a_j}(a),\qquad(i,j\in[n])\]
    \[u_{a_j}(x;a)=\sum_{k=1}^{n-1}v_{b_k}(x;\psi(a))\psi^k_{a_j}(a),\qquad(j\in[n]).\]
    であるから,$(D_au,D_{xa}^2u)$のどの$n\times n$部分行列も階数落ちである.
\end{remarks}

\begin{example}[Clairaut方程式]
    \[x\cdot Du+f(Du)=u,\qquad(f:\R^n\to\R).\]
    の完全積分は
    \[u(x;a):=a\cdot x+f(a),\qquad(x\in U,a\in\R^n)\]
    が与える.
\end{example}
\begin{Proof}[[確認]]\mbox{}
    \begin{enumerate}
        \item 次のように計算できる:
        \begin{align*}
            u_{x_i}(x;a)&=a_i,&Du&=a
        \end{align*}
        より,
        \[x\cdot a+f(a)=u(x;a).\]
        \item (1)の計算から$D^2_{xa}u=I_n$ですでにフルランクは確定であるが,一応
        \[u_{a_i}=x_i+(f'(a))_i,\qquad D_au=x+f'(a).\]
        と計算出来る.
    \end{enumerate}
\end{Proof}

\begin{example}[eikonal方程式]
    \[\abs{Du}=1\]
    の完全積分は
    \[u(x;a,b):=a\cdot x+b\qquad(x\in U,a\in\partial B(0,1),b\in\R).\]
    が与える.
\end{example}
\begin{Proof}[[確認]]\mbox{}
    \begin{enumerate}
        \item 次の計算より,$\abs{Du}=\abs{a}=1$.
        \[u_{x_i}(x;a,b)=a_i,\qquad Du=a.\]
        \item $b\in\R$も添字付けているために,自由度は$n$確保されている.
    \end{enumerate}
\end{Proof}

\begin{example}[Hamilton-Jacobi方程式]
    \[u_t+H(Du)=0,\qquad(H:\R^n\to\R).\]
    の完全積分は
    \[u(x,t;a,b):=a\cdot x-tH(a)+b,\qquad(x\in\R^n,t\ge0,a\in\R^n,b\in\R).\]
\end{example}
\begin{Proof}[[確認]]\mbox{}
    \begin{enumerate}
        \item 次の計算より,$u_t+H(Du)=-H(a)+H(a)=0$.
        \[D_xu=a,\qquad u_t=-H(a).\]
        \item 次の計算から,$(x,t;a,b)\in\R^{n+1}\times \R^{n+1}$上フルランクである:
        \[D_{(x,t)(a,b)}=\mtrx{I_n}{\b{0}_n}{-DH(a)^\top}{0}.\]
        \[D_{(a,b)}=\vctr{x}{1}\]
    \end{enumerate}
\end{Proof}

\subsection{包絡線}

\begin{tcolorbox}[colframe=ForestGreen, colback=ForestGreen!10!white,breakable,colbacktitle=ForestGreen!40!white,coltitle=black,fonttitle=\bfseries\sffamily,
title=]
    その$n$個の任意定数についての偏微分係数が消えているならば,
    完全積分(解の$n$-変数群)の包絡線として,
    任意の$n-1$変数関数に依存する解を構成することが可能になる.
\end{tcolorbox}

\begin{definition}[envelopes / singular integral]
    $u=u(x,a)$を$U\times A\subset\R^n\times\R^m$上の$C^1$-級関数,$a=\phi(x)$を方程式
    \[D_au(x;a)=0\qquad(x\in\ U,a\in A).\]
    の$C^1$-級解とする.このとき,
    関数$v(x):=u(x;\phi(x))\;(x\in U)$を関数族$\{u(-;a)\}_{a\in A}$の\textbf{包絡線}または\textbf{特異積分}という.
\end{definition}


\begin{theorem}[包絡線も解である]
    $\{u(-;a)\}_{a\in A}$を\ref{eq-5-1}の解の族,$v\in C^1(U)$をその包絡線とする.
    このとき,$v$も\ref{eq-5-1}の解である.
\end{theorem}
\begin{Proof}
    $v(x):=u(x;\phi(x))$の導関数は,$\phi$が$D_au(x;\phi(x))=0\;(x\in U)$を満たすことより,
    \[v_{x_i}(x)=u_{x_i}(x;\phi(x))+\sum_{j=1}^mu_{a_j}(x,\phi(x))\phi^j_{x_i}(x)=u_{x_i}(x;\phi(x)).\]
    と計算出来るから,
    \[F(Dv(x),v(x),x)=F(Du(x;\phi(x)),u(x;\phi),x)=0.\]
\end{Proof}
\begin{remarks}
    $v$のグラフは,$\{u(-;a)\}_{a\in A}$のうち$u(-;a)\;(a=\phi(x))$に接するから,$Dv=D_xu(-;a)\;(a=\phi(x))$が成り立つ.
\end{remarks}

\begin{example}
    \[u^2(1+\abs{Du}^2)=1.\]
    の完全積分は
    \[u(x,a)=\pm(1-\abs{x-a}^2)^{1/2},\qquad(\abs{x-a}<1).\]
    が与える.これについて,$a=\phi(x)=x$とすれば,包絡線の方程式
    \[D_au=\frac{\mp(x-a)}{(1-\abs{x-a}^2)^{1/2}}=0\]
    の解になるから,$v(x)=u(x,x)\equiv\pm1$は特異積分である.
\end{example}

\subsection{一般積分}

\begin{tcolorbox}[colframe=ForestGreen, colback=ForestGreen!10!white,breakable,colbacktitle=ForestGreen!40!white,coltitle=black,fonttitle=\bfseries\sffamily,
title=]
    $n$個の変数の間に関数関係がある場合,これを使って変数を消去でき,代わりに消去した定数の分だけの
    任意関数を含む式が得られる.これをPDEの一般解という.
\end{tcolorbox}

\begin{definition}[general integral]
    $h$に依存した\textbf{一般積分}とは,次の関数族
    \[u'(x;a')=u(x;a',h(a')),\qquad(u\in U,a'\in A')\]
    の$C^1$-級の包絡線$v'=v'(x)$をいう.
\end{definition}

\begin{example}
    $H(p)=\abs{p}^2$をHamiltonianとするHamilton-Jacobi方程式の,
    $h\equiv0$に依存した一般積分は,完全積分の(部分)族
    \[u'(x,t;a)=x\cdot a-t\abs{a}^2.\]
    の包絡線だから,包絡線の方程式の解
    \[D_au'=x-2ta=0\quad\Leftrightarrow\quad a=\frac{x}{2t}.\]
    の下で計算すると,
    \[v'(x,t)=u\paren{x,t;\frac{x}{2t}}=x\cdot\frac{x}{2t}-t\Abs{\frac{x}{2t}}^2=\frac{\abs{x^2}}{4t},\qquad(x\in\R^n,t>0).\]
    これは再びHamilton-Jacobi方程式の解である.
\end{example}
\begin{Proof}[[確認]]
    実際,$v_t'=-\frac{\abs{x}^2}{4t^2}$として,
    \[v'_{x_i}=\frac{x_i}{2t},\qquad Dv'=\frac{x}{2t}\qquad\therefore\quad\abs{Dv'}^2=\frac{\abs{x}^2}{4t^2}.\]
\end{Proof}

\section{特性曲線法}

\begin{tcolorbox}[colframe=ForestGreen, colback=ForestGreen!10!white,breakable,colbacktitle=ForestGreen!40!white,coltitle=black,fonttitle=\bfseries\sffamily,
title=]
    $U$内の曲線であって,その上では簡単な常微分方程式になるようなものを探す.
    そのような曲線の要件は$\dot{\b{x}}(s)=D_pF(\b{p}(s),z(s),\b{x}(s))$であり,
    これによって方程式系は1階系になる.
\end{tcolorbox}

\begin{problem}
    $C^1$-級関数$F,g:\Gamma\to\R\;(\Gamma\subset\partial U)$について,境界値問題
    \begin{equation}\label{eq-5-2}
        \begin{cases}
            F(Du,u,x)=0&\In U,\\
            u=g&\on\Gamma.
        \end{cases}
    \end{equation}
    を考える.
\end{problem}

\subsection{特性方程式の一般論}

\begin{tcolorbox}[colframe=ForestGreen, colback=ForestGreen!10!white,breakable,colbacktitle=ForestGreen!40!white,coltitle=black,fonttitle=\bfseries\sffamily,
title=]
    $U\osub\R^n$上の1階PDEをある曲線(=1次元多様体)上に制限し,
    さらに補助変数を増やして$2n+1$元連立とすることで,
    ODEを解く問題に変換する手法である.
\end{tcolorbox}

\begin{notation}
    $\x:I\to U$を径数曲線とする.
    \begin{enumerate}
        \item $\p(s):=Du(\x(s))$でその上の勾配ベクトル場を表す.
        \item $z(s):=u(\x(s))$でその上の$u$の値を表す.
        \item $D_pF(\p(s),z(s),\x(s))$を方程式$F$の$U$上の特性方向と呼ぼう.
    \end{enumerate}
\end{notation}

\begin{theorem}[特性方程式の構造 \cite{Evans} 3.2.1 Th'm.1]\label{thm-structure-of-characteristic-curve}
    $u\in C^2(U)$を一階PDE \ref{eq-5-2}の$C^2$-級解とし,曲線$\b{x}:I\to U$の接ベクトルは,方程式$F$の特性方向に一致するとする:
    \[\dot{\b{x}}(s)=D_pF(\b{p}(s),z(s),\b{x}(s)),\qquad s\in I.\]
    このとき,
    \begin{enumerate}
        \item $\b{p}(-)=Du(\b{x}(-))$の微分は次を満たす:
        \[\dot{\b{p}}(s)=-D_xF(\b{p}(s),z(s),\b{x}(s))-D_zF(\b{p}(s),z(s),\b{x}(s))\b{p}(s).\]
        \item $z(-)=u(\b{x}(-))$の微分は次を満たす:
        \[\dot{z}(s)=D_pF(\b{p}(s),z(s),\b{x}(s))\cdot\b{p}(s).\]
        \item 特性帯上で$F$は定値である:
        \[F(\b{p}(s),z(s),\b{x}(s))=0\qquad(s\in I).\]
    \end{enumerate}
\end{theorem}
\begin{remarks}[追加した特性曲線のパラメータの意味]
    $z(s)$は曲線上の$\b{x}(s)$での$u$の値,$\b{p}(s)$は$\b{x}(s)$の$u$の勾配ベクトルを格納している.
\end{remarks}
\begin{Proof}\mbox{}
    \begin{enumerate}
        \item \begin{enumerate}[{Step}1]
            \item 定義$\b{p}(s):=Du(\b{x}(s))$より,
            \begin{equation}\label{eq-derivarive-of-p}
                p^i(s)=u_{x_i}(\b{x}(s))
            \end{equation}
            であるから,両辺を$s$で微分すると,
            \[\dot{p^i}(s)=\sum_{j=1}^nu_{x_ix_j}(\b{x}(s))\cdot\dot{x^j}(s).\]
            \item 上式の右辺の2階の微分項を消去することを考える.元の微分方程式\ref{eq-5-2}の$x_i$での微分は
            \[\sum_{j=1}^nF_{p_j}(Du,u,x)u_{x_jx_i}+F_z(Du,u,x)u_{x_i}+F_{x_i}(Du,u,x)=0.\]
            これを$x=\b{x}(s)$上で考えて,仮定$\dot{x^j}(s)=F_{p_j}(\b{p}(s),z(s),\b{x}(s))$と$\dot{p^i}$の式\ref{eq-derivarive-of-p}を用いることより,
            \[\underbrace{\sum_{j=1}^nF_{p_j}(Du,u,x)u_{x_ix_j}(\b{x}(s))}_{=\dot{p^i}(s)}+F_z(\b{p}(s),z(s),\b{x}(s))p^i(s)+F_{x_i}(\b{p}(s),z(s),\b{x}(s))=0.\]
            \[\therefore\qquad\quad\dot{p^i}(s)=-F_{x_i}(\b{p}(s),z(s),\b{x}(s))-F_z(\b{p}(s),z(s),\b{x}(s))p^i(s).\]
        \end{enumerate}
        \item 定義$z(s):=u(\b{x}(s))$と,$\dot{p^i}$の表示\ref{eq-derivarive-of-p}と$\cdot{\b{x}}$に関する仮定を併せると,
        \[\dot{z}(s)=\sum_{j=1}^nu_{x_j}(\b{x}(s))\cdot\dot{x^j}(s)=\sum_{j=1}^np^i(s)F_{p_j}(\b{p}(s),z(s),\b{x}(s)).\]
        \item $\x$は$U$上の曲線で,$\p,z$の定義より,PDE $F(Du,u,x)=0\;\In U$の要請で当然
        \[F(\p(s),z(s),\x(s))=0,\qquad(s\in I).\]
        は成り立つ.
        %(1)と(2),を式
        %\[\dd{F}{s}(\p(s),z(s),\x(s))=F_p\cdot\dot{\p}(s)+F_z\dot{z}(s)+F_x\cdot\dot{\x}(s)\]   
        %に代入することで,
        %\begin{align*}
        %    \dd{F}{s}(\p(s),z(s),\x(s))&=F_p\cdot(-F_x-F_z\p)+F_z(F_p\cdot\p)+F_x\cdot F_p\\
        %    &=-F_p\cdot(F_z\p)+F_zF_p\cdot\p=0,
        %\end{align*}
        %が従うためである.
    \end{enumerate}
\end{Proof}
\begin{remarks}
    この定理より,特性方向場(Mongeのベクトル場)の積分曲線$\x$を取り,この形を決定するのに必要なだけ$z$の方程式,$\p$の方程式を追加して解けば,
    特性曲線上の$u$の値が得られる.
    特性方向の重要性は,この定理では証明のStep2において2階の微分項が消えて,1階系を得るという点でも代数的に表現されている.
\end{remarks}

\begin{definition}[characteristics, projected characteristics]
    関数$(\b{p},z,\b{x})\in\R^{2n+1}$を\textbf{特性(帯)}という.
    特に曲線$\b{x}$は\textbf{射影された特性曲線}という.
\end{definition}

\subsection{線型PDEでの例}

\begin{problem}[線型斉次PDEの特性方程式系]\label{prob-1st-order-linear-PDE}
    線型斉次PDE
    \[F(Du,u,x)=\b{b}(x)\cdot Du(x)+c(x)u(x)=0\qquad(x\in U).\]
    を考えると,$F(p,z,x)=\b{b}(x)\cdot p+c(x)z$であるから,$D_pF=\b{b}(x)$より,射影特性曲線の方程式は
    \[\dot{\b{x}}(s)=D_pF(\b{p}(s),z(s),\b{x}(s))=\b{b}(\b{x}(s)).\]
    これだけでは変数$s$が消去できないので,定理\ref{thm-structure-of-characteristic-curve}が示す$z$についての必要条件より,
    方程式
    \[\dot{z}(s)=\b{b}(\b{x}(s))\cdot\b{p}(s)=-c(\b{x}(s))z(s)\]
    を追加する.最後の等式は$F(\b{p}(s),z(s),\b{x}(s))=0$による.
    以上より,特性方程式系は
    \[\begin{cases}
        \dot{\b{x}}(s)=\b{b}(\b{x}(s))\\
        \dot{z}(s)=-c(\b{x}(s))z(s).
    \end{cases}\]
    となり,$\p$が出現しないためにこの2本で十分である.
    $\b{p}$の式が出て来ない理由は境界条件の特性性の議論\ref{prob-1st-order-linear-PDE-revisited}で解る.
\end{problem}

\begin{example}[特性方程式が解ける変数係数線型PDEの例]
    \[\begin{cases}
        x_1u_{x_2}-x_2u_{x_1}=u&\In \R^+\times\R^+,\\
        u=g&\on\R^+\times\{0\}.
    \end{cases}\]
    の解
    \[u(x)=g((x_1^2+x_2^2)^{1/2})e^{\arctan\paren{\frac{x_2}{x_1}}}.\]
    は特性曲線法によって得られる.
\end{example}
\begin{Proof}
    これは
    \[F(p,z,x)=\vctr{-x_2}{x_1}p-z\]
    より,特性方向は$\b{b}(x):=\vctr{-x_2}{x_1}$.
    特性方程式系は
    \[\begin{cases}
        \dot{\b{x}}=\vctr{\dot{x}^1}{\dot{x}^2}=\vctr{-x^2}{x^1}\\
        \dot{z}=z
    \end{cases}.\]
    であるから,これを解くと,一般解は
    \[\vctr{x^1}{x^2}=\mtrx{\cos s}{-\sin s}{\sin s}{\cos s}\vctr{C_1}{C_2},\qquad C_1,C_2\in\R.\]
    \[z=Ce^s,\qquad C\in\R.\]
    初期値を
    \[\b{x}(0)=\vctr{C_1}{C_2}=\vctr{x^0}{0}\in\Gamma\;(x^0>0)\]
    と要請すると,その上での値は$z(0)=u(x^0,0)=g(x^0)$で,特性曲線は次のように得られた:
    \[\begin{cases}
        x^1(s)=x^0\cos s,\quad x^2(s)=x^0\sin s\\
        z(s)=z^0e^s=g(x^0)e^s.
    \end{cases},\qquad\paren{x^0\ge0,0\le s\le\frac{\pi}{2}}.\]
    第一象限$\R^+\times\R^+$上の同心円族をなしている.

    以上より,
    任意の点$(x_1,x_2)\in U$については,
    \[\vctr{x_1}{x_2}=\vctr{x^1(s)}{x^2(s)}=\vctr{x^0\cos s}{x^0\sin s}\quad\Leftrightarrow\quad\vctr{x^0}{s}=\vctr{\sqrt{x_1^2+x_2^2}}{\arctan\paren{\frac{x_2}{x_1}}}.\]
    より,
    \[u(x)=u(x^1(s),x^2(s))=z(s)=g(x^0)e^s=g((x_1^2+x_2^2)^{1/2})e^{\arctan\paren{\frac{x_2}{x_1}}}.\]
    を得る.
\end{Proof}

\subsection{準線型PDEでの例}

\begin{tcolorbox}[colframe=ForestGreen, colback=ForestGreen!10!white,breakable,colbacktitle=ForestGreen!40!white,coltitle=black,fonttitle=\bfseries\sffamily,
title=]
    準線型方程式に対する特性方程式系は$n+1$本,実は$n=2$のときは単独の常微分方程式に退化する.
\end{tcolorbox}

\begin{problem}[準線型PDEの特性方程式系]\label{prob-1st-order-quasi-linear-PDE}
    さらに係数$\b{b}$が$u(x)$にも依存する準線型であるとき,方程式は一般に
    \[F(Du,u,x)=\b{b}(x,u(x))\cdot Du(x)+c(x,u(x))=0.\]
    と表せ\footnote{$c$に$p(s)$を含むと準戦型ではなくなる.},このとき$F(p,z,x)=\b{b}(x,z)\cdot p+c(x,z)$であるから,$D_pF=\b{b}(x,z)$より,射影特性曲線の方程式は
    \[\dot{\b{x}}(s)=\b{b}(\b{x}(s),z(s)).\]
    定理\ref{thm-structure-of-characteristic-curve}より,さらに$z(s)=u(\b{x}(s))$に関して
    \[\dot{z}(s)=\b{b}(\b{x}(s),z(s))\cdot\b{p}(s)=-c(\b{x}(s),z(s))\]
    も必要.
    やはり主要部に$\p$は出現しないから,$n+1$元連立式で十分であるから,
    以上より特性方程式系は
    \[\begin{cases}
        \dot{\b{x}}(s)=\b{b}(\b{x}(s),z(s))\\
        \dot{z}(s)=-c(\b{x}(s),z(s)).
    \end{cases}\]
\end{problem}

\begin{example}[特性方程式が解ける準線型PDEの例]
    \[\begin{cases}
        u_{x_1}+u_{x_2}=u^2&\In \R\times\R^+,\\
        u=g&\on\R\times\{0\}.
    \end{cases}\]
    の解には
    \[u(x_1,x_2)=\frac{g(x_1-x_2)}{1-x_2g(x_1-x_2)},\qquad(1-x_2g(x_1-x_2)\ne0).\]
    がある.
\end{example}
\begin{Proof}
    \[F(p,z,x)=\vctr{1}{1}\cdot p-z^2=0\]
    と表せるから,特性方程式は
    \[\begin{cases}
        \dot{\b{x}}(s)=\b{b}(\b{x}(s),z(s))=\vctr{1}{1}\\
        \dot{z}=z^2.
    \end{cases}\]
    曲線の始点を$s=0$のとき$\vctr{x^0}{0}\in\Gamma\;(x^0\in\R,s\ge0)$であるとすれば,
    \[\x(s)=\vctr{x^1}{x^2}=\vctr{x^0+s}{s}.\]
    続いて,$s=0$で与えられた初期値$z(0)=u(x^0,0)=g(x^0)$より,
    \[z(s)=\frac{z^0}{1-sz^0}=\frac{g(x^0)}{1-sg(x^0)}.\]

    以上の考察から,$(x_1,x_2)\in U$での値$u(x_1,x_2)$は,
    \[(x_1,x_2)=(x^0+s,s)\quad\Leftrightarrow (x^0,s)=(x_1-x_2,x_2)\]
    の通りに曲線$\x$の初期値$x^0$と時刻$s$を取れば良いから,
    \[u(x)=z(x_2)=\frac{g(x_1-x_2)}{1-x_2g(x_1-x_2)}.\]
\end{Proof}
\begin{remarks}[半線形輸送の奥深さ]
    この例は半線形な輸送方程式であり,$x_2$が時刻を表し,ある時刻を越すと解が発散することが解る.
    一方で,符号を変えた問題
    \[u_t-a(x)u_x=u^2\]
    は半線形熱方程式または\textbf{反応拡散方程式}といい,藤田宏が拓いて石毛先生らが継いでいる研究対象でもある.
    $u^2$を反応項,$a$を拡散項といい,2つの要素がせめぎ合って,爆発解の有無が変わる.
\end{remarks}

\subsection{完全非線形PDEでの例}

\begin{tcolorbox}[colframe=ForestGreen, colback=ForestGreen!10!white,breakable,colbacktitle=ForestGreen!40!white,coltitle=black,fonttitle=\bfseries\sffamily,
title=]
    一番の違いは,$\b{p}$の方程式も含めた$2n+1$本の全てが必要となる点である.
    今まではMonge錐が退化して一本の直線となっていたのである.
    特性帯と呼ばれる概念を考える.
\end{tcolorbox}

\begin{problem}[完全非線形PDEの特性方程式系]
    \[\begin{cases}
        u_{x_1}u_{x_2}=u&\In\R^+\times\R,\\
        u=x_2^2&\on\{0\}\times\R.
    \end{cases}\]
    の解に
    \[u(x_1,x_2)=\frac{(x_1+4x_2)^2}{16}.\]
    がある.
\end{problem}
\begin{Proof}
    \[F(p,z,x)=p_1p_2-z\]
    より,特性方程式は
    \[\begin{cases}
        \dot{\b{p}}(s)=-D_xF-(D_zF)\b{p}=\b{p}\\
        \dot{z}(s)=D_p\cdot\b{p}=2p_1p_2=2z^2\\
        \dot{\b{x}}(s)=D_pF=\vctr{p_2}{p_1}
    \end{cases}\]
    \begin{description}
        \item[特性方程式を解く] 
        \begin{enumerate}
            \item 第一式は$s=0$での初期値を$\vctr{p^0_1}{p^0_2}$とおくと,$\b{p}(s)=\vctr{p^0_1e^s}{p^0_2e^s}$.
            \item 第二式は$z(s)=z_0e^{2s}$で,初期条件$z(0)=z_0=u(0,x^0)=(x^0)^2$と併せると,$z(s)=(x^0)^2e^{2s}$.
            \item 第三式は$\dot{\x}(s)=\vctr{p^0_2e^s}{p_1^0e^s}$であるから,初期値を$\x(0)=\vctr{0}{x^0}$とすると,
            \[\x(s)=\vctr{p_2^0e^s-p_2^0}{p^0_1(e^s-1)+x^0}.\]
        \end{enumerate}
        \item[初期値を解決する] $x^0\in\R,s\ge0,z^0=(x^0)^2$に加えて,
        $p^0_2=u_{x_2}(0,x_2)=2x^0$も解る.相互関係
        \[p^0_1p^0_2=u_{x_1}(0,x^0)u_{x_2}(0,x^0)=z(\x(0))=z^0=(x^0)^2.\]
        から,$p_1^0=\frac{x^0}{2}$.
        以上より,初期値を代入すると
        \[\begin{cases}
            \x(s)=\vctr{2x^0(e^s-1)}{\frac{x^0}{2}(e^s-1)+x^0}\\
            z(s)=(x^0)^2e^{2s}\\
            \b{p}(s)=\vctr{\frac{x^0}{2}e^s}{2x^0e^s}
        \end{cases}\]
        \item[解の導出] 任意の$(x_1,x_2)\in U$について,
        \[\vctr{x_1}{x_2}=\x(s)=\vctr{2x^0(e^s-1)}{\frac{x^0}{2}(e^s+1)}\quad\Leftrightarrow\quad\vctr{x^0}{e^s}=\vctr{\frac{4x_2-x_1}{4}}{\frac{x_1+4x_2}{4x_2-x_1}}.\]
        \[u(x_1,x_2)=z(s)=(x^0)^2e^{2s}=\paren{\frac{4x_2-x_1}{4}}^2\paren{\frac{x_1+4x_2}{4x_2-x_1}}^2=\frac{(x_1+4x_2)^2}{16}.\]
    \end{description}
\end{Proof}

\section{2次元の場合の詳論と一般解}

\subsection{ODEとの違い:一般解の概念}

\begin{tcolorbox}[colframe=ForestGreen, colback=ForestGreen!10!white,breakable,colbacktitle=ForestGreen!40!white,coltitle=black,fonttitle=\bfseries\sffamily,
title=]
    ODEは任意定数であったところが,PDEでは任意関数になる.
\end{tcolorbox}

\begin{example}[\cite{Strauss-PDE}]\mbox{}
    \begin{enumerate}
        \item $u_{xx}=0$の一般解は
        \[u(x,y)=f(y)x+g(y),\qquad f,g\in\Map(\R,\R)\]
        が与える.
        \item $u_{xx}+u=0$の一般解は
        \[u(x,y)=f_0(y)\cos x+g_0(y)\sin x=f_1(y)\cos(x+g_1(y))\qquad f_i,g_i\in\Map(\R,\R)\]
        が与える.
        \item $u_{xy}=0$の一般解は,
        \[u(x,y)=F(y)+G(x).\]
    \end{enumerate}
\end{example}

\subsection{特性曲線法のアイデア}

\begin{tcolorbox}[colframe=ForestGreen, colback=ForestGreen!10!white,breakable,colbacktitle=ForestGreen!40!white,coltitle=black,fonttitle=\bfseries\sffamily,
title=]
    定数係数の場合,
    \[au_x+bu_y=0\]
    とは,$(a,b)$方向微分が零であることに同値な条件である.
    これは$(a,b)$方向=特性方向が定める積分曲線に沿って$u$は定値であることを意味する.
\end{tcolorbox}

\begin{observation}
    定数係数の場合,
    \[au_x+bu_y=0\]
    とは,$(a,b)$方向微分が零であることに同値な条件である.
    これは$(a,b)$方向=特性方向の直線
    \[bx-ay=\const\]
    上で$u$が定値であることを意味する.
    よって$u$の値はこの曲線に依存し,$u(x,y)=f(bx-ay)\;(f:\R\to\R)$と表せる.
\end{observation}

\begin{problem}[2次元上の準線型PDE]
    領域$D\osub\R^3$上の
    準線形PDE
    \[\p(x,y,u)\cdot Du=a(x,y,u)u_x+b(x,y,u)u_y=c(x,y,u),\qquad\In D.\]
    を考える.ただし,$a,b,c\in C^1(D)$は
    \[\abs{a}+\abs{b}>0,\qquad\In D\]
    を満たすとする.
\end{problem}

\begin{lemma}[Mongeベクトル場による幾何学的翻訳]
    係数が定めるベクトル場$(x,y,z)\mapsto(a,b,c)$を\textbf{Mongeのベクトル場}\cite{吉田耕作-微分方程式}または特性方向場\cite{John-PDE}という.
    関数$u$のグラフ$\Brace{z=u(x,y)}$について,次は同値:
    \begin{enumerate}
        \item $(x,y,z)\mapsto u(x,y,z)$はPDEの解曲面である.
        \item 曲面$\Brace{z=u(x,y)}$の接空間が$\R^3$の中でMongeの方向を含む.
    \end{enumerate}
\end{lemma}
\begin{Proof}
    曲面が関数のグラフ$z=u(x,y)$で与えられているとき,その法ベクトルは$\vctrr{u_x}{u_y}{-1}$で与えられる.
    いま,方程式を3つの独立変数$x,y,z\in D$についての条件
    \[\vctrr{a}{b}{c}\cdot\vctrr{u_x}{u_y}{-1}=0\]
    とみると,特性方向がグラフの法線と直交することを表す.
    よって,グラフの接空間は特性方向を含む.
\end{Proof}
\begin{remarks}
    これによって,後者の条件を$\R^3$上のODE系で表すことを考え,これを特性方程式という.
    最後に底空間$\R^2$に射影すればよい.
\end{remarks}

\begin{theorem}[\cite{John-PDE} p.10]
    データ$a,b,c$は$C^1$-級とする.
    \begin{enumerate}
        \item 点$P=(x_0,y_0,z_0)$と積分曲面$z=u(x,y)$とについて,
        $P\in\Brace{z=u(x,y)}$ならば,積分曲面は$P$を通る特性曲線全体を含む.
        \item $S_1,S_2$を2つの異なる積分曲面とし,$S_1\cap S_2$が1点(接点)ではなく,曲線になるとする.
        すると,この曲線は特性曲線である.
    \end{enumerate}
\end{theorem}
\begin{Proof}\mbox{}
    \begin{enumerate}
        \item $\gamma$を$t=t_0$にて$P$を通る特性曲線$t\mapsto(x(t),y(t),z(t))$とする.
        \[U(t):=z(t)-u(x(t),y(t)).\]
        として,特性曲線上の$u$の値と$z$の値との乖離を調べよう.
        $U(t_0)=0$である.
        微分は
        \begin{align*}
            \dd{U}{t}&=\dd{z}{t}-u_x(x(t),y(t))\dd{x}{t}-u_y(x(t),y(t))\dd{y}{t}\\
            &=c(x,y,z)-u_x(x,y)a(x,y,z)-u_y(x,y,z)b(x,y,z).
        \end{align*}
        $z$に$U+u(x,y)$を代入してこれを$(x,y)$についての$\gamma$の射影上のODEと見ると,
        $U\equiv0$はこの解である:
        \[0=c(x,y,0+u)-u_xa(x,y,u)-u_yb(x,y,u).\]
        あとはODEの解の一意性から従う.
        \item $\gamma$の方向ベクトルがMongeのベクトル場であることを確認すればよいが,
        これは$\gamma$の接空間が2つの積分曲面の接空間の合併で得られることと,
        補題から従う.
    \end{enumerate}
\end{Proof}

\subsection{変数係数線型な場合の解法}

\begin{tcolorbox}[colframe=ForestGreen, colback=ForestGreen!10!white,breakable,colbacktitle=ForestGreen!40!white,coltitle=black,fonttitle=\bfseries\sffamily,
title=]
    第3ステップで,任意初期値に一斉に初期値を割り当てる任意関数$f$をとるのだが,
    その際に非特性的な初期曲線の選び方に自由度があって当惑しがち.
    なお,係数に$u$自身が登場しない限り,$u$は特性曲線上で定値である.
\end{tcolorbox}

\begin{example}
    \[u_x+yu_y=0\]
    は各点$(x,y)$において$u$の$(1,y)$-方向微分が零であることをいう:$\dd{y}{x}=y$.
    よって,各点$(x,y)$で傾き$(1,y)$を持つ曲線族$\{y=Ce^{x}\}_{C\in\R}$が特性曲線族である.
    この上で$u$が定値になっているのは,計算
    \[\dd{}{x}u(x,Ce^x)=\pp{u}{x}+Ce^x\pp{u}{y}=y_x+yu_y=0.\]
    からわかる.さらに,$u(x,Ce^x)=u(0,C)$として$x=0$の値で統制することができ(非特性的境界),そのデータを$f:\R\to\R$と与えれば,一般解
    \[u(x,y)=f(e^{-x}y).\]
    を得る.
\end{example}
\begin{Proof}
    特性方向は$F_p=\vctr{1}{y}$であり,特性方程式は
    \[\begin{cases}
        \dot{x}=1\\
        \dot{y}=y\\
        \dot{z}=0
    \end{cases}\quad\Leftrightarrow\quad\begin{cases}
        x=s+x_0\\
        y=y_0e^s\\
        z=z_0
    \end{cases}.\]
    いま,$x=0$上で特性方向は境界と交差しており,これは非特性的境界をなしているから,
    初期条件を$\x(0)=\vctr{x_0}{y_0}=\vctr{0}{y_0}$と与えることとすると,このときの$u$の値は
    \[z_0=u(0,y_0)=u(0,ye^{-x}).\]
    今回は$z(s)$は定値であるから,
    \[u(x,y)=z(s)=z_0=u(0,ye^{-x})\]
    の最右辺に初期値$f(ye^{-x})$を分配すればよい.
\end{Proof}

\begin{example}
    \[u_x+2xy^2u_y=0\]
    \begin{enumerate}
        \item 射影特性方程式は
        \[\dd{y}{x}=\frac{2xy^2}{1}\]
        で,その解は$y=(C-x^2)^{-1}$.
        \item $u$はその上で定値である.
        \item $u(x,y)=f(C)$という構造があるから,(1)の解を$C$について解くと,
        \[u(x,y)=f\paren{x^2+\frac{1}{y}}.\]
    \end{enumerate}
\end{example}

\subsection{初期条件について}

\begin{tcolorbox}[colframe=ForestGreen, colback=ForestGreen!10!white,breakable,colbacktitle=ForestGreen!40!white,coltitle=black,fonttitle=\bfseries\sffamily,
    title=]
    初期条件も$M\times\R$上の曲線とみなせ,こちらは「初期曲線」と呼ぼう.
    一般の初期曲線は座標$z$について定値ではなく,すると特性曲線上で$u$は定値でないから,前節のような直接的な考察では足りない.
\end{tcolorbox}

\begin{problem}
    初期曲線$A:t\mapsto(x(t),y(t),u(t))\in\R^3$に対して,この近傍で定義されたなめらかな曲面$u(x,y)$を求めることを考える.
    次の2点を仮定する:
    \begin{enumerate}[{[A}1{]}]
        \item 滑らかなデータ:$x,y,u\in C^1(\R)$.
        \item 正則性:$\dot{x}(t)^2+\dot{y}(t)^2>0$.
        \item 一貫性:初期曲線の射影$A'$は自己交差を持たない.
    \end{enumerate}
\end{problem}
\begin{remarks}
    通常のC-型の境界値問題はいずれも満たすだろう.
\end{remarks}

\begin{theorem}
    $A:t\mapsto(x(t),y(t),u(t))\in\R^3$を初期条件とする.
    \begin{enumerate}
        \item $A'$は如何なる射影特性曲線$K'$とも一致せず,接しもしないとする.このとき,$A$を初期曲線とするPDEの解がただ一つ存在する.
        \item $A$がひとつの特性曲線$K$に一致する場合,$A=K$を通るPDEの解は無数に存在する.このとき,特性曲線$K$を\textbf{分岐曲線}(branch line)という.
        \item $A$は特性曲線でないが,射影$A'$が射影特性曲線になっているとする.このとき,$A$を初期曲線とするPDEの$C^1$-解は存在しない.
    \end{enumerate}
\end{theorem}
\begin{Proof}\mbox{}
    \begin{enumerate}
        \item 
        \begin{description}
            \item[存在と可微分性] このとき,係数の滑らかさから,$A$の任意の点に対して,それを通る特性曲線がただ一つ存在し$s$について$C^1$-級である.
            それらの合併$(s,t)\mapsto (x(s,t),y(s,t),u(s,t))$を曲面片$\Sigma$とすると,初期曲線$A$のパラメータ$t$についても$C^1$-級であることが逐次近似法から従う\cite{吉田耕作-微分方程式}.
            \item[$\Sigma$が解である]
            これはまず仮定によって,
            \[\vctr{\dot{x}(t)}{\dot{y}(t)}\not\varparallel\vctr{a(x(t),y(t),u(t))}{b(x(t),y(t),u(t))}.\]
            であるから,$x,y,u$が$(s,t)$について$C^1$-級であることと併せると,
            さらに$A$の近傍で$x_t:y_t\neq a:b=x_s:y_s$.
            ゆえに,$s,t$は独立で(接ベクトルが一次独立)
            連立方程式
            \[\begin{cases}
                x=x(s,t),\\
                y=y(s,t).
            \end{cases}\]
            は$(s,t)$について解けて,$C^1$-級関数$s=s(x,y),t=(x,y)$を得られ,これが定める
            \[u(s,t):=u(s(x,y),t(x,y))=U(x,y)\]
            も$C^1$-級である.
            さらに$s,t$を動かしたとき$(s,t)\mapsto (x(s,t),y(s,t),u(s,t))$はたしかに曲面を$\R^3$内に定義していて,その接ベクトルはMongeの方向に接するから,PDEの解である.
            \item[一意性] 解ならばその等高線が特性曲線を定めている必要があるが,特性曲線は一意である.
        \end{description}
        \item $K$と交わる曲線$A_1$を任意にとると,(1)よりこれを初期曲線とする解が存在する.$A_1$は無限にとれる.
        \item すなわち,$A$のあるパラメータ$\tau\mapsto(x(\tau),y(\tau),z(\tau))$が
        \[\vctr{\dd{x}{\tau}(\tau)}{\dd{y}{\tau}(\tau)}=\vctr{a(x(\tau),y(\tau),z(\tau))}{bx(\tau),y(\tau),z(\tau)}\]
        を満たす.このとき,
        仮に$u$が$A$を初期曲線とする$C^1$-級解とすると,
        $A$の射影の上で
        \[\dd{u}{\tau}=u_x\dd{x}{\tau}+u_y\dd{y}{\tau}=u_xa+u_yb=c\]
        を満たすから,結局$A$は特性曲線である必要が出てきてしまう.
    \end{enumerate}
\end{Proof}
\begin{remarks}
    (1)を満たすための条件が,境界の非特性性である\ref{remark-noncharacteristic-condition}.
    (2),(3)が境界が特性的である場合に当たる.
    (2)は境界条件が解$u$に,ある1特性曲線上の値を付与したのみで,もう1つの独立方向に自由度が1つ残ってしまう状況を表す.
    なお,2つの積分曲面がある1点を共有するならば,それを通る特性曲線の全体を共有する必要があることに注意.
    (3)は特性曲線上に「正しくない」$u$の値を付与してしまう「間違った境界条件」の場合である.
\end{remarks}

\subsection{非斉次Burgers方程式}

\begin{example}[非斉次Burgers方程式]
    \[uu_x+u_y=1\]
    の特性曲線は
    \[\begin{cases}
        \frac{dx}{z}=ds\\
        dy=ds\\
        dz=ds
    \end{cases}\]
    この一般解は
    \[\begin{cases}
        x=x_0+z_0s+\frac{s^2}{2}\\
        y=y_0+s\\
        z=z_0+s
    \end{cases},\qquad(x_0,y_0,z_0)\in\R^3.\]
    が与える.
    \begin{enumerate}
        \item $s=0$における初期値$(x_0,y_0,z_0)$を,
        \[x_0=\varphi(t),\quad y_0=\psi(t),\quad z_0=\chi(t).\]
        と与えるとする.このとき,$(s,t)\mapsto (x(s,t),y(s,t))$が$s=0$にて2つの一次独立な接ベクトルを持てば,この
        境界条件は非特性的であることがわかる.
        $(x_s,y_s)$と$(x_t,y_t)$との比の値は
        \[\vctr{x_s}{y_s}\times\vctr{x_t}{y_t}=x_sy_t-x_ty_s=s(y^0_t-z_t^0)+z^0y_t^0-x_t^0.\]
        よって,$\chi\psi_t-\varphi_t\ne0$ならば,$s,t$を$x,y$の関数として解き$u$に代入してPDEの解を得る.
        \item $s=0$における初期値を曲線
        \[\begin{cases}
            x_0=\frac{1}{2}t^2,\\
            y_0=t,\\
            z_0=t.
        \end{cases}\]
        によって与える場合に当たる.
        この初期条件はたしかに
        \[\vctrr{u_x}{u_y}{-1}\cdot\vctrr{t}{1}{1}=u_xt+u_y-1=0.\]
        という条件を満たしている.すなわち,この初期曲線の接ベクトルは,解曲面の接空間内におさまっている.
        このとき,元の式に境界条件を代入しても,
        \[\begin{cases}
            x=\frac{(s+t)^2}{2},\\
            y=s+t,\\
            z=s+t.
        \end{cases}\]
        というように,2径数$s,t\in\R$を持つにも拘わらず,ただの曲線を表している.
        そこで,この曲線を横断する方向の成分を任意に当たることで,例えば曲面
        \[\Brace{(x,y,z)\in\R^3\;\middle|\; x=\frac{1}{2}z^2+\varphi(z-y)}.\]
        は任意の$\varphi\in C^1,\varphi(0)=0$は解曲面を与える.実際,上の式を微分すると,
        \[u_x=\frac{1}{u+\varphi'(u-y)},\quad u_y=\frac{\varphi'(u-y)}{u+\varphi'(u-y)}.\]
        より,元のPDE $uu_x+u_y=1$を満たす.
        \item $s=0$における初期値を曲線
        \[\begin{cases}
            x_0=t^2,\\
            y_0=2t,\\
            z_0=t.
        \end{cases}\]
        によって与える場合に当たる.
        射影曲線$t\mapsto(t^2,2t)$上で元のPDEを満たす古典解$u$の値は$2t$である.
        このときの2径数曲面は
        \[\begin{cases}
            x=\frac{s^2}{2}+st+t^2\\
            y=s+2t\\
            z=s+t
        \end{cases}\]
        であり,$s,t$を消去すると
        \[u(x,y)=\frac{y}{2}\pm\sqrt{x-\frac{y^2}{4}}\]
        を得て,初期曲線$A$を通る=$A$が与える初期条件を満たす解ではあるが,$A$上では有界な導関数を持たない.
    \end{enumerate}
\end{example}

\subsection{準線形でもない場合}

\begin{problem}
    \[F(x,y,z,p,q)=0.\]
    をある$D\osub\R^5$上で考える.
    $F\in C^2(D)$かつ$\abs{F_p}+\abs{F_q}>0\;\In D$を仮定する.
\end{problem}
\begin{observation}
    この仮定により,$F_p,F_q$のいずれかは$D$上で零でないから,陰関数を解いて局所的に正規形
    \[p=p(x,y,z,q)\]
    を得る.これを$(x,y,z)\in\R^3$上の解曲面の接ベクトルに関する条件とみると,接ベクトルの1-径数族$\{(p(\lambda),q(\lambda))\}_{\lambda\in\R}\}$を得る.
    例えば$p(\lambda)=p(x,y,z,\lambda),q(\lambda)=\lambda$の場合を考えると,
    点$(x,y,z)$を通り,$(p(\lambda),\lambda,-1)$を法線とする接平面の$\lambda$-径数族は,$(x,y,z)$を頂点とする1つの錐面を作る.
    これを\textbf{Monge錐}という.
\end{observation}
\begin{remarks}
    準線形の場合は,方程式自体が,$(p(\lambda),q(\lambda),-1)$が係数方向$(a,b,c)$に直交することを要請しているから,$\lambda\in\R$にもはや自由度はなく,Monge錐は1つの接平面に退化する.
\end{remarks}

\begin{lemma}
    グラフ$\Brace{z=u(x,y)}$について,次は同値:
    \begin{enumerate}
        \item $\Brace{z=u(x,y)}$は解曲面である.
        \item 各点でMonge錐に接している.
    \end{enumerate}
\end{lemma}


\begin{lemma}\mbox{}
    \begin{enumerate}
        \item Monge錐の母線の方程式は
        \[\begin{cases}
            (X-x):(Y-y):(U-u)=F_p:F_q:(pF_p+qF_q)\\
            F(x,y,u,p,q)=0.
        \end{cases}\]
        である.
        \item Monge錐に各点で接する曲線は,次の解に同値:
        \[\begin{cases}
            \dd{x}{F_p}=\dd{y}{F_q}=\dd{u}{pF_p+qF_q}=ds.\\
            F(x,y,u,p,q)=0.
        \end{cases}\]
    \end{enumerate}
    なお,条件$\abs{F_p}+\abs{F_q}>0\;\In D$は,各点で$s$を$p$または$q$について解き,
    \[\begin{cases}
        \dd{y}{x}=\frac{F_q}{F_p},\quad\dd{u}{x}=p+q\frac{F_q}{F_p}\\
        F(x,y,u,p,q)=0
    \end{cases}\]
    と解けるための条件に当たる.
\end{lemma}

\begin{theorem}\mbox{}
    \begin{enumerate}
        \item 特性曲線について,
        \[\dd{z}{s}=p(s)\dd{x}{s}+q(s)\dd{y}{s}.\]
        \item 特性曲線上で$F$は定数である:
        \[F(\p(s),z(s),\x(s))=\const.\]
    \end{enumerate}
\end{theorem}
\begin{Proof}
    (1)は\ref{thm-structure-of-characteristic-curve}(2)で示した.
\end{Proof}

\begin{definition}[strip / Streifen, strip condition]
    特性曲面$t\mapsto(\p(t),z(t),\x(t))$が
    \[\dd{z}{t}=\p(t)\cdot\dot{\x}(t)\]
    を満たすとき,これを\textbf{帯}といい,この条件を\textbf{成帯条件}という.
\end{definition}
\begin{remarks}
    上の定理は,(1)が特性曲線が帯を作ることを言っている.さらに(2)より,
    初期条件$F(\p(s_0),z(s_0),\x(s_0))$を満足するならば,
    その帯全体で$F=0$であることがわかる.
    この帯を\textbf{特性帯}といい,$(x,y,z)$を\textbf{(射影)特性曲線}という.
\end{remarks}
\begin{remark}
    \cite{Evans}で特性曲線と呼んでいるものを,伝統的には特性帯という.
    すなわち,伝統的な用語法では,$\x,z$空間の曲線を特性曲線といい,$\x,z,\p$空間の曲線を特性帯という.
\end{remark}

\subsection{特性帯と初期値問題}

\begin{problem}
    前節の問題と,次の2条件を満たす帯
    \begin{enumerate}[{[A}1{]}]
        \item 正則性:$\norm{\dot{\x}(t)}>0$.
        \item $F(\p(t),z(t),\x(t))=0$.
    \end{enumerate}
    を考える.これを\textbf{初期帯}といい,初期帯とある初期曲線$A:t\mapsto(\x(t),z(t))$に沿って接しているような$C^2$-解を考える.
    $A$の射影は自己交差を持たないとする.
\end{problem}

\begin{theorem}\mbox{}\label{thm-initial-value-problem-of-fully-nonlinear-1st-order}
    \begin{enumerate}
        \item 射影$A'$はいかなる射影特性曲線とも一致せず,接しもしないとする.このとき,$A$を通って$A$に沿って初期帯と接するような$C^2$-級解がただ一つ存在する.
        \item 初期帯が特性帯になっているとする.このとき,$A$を通りこの初期帯に接する解が無限に存在する.
        \item 初期帯は特性帯ではないが,$A$の射影が射影特性曲線と一致しているとする:
        \[D_pF(\x)\cdot\nu(\x)=0.\]
        このとき,初期帯を持つ$C^2$-解は存在しない.
    \end{enumerate}
\end{theorem}

\section{境界条件の一般論}

\begin{tcolorbox}[colframe=ForestGreen, colback=ForestGreen!10!white,breakable,colbacktitle=ForestGreen!40!white,coltitle=black,fonttitle=\bfseries\sffamily,
title=境界条件が特性曲線を与えないための条件を非特性性という]
    特性方程式の境界条件$(p^0,z^0,x^0)\in\R^{2n+1}$についての一般論を展開する.
    \begin{enumerate}
        \item まず,境界条件は局所的には$\Brace{x_n=0}$の形をしているとみなして一般性を失わない.
        \item 次に,境界条件を満たす特性曲線が存在するためには,自然な必要条件が存在し,その$2n+1$元連立方程式系を整合性条件という.
        \item 最後に,この整合性条件が局所的に隣の点でも満たされ続けるように摂動出来るため
        の十分条件は
        \[D_pF(p^0,z^0,x^0)\cdot\b{\nu}(x^0)\ne0\]
        が与え,このような境界条件を\textbf{非特性的}であるという.
    \end{enumerate}
    以上を満たす特性方程式の境界値問題については,
    その解が,元もPDEの局所解を与えていることが保証される.
\end{tcolorbox}

\begin{problem}
    再び問題\ref{eq-5-2}を考える:
    $C^1$-級関数$F,g:\Gamma\to\R\;(\Gamma\subset\partial U)$についての境界値問題
    \[
        \begin{cases}
            F(Du,u,x)=0&\In U,\\
            u=g&\on\Gamma.
        \end{cases}
    \]
    を考える.
\end{problem}
\begin{definition}[一般の境界に対する非特性性条件]\label{remark-noncharacteristic-condition}
    $x^0\in\Gamma$の近傍では境界$\Gamma$は直線とみなせないとする.
    このとき,$\Gamma$が\textbf{非特性的}であるとは,
    \[D_pF(p^0,z^0,x^0)\cdot\b{\nu}(x^0)\ne0\]
    を満たすことをいう.$F_{p_n}$は変換後の$y_n$軸方向の高さの変化分を測っているから,座標変換が出来ないのならば,外側単位法線ベクトル$\b{\nu}$を用いる.
\end{definition}
\begin{remarks}
    特性曲線,すなわち,
    係数のベクトル場$x\mapsto\b{b}(x)$が境界$\Gamma$に$x^0$で接するわけではないための条件である.
    これが満たされない場合,古典解は存在しない\ref{thm-initial-value-problem-of-fully-nonlinear-1st-order}.
\end{remarks}

\subsection{境界の直線化}

\begin{tcolorbox}[colframe=ForestGreen, colback=ForestGreen!10!white,breakable,colbacktitle=ForestGreen!40!white,coltitle=black,fonttitle=\bfseries\sffamily,
title=]
    変数変換を行ったあとの問題に注目しても,特に各$x^0\in\Gamma$の近傍で境界は$\Brace{x_n=0}$と表せていると仮定しても一般性を失わない.
\end{tcolorbox}

\begin{definition}[境界の滑らかさ]
    境界$\partial U$が$C^k$-級であるとは,任意の点$x^0\in\partial U$に対して$r>0$と$C^k$-写像$\gamma:\R^{n-1}\to\R$が存在して,向きと座標の順番の違いを除いて
    \[U\cap B(x^0,r)=\Brace{x\in B(x^0,r)\mid x_n>\gamma(x_1,\cdots,x_{n-1})}\]
    と表せることをいう.
\end{definition}

\begin{lemma}[境界上での変数変換]
    $\partial U$が$C^k$-級のとき,
    任意の注目点$x^0\in\partial U$について,
    互いに逆な可微分写像$\Phi,\Psi:\R^n\to\R^n$であって,
    $x^0$の近傍で$\partial U$を直線に変換するものが取れる.
\end{lemma}
\begin{Proof}
    変数変換$y=\Phi(x)$を次のように定める:
    \[\begin{cases}
        y_i=x_i=:\Phi^i(x)\\
        y_n=x_n-\gamma(x_1,\cdots,x_{n-1})=:\Phi^n(x).
    \end{cases},\qquad(i=1,2,\cdots,n-1).\]
    逆向きの変数変換$x=\Psi(y)$は次のように定める:
    \[\begin{cases}
        x_i=y_i=:\Psi^i(y)\\
        x_n=y_n+\gamma(y_1,\cdots,y_{n-1})=:\Psi^n(y).
    \end{cases}\qquad(i=1,2,\cdots,n-1).\]
    すると,次が確認できる:
    \begin{enumerate}
        \item $\Phi=\Psi^{-1}$.
        \item $\det(D\Phi)=\det(D\Psi)=1$.
    \end{enumerate}
\end{Proof}

\begin{proposition}[変数変換は問題の構造を変えない]
    問題\ref{eq-5-2}は,$V:=\Phi(U),\Delta:=\Phi(\Gamma)$と$h(y):=g(\Psi(y))$とある$G$についての,
    $v(y):=u(\Psi(y))$についての問題
    \[\begin{cases}
        G(Dv,v,y)=0&\In V,\\
        v=h&\on\Delta.
    \end{cases}\]
    に同値.
\end{proposition}
\begin{Proof}
    このとき,$u(x)=v(\Phi(x))$であるから,この微分は
    \[u_{x_i}(x)=\sum_{k\in[n]}v_{y_k}(\Phi(x))\Phi^k_{x_i}(x),\qquad(i\in[n])\]
    であるから,
    \[Du(x)=Dv(y)D\Phi(x).\]
    これを\ref{eq-5-2}の第一式に代入して
    \[F(Dv(y)D\Phi(\Psi(y)),v(y),\Psi(y))=F(Du(x),u(x),x)=0\]
    より,$F$に$D\Phi(\Psi(y))$と$\Psi$の部分を吸収させたものを$G$と定めると,これは$G(Dv(y),v(y),y)=0\;\In V$に同値である.
\end{Proof}

\subsection{境界データの整合性条件}

\begin{tcolorbox}[colframe=ForestGreen, colback=ForestGreen!10!white,breakable,colbacktitle=ForestGreen!40!white,coltitle=black,fonttitle=\bfseries\sffamily,
title=]
    特性曲線が境界データを満たすためには,
    許容性条件という当然の必要条件がある.
\end{tcolorbox}

\begin{discussion}
    特性曲線の境界条件
    \[\b{p}(0)=p^0,\quad z(0)=z^0,\quad \b{x}(0)=x^0.\]
    について,
    \begin{enumerate}
        \item $z^0=g(x^0)$が必要.
        \item 各$x^0\in\Gamma$の近傍で,$p^0_i=g_{x_i}(x^0)\;(i=1,\cdots,n-1)$が必要.
    \end{enumerate}
\end{discussion}
\begin{Proof}\mbox{}
    \begin{enumerate}
        \item 特性曲線が$F(\b{p}(s),z(s),\b{x}(s))=0$と境界条件を満たすならば,$z^0=u(x^0)=g(x^0)$が必要.
        \item 仮定より,$x^0$の近傍では$u(x_1,\cdots,x_{n-1},0)=g(x_1,\cdots,x_{n-1})$が成り立っているから,両辺を微分して,
        \[u_{x_i}(x^0)=g_{x_i}(x^0)\qquad(i=1,2,\cdots,n-1).\]
        これが成り立つ範囲は$U$の一部も含んでいるので,$F(p^0,z^0,x^0)=0$も必要.
    \end{enumerate}
\end{Proof}

\begin{definition}[admissible, compatibility condition]
    境界値
    $(p^0,z^0,x^0)\in\R^{2n+1}$が\textbf{許容的}であるとは,次の方程式を満たすことをいう:
    \[\begin{cases}
        z^0=g(x^0),\\
        p^0_i=g_{x_i}(x^0),\quad i=1,2,\cdots,n-1\\
        F(p^0,z^0,x^0)=0.
    \end{cases}\]
    この連立方程式系を\textbf{整合性条件}という.
\end{definition}
\begin{remark}
    $z^0$は$x^0$から直ちに定まるが,整合性条件を満たす勾配ベクトル$p^0$は存在するとも限らず,一意とも限らない.
\end{remark}

\subsection{非特性的な境界データ}

\begin{tcolorbox}[colframe=ForestGreen, colback=ForestGreen!10!white,breakable,colbacktitle=ForestGreen!40!white,coltitle=black,fonttitle=\bfseries\sffamily,
title=]
    許容的な境界データ$(p^0,z^0,x^0)$に対して,
    $x^0$を通る特性曲線$\b{x}(s)$は存在する.
    これが,近傍の点$y\in B(x^0,r)$に対しても,ある境界データの摂動
    $(\b{q}(y),g(y),y)$が存在して許容的であるためには,非特性的である必要がある.
\end{tcolorbox}

\begin{observation}[問題の所在]
    $x^0\in\Gamma$の近傍$B(x^0,r)\cap\Gamma=B(x^0,r)\cap\Brace{x_n=0}\ni y$における境界条件
    \[\b{p}(0)=\b{q}(y),\quad z(0)=g(y),\quad \b{x}(0)=y.\]
    を考える.ただし,$\b{q}:B(x^0,r)\cap\Gamma\to\R^n$は$\b{q}(x^0)=p^0$を満たすような,境界条件の摂動とした.
    このとき,$(\b{q}(y),g(y),y)$が任意の$y\in B(x^0,r)\cap\Gamma$で許容的であるための条件を考える必要がある.
    これが満たされれば,特性曲線の合併として解が構成されるわけである.
\end{observation}

\begin{lemma}[摂動に対して境界データが許容的であり続けるための十分条件]
    境界値$(p^0,z^0,x^0)$は許容的で,さらに
    \[F_{p_n}(p^0,z^0,x^0)\ne0\]
    を満たすとする.
    このとき,ある関数$\b{q}$が存在して,$x^0$の近傍$B(x^0,r)$で整合性条件
    \[\begin{cases}
        \b{q}(x^0)=p^0,\\
        q^i(y)=g_{x_i}(y),\qquad i=1,2,\cdots,n-1\\
        F(\b{q}(y),g(y),y)=0.
    \end{cases}\]
    が満たされ続ける.
\end{lemma}
\begin{Proof}
    まず,$q^i(y):=g_{x_i}(y)\;(i=1,2,\cdots,n-1)$と定める他ないので,真の問題は$q^n(y)$が$q^n(x^0)=(p^0)^n$かつ
    \[F(\b{q}(y),q(y),y)=0\]
    を満たすように構成できるか,ということになる.
    これは,$F_{p_n}(p^0,z^0,x^0)\ne0$を仮定すれば,
    陰関数定理\ref{thm-Implicit-Function-Theorem}から従う.
    すなわち,パラメータ$p^n$の決め方は,ある陽関数を用いて$p^1,\cdots,p^{n-1}$から定めることが出来る.
\end{Proof}

\begin{definition}[noncharacteristic condition]
    補題の要件を\textbf{非特性性条件}という.
\end{definition}

\section{特性曲線法による局所解の存在}

\begin{tcolorbox}[colframe=ForestGreen, colback=ForestGreen!10!white,breakable,colbacktitle=ForestGreen!40!white,coltitle=black,fonttitle=\bfseries\sffamily,
title=特性的な境界の近傍では局所解が存在する]
    特性方程式の境界条件$(p^0,z^0,x^0)$が非特性的であるとき,
    $p^0$をうまく近傍で摂動させることで整合性条件を満たし続けるようにできるから,
    その近傍の境界条件$(\b{q}(y),g(y),y)$についての解との合併を考えれば,
    元のPDEの局所解になる.
    よって,このことを考慮に入れれば,興味のある境界条件$(p^0,z^0,x^0)$について特性性を確認すれば,
    これが与える特性方程式の解をPDEの解だと主張して問題ないことが解る.
\end{tcolorbox}

\begin{problem}\label{problem-5-4}
    問題\ref{eq-5-2}は
    \begin{enumerate}[{[A}1{]}]
        \item ある点$x^0\in\Gamma$の近傍で$B(x^0,r)\cap\Gamma=B(x^0,r)\cap\Brace{x_n=0}$が成り立ち,
        \item $(p^0,z^0,x^0)$は許容的な境界値で,非特性的であるとする.
    \end{enumerate}
    このとき,関数$\b{q}:B(x^0,r)\cap\Gamma\to\R^n$を補題のものとして,
    $(\b{q}(y),g(y),y)\;(y\in\Gamma\cap B(x^0,r))$が許容的である範囲で,
    特性方程式の初期値問題
    \[\begin{cases}
        \dot{\b{p}}(s)=-D_xF(\b{p}(s),z(s),\b{x}(s))-D_zF(\b{p}(s),z(s),\b{x}(s))\b{p}(s)\\
        \dot{z}(s)=D_pF(\b{p}(s),z(s),\x(s))\cdot\b{p}(s),\\
        \dot{\x}(s)=D_pF(\b{p}(s),z(s),\x(s)).
    \end{cases}\]
    \[\b{p}(0)=\b{q}(y),\quad z(0)=g(y),\quad\b{x}(0)=y.\]
    を考える.
    この解は各点$y\in\Gamma\cap B(x^0,r)$に依存するから,$\b{p}(s)=\b{p}(y,s)$というように書くこととする.
\end{problem}

\begin{lemma}[局所可逆性]
    問題\ref{problem-5-4}の仮定の下で,$0$の開近傍$I\osub\R$と$x^0$の開近傍$W\osub\Gamma\subset\R^{n-1}$と$V\osub\R^{n}$とが存在し,
    \begin{enumerate}
        \item $W$から生える特性曲線が$V$を埋め尽くす:任意の$x\in V$に対して,ある時刻$s\in I$と初期値$y\in W$が一意的に存在して,$x=\b{x}(y,s)$と表せる.
        \item この初期値と時刻への対応$V\ni x\mapsto (s,y)\in I\times W$は$C^2$-級である.
    \end{enumerate}
\end{lemma}
\begin{Proof}\mbox{}
    \begin{description}
        \item[問題の所在] $(p^0,z^0,x^0)$の非特性性が$D\b{x}(x^0,0)\ne0$を含意するならば,陰関数定理\ref{thm-Implicit-Function-Theorem}より,$(x^0,0)\in\R^n$のある近傍$V\osub\R^{n}$において,
        解を定める陰関数的条件$F(\b{p}(y,s),z(y,s),\b{x}(y,s))=0$が陽に解け,これが対応$x\mapsto (s,y)\in I\times W$を与える.$F$が$C^2$-級なのでこの対応も$C^2$-級である.
        \item[陰関数定理の要件の証明] いま任意の$y\in\Gamma$に対して$\b{x}(y,0)=(y,0)$であるから,
        \[x^j_{y_i}(z^0,0)=\begin{cases}
            \delta_{ij}&j=1,2,\cdots,n-1,\\
            0&j=n.
        \end{cases}\]
        さらに射影特性曲線の方程式より,
        \[x_{s}^j(x^0,0)=F_{p_j}(p^0,z^0,x^0).\]
        以上より,
        \[D\x(x^0,0)=\begin{pmatrix}
            1&\cdots&0&F_{p_1}(p^0,z^0,x^0)\\
            \vdots&\ddots&\vdots&\vdots\\
            0&\cdots&1&\vdots\\
            0&\cdots&0&F_{p_n}(p^0,z^0,x^0)
        \end{pmatrix}.\]
        これより,$(x^0,z^0,p^0)$が非特性的であることと,$D\x(x^0,0)\ne0$とが同値.
    \end{description}
\end{Proof}

\begin{theorem}[Local Existence Theorem \cite{Evans} 3.2.4 Th'm.2]\label{thm-Local-Existence-Theorem}
    補題の対応を用いて,$u(x):=z(\b{y}(x),s(x))$と定める.
    \begin{enumerate}
        \item 古典解である:$u\in C^2(V)$.
        \item 局所解である:$F(Du(x),u(x),x)=0\;\In V$.
        \item 境界条件を満たす:$u(x)=g(x)\;\on\Gamma\cap V$.
    \end{enumerate}
\end{theorem}
\begin{Proof}
    (1),(3)は明らかであるから,(2)を示せば良い.
    \begin{enumerate}[{Step}1]
        \item 問題の仮定の下,特性方程式系を解いて,解$\b{p}(y,s),z(y,s),\b{x}(y,s)$を得ることが出来る.
        \item $y\in\Gamma$が十分$x^0$に近いならば,
        \[f(y,s):=F(\b{p}(y,s),z(y,s),\b{x}(y,s))=0\qquad(s\in I)\]
        が成り立つ.

        実際,まず時刻$s=0$のとき,
        \[f(y,0)=F(\b{p}(y,0),z(y,0),\b{x}(y,0))=F(\b{q}(y),g(y),y)=0.\]
        が境界点$(\b{q}(y),g(y),y)$の整合性条件から成り立つ.
        さらに,$I$上$f_s(y,s)=0$でもある:
        \begin{align*}
            f_s(y,s)&=\sum_{j=1}^nF_{p_j}\dot{p}^j+F_z\dot{z}+\sum_{j=1}^nF_{x_j}\dot{x}^j\\
            &=\sum_{j=1}^nF_{p_j}(-F_{x_j}-F_zp^j)+F_z\paren{\sum_{j=1}^nF_{p_j}p^j}+\sum_{j=1}^nF_{x_j}(Fp_j)\\
            &=0.
        \end{align*}
        ただし途中の変形は特性方程式による.
        \item 以上より,$F(\b{p}(x),u(x),x)=0\;\In V$が成り立つ.続いて,$\b{p}(x)=Du(x)\;\In V$でもある.
        
        これは,任意の$s\in I,y\in W$について,
        \[z_s(y,s)=\sum_{j=1}^np^j(y,s)x_s^j(y,s),\quad z_{y_i}(y.s)=\sum_{j=1}^np^j(y,s)x^j_{y_i}(y,s)\qquad(i=1,\cdots,n-1)\]
        であることから解る.
    \end{enumerate}
\end{Proof}

\begin{example}
    方程式は線型斉次であるとする.
    \begin{enumerate}
        \item 境界が非特性的で,その近傍で解が存在しようとも,例えばすべての特性曲線が1点で交わるとき,境界条件$g$が定値でない限り大域解は存在しない.
        \item 特性曲線が$U$を交わらずに覆いつくしている場合,境界のうち特性的な部分$\Gamma$から始まる特性曲線を考慮することで大域解を構成できる.
    \end{enumerate}
\end{example}

\section{特性曲線法の応用と限界}

\subsection{線型な場合}

\begin{problem}\label{prob-1st-order-linear-PDE-revisited}
    線型斉次PDE
    \[F(Du,u,x)=\b{b}(x)\cdot Du(x)+c(x)u(x)=0\qquad(x\in U).\]
    を考える\ref{prob-1st-order-linear-PDE}.$F(p,z,x)=\b{b}(x)\cdot p+c(x)z$で,$D_pF=\b{b}(x)$である.
    \begin{enumerate}
        \item 非特性性の条件\ref{remark-noncharacteristic-condition}は$\b{b}(x^0)\cdot\bnu(x^0)\ne0$.特に,$u,Du$の情報$z^0,p^0$には依らない.
        \item $\Gamma$が変な形をしていないために,$x^0\in\Gamma$で(1)の条件が満たされるとする.
        すると任意の境界条件$u=g\;\on\Gamma$について,
        $x^0\in\Gamma$について局所的に,$\b{q}$が存在して$(\b{q}(y),g(y),y)$は$y\in\Gamma\cap B(x^0,r)$上整合性条件を満たし続ける.
        したがって
        局所存在定理\ref{thm-Local-Existence-Theorem}により,任意の$p^0\in\R^n$に対して,$(p^0,z^0,x^0)$-初期値問題の局所解の存在が解る.
        \item $p^0$に依らずに境界条件は特性性を満たすから,
        結局この場合,特性方程式は$\b{p}$とは関係ないのである.
        よって,$x^0$を通る特性方程式の解がPDEの解にもなることが確約されたいま,
        例\ref{prob-1st-order-linear-PDE}で得た,簡略化された$\b{p}$を含まない$n+1$本の特性方程式から得られた任意の曲線は,解をなす.
        \item 特性方程式系のODEとしての解の一意性定理より,特性曲線は互いに交差しない.
    \end{enumerate}
\end{problem}

\subsection{準線型な場合:保存則の方程式}

\begin{tcolorbox}[colframe=ForestGreen, colback=ForestGreen!10!white,breakable,colbacktitle=ForestGreen!40!white,coltitle=black,fonttitle=\bfseries\sffamily,
title=]
    一番の違いは,簡略化された特性方程式\ref{prob-1st-order-quasi-linear-PDE}の解が局所解は与えるが,
    大域的にそれらの曲線は交わり得る点である.
    このようなとき,$U$の内部では解が存在しない点がある(そもそも大域的な古典解が存在しない)ことが多い.
    が,保存則の形のPDEならば,適切な弱解概念で大域的な超関数解の存在と一意性が保証出来る.
\end{tcolorbox}

\begin{problem}
    準線型1階PDEは一般に
    \[F(Du,u,x)=\b{b}(x,u(x))\cdot Du(x)+c(x,u(x))=0.\]
    と表せる\ref{prob-1st-order-quasi-linear-PDE}.$F(p,z,x)=\b{b}(x,z)\cdot\b{p}+c(x,z)$,$F_p(p,z,x)=\b{b}(x,z)$である.
    \begin{enumerate}
        \item 非特性性の条件\ref{remark-noncharacteristic-condition}は$\b{b}(x^0,z^0)\cdot\bnu(x^0)\ne0$であり,特に$p^0$には依らない.
        \item 方程式が非特性的であるとき,任意の境界条件$u=g\;\on\Gamma$に対して,$(p^0,g(x^0),x^0)$が与えられたら,$\b{q}$であって$(\b{q}(y),g(y),y)$が$y\in B(x^0,r)\cap\Gamma$で整合的であり続けるような選び方$\b{q}$が一意に存在する.
        \item よって,定理\ref{thm-Local-Existence-Theorem}より,任意の$x^0,p^0$に対して,$(p^0,g(x^0),x^0)$を通る特性方程式の解は元のPDEの$x^0$での局所解を与えている.
        \item この局所解自体は,$\b{p}$のことを忘れて,簡略化された$n+1$本の特性方程式から計算出来る\ref{prob-1st-order-quasi-linear-PDE}.
    \end{enumerate}
\end{problem}

\begin{example}[保存則に由来するPDEに対する特性曲線法]
    $\b{F}=(F^1,\cdots,F^n):\R\to\R^n$に対する保存則
    \[\begin{cases}
        u_t+\div\b{F}(u)=0&\In\R^n\times\R^+=:U,\\
        u=g&\on\R^n\times\{0\}=:\Gamma.
    \end{cases}\]
    を考える.$\div\b{F}(u)=\b{F}'(u)\cdot Du$と$t:=x_{n+1}$に注意.
    このとき,$q=(p,p_{n+1}),y=(x,t)$を変数として,$G(q,z,y)=p_{n+1}+\b{F}'(z)\cdot p$と表せるから,
    \[D_qG=\vctr{\b{F}'(z)}{1},\quad D_yG=0,\quad D_zG=\b{F}''(z)\cdot p.\]
    \begin{enumerate}
        \item 任意の$y^0=(x^0,0)\in\Gamma$は非特性的.実際,$G_{p_{n+1}}=1$であるため.
        \item 射影特性曲線$\b{y}=(\b{x},x_{n+1})$に関する特性方程式は
        \[\dot{\b{x}}(s)=D_qG\quad\Leftrightarrow\quad\begin{cases}
            \dot{x}^i(s)=F^{i'}(z(s))&i\in[n]\\
            \dot{x}^{n+1}(s)=1.
        \end{cases}\]
        これを解いて,$x^{n+1}(s)=s$.$x^i$についてはまだわからない.
        \item $z(s)$に関する特性方程式は$\dot{z}(s)=D_qG\cdot q=p_{n+1}+\b{F}'(z)\cdot p=0$より,
        \[z(s)=z^0=g(x^0)\]
        と定値である.
        よって$\x=(x^1,\cdots,x^n)$の微分方程式も解けて,
        \[\b{x}(s)=\b{F}'(g(x^0))s+x^0.\]
        \item 以上より,射影特性曲線は
        \[\b{y}(s)=\vctr{\b{x}(s)}{s}=\vctr{\b{F}'(g(x^0))s+x^0}{s}\qquad(s\in\R_+).\]
        という直線となり,その上で$u$は定値である.
    \end{enumerate}
\end{example}
\begin{remarks}
    保存則に由来するPDEの特性曲線はいずれも直線で,その上で$u$は一定の値を取る.
    これがこのクラスのPDEの名前の由来である.
\end{remarks}

\begin{remark}[陰関数表示について]
    $\b{y}$の時刻を表す$n+1$番目の変数$t=x_{n+1}$は$s$に等しかったから,
    $\R^{(n+1)+1}$上の特性曲線$s\mapsto\vctr{\b{y}(s)}{z(s)}$のパラメータ$s$を消去できる:
    \[u(\b{x}(t),t)=z(t)=g(\b{x}(t)-t\b{F}'(z^0))=g(\b{x}(t)-t\b{F}'(u(\b{x}(t),t))).\]
    よって,局所解の形$u(x,t)$としては,
    \[u=g(x-t\b{F}'(u)).\]
    これは
    \[1+tDg(x-t\b{F}'(u))\cdot\b{F}''(u)\ne0\]
    が成り立つ限り,たしかに局所解を与えている(したがっていずれ成り立たなくなり,大域解を与えない).また,斉次な定数係数輸送方程式の解$u(x,t)=g(x-tb)$の一般化になっている\ref{observation-homogeneous-transport-eq}.
\end{remark}

\subsection{完全非線形な場合:Hamilton-Jacobi方程式}

\section{Hamilton-Jacobi方程式}

\begin{history}
    Hamiltonian $H$が凸の場合,「解」の概念は1970sには合意が取れていた.
    $H$が一般の場合も含めて統一的に扱えるようになったのは,Crandall and Lionsによる粘性解の概念からである.
    この概念は,超関数微分の概念の代わりに最大値原理を反映させたものである.
\end{history}

\section{保存則とBurgers方程式}

\begin{tcolorbox}[colframe=ForestGreen, colback=ForestGreen!10!white,breakable,colbacktitle=ForestGreen!40!white,coltitle=black,fonttitle=\bfseries\sffamily,
title=]
    スカラー保存則を与える方程式について,
    特性曲線法は途中で失敗し,$t>0$上大域的な解は一般に存在しない.
    そこで,やはり弱解の概念を考えることになるが,
    全ての超関数を許容すると初期値問題に対して一意性が満たされないから,
    保存則の構造と物理的意味を加味した制限をかけることで,これを回復することを考える.
\end{tcolorbox}

\begin{problem}\label{prob-conservation-law}
    1次元のスカラー保存則の初期値問題
    \[\begin{cases}
        u_t+F(u)_x=0&\In\R\times\R^+,\\
        u=g&\on\R\times\{0\}.
    \end{cases}\]
    を考える.$F,g:\R\to\R$を所与とする.
\end{problem}
\begin{example}[Burgers方程式]
    1次元のBurgers方程式
    \[\begin{cases}
        u_t+uu_x=0&x\in\R,t>0,\\
        u(x,0)=g(x)&x\in\R.
    \end{cases}\]
    は$F(u)=\frac{u^2}{2}$とした場合である.
    これは流体力学の非線形波動を記述したものと捉えられる.
    Johannes Burgers 95-81 蘭による.
\end{example}

\subsection{積分解:積分方程式への変換}

\begin{tcolorbox}[colframe=ForestGreen, colback=ForestGreen!10!white,breakable,colbacktitle=ForestGreen!40!white,coltitle=black,fonttitle=\bfseries\sffamily,
title=]
    $v\in C_c^\infty(\R\times\R_+)$を乗じて部分積分をする,これは実は背後に超関数論を想定した行為である.
\end{tcolorbox}

\begin{observation}
    問題に$v\in C_c^\infty(\R\times\R_+)$を乗じて部分積分をすると,
    \[\int_0^\infty\int^\infty_{-\infty}\Paren{uv_t+F(u)v_x}dxdt+\int^\infty_{-\infty}gv\,dx\biggr|_{t=0}=0.\]
\end{observation}

\begin{definition}[integral solution]
    有界関数$u\in L^\infty(\R\times\R^+)$が任意の試験関数$v\in C_c^\infty(\R\times\R_+)$について上式を満たすならば,元の式\ref{prob-conservation-law}の\textbf{積分解}であるという.
\end{definition}

\subsection{積分解の性質:不連続点が線をなすとき}

\begin{problem}
    領域$V\osub\R\times\R^+$とこれを通る可微分曲線$C$を考え,これによって$V$は左側と右側$V_l,V_r$に分割されるとし,
    各$V_l,V_r$上で積分解$u$は$C^1$-級で,さらに$u$とその導関数が各$V_r,V_l$にて一様連続である状況を考える.
    \begin{enumerate}
        \item $\b{\nu}=(\nu^1,\nu^2)$を$C$に直交する単位ベクトルであって,向きは$V_l$から$V_r$へ向かうものとする.
        \item $u_l,u_r$をそれぞれ$V_l,V_r$上からの$\nu$の方向に沿った極限値であるとする.
    \end{enumerate}
\end{problem}

\begin{proposition}[積分解であるための必要条件 (Rankine-Hugoniot)]
    上述の条件を満たす積分解$u$について,
    \begin{enumerate}
        \item 弱い意味で,$V_l,V_r$上で元の方程式を満たす:
        \[u_t+F(u)_x=0\qquad\In V_l\cup V_r.\]
        \item $C$上で弱い意味で次の方程式を満たす:
        \[(F(u_l)-F(u_r))\nu^1+(u_l-u_r)\nu^2=0\qquad\on C.\]
        特に,$C=\Brace{(x,t)\in\R^2\mid x=s(t)}$と可微分関数$s:\R_+\to\R$を用いて表現されているとき,
        \[\b{\nu}=\frac{1}{\sqrt{1+\dot{s}^2}}\vctr{1}{-\dot{s}}\]
        であるから,
        \[F(u_l)-F(u_r)=\dot{s}(u_l-u_r)\qquad\on C.\]
    \end{enumerate}
\end{proposition}
\begin{Proof}\mbox{}
    \begin{enumerate}
        \item $V_l,V_r$に台を持つ試験関数$v\in\D(V_l),\D(V_r)$について評価することで得る.
        \item 一般の試験関数$v\in\D(V)$について評価することで得る.
    \end{enumerate}
\end{Proof}

\begin{definition}[jump in $u$ across the curve $C$]\mbox{}
    \begin{enumerate}
        \item $[[u]]:=u_l-u_r$を$u$の$C$での\textbf{ジャンプ}という.
        \item $[[F(u)]]:=F(u_l)-F(u_r)$を$F(u)$の$C$での\textbf{ジャンプ}という.
        \item $\sigma:=\dot{s}$を曲線$C$の\textbf{速さ}という.
        \item 命題の(2)の条件は$[[F(u)]]=\sigma[[u]]$とも表せる.これを\textbf{衝撃曲線$C$に沿ったRankine-Hugoniot条件}という.
    \end{enumerate}
\end{definition}

\subsection{例}

\begin{example}[Burgers方程式の衝撃波]
    \[\begin{cases}
        u_t+uu_x=0&x\in\R,t>0,\\
        u(x,0)=g(x)&x\in\R.
    \end{cases}\]
    について,$t=0$における初期条件
    \[g(x)=\begin{cases}
        1&x\le 0,\\
        1-x&0\le x\le 1,\\
        0&x\ge1.
    \end{cases}\]
    を考える.$\R\times\cointerval{1,\infty}$上で特性曲線は交差してしまう.
    $t\ge1$上での関数
    \[u(x,t):=\begin{cases}
        1&x<s(t),\\
        0&s(t)<x.
    \end{cases},\qquad s(t)=\frac{1+t}{2}.\]
    はRankine-Hugoniot条件を満たす.
\end{example}
\begin{Proof}\mbox{}
    \begin{description}
        \item[問題の所在] 特性方向は$\b{b}=\vctr{1}{u}$で,これを解くと
        \[\vctr{t}{x}=\vctr{t}{g(x_0)t+x_0}.\]
        より,$y$切片が$x_0$で傾きが$g(x_0)$の直線の族である.$u$の値は,
        \[u(x,t)=\begin{cases}
            1&x\le t,\\
            \frac{1-x}{1-t}&t\le x,\\
            0&x\ge1.
        \end{cases},\qquad t\in[0,1].\]
        であり,それ以降は特性曲線が交点を持ってしまう.
        \item[Rankine-Hugoniot条件] $s(t)=\frac{1+t}{2}$が表す曲線は$y$切片$1/2$で傾き$1/2$の直線であり,点$(1,1)$を通る.
        これについて,$u_l=1,u_r=0$であるから$[[u]]=1$.よって,
        \[[[F(u)]]=F(u_l)-F(u_r)=\frac{1}{2}(u_l^2-u_r^2)=\frac{1}{2}=\sigma[[u]].\]
    \end{description}
\end{Proof}

\subsection{積分解の多様性とエントロピー条件}

\begin{tcolorbox}[colframe=ForestGreen, colback=ForestGreen!10!white,breakable,colbacktitle=ForestGreen!40!white,coltitle=black,fonttitle=\bfseries\sffamily,
title=]
    非連続な積分解を,射影特性曲線上で見ていく.
    $t\nearrow\infty$と探していくと崩壊するが,
    $t\searrow\infty$という逆方向の探索には如何なる不連続点も見つからない,という性質は期待できる.
\end{tcolorbox}

\begin{example}
    今度は$t=0$における初期条件
    \[g(x)=\begin{cases}
        0&x<0,\\
        1&x>0.
    \end{cases}\]
    を考える.特性曲線は交差しないが,楔形の領域$\Brace{0<x<t}$上には如何なる特性曲線も存在しない.
    \begin{enumerate}
        \item \[u_1(x,t)=\begin{cases}
            0&x<t/2,\\
            1&x>t/2.
        \end{cases}\]
        は積分解である.
        \item \[u_2(x,t)=\begin{cases}
            1&x>t,\\
            \frac{x}{t}&0<x<t,\\
            0&x<0.
        \end{cases}\]
        も積分解である.$u_2$は\textbf{希薄解}(rarefaction wave)という.
    \end{enumerate}
    いずれも傾きは増加しており,エントロピー解ではない.
\end{example}

\begin{definition}[Lax's entropy condition, shock]
    保存則型の方程式
    \[u_t+F(u)_x=0.\]
    の積分解$u$は曲線$C$上で不連続であるとする.
    \begin{enumerate}
        \item $C$上の点を特性曲線が通っており,これに沿った左極限$u_l$と右極限$u_r$が存在するとする.条件
        \[F'(u_l)>\sigma>F'(u_r)\]
        を\textbf{エントロピー条件}という.
        \item Rankine-Hugoniot条件とエントロピー条件の双方を満たすような,積分解$u$の不連続点がなす曲線$C$を\textbf{衝撃}という.
    \end{enumerate}
\end{definition}
\begin{remarks}[エントロピー条件の意味]
    射影特性曲線は
    \[\x(s)=(F'(g(x^0))s+x^0,s),\qquad s\in\R_+.\]
    とパラメタ付けられる.
    \begin{enumerate}
        \item $\sigma$は衝撃曲線の傾きと解釈できる.
        \item $F'(u_l)$は特性曲線の左側における傾き,$F'(u_r)$は右側における傾きである.
    \end{enumerate}
    したがって,「積分解$u$の不連続曲線を通過することによって傾きが減少する」ということが,$\R\times\R_+$上でのエントロピー条件の解釈である.
    最後に,$\R\times\R_+$上における特性曲線の傾きは小さいほど波速は速いとみなせる.
    したがって,「shock後には波速は増加する」という意味で物理的に自然な解を選びだしている行為ともみなせる.
    名前の由来は,熱力学において,$t$の増加にしたがってエントロピーは減少し得ないという条件である.
\end{remarks}

\subsection{エントロピー条件の一意性}

\begin{theorem}[\cite{Evans} 3.4.2 Th'm.3]
    $F\in C^\infty(\R^n)$は凸であるとする.このとき,エントロピー解はa.e.の違いを除いて一意である.
\end{theorem}

\subsection{Lax-Oleinikの公式}

flux関数$F$が一様凸である際に与えられる弱解の公式である.

\chapter{2階方程式}

\section{解の存在の概観}

\begin{tcolorbox}[colframe=ForestGreen, colback=ForestGreen!10!white,breakable,colbacktitle=ForestGreen!40!white,coltitle=black,fonttitle=\bfseries\sffamily,
title=]
    複素領域の常微分方程式は,データが正則である限り,正則な解(=収束冪級数解)を持つ\cite{竹井義次-RIMS-テキスト}.
\end{tcolorbox}

\begin{remarks}[正規形ODEの性質まとめ]\mbox{}
    \begin{itemize}
        \item 領域$\Om\osub\bF^{n+1}$
        \item その上の関数$f:\Om\to\C$
    \end{itemize}
    が与える
    連立1階系の初期値問題
    \[\dd{u}{x}=f(x,u),\quad u(x_0)=a,\qquad u:={}^t\!(u^1,\cdots,u^n),\,(x_0,a)\in\Om\]
    は,
    \begin{enumerate}
        \item $\bF=\R$で$f$が連続ならば,$C^1$-級の局所解が存在する.一意性は不問.
        \item さらに,$f$が$u$についてLipschitz連続ならば,解も一意で,$C^1$-級になる.
        \item $\bF=\C$で$f$が正則ならば,解も正則である.
    \end{enumerate}
\end{remarks}

\begin{theorem}[Cauchy]
    領域$\Om\osub\C^{n+1}$上の正則関数$f:\Om\to\C$が与える
    連立1階系の初期値問題
    \[\dd{u}{x}=f(x,u),\quad u(x_0)=a,\qquad u:={}^t\!(u^1,\cdots,u^n),\,(x_0,a)\in\Om\]
    は,$x_0$の近傍で唯一つの解を持ち,それは正則である.
\end{theorem}

\begin{theorem}[Cauchy-Kowalevski]
    \[\paren{\pp{}{x_1}}^mu(x)+\sum_{j=0}^{m-1}\sum_{\abs{\nu}\le m-j}a_{j,\nu}(x)\paren{\pp{}{x_1}}^j\paren{\pp{}{x'}}^\nu u(x)=f(x),\qquad x':=(x_2,\cdots,x_n),\nu\in\N^{n-1}.\]
    について,係数$a_{j,\nu}$と右辺$f$は$x=0$にて正則とする.このとき,任意の$x'\in\C^{n-1}$近傍の正則関数$g_k(x')\;(k\in m)$について,初期条件
    \[\paren{\pp{^ku}{x_1^k}}(0,x')=g_k(x'),\qquad(k\in m)\]
    を満たす解が$x=0$の近傍で唯一つ存在し,正則である.
\end{theorem}

\section{Cauchy-Kovalevskayaの定理}

\section{Hans Lewy (57)}

\begin{tcolorbox}[colframe=ForestGreen, colback=ForestGreen!10!white,breakable,colbacktitle=ForestGreen!40!white,coltitle=black,fonttitle=\bfseries\sffamily,
title=]
    偏微分方程式はODEと本質的に異なるもの.
\end{tcolorbox}

\begin{theorem}[Lewy (57)]
    $\phi\in C^\infty(\R)$が定める方程式
    \[-\pp{u}{x_1}-i\pp{u}{x_2}+2i(x_1+ix_2)\pp{u}{x_3}=\phi'(x_3).\]
    を考える.$u$が原点近傍での$C^1$-級解ならば,$\phi$は$x_3=0$の近傍で実解析的である必要がある.
\end{theorem}
\begin{remarks}
    したがって,$\phi$として$C^\infty$-級であるが実解析的ではない関数を持って来れば,方程式は原点の近傍で局所解を持たない.
    この方程式は$\C^2$上のCauchy-Riemann方程式のある領域の境界への制限の形をしており,その構造を利用することで複素解析的な議論から証明した.
\end{remarks}

\chapter{関数解析による議論}

\begin{quotation}
    偏微分方程式を真に作用素論的に見るには,定義域を明確化する必要があり,殆どの場合,自然な定義域はSobolev空間として与えられる.
\end{quotation}

\section{ペアリングを通じた微分の一般化}

\begin{tcolorbox}[colframe=ForestGreen, colback=ForestGreen!10!white,breakable,colbacktitle=ForestGreen!40!white,coltitle=black,fonttitle=\bfseries\sffamily,
title=]
    Sobolev, S. L. (1938)は微分を一般化する.
    これは現代的には,超関数微分$\partial:\D'(U)\to\D'(U)$の$L^1_\loc(U)$への制限に他ならない.
    このとき,像は$L^1_\loc(U)$には収まらない.例えば
    $\log\abs{x}\in L^1_\loc(\R)$の超関数微分は$L^1_\loc(\R)$から飛び出してしまうため,「弱微分可能ではない」ということになる.
\end{tcolorbox}

\begin{observation}
    $m$階連続微分可能かつ$m$階までの偏導関数が$L^p$-に属する関数の全体は完備ではない.
    これは,「連続微分可能」というときの微分の概念が強すぎることによる.
    かと言って,殆ど至る所の微分にしたからと言って解決はしない.
\end{observation}

\subsection{Holder空間}

\begin{tcolorbox}[colframe=ForestGreen, colback=ForestGreen!10!white,breakable,colbacktitle=ForestGreen!40!white,coltitle=black,fonttitle=\bfseries\sffamily,
title=]
    導関数の滑らかさの情報を半ノルムとして加えることで,Banach空間を構成できる.
    この手法を弱微分の概念を併せるとSobolev空間を得る.
\end{tcolorbox}

\begin{definition}
    $U\osub\R^n,0<\gamma\le1$について,
    \begin{enumerate}
        \item 次が成り立つとき,$u:U\to\R$は\textbf{$\gamma$-Holder連続}であるという:
        \[\abs{u(x)-u(y)}\le C\abs{x-y}^\gamma,\qquad(x,y\in U).\]
        \item $u\in C_b(U)$のノルム$\norm{u}_{C(\o{U})}:=\sup_{x\in U}\abs{u(x)}$を考える.
        \item \textbf{$\gamma$-Holder半ノルム}を
        \[[u]_{C^{0,\gamma}(\o{U})}:=\sup_{x\ne y\in U}\frac{\abs{u(x)-u(y)}}{\abs{x-y}^\gamma}\]
        で定める.
        \item \textbf{$\gamma$-Holderノルム}を
        \[\norm{u}_{C^{0,\gamma}(\o{U})}:=\norm{u}_{C(\o{U})}+[u]_{C^{0,\gamma}(\o{U})}\]
        で定める.
        \item 次のノルムが有限な関数$u\in C^k(\o{U})$の全体がなす部分空間を$C^{k,\gamma}(\o{U})\subset C^k(\o{U})$で表す:
        \[\norm{u}_{C^{k,\gamma}(\o{U})}:=\sum_{\abs{\al}\le k}\norm{D^\al u}_{C(\o{U})}+\sum_{\abs{\al}=k}[D^\al u]_{C^{0,\gamma}(\o{U})}.\]
        すなわち,$k$-階連続微分可能で,その$k$階の偏導関数が全て有界かつ$\gamma$-Holder連続なものの全体である.
    \end{enumerate}
\end{definition}

\begin{theorem}
    $C^{k,\gamma}(\o{U})$はBanach空間をなす.
\end{theorem}

\subsection{弱微分の定義}

\begin{definition}[weak partial derivative]
    $u,v\in L^1_\loc(U)$と$\al\in\N^n$について,$v$が$u$の\textbf{$\al$-階の弱微分}であるとは,
    \[\forall_{\varphi\in\D(U)}\quad\int_UuD^\al\varphi dx=(-1)^{\abs{\al}}\int_Uv\varphi dx.\]
    を満たすことをいう.これを$D^\al u=v$と表す.
\end{definition}

\begin{lemma}[弱微分の一意性]
    $u$の$\al$-階の弱微分は存在すれば一意である.
\end{lemma}
\begin{Proof}
    仮に2つの
    $v,\wt{v}\in L^1_\loc(U)$がいずれも弱微分であるとする:
    \[\int_UuD^\al\varphi dx=(-1)^{\abs{\al}}\int_Uv\varphi dx=(-1)^{\abs{\al}}\int_U\wt{v}\varphi dx,\qquad(\varphi\in\D(U)).\]
    このとき,特に
    \[\forall_{\varphi\in\D(U)}\quad\int_U(v-\wt{v})\varphi dx=0\]
    が必要.変分法の基本補題より,$v=\wt{v}=0\;\ae$である.
\end{Proof}

\subsection{弱微分の性質}

\begin{proposition}[コンパクト台を持つ$C^m$-級関数は滑らかな関数で一様近似可能,]\mbox{}
    \begin{enumerate}
        \item 任意の$\varphi\in C_c^m(\Om)$について,$C_c^\infty(\Om)$の列$(\varphi_n)$であって,任意の$\abs{\al}\le m$について$D^\al\varphi_n$が$D^\al\varphi$に一様収束するものが存在する.
        \item $u\in L^1_\loc(\Om)$に弱微分$D^\al u=u_\al$が存在するとする.このとき,任意の$\varphi\in C^{\abs{\al}}_c(\Om)$についても,関係
        \[\forall_{\varphi\in C_c^{\abs{\al}}(\Om)}\;\int_\Om u(x)D^\al\varphi(x)dx=(-1)^{\abs{\al}}\int_\Om u_\al(x)\varphi(x)dx\]
        を満たす.
    \end{enumerate}
\end{proposition}

\begin{proposition}[線型作用素]
    $u,v\in L^1_\loc(\Om)$に対して$D^\al u,D^\al v$が存在するならば,$au+bv\;(a,b\in\C)$も弱微分可能で,$D^\al(au+bv)=aD^\al u+bD^\al v$.
\end{proposition}

\begin{theorem}[Leibniz則]
    $u\in L^1_\loc(\Om)$は1階の弱偏導関数$D_ju$を持つとする.
    $\psi\in C^1(\Om)$に対して,$D_j(\psi u)$が存在し,$D_j(\psi u)=D_j\psi\cdot u+\psi\cdot D_ju$.
\end{theorem}

\begin{proposition}[畳み込みとの可換性]
    $u\in L^1_\loc(\R^n)$に対して$D^\al u$が存在するならば,
    \[\forall_{\varphi\in C_c^{\abs{\al}}(\R^n)}\;D^\al(\varphi *u)=(D^\al\varphi)*u=\varphi *D^\al u.\]
\end{proposition}

\begin{proposition}
    領域$\Om\osub\R^n$上の$u\in L^1_\loc(\Om)$が,1階導関数$D_j(u)\;(j\in[n])$をすべて持ち,すべて$0$に等しいならば,$u$は殆ど至る所定数である:$\exists_{\al\in\C}\;u(x)=\al\;\ae$.
\end{proposition}

\subsection{微分の全射性}

\begin{tcolorbox}[colframe=ForestGreen, colback=ForestGreen!10!white,breakable,colbacktitle=ForestGreen!40!white,coltitle=black,fonttitle=\bfseries\sffamily,
title=]
    $W^{1,p}$の(商空間の)元は$L^p$の原始関数になっている.
\end{tcolorbox}

\begin{theorem}[連続代表元を取れば,原始関数である]
    任意の$u\in W^{1,p}(I)$に対して,ある連続な「修正」$\wt{u}\in C(I)$が存在して,$u=\wt{u}\;\ae$で,
    \[\forall_{x,y\in I}\quad\wt{u}(x)-\wt{u}(y)=\int^x_yu'(t)dt\]
    が成り立つ.このとき,$\wt{u}\in W^{1,p}(I)$となることに注意.
\end{theorem}

\section{Sobolev空間}

\begin{tcolorbox}[colframe=ForestGreen, colback=ForestGreen!10!white,breakable,colbacktitle=ForestGreen!40!white,coltitle=black,fonttitle=\bfseries\sffamily,
title=]
    導関数までノルムに寄与させることで,関数の滑らかさに関する情報を反映させることが出来る.
    $C_b^m(\Om)$がその発想であったが,一様ノルムを$L^p$-ノルムに一般化することが考え得る.
\end{tcolorbox}

\subsection{Sobolev空間の定義}

\begin{tcolorbox}[colframe=ForestGreen, colback=ForestGreen!10!white,breakable,colbacktitle=ForestGreen!40!white,coltitle=black,fonttitle=\bfseries\sffamily,
title=]
    $W^{k,p}(U)\subset L^1_\loc(U)\cap L^p(U)$に注意.
\end{tcolorbox}

\begin{definition}[Sobolev spaces]
    $p\in[1,\infty],k\in\N$とする.
    \begin{enumerate}
        \item \textbf{Sobolev空間}$W^{k,p}(U)$とは,
        \[W^{k,p}(U):=\Brace{u\in L^1_\loc(U)\mid \forall_{\al\in\N^n}\;\abs{\al}\le k\Rightarrow D^\al u\in L^p(U)\cap L^1_\loc(U)}.\]
        と定める.この空間の元であることを\textbf{$L^p$-の意味で$k$階微分可能}であるともいう.
        $k=0$のとき,$W^{0,p}(U)=L^p(U)$である.
        \item 特に$p=2$のとき,
        \[H^k(U):=W^{k,2}(U),\quad H^k_0(U):=W^{k,2}_0(U)\qquad(k\in\N)\]
        と表し,内積を
        \[(u|v)_{H^k(U)}:=\sum_{\abs{\al}\le k}(D^\al u|D^\al v)_{L^2(U)}\]
        と定める.
        \item $u\in W^{k,p}(U)$の\textbf{Sobolevノルム}を
        \[\norm{u}_{W^{k,p}(U)}:=\begin{cases}
            \paren{\sum_{\abs{\al}\le k}\int_U\abs{D^\al u}^pdx}^{1/p}&1\le p<\infty,\\
            \sum_{\abs{\al}\le k}\esssup_U\abs{D^\al u}&p=\infty.
        \end{cases}\]
        と定める.
        \item $\{u_m\}\subset W^{k,p}(U)$が$u\in W^{k,p}(U)$に\textbf{$W^{k,p}_\loc(U)$-収束}するとは,任意の開集合$V\ssub U$について$u_m\to u\;\In W^{k,p}(V)$が成り立つことをいう.
        \item $\D(U)\subset W^{k,p}(U)$の閉包を$W_0^{k,p}(U)$で表す.その元は$\forall_{\al\in\N^n}\;\abs{\al}\le k-1\Rightarrow D^\al u=0\;\on\partial U$を満たす元の全体とみなすことが出来る\ref{thm-trace-zero-function-in-W1p}.
    \end{enumerate}
\end{definition}

\begin{example}[1次元の場合は簡単な特徴付けが存在する]
    $U\osub\R$を開区間とする.次は同値:
    \begin{enumerate}
        \item $u\in W^{1,p}(U)$.
        \item $u$は絶対連続な修正を持ち,その殆ど至る所の微分は$L^p(U)$に属する.
    \end{enumerate}
\end{example}

\begin{example}
    開球$U:=B^\circ(0,1)\osub\R^n$上の関数$u(x):=\abs{x}^{-\al}\;(x\in U\setminus\{0\},\al>0)$について,次は同値:
    \begin{enumerate}
        \item $u\in W^{1,p}(U)$.
        \item $\al<\frac{n-p}{p}$.
    \end{enumerate}
    特に,$p\ge n$のとき,$\al>0$に依らず常に$u\notin W^{1,p}(U)$.
    一般に,$p(\al+k)\le n$のとき,$L^p$-の意味で$k$-階微分可能である.
\end{example}

\subsection{Sobolev空間上の弱微分の性質}

\begin{theorem}
    $u,v\in W^{k,p}(U),\abs{\al}\le k$とする.
    \begin{enumerate}
        \item 微分の結合性:任意の$\abs{\al}+\abs{\beta}\le k$について,$D^\al u\in W^{k-\abs{\al},p}(U)$かつ$D^\beta(D^\al u)=D^\al(D^\beta u)=D^{\al+\beta}u$.
        \item 微分の線形性:任意の$\lambda,\mu\in\R$について,$\lambda u+\mu v\in W^{k,p}(U)$かつ$D^\al(\lambda u+\mu v)=\lambda D^\al u+\mu D^\al v\;(\abs{\al}\le k)$.
        \item 制限:$V\osub U$ならば,$u\in W^{k,p}(V)$.
        \item $\D(U)$の作用とLeibniz則:$\zeta\in\D(U)$ならば$\zeta u\in W^{k,p}(U)$で,
        \[D^\al(\zeta u)=\sum_{\beta\le\al}\comb{\al}{\beta}D^\beta\zeta D^{\al-\beta}u.\]
    \end{enumerate}
\end{theorem}
\begin{Proof}\mbox{}
    \begin{enumerate}
        \item 任意に$\varphi\in\D(U)$を取ると,部分積分より,
        \begin{align*}
            \frac{1}{(-1)^{\abs{\beta}}}(D^\beta(D^\al u))(\varphi)&=(D^\al u|D^\beta\varphi)=\int_UD^\al uD^\beta\varphi dx=(-1)^{\abs{\al}}\int_UuD^{\al+\beta}\varphi dx\\
            &=(-1)^{\abs{\al}}(-1)^{\abs{\al+\beta}}\int_UD^{\al+\beta}u\varphi dx\\
            &=(-1)^{\abs{\beta}}\int_UD^{\al+\beta}u\varphi dx=(-1)^{\abs{\beta}}(D^{\al+\beta}u|\varphi).
        \end{align*}
        \item a
        \item b
        \item $\abs{\al}=1$とする.任意の$\varphi\in\D(U)$について,
        \begin{align*}
            (D^\al(\zeta u)|\varphi)&=-(\zeta u|D^\al\varphi)=-\int_U\zeta uD^\al\varphi dx\\
            &=-\int_U\Paren{uD^\al(\zeta\varphi)-u(D^\al\zeta)\varphi}dx\\
            &=\int_U(\zeta D^\al u+uD^\al\zeta)\varphi dx=(\zeta D^\al u+uD^\al\zeta|\varphi).
        \end{align*}
    \end{enumerate}
\end{Proof}

\begin{theorem}
    任意の$k\in\N,p\in[1,\infty]$について,$W^{k,p}(U)$はBanach空間である.
\end{theorem}
\begin{Proof}\mbox{}
    \begin{enumerate}[{Step}1]
        \item Sobolevノルム$\norm{u}_{W^{k,p}}(U)$はたしかにノルムをなす.
        \item 任意のCauchy列$\{u_m\}\subset W^{k,p}(U)$を取る.任意の$\abs{\al}\le k$について,$\{D^\al u_m\}\subset L^p(U)$はCauchy列であるから,極限$u_\al\in L^p(U)$を持つ:$D^\al u_m\to u_\al\;\In L^p(U)$.
        特に,$u_m\to u_0=:u\;\In L^p(U)$.
        これについて,$u\in W^{k,p}(U)$かつ$D^\al u=u_\al$である.

        実際,
        任意の$\varphi\in\D(U)$について,
        \begin{align*}
            (D^\al u|\varphi)&=(-1)^{\abs{\al}}\int_UuD^\al\varphi dx=(-1)^{\abs{\al}}\lim_{m\to\infty}\int_Uu_mD^\al\varphi dx\\
            &=\lim_{m\to\infty}(-1)^{2\abs{\al}}\int_UD^\al u_m\varphi dx\\
            &=\int_Uu_\al\varphi dx=(u_\al|\varphi).
        \end{align*}
        \item よって,$\forall_{\abs{\al}\le k}\;D^\al u_m\to D^\al u\;\In L^p(U)$を得た.
        すると,Sobolevノルムは各導関数の$L^p$-ノルムの和よりも大きくないから,
        $u_m\to u\;\In W^{k,p}(U)$である.
    \end{enumerate}
\end{Proof}

\subsection{稠密部分空間}

\begin{theorem}
    $1\le p<\infty$について,
    $C^{k,p}(\R^n)$は$W^{k,p}(\R^n)$上稠密である.
\end{theorem}

\begin{theorem}
    $1\le p<\infty$について,
    $\D(\R^n)$は$W^{k,p}(\R^n)$上稠密である.
    すなわち,$W^{k,p}_0(\R^n)=W^{k,p}(\R^n)$.
\end{theorem}

\subsection{Sobolev空間の例}

\begin{tcolorbox}[colframe=ForestGreen, colback=ForestGreen!10!white,breakable,colbacktitle=ForestGreen!40!white,coltitle=black,fonttitle=\bfseries\sffamily,
title=]
    滑らかさの尺度$p$の大きさを調節することで,多様な滑らかさを表現できる.
    強い順に,連続微分可能,Lipschitz連続,$\al$-Holder連続($0<\al\le 1$),絶対連続,一様連続,連続関数となる.
\end{tcolorbox}

\begin{example}[一次元の場合]
    $I\osub\R$とする.
    滑らかさの尺度$p$の大きさを調節することで,多様な滑らかさを表現できる.
    強い順に,連続微分可能,Lipschitz連続,$\al$-Holder連続($0<\al\le 1$),絶対連続,一様連続,連続関数となる.
    \begin{enumerate}
        \item $W^{1,1}(I)\subset C(I)$は絶対連続な関数のなす空間\ref{thm-absolutely-continuous}.
        \item $W^{1,2}(I)=H^1(I)$は$1/2$次Holder連続な関数のなす空間である.
        \item $W^{1,\infty}(I)=\Lip(I)$はLipschitz連続関数全体のなす空間となる.
    \end{enumerate}
    Cantor関数は絶対連続でないが,連続かつ殆ど至る所微分可能で,Lebesgue可積分な導関数を持つ.
\end{example}

\begin{proposition}
    高次の場合は,もはや連続でない関数を含む.
    しかし,次のような形一般化出来る.
    $\Om\osub\R^n$とする.
    \begin{enumerate}
        \item $f\in W^{1,p}(\Om)\;(1\le p\le\infty)$は$\R^n$の座標方向に平行な殆どすべての直線への制限が絶対連続であるような関数と,殆ど至る所等しい.
        \item $f\in W^{1,p}(\Om)\;(n\le p\le\infty)$は$\gamma:=1-\frac{n}{p}$次Holder連続な関数と殆ど至る所等しい.特に$p=\infty$のときは局所Lipschitzな関数と殆ど至る所等しい.
    \end{enumerate}
\end{proposition}

\subsection{Banach空間としての性質}

\begin{proposition}\mbox{}
    \begin{enumerate}
        \item $W^{k,p}(\Om)\;(1\le p<\infty)$は可分である.
        \item $W^{k,\infty}$はノルム代数となる.
    \end{enumerate}
\end{proposition}



\subsection{区間上の例}

\begin{tcolorbox}[colframe=ForestGreen, colback=ForestGreen!10!white,breakable,colbacktitle=ForestGreen!40!white,coltitle=black,fonttitle=\bfseries\sffamily,
title=]
    $W^{1,p}(a,b)$は,偏微分方程式の解空間みたいになっている.
    この範囲で探す解を,$C^{1,p}(a,b)$-古典解とは別に,弱解と呼ぶ.
    $W^{1,p}_0(a,b)$はそのうち,境界上で消えるもののなす部分空間となっている.
\end{tcolorbox}

\begin{theorem}
    $u\in L^p((a,b))\;(a<b\in\oR)$について,次の2条件は同値.
    \begin{enumerate}
        \item $u\in W^{1,p}(a,b)$.
        \item ある$v\in L^p(a,b),\al\in\C,c\in(a,b)$が存在して,$u(x)=\int^x_cv(y)dy+\al\;\ae$と表せる.
    \end{enumerate}
    このとき,$u$は至る所連続で,$u(c)=\al$かつ$v=Du$であることに注意.
\end{theorem}

\begin{definition}
    $a,b\in\R$で,区間が有界である場合を考える.
    このとき,$u(x)=\int^x_cDu(y)dy+\al$は$[a,b]$上に連続延長出来る.
\end{definition}

\begin{theorem}
    $(a,b)$が有界であるとき,$u\in W^{1,p}(a,b)$について,次の2条件は同値.
    \begin{enumerate}
        \item $u\in W^{1,p}_0(a,b)$.
        \item $u(a)=u(b)=0$.
    \end{enumerate}
\end{theorem}

\section{分数階のSobolev空間}

\subsection{分数階のSobolev-Hilbert空間}

\begin{tcolorbox}[colframe=ForestGreen, colback=ForestGreen!10!white,breakable,colbacktitle=ForestGreen!40!white,coltitle=black,fonttitle=\bfseries\sffamily,
title=]
    $H^k(U)$上には,Fourier変換先の消息から同値なノルムを定義することが出来る.
    このノルムが極めて有用になるのは,Hardy空間の理論と似ている理論になる.
    \cite{黒田成俊-関数解析} 6.5節.
\end{tcolorbox}

\begin{proposition}
    $u\in H^k(\R^n)$について,
    \[\norm{u}_{H^k(\R^n)}:=\paren{\int_{\R^n}\abs{\wh{u}(\xi)}^2(1+\abs{\xi}^2)^kd\xi}^{1/2},\qquad(u\in H^k(\R^n))\]
    と定めると,これは$\norm{u}_{W^{k,2}(\R^n)}$と同値なノルムになる.
\end{proposition}

\begin{definition}
    $s\in\R_+$について,Sobolev空間$H^s(\R^n)$とは,
    \[\norm{u}_s:=\paren{\int_{\R^n}\abs{\wh{u}(\xi)}^2(1+\abs{\xi}^2)^sd\xi}^{1/2},\qquad(u\in L^2(\R^n))\]
    を有限とする元の全体とする.
\end{definition}

\begin{theorem}
    任意の$s\in\R_+$について,
    $\D(\R^n)$は$H^s(\R^n)$で稠密である.
\end{theorem}

\begin{theorem}[Sobolevの埋蔵定理の特別な場合]
    $s>n/2$とする.
    \begin{enumerate}
        \item 任意の$u\in H^s(\R^n)$は一様連続な修正$\wt{u}$を持つ.
        \item ある$C>0$が存在して,任意の$u\in H^s(\R^n)$について,$\norm{\wt{u}}_{\infty}\le c\norm{u}_s$.
    \end{enumerate}
\end{theorem}
\begin{remarks}
    $H^s$は弱微分と$L^2$-ノルムで,$C_b^k$は微分と一様ノルムで,関数の滑らかさを測っている.この2つの尺度の対応は,$H^s$を$n/2$だけずらせば$C_b^k$の中に埋め込むことが出来る.
\end{remarks}

\subsection{関数の大域延長}

\begin{theorem}[extension theorem]
    $p\in[1,\infty],U\osub\R^n$を有界開集合で$\partial U$は$C^1$-級,
    $U\ssub V$を有界開近傍とする.このとき,有界線型作用素$E:W^{1,p}(U)\to W^{1,p}(\R^n)$が存在して,任意の$u\in W^{1,p}(U)$に対して次を満たす:
    \begin{enumerate}
        \item $Eu=u\;\ae\;\In U$.
        \item $\supp(Eu)\subset V$.
        \item ある$C>0$が存在して,任意の$u\in W^{1,p}(U)$について$\norm{Eu}_{W^{1,p}(\R^n)}\le C\norm{u}_{W^{1,p}(U)}$.
    \end{enumerate}
\end{theorem}

\subsection{境界での値と跡}

\begin{theorem}[trace theorem]
    $p\in[1,\infty],U\osub\R^n$を有界開集合で$\partial U$は$C^1$-級とする.
    このとき,有界線型作用素$T:W^{1,p}(U)\to L^p(\partial U)$が存在して,次を満たす:
    \begin{enumerate}
        \item 任意の$u\in W^{1,p}(U)\cap C(\o{U})$について,$Tu=u|_{\partial U}$.
        \item ある$C>0$が存在して,任意の$u\in W^{1,p(U)}$について,$\norm{Tu}_{L^p(\partial U)}\le C\norm{u}_{W^{1,p}(U)}$.
    \end{enumerate}
    $Tu\in L^p(\partial U)$を,$u$の$\partial U$上の\textbf{跡}という.
\end{theorem}

\begin{theorem}[trace-zero function in $W^{1,p}$]\label{thm-trace-zero-function-in-W1p}
    $p\in[1,\infty],U\osub\R^n$を有界開集合で$\partial U$は$C^1$-級とする.
    $u\in W^{1,p}(U)$について,次の2条件は同値:
    \begin{enumerate}
        \item $u\in W_0^{1,p}(U)$.
        \item $Tu=0\;\on\partial U$.
    \end{enumerate}
\end{theorem}

\section{Sobolev不等式}

\subsection{Sobolev不等式}

\subsection{Gagliardo-Nirenberg-Sobolevの不等式}

\begin{definition}
    $1\le p<n$について,$p$の\textbf{Sobolev共役}とは,
    \[p^*:=\frac{np}{n-p}\]
    をいう.このとき,
    \[\frac{1}{p^*}=\frac{1}{p}-\frac{1}{n},\qquad(p^*>p).\]
    が成り立つ.
\end{definition}

\begin{theorem}[Gagliardo-Nirenberg-Sobolev]
    $1\le p<n$とする.ある$C>0$が存在して,任意の$u\in C_c^1(\R^n)$に対して,
    \[\norm{u}_{L^{p^*}(\R^n)}\le C\norm{Du}_{L^p(\R^n)}.\]
\end{theorem}

\subsection{Sobolev埋め込み}

\begin{tcolorbox}[colframe=ForestGreen, colback=ForestGreen!10!white,breakable,colbacktitle=ForestGreen!40!white,coltitle=black,fonttitle=\bfseries\sffamily,
title=]
    $W^{1,p}(U)$はコンパクトに$L^q(U)\;(1\le q<p^*)$に埋め込める.
    特に,任意の$p\in[1,\infty]$について,$W^{1,p}(U)\mono L^p(U)$はコンパクトな埋め込みである.
\end{tcolorbox}

\begin{definition}
    $X,Y$をBanach空間とする.$i:X\mono Y$が\textbf{コンパクトな埋め込み}であるとは,次の2条件を満たすことをいう:
    \begin{enumerate}
        \item $i:X\mono Y$は有界である.
        \item $X$の任意の有界列は$Y$にて収束部分列を持つ.
    \end{enumerate}
\end{definition}

\begin{theorem}[Relich-Kondrachov compactness theorem]
    $p\in{[1,n)},U\osub\R^n$を有界開集合で$\partial U$は$C^1$-級とする.
    このとき,任意の$1\le q<p^*$に対して,$W^{1,p}(U)\mono L^q(U)$はコンパクトな埋め込みである.
    特に,任意の$p\in[1,\infty]$について,$W^{1,p}(U)\mono L^p(U)$はコンパクトな埋め込みである.
\end{theorem}

\section{Hilbert空間上の完全連続作用素}

\begin{tcolorbox}[colframe=ForestGreen, colback=ForestGreen!10!white,breakable,colbacktitle=ForestGreen!40!white,coltitle=black,fonttitle=\bfseries\sffamily,
title=]
    線形PDEは要は関数空間上の線形方程式
    $zu-Au=f$である.
    $A$が完全連続で$\C\ni z\ne0$のとき,全く同様に解くことができて,
    $z$は解素の元$z\in\rho(A)$または固有値$z\in\Sp(A)$のいずれかが起こる.
\end{tcolorbox}

\subsection{完全連続性の定義}

\begin{definition}[completely continuous]
    弱-ノルム連続な作用素を\textbf{完全連続}という.Hilbert空間ではコンパクト性で同値であり,一般のBanach空間でも完全連続ならばコンパクトであるが,逆は成り立たない.
    二乗可積分な核に関する積分作用素は完全連続になる.実は跡が有限になるコンパクト作用素のクラスに一致する.
    二乗可積分な核の$L^2$-ノルムを,作用素のSchimidtノルムという.
\end{definition}

\begin{theorem}[Riesz-Schauderの定理のHilbert空間における消息]
    $X$をHilbert空間,$A\in B_0(X)$を完全連続,$z\in\C\setminus\{0\}$とする.
    方程式$zu-Au=f$について,次のいずれかが成り立つ:
    \begin{enumerate}
        \item $z\in\Sp(A)$で,一意可解でない.
        \item $z$は解素の元で,任意の$f\in X$について一意の解を持ち,対応$f\mapsto u=(z-A)^{-1}f$は連続である.
    \end{enumerate}
    この(2)の場合をwell-posedというのである.
\end{theorem}

\subsection{完全連続作用素の固有値}

\begin{theorem}
    $X$をHilbert空間,
    $A:X\to X$を完全連続とする.
    $A$の固有値$\lambda\in\Sp(A)$について,
    \begin{enumerate}
        \item $\lambda\ne0$ならば,重複度は1以上の有限の値である.
        \item $0$以外の点に集積することはない.
    \end{enumerate}
\end{theorem}

\begin{theorem}
    さらに$A\in B(X)_+$を(半正定値な)自己共役作用素とする.
    このとき,$A$の$0$に収束する正の固有値の減少列$(\lambda_n)$であって,それぞれに属する正規化された固有ベクトルの列$(\varphi_n)$が$\oo{\Im A}$の正規直交基底であるものが取れる:
    \[\forall_{u\in X}\;Au=\sum_{n=0}^\infty(u|\varphi_n)\lambda_n\varphi_n.\]
    また,半正定値性の仮定は取り去れる.
\end{theorem}

\section{Laplacianの固有値問題}

\begin{problem}
    有界領域$\Om\subset\R^n$上の
    Dirichlet固有値問題
    \[\begin{cases}
        -\Lap u=\lambda u\quad\on\Om\\
        u=0\quad\on\partial\Om.
    \end{cases}\]
    の弱解
    \begin{enumerate}
        \item $u\in H^1_0(\Om),u\ne0$.
        \item $\forall_{h\in H^1_0(\Om)}\;(\nabla u,\nabla h)_{L^2}=\lambda(u,h)_{L^2}$.
    \end{enumerate}
    を考える.
    $X:=L^2(\Om)$とし,Green作用素を$G:X\to H^1_0(\Om)\subset X$で表す.
    すると,$G$の固有関数が弱解になる.
\end{problem}

\begin{proposition}
    $G$はコンパクトな自己共役作用素である.
\end{proposition}

\begin{theorem}\mbox{}
    \begin{enumerate}
        \item $G$の正の固有値の$0$に収束する列$\mu_n$が存在して,それぞれに属する固有関数の列$\varphi_n$からなる$X$の正規直交基底が存在する.
        \item 逆作用素を$-\Lap:=G$と定めれば,これの
        無限大に発散する正の固有値の列$\lambda_n$が存在して,それぞれに属する固有関数の列$\varphi_n$であって,$X$の直交基底であるものが取れる.
    \end{enumerate}
\end{theorem}

\section{固有値問題における変分原理}

\begin{tcolorbox}[colframe=ForestGreen, colback=ForestGreen!10!white,breakable,colbacktitle=ForestGreen!40!white,coltitle=black,fonttitle=\bfseries\sffamily,
title=]
    最小の固有値はRayleigh商の最小値,次に小さい固有値は直交補空間$(\C\varphi_0)^\perp$での最小値として特徴付けられる.
    また,$n$番目に小さい固有値は,$n$次元部分空間に制限することで別の方法でも探せる.
\end{tcolorbox}

\subsection{Rayleighの原理}

\begin{observation}
    $G=(-\Lap)^{-1}$の最大の固有値$\mu_0$は,Rayleigh商の最大値である:
    \[\mu_0=\max_{u\in X}\frac{(Gf|f)}{\norm{f}^2}\]
    すなわち,これは汎関数
    \[J[u]:=\frac{(Gu|u)}{\norm{u}^2}\]
    の$L^2(\Om)$上における最大化問題の解である.
    これの双対命題が次のとおりである.
\end{observation}

\begin{theorem}
    有界領域$\Om\osub\R^n$上のDirichlet条件の下での$-\Lap$の最小固有値$\lambda_0$とそれに属する固有関数$\varphi_0$は,Rayleigh商
    \[R[u]:=\frac{\norm{\nabla u}^2}{\norm{u}^2}\]
    を$H^1_0(\Om)$において最小にする変分問題の解として特徴付けられる.
\end{theorem}

\subsection{min-max原理}

\begin{theorem}[Courantの最大最小原理]
    有界領域$\Om\osub\R^n$上のDirichlet条件の下での$-\Lap$の$n$番目に小さい固有値$\lambda_n$は,次のように表せる.
    線形独立な$f_1,\cdots,f_n\in H^1_0(\Om)$についての直交補空間を
    \[V(f_1,\cdots,f_n):=\Brace{u\in H^1_0(\Om)\mid u\perp f_i}\]
    と表すと,
    \[\lambda_n=\max_{(f_1,\cdots,f_n)\in (H^1_0(\Om))^n}\paren{\min_{u\in V(f_1,\cdots,f_n)}R[u]}.\]
\end{theorem}

\chapter{参考文献}



\bibliography{../StatisticalSciences.bib,../SocialSciences.bib,../mathematics.bib,../statistics.bib}
\begin{thebibliography}{99}
    \item 
    大島利雄,小松彦三郎 (1977). 『1階偏微分方程式』(岩波基礎数学,岩波講座基礎数学,解析学(II) iii).
    \item 
    小松彦三郎 (1978). 『超関数論入門』(岩波講座基礎数学,解析学(II) ix).
    \item 
    金子晃 (1976). 『定数係数線型偏微分方程式』(岩波講座基礎数学,解析学(II) v).
    \item 
    小松彦三郎 (1978). 『Fourier解析』(岩波講座基礎数学,解析学(II) vi).
    \item 
    金子晃 (1998). 『偏微分方程式入門』(東京大学出版会,基礎数学12).
    \item 
    熊ノ郷準 (1978). 『偏微分方程式』(共立数学講座14).

    \item 
    Evans, L. C. (2010). \textit{Partial Differential Equations}. AMS.
    \item 
    Gilbarg, D. and Trudinger, N. S. (2001). \textit{Elliptic Partial Differential Equations of Second Order}. Springer.
    \item 
    相川弘明 (2008). 『複雑領域上のディリクレ問題』(岩波数学叢書).
    \item 
    Walter Rudin \textit{Real and Complex Analysis}.

    \item 
    Petrovsky, I. G. (1954). \textit{Lectures On Partial Differential Equations}. Interscience Publishers Inc.
\end{thebibliography}

\end{document}