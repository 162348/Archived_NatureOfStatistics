\documentclass[uplatex,dvipdfmx]{jsarticle}
\title{現象数理I レポート\\担当:加藤晃史先生}\author{05-210520 司馬博文}\date{\today}
\pagestyle{headings}\setcounter{secnumdepth}{4}
\input{/Users/Hirofumi Shiba/NatureOfStatistics/preamble_no_fonts.tex}
%\input{/Users/hirofumi.shiba48/NatureOfStatistics/preamble_no_fonts.tex}
\usepackage[math]{anttor}
\begin{document}
\maketitle

\section*{まえがき}

授業中に出題された
Exerciseにのみ,番号を振って下線を引いてあり,
必要に応じて準備した定理や補題は太字にはしましたが
番号は振らず,下線も付しませんでした.

\section{解析力学}

\subsection{第2回授業:Hamilton系の例}

\begin{theorem*}[積分領域が動く積分に対する微分\footnote{\cite{Evans}}]\label{thm-differentiation-of-integral-on-moving-region}
    $U_\tau\subset\R^n$を滑らかな境界を持つ領域の滑らかな族,
    $\partial U_\tau$の各点の速度ベクトルを$\b{v}$で表す.
    このとき,任意の滑らかな関数$f\in C^1(\R^n\times\R)$に対して,
    \[\dd{}{\tau}\int_{U_\tau}fdx=\int_{\partial U_\tau}f\b{v}\cdot\nu dS+\int_{U_\tau}f_\tau dx.\]
\end{theorem*}
\begin{remark*}
    $n=2$ではLeibnizの微分則,$n=3$ではReynolds transport theorem
    と呼ばれ,連続体力学の分野で知られていた.
\end{remark*}

\begin{exercise}[閉軌道の周期の特徴付け]
    $f$のある等高線
    \[C_E:=\Brace{(x,y)\in\R^2\mid f(x,y)=E}\]
    が単純閉曲線であり,囲む有界領域の面積を$S(E)$とすると,これはplanck定数と同じ次元を持ち,
    \[S(E):=\oint_{C_E}ydx=\int_{D_E}dx\wedge dy.\]
    が成り立つ.これが表す運動の周期は
    \[T(E)=\dd{S(E)}{E}.\]
    で計算できる.
\end{exercise}
\begin{proof}
    \[f=\frac{y^2}{2}+U(x)\quad\Leftrightarrow\quad y^2=2E-2U(x)\]
    に注意すると,
    \begin{align*}
        S(E)&=\int_Cydx\\
        &=\int^{x_{\mathrm{max}}}_{x_{\mathrm{min}}}\sqrt{2}\sqrt{E-U(x)}dx-\int^{x_{\mathrm{max}}}_{x_{\mathrm{min}}}(-\sqrt{2}\sqrt{E-U(x)})dx\\
        &=2\sqrt{2}\int^{x_{\mathrm{max}}}_{x_{\mathrm{min}}}\sqrt{E-U(x)}dx
    \end{align*}
    と変形出来る.ここで,$x_{\mathrm{max}},x_{\mathrm{min}}$の$E$に対する微分係数をそれぞれ$v(x_{\mathrm{max}}),v(x_{\mathrm{min}})$とすると,
    微分は
    \begin{align*}
        \dd{S(E)}{E}&=2\sqrt{2}\SQuare{\sqrt{E-U(x)}v(x)}^{x=x_{\mathrm{max}}}_{x=x_{\mathrm{min}}}+2\sqrt{2}\int^{x_{\mathrm{max}}}_{x_{\mathrm{min}}}\frac{1}{2\sqrt{E-U(x)}}dx\\
        &=\sqrt{2}\int^{x_{\mathrm{max}}}_{x_{\mathrm{min}}}\frac{1}{\sqrt{E-U(x)}}dx=T(E).
    \end{align*}
    と計算できる(前の定理を用いた).
\end{proof}

\begin{exercise}[Hamilton系の周期解は摂動に対して安定]
    $f,g\in C^\infty(\R^2)$について,摂動されたHamiltonian
    \[H_\ep:=f+\ep g\]
    を考える.$H_0$が$(x_0,y_0)$を初期値とする周期解を持つとき,
    $\ep$が十分小さいならば$H_\ep$も同様の初期値を持つ周期解を持つ.
\end{exercise}
\begin{proof}
    \[C_\ep:=\Brace{(x,y)\in\R^2\mid H_\ep(x,y)=H_\ep(x_0,y_0)=:E_\ep}.\]
    は$\ep=0$のとき閉曲線であり,
    微分が消えないパラメータ付け$\gamma:\R\to C_\ep$を持つとする.
    \begin{enumerate}[{Step}1]
        \item このとき,十分小さい$\ep>0$について,
        \[C_\ep=\Brace{f(x,y)+\ep g(x,y)=E_\ep}\]
        も閉曲線である.実際,任意の点$(x,y)\in C_\ep$において,
        $\grad(f+\ep g)$は消えず,したがって陰関数定理より局所的な
        陽な表示$\Brace{h_\ep(x)=y}\;(h_\ep\in C^\infty)$を持つ.
        これは$\ep\to0$のとき,$(x,y)$の近傍で$C_0$の一部に連続に変形する.
        \item $\ep>0$が十分小さいとき,$\grad(f+\ep g)$は消えず,したがってHamiltonian vector fieldも消えない.
        これより,解は$C_\ep$上を止まらず動き続け,周期性を持つ.
    \end{enumerate}
\end{proof}
\begin{remark*}
    すなわち,Hamiltonian flowに対して,「周期解を持つ」という性質は開な性質である.
    一方,ベクトル場に対しては同様の事実は成り立たない.
\end{remark*}

\begin{exercise}[振り子の周期の挙動]
    振り子の周期$T$を,$E\in(-1,1)$の関数として調べよ.
    \begin{enumerate}
        \item $E=-1$近傍で$T'$は消える.これを振り子の等時性という.このとき,$E\searrow-1$での漸近挙動は?
        \item $E=+1$ではHamiltonianは$f=\frac{1}{2}y^2-\frac{1}{2}x^2+1$とTaylor展開出来る.$f=1$の一般解は
        \[\begin{cases}
            x=A\sinh(t),\\
            y=B\cosh(t)
        \end{cases},\qquad A,B\in\R\]
        が与える.
        \item $E\nearrow1$での漸近挙動は?
    \end{enumerate}
\end{exercise}

\begin{exercise}[連成振動の系は多自由度系であるが可積分である]
    \[\dd{^2x_j}{t^2}=(x_{j+1}-x_j)-(x_j-x_{j-1})=x_{j+1}-2x_j+x_{j-1}\qquad(j\in[n]).\]
    なる方程式系を考える.対角成分が全て$2$で副対角成分が全て$-1$である$n$次正方行列を$A_n$とすると,
    \[\dd{^2}{t^2}\begin{pmatrix}x_1\\\vdots\\x_n\end{pmatrix}=-A_n\begin{pmatrix}x_1\\\vdots\\x_n\end{pmatrix}\]
    と表せる.$A_n$は$A_n$-型Cartan行列といい,複素単純Lie代数の分類の1つである.
    この系のHamiltonianは
    \[f(x,y):=\frac{1}{2}\sum_{i\in[n]}y_i^2+\frac{1}{2}x^\top A_nx.\]
    Cartan行列の対角化
    \[A_n=U^\top DU,\qquad D:=\diag(\lambda_1,\cdots,\lambda_n)\]
    により,変数変換を経たHamiltonianは
    \[f(x,y)=\sum_{i\in[n]}\paren{\frac{1}{2}(y'_i)^2+\frac{1}{2}\lambda_i(x_i')^2},\qquad(Ux':=x,Uy':=y).\]
    と表せ,各$(x_i,y_i)$の部分空間上で1-自由度系であり,全体としても求積可能.

    Cartan行列$A_n$の固有値$\lambda_1,\cdots,\lambda_n$を求めて,連成振動の一般解を書き下せ.
\end{exercise}
\begin{proof}
    次の補題を$a=2,b=c=-1$として用いることで,$A_n$の固有値は
    \[\lambda_j=2-2\cos\frac{j\pi}{n+1}=4\sin^2\frac{j\pi}{2(n+1)},\qquad j\in[n].\]
    と表せる.
    各自由度1のHamilton系$(\R^2,f_i)$を
    \[f_i(x_i',y_i')=\frac{(y_i')^2+\lambda_i(x_i')^2}{2}\]
    で与えると,これは調和振動を表す系で,
    \[\vctr{x_i'}{y_i'}=\mtrx{\lambda_i^{-1/2}\cos t}{\lambda_i^{-1/2}\sin t}{-\sin t}{\cos t}c,\qquad(c\in\R^2).\]
    が一般解を与える.
\end{proof}

\begin{lemma*}\footnote{\cite{Smith86-NumericalSolutionOfPDE} p.154}
    三重対角なToeplitz行列
    \[\begin{pmatrix}
        a&b\\
        c&a&b\\
        &\ddots&\ddots&\ddots\\
        &&\ddots&\ddots&\ddots\\
        &&&c&a&b\\
        &&&&c&a
    \end{pmatrix}\in M_n(\C)\]
    の固有値は
    \[\lambda_j=a+2b\sqrt{\frac{c}{b}}\cos\frac{j\pi}{n+1},\qquad j\in[n]\]
    で与えられる.
\end{lemma*}
\begin{proof}\mbox{}
    \begin{enumerate}[{Step}1]
        \item 任意の固有値$\lambda$と,これに属する零でない固有ベクトル$v=(v_1,\cdots,v_n)^\top$を取ると,
        \[Av=\lambda v\quad\Leftrightarrow\quad 
        \begin{pmatrix}
            a-\lambda&b\\
            c&a-\lambda&b\\
            &\ddots&\ddots&\ddots\\
            &&\ddots&\ddots&\ddots\\
            &&&c&a-\lambda&b\\
            &&&&c&a-\lambda
        \end{pmatrix}\begin{pmatrix}v_1\\v_2\\\vdots\\\vdots\\v_{n-1}\\v_n\end{pmatrix}=0\]
        を満たすが,これは$v_0=v_{n+1}=0$と置くと,三項間漸化式
        \[cv_{j-1}+(a-\lambda)v_j+bv_{j+1}=0,\qquad(j=1,\cdots,n).\]
        を満たす.
        \item この三項間漸化式の特性方程式
        \[c+(a-\lambda)m+bm^2=0\]
        は重根を持たないことを示す.仮に2解が$m_1=m_2$を満たすとすると,
        $v_j=(B+Cj)m_1^j\;(B,C\in\R)$は上の三項間漸化式の解であるが,
        初期条件$v_0=v_{n+1}=0$の下では$B=C=0$が必要であるから,結局$v=0$.
        これは$\lambda$を固有値,$v$をその非零固有ベクトルとした仮定に矛盾.
        \item よって,特性根を$m_1\ne m_2$とすると,
        \[v_j=Bm_1^j+Cm_2^j\qquad(j=0,1,\cdots,n,n+1).\]
        は解を与える.初期条件
        \[\begin{cases}
            0=B+C=v_0,\\
            0=Bm_1^{n+1}+Cm_2^{n+1}=v_{n+1}.
        \end{cases}\]
        から,$m_2\ne0$を仮定しても一般性を失わないことに注意して,
        \[\paren{\frac{m_1}{m_2}}^{n+1}=-\frac{C}{B}=1.\]
        が必要.特性方程式の解と係数の関係から$m_1m_2=\frac{C}{B}$と併せると,
        \[\begin{cases}
            m_1=(c/b)^{1/2}e^{\frac{is\pi}{n+1}},\\
            m_2=(c/b)^{1/2}e^{-\frac{is\pi}{n+1}}.
        \end{cases}\qquad s=1,2,\cdots,n.\]
        を得る.
        \item 再び解と係数の関係$m_1+m_2=\frac{\lambda-a}{b}$から,固有値は
        \[\lambda=a+b\sqrt{\frac{c}{b}}\paren{e^{\frac{is\pi}{n+1}}+e^{-\frac{is\pi}{n+1}}}=a+b\sqrt{\frac{c}{b}}\cos\frac{s\pi}{n+1},\qquad s=1,\cdots,n.\]
        が必要であることが従う.
    \end{enumerate}
\end{proof}

\subsection{第3回授業:変分原理の例}

\begin{exercise}[Fermatの原理からの反射の法則の導出]
    Fermatの原理を認めると,反射の法則が従う.すなわち,
    下図のように座標と$x_2,y_1,y_2>0$と$\theta_1,\theta_2\in(0,\pi/2)$を定めたとき,$\sin\theta_1=\sin\theta_2$.
    \begin{figure}[h]\centering
        \includegraphics[width=7cm]{ReflectionPrinciple.jpg}
    \end{figure}
\end{exercise}
\begin{proof}
    \[\Om:=\Brace{\gamma:\text{A,Bを結ぶ曲線}\mid\gamma\text{は}x\text{軸と交差する}}.\]
    とおいてFermatの原理を考えるが,$x$軸との交点の1つを$P$とすれば,A$\to$P, P$\to$Bの経路は直線である必要があることは明らか.
    よって,A, P, Bを結ぶ折れ線のみを考えれば十分である.その際の所要時間は,関数
    \[T(x)=\sqrt{x^2+y_1^2}+\sqrt{(x_2-x)^2+y_2^2}\]
    の定数倍になっているから,特に極値点は$T$について調べれば十分.$x\in\R_+$が極値点であるためには,
    \begin{align*}
        T'(x)=\frac{x}{\sqrt{x^2+y_1^2}}-\frac{x_2-x}{\sqrt{(x_2-x)^2+y_2^2}}=\sin\theta_1-\sin\theta_2=0
    \end{align*}
    より,$\theta_1=\theta_2$が必要.
\end{proof}

\begin{exercise}[affine空間上で微分が消えることの特徴付け]
    $A$を(有限次元とは限らない)affine空間,$f:A\to\R$を可微分関数とする.
    次は同値:
    \begin{enumerate}
        \item $df|_a=0$.
        \item 任意の方向$v\in A$について,これを接ベクトルに持つ直線$\varphi(s):=a+sv\;(s\in(-\ep,\ep))$に対して,
        \[D_\varphi f|_a:=\dd{}{s}f(\varphi(s))\biggr|_{s=0}=0.\]
    \end{enumerate}
\end{exercise}
\begin{proof}
    (1)$\Rightarrow$(2)は定義に含意されているから,(2)$\Rightarrow$(1)を示せば良い.
    任意の可微分関数$\psi:(-\ep,\ep)\to A$であって$\psi(0)=a$を満たすものを取ると,ある$v\in A$が一意的に存在して,
    \[\lim_{h\to0}\frac{\psi(h)-(\psi(0)+hv)}{h}=\lim_{h\to0}\frac{\psi(h)-\varphi(h)}{h}=0.\]
    が成り立つ.
    これを$f$の可微分性と併せると,
    \begin{align*}
        \dd{}{s}f(\psi(s))\biggr|_{s=0}&=\lim_{h\to0}\frac{f(\psi(h))-f(a)}{h}\\
        &=\lim_{h\to0}\frac{\Paren{f(\psi(h))-f(\varphi(h))}+\Paren{f(\varphi(h))-f(a)}}{h}\\
        &=0+\dd{}{s}f(\varphi(s))\biggr|_{s=0}=0.
    \end{align*}
\end{proof}

\begin{exercise}[最急降下線の変分法による導出]
    2点の最急降下線はcycloidが与える.すなわち,
    \[\Om:=\Brace{\gamma:[t_0,t_1]\to\R^2\;\middle|\;\begin{array}{l}
    \gamma(t_0)=(0,0),\gamma(t_1)=(a,b).\\
    C^1\text{級の陰関数表示}y:[a,b]\to\R_+\text{を持つ}\\
    \gamma'(t_0)=(0,0).
    \end{array}}\]
    のうち,一様重力場の下で原点からより低い点P$(a,b)$まで最も速く移動する際の運動はcycloid
    \[\gamma(t)=C\vctr{t-\frac{1}{2}\sin 2t}{\frac{1}{2}-\frac{1}{2}\cos 2t},\qquad C>0,t\in[0,t_1],Ct_1-\frac{1}{2}\sin 2t_1=a.\]
    が与える.
\end{exercise}
\begin{proof}\mbox{}
    \begin{enumerate}[{Step}1]
        \item まず,所要時間を表す汎関数$S:\Om\to\R$を求める.
        
        位置$x$での速さを$v(x)$とすると,初期条件$v(0)=0$より,エネルギー保存則から
        \[\frac{1}{2}mv(x)^2=mgy(x)\quad\Leftrightarrow\quad v(x)=\sqrt{2gy(x)}.\]
        位置$x$までの運動の軌跡の長さ$l(x)$は
        \[l(x)=\int^x_0\sqrt{1+y'(x)^2}dx\]
        である.以上から,
        \begin{align*}
            S(\gamma)&=\int^{l(a)}_0\frac{dl}{v(x)}\\
            &=\int^a_0\frac{\sqrt{1+y'(x)^2}}{v(x)}dx=\int^a_0\sqrt{\frac{1+y'(x)^2}{2gy(x)}}dx.
        \end{align*}
        改めて,$S(\gamma)$の$\sqrt{2g}$倍を$S(\gamma)$として取り直しても,極値点は変わらない.
        特に,Lagrangianは
        \[L(y,y'):=\sqrt{\frac{1+y'^2}{y}}.\]
        で与えられる.
        \item 計算
        \[\pp{L}{y}=\frac{1}{2}\paren{\frac{1+y'^2}{y}}^{-\frac{1}{2}}\paren{-\frac{1+y'^2}{y^2}}=-\frac{1}{2}\frac{(1+y'^2)^{\frac{1}{2}}}{y^{\frac{3}{2}}}\]
        \[\pp{L}{y'}=\frac{1}{2}\paren{\frac{1+y'^2}{y}}^{-\frac{1}{2}}2\frac{y'}{y}=\frac{y'}{\sqrt{y(1+y'^2)}}\]
        から,Euler-Lagrange方程式は
        \begin{align*}
            -\frac{1}{2}\frac{(1+y'^2)^{\frac{1}{2}}}{y^{\frac{3}{2}}}&=\dd{}{x}\frac{y'}{\sqrt{y(1+y'^2)}}\\
            &=\frac{y''}{\sqrt{y(1+y'^2)}}+\paren{-\frac{1}{2}}\frac{y'}{\Paren{y(1+y'^2)}^{\frac{3}{2}}}\Paren{(1+y'^2)+2yy'y''}\\
            &=\frac{y''}{\sqrt{y(1+y'^2)}}-\frac{1}{2}\frac{y'}{\sqrt{y^3(1+y'^2)}}-\frac{y'^2y''}{\sqrt{y(1+y'^2)^3}}.
        \end{align*}
        となる.両辺に$\sqrt{y(1+y'^2)}$を乗じることで
        \begin{align*}
            -\frac{1}{2}\frac{1+y'^2}{y}&=y''-\frac{1}{2}\frac{y'}{y}-\frac{y'^2y''}{1+y'^2}\\
            &=\frac{y''}{1+y'^2}-\frac{1}{2}\frac{y'}{y}
        \end{align*}
        より,
        \[-\frac{1}{2y}=\frac{y''}{1+y'^2}\quad\Leftrightarrow y'^2+2yy''+1=0\]
        の形に同値変形出来る.同値性は因子$y(1+y'^2)$が$x=0$の場合を除いて零にならないことによる.
        \item この常備分方程式には積分因子$y'$が見つかり,これを両辺に乗じることで
        \[y'+2yy'y''+y'^3=(y+yy'^2)'=0\]
        を得る.よって,この第一積分を$y+yy'^2=C>0$とおいて解を求める.
        なお,$y\ge 0$として良いから$C\ge0$が必要で,$C=0$のとき$y=0$より$\Om$の元ではない.
        この条件は正規形の常微分方程式
        \[y'=\sqrt{\frac{C-y}{y}}\]
        に帰着するが,これは変数分離型であることに注目すれば,
        \begin{align*}
            x&=\int\sqrt{\frac{y}{C-y}}dy+C'\qquad C'\in\R\\
            &=\int\frac{\sin t}{\cos t}2C\sin t\cos tdt+C'\qquad y=:C\sin^2t\\
            &=C\int(1-\cos 2t)dt+C'\\
            &=C\paren{t-\frac{\sin 2t}{t}}+C'.
        \end{align*}
        と積分出来る.$y=0$のとき$t=0$で,このとき$x=0$が必要だから,$C'=0$を得る.
        総じて,
        \[\begin{cases}
            x=Ct-\frac{C}{2}\sin 2t,\\
            y=C\sin^2t=\frac{C}{2}-\frac{C}{2}\cos 2t.
        \end{cases}\qquad C>0.\]
    \end{enumerate}
\end{proof}

\section{解析力学から幾何学へ}

\subsection{第5回授業:変分原理から見た微分幾何}

\begin{exercise}[停留曲面は極小曲面に同値\footnote{\cite{小林昭七-曲面}問2.4dを参考にした}]
    $R\subset\R^2$を曲面,$C$を$\partial R$の$\R^3$へのある滑らかな埋め込み像とし,
    \[\Om(C):=\Brace{\Sigma\subset\R^3\mid\exists_{\varphi\in C^3(R,\R^3)}\;\varphi(R)=\Sigma,\partial\Sigma=C}.\]
    とし,$S(\Sigma)$をその面積として変分問題$(\Om(C),S)$を定める.このとき,次は同値:
    \begin{enumerate}
        \item $\Sigma$は停留曲面である.
        \item $\Sigma$の平均曲率$H$は消える:$H\equiv0$.
    \end{enumerate}
\end{exercise}
\begin{proof}
    任意の曲面片$\Sigma\in\Om(C)$とそのパラメータ付け$\p(R)=\Sigma$に対して,任意に$f(\partial R)=\{0\}$を満たす可微分関数$f:R\to\R$を取る.
    $\e$を曲面片$\Sigma$上の単位法ベクトル場とし,関数$f$が定める
    曲面$\Sigma$のパラメータ付け$\p$の摂動
    \[\o{\p}:=\p+\ep f\e\qquad(\ep\in\R)\]
    を考え,$\Sigma_\ep:=\o{\p}(R)$で表す.
    補題より,$A(\ep):=S(\Sigma_\ep)$とすると,
    \[A'(0)=-\iint_RfH\,dA=-\iint_RfH\sqrt{g_{11}g_{22}-g_{12}^2}dudv\]
    が成り立つ.
    \begin{description}
        \item[(1)$\Rightarrow$(2)] $\Sigma$が停留曲面であるとき,任意の$f$と$\ep$について$A'(0)=0$が成り立つから,これは変分法の基本補題より,$H\equiv0$を意味する.
        実際,もし$H$がある正の測度を持つ集合上で非零ならば,$\varphi$を$R$内部で正で$\partial R$上で$0$となる関数として,$f$として$\varphi H$を取ると,
        \[A'(0)=-\iint_R\varphi H^2\,dA<0\]
        より矛盾.
        \item[(2)$\Rightarrow$(1)] $H\equiv0$ならば,任意の$f,\ep$について$A'(0)=0$であるから,$\Sigma$は停留曲面である.
    \end{description}
\end{proof}

\begin{lemma*}[曲面の変分に対する面積変化\footnote{\cite{佐々木}4.1節}]
    任意の曲面片$\Sigma\in\Om(C)$とそのパラメータ付け$\p(R)=\Sigma$に対して,任意に$f(\partial R)=\{0\}$を満たす可微分関数$f:R\to\R$を取る.
    $\e$を曲面片$\Sigma$上の単位法ベクトル場とし,関数$f$が定める
    曲面$\Sigma$のパラメータ付け$\p$の摂動
    \[\o{\p}:=\p+\ep f\e\qquad(\ep\in\R)\]
    を考え,$\Sigma_\ep:=\o{\p}(R)$で表す.
    いま,$A(\ep):=S(\Sigma_\ep)$とすると,
    \[A'(0)=-\iint_RfH\,dA=-\iint_RfH\sqrt{g_{11}g_{22}-g_{12}^2}dudv\]
    が成り立つ.
\end{lemma*}
\begin{proof}
    \begin{align*}
        \o{g}_{ij}&:=\o{\p}_i\cdot\o{\p}_j=(\p_i+\ep f_i\e+\ep f\e_i)\cdot(\p_j+\ep f_j\e+\ep f\e_j)\\
        &=g_{ij}+\ep f\e_i\cdot\p_j+\ep^2f_if_j\e\cdot\e+\ep f\e_i\cdot\p_j+(\ep f)^2\e_i\cdot\e_j\\
        &=g_{ij}-2\ep f\cdot h_{ij}+\ep^2(f_if_j+f^2k_{ij}).
    \end{align*}
    最後の等式は,Weingartenの式から$\e_i\cdot\p_j=-h_{ij}$であることによる.
    ただし,$\p_i,f_i,\e_i$は$i=1$のとき
    $u$での偏微分,$i=2$のとき$v$での偏微分とし,
    \[h_{ij}:=\e_i\cdot\p_j,\quad k_{ij}:=\e_i\cdot\e_j.\]
    を第二基本形式の係数と第三基本形式の係数とした.
    以上より,面積要素の変換は,Einsteinの記法に注意して,
    \begin{align*}
        \o{g}_{11}\o{g}_{22}-\o{g}_{12}^2&=\Paren{g_{11}-2\ep fh_{11}+\ep^2(f_1^2+f^2k_{11})}\Paren{g_{22}+2\ep fh_{22}+\ep^2(f^2_2+f^2k_{22})}\\
        &\qquad\qquad-\Paren{g_{12}+2\ep f_{12}+\ep^2(f_1f_2+f^2k_{12})}^2\\
        &=(g_{11}g_{22}-g_{12}^2)\Biggl(1 - 2\ep f\underbrace{\frac{h_{11}g_{22}+g_{11}h_{22}-2h_{12}g_{12}}{\abs{g_{ij}}}}_{=g^{ij}h_{ij}=2H} +4 \ep^2f^2\underbrace{\frac{\abs{h_{ij}}}{\abs{g_{ij}}}}_{=K}\\
        &\qquad\qquad+\ep^2f^2\underbrace{\frac{g_{11}k_{22}+g_{22}k_{11}-g_{12}k_{12}-g_{21}k_{21}}{\abs{g_{ij}}}}_{=g^{ij}k_{ij}}+\ep^2\frac{f_1^2g_{22}+f_2^fg_{11}-2f_1f_2g_{12}}{\abs{g_{ij}}}\Biggr)+o(\ep^3)\\
        &=(g_{11}g_{22}-g_{12}^2)\Paren{1-4\ep fH+\ep^2f^2(4K+g^{ij}k_{ij}+g^{ij}f_if_j)}+o(\ep^3)
    \end{align*}
    と変換される.
    ただし,$\abs{g_{ij}}$は行列$\mtrx{g_{11}}{g_{12}}{g_{21}}{g_{22}}$の行列式,$K$はGauss曲率とした.
    この計算から,ある$c>0$が存在して,任意の十分に小さい$\ep>0$について,
    \[\Abs{\sqrt{\o{g}_{11}\o{g}_{22}-\o{g}_{12}^2}-\sqrt{g_{11}g_{22}-g_{12}^2}(1-2\ep fH)}<c\ep^2\]
    すなわち,
    \[\Abs{A(\ep)-A(0)+2\ep\iint_RfHdA}<c'\ep^2\]
    \[\Abs{\frac{A(\ep)-A(0)}{\ep}+2\iint_RfH\,dA}<c'\ep.\]
    以上より,
    \[A'(0)=-2\iint_RfH^,dA\]
    を得た.
\end{proof}

\begin{exercise}[中心力場の特徴付け]
    任意の有界な軌道が閉曲線になる中心力場は,次の2つに限る:
    \begin{enumerate}
        \item Kepler問題:$U=-\frac{\kappa}{r}\;(\kappa>0)$.
        \item 調和振動子:$U=\frac{a}{2}r^2\;(a>0)$.
    \end{enumerate}
\end{exercise}

\subsection{第6回授業:Gauge固定とエネルギー汎関数}

\begin{exercise}[長さ汎関数のgauge変換不変性]
    $M$をRiemann多様体,$\gamma:[a,b]\to M$を
    可微分曲線,
    $f:[c,d]\iso[a,b]$を$f(a)=c,f(d)=b$を満たす可微分同相とする.
    このとき,
    \[\L(\gamma)=\L(\gamma\circ f).\]
\end{exercise}
\begin{proof}
    \[G:=\mtrx{g_{11}}{g_{12}}{g_{21}}{g_{22}}\]
    をRiemann計量の係数行列,$\wgamma:=\gamma\circ f$とすると,
    \begin{align*}
        \L(\gamma\circ f)&=\int^d_c\sqrt{(\wt{\gamma}'|G\circ\wgamma\cdot\wt{\gamma}')}ds\\
        &=\int^d_c\sqrt{(\gamma'(f(s))f'(s)|G\circ\wgamma\cdot\gamma'(f(s))f'(s))}ds\\
        &=\int^d_c\sqrt{(\gamma'(f(s))|G(\gamma(f(s)))\cdot\gamma'(f(s)))}f'(s)ds\\
        &=\int^b_a\sqrt{(\gamma'(t)|G(\gamma(t))\cdot\gamma'(t))}dt=\L(\gamma).
    \end{align*}
    ただし,$(-|-)$は各点$\gamma(t)$での接空間の内積とした.
\end{proof}

\begin{exercise}[affineパラメータへの取替]
    任意の微分が消えない曲線$\gamma:[a,b]\to M$に対して,パラメータの変換$f:[c,d]\to[a,b]$であって,
    $\gamma\circ f$がaffineパラメータになるようなものが存在する.
\end{exercise}
\begin{proof}
    $c:=0,d:=L(\gamma)$とし,
    $\varphi:[a,b]\to[c,d]$を
    \[\varphi(t):=\int^{t}_a\norm{\gamma'(t)}dt\]
    と定めると,$\varphi'(t)=\norm{\gamma'(t)}\;(t\in[a,b])$が成り立つ.
    $\gamma$は正則としたから$\varphi'>0$
    より狭義単調増加であり,
    特に可微分同相である.よって,可微分な逆$\varphi^{-1}$を持つから,これについて$\wgamma:=\gamma(\psi)$とすれば良い.
    実際,
    \[\wgamma'(s)=\gamma'(\psi(s))\psi'(s)=\frac{\gamma'(\psi(s))}{\norm{\gamma'(\psi(s))}}\qquad(s\in[c,d]).\]
    より,これは弧長が定めるパラメータである.
\end{proof}

\begin{exercise}[正規座標系の定義\footnote{\cite{Petersen16-RiemannianGeometry}命題5.5.1}]
    $n$次元
    Riemann多様体$M^n$の任意の点$x\in M$について,
    $(x,v)\in T(M)$を$t=0$に通る測地線$c_v:(-\ep,\ep)\to M$は唯一つに定まる.
    このとき,接空間内のある開集合$U\osub T_x(M)$が存在して,
    \[\xymatrix@R-2pc{
        \exp_x:U\subset T_x(M)\ar[r]&M\\
        \rotatebox[origin=c]{90}{$\in$}&\rotatebox[origin=c]{90}{$\in$}\\
        v\ar@{|->}[r]&c_v(1)
    }\]
    は$x\in M$の近傍座標を与える.
\end{exercise}
\begin{proof}\mbox{}
    \begin{description}
        \item[方針] \[O_x:=\Brace{v\in T_x(M)\mid v\text{を初速度とする測地線}c_v\text{は}[0,1]\text{の近傍で定まる}}.\]
        とすると,$0\in O_x$である.写像$\exp_x:O_x\to M$の微分の$0\in T_x(M)$での値
        \[d\exp_x:T_0(T_x(M))\to T_x(M)\]
        が非特異であることを示せば良い.
        すると,逆関数定理から$\exp_x$は局所微分同相であるから,$0\in T_x(M)$のある開近傍$U\osub T_x(M)$が存在して$\exp_x|_U$は可微分同相であり,
        $x\in\exp_x(U)$である.
        \item[準備] 任意の$\al>0$と,$c_{\al v}$の定義されている任意の$t\in\R_+$について,$c_{\al v}(t)=c_v(\al t)$が成り立つ.
        実際,$t\mapsto c_v(\al t)$は$t=0$にて$(x,\al v)$を通る$T(M)$上の曲線であるが,常微分方程式系の初期値問題の解の一意性から,これは$c_{\al v}$に等しい.
        \item[証明] よって,$O_x$の定め方から,任意の$v\in O_x$について$c_v(t)=c_{tv}(1)\;(t\in[0,1])$が成り立つ.さらに,
        \[I_0(v):=\dd{}{t}(tv)\biggr|_{t=0},\qquad(v\in T_p(M)).\]
        により定まる線型写像$T_p(M)\to T_0(T_p(M))$は同型である.これは$T_p(M)$の局所自明性から明らか.
        いま,
        \begin{align*}
            d\exp_x(I_0(v))&=\dd{}{t}\exp_x(tv)\biggr|_{t_0}&\text{写像の微分の定義}\\
            &=\dd{}{t}c_{tv}(1)\biggr|_{t_0}&\text{指数写像の定義}\\
            &=\dd{}{t}c_v(t)\biggr|_{t=0}=v.
        \end{align*}
        より,$d\exp_x\circ I_0=\id_{T_p(M)}$を得る.特に,$T_0(T_p(M))$の原点にて非特異である.
    \end{description}
\end{proof}

\subsection{第7回授業:Lie群の測地線}

\begin{exercise}[指数写像と行列の指数関数]
    Lie群$G\subset\GL_n(\C)$の測地線は$\gamma(t)=\gamma(0)e^{tX}$の形で与えられる.
\end{exercise}

\begin{exercise}[de Sitter空間の例]
    $\SL_2(\R)$のLie環は
    \[\r{sl}_2(\R):=\Brace{A\in M_2(\R)\mid\tr(A)=0}.\]
    なる3次元空間になる.一般に
    $\det(e^A)=e^{\tr(A)}$が成り立つことに注意.このとき,
    指数写像$\exp:\r{sl}_2(\R)\to\SL_2(\R)$は全射でない.
\end{exercise}
\begin{proof}
    $g\in\SL_2(\R)$であって,$\tr g<-2$を満たすものを取れば,
    これは$e^X\;(X\in \SL_2(\R))$の形では表せない.例えば
    \[g=\mtrx{-2}{0}{0}{-\frac{1}{2}}\]
    など.
    
    任意に$X\in \SL_2(\R)$を取り,その固有値を$\al,\beta$とすると,
    \[\tr(e^X)=e^\al+e^\beta\]
    が成り立つことはすぐに分かる.
    $\al,\beta$がいずれも実数である場合は,これは$0$より大きい.
    $\al,\beta$が実数でない場合は,ある$a,b\in\R$を用いて$\{\al,\beta\}=\{a\pm bi\}$と表せるが,
    $\tr(X)=0$より$a=0$が必要.
    よって,
    \[\tr(e^X)=e^{bi}+e^{-bi}=2\cos b\ge-2.\]
    いずれの場合も,$\tr(g)=\tr(e^X)<-2$としたことに矛盾.
\end{proof}

\subsection{第8回授業:Legendre変換の応用}

\begin{theorem*}[Minkowski内積の双曲角による表示\footnote{\cite{Callahan00-Spacetime}Th'm 2.8}]
    任意の未来方向の時間的事象$\r{E_1,E_2}$について,
    その双曲角を$\angle\r{E_1E_2}=:\beta$とすると,
    \[\rE_1\cdot\rE_2=\norm{\rE_1}\norm{\rE_2}\cosh\beta.\]
\end{theorem*}
\begin{proof}
    Minkowski空間を$\R^2$として考える.一般の場合も,空間変数$z$を多次元に解釈し,
    $z^2=\sum_{i\in[n]}z_i^2$と読み替えることで
    同様に示せる.
    \begin{enumerate}[{Step}1]
        \item 単位化を$\rU_i:=\frac{\rE_i}{\norm{\rE_i}}\;(i=1,2)$とすると,これは標準双曲線上の点である.
        Lorentz変換は双曲角を保つから,
        ある$u\in\R$が存在して,
        \[H_u(\rU_1)=\vctr{1}{0},\quad H_u(\rU_2)=\vctr{\cosh\beta}{\sinh\beta}.\]
        と表せる.これについて,
        \[H_u(\rU_1)\cdot H_u(\rU_2)=(1\;0)\mtrx{1}{0}{0}{-1}\vctr{\cosh\beta}{\sinh\beta}=\cosh\beta\]
        がわかり,Lorentz変換はMinkowski内積を保つから,$\rU_1\cdot \rU_2=H_u(\rU_1)\cdot H_u(\rU_2)=\cosh\beta$である.
        \item 元のベクトルについても,Minkowski内積の双線型性から,
        \[\rE_1\cdot\rE_2=(\norm{\rE_1}\rU_1)\cdot(\norm{\rE_2}\rU_2)=\norm{\rE_1\rE_2}(\rU_1\cdot \rU_2)=\norm{\rE_1\rE_2}\cosh\beta.\]
    \end{enumerate}
\end{proof}

\begin{exercise}[Minkowski空間に於ける逆向きの三角不等式]
    任意の未来方向の時間的事象$\r{E_1,E_2}$について,$\angle\r{E_1E_2}=:\beta$とする.
    \begin{enumerate}
        \item $\norm{\r{E_1+E_2}}^2=\norm{\r{E_1}}^2+\norm{\r{E_2}}^2+2\norm{\r{E_1}}\norm{\r{E_2}}$.
        \item 逆向きの三角不等式:$\norm{\r{E_1+E_2}}^2\ge\norm{\r{E_1}}+\norm{\r{E_2}}$.
    \end{enumerate}
\end{exercise}
\begin{proof}\mbox{}
    \begin{enumerate}
        \item $\rE_1+\rE_2$は再び未来方向かつ時間的事象である.実際,
        \[\rE_1=\vctr{t_1}{z_1},\quad\rE_2=\vctr{t_2}{z_2}\]
        とすると,$-(t_1+t_2)<z_1+z_2<t_1+t_2$であることが解る.
        これについて,
        \begin{align*}
            \norm{\rE_1+\rE_2}^2&=(\rE_1+\rE_2)\cdot(\rE_1+\rE_2)=\norm{\rE_1}^2+\norm{\rE_2}^2+2\rE_1\cdot\rE_2\\
            &=\norm{\rE_1}^2+\norm{\rE_2}^2+2\norm{\rE_1}\norm{\rE_2}\cosh\beta.
        \end{align*}
        \item $\cosh\beta=\frac{e^\beta+e^{-\beta}}{2}\ge\sqrt{e^\beta\cdot e^{-\beta}}=1$より,(1)から
        \begin{align*}
            \norm{\rE_1+\rE_2}&=\sqrt{\norm{\rE_1}^2+\norm{\rE_2}^2+2\norm{\rE_1}\norm{\rE_2}\cosh\beta}\\
            &\ge\sqrt{\norm{\rE_1}^2+\norm{\rE_2}^2+2\norm{\rE_1}\norm{\rE_2}}=\norm{\rE_1}+\norm{\rE_2}.
        \end{align*}
    \end{enumerate}
\end{proof}

\section{解析力学再論}

\subsection{第9回授業:Hamiltonの変分原理}

\begin{lemma*}[逆行列の微分則]
    可微分写像$x\mapsto A(x)\in\GL_n(\C)$について,
    \[(A^{-1})'=-A^{-1}A'A^{-1}\]
\end{lemma*}
\begin{proof}
    等式$AA^{-1}=I$の両辺を微分すると,
    \[A'A^{-1}+A(A^{-1})'=O\]
    より従う.
\end{proof}

\begin{exercise}[測地流の方程式のHamilton系としての特徴付け]
    Riemann多様体上の曲線について,次の2条件は同値:
    \begin{enumerate}
        \item Lagrange形式:弧長の定めるパラメータ$s$について,
        \[\dd{^2\gamma^m}{t^2}+\Gamma_{kl}^i\dd{\gamma^k}{s}\dd{\gamma^l}{s}=0,\qquad(m\in[n]).\]
        \item Hamilton形式:次の方程式を満たす
        \[\begin{cases}
            \dd{x^m}{t}=g^{mi}p_i,\\
            \dd{p^i}{t}=-\frac{1}{2}\pp{g^{lk}}{x^i}p_lp_k.
        \end{cases}\]
    \end{enumerate}
\end{exercise}
\begin{proof}
    (2)の2式を併せると,
    \begin{align*}
        \dd{^2x^m}{t^2}&=g^{mi}\dd{p^i}{t}\\
        &=g^{mi}\paren{-\frac{1}{2}\pp{g^{lk}}{x^i}p_lp_k}.
    \end{align*}
    を得る.さらに(2)の第1式は
    \[p_l=(g^{kl})^{-1}\dd{x^k}{t}\]
    に同値で,これを代入することで,
    \begin{align*}
        \dd{^2x^m}{t^2}&=\frac{1}{2}g^{mi}\paren{-\pp{g^{lk}}{x^i}(g^{kl})^{-1}(g^{lk})^{-1}}\dd{x^k}{t}\dd{x^l}{t}\\
        &=\frac{1}{2}g^{mi}\pp{g_{kl}}{x^i}\dd{x^k}{t}\dd{x^l}{t}.
    \end{align*}
    と計算を進められる.ただし,途中で逆行列の微分$(A^{-1})'=-A^{-1}A'A^{-1}$を用いた.
    最後に,$t$の代わりに弧長の定めるパラメータ$s$に取り替えることを考えると,
    \[\frac{1}{2}\pp{g_{kl}}{x^i}\dd{x^k}{t}\dd{x^l}{t}=-\Gamma^{i}_{kl}\dd{x^k}{s}\dd{x^l}{s}\dd{s}{t}\]
    以上を併せると,
    \[\paren{\dd{^2\gamma^m}{t^2}+\Gamma_{kl}^i\dd{\gamma^k}{s}\dd{\gamma^l}{s}}\dd{s}{t}=0\]
    いま$t$は正則なパラメータと考えて良いから,$\dd{s}{t}\ne0$より,結論を得る(以上の変形はいずれも逆に辿れる同値変形である).
\end{proof}

\subsection{第10回授業:Noetherの定理}

\begin{exercise}[ベクトル場の方向微分としての特徴付け\footnote{\cite{志賀浩二-多様体}定理3.2}]
    $n$次元可微分多様体$M^n$について,
    \[\Der(C^\infty(M)):=\Brace{D\in\End(C(M))\mid\forall_{f,g\in C^\infty(M)} D(fg)=fD(g)+gD(f)}.\]
    とすると,$\Der(C^\infty(M))\simeq_\Set\X(M)$である.
\end{exercise}
\begin{proof}\mbox{}
    \begin{description}
        \item[$\supset$] 任意のベクトル場$X\in\X(M)$を取ると,任意の点$x\in M$で$X_x(fg)=(X_xf)g(x)+f(x)X_xg$を満たす.
        \item[$\subset$] 
        \begin{description}
            \item[問題の所在] 任意に$D\in\Der(C^\infty(M))$を取る.任意の点$x\in M$において,対応$f\mapsto D(f)(x)$はLeibniz則を満たす線型変換だから,
            ある接ベクトル$X_x\in T_x(M)$が存在して,
            \[X_x(f)=D(f)(x)\]
            を満たす.あとは,この対応$x\mapsto X_x$が接束$T(M)$の滑らかな切断を定めることを示せば良い.
            \item[証明] 任意の$x\in M$とその近傍座標$U$を取り,係数を
            \[X_x=\sum_{\mu=1}^nX^\mu(x)\pp{}{x^\mu}.\]
            とおくと,各係数$X^\mu$は$U$上で滑らかである.実際,$U$上で座標関数$x^\mu$に一致するような,$M$上への滑らかな延長$f\in C^\infty(M)$を取ると,$D(f)\in C^\infty(M)$であるから,
            \[D(f)(x)=X_x(f)=X^\mu(x)\qquad(x\in U).\]
            より$X^\mu\in C^\infty(U)$が従う.

            同様の条件を満たすベクトル場は$X$と同じ係数を持つ必要があるから,存在と一意性が判った.
        \end{description}
    \end{description}
\end{proof}

\section{Symplectic幾何学}

\subsection{第11回授業:Symplectic幾何学}

\begin{exercise}[symplectic空間が定める複素内積空間]
    任意の$2n$次元symplectic線型空間$(V,\om)$に対して,
    ある複素構造$J^2=-\id_V$が存在して,
    $g(x,y):=\om(x,Jy)$について,
    $(V,J)$は$n$次元複素内積空間となる.
\end{exercise}
\begin{proof}\mbox{}
    \begin{description}
        \item[複素線型空間として$V$を見る] $V$の標準基底$\al_1,\cdots,\al_n,\beta_1,\cdots,\beta_n$を取ると,これについて$\om$の行列表示は
        $J_n:=\mtrx{O_n}{I_n}{-I_n}{O_n}$となる.
        これを用いて,複素構造$J:V\to V$を$-J_n$倍線型写像とする.
        このとき,$J(\al_i)=-J_n\al_i=\beta_i$に注意すれば,作用
        \[\xymatrix@R-2pc{
            \rho:\C\times V\ar[r]&V\\
            \rotatebox[origin=c]{90}{$\in$}&\rotatebox[origin=c]{90}{$\in$}\\
            (\xi+i\eta,x)\ar@{|->}[r]&\xi x+\eta J(x)
        }\]
        の$\al_i$-軌道は$\brac{\al_i,\beta_i}$という形の部分空間となる.
        よって,$V$は上記の$\rho$をスカラー倍とすることで,
        $\al_1,\cdots,\al_n$を基底とする$n$次元複素線型空間とみなせる.
        これを$V'$と表す.
        \item[$g$が内積となる] 
        \[g(x,y):=\om(x,Jy)\]
        と定めると,$g$は$V'$の内積である.
        実際,その$\al_1,\cdots,\al_n$に関する行列表示は
        \[g(\al_i,\al_j)=\om(\al_i,J\al_j)=\om(\al_i,\beta_j)=\delta_{ij}.\]
        より,$g$は確かに半正定値な対称形式である.
        半線形性は,第一引数に関する線形性は明らかで,第二引数についても,任意の$x,y\in V$と$z\in\C$について,
        \[g(x,(\xi+i\eta)y)=\om(x,J(\xi+i\eta)y)=\om(x,J\xi y)+\om(x,J^2\eta y)=\om\]
        非退化性も明らかである.
    \end{description}
\end{proof}

\begin{exercise}
    $(V^{2n},\om)$をsymplectic部分空間,$W,W_1,W_2\subset V$を線型部分空間とする.
    \begin{enumerate}
        \item $\dim W+\dim W^\perp=\dim V$.
        \item $(W^\perp)^\perp=W$.
        \item $W_1\subset W_2\Leftrightarrow W_1^\perp\supset W_2^\perp$.
        \item $(W_1+W_2)^\perp=W_1^\perp\cap W_2^\perp$.
        \item $(W_1\cap W_2)^\perp=W_1^\perp+W_2^\perp$.
    \end{enumerate}
\end{exercise}
\begin{proof}
    $V$のsymplectic基底を$e_1,\cdots,e_{2n}$とすると,これについて$\om$は行列$J_n:=\mtrx{O_n}{I_n}{-I_n}{O_n}$によって表現される.
    \begin{enumerate}
        \item $\om^\#:V\to V^*$の誘導する線型写像
        \[\xymatrix@R-2pc{
            f:V\ar[r]&W^*\\
            \rotatebox[origin=c]{90}{$\in$}&\rotatebox[origin=c]{90}{$\in$}\\
            x\ar@{|->}[r]&\om(x,-)|_{W}
        }\]
        を考えると,$\Ker f=W^\perp$である.
        またこの線型写像は全射であることは,$\om^\#$と全射$i^*:V^*\epi W^*$の合成であることによる.
        よって,$\dim V=\dim W+\dim W^\perp$.
        \item \begin{description}
            \item[$W\subset(W^\perp)^\perp$] $(W^\perp)^\perp$の元であるための必要十分条件は,$\forall_{y\in W^\perp}\;\om(x,y)=0$を満たすことであるが,
            そもそも任意の$y\in W^\perp$は任意の$x\in W$に対して$\om(x,y)=0$を満たすから,任意の$x\in W$は$(W^\perp)^\perp$の元である.
            \item[$W=(W^\perp)^\perp$] 
            (1)から
            \[\dim W+\dim W^\perp=\dim V=\dim W^\perp+\dim (W^\perp)^\perp\]
            より,$\dim W=\dim(W^\perp)^\perp$.よって特に等号が成り立つことがわかる.
        \end{description}
        \item 
        \begin{description}
            \item[$\Rightarrow$] $x\in W_2^\perp$を任意にとる.
            すると$\forall_{y\in W_2}\;\om(y,x)=0$であるから,特に$x\in W^\perp_1$.
            \item[$\Leftarrow$] $\Rightarrow$の結果と(2)から
            \[W_1=(W_1^\perp)^\perp\subset(W_2^\perp)^\perp=W_2.\]
        \end{description}
    \end{enumerate}
    \begin{description}
        \item[(4),(5)] \mbox{}
        \begin{enumerate}[{Step}1]
            \item 任意の$x+y\in W_1^\perp+W_2^\perp$を取ると,
            \[\forall_{z\in W_1\cap W_2}\;\om(x+y,z)=\om(x,z)+\om(y,z)=0.\]
            であるから,$x+y\in(W_1\cap W_2)^\perp$.
            これより$W_1^\perp+W_2^\perp\subset(W_1\cap W_2)^\perp$である.
            両辺の直交を取ると$(W_1^\perp+W_2^\perp)^\perp\subset W_1\cap W_2$であるが,
            $W_1,W_2$をその直交に取り直すことで$(W_1+W_2)^\perp\subset W_1^\perp\cap W_2^\perp$も成り立つ.
            \item 任意の$x\in(W_1+W_2)^\perp$を取ると,
            \[\forall_{y\in W_1}\;\forall_{z\in W_2}\;\om(x,y+z)=\om(x,y)+\om(x,z)=0.\]
            が成り立つ.特に$y=0,z=0$の場合をそれぞれ考えると,$x\in W_1^\perp$かつ$x\in W_2^\perp$であることがわかるから,
            $(W_1+W_2)^\perp\subset W_1^\perp\cap W_2^\perp$.
            両辺の直交を取ると$(W^\perp_1\cap W^\perp_2)^\perp\subset W_1+W_2$であり,
            $W_1,W_2$をその直交に取り直すことで$(W_1\cap W_2)^\perp\subset W_1^\perp+W_2^\perp$.
        \end{enumerate}
    \end{description}
\end{proof}

\begin{exercise}
    $\dim V=2n$,$W\subset V$を線型部分空間とする.
    \begin{enumerate}
        \item $W$がisotropicならば,$\dim W\le n$.
        \item $W$がcoisotropicならば,$\dim W\ge n$.
    \end{enumerate}
    特に,$W$がLagrangeならば,$\dim W=n$である.
\end{exercise}
\begin{proof}\mbox{}
    \begin{enumerate}
        \item $W$がisotropicとは,$W\subset W^\perp$ということである.このとき当然$\dim W\le\dim W^\perp$.
        いま,等式$\dim W+\dim W^\perp=2n$より,$2\dim W\le\dim W+\dim W^\perp=2n$.
        \item $W$がcoisotropicのとき,$W^\perp$はisotropicであるから,(1)より$\dim W^\perp\le n$.よって,$\dim W\ge n$.
    \end{enumerate}
\end{proof}

\subsection{第12回授業:Symplectic多様体}

\begin{exercise}
    $A$をsymplectic空間$(V,\om)$の自己準同型とする.
    $\det A=1$である.
\end{exercise}
\begin{proof}
    $A$が定める線型写像$f_A:V\to V$の外積
    \[\Lambda^2f_A:\Lambda^2V\to\Lambda^2V\]
    について,symplectic形式は交代形式だから,次の図式は可換になる:
    \[\xymatrix{
        \Lambda^2V\ar[dr]_-\om\ar[rr]^-{\Lambda^2f_A}&&\Lambda^2V\ar[dl]^-\om\\
        &\R
    }\]
    一方で,$p=\dim V$の場合,$\Lambda^pf_A$とは$\det A$倍写像になる.
    この2点と$\Lambda^{2n}V=\Lambda^n(\Lambda^2V)$に注意すれば,
    次の図式の大外周りを見比べることで,$\det f_A=1$.
    \[\xymatrix{
        \R\ar[r]^-1&\R\\
        \Lambda^n(\Lambda^2V)\ar[r]^-{\Lambda^n(\Lambda^2f_A)}\ar[u]^-\om\ar@{=}[d]&\Lambda^n(\Lambda^2V)\ar[u]_-\om\ar@{=}[d]\\
        \Lambda^{2n}V\ar[r]^-{\Lambda^{2n}f_A}\ar@{=}[d]&\Lambda^{2n}V\ar@{=}[d]\\
        \R\ar[r]_-{\det A}&\R
    }\]
\end{proof}



\subsection{第13回授業:Arnold-Liouvilleの定理}

\begin{exercise}
    $f$をHamiltonianとする$M$上のベクトル場を$X_f$とする.このとき,任意の$Y\in\X(M)$に対して,
    \[\om(Y,X_f)=(df|Y).\]
    が成り立つ.
\end{exercise}
\begin{proof}
    任意の点$p\in M$について,
    近傍座標$p_1,\cdots,p_n,q_1,\cdots,q_n$であってこれについて$\om$が
    \[\om=\sum_{i\in[n]}dp_i\wedge dq_i\]
    と表せるものを取ると,
    \begin{align*}
        \om_p(Y,X_f)&=\sum_{i\in[n]}(dp_i\wedge dq^i)(Y,X_f)\\
        &=\sum_{i\in[n]}\paren{\pp{f}{p_i}dp_i\wedge dq^i\paren{Y,\pp{}{q_i}-\pp{f}{q_i}dp_i\wedge dq_i\paren{Y,\pp{}{p_i}}}}\\
        &=\sum_{i\in[n]}\paren{\pp{f}{p_i}dp_i(Y)+\pp{f}{q^i}dq^i(Y)}=(df|Y).
    \end{align*}
    と各点$p\in M$上で計算できる.
\end{proof}

\bibliography{../../StatisticalSciences.bib,../../SocialSciences.bib,../../mathematics.bib,../../statistics.bib}

\end{document}

\begin{exercise}
    $A\in\Sp(V,\om)$の固有多項式$\Phi_A$は相反多項式(reciprocal polynomial)である.
\end{exercise}

\begin{exercise}[三すくみの関係]
    \begin{align*}
        \rO_{2n}(\R)\cap\Sp_{2n}(\R)&=\Sp_{2n}(\R)\cap\GL_n(\C)\\
        &=\GL_n(\C)\cap\rO_{2n}(\R)\\
        &=U_n(\C)
    \end{align*}
\end{exercise}

\begin{exercise}[Liouvilleの1-形式の同値な定義]
    $\theta\in\Om^1(M)$について,次の2条件は同値:
    \begin{enumerate}
        \item $\theta=p_1dq^1+\cdots+p_ndq^n$と表せる.
        \item 余接束$\pi:T^*(M)\to M$が引き起こす線型写像$\pi_*:T(T^*(M))\to T(M)$について,$\theta(\xi):=(y|\pi_*\xi)\;(y\in T_x^*(M),\xi\in T_y(T^*(M)))$が成り立つ.
    \end{enumerate}
\end{exercise}