\documentclass[uplatex,dvipdfmx]{jsarticle}
\title{偏微分方程式論 レポート\\1月12日発表分}\author{05-210520 司馬博文}\date{\today}
\pagestyle{plain} \setcounter{secnumdepth}{4}
\input{/Users/Hirofumi Shiba/NatureOfStatistics/preamble_no_fonts.tex}
%\input{/Users/hirofumi.shiba48/NatureOfStatistics/preamble_no_fonts.tex}
%\input{/Users/hirof/NatureOfStatistics/preamble_no_fonts.tex}
\usepackage[math]{anttor}
\begin{document}
\setcounter{section}{1}
\maketitle

\begin{problem}
    $p>0$について,$\R$上の関数を
    \[g(x):=e^{-x^p}1_{\Brace{x>0}}.\]
    と定める.
    \begin{enumerate}
        \item ある定数$\theta\in\R$が存在し,任意の$y>0,\al\in\N$について,
        \[\abs{g^{(k)}(y)}\le\frac{k!}{(\theta y)^k}e^{-\frac{y^{-p}}{2}}.\]
        \item ある定数$C,r\in\R$が存在し,任意の$x\in\R,\al\in\N$について,
        \[\abs{g^{(k)}(x)}\le C(k!)^{1+\frac{1}{p}}r^{-k}.\]
    \end{enumerate}
\end{problem}
\begin{remark}[Gervey class]
    Cauchyの積分公式がヒント.$g\in C^\infty(\Om)$が次を満たすとき,\textbf{Gervey class $\sigma$}であるという:
    任意の$\Om'\ssub\Om$に対して,ある$M,r\in\R$が存在して,任意の$y\in\Om',\beta\in\N^N$について$\abs{D^\beta g(y)}\le M(\abs{\beta}!)^\sigma r^{-\abs{\beta}}$.
    $g$が解析的であることと,Gervey class $1$であることは同値であることが知られていいる.
\end{remark}

\end{document}