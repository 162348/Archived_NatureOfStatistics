\documentclass[uplatex,dvipdfmx]{jsarticle}
\title{偏微分方程式論 レポート\\12月8日発表分}\author{05-210520 司馬博文}\date{\today}
\pagestyle{plain} \setcounter{secnumdepth}{4}
\input{/Users/Hirofumi Shiba/NatureOfStatistics/preamble_no_fonts.tex}
%\input{/Users/hirofumi.shiba48/NatureOfStatistics/preamble_no_fonts.tex}
\usepackage[math]{anttor}
\begin{document}
\setcounter{section}{1}

\begin{problem}
    対流付き拡散方程式の初期値問題
    \[\begin{cases}
        u_t=u_{xx}-u_x&\In\R\times\R^+,\\
        u=\varphi&\on\R\times\{0\}.
    \end{cases}\qquad \varphi\in L^\infty(\R)\]
    を考える.
    \begin{enumerate}
        \item $v(x,t):=u(x+t,t)$の満たすべき方程式を求めよ.
        \item $u:=\psi w$によって定めた$w$が熱方程式を満たすような$\psi(x,t)$を見つけよ.
    \end{enumerate}
\end{problem}
\begin{proof}[\bf\underline{[解]}]\mbox{}
    \begin{enumerate}
        \item 
    \end{enumerate}
\end{proof}

\begin{problem}
    非斉次な熱方程式の初期値境界値問題
    \[\begin{cases}
        u_t=u_{xx}+f(x,t)&\In\R^+\times\R^+,\\
        u=h&\on\{0\}\times\R^+,\\
        u=\varphi&\on\R^+\times\{0\}.
    \end{cases}\qquad f\in C^1_b(\R_+\times\R_+),h\in C^1_b(\R_+),\varphi\in C_b(\R_+),h(0)=\varphi(0).\]
    を考える.
    $v(x,t):=u(x,t)-h(t)$を考えよ.
\end{problem}
\begin{proof}[\bf\underline{[解]}]
    
\end{proof}

\begin{problem}
    方程式$u_t=xu_{xx}$を考える.
    \begin{enumerate}
        \item $u(x,t)=-2xt-x^2$は$u_t=xu_{xx}$を満たす.
        \item $u$の$R:=[-2,2]\times[0,1]$上での最大値を求めよ.
        \item $u$は放物型境界$([-2,2]\times\{0\})\cup(\{-2,2\}\times[0,1])$上で最大値を達成するか?
    \end{enumerate}
\end{problem}
\begin{proof}[\bf\underline{[解]}]
    
\end{proof}

\begin{problem}
    退化放物型方程式
    \[u_t=(xu_x)_x\quad\In\R^+\times\R^+\]
    を考える.
    \begin{enumerate}
        \item 任意の解$u$と$\lambda>0$について,$v(x,t):=u(\lambda^\al x,\lambda t)$も解になる.
        \item $u(x,t):=t^{-1}\varphi(t^{-\al}x)$が解になるような関数$\varphi$はどんな微分方程式を満たすか?
        \item $\varphi(0)=1$を満たす解により,自己相似解であって$u(0,t)=t^{-1}\;(t>0)$を満たすものを得よ.
    \end{enumerate}
\end{problem}
\begin{proof}[\bf\underline{[解]}]\mbox{}
    \begin{enumerate}
        \item 方程式の$(x,t)$に$(\lambda^\al x,\lambda t)$を代入すれば,$u_t(\lambda^\al x,\lambda t)=u_x(\lambda^\al x,\lambda t)+\lambda^\al xu_{xx}(\lambda^\al x,\lambda t)$をみたす.
        また,\[v_t(\lambda^\al x,\lambda t)=\lambda u_t(\lambda^\al x,\lambda t)=\lambda u_x+\lambda^{\al+1}u_{xx}=\lambda^{1-\al}v_x+\lambda^{1-\al}xv_{xx}.\]
        より,$\al=1$が必要.
        \item $u(x,t)=\frac{1}{t}\varphi\paren{\frac{x}{t}}$を微分すると,$u_t=u_x+xu_{xx}$は
        \[-\frac{1}{t^2}\varphi\paren{\frac{x}{t}}-\frac{x}{t^3}\varphi'\paren{\frac{x}{t}}=\frac{1}{t^2}\varphi'\paren{\frac{x}{t}}+\frac{x}{t^3}\varphi''\paren{\frac{x}{t}}\]
        となる.$s:=x/t$とすると,
        \[s\varphi''+(1+s)\varphi'+\varphi=0.\]
        \item 特性関数$st^2+(1+s)t+1=0$からの類推から,$\varphi(x)=e^{-x}$が解の1つだと予想でき,実際その通りである.
        よって,$u(x,t):=\frac{1}{t}e^{-\frac{x}{t}}$は1つの自己相似解である.
    \end{enumerate}
\end{proof}
\begin{remark*}[途中で出現したKummerの微分方程式について]
    途中の変数係数ODE
    \[s\varphi''+(1+s)\varphi'+\varphi=0.\]
    は$s=0,\infty$に特異点を持っている.この解空間は1次元と決まっているのであろうか?
    実は特異点が2つしかないように,Kummerの方程式
    \[tx''+(b-t)x'-ax=0\qquad(a,b\in\C)\]
    と関係が深く,
    一般解は$e^{-s}$と$e^{-s}\Ei(s)$で張られるらしい.
    Eiは指数積分というが,$0,\infty$に分岐点を持つ(したがって問題文の中の条件$\varphi(0)=1$は解を一意に定めている).
    Eiを複素関数とみるときは適切な分枝を取って$E_1$と表され,$-E_1(x)=\Ei(x)\;(x>0)$の関係を持つ.
    $b=1,a=0$としたKummerの微分方程式も通常は合流型超幾何関数$M$と$U$で張られる2次元の解空間を持つが,
    $E_1(-z)$も解にもつ.合流型超幾何関数と
    \[E_1(z)=e^{-z}U(1,1;z)\]
    の関係にある.
\end{remark*}


\begin{problem}[熱方程式の球面平均定理]
    $\Om\subset\R^n$を領域,$Q_T:=Q\times{(0,T]}\;(T>0)$とし,$u\in C^{2,1}(Q_T)$は$u_t-\Lap u=0\;\In Q_T$を満たすとする.
    このとき,任意の熱球$E(x,t;r)\ssub Q_T$に対して,次が成り立つ:
    \[u(x,t)=\frac{1}{4r^n}\iint_{E(x,t;r)}u(y,s)\frac{\abs{x-y}^2}{(t-s)^2}dyds.\]
\end{problem}
\begin{proof}
    熱球$E(x,t;r)\ssub Q_T$を任意にとる.すると,平行移動した関数$v(y,s):=u(y+x,s+t)$を考えると,
    \[(y+x,s+t)\in E(x,t;r)\quad\Leftrightarrow\quad(y,s)\in E(0,0;r)\]
    に注意すれば,
    \[v(x,t)=\frac{1}{4r^n}\iint_{E(0,0;r)}v(y,s)\frac{\abs{x-y}^2}{(t-s)^2}dyds\]
    を示せばよい.以降,$v$を$u$と書き,$E(0,0;r)$を$E(r)$と書く.
    \begin{enumerate}[{Step}1]
        \item 関数
        \[\phi(r):=\frac{1}{r^n}\iint_{E(r)}u(y,s)\frac{\abs{y}^2}{s^2}dyds\]
        の微分を考える.まず,
        \[r^n\Phi(y,s)=\frac{1}{\sqrt{4\pi(s/r^2)}^n}e^{-\frac{\abs{y/r}^2}{s/r^2}}=\Phi(y/r,s/r^2)\]
        に注意すれば,変数変換$x:=y/r,t:=s/r^2$により
        \begin{align*}
            \phi(r)&=\frac{1}{r^n}\iint_{E(1)}u(rx,r^2t)\frac{r^2\abs{x}^2}{r^4t^2}r^ndxr^2dt=\iint_{E(1)}u(rx,r^2t)\frac{\abs{x}^2}{t^2}dxdt.
        \end{align*}
        と書き直せる.するとこの微分は
        \begin{align*}
            \phi'(r)&=\iint_{E(1)}\paren{\sum_{i=1}^nu_{x_i}x_i\frac{\abs{x}^2}{t^2}+2ru_t\frac{\abs{x}^2}{t}}dxdt\\
            &=\frac{1}{r^{n+2}}\iint_{E(r)}\paren{\sum_{i=1}^nu_{y_i}\frac{y_i}{r}\frac{r^2\abs{y}^2}{s^2}+2ru_s\frac{\abs{y}^2}{s}}dyds\\
            &=\frac{1}{r^{n+1}}\iint_{E(r)}\paren{\sum_{i=1}^nu_{y_i}y_i\frac{\abs{y}^2}{s^2}+2u_s\frac{\abs{y}^2}{s}}dyds=:A+B.
        \end{align*}
        と計算できる.
        \item まず$B$を評価することを考える.いま,
        \[\partial E(r)=\Brace{(y,s)\in\R^{n+1}\;\middle|\;s\le0,\Phi(-y,-s)=\frac{1}{r^n}}\]
        上では
        \[\Phi(-y,-s)=\frac{1}{(-4\pi s)^{n/2}}e^{\frac{\abs{y}^2}{4s}}=\frac{1}{r^n}\]
        より,特に
        \[\psi(y,s):=\log\Phi(-y,-s)-\log r^{-n}=-\frac{n}{2}\log(-4\pi s)+\frac{\abs{y}^2}{4s}+n\log r=0,\qquad(y,s)\in\partial E(r)\]
        に注目する.
        以降,
        \[\psi_s(y,s)=-\frac{n}{2}\frac{1}{s}-\frac{\abs{y}^2}{4s^2},\quad\psi_{y_i}(y,s)=\frac{y_i}{2s}.\]
        を用いる.
        $y_i$に関する
        部分積分により,$B$は
        \begin{align*}
            B&=\frac{1}{r^{n+1}}\iint_{E(r)}\paren{\sum_{i=1}^n2u_s\frac{y_i^2}{s}}dyds\\
            &=\frac{1}{r^{n+1}}\iint_{E(r)}\paren{4\sum_{i=1}^nu_sy_i\psi_{y_i}}dyds\\
            &=-\frac{1}{r^{n+1}}\iint_{E(r)}4\sum_{i=1}^n\paren{u_{sy_i}y_i\psi+_s\psi}dyds\\
            &=-\frac{1}{r^{n+1}}\iint_{E(r)}\paren{4nu_s\psi+4\sum_{i=1}^nu_{sy_i}y_i\psi}dyds.
        \end{align*}
        と変形できる.引き続き第二項に,$s$に関する部分積分を考えることで,
        \begin{align*}
            B&=\frac{1}{r^{n+1}}\iint_{E(r)}\paren{-4nu_s\psi+4\sum_{i=1}^nu_{y_i}y_i\psi_s}dyds\\
            &=\frac{1}{r^{n+1}}\iint_{E(r)}\paren{-4nu_s\psi+4\sum_{i=1}^nu_{y_i}y_i\paren{-\frac{n}{2s}-\frac{\abs{y}^2}{4s^2}}}dyds\\
            &=\frac{1}{r^{n+1}}\iint_{E(r)}\paren{-4nu_s\psi-\frac{2n}{s}\sum_{i=1}^nu_{y_i}y_i}dyds-\underbrace{\frac{1}{r^{n+1}}\iint_{E(r)}\frac{\abs{y}^2}{s^2}\sum_{i=1}^nu_{y_i}y_idyds}_{=A}
        \end{align*}
        \item 以上の計算と,$u$が熱方程式の解であることを併せれば,部分積分から
        \begin{align*}
            \psi'(r)&=A+B=\frac{1}{r^{n+1}}\iint_{E(r)}\paren{-4nu_s\psi-\frac{2n}{s}\sum_{i=1}^nu_{y_i}y_i}dyds\\
            &=\frac{1}{r^{n+1}}\iint_{E(r)}\paren{-4n\underbrace{\Lap u\psi}_{=\sum_{i=1}^nu_{y_iy_i}\psi}-\frac{2n}{s}\sum_{i=1}^nu_{y_i}y_i}dyds\\
            &=\frac{1}{r^{n+1}}\iint_{E(r)}\sum_{i=1}^n\paren{4nu_{y_i}\psi_{y_i}-\frac{2n}{s}u_{y_i}y_i}dyds\\
            &=\frac{1}{r^{n+1}}\sum_{i=1}^n\iint_{E(r)}u_{y_i}\paren{4n\psi_{y_i}-\frac{2n}{s}y_i}dyds=0.
        \end{align*}
        が解る.よって関数$\psi$は定数であるから,次の補題より,
        \begin{align*}
            \phi(r)&=\lim_{r\to0}\phi(r)=\lim_{r\to0}\frac{1}{r^n}\iint_{E(r)}u(y,s)\frac{\abs{y}^2}{s^2}dyds\\
            &=\lim_{r\to0}\iint_{E(1)}u(rx,r^2t)\frac{\abs{x}^2}{t^2}dxdt\\
            &=u(0,0)\iint_{E(1)}\frac{\abs{x}^2}{t^2}dxdt=4u(0,0).
        \end{align*}
        を得る.途中の収束は,$E(1)$がコンパクトであることに注意すれば,
        $\{u(rx,r^2t)\}_{0\le r\le 1}$がすべて可積分であるため,Lebesgueの優収束定理による.
    \end{enumerate}
\end{proof}

\begin{lemma*}
    \[\iint_{E(1)}\frac{\abs{x}^2}{t^2}dxdt=4.\]
\end{lemma*}
\begin{proof}
    \[E(1)=\Brace{(y,s)\in\R^{n+1}\;\middle|\;s\le0,\Phi(-y,-s)\ge1}\]
    であるが,
    \begin{align*}
        \Phi(-y,-s)&=(-4\pi s)^{-\frac{n}{2}}e^{\frac{\abs{y}^2}{4s}}\ge1\\
        &\Leftrightarrow e^{\frac{\abs{y}^2}{4s}}\ge(-4\pi s)^{\frac{n}{2}}\\
        &\Leftrightarrow \frac{\abs{y}^2}{4s}\ge\frac{n}{2}\log(-4\pi s)\\
        &\Leftrightarrow\abs{y}^2\le 2ns\log(-4\pi s).
    \end{align*}
    と同値変形できる.$e^{\frac{\abs{y}^2}{4s}}\le1$に注意すれば,$s\in\Square{-\frac{1}{4\pi},0}$が必要であるため,特に$E(1)$はコンパクトで,次のように表せる:
    \[E(1)=\Brace{(y,s)\in\R^{n+1}\;\middle|\;-\frac{1}{4\pi}\le s\le0,\abs{y}^2\le 2ns\log(-4\pi s)}.\]
    $R:=\sqrt{2ns\log(-4\pi s)}$とおく.
    よって,
    積分は次のように計算できる:
    \begin{align*}
        \iint_{E(1)}\frac{\abs{y}^2}{s^2}dyds&=\int^0_{-1/4\pi}\int^R_0\int_{\partial B(0,r)}\frac{r^2}{s^2}dSdrds\\
        &=\int^0_{-1/4\pi}\int^R_0\frac{n\om_nr^{n+1}}{s^2}drds&\because\abs{\partial B(0,r)}=r^{n-1}n\om_n\\
        &=\int^0_{-1/4\pi}\frac{n\om_n}{s^2}\frac{R^{n+2}}{n+2}ds\\
        &=\frac{2n}{n+2}(2n)^{n/2}n\om_n\int^0_{-1/4\pi}s^{\frac{n-2}{2}}(\log(-4\pi s))^{\frac{n+2}{2}}ds.
    \end{align*}
    ここで,変数変換$z:=-\log(-4\pi s)$より,
    \begin{align*}
        \int^0_{-1/4\pi}s^{\frac{n-2}{2}}(\log(-4\pi s))^{\frac{n+2}{2}}ds&=\int^\infty_0\paren{\frac{e^{-z}}{4\pi}}^{\frac{n}{2}}z^{\frac{n+2}{2}}dz\\
        &=\frac{1}{(4\pi)^{\frac{n}{2}}}\int^\infty_0e^{-w}w^{\frac{n}{2}+2-1}\paren{\frac{2}{n}}^{\frac{n}{2}+2}dw\\
        &=\frac{1}{(2n\pi)^{n/2}}\frac{4}{n^2}\Gamma\paren{\frac{n}{2}+2}.
    \end{align*}
    と計算できるから,
    $n$次元単位球の表面積の関係
    \[n\om_n=\frac{2\pi^{n/2}}{\Gamma\paren{\frac{n}{2}}}\]
    に注意すれば,総じて,
    \begin{align*}
        \iint_{E(1)}\frac{\abs{y}^2}{s^2}dyds&=\frac{2n}{n+2}(2n)^{n/2}n\om_n\frac{1}{(2n\pi)^{n/2}}\frac{4}{n^2}\Gamma\paren{\frac{n}{2}+2}\\
        &=\frac{2n}{n+2}2\frac{4}{n^2}\frac{n+2}{2}\frac{n}{2}=4.
    \end{align*}
\end{proof}
\begin{problem}[強最大値原理]
    $\Om\subset\R^n$を領域,$Q_T:=Q\times{(0,T]}\;(T>0)$とし,$u\in C^{2,1}(Q_T)$は$u_t-\Lap u=0\;\In Q_T$を満たすとする.
    このとき,ある点$(x_0,t_0)\in Q_T$において$u(x_0,t_0)=\max_{\o{Q_T}}u$を満たすならば,$u$は$\o{Q_{t_0}}$上定数であることを示せ.
\end{problem}
\begin{proof}
    点$(x_0,t_0)\in Q_T$にて$u(x_0,t_0)=\max_{\o{Q_T}}u=:M$を達成するとする.
    \begin{enumerate}[{Step}1]
        \item 任意の$E(x_0,t_0;r)\subset Q_T$について,平均値の定理から
        \[M=u(x_0,t_0)=\frac{1}{4r^n}\iint_{E(x_0,t_0;r)}u(y,s)\frac{\abs{x_0-y}^2}{(t_0-s)^2}dyds\]
        が成り立つが,
        補題より,
        \[\frac{1}{4r^n}\iint_{E(x_0,t_0;r)}\frac{\abs{x_0-y}^2}{(t_0-s)^2}dyds=1\]
        であることから,
        \[\iint_{E(x_0,t_0;r)}\Paren{u(y,s)-M}\frac{\abs{x_0-y}^2}{(t_0-s)^2}dyds=0.\]
        $u$が連続であることと併せれば,$u\equiv M\;\on E(x_0,t_0;r)$を得る.
        \item $(y_0,s_0)\in Q_T$で$s_0<t_0$を満たす点と$(x_0,t_0)$とを結ぶ線分$L$は$Q_T$に含まれるとすると,$u\equiv M\;\on L$である.
        
        実際,
        \[r_0:=\min\Brace{s\ge s_0\mid \forall_{(x,t)\in L}\;s\le t\le t_0\Rightarrow u(x,t)=M}.\]
        が$r_0>s_0$を満たすとすると矛盾が導ける.$u$は連続であるから,$r_0$の定義でminを取っている所の集合は閉集合である.特に,$r_0$について,
        $(z_0,r_0)\in L$を満たす任意の$z_0\in Q$について,$u(z_0,r_0)=M$を満たす.
        よってStep1から,任意の$E(z_0,r_0;r)\subset Q_T$について,$u(z_0,r_0)\equiv M\;\on E(z_0,r_0;r)$を満たす.
        このとき,$E(z_0,r_0;r)$はある$\ep>0$について$L\cap\Brace{r_0-\ep\le t\le r_0}$という形の集合を含むから,$r_0$の最小性に矛盾する.
        \item あとは,任意の点$(x,t)\in Q\times\cointerval{0,T}$が$(x_0,t_0)$と線分の有限個のつなぎ合わせによって結べることを示せばよい.
        
        まず,$Q$が連結であるために,ある有限個の点$x_0,x_1,\cdots,x_m=x$であって,これらを結ぶ折れ線は$Q$内に存在するように出来る.
        これは,この性質を持つ$Q$の点$x\in Q$の全体$V\subset Q$は,非空の,$Q$の開かつ閉集合であるためである.
        $V$が開集合であることは明らか.閉集合であることも,$V$の任意の収束列$\{x_n\}\subset V$に対して,$Q$内の開球$B\osub Q$であって$x_n$を無限個含むものが取れるから,$n_0:=\min\Brace{n\in\N\mid x_n\in B}$とすれば,$[x_0,x_{n_0}],[x_{n_0},x_\infty]$は$x_0$と$x_\infty:=\lim_{n\to\infty}x_n$を結ぶから,$x_\infty\in V$である.

        これに対して,対応するだけの$t_0>t_1>\cdots>t_m=t$を任意に取れば,$(x_0,t_0),(x_1,t_1),\cdots,(x_m,t_m)$を結ぶ折れ線は$Q_T$に含まれる.
    \end{enumerate}
\end{proof}

\begin{problem}[時間後方一意性]
    $Q\subset\R^n$を有界領域,$Q_T:=Q\times{(0,T]}\;(T>0)$とし,$u,v\in C^{2,1}(Q_T)$は
    いずれも熱方程式
    $u_t-\Lap u=0\;\In Q_T$を満たすとする.
    このとき,$u=v\;\on\partial Q\times(0,T]$ならば,次が成り立つ:
    \begin{enumerate}
        \item $w:=u-v,E(t):=\int_\Om w^2dx$とおくと,$\forall_{t\in(0,T)}\;(E'(t))^2\le E(t)E''(t)$.
        \item $u(-,T)=v(-,T)\;\on\o{Q}$ならば,$u=v\;\on\o{Q_T}$.
    \end{enumerate}
\end{problem}
\begin{proof}\mbox{}
    \begin{enumerate}
        \item 
        \begin{description}
            \item[2階微分の計算] \[E(t):=\int_Qw^2(x,t)dx,\qquad t\in[0,T].\]
            の微分は,部分積分より
            \begin{align*}
                \dot{E}(t)&=2\int_Qww_tdx\\
                &=2\int_Qw\Lap wdx=-2\int_Q\abs{Dw}^2dx.
            \end{align*}
            さらにもう一度微分すると,
            \begin{align*}
                \ddot{E}(t)&=-4\int_Q(Dw|Dw_t)dx\\
                &=4\int_Q\Lap ww_tdx=4\int_Q(\Lap w)^2dx.
            \end{align*}
            \item[2階微分の評価] 任意の$t\in[0,T]$について$w(-,t)=0\;\on\partial Q$より,部分積分とCauchy-Schwarz不等式から
            \begin{align*}
                (\dot{E}(t))^2&=4\paren{\int_Q\abs{Dw}^2dx}^2=4\paren{\int_Qw\Lap wdx}^2\\
                &\le\paren{\int_Qw^2dx}\paren{4\int_Q(\Lap w)^2dx}=E(t)\ddot{E}(t).
            \end{align*}
        \end{description}
        \item $E\equiv0$を示せば良いから,ある$[t_1,t_2]\subset[0,T]\;(t_1<t_2)$が存在して$\forall_{t\in\cointerval{t_1,t_2}}\;E(t)>0$かつ$E(t_2)=0$を満たすと仮定して矛盾を導く.
        \[f(t):=\log E(t),\qquad t\in\cointerval{t_1,t_2}.\]
        を考えると,(1)での議論より,
        \[f''(t)=\frac{\ddot{E}(t)}{E(t)}-\frac{\dot{E}(t)^2}{E(t)^2}\ge0.\]
        よって,$f$は$(t_1,t_2)$上凸である:
        \[f((1-\tau)t_1+\tau t)\le(1-\tau)f(t_1)+\tau f(t),\qquad t\in(t_1,t_2),\tau\in(0,1).\]
        すなわち,
        \[E((1-\tau)t_1+\tau t)\le E(t_1)^{1-\tau}E(t)^{\tau},\qquad t\in(t_1,t_2),\tau\in(0,1).\]
        であるが,$t=t_2$と取ると,特に
        \[(0\le)E((1-\tau)t_1+\tau t_2)\le E(t_1)^{1-\tau}E(t_2)^\tau=0,\qquad\tau\in(0,1).\]
        より,$E=0\;\on[t_1,t_2]$.
    \end{enumerate}
\end{proof}

\begin{problem}[正則性]
    $\Om\subset\R^n$を領域,$Q_T:=Q\times{(0,T]}\;(T>0)$とし,$u\in C^{2,1}(Q_T)$は$u_t-\Lap u=0\;\In Q_T$を満たすとする.
    このとき,次が成り立つ:
    \begin{enumerate}
        \item $u\in C^\infty(Q_T)$.
        \item 任意の
        \[C(x,t;r):=\Brace{(y,s)\in\R^n\times\R\mid\abs{x-y}\le r,t-r^2\le s\le t}\subset Q_T\]
        と$k,l\in\N$に対して,ある定数$C\ge0$が存在して,
        \[\max_{(y,s)\in C\paren{x,t;\frac{r}{2}}}\Abs{D^k_xD^l_tu(y,s)}\le\frac{C}{r^{k+2l+n+2}}\norm{u}_{L^1(C(x,t;r))}.\]
    \end{enumerate}
\end{problem}
\begin{proof}\mbox{}
    \begin{enumerate}
        \item 任意の$(x_0,t_0)\in Q_T$と$C:=C(x_0,t_0;r)\subset Q_T$をとる.
        $(\eta_\ep)_{\ep>0}$を$(x,t)\in\R^{n+1}$上の軟化子とし,
        $u^\ep:=\eta_\ep*u$と表す.
        $n$次元閉球を$B(x,r):=\Brace{y\in\R^n\mid\abs{y-x}\le r}$と表す.
        $C':=C(x_0,t_0;3r/4)$上で$1$で,$C$の放物境界$\partial B(x_0,r)\times[t_0-r^2,t_0]\cup B(x_0,r)\times\{t_0-r^2\}$の近傍で$0$な,$C$上に台を持つ可微分関数$\zeta$を取り,
        これを用いてカットオフをかけたものを
        \[v^\ep(x,t):=\zeta(x,t)u^\ep(x,t),\qquad x\in\R^n,t\in[0,t_0]\]
        と表す.
        \begin{enumerate}[{Step}1]
            \item いま,
            \[v_t^\ep=\zeta u^\ep_t+\zeta_tu^\ep,\quad\Lap v^\ep=\zeta\Lap u^\ep+2(D\zeta|Du^\ep)+u^\ep\Lap\zeta\]
            であるから,
            \[\begin{cases}
                v^\ep_t-\Lap v^\ep=\zeta_tu^\ep-2(D\zeta|Du^\ep)-u^\ep\Lap\zeta=:\wt{f}&\In\R^n\times(0,t_0),\\
                v^\ep=0&\on\R^n\times\{0\}.
            \end{cases}\]
            を満たす.
            $u^\ep$は$Q_T$の近傍で可微分であるために,解公式を用いるための$f$の可微分性の条件はみたされており,また$\zeta$の存在よりコンパクトな台を持つ.
            さらに$v^\ep$はこの有界な解であるから,一意性から
            \[v^\ep(x,t)=\int^t_0\int_{\R^n}\Phi(x-y,t-s)\wt{f}(y,s)dyds\]
            を結論付けられる.
            \item 任意の$(x,t)\in C'':=C(x_0,t_0;r/2)$について,十分小さい$\ep>0$についてこの上で$u^\ep=v^\ep$であるから,
            \begin{align*}
                u^\ep(x,t)&=\iint_C\Phi(x-y,t-s)\wt{f}(y,s)dyds\\
                &=\iint_C\Phi(x-y,t-s)\Paren{(\zeta_s(y,s)-\Lap\zeta(y,s))u^\ep(y,s)-2(D\zeta(y,s)|Du^\ep(y,s))}dyds.
            \end{align*}
            $\Phi(x-y,t-s)$は$(y,s)=(x,t)$で特異性を持つが,
            $(x,t)\in C''$の近傍で$\wh{f}=v_t^\ep-\Lap v^\ep=Q_T^\ep-\Lap u^\ep=0$より,部分積分によって次のように計算を進めることができる:
            \begin{align*}
                u^\ep(x,t)&=\iint_C\Phi(x-y,t-s)(\zeta_s-\Lap\zeta)u^\ep dyds\\
                &\qquad\quad+2\iint_C\Paren{(D_y\Phi(x-y,t-s)|D\zeta)u^\ep+\Phi(x-y,t-s) u^\ep\Lap\zeta} dyds\\
                &=\iint_C\Paren{\Phi(x-y,t-s)(\zeta_s+\Lap\zeta)+2(D_y\Phi(x-y,t-s)|D\zeta)}u^\ep dyds.
            \end{align*}
            $\ep\to0$の極限を取ることで,
            \[u(x,t)=\iint_C\Paren{\Phi(x-y,t-s)(\zeta_s+\Lap\zeta)+2(D_y\Phi(x-y,t-s)|D\zeta)}u dyds.\]
            を得る.
            \item Step2で得た$C''$上での$u$の表示の積分核
            \[K(x,t,y,s):=\Phi(x-y,t-s)(\zeta_s+\Lap\zeta)+2(D_y\Phi(x-y,t-s)|D\zeta)\]
            は$C'$上で$0$で,$C$上で可微分である.よって,$u$は$C''$上で可微分である.
        \end{enumerate}
        \item $(x,t)=(0,0)$と仮定して議論する.一般の$(x,t)\in Q_T$については,$v(y,s):=u(y+x,s+t)$についての議論の帰着させることが出来る.
        \begin{enumerate}[{Step}1]
            \item $C(1):=C(0,0,1)\subset Q_T$が成り立つとする.このとき,(1)と同様にして
            \[u(x,t)=\iint_{C(1)}K(x,t,y,s)u(y,s)dyds\qquad(x,t)\in C(1/2).\]
            という表示を得る.$K\in C_c^\infty$より,この微分は$C(1/2)$上で
            \[\abs{D_x^kD_t^lu(x,t)}\le\iint_{C(1)}\abs{D_x^kD_t^lK(x,t,y,s)}\abs{u(y,s)}dyds\le C\norm{u}_{L^1(C(1))}.\]
            と評価できる.
            \item 任意の$C(r)\subset Q_T$を取ると,
            \[v(x,t):=u(rx,r^2t)\]
            は$C(1)$の近傍で定義されており,$C(1)$上の熱方程式を満たす.
            よって,Step1での議論から,
            \[\abs{D_x^kD_t^lv(x,t)}\le C\norm{v}_{L^1(C(1))},\qquad (x,t)\in C(1/2).\]
            という評価を得る.
            関係
            \[D_x^kD_t^lv(x,t)=r^{2l+k}D_x^kD_t^lu(rx,r^2t).\]
            \[\norm{v}_{L^1(C(1))}=\iint_{C(1)}\abs{u(rx,r^2t)}dxdt=\frac{1}{r^{n+2}}\iint_{C(r)}\abs{u(y,s)}dyds=\frac{1}{r^{n+2}}\norm{u}_{L^1(C(r))}.\]
            に注意して,結論を得る.
        \end{enumerate}
    \end{enumerate}
\end{proof}


\bibliography{../../StatisticalSciences.bib,../../SocialSciences.bib,../../mathematics.bib,../../statistics.bib}

\end{document}

