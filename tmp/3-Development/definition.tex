\documentclass[uplatex,dvipdfmx]{jsarticle}
\title{経済安全保障プログラム\\Webサイト案}
\author{司馬博文,富澤新太郎}
\date{\today}
\pagestyle{headings} \setcounter{secnumdepth}{4}
\input{/Users/Hirofumi Shiba/NatureOfStatistics/preamble_no_fonts.tex}
%\input{/Users/hirofumi.shiba48/NatureOfStatistics/preamble_no_fonts.tex}
%\input{/Users/hirof/NatureOfStatistics/preamble_no_fonts.tex}
\usepackage[math]{anttor}
\begin{document}
\maketitle

\section{予算と納期}

\begin{description}
    \item[予算] 150万$\pm$50万円.
    \item[締め切り] 3月末頃の公開が目標.
\end{description}

\section{必須要素}

\begin{tcolorbox}[colframe=ForestGreen, colback=ForestGreen!10!white,breakable,colbacktitle=ForestGreen!40!white,coltitle=black,fonttitle=\bfseries\sffamily,
title=]
    絶対に譲れない3要素をまず挙げる.
    続いて第3節で(1)について,第4節で(2),(3)を詳述し,
    サイトの全体像を見る.
\end{tcolorbox}

\begin{enumerate}
    \item 洗練された印象を持つランディングページを持ち,日本の他のシンクタンクとはっきり差別化できること.
    \item 研究者の一覧ページと個別ページが統一されたフォーマットを持ち,比較・一覧しやすいこと.
    \item 記事のタグ付け機能:前述した研究者の一覧と対応が付き,簡単に往来できるタグ付け機能を持つこと.さらに,記事の「ジャンル」のタグが豊富にあり,これを用いても各記事・イベントと研究者の繋がりが可視化出来る機能を備えていること.
\end{enumerate}

次に重要な点として,上述の3点にあと2点付け加えるならば,次の2点が挙げられる:
\begin{enumerate}\setcounter{enumi}{3}
    \item 多言語対応していること:優先するのは英語ページであるが,理想的にはサイト上部のどこかに配置されたボタンによって日本語・英語の切り替えが出来ることが望ましい.
    \item 写真・ロゴの意匠性が高いこと:\href{https://www.cfr.org/}{CFR}, \href{https://www.aspi.org.au/}{ASPI}, \href{https://www.brookings.edu/}{Brookings}のどれを挙げても,どの写真も光度が一様でなく明暗がはっきりしており,写真に立体感がある.
\end{enumerate}

\section{デザインの基本精神と参考サイト}

\begin{tcolorbox}[colframe=ForestGreen, colback=ForestGreen!10!white,breakable,colbacktitle=ForestGreen!40!white,coltitle=black,fonttitle=\bfseries\sffamily,
title=基本哲学]
    日本の研究所の中で最も洗練されたウェブサイトを作成すること.
    特に,ランディングページが与える印象で差別化すること.
\end{tcolorbox}

\begin{example}[差別化したいところの日本サイト]\mbox{}\label{exp-domestic}
    \begin{enumerate}
        \item \href{https://www.jiia.or.jp/}{日本国際問題研究所}や\href{https://www.rips.or.jp/}{平和安全保障研究所}のHPは,
        \begin{itemize}
            \item 小さな文字が多く,
            \item ブロックも上下左右に詰まっており,
            \item 従ってどこに何が書いてあるか,そして重要度の順番がわかりにくい.
        \end{itemize}
        そこで,我々は
        \begin{itemize}
            \item \textbf{文字ではなく画像中心}で,
            \item \textbf{ブロックは上下のスクロールが中心}とし,
            \item \textbf{上から下に従って重要度が下がっていく}
        \end{itemize}
        作りを徹底したい.
        \item \href{https://www.jri.co.jp/}{日本総研}や\href{https://thinktank.php.co.jp/}{PHP総研}も,
        情報を詰め込もうとするあまり,(1)と同様の弊に陥っている.特に,
        \begin{itemize}
            \item 文字中心のサイト構成
            \item 背景色は白一色に統一されており,
            \item 画像の移り変わりが激しすぎて注意が定まらない
        \end{itemize}
        点の同じ轍を踏まないことを徹底したい.
        \item \href{https://apinitiative.org/}{アジアパシフィックイニシアティブ}や\href{https://www.jfir.or.jp/}{日本国際フォーラム}は可もなく不可もない立派なデザインである.
        特に,
        \begin{itemize}
            \item 上下のみにブロックを構成しており,
            \item 余白も十分に取っており,
            \item 動画・画像の使用も豊富で,
        \end{itemize}
        素晴らしく見やすい点は見習いたい.
        
        また特に
        後者は,動的な要素も最小限に採用しており,また背景色も一様にせずアクセントを付けているために
        非常に見やすく,上述の欠点のいずれも持たない理想的なサイトである.
    \end{enumerate}
    そこで,我々はこの2つのサイトとはっきりと差別化されるデザインを用意する必要がある.
    これを,次の「参考にしたい海外サイト」から取り入れたい.
\end{example}

\begin{example}[参考にしたい海外サイト]\mbox{}\label{exp-abroad}
    \begin{enumerate}
        \item \href{https://www.brookings.edu/}{Brookings}のデザインを最も重要な参考先としたい.特に,次の点を取り入れたい:
        \begin{itemize}
            \item 最重要なランディングページのトップ部分には背景色を黒とし,文字色は白,強調色は赤の3色のみで構成する.
            \item 下にスクロールし,いくつかのサイトを列挙する構造が続く部分では,背景色を白とし,文字色は黒,強調色は青の3色のみで構成する.
        \end{itemize}
        この使い分けは特に\href{https://www.apple.com/jp/mac/}{Apple}のHPで採用されており,
        洗練された印象と豊富な情報量を両立させる貴重な方法である.
        \item \href{http://www.defenddemocracy.org/}{FDD}, 
        \href{https://www.csis.org/}{CSIS}, 
        \href{https://www.heritage.org/}{Heritage Foundation}, 
        \href{https://carnegieendowment.org/}{Carnegie}, 
        \href{https://www.piie.com/}{PIIE}, 
        \href{https://spfusa.org/}{笹川USA}などは,
        上で指摘してきた要素を漏れなく採用している.
        \item 一方で,\href{https://www.aspi.org.au/}{ASPI}, 
        \href{https://www.cfr.org/}{CFR}は文字もうまく取り入れており,場合によっては参考にしたいランディングページを持っている.
        特に,前者や\href{https://www.brookings.edu/}{Brookings}は太いフォントを採用しており,視認性が高いものとそうでないものを
        はっきりと使い分けている点は取り入れたい.
    \end{enumerate}
\end{example}

\begin{tcolorbox}[colframe=ForestGreen, colback=ForestGreen!10!white,breakable,colbacktitle=ForestGreen!40!white,coltitle=black,fonttitle=\bfseries\sffamily,
title=まとめ]
    上述の,「差別化したい日本サイト\ref{exp-domestic}」と「参考にしたい海外サイト\ref{exp-abroad}」で指摘した点は次の5つに尽きる:
    \begin{enumerate}[{[}1{]}]
        \item 文字ではなく画像中心にランディングページを構成する.
        \item 左右にブロックを作らず,上下スクロール一方向に導線を作り,上下の余白も適度に用意する.
        \item 重要な最上部のみ黒背景を採用し,インパクトを持たせる.下部は白背景を採用し,見やすさを確保する.
        \item 動的な要素(画像のスライド運動や動画の採用など)は必要最小限に留める(マウスオーバーなど).
        \item 文字は,重要なもの(記事のタイトルなど)とそうでないもの(執筆者,タグ,本文など)にはっきり峻別し,
        使うフォントの太さをはっきり分ける.
    \end{enumerate}
\end{tcolorbox}

\section{サイトの構成}

\begin{tcolorbox}[colframe=ForestGreen, colback=ForestGreen!10!white,breakable,colbacktitle=ForestGreen!40!white,coltitle=black,fonttitle=\bfseries\sffamily,
title=]
    サイトの主な固定ページは次の10ページを想定している.
\end{tcolorbox}

\begin{enumerate}\setcounter{enumi}{-1}
    \item ランディングページ
    \begin{itemize}
        \item どこを押せば何の情報が得られるかが直観的に判るデザインがなされたナビゲーション機能を備える.
        \item 上述した要素を持ち合わせること.
    \end{itemize}
    \item 研究所の概要
    \begin{itemize}
        \item 研究所説明(文字中心になるでしょう).
        \item それに付される授業・セミナー風景の大きな写真.
        \item 参考:\href{https://carnegieendowment.org/about/}{Carnegie}, \href{https://www.brookings.edu/about-us/}{Brookings}.
    \end{itemize}
    \item 専門家一覧
    \begin{itemize}
        \item 画像・名前・簡易説明・分野タグからなる「専門家カード」が一覧となったページ
        \item 参考:\href{https://www.brookings.edu/experts/}{Brookings}, \href{https://www.csis.org/about/people/experts}{CSIS}, \href{https://www.piie.com/experts}{PIIE}.
        \item 「専門家カード」を押すと個人ページに飛ぶ.個人ページは,上述のより詳しい内容と,下にスクロールすることで当人に関連する記事・論文・新聞抜粋などが,それぞれのカテゴリ別に表示される構成.
        \item 参考:\href{https://www.brookings.edu/author/hady-amr/}{Brookings}, \href{https://www.csis.org/people/michael-j-green}{CSIS}.
    \end{itemize}
    \item 経済安全保障プログラムのプログラム概要
    \begin{itemize}
        \item 経済安全保障プログラムで走っている5,6個のサブプロジェクトの一覧.
        \item 参考:\href{https://www.fdd.org/projects/}{FDD}.
        \item 画像を押すと,個別のサブプロジェクトのページに飛ぶ.そこでは,プロジェクトの簡単な説明と関連人物,そしてタグ付けされた記事が,スクロールによって現れる.
        \item 参考:\href{https://www.fdd.org/projects/center-on-cyber-and-technology-innovation/}{FDD}.
    \end{itemize}
    \item イベント一覧
    \begin{itemize}
        \item サイト上部に黒背景で直近のイベントを表示し,下部は白背景で今後のイベントが列挙されたページ.
        \item 参考:\href{https://www.brookings.edu/events/}{Brookings}.
        \item 日付のアイコンを押すことで個別イベントページに飛ぶ.
        \item 個別イベントページでは関連するポスターのPDFや,物によっては参加フォームが埋め込まれているイメージ.
        \item 参考:\href{https://spfusa.org/event/u-s-japan-relations-portland-oregon-impact-trade-economy-community/}{笹川USA}, \href{https://carnegieendowment.org/2018/02/15/taiwan-strait-update-event-6818}{Carnegie}.
    \end{itemize}
    \item ニュース・お知らせ一覧
    \begin{itemize}
        \item サイト上部に黒背景で取材受付の連絡フォーム,下部は白背景で直近のニュースとお知らせの一覧が並ぶ.
        「出版物」「新聞引用」「論文発表」「翻訳」「報告書」「イベント」等のカテゴリがあるが,下にスクロールしていくことでジャンルごとに横に列挙されているイメージ.
        \item 参考:\href{https://www.fdd.org/media/}{FDD}, \href{https://www.piie.com/newsroom}{PIIE}.
        \item アイコンを押すと個別ページに飛ぶ.参考:\href{https://www.piie.com/newsroom/press-releases/douglas-irwin-and-mary-lovely-join-peterson-institute-international}{PIIE}.
    \end{itemize}
    \item メーリングリスト登録ページ
    \begin{itemize}
        \item \href{https://www.brookings.edu/}{Brookings}のように,ランディングページ下部にポップアップを表示して,このページに飛ぶ仕様としたい.
        \item このページでは登録はもちろん,過去のメルマガ一覧も見れるようにしたい.詳しくは未定.
    \end{itemize}
    \item 授業一覧
    \begin{itemize}
        \item 授業名・内容などのシラバスの内容の抜粋と,シラバスへのリンクなどを載せる予定.
        \item 担当教員をタグ化し,専門家一覧と連関させたい.
        \item 詳しくは未定.
    \end{itemize}
    \item スポンサー一覧
    \begin{itemize}
        \item 上部に黒背景でスポンサー募集の内容を掲載.文書や画像でスポンサーになることのメリットやスポンサーになる手続き等の説明など.
        \item その下に白背景で企業スポンサーのロゴ一覧を掲載.
        \item 参考:\href{https://www.aspi.org.au/sponsors}{ASPI}.少しロゴの陳列が変だが.
    \end{itemize}
    \item お問い合わせフォーム
    \begin{itemize}
        \item 参考:\href{https://www.heritage.org/contact}{Heritage Foundation}.
    \end{itemize}
    \item プライバシーポリシー
    \begin{itemize}
        \item 参考:\href{https://www.csis.org/privacy-policy}{CSIS}, \href{https://www.heritage.org/article/privacy-policy}{Heritage Foundation}, \href{https://spfusa.org/privacy-policy/}{笹川USA}.
    \end{itemize}
\end{enumerate}

\end{document}