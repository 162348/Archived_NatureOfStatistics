\documentclass[uplatex, dvipdfmx]{jsarticle}
\title{志望理由書}
\date{\today}
\pagestyle{empty} \setcounter{secnumdepth}{4}
\input{/Users/Hirofumi Shiba/NatureOfStatistics/preamble_no_fonts.tex}
%\input{/Users/hirofumi.shiba48/NatureOfStatistics/preamble_no_fonts.tex}
\usepackage[math]{anttor}
\begin{document}

\begin{description}
    \item[(1) 大学院での志望研究内容] \mbox{}\\
    筆者は主に因果推論とロバスト統計,具体的には次の3つの主題に特に注目している.
    \vspace{1em}
    \begin{description}
        \item[i. 因果モデルの実用化] 因果推論のための統計モデルは,従来のものよりも遥かに多くの仮定を要する.このことが実際のデータへの適用の妨げになり得るため,モデルを真に実用に耐えるものにするためには,種々の仮定に対する感度分析や,仮定への離反に対して頑健な手法の開発が極めて肝要である.
        この課題について近年\textbf{二重に頑健な推定量(doubly robust estimator)}が解決の道筋を与えるとして注目されている\cite{Robins-Rotnitzky01-Comments}.
        \vspace{1em}\item[ii. セミパラメトリック・ロバスト理論] しかし上述の推定量を構成するための手段は,大別して2つほど提案されているのみ\cite{Rotnitzky-Smucler-Robins21-MixedBiasProperty}で,
        必要条件を与えて特徴付ける方向の試みは未だ見当たらない.
        その障碍は専ら対象の数学的な複雑さにあると筆者は目しており,特徴付けを得て組織立った手法としての全貌を示すことで広く使いやすいものとし,更にはセミパラメトリックモデルにおける頑健統計に対して人類の理解を深める貢献を目指したい.
        \vspace{1em}\item[iii. 高次理論] 因果推論の幅広い応用を目指すに当たり,データが複雑な依存構造を持つ状況も考えたい.
        頑健性の達成に加え,非独立・非正規なデータにも対応できる高次理論が得られれば,爆発的な応用が期待できる.
        この困難極まる途にも近年進展が見られる\cite{Robins-Li-Tchetgen-vanderVaart08-HigherOrder}.
        例えば被験者の割当を,得られたデータを基に%所定のルールで
        変えていく実験計画を採用した場合(batched bandit問題),
        データの時系列は当然無相関でも正規でもないが,partial mixingと言われる構造を持ち,これに注目することで統計量の高次な漸近的近似を与えることが出来る\cite{YechanPark-Zhan-Yoshida22}.
        %このように,データの非独立性・非正規性とモデルの非線型性に対応するための高次理論は困難であるが進展が見られる\cite{Robins-Li-Tchetgen-vanderVaart08-HigherOrder}, \cite{vanderVaart14-HigherOrder}.
    \end{description}
    \vspace{1em}
    上述が筆者の現時点の知識の限界であり,数理を中心に情報収集していることは明らかである.
    しかし,
    因果推論はそもそも,\textbf{元来統計分野では忌避されたが,殆どの実証科学分野では喫緊事であったために,厳密な議論を後回しにしながらも整備されていった分野}という側面を持ち,近年\cite{Rubin74-Causal}や\cite{Pearl95-CausalDiagram}で理論的輪郭が露わになったことで,再び統計分野でも注目を集めている分野である.
    筆者はこの点でも因果推論が好きで,
    ゆくゆくは筆者も各応用分野との協業の中で本当に必要としている手法を見極め,その為の知識と技術を適宜準備する柔軟さと数理的な基礎体力を持ちたいと考える.
    \vspace{1em}\clearpage
    \item[(2) 強調したい自身の能力や実績] \mbox{}\vspace{1em}
    \begin{description}
        \item[i. 数理能力] 前項(1) ii.に言う「数学的な複雑さ」は主に局外母数の空間が無限次元になり微分構造が混迷を来すことに起因するが,これには数学の分野・関数解析の約100年の蓄積がある.
        (1) iii.で紹介したセミパラメトリックモデルの高次理論の研究群は多くが確率過程論の極限定理の数学的結果を応用する段階が本質的な要素である.
        筆者は4年生時,上述の2点を中心に必要と思われる準備を\cite{YechanPark-Zhan-Yoshida22}の第三著者である吉田朋広先生(筆者指導教官)の下での1対1での数学セミナーと,同第一著者(筆者と同大学経済学部同期生)らとの自主輪読活動を通じて行った.
        \vspace{1em}\item[ii. 実務経験] 前項(1) 末文で述べたように筆者は統計手法の開発を通じた各科学分野への貢献を志している.それには,LISRELが構造方程式モデルの普及を後押ししたように,計算機プログラムを通じて多くの人に使いやすい形に落とし込む社会実装が欠かせないと考える.
        筆者は,経営学研究を基礎に産学連携とコンサルティングを営むIMIS研究所にて,
        顧客アンケートなどのテキストデータを解析して企業の経営を支援するソリューションの中核となるPythonプログラムを作成する特別研究員として
        データ解析とパッケージ作成を行う業務に携わり,現在も継続中である.
    \end{description}
    \item[(3) これまで読んだ本・科学論文などのうち興味を持ったものについての自分の考え] 
    \mbox{}
    \vspace{1em}
    
     高校生時,行動経済学の一般書\cite{DanielKahneman13-FastAndSlow}の
    第4章にて,
    恥の感情を誘発された被験者は無意識下に手を洗う衝動が起きる\cite{06-LadyMacbeth} (Macbeth効果)という研究と出会った.
    その後も心理学,経済学から行動科学,疫学と実証科学の大海を渡るほどに,
    我々の日々の生活に深く関わる信じがたい知見に山ほど出会い,
    これらの知見がそもそも,統計分析を通じて
    発見可能な科学的対象であることに驚嘆した.
    また当該論文でも議論されているような,
    これら統計を通じた数々の観察から構成される諸科学分野固有の理論・仮説は,
    日頃の素朴な「なぜ?」の疑問に答え得る非常に魅惑的なものである.
    %また当論文でも議論されている通り,これらの知見から提案される仮説(natural re-use仮説など)は「我々あはどこから来たのか」に答え得るもので,なるほど心理学と進化論は統計学最初にして最大の応用分野の一つだと首肯させるものであった.
    
     しかし大学進学して後,当該研究も「実証科学再現性の危機」の例に漏れないことを知った.例えば\cite{14-AntiLadyMacbeth}は2倍のサイズの,より文化的に多様な標本では有意性が消えたとしている.これはMacbeth効果は文化依存性を持つより繊細な消息であったということだろう.筆者は更なる解明を待っているが,ただ指を咥えて待つだけではない.Pearlの構造的因果モデルは,交絡がある中で因果を結論付ける為の必要条件を明確にする試みであり,これに頼ればより精緻な議論が可能だろう.
    このPearlの仕事のように,筆者は統計の語彙を,厳密性とわかり易さを両立させたまま豊かにすることによって,日常の根源的な疑問に鮮やかに応える科学の研究に微力でも貢献したいと青春の思い出を燃やしている.
\end{description}

\small
\bibliography{../../StatisticalSciences.bib,../../SocialSciences.bib,../../mathematics.bib,../../statistics.bib}

\end{document}