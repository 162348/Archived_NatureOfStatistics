\documentclass[uplatex,dvipdfmx]{jsarticle}
\title{総合研究大学院大学複合科学研究科統計科学専攻\\
博士課程(5年一貫制)入学試験(1/19/2021実施)\\
問題と解答}
\author{あの\footnote{e-mail address : anomath57@gmail.com\\URL : \url{https://anomath.com/}}}
\date{\today}
\pagestyle{headings} \setcounter{secnumdepth}{4}
\input{/Users/Hirofumi Shiba/NatureOfStatistics/preamble_no_fonts.tex}
%\input{/Users/hirofumi.shiba48/NatureOfStatistics/preamble_no_fonts.tex}
\usepackage[math]{anttor}
\begin{document}
\maketitle

\begin{tcolorbox}[title=記法についての注意]
    次の記法は以後断りなく用いる.
    \begin{enumerate}
        \item $n=1,2,\cdots$について,
        \[n:=\Brace{0,1,2,\cdots,n-1},[n]:=\{1,2,\cdots,n\}.\N=\{0,1,2,\cdots\},\N^+=\N_{>0}=\{1,2,3,\cdots\}.\]
        \item 同様にして,$\R_+:=\Brace{x\in\R\mid x\ge0}$,$\R^+:=\Brace{x\in\R\mid x>0}$,$\o{\R_+}:=[0,\infty]$.
        \item $M_{mn}(\R)$で$(m,n)$-実正方行列の全体,$\GL_n(\R)\subset M_n(\R)$でそのうち可逆なものの全体を表す.
        \item $I_d\in M_d(\R)$を単位行列,$O_d\in M_d(\R)$を零行列とする.
        \item $\Sp(A)\subset\C$で,行列$A\in M_n(\R)$の固有値全体の集合を表す.
        \item $f(x)=O(x^n)\;(x\to0)$で$\limsup_{x\to0}\Abs{\frac{f(x)}{x^n}}<\infty$を表す.
        \item $\R$上のLebesgue測度を$m$,距離空間$S$上のBorel $\sigma$-代数を$\B(S)$で表す.
        \item $1_A$で集合$A$の指示関数を表す.
        \item 確率変数$X,Y$に対して,期待値を$E[X]$,分散を$\Var[X]$,共分散を$\Cov[X,Y]$で表す.
        \item $U(S)$で集合$S$上の一様分布,$N(\mu,\sigma^2)$で平均$\mu$分散$\sigma^2$の正規分布を表す.
        \item $\Exp(\gamma)\;(\gamma>0)$で指数分布$f(x)=\gamma e^{-\gamma x}1_{\Brace{x>0}}$を表す.
    \end{enumerate}
    問題文の表現は筆者の都合で一部変えています.
    過去に実施された入学試験問題は\href{https://www.ism.ac.jp/senkou/admission/kakomon.html}{統計数理研究所HP}から見れます.
\end{tcolorbox}

\section{問題}

\begin{tcolorbox}[colframe=ForestGreen, colback=ForestGreen!10!white,breakable,colbacktitle=ForestGreen!40!white,coltitle=black,fonttitle=\bfseries\sffamily,
    title=概観]
    \begin{enumerate}[{第}1{問}]
        \item 
    \end{enumerate}
\end{tcolorbox}

\subsection{第一問}

\begin{tcolorbox}[colframe=ForestGreen, colback=ForestGreen!10!white,breakable,colbacktitle=ForestGreen!40!white,coltitle=black,fonttitle=\bfseries\sffamily,
    title=第1問]
    \begin{problem}\mbox{}
        \begin{enumerate}[{問}1]
            \item 
        \end{enumerate}
    \end{problem}
\end{tcolorbox}
\begin{proof}[\textbf{\underline{[解答例]}}]\mbox{}
    \begin{enumerate}
        \item 
    \end{enumerate}
\end{proof}

\subsection{第二問}

\begin{tcolorbox}[colframe=ForestGreen, colback=ForestGreen!10!white,breakable,colbacktitle=ForestGreen!40!white,coltitle=black,fonttitle=\bfseries\sffamily,
    title=第2問]
    \begin{problem}\mbox{}
        \begin{enumerate}[{問}1]
            \item 
        \end{enumerate}
    \end{problem}
\end{tcolorbox}
\begin{proof}[\textbf{\underline{[解答例]}}]\mbox{}
    \begin{enumerate}
        \item 
    \end{enumerate}
\end{proof}

\subsection{第三問}

\begin{tcolorbox}[colframe=ForestGreen, colback=ForestGreen!10!white,breakable,colbacktitle=ForestGreen!40!white,coltitle=black,fonttitle=\bfseries\sffamily,
    title=第3問]
    \begin{problem}\mbox{}
        \begin{enumerate}[{問}1]
            \item 
        \end{enumerate}
    \end{problem}
\end{tcolorbox}
\begin{proof}[\textbf{\underline{[解答例]}}]\mbox{}
    \begin{enumerate}
        \item 
    \end{enumerate}
\end{proof}

\subsection{第四問}

\begin{tcolorbox}[colframe=ForestGreen, colback=ForestGreen!10!white,breakable,colbacktitle=ForestGreen!40!white,coltitle=black,fonttitle=\bfseries\sffamily,
    title=第4問]
    \begin{problem}\mbox{}
        \begin{enumerate}[{問}1]
            \item 
        \end{enumerate}
    \end{problem}
\end{tcolorbox}
\begin{proof}[\textbf{\underline{[解答例]}}]\mbox{}
    \begin{enumerate}
        \item 
    \end{enumerate}
\end{proof}

\end{document}