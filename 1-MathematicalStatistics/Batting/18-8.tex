\documentclass[uplatex,dvipdfmx]{jsarticle}
\title{総合研究大学院大学複合科学研究科統計科学専攻\\
博士課程(5年一貫制)入学試験(1/19/2021実施)\\
問題と解答}
\author{あの\footnote{e-mail address : anomath57@gmail.com\\URL : \url{https://anomath.com/}}}
\date{\today}
\pagestyle{headings} \setcounter{secnumdepth}{4}
\input{/Users/Hirofumi Shiba/NatureOfStatistics/preamble_no_fonts.tex}
%\input{/Users/hirofumi.shiba48/NatureOfStatistics/preamble_no_fonts.tex}
\usepackage[math]{anttor}
\begin{document}
\maketitle

\begin{tcolorbox}[title=記法についての注意]
    次の記法は以後断りなく用いる.
    \begin{enumerate}
        \item $n=1,2,\cdots$について,
        \[n:=\Brace{0,1,2,\cdots,n-1},[n]:=\{1,2,\cdots,n\}.\N=\{0,1,2,\cdots\},\N^+=\N_{>0}=\{1,2,3,\cdots\}.\]
        \item 同様にして,$\R_+:=\Brace{x\in\R\mid x\ge0}$,$\R^+:=\Brace{x\in\R\mid x>0}$,$\o{\R_+}:=[0,\infty]$.
        \item $M_{mn}(\R)$で$(m,n)$-実正方行列の全体,$\GL_n(\R)\subset M_n(\R)$でそのうち可逆なものの全体を表す.
        \item $I_d\in M_d(\R)$を単位行列,$O_d\in M_d(\R)$を零行列とする.
        \item $\Sp(A)\subset\C$で,行列$A\in M_n(\R)$の固有値全体の集合を表す.
        \item $f(x)=O(x^n)\;(x\to0)$で$\limsup_{x\to0}\Abs{\frac{f(x)}{x^n}}<\infty$を表す.
        \item $\R$上のLebesgue測度を$m$,距離空間$S$上のBorel $\sigma$-代数を$\B(S)$で表す.
        \item $1_A$で集合$A$の指示関数を表す.
        \item 確率変数$X,Y$に対して,期待値を$E[X]$,分散を$\Var[X]$,共分散を$\Cov[X,Y]$で表す.
        \item $U(S)$で集合$S$上の一様分布,$N(\mu,\sigma^2)$で平均$\mu$分散$\sigma^2$の正規分布を表す.
        \item $\Exp(\gamma)\;(\gamma>0)$で指数分布$f(x)=\gamma e^{-\gamma x}1_{\Brace{x>0}}$を表す.
    \end{enumerate}
    問題文の表現は筆者の都合で一部変えています.
    過去に実施された入学試験問題は\href{https://www.ism.ac.jp/senkou/admission/kakomon.html}{統計数理研究所HP}から見れます.
\end{tcolorbox}

\section{問題}

\subsection{第一問}

\begin{tcolorbox}[colframe=ForestGreen, colback=ForestGreen!10!white,breakable,colbacktitle=ForestGreen!40!white,coltitle=black,fonttitle=\bfseries\sffamily,
    title=第1問]
    \begin{problem}\mbox{}
        \begin{enumerate}[{問}1]
            \item 次の行列が可逆か判定し,可逆ならば逆行列を,非可逆ならば右零因子を1つ求めよ.
            \[A:=\mtrx{1}{3}{2}{4},\qquad B:=\mtrx{1}{3}{3}{9}\]
            \item 次の関数の導関数を求めよ:
            \[f(x):=5^{x^2-3x+1},\qquad g(x):=\log\sqrt{\frac{1-x}{1+x}},\quad(\abs{x}<1).\]
            \item $f(x):=(1+x)^{\frac{1}{x}}$の2次までのMaclaurin展開を求めよ.
            \item 次の3つのベクトルが生成する$\R^4$の線型部分空間の正規直交基底を1組求めよ.
            \[a:=\begin{pmatrix}
                1\\2\\0\\1
            \end{pmatrix},\quad b:=\begin{pmatrix}
                2\\1\\0\\1
            \end{pmatrix},\quad c:=\begin{pmatrix}
                1\\0\\1\\2
            \end{pmatrix}\]
        \end{enumerate}
    \end{problem}
\end{tcolorbox}
\begin{proof}[\textbf{\underline{[解答例]}}]\mbox{}
    \begin{enumerate}
        \item $\det A=-2$より可逆で,逆行列は$\frac{1}{-2}\mtrx{4}{-3}{-2}{1}$.$\det B=0$より特異で,$\mtrx{-3}{-3}{1}{1}$は右の零因子である.
        \item $g'$の計算は,$\abs{x}<1$のとき$1-x,1+x>0$に注意して,
        \begin{align*}
            f'(x)&=(e^{\log 5(x^2-3x+1)})'=\log 5(2x-3)5^{x^2-3x+1}.\\
            g'(x)&=\paren{\frac{1}{2}\paren{\log(1-x)-\log(1+x)}}'=-\frac{1}{2}\frac{1}{1-x}-\frac{1}{2}\frac{1}{1+x}.
        \end{align*}
        \item $f(0)=\lim_{x\to 0}(1+x)^{\frac{1}{x}}=e$.両辺の対数を取って微分すると,
        \[\frac{f'(x)}{f(x)}=-\frac{\log(1+x)}{x^2}+\frac{1}{x(1+x)}=\frac{-(1+x)\log(1+x)+x}{x^2(1+x)}.\]
        右辺の$x=0$における極限はl'Hospitalの定理を繰り返し適用することより計算出来て,$f'(0)=f(0)\paren{-\frac{1}{2}}=-\frac{e}{2}$を得る.
        二次の微分係数も同様にして,
        \[f(x)=e-\frac{e}{2}x+\frac{11e}{24}x^2+O(x^3)\qquad(\abs{x}<1).\]
        \item \[e_1:=\frac{1}{\norm{a}}=\frac{1}{\sqrt{6}}
        \begin{pmatrix}
            1\\2\\0\\1
        \end{pmatrix}.\]
        次に,$b$の$e_1$の直交成分は
        \[b-\frac{a\cdot b}{\norm{a}^2}a=\begin{pmatrix}
            2\\1\\0\\1
        \end{pmatrix}-\frac{5}{6}\begin{pmatrix}
            1\\2\\0\\1
        \end{pmatrix}=\frac{1}{6}\begin{pmatrix}7\\-4\\0\\1\end{pmatrix}=:\frac{1}{6}b'\]
        より,
        \[e_2:=\frac{1}{\sqrt{66}}\begin{pmatrix}7\\-4\\0\\1\end{pmatrix}\]
        最後に,$c$の$e_1,e_2$直交成分は,
        \[c-\frac{a\cdot c}{\norm{a}^2}a-\frac{b'\cdot c}{\norm{b'}^2}b'=\begin{pmatrix}
            1\\0\\1\\2
        \end{pmatrix}-\frac{3}{6}\begin{pmatrix}
            1\\2\\0\\1
        \end{pmatrix}-\frac{9}{66}\begin{pmatrix}
            7\\-4\\0\\1
        \end{pmatrix}=\frac{1}{11}\begin{pmatrix}-5\\-5\\11\\15\end{pmatrix}\]
        より,
        \[e_3:=\frac{1}{3\sqrt{43}}\begin{pmatrix}-5\\-5\\11\\15\end{pmatrix}\]
        以上より,$e_1,e_2,e_3$は所与の空間の正規直交基底をなす.
    \end{enumerate}
\end{proof}

\subsection{第二問}

\begin{tcolorbox}[colframe=ForestGreen, colback=ForestGreen!10!white,breakable,colbacktitle=ForestGreen!40!white,coltitle=black,fonttitle=\bfseries\sffamily,
    title=第2問]
    \begin{problem}\mbox{}
        \begin{enumerate}[{問}1]
            \item 任意の実行列$B$に対して,$B^\top B$の固有値は非負であることを示せ.
            \item 次の$B\in M_{23}(\R)$に対して,$B^\top B$の固有値と固有ベクトルを求めよ:
            \[B:=\begin{pmatrix}0&\sqrt{3}&1\\0&0&2\end{pmatrix}\]
            \item 同様の$B\in M_{23}(\R)$に対して,$VB^\top BV^{-1}$が対角行列になるような直交行列$V\in\r{O}_3(\R)$を1つ求めよ.
        \end{enumerate}
    \end{problem}
\end{tcolorbox}
\begin{proof}[\textbf{\underline{[解答例]}}]\mbox{}
    \begin{enumerate}
        \item 任意の$x\in\R^n$に対して,$(B^\top Bx|x)=(Bx|Bx)=\norm{Bx}^2\ge0$より,$B^\top B$は半正定値行列である.よって,$B^\top B$の固有値は非負である.
        \item まず,
        \[B^\top B=\begin{pmatrix}0&0&0\\0&3&\sqrt{3}\\0&\sqrt{3}&5\end{pmatrix}\]
        と計算出来,固有方程式$\lambda(\lambda-2)(\lambda-6)=0$を考えることにより,固有値は$0,2,6$.
        それぞれに属する固有ベクトルは,$(1,0,0)^\top$,$(0,\sqrt{3},-1)^\top$,$(0,1,\sqrt{3})^\top$が取れる.
        \item (2)の議論より,
        \[V:=\frac{1}{2}\begin{pmatrix}1&0&0\\0&\sqrt{3}&1\\0&-1&\sqrt{3}\end{pmatrix}^{-1}=\frac{1}{2}\begin{pmatrix}1&0&0\\0&\sqrt{3}&1\\0&-1&\sqrt{3}\end{pmatrix}\]
        と定めるよい.
    \end{enumerate}
\end{proof}
\begin{remark*}
    固有ベクトルからなる正規直交基底を列ベクトルとする直交行列を$U$とすると,$U^{-1}B^\top BU$は対角行列になる.
    しかし,第3問の問題分では$VB^\top BV^{-1}$となっている.その点,$V$は$U^{-1}$と取る必要がある.
\end{remark*}

\subsection{第三問}

\begin{tcolorbox}[colframe=ForestGreen, colback=ForestGreen!10!white,breakable,colbacktitle=ForestGreen!40!white,coltitle=black,fonttitle=\bfseries\sffamily,
    title=第3問]
    \begin{problem}\mbox{}
        \begin{enumerate}[{問}1]
            \item 数直線$\R$上の点Pの$x$座標$X$は$\rN(0,1)$に従うとする.
            Pの原点からの距離の自乗の確率密度関数が
            \[\frac{1}{\sqrt{2\pi x}}e^{-\frac{x}{2}},\qquad(x>0)\]
            であることを示せ.
            \item Euclid空間$\R^n$内の点Qの座標$(X_1,\cdots,X_n)$は$\rN_n(0,I_n)$に従うとする.
            Qの原点からの距離の自乗の確率密度関数が
            \[\frac{1}{\Gamma\paren{\frac{n}{2}}2^{\frac{n}{2}}}x^{\frac{n}{2}-1}e^{-\frac{x}{2}},\qquad(x>0)\]
            であることを示せ.
            \item (2)の確率密度関数を持つ分布を$\chi^2(n)$という.
            確率変数$X,Y$は独立で$X\sim\chi^2(n),Y\sim\chi^2(m)$であるとする.このとき,
            \[X+Y\sim\chi^2(n+m),\quad\frac{X}{X+Y}\sim\Beta(n/2,m/2)\]
            であり,互いに独立であることを示せ.
        \end{enumerate}
    \end{problem}
\end{tcolorbox}
\begin{proof}[\textbf{\underline{[解答例]}}]\mbox{}
    \begin{enumerate}
        \item $X^2$の分布関数$F$は
        \[F(x)=P[X^2\le x]=\frac{1}{\sqrt{2\pi}}\int^{\sqrt{x}}_{-\sqrt{x}}e^{-\frac{y^2}{2}}dy.\]
        よって,
        \[F'(x)=\frac{1}{\sqrt{2\pi}}\frac{1}{2\sqrt{x}}2e^{-\frac{(\sqrt{x})^2}{2}}=\frac{1}{\sqrt{2\pi x}}e^{-\frac{x}{2}}.\]
        \item Qの原点からの距離の自乗は$X_1^2+\cdots+X_n^2$と表せ,$X_1,\cdots,X_n$は互いに独立であることに注意すると,
        \[\varphi_n(x):=\frac{1}{\Gamma\paren{\frac{n}{2}}2^{\frac{n}{2}}}x^{\frac{n}{2}-1}e^{-\frac{x}{2}}\]
        の積率母関数が$\frac{1}{\sqrt{2\pi x}}e^{-\frac{x}{2}}$の積率母関数
        \[g(u):=\int^\infty_0e^{-ux}\frac{1}{\sqrt{2\pi x}}e^{-\frac{x}{2}}dx=\frac{1}{\sqrt{2}}\frac{1}{\paren{\frac{1}{2}+u}^{1/2}}\]
        の$n$乗であることを示せば良い.
        \item 
    \end{enumerate}
\end{proof}

\subsection{第四問}

\begin{tcolorbox}[colframe=ForestGreen, colback=ForestGreen!10!white,breakable,colbacktitle=ForestGreen!40!white,coltitle=black,fonttitle=\bfseries\sffamily,
    title=第4問]
    \begin{problem}
        $x\in\R^p$の関数$f$
        \[f(\b{x}):=\sum_{i=1}^mg(\b{x},a_i,\b{b}_i),\qquad g(\b{x},a,\b{b}):=e^{-(a-\b{b}^\top\b{x})^2},\quad a_i\in\R,\b{b}_i\in\R^n.\]
        の最大化を考える.ただし,$\rank(\b{b}_1,\cdots,\b{b}_m)=p\le m$とする.
        関数$F$は
        \[F(\b{x},\b{y}):=\sum_{i\in[m]}g(\b{y},a_i,\b{b}_i)(a_i-\b{b}_i^\top\b{x})^2\]
        と定め,$\b{x}_1\in\R^p$から順に$\b{x}_{t+1}:=\argmin_{\b{x}\in\R^p}F(\b{x},\b{x}_t)$とする.
        \begin{enumerate}[{問}1]
            \item $\b{x}_{t+1}$を$\b{x}_t$の関数として表せ.
            \item 任意のスカラー$A,A_0$に対して$e^{A}-e^{A_0}\ge(A-A_0)e^{A_0}$を示せ.
            \item 任意の$\b{x},\b{y}\in\R^p$に対して,$f(\b{x})-f(\b{y})\ge F(\b{y},\b{y})-F(\b{x},\b{y})$を示せ.
            \item $\forall_{t\in\N^+}\;f(\b{x}_{t+1})\ge f(\b{x}_t)$を示せ.
        \end{enumerate}
    \end{problem}
\end{tcolorbox}
\begin{proof}[\textbf{\underline{[解答例]}}]\mbox{}
    \begin{enumerate}
        \item \[B:=(\b{b}_1,\cdots,\b{b}_m)\in M_{pm}(\R),\quad G:=(g(\b{y},a_1,\b{b}_1)\b{b}_m,\cdots,g(\b{y},a_m,\b{b}_m)\b{b}_m)\in M_{pm}(\R)\]
        と定めると,
        \[D_xF(\b{x},\b{y})=0\Leftrightarrow G \begin{pmatrix}a_1-\b{b}_1^\top\b{x}\\\vdots\\a_m-\b{b}_m^\top\b{x}\end{pmatrix}=0\]
        となる.ただし,$D_xF$とは,$F$の第一引数$\b{x}$に関する勾配とした.
        よって,
        \[G \begin{pmatrix}a_1\\\vdots\\a_m\end{pmatrix}=GB\b{x}_p.\]
        $GB\in\GL_p(\R)$に注意すれば,
        \[\b{x}=(GB)^{-1}G \begin{pmatrix}a_1\\\vdots\\a_m\end{pmatrix}.\]
        すなわち,$\b{y}:=\b{x}_t$を代入すれば,
        \[\b{x}_{t+1}=\Paren{(g(\b{x}_t,a_1,\b{b}_1)\b{b}_m,\cdots,g(\b{x}_t,a_m,\b{b}_m)\b{b}_m)(\b{b}_1,\cdots,\b{b}_m)}^{-1}(g(\b{x}_t,a_1,\b{b}_1)\b{b}_m,\cdots,g(\b{x}_t,a_m,\b{b}_m)\b{b}_m)\begin{pmatrix}a_1\\\vdots\\a_m\end{pmatrix}.\]
        \item $A>A_0$のとき,平均値の定理より,ある$c\in(A_0,A)$が存在して,
        \[\frac{e^A-e^{A_0}}{A-A_0}=e^c\ge e^{A_0}.\]
        $A<A_0$の場合も同様.$A=A_0$の場合は$e^A-e^{A_0}=0=(A-A_0)e^{A_0}$である.
        \item 任意の$\b{x},\b{y}\in\R^p$を取り,$A^i:=-(a_i-\b{b}_i^\top\b{x})^2,A_0^i:=-(a_i-\b{b}_i^\top\b{y})^2$と定める.すると,問2より,
        \[f(\b{x})-f(\b{y})=\sum_{i\in[m]}(e^{A^i}-e^{A_0^i})\ge\sum_{i\in[m]}(A^i-A_0^i)e^{A_0^i}=F(\b{y},\b{y})-F(\b{x},\b{y}).\]
        \item 問3の不等式と列$\{\b{x}_t\}$の取り方より,
        \[f(\b{x}_{t+1})-f(\b{x}_{t})\ge F(\b{x}_t,\b{x}_t)-F(\b{x}_{t+1},\b{x}_t)\ge0.\]
    \end{enumerate}
\end{proof}
\begin{remarks}
    幾何学的に言えば,affine超平面$a_i-\b{b}_i^\top\b{x}=0$との距離の何らかの意味の和を最小化する問題を考えている.
\end{remarks}

\end{document}