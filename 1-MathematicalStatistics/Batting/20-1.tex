\documentclass[uplatex,dvipdfmx]{jsarticle}
\title{総合研究大学院大学複合科学研究科統計科学専攻\\
博士課程(5年一貫制)入学試験(1/21/2020実施)\\
問題と解答}
\author{あの\footnote{e-mail address : anomath57@gmail.com\\URL : \url{https://anomath.com/}}}
\date{\today}
\pagestyle{headings} \setcounter{secnumdepth}{4}
\input{/Users/Hirofumi Shiba/NatureOfStatistics/preamble_no_fonts.tex}
%\input{/Users/hirofumi.shiba48/NatureOfStatistics/preamble_no_fonts.tex}
\usepackage[math]{anttor}
\begin{document}
\maketitle

\begin{tcolorbox}[title=記法についての注意]
    次の記法は以後断りなく用いる.
    \begin{enumerate}
        \item $n=1,2,\cdots$について,$n:=\Brace{0,1,2,\cdots,n-1}$,$[n]:=\{1,2,\cdots,n\}$.$\N=\{0,1,2,\cdots\}$,$\N^+=\N_{>0}=\{1,2,3,\cdots\}$.
        \item 同様にして,$\R_+:=\Brace{x\in\R\mid x\ge0}$,$\R^+:=\Brace{x\in\R\mid x>0}$,$\o{\R_+}:=[0,\infty]$.
        \item $M_{mn}(\R)$で$(m,n)$の実正方行列の全体,$\GL_n(\R)\subset M_n(\R)$でそのうち可逆なものの全体を表す.
        \item $I_d\in M_d(\R)$を単位行列,$O_d\in M_d(\R)$を零行列とする.
        \item $f(x)=O(x^n)\;(x\to0)$で$\limsup_{x\to0}\Abs{\frac{f(x)}{x^n}}<\infty$を表す.
        \item $\R$上のLebesgue測度を$m$,距離空間$S$上のBorel $\sigma$-代数を$\B(S)$で表す.
        \item $1_A$で集合$A$の指示関数を表す.
        \item $U(S)$で集合$S$上の一様分布,$N(\mu,\sigma^2)$で平均$\mu$分散$\sigma^2$の正規分布を表す.
        \item $\Exp(\gamma)\;(\gamma>0)$で指数分布$f(x)=\gamma e^{-\gamma x}1_{\Brace{x>0}}$を表す.
    \end{enumerate}
    問題文の表現は筆者の都合で一部変えています.
    過去3年分の入学試験問題は\href{https://www.ism.ac.jp/senkou/admission/kakomon.html}{こちら}から見れます.
\end{tcolorbox}

\begin{tcolorbox}[colframe=ForestGreen, colback=ForestGreen!10!white,breakable,colbacktitle=ForestGreen!40!white,coltitle=black,fonttitle=\bfseries\sffamily,
    title=第1問]
    \begin{enumerate}
        \item 次の行列の逆行列を求めよ:
        \[\begin{pmatrix}2&1&1\\-1&0&2\\1&1&2\end{pmatrix}.\]
        \item 確率変数列$\{X_n\}\subset L^2(\Om)$が$\lim_{n\to\infty}E[X_n^2]=0$を満たすならば$\forall_{\ep>0}\;P[\abs{X_n}>\ep]=0$が成り立つことを示す.
        \item 確率変数列$\{X_n\}\subset L^2(\Om)$であって,$\forall_{\ep>0}\;P[\abs{X_n}>\ep]=0$であるが$\lim_{n\to\infty}E[X_n^2]\ne0$であるものの例を挙げよ.
    \end{enumerate}
\end{tcolorbox}
\begin{proof}[\textbf{\underline{[解答例]}}]\mbox{}
    \begin{enumerate}
        \item 略.
        \item 任意の$\ep>0$を取る.$\ep<\abs{X_n}$ならば$\ep^2<\abs{X_n}^2$であるから,事象集合について
        \[\Brace{\om\in\Om\mid\ep<\abs{X_n(\om)}}\subset\Brace{\om\in\Om\mid\ep^2<\abs{X_n^2(\om)}}\]
        が成り立つ.よって,
        \begin{align*}
            (0\le)P[\ep<\abs{X_n}]&\le P[\ep^2<\abs{X_n}^2]\le\frac{E[\abs{X_n}^2]}{\ep^2}\xrightarrow{n\to\infty}0.
        \end{align*}
        \item 確率空間$([0,1],\B([0,1]),m)$上の実確率変数$Y_n:[0,1]\to\R$を
        \[Y_n:=\sqrt{n}1_{[0,1/n]}\quad(n\in\N^+)\]
        と定める.すると,$\lim_{n\to\infty}\frac{1}{n}=0$より,$\forall_{\ep>0}\;\exists_{N>0}\;\forall_{n\ge N}\;\frac{1}{n}<\ep$.
        よって,$\forall_{n\ge N}\;P[\abs{Y_n}>\ep]=0$より当然$\lim_{n\to\infty}P[\abs{Y_n}>\ep]=0$.
        一方で,$\forall_{n\in\N^+}\;E[Y_n^2]=1$である.
    \end{enumerate}
\end{proof}

\begin{tcolorbox}[colframe=ForestGreen, colback=ForestGreen!10!white,breakable,colbacktitle=ForestGreen!40!white,coltitle=black,fonttitle=\bfseries\sffamily,
    title=第2問]
    次の定積分を求めよ:
    \begin{enumerate}
        \item \[\int^1_0\frac{dx}{x^2-x+1}.\]
        \item \[\int^1_0\frac{dx}{x^3+1}.\]
    \end{enumerate}
\end{tcolorbox}
\begin{proof}[\textbf{\underline{[解答例]}}]\mbox{}
    \begin{enumerate}
        \item $x-\frac{1}{2}=\frac{\sqrt{3}}{2}\tan\theta$によって変数変換を行うと,次のように計算できる:
        \begin{align*}
            \int^1_0\frac{dx}{x^2-x+1}&=\int^1_0\frac{1}{\paren{x-\frac{1}{2}}^2+\frac{3}{4}}dx\\
            &=\int^{\pi/6}_{-\pi/6}\frac{4}{3}\frac{1}{\tan^2\theta+1}\frac{\sqrt{3}}{2}\frac{d\theta}{\cos^2\theta}\\
            &=\int^{\pi/6}_{-\pi/6}\frac{4}{3}\frac{\sqrt{3}}{2}d\theta
            =\int^{\pi/6}_{-\pi/6}\frac{2}{\sqrt{3}}=\frac{2\pi}{3\sqrt{3}}.
        \end{align*}
        \item 被積分関数の部分分数分解
        \[\frac{1}{x^3+1}=\frac{1}{3}\paren{\frac{1}{x+1}-\frac{x-2}{x^2-x+1}}\]
        を考えることにより,次のように計算できる.
        \begin{align*}
            \int^1_0\frac{dx}{x^3+1}&=\frac{1}{3}\int^1_0\paren{\frac{1}{x+1}-\frac{x-2}{x^2-x+1}}dx\\
            &=\frac{1}{3}\int^1_0\frac{dx}{x+1}-\frac{1}{6}\int^1_0\frac{(x^2-x+1)'}{x^2-x+1}dx+\frac{2}{3}\int^1_0\frac{dx}{x^2-x+1}\\
            &=\frac{\log 2}{3}+\frac{4}{9\sqrt{3}}\pi.
        \end{align*}
    \end{enumerate}
\end{proof}

\begin{tcolorbox}[colframe=ForestGreen, colback=ForestGreen!10!white,breakable,colbacktitle=ForestGreen!40!white,coltitle=black,fonttitle=\bfseries\sffamily,
    title=第3問]
    この問題で登場する行列は全て$n\times n$の可逆行列であるとする.ブロック行列
    \[A=\begin{pmatrix}
        A_{11}&A_{12}\\A_{21}&A_{22}
    \end{pmatrix},\quad B=\begin{pmatrix}B_{11}&O\\B_{21}&B_{22}\end{pmatrix},\quad A_{11},A_{12},A_{21},A_{22},B_{11},B_{21},B_{22}\in\GL_n(\R).\]
    を考える.
    \begin{enumerate}
        \item 次を満たす行列$Q_1\in M_n(\R)$を$A_{11},A_{12},A_{21},A_{22}$とその逆行列を用いて表せ:
        \[\begin{pmatrix}
            A_{11}&A_{12}\\A_{21}&A_{22}
        \end{pmatrix}=\begin{pmatrix}A_{11}&O\\A_{21}&Q_1\end{pmatrix}\begin{pmatrix}I_n&Q_2\\O&I_n\end{pmatrix}.\]
        \item 次を満たす行列$Q_3\in M_n(\R)$を$B_{11},B_{21},B_{22}$とその逆行列を用いて表せ:
        \[\begin{pmatrix}
            B_{11}&O\\B_{21}&B_{22}
        \end{pmatrix}^{-1}=\begin{pmatrix}
            B_{11}^{-1}&O\\Q_3&B_{22}^{-1}
        \end{pmatrix}\]
        \item 次の行列の成分$A^{11},A^{12},A^{21},A^{22}$を$A_{11},A_{12},A_{21},A_{22}$とその逆行列を用いて表せ:
        \[A^{-1}=\begin{pmatrix}
            A^{11}&A^{12}\\A^{21}&A^{22}
        \end{pmatrix}\]
        \item 次の等式を示せ:
        \[(A_{22}-A_{21}A_{11}^{-1}A_{12})^{-1}=A_{22}^{-1}+A_{22}^{-1}A_{21}(A_{11}-A_{12}A_{22}^{-1}A_{21})^{-1}A_{12}A_{22}^{-1}.\]
    \end{enumerate}
\end{tcolorbox}
\begin{proof}[\textbf{\underline{[解答例]}}]\mbox{}
    \begin{enumerate}
        \item 右辺を計算すると$\begin{pmatrix}A_{11}&A_{11}Q_2\\A_{21}&A_{21}Q_2+Q_1\end{pmatrix}$となるから,
        \[\begin{cases}
            A_{12}=A_{11}Q_2\quad\Leftrightarrow\quad Q_2=A_{11}^{-1}A_{12}\\
            A_{22}=A_{21}Q_2+Q_1
        \end{cases}\]
        が必要.よって,$Q_1=A_{22}-A_{21}A_{11}^{-1}A_{12}$.
        \item 
        \[\begin{pmatrix}I_n&O\\B_{21}B_{11}^{-1}+B_{22}Q_3&I_n\end{pmatrix}=\begin{pmatrix}
            B_{11}&O\\B_{21}&B_{22}
        \end{pmatrix}\begin{pmatrix}B_{11}^{-1}&O\\Q_3&B_{22}^{-1}\end{pmatrix}=I_{2n}\]
        が必要.
        よって,$Q_3=-B_{22}^{-1}B_{21}B_{11}^{-1}$.
        \item まず,$\begin{pmatrix}I_n&Q_2\\O&I_n\end{pmatrix}$の逆行列は$\begin{pmatrix}I_n&-Q_2\\O&I_n\end{pmatrix}$である.
        実際,
        \[\begin{pmatrix}I_n&Q_2\\O&I_n\end{pmatrix}\begin{pmatrix}I_n&-Q_2\\O&I_n\end{pmatrix}=I_{2n},\quad \begin{pmatrix}I_n&-Q_2\\O&I_n\end{pmatrix}\begin{pmatrix}I_n&Q_2\\O&I_n\end{pmatrix}=I_{2n}\]
        が成り立つ.よって,
        \begin{align*}
            \begin{pmatrix}
                A_{11}&A_{12}\\A_{21}&A_{22}
            \end{pmatrix}^{-1}&=\begin{pmatrix}I_n&Q_2\\O&I_n\end{pmatrix}^{-1}\begin{pmatrix}A_{11}&O\\A_{21}&Q_1\end{pmatrix}^{-1}\\
            &=\begin{pmatrix}I_n&-Q_2\\O&I_n\end{pmatrix}\begin{pmatrix}A_{11}^{-1}&O\\-Q_1^{-1}A_{21}A_{11}^{-1}&Q_1^{-1}\end{pmatrix}\\
            &=\begin{pmatrix}I_n&-A_{11}^{-1}A_{12}\\O&I_n\end{pmatrix}\begin{pmatrix}A_{11}^{-1}&O\\-(A_{22}-A_{21}A_{11}^{-1}A_{12})^{-1}A_{21}A_{11}^{-1}&(A_{22}-A_{21}A_{11}^{-1}A_{12})^{-1}\end{pmatrix}\\
            &=\begin{pmatrix}A_{11}^{-1}+A_{11}^{-1}A_{12}(A_{22}-A_{21}A_{11}^{-1}A_{12})^{-1}A_{21}A_{11}^{-1}&-A_{11}^{-1}A_{12}(A_{22}-A_{21}A_{11}^{-1}A_{12})^{-1}\\-(A_{22}-A_{21}A_{11}^{-1}A_{12})^{-1}A_{21}A_{11}^{-1}&(A_{22}-A_{21}A_{11}^{-1}A_{12})^{-1}\end{pmatrix}
        \end{align*}
        \item $A^{-1}$をもう一通りで表す.元の行列のUL分解を考えると,
        \[\mtrx{A_{11}}{A_{12}}{A_{21}}{A_{22}}=\mtrx{I}{R_2}{O}{I}\mtrx{R_1}{O}{A_{21}}{A_{22}}=\mtrx{R_1+R_2A_{21}}{R_2A_{22}}{A_{21}}{A_{22}}\]
        より,
        \[R_2=A_{12}A_{22}^{-1},\quad R_1=A_{11}-A_{12}A_{22}^{-1}A_{21}.\]
        続いて,このL部分の逆行列は
        \[\mtrx{R_1^{-1}}{O}{-A_{22}^{-1}A_{21}R_1^{-1}}{A_{22}^{-1}}\]
        と表せるから,
        \begin{align*}
            \mtrx{A_{11}}{A_{12}}{A_{21}}{A_{22}}^{-1}&=\paren{\mtrx{I}{R_2}{O}{I}\mtrx{R_1}{O}{A_{21}}{A_{22}}}^{-1}\\
            &=\mtrx{R_1^{-1}}{O}{-A_{22}^{-1}A_{21}R_1^{-1}}{A_{22}^{-1}}\mtrx{I}{-R_2}{O}{I}\\
            &=\mtrx{R_1^{-1}}{-R_1^{-1}R_2}{-A_{22}^{-1}A_{21}R_1^{-1}}{A_{22}^{-1}A_{21}R_1^{-1}R_2+A_{22}^{-1}}\\
            &=\mtrx{(A_{11}-A_{12}A_{22}^{-1}A_{21})^{-1}}{-(A_{11}-A_{12}A_{22}^{-1}A_{21})^{-1}A_{12}A_{22}^{-1}}{-A_{22}^{-1}A_{12}(A_{11}-A_{12}A_{22}^{-1}A_{21})^{-1}}{A_{22}^{-1}A_{21}(A_{11}-A_{12}A_{22}^{-1}A_{21})^{-1}A_{12}A_{22}^{-1}+A_{22}^{-1}}.
        \end{align*}
        以上より,$A^{-1}$の成分$A^{22}$を比べることで,
        \[(A_{22}-A_{21}A_{11}^{-1}A_{12})^{-1}=A_{22}^{-1}A_{21}(A_{11}-A_{12}A_{22}^{-1}A_{21})^{-1}A_{12}A_{22}^{-1}+A_{22}^{-1}\]
    \end{enumerate}
\end{proof}

\begin{tcolorbox}[colframe=ForestGreen, colback=ForestGreen!10!white,breakable,colbacktitle=ForestGreen!40!white,coltitle=black,fonttitle=\bfseries\sffamily,
    title=第4問]
    $X_1,X_2,X_3$を独立同分布確率変数とし,その分布関数を$F$で表す.
    $F$は狭義単調増加かつ連続で,従って逆関数$F^{-1}$を持つとする.
    $E_1,E_2,E_3$を標準指数分布$\Exp(1)$に従う独立同分布確率変数とする.
    それぞれ3つの中で,昇順の並び替えを$X_{(1)}\le X_{(2)}\le X_{(3)},E_{(1)}\le E_{(2)}\le E_{(3)}$とする.
    \begin{enumerate}
        \item $(F^{-1}(e^{-E_{(1)}}),F^{-1}(e^{-E_{(2)}}),F^{-1}(e^{-E_{(3)}}))$と$(X_{(3)},X_{(2)},X_{(1)})$は同分布であることを示せ.
        \item 
    \end{enumerate}
\end{tcolorbox}
\begin{proof}[\textbf{\underline{[解答例]}}]\mbox{}
    \begin{enumerate}
        \item 一般に,$E_1\sim\Exp(1)$のとき,$e^{-E_1}\sim U([0,1])$であるから,
        \[P[F^{-1}(e^{-E_1})\le a]=P[e^{-E_1}\le F(a)]=F(a)\]
        より,$F^{-1}(e^{-E_1})$は$X_1$に分布が等しい.
        後は順序を考えると,$e^{-E_{(1)}}\ge e^{-E_{(2)}}\ge e^{-E_{(3)}}$で,$F^{-1}$も順序を保存するから,
        分布は$X_1,X_2,X_3$を降順に並び替えた確率ベクトル,すなわち$(X_{(3)},X_{(2)},X_{(1)})$に等しい.
        \item 
    \end{enumerate}
\end{proof}

\end{document}