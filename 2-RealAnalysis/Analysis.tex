\documentclass[uplatex, dvipdfmx]{jsreport}
\title{微分積分論再建}
\author{司馬博文}
\date{\today}
\pagestyle{headings} \setcounter{secnumdepth}{4}
%%%%%%%%%%%%%%% 数理文書の組版 %%%%%%%%%%%%%%%

\usepackage{mathtools} %内部でamsmathを呼び出すことに注意.
%\mathtoolsset{showonlyrefs=true} %labelを附した数式にのみ附番される設定.
\usepackage{amsfonts} %mathfrak, mathcal, mathbbなど.
\usepackage{amsthm} %定理環境.
\usepackage{amssymb} %AMSFontsを使うためのパッケージ.
\usepackage{ascmac} %screen, itembox, shadebox環境.全てLATEX2εの標準機能の範囲で作られたもの.
\usepackage{comment} %comment環境を用いて,複数行をcomment outできるようにするpackage
\usepackage{wrapfig} %図の周りに文字をwrapさせることができる.詳細な制御ができる.
\usepackage[usenames, dvipsnames]{xcolor} %xcolorはcolorの拡張.optionの意味はdvipsnamesはLoad a set of predefined colors. forestgreenなどの色が追加されている.usenamesはobsoleteとだけ書いてあった.
\setcounter{tocdepth}{2} %目次に表示される深さ.2はsubsectionまで
\usepackage{multicol} %\begin{multicols}{2}環境で途中からmulticolumnに出来る.
\usepackage{mathabx}\newcommand{\wc}{\widecheck} %\widecheckなどのフォントパッケージ

%%%%%%%%%%%%%%% フォント %%%%%%%%%%%%%%%

\usepackage{textcomp, mathcomp} %Text Companionとは,T1 encodingに入らなかった文字群.これを使うためのパッケージ.\textsectionでブルバキに!
\usepackage[T1]{fontenc} %8bitエンコーディングにする.comp系拡張数学文字の動作が安定する.

%%%%%%%%%%%%%%% 一般文書の組版 %%%%%%%%%%%%%%%

\definecolor{花緑青}{cmyk}{1,0.07,0.10,0.10}\definecolor{サーモンピンク}{cmyk}{0,0.65,0.65,0.05}\definecolor{暗中模索}{rgb}{0.2,0.2,0.2}
\usepackage{url}\usepackage[dvipdfmx,colorlinks,linkcolor=花緑青,urlcolor=花緑青,citecolor=花緑青]{hyperref} %生成されるPDFファイルにおいて、\tableofcontentsによって書き出された目次をクリックすると該当する見出しへジャンプしたり、さらには、\label{ラベル名}を番号で参照する\ref{ラベル名}やthebibliography環境において\bibitem{ラベル名}を文献番号で参照する\cite{ラベル名}においても番号をクリックすると該当箇所にジャンプする.囲み枠はダサいので,colorlinksで囲み廃止し,リンク自体に色を付けることにした.
\usepackage{pxjahyper} %pxrubrica同様,八登崇之さん.hyperrefは日本語pLaTeXに最適化されていないから,hyperrefとセットで,(u)pLaTeX+hyperref+dvipdfmxの組み合わせで日本語を含む「しおり」をもつPDF文書を作成する場合に必要となる機能を提供する
\usepackage{ulem} %取り消し線を引くためのパッケージ
\usepackage{pxrubrica} %日本語にルビをふる.八登崇之(やとうたかゆき)氏による.

%%%%%%%%%%%%%%% 科学文書の組版 %%%%%%%%%%%%%%%

\usepackage[version=4]{mhchem} %化学式をTikZで簡単に書くためのパッケージ.
\usepackage{chemfig} %化学構造式をTikZで描くためのパッケージ.
\usepackage{siunitx} %IS単位を書くためのパッケージ

%%%%%%%%%%%%%%% 作図 %%%%%%%%%%%%%%%

\usepackage{tikz}\usetikzlibrary{positioning,automata}\usepackage{tikz-cd}\usepackage[all]{xy}
\def\objectstyle{\displaystyle} %デフォルトではxymatrix中の数式が文中数式モードになるので,それを直す.\labelstyleも同様にxy packageの中で定義されており,文中数式モードになっている.

\usepackage{graphicx} %rotatebox, scalebox, reflectbox, resizeboxなどのコマンドや,図表の読み込み\includegraphicsを司る.graphics というパッケージもありますが,graphicx はこれを高機能にしたものと考えて結構です(ただし graphicx は内部で graphics を読み込みます)
\usepackage[top=15truemm,bottom=15truemm,left=10truemm,right=10truemm]{geometry} %足助さんからもらったオプション

%%%%%%%%%%%%%%% 参照 %%%%%%%%%%%%%%%
%参考文献リストを出力したい箇所に\bibliography{../mathematics.bib}を追記すると良い.

%\bibliographystyle{jplain}
%\bibliographystyle{jname}
\bibliographystyle{apalike}

%%%%%%%%%%%%%%% 計算機文書の組版 %%%%%%%%%%%%%%%

\usepackage[breakable]{tcolorbox} %加藤晃史さんがフル活用していたtcolorboxを,途中改ページ可能で.
\tcbuselibrary{theorems} %https://qiita.com/t_kemmochi/items/483b8fcdb5db8d1f5d5e
\usepackage{enumerate} %enumerate環境を凝らせる.

\usepackage{listings} %ソースコードを表示できる環境.多分もっといい方法ある.
\usepackage{jvlisting} %日本語のコメントアウトをする場合jlistingが必要
\lstset{ %ここからソースコードの表示に関する設定.lstlisting環境では,[caption=hoge,label=fuga]などのoptionを付けられる.
%[escapechar=!]とすると,LaTeXコマンドを使える.
  basicstyle={\ttfamily},
  identifierstyle={\small},
  commentstyle={\smallitshape},
  keywordstyle={\small\bfseries},
  ndkeywordstyle={\small},
  stringstyle={\small\ttfamily},
  frame={tb},
  breaklines=true,
  columns=[l]{fullflexible},
  numbers=left,
  xrightmargin=0zw,
  xleftmargin=3zw,
  numberstyle={\scriptsize},
  stepnumber=1,
  numbersep=1zw,
  lineskip=-0.5ex
}
%\makeatletter %caption番号を「[chapter番号].[section番号].[subsection番号]-[そのsubsection内においてn番目]」に変更
%    \AtBeginDocument{
%    \renewcommand*{\thelstlisting}{\arabic{chapter}.\arabic{section}.\arabic{lstlisting}}
%    \@addtoreset{lstlisting}{section}
%    }
%\makeatother
\renewcommand{\lstlistingname}{算譜} %caption名を"program"に変更

\newtcolorbox{tbox}[3][]{%
colframe=#2,colback=#2!10,coltitle=#2!20!black,title={#3},#1}

% 証明内の文字が小さくなる環境.
\newenvironment{Proof}[1][\bf\underline{[証明]}]{\proof[#1]\color{darkgray}}{\endproof}

%%%%%%%%%%%%%%% 数学記号のマクロ %%%%%%%%%%%%%%%

%%% 括弧類
\newcommand{\abs}[1]{\lvert#1\rvert}\newcommand{\Abs}[1]{\left|#1\right|}\newcommand{\norm}[1]{\|#1\|}\newcommand{\Norm}[1]{\left\|#1\right\|}\newcommand{\Brace}[1]{\left\{#1\right\}}\newcommand{\BRace}[1]{\biggl\{#1\biggr\}}\newcommand{\paren}[1]{\left(#1\right)}\newcommand{\Paren}[1]{\biggr(#1\biggl)}\newcommand{\bracket}[1]{\langle#1\rangle}\newcommand{\brac}[1]{\langle#1\rangle}\newcommand{\Bracket}[1]{\left\langle#1\right\rangle}\newcommand{\Brac}[1]{\left\langle#1\right\rangle}\newcommand{\bra}[1]{\left\langle#1\right|}\newcommand{\ket}[1]{\left|#1\right\rangle}\newcommand{\Square}[1]{\left[#1\right]}\newcommand{\SQuare}[1]{\biggl[#1\biggr]}
\renewcommand{\o}[1]{\overline{#1}}\renewcommand{\u}[1]{\underline{#1}}\newcommand{\wt}[1]{\widetilde{#1}}\newcommand{\wh}[1]{\widehat{#1}}
\newcommand{\pp}[2]{\frac{\partial #1}{\partial #2}}\newcommand{\ppp}[3]{\frac{\partial #1}{\partial #2\partial #3}}\newcommand{\dd}[2]{\frac{d #1}{d #2}}
\newcommand{\floor}[1]{\lfloor#1\rfloor}\newcommand{\Floor}[1]{\left\lfloor#1\right\rfloor}\newcommand{\ceil}[1]{\lceil#1\rceil}
\newcommand{\ocinterval}[1]{(#1]}\newcommand{\cointerval}[1]{[#1)}\newcommand{\COinterval}[1]{\left[#1\right)}


%%% 予約語
\renewcommand{\iff}{\;\mathrm{iff}\;}
\newcommand{\False}{\mathrm{False}}\newcommand{\True}{\mathrm{True}}
\newcommand{\otherwise}{\mathrm{otherwise}}
\newcommand{\st}{\;\mathrm{s.t.}\;}

%%% 略記
\newcommand{\M}{\mathcal{M}}\newcommand{\cF}{\mathcal{F}}\newcommand{\cD}{\mathcal{D}}\newcommand{\fX}{\mathfrak{X}}\newcommand{\fY}{\mathfrak{Y}}\newcommand{\fZ}{\mathfrak{Z}}\renewcommand{\H}{\mathcal{H}}\newcommand{\fH}{\mathfrak{H}}\newcommand{\bH}{\mathbb{H}}\newcommand{\id}{\mathrm{id}}\newcommand{\A}{\mathcal{A}}\newcommand{\U}{\mathfrak{U}}
\newcommand{\lmd}{\lambda}
\newcommand{\Lmd}{\Lambda}

%%% 矢印類
\newcommand{\iso}{\xrightarrow{\,\smash{\raisebox{-0.45ex}{\ensuremath{\scriptstyle\sim}}}\,}}
\newcommand{\Lrarrow}{\;\;\Leftrightarrow\;\;}

%%% 注記
\newcommand{\rednote}[1]{\textcolor{red}{#1}}

% ノルム位相についての閉包 https://newbedev.com/how-to-make-double-overline-with-less-vertical-displacement
\makeatletter
\newcommand{\dbloverline}[1]{\overline{\dbl@overline{#1}}}
\newcommand{\dbl@overline}[1]{\mathpalette\dbl@@overline{#1}}
\newcommand{\dbl@@overline}[2]{%
  \begingroup
  \sbox\z@{$\m@th#1\overline{#2}$}%
  \ht\z@=\dimexpr\ht\z@-2\dbl@adjust{#1}\relax
  \box\z@
  \ifx#1\scriptstyle\kern-\scriptspace\else
  \ifx#1\scriptscriptstyle\kern-\scriptspace\fi\fi
  \endgroup
}
\newcommand{\dbl@adjust}[1]{%
  \fontdimen8
  \ifx#1\displaystyle\textfont\else
  \ifx#1\textstyle\textfont\else
  \ifx#1\scriptstyle\scriptfont\else
  \scriptscriptfont\fi\fi\fi 3
}
\makeatother
\newcommand{\oo}[1]{\dbloverline{#1}}

% hslashの他の文字Ver.
\newcommand{\hslashslash}{%
    \scalebox{1.2}{--
    }%
}
\newcommand{\dslash}{%
  {%
    \vphantom{d}%
    \ooalign{\kern.05em\smash{\hslashslash}\hidewidth\cr$d$\cr}%
    \kern.05em
  }%
}
\newcommand{\dint}{%
  {%
    \vphantom{d}%
    \ooalign{\kern.05em\smash{\hslashslash}\hidewidth\cr$\int$\cr}%
    \kern.05em
  }%
}
\newcommand{\dL}{%
  {%
    \vphantom{d}%
    \ooalign{\kern.05em\smash{\hslashslash}\hidewidth\cr$L$\cr}%
    \kern.05em
  }%
}

%%% 演算子
\DeclareMathOperator{\grad}{\mathrm{grad}}\DeclareMathOperator{\rot}{\mathrm{rot}}\DeclareMathOperator{\divergence}{\mathrm{div}}\DeclareMathOperator{\tr}{\mathrm{tr}}\newcommand{\pr}{\mathrm{pr}}
\newcommand{\Map}{\mathrm{Map}}\newcommand{\dom}{\mathrm{Dom}\;}\newcommand{\cod}{\mathrm{Cod}\;}\newcommand{\supp}{\mathrm{supp}\;}


%%% 線型代数学
\newcommand{\vctr}[2]{\begin{pmatrix}#1\\#2\end{pmatrix}}\newcommand{\vctrr}[3]{\begin{pmatrix}#1\\#2\\#3\end{pmatrix}}\newcommand{\mtrx}[4]{\begin{pmatrix}#1&#2\\#3&#4\end{pmatrix}}\newcommand{\smtrx}[4]{\paren{\begin{smallmatrix}#1&#2\\#3&#4\end{smallmatrix}}}\newcommand{\Ker}{\mathrm{Ker}\;}\newcommand{\Coker}{\mathrm{Coker}\;}\newcommand{\Coim}{\mathrm{Coim}\;}\DeclareMathOperator{\rank}{\mathrm{rank}}\newcommand{\lcm}{\mathrm{lcm}}\newcommand{\sgn}{\mathrm{sgn}\,}\newcommand{\GL}{\mathrm{GL}}\newcommand{\SL}{\mathrm{SL}}\newcommand{\alt}{\mathrm{alt}}
%%% 複素解析学
\renewcommand{\Re}{\mathrm{Re}\;}\renewcommand{\Im}{\mathrm{Im}\;}\newcommand{\Gal}{\mathrm{Gal}}\newcommand{\PGL}{\mathrm{PGL}}\newcommand{\PSL}{\mathrm{PSL}}\newcommand{\Log}{\mathrm{Log}\,}\newcommand{\Res}{\mathrm{Res}\,}\newcommand{\on}{\mathrm{on}\;}\newcommand{\hatC}{\widehat{\C}}\newcommand{\hatR}{\hat{\R}}\newcommand{\PV}{\mathrm{P.V.}}\newcommand{\diam}{\mathrm{diam}}\newcommand{\Area}{\mathrm{Area}}\newcommand{\Lap}{\Laplace}\newcommand{\f}{\mathbf{f}}\newcommand{\cR}{\mathcal{R}}\newcommand{\const}{\mathrm{const.}}\newcommand{\Om}{\Omega}\newcommand{\Cinf}{C^\infty}\newcommand{\ep}{\epsilon}\newcommand{\dist}{\mathrm{dist}}\newcommand{\opart}{\o{\partial}}\newcommand{\Length}{\mathrm{Length}}
%%% 集合と位相
\renewcommand{\O}{\mathcal{O}}\renewcommand{\S}{\mathcal{S}}\renewcommand{\U}{\mathcal{U}}\newcommand{\V}{\mathcal{V}}\renewcommand{\P}{\mathcal{P}}\newcommand{\R}{\mathbb{R}}\newcommand{\N}{\mathbb{N}}\newcommand{\C}{\mathbb{C}}\newcommand{\Z}{\mathbb{Z}}\newcommand{\Q}{\mathbb{Q}}\newcommand{\TV}{\mathrm{TV}}\newcommand{\ORD}{\mathrm{ORD}}\newcommand{\Tr}{\mathrm{Tr}}\newcommand{\Card}{\mathrm{Card}\;}\newcommand{\Top}{\mathrm{Top}}\newcommand{\Disc}{\mathrm{Disc}}\newcommand{\Codisc}{\mathrm{Codisc}}\newcommand{\CoDisc}{\mathrm{CoDisc}}\newcommand{\Ult}{\mathrm{Ult}}\newcommand{\ord}{\mathrm{ord}}\newcommand{\maj}{\mathrm{maj}}\newcommand{\bS}{\mathbb{S}}\newcommand{\PConn}{\mathrm{PConn}}

%%% 形式言語理論
\newcommand{\REGEX}{\mathrm{REGEX}}\newcommand{\RE}{\mathbf{RE}}
%%% Graph Theory
\newcommand{\SimpGph}{\mathrm{SimpGph}}\newcommand{\Gph}{\mathrm{Gph}}\newcommand{\mult}{\mathrm{mult}}\newcommand{\inv}{\mathrm{inv}}

%%% 多様体
\newcommand{\Der}{\mathrm{Der}}\newcommand{\osub}{\overset{\mathrm{open}}{\subset}}\newcommand{\osup}{\overset{\mathrm{open}}{\supset}}\newcommand{\al}{\alpha}\newcommand{\K}{\mathbb{K}}\newcommand{\Sp}{\mathrm{Sp}}\newcommand{\g}{\mathfrak{g}}\newcommand{\h}{\mathfrak{h}}\newcommand{\Exp}{\mathrm{Exp}\;}\newcommand{\Imm}{\mathrm{Imm}}\newcommand{\Imb}{\mathrm{Imb}}\newcommand{\codim}{\mathrm{codim}\;}\newcommand{\Gr}{\mathrm{Gr}}
%%% 代数
\newcommand{\Ad}{\mathrm{Ad}}\newcommand{\finsupp}{\mathrm{fin\;supp}}\newcommand{\SO}{\mathrm{SO}}\newcommand{\SU}{\mathrm{SU}}\newcommand{\acts}{\curvearrowright}\newcommand{\mono}{\hookrightarrow}\newcommand{\epi}{\twoheadrightarrow}\newcommand{\Stab}{\mathrm{Stab}}\newcommand{\nor}{\mathrm{nor}}\newcommand{\T}{\mathbb{T}}\newcommand{\Aff}{\mathrm{Aff}}\newcommand{\rsub}{\triangleleft}\newcommand{\rsup}{\triangleright}\newcommand{\subgrp}{\overset{\mathrm{subgrp}}{\subset}}\newcommand{\Ext}{\mathrm{Ext}}\newcommand{\sbs}{\subset}\newcommand{\sps}{\supset}\newcommand{\In}{\mathrm{in}\;}\newcommand{\Tor}{\mathrm{Tor}}\newcommand{\p}{\b{p}}\newcommand{\q}{\mathfrak{q}}\newcommand{\m}{\mathfrak{m}}\newcommand{\cS}{\mathcal{S}}\newcommand{\Frac}{\mathrm{Frac}\,}\newcommand{\Spec}{\mathrm{Spec}\,}\newcommand{\bA}{\mathbb{A}}\newcommand{\Sym}{\mathrm{Sym}}\newcommand{\Ann}{\mathrm{Ann}}\newcommand{\Her}{\mathrm{Her}}\newcommand{\Bil}{\mathrm{Bil}}\newcommand{\Ses}{\mathrm{Ses}}\newcommand{\FVS}{\mathrm{FVS}}
%%% 代数的位相幾何学
\newcommand{\Ho}{\mathrm{Ho}}\newcommand{\CW}{\mathrm{CW}}\newcommand{\lc}{\mathrm{lc}}\newcommand{\cg}{\mathrm{cg}}\newcommand{\Fib}{\mathrm{Fib}}\newcommand{\Cyl}{\mathrm{Cyl}}\newcommand{\Ch}{\mathrm{Ch}}
%%% 微分幾何学
\newcommand{\rE}{\mathrm{E}}\newcommand{\e}{\b{e}}\renewcommand{\k}{\b{k}}\newcommand{\Christ}[2]{\begin{Bmatrix}#1\\#2\end{Bmatrix}}\renewcommand{\Vec}[1]{\overrightarrow{\mathrm{#1}}}\newcommand{\hen}[1]{\mathrm{#1}}\renewcommand{\b}[1]{\boldsymbol{#1}}

%%% 函数解析
\newcommand{\HS}{\mathrm{HS}}\newcommand{\loc}{\mathrm{loc}}\newcommand{\Lh}{\mathrm{L.h.}}\newcommand{\Epi}{\mathrm{Epi}\;}\newcommand{\slim}{\mathrm{slim}}\newcommand{\Ban}{\mathrm{Ban}}\newcommand{\Hilb}{\mathrm{Hilb}}\newcommand{\Ex}{\mathrm{Ex}}\newcommand{\Co}{\mathrm{Co}}\newcommand{\sa}{\mathrm{sa}}\newcommand{\nnorm}[1]{{\left\vert\kern-0.25ex\left\vert\kern-0.25ex\left\vert #1 \right\vert\kern-0.25ex\right\vert\kern-0.25ex\right\vert}}\newcommand{\dvol}{\mathrm{dvol}}\newcommand{\Sconv}{\mathrm{Sconv}}\newcommand{\I}{\mathcal{I}}\newcommand{\nonunital}{\mathrm{nu}}\newcommand{\cpt}{\mathrm{cpt}}\newcommand{\lcpt}{\mathrm{lcpt}}\newcommand{\com}{\mathrm{com}}\newcommand{\Haus}{\mathrm{Haus}}\newcommand{\proper}{\mathrm{proper}}\newcommand{\infinity}{\mathrm{inf}}\newcommand{\TVS}{\mathrm{TVS}}\newcommand{\ess}{\mathrm{ess}}\newcommand{\ext}{\mathrm{ext}}\newcommand{\Index}{\mathrm{Index}\;}\newcommand{\SSR}{\mathrm{SSR}}\newcommand{\vs}{\mathrm{vs.}}\newcommand{\fM}{\mathfrak{M}}\newcommand{\EDM}{\mathrm{EDM}}\newcommand{\Tw}{\mathrm{Tw}}\newcommand{\fC}{\mathfrak{C}}\newcommand{\bn}{\boldsymbol{n}}\newcommand{\br}{\boldsymbol{r}}\newcommand{\Lam}{\Lambda}\newcommand{\lam}{\lambda}\newcommand{\one}{\mathbf{1}}\newcommand{\dae}{\text{-a.e.}}\newcommand{\das}{\text{-a.s.}}\newcommand{\td}{\text{-}}\newcommand{\RM}{\mathrm{RM}}\newcommand{\BV}{\mathrm{BV}}\newcommand{\normal}{\mathrm{normal}}\newcommand{\lub}{\mathrm{lub}\;}\newcommand{\Graph}{\mathrm{Graph}}\newcommand{\Ascent}{\mathrm{Ascent}}\newcommand{\Descent}{\mathrm{Descent}}\newcommand{\BIL}{\mathrm{BIL}}\newcommand{\fL}{\mathfrak{L}}\newcommand{\De}{\Delta}
%%% 積分論
\newcommand{\calA}{\mathcal{A}}\newcommand{\calB}{\mathcal{B}}\newcommand{\D}{\mathcal{D}}\newcommand{\Y}{\mathcal{Y}}\newcommand{\calC}{\mathcal{C}}\renewcommand{\ae}{\mathrm{a.e.}\;}\newcommand{\cZ}{\mathcal{Z}}\newcommand{\fF}{\mathfrak{F}}\newcommand{\fI}{\mathfrak{I}}\newcommand{\E}{\mathcal{E}}\newcommand{\sMap}{\sigma\textrm{-}\mathrm{Map}}\DeclareMathOperator*{\argmax}{arg\,max}\DeclareMathOperator*{\argmin}{arg\,min}\newcommand{\cC}{\mathcal{C}}\newcommand{\comp}{\complement}\newcommand{\J}{\mathcal{J}}\newcommand{\sumN}[1]{\sum_{#1\in\N}}\newcommand{\cupN}[1]{\cup_{#1\in\N}}\newcommand{\capN}[1]{\cap_{#1\in\N}}\newcommand{\Sum}[1]{\sum_{#1=1}^\infty}\newcommand{\sumn}{\sum_{n=1}^\infty}\newcommand{\summ}{\sum_{m=1}^\infty}\newcommand{\sumk}{\sum_{k=1}^\infty}\newcommand{\sumi}{\sum_{i=1}^\infty}\newcommand{\sumj}{\sum_{j=1}^\infty}\newcommand{\cupn}{\cup_{n=1}^\infty}\newcommand{\capn}{\cap_{n=1}^\infty}\newcommand{\cupk}{\cup_{k=1}^\infty}\newcommand{\cupi}{\cup_{i=1}^\infty}\newcommand{\cupj}{\cup_{j=1}^\infty}\newcommand{\limn}{\lim_{n\to\infty}}\renewcommand{\l}{\mathcal{l}}\renewcommand{\L}{\mathcal{L}}\newcommand{\Cl}{\mathrm{Cl}}\newcommand{\cN}{\mathcal{N}}\newcommand{\Ae}{\textrm{-a.e.}\;}\newcommand{\csub}{\overset{\textrm{closed}}{\subset}}\newcommand{\csup}{\overset{\textrm{closed}}{\supset}}\newcommand{\wB}{\wt{B}}\newcommand{\cG}{\mathcal{G}}\newcommand{\Lip}{\mathrm{Lip}}\DeclareMathOperator{\Dom}{\mathrm{Dom}}\newcommand{\AC}{\mathrm{AC}}\newcommand{\Mol}{\mathrm{Mol}}
%%% Fourier解析
\newcommand{\Pe}{\mathrm{Pe}}\newcommand{\wR}{\wh{\mathbb{\R}}}\newcommand*{\Laplace}{\mathop{}\!\mathbin\bigtriangleup}\newcommand*{\DAlambert}{\mathop{}\!\mathbin\Box}\newcommand{\bT}{\mathbb{T}}\newcommand{\dx}{\dslash x}\newcommand{\dt}{\dslash t}\newcommand{\ds}{\dslash s}
%%% 数値解析
\newcommand{\round}{\mathrm{round}}\newcommand{\cond}{\mathrm{cond}}\newcommand{\diag}{\mathrm{diag}}
\newcommand{\Adj}{\mathrm{Adj}}\newcommand{\Pf}{\mathrm{Pf}}\newcommand{\Sg}{\mathrm{Sg}}

%%% 確率論
\newcommand{\Prob}{\mathrm{Prob}}\newcommand{\X}{\mathcal{X}}\newcommand{\Meas}{\mathrm{Meas}}\newcommand{\as}{\;\mathrm{a.s.}}\newcommand{\io}{\;\mathrm{i.o.}}\newcommand{\fe}{\;\mathrm{f.e.}}\newcommand{\F}{\mathcal{F}}\newcommand{\bF}{\mathbb{F}}\newcommand{\W}{\mathcal{W}}\newcommand{\Pois}{\mathrm{Pois}}\newcommand{\iid}{\mathrm{i.i.d.}}\newcommand{\wconv}{\rightsquigarrow}\newcommand{\Var}{\mathrm{Var}}\newcommand{\xrightarrown}{\xrightarrow{n\to\infty}}\newcommand{\au}{\mathrm{au}}\newcommand{\cT}{\mathcal{T}}\newcommand{\wto}{\overset{w}{\to}}\newcommand{\dto}{\overset{d}{\to}}\newcommand{\pto}{\overset{p}{\to}}\newcommand{\vto}{\overset{v}{\to}}\newcommand{\Cont}{\mathrm{Cont}}\newcommand{\stably}{\mathrm{stably}}\newcommand{\Np}{\mathbb{N}^+}\newcommand{\oM}{\overline{\mathcal{M}}}\newcommand{\fP}{\mathfrak{P}}\newcommand{\sign}{\mathrm{sign}}\DeclareMathOperator{\Div}{Div}
\newcommand{\bD}{\mathbb{D}}\newcommand{\fW}{\mathfrak{W}}\newcommand{\DL}{\mathcal{D}\mathcal{L}}\renewcommand{\r}[1]{\mathrm{#1}}\newcommand{\rC}{\mathrm{C}}
%%% 情報理論
\newcommand{\bit}{\mathrm{bit}}\DeclareMathOperator{\sinc}{sinc}
%%% 量子論
\newcommand{\err}{\mathrm{err}}
%%% 最適化
\newcommand{\varparallel}{\mathbin{\!/\mkern-5mu/\!}}\newcommand{\Minimize}{\text{Minimize}}\newcommand{\subjectto}{\text{subject to}}\newcommand{\Ri}{\mathrm{Ri}}\newcommand{\Cone}{\mathrm{Cone}}\newcommand{\Int}{\mathrm{Int}}
%%% 数理ファイナンス
\newcommand{\pre}{\mathrm{pre}}\newcommand{\om}{\omega}

%%% 偏微分方程式
\let\div\relax
\DeclareMathOperator{\div}{div}\newcommand{\del}{\partial}
\newcommand{\LHS}{\mathrm{LHS}}\newcommand{\RHS}{\mathrm{RHS}}\newcommand{\bnu}{\boldsymbol{\nu}}\newcommand{\interior}{\mathrm{in}\;}\newcommand{\SH}{\mathrm{SH}}\renewcommand{\v}{\boldsymbol{\nu}}\newcommand{\n}{\mathbf{n}}\newcommand{\ssub}{\Subset}\newcommand{\curl}{\mathrm{curl}}
%%% 常微分方程式
\newcommand{\Ei}{\mathrm{Ei}}\newcommand{\sn}{\mathrm{sn}}\newcommand{\wgamma}{\widetilde{\gamma}}
%%% 統計力学
\newcommand{\Ens}{\mathrm{Ens}}
%%% 解析力学
\newcommand{\cl}{\mathrm{cl}}\newcommand{\x}{\boldsymbol{x}}

%%% 統計的因果推論
\newcommand{\Do}{\mathrm{Do}}
%%% 応用統計学
\newcommand{\mrl}{\mathrm{mrl}}
%%% 数理統計
\newcommand{\comb}[2]{\begin{pmatrix}#1\\#2\end{pmatrix}}\newcommand{\bP}{\mathbb{P}}\newcommand{\compsub}{\overset{\textrm{cpt}}{\subset}}\newcommand{\lip}{\textrm{lip}}\newcommand{\BL}{\mathrm{BL}}\newcommand{\G}{\mathbb{G}}\newcommand{\NB}{\mathrm{NB}}\newcommand{\oR}{\o{\R}}\newcommand{\liminfn}{\liminf_{n\to\infty}}\newcommand{\limsupn}{\limsup_{n\to\infty}}\newcommand{\esssup}{\mathrm{ess.sup}}\newcommand{\asto}{\xrightarrow{\as}}\newcommand{\Cov}{\mathrm{Cov}}\newcommand{\cQ}{\mathcal{Q}}\newcommand{\VC}{\mathrm{VC}}\newcommand{\mb}{\mathrm{mb}}\newcommand{\Avar}{\mathrm{Avar}}\newcommand{\bB}{\mathbb{B}}\newcommand{\bW}{\mathbb{W}}\newcommand{\sd}{\mathrm{sd}}\newcommand{\w}[1]{\widehat{#1}}\newcommand{\bZ}{\boldsymbol{Z}}\newcommand{\Bernoulli}{\mathrm{Ber}}\newcommand{\Ber}{\mathrm{Ber}}\newcommand{\Mult}{\mathrm{Mult}}\newcommand{\BPois}{\mathrm{BPois}}\newcommand{\fraks}{\mathfrak{s}}\newcommand{\frakk}{\mathfrak{k}}\newcommand{\IF}{\mathrm{IF}}\newcommand{\bX}{\mathbf{X}}\newcommand{\bx}{\boldsymbol{x}}\newcommand{\indep}{\raisebox{0.05em}{\rotatebox[origin=c]{90}{$\models$}}}\newcommand{\IG}{\mathrm{IG}}\newcommand{\Levy}{\mathrm{Levy}}\newcommand{\MP}{\mathrm{MP}}\newcommand{\Hermite}{\mathrm{Hermite}}\newcommand{\Skellam}{\mathrm{Skellam}}\newcommand{\Dirichlet}{\mathrm{Dirichlet}}\newcommand{\Beta}{\mathrm{Beta}}\newcommand{\bE}{\mathbb{E}}\newcommand{\bG}{\mathbb{G}}\newcommand{\MISE}{\mathrm{MISE}}\newcommand{\logit}{\mathtt{logit}}\newcommand{\expit}{\mathtt{expit}}\newcommand{\cK}{\mathcal{K}}\newcommand{\dl}{\dot{l}}\newcommand{\dotp}{\dot{p}}\newcommand{\wl}{\wt{l}}\newcommand{\Gauss}{\mathrm{Gauss}}\newcommand{\fA}{\mathfrak{A}}\newcommand{\under}{\mathrm{under}\;}\newcommand{\whtheta}{\wh{\theta}}\newcommand{\Em}{\mathrm{Em}}\newcommand{\ztheta}{{\theta_0}}
\newcommand{\rO}{\mathrm{O}}\newcommand{\Bin}{\mathrm{Bin}}\newcommand{\rW}{\mathrm{W}}\newcommand{\rG}{\mathrm{G}}\newcommand{\rB}{\mathrm{B}}\newcommand{\rN}{\mathrm{N}}\newcommand{\rU}{\mathrm{U}}\newcommand{\HG}{\mathrm{HG}}\newcommand{\GAMMA}{\mathrm{Gamma}}\newcommand{\Cauchy}{\mathrm{Cauchy}}\newcommand{\rt}{\mathrm{t}}
\DeclareMathOperator{\erf}{erf}

%%% 圏
\newcommand{\varlim}{\varprojlim}\newcommand{\Hom}{\mathrm{Hom}}\newcommand{\Iso}{\mathrm{Iso}}\newcommand{\Mor}{\mathrm{Mor}}\newcommand{\Isom}{\mathrm{Isom}}\newcommand{\Aut}{\mathrm{Aut}}\newcommand{\End}{\mathrm{End}}\newcommand{\op}{\mathrm{op}}\newcommand{\ev}{\mathrm{ev}}\newcommand{\Ob}{\mathrm{Ob}}\newcommand{\Ar}{\mathrm{Ar}}\newcommand{\Arr}{\mathrm{Arr}}\newcommand{\Set}{\mathrm{Set}}\newcommand{\Grp}{\mathrm{Grp}}\newcommand{\Cat}{\mathrm{Cat}}\newcommand{\Mon}{\mathrm{Mon}}\newcommand{\Ring}{\mathrm{Ring}}\newcommand{\CRing}{\mathrm{CRing}}\newcommand{\Ab}{\mathrm{Ab}}\newcommand{\Pos}{\mathrm{Pos}}\newcommand{\Vect}{\mathrm{Vect}}\newcommand{\FinVect}{\mathrm{FinVect}}\newcommand{\FinSet}{\mathrm{FinSet}}\newcommand{\FinMeas}{\mathrm{FinMeas}}\newcommand{\OmegaAlg}{\Omega\text{-}\mathrm{Alg}}\newcommand{\OmegaEAlg}{(\Omega,E)\text{-}\mathrm{Alg}}\newcommand{\Fun}{\mathrm{Fun}}\newcommand{\Func}{\mathrm{Func}}\newcommand{\Alg}{\mathrm{Alg}} %代数の圏
\newcommand{\CAlg}{\mathrm{CAlg}} %可換代数の圏
\newcommand{\Met}{\mathrm{Met}} %Metric space & Contraction maps
\newcommand{\Rel}{\mathrm{Rel}} %Sets & relation
\newcommand{\Bool}{\mathrm{Bool}}\newcommand{\CABool}{\mathrm{CABool}}\newcommand{\CompBoolAlg}{\mathrm{CompBoolAlg}}\newcommand{\BoolAlg}{\mathrm{BoolAlg}}\newcommand{\BoolRng}{\mathrm{BoolRng}}\newcommand{\HeytAlg}{\mathrm{HeytAlg}}\newcommand{\CompHeytAlg}{\mathrm{CompHeytAlg}}\newcommand{\Lat}{\mathrm{Lat}}\newcommand{\CompLat}{\mathrm{CompLat}}\newcommand{\SemiLat}{\mathrm{SemiLat}}\newcommand{\Stone}{\mathrm{Stone}}\newcommand{\Mfd}{\mathrm{Mfd}}\newcommand{\LieAlg}{\mathrm{LieAlg}}
\newcommand{\Sob}{\mathrm{Sob}} %Sober space & continuous map
\newcommand{\Op}{\mathrm{Op}} %Category of open subsets
\newcommand{\Sh}{\mathrm{Sh}} %Category of sheave
\newcommand{\PSh}{\mathrm{PSh}} %Category of presheave, PSh(C)=[C^op,set]のこと
\newcommand{\Conv}{\mathrm{Conv}} %Convergence spaceの圏
\newcommand{\Unif}{\mathrm{Unif}} %一様空間と一様連続写像の圏
\newcommand{\Frm}{\mathrm{Frm}} %フレームとフレームの射
\newcommand{\Locale}{\mathrm{Locale}} %その反対圏
\newcommand{\Diff}{\mathrm{Diff}} %滑らかな多様体の圏
\newcommand{\Quiv}{\mathrm{Quiv}} %Quiverの圏
\newcommand{\B}{\mathcal{B}}\newcommand{\Span}{\mathrm{Span}}\newcommand{\Corr}{\mathrm{Corr}}\newcommand{\Decat}{\mathrm{Decat}}\newcommand{\Rep}{\mathrm{Rep}}\newcommand{\Grpd}{\mathrm{Grpd}}\newcommand{\sSet}{\mathrm{sSet}}\newcommand{\Mod}{\mathrm{Mod}}\newcommand{\SmoothMnf}{\mathrm{SmoothMnf}}\newcommand{\coker}{\mathrm{coker}}\newcommand{\Ord}{\mathrm{Ord}}\newcommand{\eq}{\mathrm{eq}}\newcommand{\coeq}{\mathrm{coeq}}\newcommand{\act}{\mathrm{act}}

%%%%%%%%%%%%%%% 定理環境(足助先生ありがとうございます) %%%%%%%%%%%%%%%

\everymath{\displaystyle}
\renewcommand{\proofname}{\bf\underline{[証明]}}
\renewcommand{\thefootnote}{\dag\arabic{footnote}} %足助さんからもらった.どうなるんだ?
\renewcommand{\qedsymbol}{$\blacksquare$}

\renewcommand{\labelenumi}{(\arabic{enumi})} %(1),(2),...がデフォルトであって欲しい
\renewcommand{\labelenumii}{(\alph{enumii})}
\renewcommand{\labelenumiii}{(\roman{enumiii})}

\newtheoremstyle{StatementsWithUnderline}% ?name?
{3pt}% ?Space above? 1
{3pt}% ?Space below? 1
{}% ?Body font?
{}% ?Indent amount? 2
{\bfseries}% ?Theorem head font?
{\textbf{.}}% ?Punctuation after theorem head?
{.5em}% ?Space after theorem head? 3
{\textbf{\underline{\textup{#1~\thetheorem{}}}}\;\thmnote{(#3)}}% ?Theorem head spec (can be left empty, meaning ‘normal’)?

\usepackage{etoolbox}
\AtEndEnvironment{example}{\hfill\ensuremath{\Box}}
\AtEndEnvironment{observation}{\hfill\ensuremath{\Box}}

\theoremstyle{StatementsWithUnderline}
    \newtheorem{theorem}{定理}[section]
    \newtheorem{axiom}[theorem]{公理}
    \newtheorem{corollary}[theorem]{系}
    \newtheorem{proposition}[theorem]{命題}
    \newtheorem{lemma}[theorem]{補題}
    \newtheorem{definition}[theorem]{定義}
    \newtheorem{problem}[theorem]{問題}
    \newtheorem{exercise}[theorem]{Exercise}
\theoremstyle{definition}
    \newtheorem{issue}{論点}
    \newtheorem*{proposition*}{命題}
    \newtheorem*{lemma*}{補題}
    \newtheorem*{consideration*}{考察}
    \newtheorem*{theorem*}{定理}
    \newtheorem*{remarks*}{要諦}
    \newtheorem{example}[theorem]{例}
    \newtheorem{notation}[theorem]{記法}
    \newtheorem*{notation*}{記法}
    \newtheorem{assumption}[theorem]{仮定}
    \newtheorem{question}[theorem]{問}
    \newtheorem{counterexample}[theorem]{反例}
    \newtheorem{reidai}[theorem]{例題}
    \newtheorem{ruidai}[theorem]{類題}
    \newtheorem{algorithm}[theorem]{算譜}
    \newtheorem*{feels*}{所感}
    \newtheorem*{solution*}{\bf{[解]}}
    \newtheorem{discussion}[theorem]{議論}
    \newtheorem{synopsis}[theorem]{要約}
    \newtheorem{cited}[theorem]{引用}
    \newtheorem{remark}[theorem]{注}
    \newtheorem{remarks}[theorem]{要諦}
    \newtheorem{memo}[theorem]{メモ}
    \newtheorem{image}[theorem]{描像}
    \newtheorem{observation}[theorem]{観察}
    \newtheorem{universality}[theorem]{普遍性} %非自明な例外がない.
    \newtheorem{universal tendency}[theorem]{普遍傾向} %例外が有意に少ない.
    \newtheorem{hypothesis}[theorem]{仮説} %実験で説明されていない理論.
    \newtheorem{theory}[theorem]{理論} %実験事実とその(さしあたり)整合的な説明.
    \newtheorem{fact}[theorem]{実験事実}
    \newtheorem{model}[theorem]{模型}
    \newtheorem{explanation}[theorem]{説明} %理論による実験事実の説明
    \newtheorem{anomaly}[theorem]{理論の限界}
    \newtheorem{application}[theorem]{応用例}
    \newtheorem{method}[theorem]{手法} %実験手法など,技術的問題.
    \newtheorem{test}[theorem]{検定}
    \newtheorem{terms}[theorem]{用語}
    \newtheorem{solution}[theorem]{解法}
    \newtheorem{history}[theorem]{歴史}
    \newtheorem{usage}[theorem]{用語法}
    \newtheorem{research}[theorem]{研究}
    \newtheorem{shishin}[theorem]{指針}
    \newtheorem{yodan}[theorem]{余談}
    \newtheorem{construction}[theorem]{構成}
    \newtheorem{motivation}[theorem]{動機}
    \newtheorem{context}[theorem]{背景}
    \newtheorem{advantage}[theorem]{利点}
    \newtheorem*{definition*}{定義}
    \newtheorem*{remark*}{注意}
    \newtheorem*{question*}{問}
    \newtheorem*{problem*}{問題}
    \newtheorem*{axiom*}{公理}
    \newtheorem*{example*}{例}
    \newtheorem*{corollary*}{系}
    \newtheorem*{shishin*}{指針}
    \newtheorem*{yodan*}{余談}
    \newtheorem*{kadai*}{課題}

\raggedbottom
\allowdisplaybreaks
%%%%%%%%%%%%%%%% 数理文書の組版 %%%%%%%%%%%%%%%

\usepackage{mathtools} %内部でamsmathを呼び出すことに注意.
%\mathtoolsset{showonlyrefs=true} %labelを附した数式にのみ附番される設定.
\usepackage{amsfonts} %mathfrak, mathcal, mathbbなど.
\usepackage{amsthm} %定理環境.
\usepackage{amssymb} %AMSFontsを使うためのパッケージ.
\usepackage{ascmac} %screen, itembox, shadebox環境.全てLATEX2εの標準機能の範囲で作られたもの.
\usepackage{comment} %comment環境を用いて,複数行をcomment outできるようにするpackage
\usepackage{wrapfig} %図の周りに文字をwrapさせることができる.詳細な制御ができる.
\usepackage[usenames, dvipsnames]{xcolor} %xcolorはcolorの拡張.optionの意味はdvipsnamesはLoad a set of predefined colors. forestgreenなどの色が追加されている.usenamesはobsoleteとだけ書いてあった.
\setcounter{tocdepth}{2} %目次に表示される深さ.2はsubsectionまで
\usepackage{multicol} %\begin{multicols}{2}環境で途中からmulticolumnに出来る.
\usepackage{mathabx}\newcommand{\wc}{\widecheck} %\widecheckなどのフォントパッケージ

%%%%%%%%%%%%%%% フォント %%%%%%%%%%%%%%%

\usepackage{textcomp, mathcomp} %Text Companionとは,T1 encodingに入らなかった文字群.これを使うためのパッケージ.\textsectionでブルバキに!
\usepackage[T1]{fontenc} %8bitエンコーディングにする.comp系拡張数学文字の動作が安定する.

%%%%%%%%%%%%%%% 一般文書の組版 %%%%%%%%%%%%%%%

\definecolor{花緑青}{cmyk}{1,0.07,0.10,0.10}\definecolor{サーモンピンク}{cmyk}{0,0.65,0.65,0.05}\definecolor{暗中模索}{rgb}{0.2,0.2,0.2}
\usepackage{url}\usepackage[dvipdfmx,colorlinks,linkcolor=花緑青,urlcolor=花緑青,citecolor=花緑青]{hyperref} %生成されるPDFファイルにおいて、\tableofcontentsによって書き出された目次をクリックすると該当する見出しへジャンプしたり、さらには、\label{ラベル名}を番号で参照する\ref{ラベル名}やthebibliography環境において\bibitem{ラベル名}を文献番号で参照する\cite{ラベル名}においても番号をクリックすると該当箇所にジャンプする.囲み枠はダサいので,colorlinksで囲み廃止し,リンク自体に色を付けることにした.
\usepackage{pxjahyper} %pxrubrica同様,八登崇之さん.hyperrefは日本語pLaTeXに最適化されていないから,hyperrefとセットで,(u)pLaTeX+hyperref+dvipdfmxの組み合わせで日本語を含む「しおり」をもつPDF文書を作成する場合に必要となる機能を提供する
\usepackage{ulem} %取り消し線を引くためのパッケージ
\usepackage{pxrubrica} %日本語にルビをふる.八登崇之(やとうたかゆき)氏による.

%%%%%%%%%%%%%%% 科学文書の組版 %%%%%%%%%%%%%%%

\usepackage[version=4]{mhchem} %化学式をTikZで簡単に書くためのパッケージ.
\usepackage{chemfig} %化学構造式をTikZで描くためのパッケージ.
\usepackage{siunitx} %IS単位を書くためのパッケージ

%%%%%%%%%%%%%%% 作図 %%%%%%%%%%%%%%%

\usepackage{tikz}\usetikzlibrary{positioning,automata}\usepackage{tikz-cd}\usepackage[all]{xy}
\def\objectstyle{\displaystyle} %デフォルトではxymatrix中の数式が文中数式モードになるので,それを直す.\labelstyleも同様にxy packageの中で定義されており,文中数式モードになっている.

\usepackage{graphicx} %rotatebox, scalebox, reflectbox, resizeboxなどのコマンドや,図表の読み込み\includegraphicsを司る.graphics というパッケージもありますが,graphicx はこれを高機能にしたものと考えて結構です(ただし graphicx は内部で graphics を読み込みます)
\usepackage[top=15truemm,bottom=15truemm,left=10truemm,right=10truemm]{geometry} %足助さんからもらったオプション

%%%%%%%%%%%%%%% 参照 %%%%%%%%%%%%%%%
%参考文献リストを出力したい箇所に\bibliography{../mathematics.bib}を追記すると良い.

%\bibliographystyle{jplain}
%\bibliographystyle{jname}
\bibliographystyle{apalike}

%%%%%%%%%%%%%%% 計算機文書の組版 %%%%%%%%%%%%%%%

\usepackage[breakable]{tcolorbox} %加藤晃史さんがフル活用していたtcolorboxを,途中改ページ可能で.
\tcbuselibrary{theorems} %https://qiita.com/t_kemmochi/items/483b8fcdb5db8d1f5d5e
\usepackage{enumerate} %enumerate環境を凝らせる.

\usepackage{listings} %ソースコードを表示できる環境.多分もっといい方法ある.
\usepackage{jvlisting} %日本語のコメントアウトをする場合jlistingが必要
\lstset{ %ここからソースコードの表示に関する設定.lstlisting環境では,[caption=hoge,label=fuga]などのoptionを付けられる.
%[escapechar=!]とすると,LaTeXコマンドを使える.
  basicstyle={\ttfamily},
  identifierstyle={\small},
  commentstyle={\smallitshape},
  keywordstyle={\small\bfseries},
  ndkeywordstyle={\small},
  stringstyle={\small\ttfamily},
  frame={tb},
  breaklines=true,
  columns=[l]{fullflexible},
  numbers=left,
  xrightmargin=0zw,
  xleftmargin=3zw,
  numberstyle={\scriptsize},
  stepnumber=1,
  numbersep=1zw,
  lineskip=-0.5ex
}
%\makeatletter %caption番号を「[chapter番号].[section番号].[subsection番号]-[そのsubsection内においてn番目]」に変更
%    \AtBeginDocument{
%    \renewcommand*{\thelstlisting}{\arabic{chapter}.\arabic{section}.\arabic{lstlisting}}
%    \@addtoreset{lstlisting}{section}
%    }
%\makeatother
\renewcommand{\lstlistingname}{算譜} %caption名を"program"に変更

\newtcolorbox{tbox}[3][]{%
colframe=#2,colback=#2!10,coltitle=#2!20!black,title={#3},#1}

% 証明内の文字が小さくなる環境.
\newenvironment{Proof}[1][\bf\underline{[証明]}]{\proof[#1]\color{darkgray}}{\endproof}

%%%%%%%%%%%%%%% 数学記号のマクロ %%%%%%%%%%%%%%%

%%% 括弧類
\newcommand{\abs}[1]{\lvert#1\rvert}\newcommand{\Abs}[1]{\left|#1\right|}\newcommand{\norm}[1]{\|#1\|}\newcommand{\Norm}[1]{\left\|#1\right\|}\newcommand{\Brace}[1]{\left\{#1\right\}}\newcommand{\BRace}[1]{\biggl\{#1\biggr\}}\newcommand{\paren}[1]{\left(#1\right)}\newcommand{\Paren}[1]{\biggr(#1\biggl)}\newcommand{\bracket}[1]{\langle#1\rangle}\newcommand{\brac}[1]{\langle#1\rangle}\newcommand{\Bracket}[1]{\left\langle#1\right\rangle}\newcommand{\Brac}[1]{\left\langle#1\right\rangle}\newcommand{\bra}[1]{\left\langle#1\right|}\newcommand{\ket}[1]{\left|#1\right\rangle}\newcommand{\Square}[1]{\left[#1\right]}\newcommand{\SQuare}[1]{\biggl[#1\biggr]}
\renewcommand{\o}[1]{\overline{#1}}\renewcommand{\u}[1]{\underline{#1}}\newcommand{\wt}[1]{\widetilde{#1}}\newcommand{\wh}[1]{\widehat{#1}}
\newcommand{\pp}[2]{\frac{\partial #1}{\partial #2}}\newcommand{\ppp}[3]{\frac{\partial #1}{\partial #2\partial #3}}\newcommand{\dd}[2]{\frac{d #1}{d #2}}
\newcommand{\floor}[1]{\lfloor#1\rfloor}\newcommand{\Floor}[1]{\left\lfloor#1\right\rfloor}\newcommand{\ceil}[1]{\lceil#1\rceil}
\newcommand{\ocinterval}[1]{(#1]}\newcommand{\cointerval}[1]{[#1)}\newcommand{\COinterval}[1]{\left[#1\right)}


%%% 予約語
\renewcommand{\iff}{\;\mathrm{iff}\;}
\newcommand{\False}{\mathrm{False}}\newcommand{\True}{\mathrm{True}}
\newcommand{\otherwise}{\mathrm{otherwise}}
\newcommand{\st}{\;\mathrm{s.t.}\;}

%%% 略記
\newcommand{\M}{\mathcal{M}}\newcommand{\cF}{\mathcal{F}}\newcommand{\cD}{\mathcal{D}}\newcommand{\fX}{\mathfrak{X}}\newcommand{\fY}{\mathfrak{Y}}\newcommand{\fZ}{\mathfrak{Z}}\renewcommand{\H}{\mathcal{H}}\newcommand{\fH}{\mathfrak{H}}\newcommand{\bH}{\mathbb{H}}\newcommand{\id}{\mathrm{id}}\newcommand{\A}{\mathcal{A}}\newcommand{\U}{\mathfrak{U}}
\newcommand{\lmd}{\lambda}
\newcommand{\Lmd}{\Lambda}

%%% 矢印類
\newcommand{\iso}{\xrightarrow{\,\smash{\raisebox{-0.45ex}{\ensuremath{\scriptstyle\sim}}}\,}}
\newcommand{\Lrarrow}{\;\;\Leftrightarrow\;\;}

%%% 注記
\newcommand{\rednote}[1]{\textcolor{red}{#1}}

% ノルム位相についての閉包 https://newbedev.com/how-to-make-double-overline-with-less-vertical-displacement
\makeatletter
\newcommand{\dbloverline}[1]{\overline{\dbl@overline{#1}}}
\newcommand{\dbl@overline}[1]{\mathpalette\dbl@@overline{#1}}
\newcommand{\dbl@@overline}[2]{%
  \begingroup
  \sbox\z@{$\m@th#1\overline{#2}$}%
  \ht\z@=\dimexpr\ht\z@-2\dbl@adjust{#1}\relax
  \box\z@
  \ifx#1\scriptstyle\kern-\scriptspace\else
  \ifx#1\scriptscriptstyle\kern-\scriptspace\fi\fi
  \endgroup
}
\newcommand{\dbl@adjust}[1]{%
  \fontdimen8
  \ifx#1\displaystyle\textfont\else
  \ifx#1\textstyle\textfont\else
  \ifx#1\scriptstyle\scriptfont\else
  \scriptscriptfont\fi\fi\fi 3
}
\makeatother
\newcommand{\oo}[1]{\dbloverline{#1}}

% hslashの他の文字Ver.
\newcommand{\hslashslash}{%
    \scalebox{1.2}{--
    }%
}
\newcommand{\dslash}{%
  {%
    \vphantom{d}%
    \ooalign{\kern.05em\smash{\hslashslash}\hidewidth\cr$d$\cr}%
    \kern.05em
  }%
}
\newcommand{\dint}{%
  {%
    \vphantom{d}%
    \ooalign{\kern.05em\smash{\hslashslash}\hidewidth\cr$\int$\cr}%
    \kern.05em
  }%
}
\newcommand{\dL}{%
  {%
    \vphantom{d}%
    \ooalign{\kern.05em\smash{\hslashslash}\hidewidth\cr$L$\cr}%
    \kern.05em
  }%
}

%%% 演算子
\DeclareMathOperator{\grad}{\mathrm{grad}}\DeclareMathOperator{\rot}{\mathrm{rot}}\DeclareMathOperator{\divergence}{\mathrm{div}}\DeclareMathOperator{\tr}{\mathrm{tr}}\newcommand{\pr}{\mathrm{pr}}
\newcommand{\Map}{\mathrm{Map}}\newcommand{\dom}{\mathrm{Dom}\;}\newcommand{\cod}{\mathrm{Cod}\;}\newcommand{\supp}{\mathrm{supp}\;}


%%% 線型代数学
\newcommand{\vctr}[2]{\begin{pmatrix}#1\\#2\end{pmatrix}}\newcommand{\vctrr}[3]{\begin{pmatrix}#1\\#2\\#3\end{pmatrix}}\newcommand{\mtrx}[4]{\begin{pmatrix}#1&#2\\#3&#4\end{pmatrix}}\newcommand{\smtrx}[4]{\paren{\begin{smallmatrix}#1&#2\\#3&#4\end{smallmatrix}}}\newcommand{\Ker}{\mathrm{Ker}\;}\newcommand{\Coker}{\mathrm{Coker}\;}\newcommand{\Coim}{\mathrm{Coim}\;}\DeclareMathOperator{\rank}{\mathrm{rank}}\newcommand{\lcm}{\mathrm{lcm}}\newcommand{\sgn}{\mathrm{sgn}\,}\newcommand{\GL}{\mathrm{GL}}\newcommand{\SL}{\mathrm{SL}}\newcommand{\alt}{\mathrm{alt}}
%%% 複素解析学
\renewcommand{\Re}{\mathrm{Re}\;}\renewcommand{\Im}{\mathrm{Im}\;}\newcommand{\Gal}{\mathrm{Gal}}\newcommand{\PGL}{\mathrm{PGL}}\newcommand{\PSL}{\mathrm{PSL}}\newcommand{\Log}{\mathrm{Log}\,}\newcommand{\Res}{\mathrm{Res}\,}\newcommand{\on}{\mathrm{on}\;}\newcommand{\hatC}{\widehat{\C}}\newcommand{\hatR}{\hat{\R}}\newcommand{\PV}{\mathrm{P.V.}}\newcommand{\diam}{\mathrm{diam}}\newcommand{\Area}{\mathrm{Area}}\newcommand{\Lap}{\Laplace}\newcommand{\f}{\mathbf{f}}\newcommand{\cR}{\mathcal{R}}\newcommand{\const}{\mathrm{const.}}\newcommand{\Om}{\Omega}\newcommand{\Cinf}{C^\infty}\newcommand{\ep}{\epsilon}\newcommand{\dist}{\mathrm{dist}}\newcommand{\opart}{\o{\partial}}\newcommand{\Length}{\mathrm{Length}}
%%% 集合と位相
\renewcommand{\O}{\mathcal{O}}\renewcommand{\S}{\mathcal{S}}\renewcommand{\U}{\mathcal{U}}\newcommand{\V}{\mathcal{V}}\renewcommand{\P}{\mathcal{P}}\newcommand{\R}{\mathbb{R}}\newcommand{\N}{\mathbb{N}}\newcommand{\C}{\mathbb{C}}\newcommand{\Z}{\mathbb{Z}}\newcommand{\Q}{\mathbb{Q}}\newcommand{\TV}{\mathrm{TV}}\newcommand{\ORD}{\mathrm{ORD}}\newcommand{\Tr}{\mathrm{Tr}}\newcommand{\Card}{\mathrm{Card}\;}\newcommand{\Top}{\mathrm{Top}}\newcommand{\Disc}{\mathrm{Disc}}\newcommand{\Codisc}{\mathrm{Codisc}}\newcommand{\CoDisc}{\mathrm{CoDisc}}\newcommand{\Ult}{\mathrm{Ult}}\newcommand{\ord}{\mathrm{ord}}\newcommand{\maj}{\mathrm{maj}}\newcommand{\bS}{\mathbb{S}}\newcommand{\PConn}{\mathrm{PConn}}

%%% 形式言語理論
\newcommand{\REGEX}{\mathrm{REGEX}}\newcommand{\RE}{\mathbf{RE}}
%%% Graph Theory
\newcommand{\SimpGph}{\mathrm{SimpGph}}\newcommand{\Gph}{\mathrm{Gph}}\newcommand{\mult}{\mathrm{mult}}\newcommand{\inv}{\mathrm{inv}}

%%% 多様体
\newcommand{\Der}{\mathrm{Der}}\newcommand{\osub}{\overset{\mathrm{open}}{\subset}}\newcommand{\osup}{\overset{\mathrm{open}}{\supset}}\newcommand{\al}{\alpha}\newcommand{\K}{\mathbb{K}}\newcommand{\Sp}{\mathrm{Sp}}\newcommand{\g}{\mathfrak{g}}\newcommand{\h}{\mathfrak{h}}\newcommand{\Exp}{\mathrm{Exp}\;}\newcommand{\Imm}{\mathrm{Imm}}\newcommand{\Imb}{\mathrm{Imb}}\newcommand{\codim}{\mathrm{codim}\;}\newcommand{\Gr}{\mathrm{Gr}}
%%% 代数
\newcommand{\Ad}{\mathrm{Ad}}\newcommand{\finsupp}{\mathrm{fin\;supp}}\newcommand{\SO}{\mathrm{SO}}\newcommand{\SU}{\mathrm{SU}}\newcommand{\acts}{\curvearrowright}\newcommand{\mono}{\hookrightarrow}\newcommand{\epi}{\twoheadrightarrow}\newcommand{\Stab}{\mathrm{Stab}}\newcommand{\nor}{\mathrm{nor}}\newcommand{\T}{\mathbb{T}}\newcommand{\Aff}{\mathrm{Aff}}\newcommand{\rsub}{\triangleleft}\newcommand{\rsup}{\triangleright}\newcommand{\subgrp}{\overset{\mathrm{subgrp}}{\subset}}\newcommand{\Ext}{\mathrm{Ext}}\newcommand{\sbs}{\subset}\newcommand{\sps}{\supset}\newcommand{\In}{\mathrm{in}\;}\newcommand{\Tor}{\mathrm{Tor}}\newcommand{\p}{\b{p}}\newcommand{\q}{\mathfrak{q}}\newcommand{\m}{\mathfrak{m}}\newcommand{\cS}{\mathcal{S}}\newcommand{\Frac}{\mathrm{Frac}\,}\newcommand{\Spec}{\mathrm{Spec}\,}\newcommand{\bA}{\mathbb{A}}\newcommand{\Sym}{\mathrm{Sym}}\newcommand{\Ann}{\mathrm{Ann}}\newcommand{\Her}{\mathrm{Her}}\newcommand{\Bil}{\mathrm{Bil}}\newcommand{\Ses}{\mathrm{Ses}}\newcommand{\FVS}{\mathrm{FVS}}
%%% 代数的位相幾何学
\newcommand{\Ho}{\mathrm{Ho}}\newcommand{\CW}{\mathrm{CW}}\newcommand{\lc}{\mathrm{lc}}\newcommand{\cg}{\mathrm{cg}}\newcommand{\Fib}{\mathrm{Fib}}\newcommand{\Cyl}{\mathrm{Cyl}}\newcommand{\Ch}{\mathrm{Ch}}
%%% 微分幾何学
\newcommand{\rE}{\mathrm{E}}\newcommand{\e}{\b{e}}\renewcommand{\k}{\b{k}}\newcommand{\Christ}[2]{\begin{Bmatrix}#1\\#2\end{Bmatrix}}\renewcommand{\Vec}[1]{\overrightarrow{\mathrm{#1}}}\newcommand{\hen}[1]{\mathrm{#1}}\renewcommand{\b}[1]{\boldsymbol{#1}}

%%% 函数解析
\newcommand{\HS}{\mathrm{HS}}\newcommand{\loc}{\mathrm{loc}}\newcommand{\Lh}{\mathrm{L.h.}}\newcommand{\Epi}{\mathrm{Epi}\;}\newcommand{\slim}{\mathrm{slim}}\newcommand{\Ban}{\mathrm{Ban}}\newcommand{\Hilb}{\mathrm{Hilb}}\newcommand{\Ex}{\mathrm{Ex}}\newcommand{\Co}{\mathrm{Co}}\newcommand{\sa}{\mathrm{sa}}\newcommand{\nnorm}[1]{{\left\vert\kern-0.25ex\left\vert\kern-0.25ex\left\vert #1 \right\vert\kern-0.25ex\right\vert\kern-0.25ex\right\vert}}\newcommand{\dvol}{\mathrm{dvol}}\newcommand{\Sconv}{\mathrm{Sconv}}\newcommand{\I}{\mathcal{I}}\newcommand{\nonunital}{\mathrm{nu}}\newcommand{\cpt}{\mathrm{cpt}}\newcommand{\lcpt}{\mathrm{lcpt}}\newcommand{\com}{\mathrm{com}}\newcommand{\Haus}{\mathrm{Haus}}\newcommand{\proper}{\mathrm{proper}}\newcommand{\infinity}{\mathrm{inf}}\newcommand{\TVS}{\mathrm{TVS}}\newcommand{\ess}{\mathrm{ess}}\newcommand{\ext}{\mathrm{ext}}\newcommand{\Index}{\mathrm{Index}\;}\newcommand{\SSR}{\mathrm{SSR}}\newcommand{\vs}{\mathrm{vs.}}\newcommand{\fM}{\mathfrak{M}}\newcommand{\EDM}{\mathrm{EDM}}\newcommand{\Tw}{\mathrm{Tw}}\newcommand{\fC}{\mathfrak{C}}\newcommand{\bn}{\boldsymbol{n}}\newcommand{\br}{\boldsymbol{r}}\newcommand{\Lam}{\Lambda}\newcommand{\lam}{\lambda}\newcommand{\one}{\mathbf{1}}\newcommand{\dae}{\text{-a.e.}}\newcommand{\das}{\text{-a.s.}}\newcommand{\td}{\text{-}}\newcommand{\RM}{\mathrm{RM}}\newcommand{\BV}{\mathrm{BV}}\newcommand{\normal}{\mathrm{normal}}\newcommand{\lub}{\mathrm{lub}\;}\newcommand{\Graph}{\mathrm{Graph}}\newcommand{\Ascent}{\mathrm{Ascent}}\newcommand{\Descent}{\mathrm{Descent}}\newcommand{\BIL}{\mathrm{BIL}}\newcommand{\fL}{\mathfrak{L}}\newcommand{\De}{\Delta}
%%% 積分論
\newcommand{\calA}{\mathcal{A}}\newcommand{\calB}{\mathcal{B}}\newcommand{\D}{\mathcal{D}}\newcommand{\Y}{\mathcal{Y}}\newcommand{\calC}{\mathcal{C}}\renewcommand{\ae}{\mathrm{a.e.}\;}\newcommand{\cZ}{\mathcal{Z}}\newcommand{\fF}{\mathfrak{F}}\newcommand{\fI}{\mathfrak{I}}\newcommand{\E}{\mathcal{E}}\newcommand{\sMap}{\sigma\textrm{-}\mathrm{Map}}\DeclareMathOperator*{\argmax}{arg\,max}\DeclareMathOperator*{\argmin}{arg\,min}\newcommand{\cC}{\mathcal{C}}\newcommand{\comp}{\complement}\newcommand{\J}{\mathcal{J}}\newcommand{\sumN}[1]{\sum_{#1\in\N}}\newcommand{\cupN}[1]{\cup_{#1\in\N}}\newcommand{\capN}[1]{\cap_{#1\in\N}}\newcommand{\Sum}[1]{\sum_{#1=1}^\infty}\newcommand{\sumn}{\sum_{n=1}^\infty}\newcommand{\summ}{\sum_{m=1}^\infty}\newcommand{\sumk}{\sum_{k=1}^\infty}\newcommand{\sumi}{\sum_{i=1}^\infty}\newcommand{\sumj}{\sum_{j=1}^\infty}\newcommand{\cupn}{\cup_{n=1}^\infty}\newcommand{\capn}{\cap_{n=1}^\infty}\newcommand{\cupk}{\cup_{k=1}^\infty}\newcommand{\cupi}{\cup_{i=1}^\infty}\newcommand{\cupj}{\cup_{j=1}^\infty}\newcommand{\limn}{\lim_{n\to\infty}}\renewcommand{\l}{\mathcal{l}}\renewcommand{\L}{\mathcal{L}}\newcommand{\Cl}{\mathrm{Cl}}\newcommand{\cN}{\mathcal{N}}\newcommand{\Ae}{\textrm{-a.e.}\;}\newcommand{\csub}{\overset{\textrm{closed}}{\subset}}\newcommand{\csup}{\overset{\textrm{closed}}{\supset}}\newcommand{\wB}{\wt{B}}\newcommand{\cG}{\mathcal{G}}\newcommand{\Lip}{\mathrm{Lip}}\DeclareMathOperator{\Dom}{\mathrm{Dom}}\newcommand{\AC}{\mathrm{AC}}\newcommand{\Mol}{\mathrm{Mol}}
%%% Fourier解析
\newcommand{\Pe}{\mathrm{Pe}}\newcommand{\wR}{\wh{\mathbb{\R}}}\newcommand*{\Laplace}{\mathop{}\!\mathbin\bigtriangleup}\newcommand*{\DAlambert}{\mathop{}\!\mathbin\Box}\newcommand{\bT}{\mathbb{T}}\newcommand{\dx}{\dslash x}\newcommand{\dt}{\dslash t}\newcommand{\ds}{\dslash s}
%%% 数値解析
\newcommand{\round}{\mathrm{round}}\newcommand{\cond}{\mathrm{cond}}\newcommand{\diag}{\mathrm{diag}}
\newcommand{\Adj}{\mathrm{Adj}}\newcommand{\Pf}{\mathrm{Pf}}\newcommand{\Sg}{\mathrm{Sg}}

%%% 確率論
\newcommand{\Prob}{\mathrm{Prob}}\newcommand{\X}{\mathcal{X}}\newcommand{\Meas}{\mathrm{Meas}}\newcommand{\as}{\;\mathrm{a.s.}}\newcommand{\io}{\;\mathrm{i.o.}}\newcommand{\fe}{\;\mathrm{f.e.}}\newcommand{\F}{\mathcal{F}}\newcommand{\bF}{\mathbb{F}}\newcommand{\W}{\mathcal{W}}\newcommand{\Pois}{\mathrm{Pois}}\newcommand{\iid}{\mathrm{i.i.d.}}\newcommand{\wconv}{\rightsquigarrow}\newcommand{\Var}{\mathrm{Var}}\newcommand{\xrightarrown}{\xrightarrow{n\to\infty}}\newcommand{\au}{\mathrm{au}}\newcommand{\cT}{\mathcal{T}}\newcommand{\wto}{\overset{w}{\to}}\newcommand{\dto}{\overset{d}{\to}}\newcommand{\pto}{\overset{p}{\to}}\newcommand{\vto}{\overset{v}{\to}}\newcommand{\Cont}{\mathrm{Cont}}\newcommand{\stably}{\mathrm{stably}}\newcommand{\Np}{\mathbb{N}^+}\newcommand{\oM}{\overline{\mathcal{M}}}\newcommand{\fP}{\mathfrak{P}}\newcommand{\sign}{\mathrm{sign}}\DeclareMathOperator{\Div}{Div}
\newcommand{\bD}{\mathbb{D}}\newcommand{\fW}{\mathfrak{W}}\newcommand{\DL}{\mathcal{D}\mathcal{L}}\renewcommand{\r}[1]{\mathrm{#1}}\newcommand{\rC}{\mathrm{C}}
%%% 情報理論
\newcommand{\bit}{\mathrm{bit}}\DeclareMathOperator{\sinc}{sinc}
%%% 量子論
\newcommand{\err}{\mathrm{err}}
%%% 最適化
\newcommand{\varparallel}{\mathbin{\!/\mkern-5mu/\!}}\newcommand{\Minimize}{\text{Minimize}}\newcommand{\subjectto}{\text{subject to}}\newcommand{\Ri}{\mathrm{Ri}}\newcommand{\Cone}{\mathrm{Cone}}\newcommand{\Int}{\mathrm{Int}}
%%% 数理ファイナンス
\newcommand{\pre}{\mathrm{pre}}\newcommand{\om}{\omega}

%%% 偏微分方程式
\let\div\relax
\DeclareMathOperator{\div}{div}\newcommand{\del}{\partial}
\newcommand{\LHS}{\mathrm{LHS}}\newcommand{\RHS}{\mathrm{RHS}}\newcommand{\bnu}{\boldsymbol{\nu}}\newcommand{\interior}{\mathrm{in}\;}\newcommand{\SH}{\mathrm{SH}}\renewcommand{\v}{\boldsymbol{\nu}}\newcommand{\n}{\mathbf{n}}\newcommand{\ssub}{\Subset}\newcommand{\curl}{\mathrm{curl}}
%%% 常微分方程式
\newcommand{\Ei}{\mathrm{Ei}}\newcommand{\sn}{\mathrm{sn}}\newcommand{\wgamma}{\widetilde{\gamma}}
%%% 統計力学
\newcommand{\Ens}{\mathrm{Ens}}
%%% 解析力学
\newcommand{\cl}{\mathrm{cl}}\newcommand{\x}{\boldsymbol{x}}

%%% 統計的因果推論
\newcommand{\Do}{\mathrm{Do}}
%%% 応用統計学
\newcommand{\mrl}{\mathrm{mrl}}
%%% 数理統計
\newcommand{\comb}[2]{\begin{pmatrix}#1\\#2\end{pmatrix}}\newcommand{\bP}{\mathbb{P}}\newcommand{\compsub}{\overset{\textrm{cpt}}{\subset}}\newcommand{\lip}{\textrm{lip}}\newcommand{\BL}{\mathrm{BL}}\newcommand{\G}{\mathbb{G}}\newcommand{\NB}{\mathrm{NB}}\newcommand{\oR}{\o{\R}}\newcommand{\liminfn}{\liminf_{n\to\infty}}\newcommand{\limsupn}{\limsup_{n\to\infty}}\newcommand{\esssup}{\mathrm{ess.sup}}\newcommand{\asto}{\xrightarrow{\as}}\newcommand{\Cov}{\mathrm{Cov}}\newcommand{\cQ}{\mathcal{Q}}\newcommand{\VC}{\mathrm{VC}}\newcommand{\mb}{\mathrm{mb}}\newcommand{\Avar}{\mathrm{Avar}}\newcommand{\bB}{\mathbb{B}}\newcommand{\bW}{\mathbb{W}}\newcommand{\sd}{\mathrm{sd}}\newcommand{\w}[1]{\widehat{#1}}\newcommand{\bZ}{\boldsymbol{Z}}\newcommand{\Bernoulli}{\mathrm{Ber}}\newcommand{\Ber}{\mathrm{Ber}}\newcommand{\Mult}{\mathrm{Mult}}\newcommand{\BPois}{\mathrm{BPois}}\newcommand{\fraks}{\mathfrak{s}}\newcommand{\frakk}{\mathfrak{k}}\newcommand{\IF}{\mathrm{IF}}\newcommand{\bX}{\mathbf{X}}\newcommand{\bx}{\boldsymbol{x}}\newcommand{\indep}{\raisebox{0.05em}{\rotatebox[origin=c]{90}{$\models$}}}\newcommand{\IG}{\mathrm{IG}}\newcommand{\Levy}{\mathrm{Levy}}\newcommand{\MP}{\mathrm{MP}}\newcommand{\Hermite}{\mathrm{Hermite}}\newcommand{\Skellam}{\mathrm{Skellam}}\newcommand{\Dirichlet}{\mathrm{Dirichlet}}\newcommand{\Beta}{\mathrm{Beta}}\newcommand{\bE}{\mathbb{E}}\newcommand{\bG}{\mathbb{G}}\newcommand{\MISE}{\mathrm{MISE}}\newcommand{\logit}{\mathtt{logit}}\newcommand{\expit}{\mathtt{expit}}\newcommand{\cK}{\mathcal{K}}\newcommand{\dl}{\dot{l}}\newcommand{\dotp}{\dot{p}}\newcommand{\wl}{\wt{l}}\newcommand{\Gauss}{\mathrm{Gauss}}\newcommand{\fA}{\mathfrak{A}}\newcommand{\under}{\mathrm{under}\;}\newcommand{\whtheta}{\wh{\theta}}\newcommand{\Em}{\mathrm{Em}}\newcommand{\ztheta}{{\theta_0}}
\newcommand{\rO}{\mathrm{O}}\newcommand{\Bin}{\mathrm{Bin}}\newcommand{\rW}{\mathrm{W}}\newcommand{\rG}{\mathrm{G}}\newcommand{\rB}{\mathrm{B}}\newcommand{\rN}{\mathrm{N}}\newcommand{\rU}{\mathrm{U}}\newcommand{\HG}{\mathrm{HG}}\newcommand{\GAMMA}{\mathrm{Gamma}}\newcommand{\Cauchy}{\mathrm{Cauchy}}\newcommand{\rt}{\mathrm{t}}
\DeclareMathOperator{\erf}{erf}

%%% 圏
\newcommand{\varlim}{\varprojlim}\newcommand{\Hom}{\mathrm{Hom}}\newcommand{\Iso}{\mathrm{Iso}}\newcommand{\Mor}{\mathrm{Mor}}\newcommand{\Isom}{\mathrm{Isom}}\newcommand{\Aut}{\mathrm{Aut}}\newcommand{\End}{\mathrm{End}}\newcommand{\op}{\mathrm{op}}\newcommand{\ev}{\mathrm{ev}}\newcommand{\Ob}{\mathrm{Ob}}\newcommand{\Ar}{\mathrm{Ar}}\newcommand{\Arr}{\mathrm{Arr}}\newcommand{\Set}{\mathrm{Set}}\newcommand{\Grp}{\mathrm{Grp}}\newcommand{\Cat}{\mathrm{Cat}}\newcommand{\Mon}{\mathrm{Mon}}\newcommand{\Ring}{\mathrm{Ring}}\newcommand{\CRing}{\mathrm{CRing}}\newcommand{\Ab}{\mathrm{Ab}}\newcommand{\Pos}{\mathrm{Pos}}\newcommand{\Vect}{\mathrm{Vect}}\newcommand{\FinVect}{\mathrm{FinVect}}\newcommand{\FinSet}{\mathrm{FinSet}}\newcommand{\FinMeas}{\mathrm{FinMeas}}\newcommand{\OmegaAlg}{\Omega\text{-}\mathrm{Alg}}\newcommand{\OmegaEAlg}{(\Omega,E)\text{-}\mathrm{Alg}}\newcommand{\Fun}{\mathrm{Fun}}\newcommand{\Func}{\mathrm{Func}}\newcommand{\Alg}{\mathrm{Alg}} %代数の圏
\newcommand{\CAlg}{\mathrm{CAlg}} %可換代数の圏
\newcommand{\Met}{\mathrm{Met}} %Metric space & Contraction maps
\newcommand{\Rel}{\mathrm{Rel}} %Sets & relation
\newcommand{\Bool}{\mathrm{Bool}}\newcommand{\CABool}{\mathrm{CABool}}\newcommand{\CompBoolAlg}{\mathrm{CompBoolAlg}}\newcommand{\BoolAlg}{\mathrm{BoolAlg}}\newcommand{\BoolRng}{\mathrm{BoolRng}}\newcommand{\HeytAlg}{\mathrm{HeytAlg}}\newcommand{\CompHeytAlg}{\mathrm{CompHeytAlg}}\newcommand{\Lat}{\mathrm{Lat}}\newcommand{\CompLat}{\mathrm{CompLat}}\newcommand{\SemiLat}{\mathrm{SemiLat}}\newcommand{\Stone}{\mathrm{Stone}}\newcommand{\Mfd}{\mathrm{Mfd}}\newcommand{\LieAlg}{\mathrm{LieAlg}}
\newcommand{\Sob}{\mathrm{Sob}} %Sober space & continuous map
\newcommand{\Op}{\mathrm{Op}} %Category of open subsets
\newcommand{\Sh}{\mathrm{Sh}} %Category of sheave
\newcommand{\PSh}{\mathrm{PSh}} %Category of presheave, PSh(C)=[C^op,set]のこと
\newcommand{\Conv}{\mathrm{Conv}} %Convergence spaceの圏
\newcommand{\Unif}{\mathrm{Unif}} %一様空間と一様連続写像の圏
\newcommand{\Frm}{\mathrm{Frm}} %フレームとフレームの射
\newcommand{\Locale}{\mathrm{Locale}} %その反対圏
\newcommand{\Diff}{\mathrm{Diff}} %滑らかな多様体の圏
\newcommand{\Quiv}{\mathrm{Quiv}} %Quiverの圏
\newcommand{\B}{\mathcal{B}}\newcommand{\Span}{\mathrm{Span}}\newcommand{\Corr}{\mathrm{Corr}}\newcommand{\Decat}{\mathrm{Decat}}\newcommand{\Rep}{\mathrm{Rep}}\newcommand{\Grpd}{\mathrm{Grpd}}\newcommand{\sSet}{\mathrm{sSet}}\newcommand{\Mod}{\mathrm{Mod}}\newcommand{\SmoothMnf}{\mathrm{SmoothMnf}}\newcommand{\coker}{\mathrm{coker}}\newcommand{\Ord}{\mathrm{Ord}}\newcommand{\eq}{\mathrm{eq}}\newcommand{\coeq}{\mathrm{coeq}}\newcommand{\act}{\mathrm{act}}

%%%%%%%%%%%%%%% 定理環境(足助先生ありがとうございます) %%%%%%%%%%%%%%%

\everymath{\displaystyle}
\renewcommand{\proofname}{\bf\underline{[証明]}}
\renewcommand{\thefootnote}{\dag\arabic{footnote}} %足助さんからもらった.どうなるんだ?
\renewcommand{\qedsymbol}{$\blacksquare$}

\renewcommand{\labelenumi}{(\arabic{enumi})} %(1),(2),...がデフォルトであって欲しい
\renewcommand{\labelenumii}{(\alph{enumii})}
\renewcommand{\labelenumiii}{(\roman{enumiii})}

\newtheoremstyle{StatementsWithUnderline}% ?name?
{3pt}% ?Space above? 1
{3pt}% ?Space below? 1
{}% ?Body font?
{}% ?Indent amount? 2
{\bfseries}% ?Theorem head font?
{\textbf{.}}% ?Punctuation after theorem head?
{.5em}% ?Space after theorem head? 3
{\textbf{\underline{\textup{#1~\thetheorem{}}}}\;\thmnote{(#3)}}% ?Theorem head spec (can be left empty, meaning ‘normal’)?

\usepackage{etoolbox}
\AtEndEnvironment{example}{\hfill\ensuremath{\Box}}
\AtEndEnvironment{observation}{\hfill\ensuremath{\Box}}

\theoremstyle{StatementsWithUnderline}
    \newtheorem{theorem}{定理}[section]
    \newtheorem{axiom}[theorem]{公理}
    \newtheorem{corollary}[theorem]{系}
    \newtheorem{proposition}[theorem]{命題}
    \newtheorem{lemma}[theorem]{補題}
    \newtheorem{definition}[theorem]{定義}
    \newtheorem{problem}[theorem]{問題}
    \newtheorem{exercise}[theorem]{Exercise}
\theoremstyle{definition}
    \newtheorem{issue}{論点}
    \newtheorem*{proposition*}{命題}
    \newtheorem*{lemma*}{補題}
    \newtheorem*{consideration*}{考察}
    \newtheorem*{theorem*}{定理}
    \newtheorem*{remarks*}{要諦}
    \newtheorem{example}[theorem]{例}
    \newtheorem{notation}[theorem]{記法}
    \newtheorem*{notation*}{記法}
    \newtheorem{assumption}[theorem]{仮定}
    \newtheorem{question}[theorem]{問}
    \newtheorem{counterexample}[theorem]{反例}
    \newtheorem{reidai}[theorem]{例題}
    \newtheorem{ruidai}[theorem]{類題}
    \newtheorem{algorithm}[theorem]{算譜}
    \newtheorem*{feels*}{所感}
    \newtheorem*{solution*}{\bf{[解]}}
    \newtheorem{discussion}[theorem]{議論}
    \newtheorem{synopsis}[theorem]{要約}
    \newtheorem{cited}[theorem]{引用}
    \newtheorem{remark}[theorem]{注}
    \newtheorem{remarks}[theorem]{要諦}
    \newtheorem{memo}[theorem]{メモ}
    \newtheorem{image}[theorem]{描像}
    \newtheorem{observation}[theorem]{観察}
    \newtheorem{universality}[theorem]{普遍性} %非自明な例外がない.
    \newtheorem{universal tendency}[theorem]{普遍傾向} %例外が有意に少ない.
    \newtheorem{hypothesis}[theorem]{仮説} %実験で説明されていない理論.
    \newtheorem{theory}[theorem]{理論} %実験事実とその(さしあたり)整合的な説明.
    \newtheorem{fact}[theorem]{実験事実}
    \newtheorem{model}[theorem]{模型}
    \newtheorem{explanation}[theorem]{説明} %理論による実験事実の説明
    \newtheorem{anomaly}[theorem]{理論の限界}
    \newtheorem{application}[theorem]{応用例}
    \newtheorem{method}[theorem]{手法} %実験手法など,技術的問題.
    \newtheorem{test}[theorem]{検定}
    \newtheorem{terms}[theorem]{用語}
    \newtheorem{solution}[theorem]{解法}
    \newtheorem{history}[theorem]{歴史}
    \newtheorem{usage}[theorem]{用語法}
    \newtheorem{research}[theorem]{研究}
    \newtheorem{shishin}[theorem]{指針}
    \newtheorem{yodan}[theorem]{余談}
    \newtheorem{construction}[theorem]{構成}
    \newtheorem{motivation}[theorem]{動機}
    \newtheorem{context}[theorem]{背景}
    \newtheorem{advantage}[theorem]{利点}
    \newtheorem*{definition*}{定義}
    \newtheorem*{remark*}{注意}
    \newtheorem*{question*}{問}
    \newtheorem*{problem*}{問題}
    \newtheorem*{axiom*}{公理}
    \newtheorem*{example*}{例}
    \newtheorem*{corollary*}{系}
    \newtheorem*{shishin*}{指針}
    \newtheorem*{yodan*}{余談}
    \newtheorem*{kadai*}{課題}

\raggedbottom
\allowdisplaybreaks
\usepackage[math]{anttor}
\begin{document}
\tableofcontents

\chapter{微分論}

\section{l'Hospitalの定理}

\begin{tcolorbox}[colframe=ForestGreen, colback=ForestGreen!10!white,breakable,colbacktitle=ForestGreen!40!white,coltitle=black,fonttitle=\bfseries\sffamily,
title=]
    平均値の定理の消息であり,多変数の場合(または複素関数の場合)には崩れる.
\end{tcolorbox}

\subsection{一般形}

\begin{theorem}[l'Hospitalの定理\footnote{\cite{Rudin-Principles} Th'm 5.13}]
    $f,g:(a,b)\to\R$を可微分関数,$g$の微分は$(a,b)$上で消えないとする:$\forall_{x\in(a,b)}\;g'(x)\ne0$.
    次の条件(1),(2)のいずれかが成り立てば,
    \[\lim_{x\to a}\frac{f'(x)}{g'(x)}=A\quad\Rightarrow\quad\lim_{x\to a}\frac{f(x)}{g(x)}=A,\qquad A\in[-\infty,\infty].\]
    \begin{enumerate}
        \item $\lim_{x\to a}f(x)=\lim_{x\to a}g(x)=0$.
        \item $\lim_{x\to a}g(x)\in\{\pm\infty\}$.
    \end{enumerate}
\end{theorem}
\begin{Proof}\mbox{}
    \begin{description}
        \item[方針] まず$A\in[-\infty,\infty)$と仮定し,任意の$A<q$に対して$\frac{f(x)}{g(x)}< q$を示す.
        すると全く同様にして,$A\in(-\infty,\infty]$の仮定の下で,任意の$p<A$に対して$p<\frac{f(x)}{g(x)}$が示せ,結論を得る.
        欲しい等式は真の不等号であるから,新たに$A<r<q$を取ることに注意.
        \item[準備]
        仮定$\lim_{x\to a}\frac{f'(x)}{g'(x)}=A$より,
        任意の$A<q$と$r\in(A,q)$について,ある$c\in(a,b)$が存在して,
        \[x\in(a,c)\quad\Rightarrow\quad\frac{f'(x)}{g'(x)}<r.\]
        このとき,Cauchyの平均値の定理より,任意の$(x,y)\subset(a,c)$に対して,ある$t\in(x,y)$が存在して,
        \begin{equation}\label{eq-for-lHospital}
            \frac{f(x)-f(y)}{g(x)-g(y)}=\frac{f'(t)}{g'(t)}<r.
        \end{equation}
    \end{description}
    \begin{enumerate}
        \item $\lim_{x\to a}f(x)=\lim_{x\to a}g(x)=0$のとき,式\ref{eq-for-lHospital}の極限$x\to a$を考えて,
        \[\frac{f(y)}{g(y)}\le r<q,\qquad(y\in(a,c)).\]
        \item $\lim_{x\to a}g(x)=\infty$のとき,ある$c_1\in(a,y)$が存在して,任意の$x\in(a,c_1)$について$g(x)>g(y)$かつ$g(x)>0$が成り立つように出来る.これより式\ref{eq-for-lHospital}の両辺を$\frac{g(x)-g(y)}{g(x)}$倍することで,
        \[\frac{f(x)}{g(x)}<r-r\frac{g(y)}{g(x)}+\frac{f(y)}{g(x)},\qquad(a<x<c_1).\]
        を得る.この変形に対して極限$x\to a$を考えると,ある$c_2\in(a,c_1)$が存在して,
        \[\frac{f(x)}{g(x)}\le r<q,\qquad(x\in(a,c_2)).\]
    \end{enumerate}
\end{Proof}
\begin{remarks}
    すなわち,分数関数$f/g$の$a\in\R$での$0/0$または$\infty/\infty$の不定型極限について,
    ある片側近傍$(a,b)$上で$f,g$が可微分かつ$g'$が消えないならば,$f'/g'$と同じ$x\to a$極限を持つ.
    $x\to a$に対して,左側近傍$(b,a)$を考えてももちろん良い.
\end{remarks}

\subsection{複素関数について}

\begin{proposition}[複素関数にも成り立つ消息]
    $f,g:(a,b)\to\C$を可微分関数,$g'(x)\ne0,f(x)=g(x)=0$とする.このとき,
    \[\lim_{t\to x}\frac{f(t)}{g(t)}=\frac{f'(x)}{g'(x)}.\]
\end{proposition}

\begin{proposition}
    $f,g\in H((0,1);\C)$を正則関数とし,$f(x),g(x)\to0,f'(x)\to A,g'(x)\to B\;(x\to0)$とする.$B\ne0$のとき,
    \[\lim_{x\to0}\frac{f(x)}{g(x)}=\frac{A}{B}.\]
\end{proposition}

\subsection{多変数の平均値の定理}

\begin{theorem}
    $f\in C([a,b];\R^k)$は微分可能であるとする.このとき,
    \[\exists_{x\in(a,b)}\;\abs{f(b)-f(a)}\le(b-a)\sup_{x\in(a,b)}\abs{f'(x)}.\]
\end{theorem}

\section{最適化問題}

\subsection{極値判定法}

\begin{theorem}
    $A\osub\R^n$を開集合,$f:A\to\R$を$C^2$-級,$a\in A$で$f'$は消えるとし,$Q(X_1,\cdots,X_n):=\sum^n_{i,j=1}f_{x_ix_j}(a)X_iX_j\in\R[X_1,\cdots,X_n]$を2次形式とする.
    \begin{enumerate}
        \item $Q$が正定値ならば,$f$は$a$で極小.
        \item $Q$が負定値ならば,$f$は$a$で極大.
        \item $Q$が不定符号ならば,$f$は$a$で極小でも極大でもない.
    \end{enumerate}
\end{theorem}

\subsection{等式制約付き最適化}

\begin{theorem}
    $f,g_1,\cdots,g_m\in C^1(A)\;(A\osub\R^n)$,$S:=\cap_{i\in[m]}g_i^{-1}(0)$を実行可能領域とする.
    $f$が$a$において極値を取り(局所最適解),$\rank\paren{Dg}{Dx}=m$ならば,
    \[\exists_{\lambda_1,\cdots,\lambda_m\in\R}\forall_{j\in[n]}\quad\pp{f}{x_j}(a)=\sum^m_{i=1}\lambda_i\pp{g_i}{x_j}(a)\]
\end{theorem}
\begin{remarks}
    Jacobianがランク落ちしていないという仮定は,勾配$\nabla g_i$が1次独立であることをいう(1次独立制約想定).
    $F(x,\lambda):=f-\sum^m_{i=1}\lambda_ig_i$で定まる$\R^n\times\R^m\to\R$を\textbf{Lagrange関数}といい,極値点の候補を探す問題は,Lagrange関数の微分係数に関する連立方程式(停留条件)に還元される.
\end{remarks}

\begin{remark}
    不等式制約付き最適化は,さらに多様な制約想定の下で理論展開されている.SlaterとMangasarian-Fromovitzである.
    このとき,Lagrange乗数(の一般化)が満たすべき連立方程式(停留条件と相補性条件と呼ばれる条件)を\textbf{Karush-Kuhn-Tucker条件}という.ここには明らかに,
\end{remark}

\section{Taylorの定理}

\subsection{理論標準形}

\begin{tcolorbox}[colframe=ForestGreen, colback=ForestGreen!10!white,breakable,colbacktitle=ForestGreen!40!white,coltitle=black,fonttitle=\bfseries\sffamily,
title=]
    可微分実関数については,多項式による最良近似が標準的に存在して,近似誤差が$n$階の微分係数で予測出来る.
\end{tcolorbox}

\begin{lemma}[可微分関数の多項式近似]
    $f:[a,b]\to\R$は$n$階微分可能とし,$\al\in[a,b]$を点とする.
    ある$n-1$次の多項式$P_\al$について,
    \[f(x)-P_\al(x)=O((x-\al)^n)\quad(x\to\al)\]
    が成り立つならば,
    \[P_\al(t):=\sum^{n-1}_{k=0}\frac{f^{(k)}(\al)}{k!}(t-\al^k)\]
    と表せる.
\end{lemma}
\begin{Proof}
    \[Q_\al(t):=a_0+a_1(x-\al)+\cdots+a_n(x-\al)^n\]
    も条件を満たすとする.$P_\al(\al)=Q_\al(\al)$より,$a_0=f(a)$.
    続いて,$P'_\al(\al)=Q'_\al(\al)$と比較して行けば良い.
\end{Proof}

\begin{theorem}[Taylorの定理 \cite{Rudin-Principles} Th'm 5.15]
    $f:[a,b]\to\R$は$n$階微分可能とする.
    任意の$\{\al<\beta\}\subset[a,b]$について,$\al\in[a,b]$での$n-1$次多項式近似$P_\al$の近似誤差は,ある$x\in(\al,\beta)$が存在して,
    \[f(\beta)-\underbrace{\sum_{k=0}^{n-1}\frac{f^{(k)}(\al)}{k!}(\beta-\al)^k}_{=P_\al(\beta)}=\frac{f^{(n)}(x)}{n!}(\beta-\al)^n.\]
    と表せる.右辺を\textbf{$n$次の剰余項}という.
\end{theorem}
\begin{Proof}\mbox{}
    \begin{enumerate}[{Step}1]
        \item 任意の$\{\al<\beta\}\subset[a,b]$を取り,問題の係数を
        \[M:=\frac{f(\beta)-P_\al(\beta)}{(\beta-\al)^n}\]
        と定め,値の変化
        \[g(t):=f(t)-P_\al(t)-M(t-\al)^n\qquad(t\in[a,b])\]
        を考える.両辺を$n$階微分すると,
        \[g^{(n)}(t)=f^{(n)}(t)-n!M\qquad(t\in(a,b))\]
        であるが,このとき$\exists_{x\in(\al,\beta)}g^{(n)}(\al)=0$より結論を得る.
        \item いま$\forall_{k\in n}\;P^{(k)}(\al)=f^{(k)}(\al)$より,$g(\al)=g'(\al)=\cdots=g^{(n-1)}(\al)=0$が成り立っている.
        よって,$g(\beta)=0$であることは,平均値の定理より$\exists_{x_1\in(\al,\beta)}\;g'(x_1)=0$を含意する.
        これを繰り返すと,$\exists_{x_n\in(\al,x_{n-1})}\;g^{(n)}(x_n)=0$.
    \end{enumerate}
\end{Proof}
\begin{remarks}
    このとき$f$は$C^{n-1}([a,b])$-級ではあるから,
    \[f(x)=P_\al(x)+O((x-\al)^n),\quad(x\to\al)\]
    が従う.
\end{remarks}

\begin{proposition}[Lagrange's form]
    さらに$f:[a,b]\to\R$が$C^n$-級のとき,$n$次の剰余項は
    \[f(\beta)-P_\al(\beta)=\int^x_\al\frac{f^{(n)}(t)}{(n-1)!}(x-t)^{n-1}dt\]
    と表示出来る.
\end{proposition}

\subsection{Cauchyの平均値の定理}

\begin{tcolorbox}[colframe=ForestGreen, colback=ForestGreen!10!white,breakable,colbacktitle=ForestGreen!40!white,coltitle=black,fonttitle=\bfseries\sffamily,
title=]
    随伴が存在する.いまなら超関数法の萌芽に見える.
\end{tcolorbox}

\begin{theorem}[generalized mean value theorem]
    $f,g\in C([a,b];\R)$は微分可能とする.このとき,
    \[\exists_{x\in(a,b)}\quad (f(b)-f(a))g'(x)=(g(b)-g(a))f'(x).\]
\end{theorem}

\section{導関数の連続性}

\begin{definition}[simple discontinuity / first kind]
    $f\in\Map((a,b);\R)$は$x\in(a,b)$で不連続とする.
    $f(x+),f(x-)$がいずれも存在するとき,\textbf{第一種不連続}という.
    そうでない場合を第二種という.
    第一種不連続性は,$f(x+)\ne f(x-)$と$f(x+)=f(x-)\ne f(x)$との2通りに分類出来る.
\end{definition}

\begin{theorem}
    $f\in\Map([a,b];\R)$は微分可能で,$f'(a)<\lambda<f'(b)$を満たすとする.
    このとき,ある$x\in(a,b)$が存在して$f'(x)=\lambda$を満たす.
\end{theorem}

\begin{corollary}
    $f\in\Map([a,b];\R)$は微分可能ならば,導関数$f'$は第一種の不連続点を持ち得ない.
\end{corollary}

\chapter{Riemann-Stieltjes積分}

\begin{quotation}
    Lebesgue積分とは違って,$\R$の順序構造に強く依存した,Euclid空間上にオーダーメイドの積分が定義できる.
    これについての古典論を復習する.
    \begin{enumerate}
        \item 多変数の積分の変数変換が苦手.Jacobi行列周り.
    \end{enumerate}
\end{quotation}

\section{定義と存在}

\begin{definition}[(Riemann-)Stieltjes integral]
    $I:=[a,b]$を閉区間とし,$f:[a,b]\to\R$を有界関数,$\al:[a,b]\to\R$を単調増加関数とする.
    \begin{enumerate}
        \item \textbf{分割}$P$とは,$[a,b]$の有限集合$P=\Brace{a=x_0\le x_1\le\cdots\le x_n=b}$をいう.
        \item 各分割$P\in P([a,b])$に対して,$\Delta\al_i:=\al(x_i)-\al(x_{i-1})$と表し,
        \[M_i(P):=\sup_{x\in[x_{i-1},x_i]}f(x),\qquad\qquad m_i(P):=\inf_{x\in[x_{i-1},x_i]}f(x)\;(i\in[n])\]
        とし,
        \[U(P,f,\al):=\sum^n_{i=1}M_i(P)\Delta\al_i,\qquad\qquad L(P,f,\al):=\sum^n_{i=1}m_i(P)\Delta\al_i\]
        とする.
        \item 分割の全体$\P:=\Brace{P\in P([a,b])\mid\abs{P}<\infty}$は有向集合をなす.包含関係$\subset$について分割は順序をなし,任意の2つの分割$P,Q$について$P\cup Q$は上界である.
        このとき,2つのネット$(U(P,f,\al))_{P\in\P},(L(P,f,\al))_{P\in\P}$は収束する.
        すなわち,
        \[\o{\int^b_a}fd\al:=\inf_{P\in P([a,b]),\abs{P}<\infty}U(P,f,\al),\qquad\qquad\underline{\int^b_a}fd\al=\sup_{P\in P([a,b]),\abs{P}<\infty}L(P,f,\al).\]
        として得る実数を,\textbf{上/下Stieltjes積分}と呼ぶ.
        \item 上積分と下積分が一致するとき,\textbf{Stieltjes可積分}であるといい,$f\in\cR([a,b],\al)$と表す.
    \end{enumerate}
\end{definition}

\section{可積分性}

\begin{tcolorbox}[colframe=ForestGreen, colback=ForestGreen!10!white,breakable,colbacktitle=ForestGreen!40!white,coltitle=black,fonttitle=\bfseries\sffamily,
title=]
    復習する.
\end{tcolorbox}

\begin{theorem}[可積分性の特徴付け]
    関数$f:[a,b]\to\R$について,次の2条件は同値.
    \begin{enumerate}
        \item $f\in\cR(\al)$.
        \item $\forall_{\ep>0}\;\exists_{P\in P([a,b])}\;\abs{P}<\infty\land U(P,f,\al)-L(P,f,\al)<\ep$.
    \end{enumerate}
\end{theorem}

\begin{theorem}[可積分条件]\mbox{}
    関数$f:[a,b]\to\R$は,
    \begin{enumerate}
        \item 連続ならば$f\in\cR(\al)$.
        \item 単調ならば,$\al$が連続ならば$f\in\cR(\al)$.
        \item 有界であり,$[a,b]$上に高々有限の不連続点をもち,その任意の点で$\al$は連続であるならば,$f\in\cR(\al)$.
    \end{enumerate}
\end{theorem}

\chapter{多変数関数論}

\section{行列のBanach代数}

\begin{tcolorbox}[colframe=ForestGreen, colback=ForestGreen!10!white,breakable,colbacktitle=ForestGreen!40!white,coltitle=black,fonttitle=\bfseries\sffamily,
title=]
    Euclid空間の間の写像について,接空間の間に引き起こされる微分の空間$B(\R^n,\R^m)$がJacobi行列の空間である.
    Jacobi行列への対応$x\mapsto Jf(x)$が連続であるとき,$f$を$C^1$-級という.
\end{tcolorbox}

\begin{theorem}[行列の代数]
    任意の行列$A,B\in B(\R^n,\R^m)$について,
    \begin{enumerate}
        \item $A$は一様連続である.
        \item $B(\R^n,\R^m)$は有限次元Banach空間である.
        \item 劣乗法性が成り立つ:$\norm{BA}\le\norm{B}\norm{A}$.
    \end{enumerate}
\end{theorem}

\begin{theorem}[行列の空間]\mbox{}
    \begin{enumerate}
        \item $A\in\Iso(\R^n),B\in B(\R^n)$について,$\norm{B-A}\norm{A^{-1}}<1$ならば,$B\in\Iso(\R^n)$.
        \item $\Iso(\Om)\subset B(\R^n)$は開集合で,$A\mapsto A^{-1}$は$\Iso(\Om)$上の位相同型である.
    \end{enumerate}
\end{theorem}

\begin{theorem}
    $S$を距離空間,$a_{11},\cdots,a_{mn}:S\to\R$を連続とする.
    このとき,$S\to B(\R^n,\R^m);p\mapsto A_p:=(a_{ij}(p))$は連続である.
\end{theorem}

\begin{theorem}[Jacobi行列からJacobianへの対応]
    $\det:B(\R^n)\to\R$は行列積と実数積について群準同型を与える:$\det(BA)=\det(B)\det(A)$.
\end{theorem}
\begin{definition}
    可微分関数$f:E\to\R^m$に対して,微分と行列式の合成
    $J_f:=\det\circ D:E\to B(\R^m)\to\R$をJacobianという.
\end{definition}

\section{微分}

\begin{definition}
    $f:\R^n\osup E\to\R^m$が
    \begin{enumerate}
        \item $x\in E$で\textbf{微分可能}であるとは,
        \[\exists_{A\in B(\R^n,\R^m)}\;\lim_{h\to0\in\R^n}\frac{\abs{f(x+h)-f(x)-Ah}}{\abs{h}}=0\]
        が成り立つことをいう.このとき,$f'(x)=df_x=A$として$f':E\to B(\R^n;\R^m)$を定める.
        \item \textbf{$C^1$-級}であるとは,$f'=df:E\to B(\R^n;\R^m)$が連続であることをいう:$\forall_{x\in E}\;\forall_{\ep>0}\;\exists_{\delta>0}\;\forall_{y\in E}\;\abs{x-y}<\delta\Rightarrow\norm{f'(y)-f'(x)}<\ep$.
        \item 写像$f:E\to\R^m$の微分$Df:R^n=T(\R^n)\to T(\R^m)=\R^m$は,係数の対応$D_if:E\to\R$によって完全に定まる.これを\textbf{偏微分}という.
        \item $\nabla f(x):=\sum_{i=1}^nD_if(x)e_i$によって定まる対応$\nabla f:E\to T(\R^m)=\R^m$を\textbf{勾配}または\textbf{発散}という.
        \item 任意の$u\in\R^n$に対して,$(D_uf)(x):=(\nabla f)(x)\cdot u$によって定まる対応$D_uf:E\to\R$を\textbf{方向微分}という.
    \end{enumerate}
\end{definition}

\begin{proposition}\mbox{}
    \begin{enumerate}
        \item $A_1,A_2\in M_{mn}(\R)$がいずれも$f$の$x\in E$における微分係数ならば,$A_1=A_2$.
        \item $f$が$C^1$-級であることは,任意の偏微分$D_if_i\;(i\in[m])$が$E$上連続であることに同値.
    \end{enumerate}
\end{proposition}

\begin{proposition}
    $f:E\to\R$を関数,$\gamma:I\to E$を可微分曲線とする.$g:=f\circ\gamma$とすると,
    \[g'(t)=(\nabla f)(\gamma(t))\cdot\gamma'(t).\]
\end{proposition}

\begin{theorem}[平均値の定理の一般化]
    $E$は凸開集合,$f:E\to\R^m$は可微分で$E$上有界な導関数を持つとする.このとき,$\forall_{a,b\in E}\;\abs{f(b)-f(a)}\le\sup_{x\in E}\norm{f'(x)}\abs{b-a}$.
\end{theorem}
\begin{corollary}
    $f'=0\;\on E$ならば$f$は定数.
\end{corollary}

\section{逆関数定理}

\begin{theorem}
    $f:E\to\R^n$を$C^1$-級で,$f(a)=b$において$f'(a)$は可逆であるとする.
    \begin{enumerate}
        \item 開近傍$U\in\O(a),V\in\O(b)$が存在して,$f|_U$は全単射$U\simeq_\Set V$を定める.
        \item 逆写像$g:V\to U$も$C^1$-級である.
    \end{enumerate}
\end{theorem}
\begin{remarks}[陽関数定理としての見方]
    成分毎に表せば,$n$元連立$n$次方程式$y_i=f_i(x_1,\cdots,x_n)\;(i\in[n])$は$f$が$C^1$-級で$f'(x)$が可逆ならば必ず局所解$x_i=g_i(y_1,\cdots,y_n)$を持ち,$g$も$C^1$-級になる.
\end{remarks}

\begin{corollary}
    $f:E\to\R^n$を$C^1$-級で,$f'(E)\subset\Iso(\R^n)$とする.このとき,$f$は開写像である.
\end{corollary}

\section{陰関数定理}

\begin{remark}
    $A\in B(\R^{n+m},\R^m)$について,$A(h,k)=A(h,0)+A(0,k)$であるから,$A_x(h):=A(h,0),A_y(k):=A(0,k)$と定めると$A_x\in B(\R^n),A_y\in B(\R^m,\R^n)$で,$A(h,k)=A_xh+A_yk$が成り立つ.
    $(h,k)=(h,0)+(0,k)$は縦ベクトルに関する分解,$A=[A_x;A_y]$は横長行列の分解である.
\end{remark}

\begin{lemma}[$n$元連立方程式の解]
    線型写像$A\in B(\R^{n+m},\R^n)$は$A_x\in\Iso(\R^n)$が可逆であるとする.このとき,
    \begin{enumerate}
        \item $\forall_{k\in\R^m}\;\exists_{h\in\R^n}\;A(h,k)=0$.
        \item $h=-(A_x)^{-1}A_yk$を満たす.
    \end{enumerate}
\end{lemma}
\begin{remarks}
    陰関数表示された$n+m$元連立$n$次方程式系$A(h,k)=0$は,$\rank A\ge n$ならば,与えられた$k$に対して$h$について解け,解の対応は線形写像として表される.
\end{remarks}

\begin{theorem}[陰関数定理]
    $f:\R^{n+m}\osup E\to\R^n$を$C^1$-級,$f(a,b)=0$とする.$A:=f'(a,b)\in B(\R^{n+m},\R^n)$の首座行列$A_x\in\Iso(\R^n)$は可逆とする.
    \begin{enumerate}
        \item 開近傍$U\in\O_{\R^{n+m}}(a,b),W\in\O_{\R^m}(b)$が存在して,$\forall_{y\in W}\;\exists_{x\in U_y}\;f(x,y)=0$.
        \item $x=:g(y)$で定まる対応$\R^m\osup W\to\R^n$は$C^1$-級で$g(b)=a$を満たし,$\forall_{y\in W}\;f(g(y),y)=0$かつ$g'(b)=-(A_x)^{-1}A_y$を満たすものとする.
    \end{enumerate}
\end{theorem}
\begin{remarks}
    式$g'(b)=-(A_x)^{-1}A_y$は$f(g(b),b)=0$のChain Ruleに関する必要条件
    \[\sum^n_{i=j}\paren{\pp{f_i}{x_j}}\paren{\pp{g_j}{y_k}}=-\paren{\pp{f_i}{y_k}}\]
    である.
\end{remarks}

\begin{theorem}[変数変換の存在 (rank theorem)]
    $m,n\ge r$について,$C^1$-級写像$F:E\to\R^m$の微分$F'(x)$は$E$上常に階数$r$を持つとする.
    このとき,任意の$a\in E,A:=F'(a)\in M_{mn}(\R)$に対して,ある開近傍$U,V\in\O_E(a)$と$C^1$-級全単射$H:V\iso U$と$C^1$-級写像$\varphi:A(V)\to(\Im A)^\perp$が存在して,$\forall_{x\in V}\;F(H(x))=Ax+\varphi(Ax)$.
\end{theorem}

\section{高階微分}

\begin{theorem}[平均値の定理]
    $f:\R^2\osup E\to\R$は$D_1f,D_{21}f$を$E$上で持つとし,$Q:=[a,a+k]\times[b,b+h]\subset E$を閉矩形とする.
    このとき,ある$(x,y)\in Q^\circ$が存在して,
    \[\Delta(f,Q):=f(a+h,b+k)-f(a+h,b)-f(a,b+k)+f(a,b)=hk(D_{21}f)(x,y).\]
\end{theorem}

\begin{theorem}
    $f:\R^2\osup E\to\R$は$D_1f,D_{21}f,D_2f$を$E$上で持ち,$D_{21}f$はある$(a,b)\in E$上で連続とする.このとき,$D_{12}f$も$(a,b)$上で存在し,値が$D_{21}f(a,b)$に一致する.
    特に,$C^2$-級ならば,2つの偏微分作用素は可換.
\end{theorem}

\section{定積分の微分}

\begin{notation}\mbox{}
    \begin{enumerate}
        \item $\varphi^t(x):=\varphi(x,t)$を$[a,b]\times[c,d]\to\R$とする.
        \item $\al$は$[a,b]$上の増加関数とする.
    \end{enumerate}
\end{notation}

\begin{theorem}
    次が成り立つとき,$(D_2\varphi)(-,s)\in\R(\al)$で,
    \[\dd{}{t}\int^b_a\varphi(x,t)d\al(x)=\int^b_a(D_2\varphi)(x,s)d\al(x).\]
    \begin{enumerate}
        \item 第1引数可積分性:$t\mapsto\varphi^t$は$[c,d]\to\cR(\al)$を定める.
        \item 第2引数一様連続性:$\forall_{s\in(c,d)}\;\forall_{\ep>0}\;\exists_{\delta>0}\;\forall_{x\in[a,b]}\;\forall_{t\in(s-\delta,s+\delta)}\;\abs{(D_2\varphi)(x,t)-(D_2\varphi)(x,s)}<\ep$.
    \end{enumerate}
\end{theorem}

\section{写像の分解}

\begin{definition}[primitive, flip]\mbox{}
    \begin{enumerate}
        \item $G:E\to\R^n$が高々1つの成分しか変えない場合,\textbf{原始的}であるという.
        \item $B\in B(\R^n)$がある2つの成分を入れ替え,その他を変えない場合,\textbf{入れ替え}という.
    \end{enumerate}
\end{definition}

\begin{theorem}[原始的写像への分解]
    $C^1$-級写像$F:\O_{\R^n}(0)\ni E\to\R^n$は$F(0)=0$かつ$F'(0)$は可逆とする.
    このとき,ある近傍$U\in\O_{\R^n}(0)$と$U$上原始的な写像$G_i$で可逆なものと入れ替えまたは恒等作用素$B_i$とが存在して,
    \[F=B_1\cdots B_{n-1}G_n\circ\cdots G_1\;\on U.\]
\end{theorem}

\begin{theorem}
    $K\compsub\R^n$,$(V_\al)$を$K$の被覆とする.$\{\psi_i\}_{i\in[s]}\subset C(\R^n;[0,1])$が存在して,
    \begin{enumerate}
        \item 従属性:$\exists_{\al\in A}\;\supp\psi_i\subset V_\al$.
        \item 1の分解:$\psi_1+\cdots+\psi_s=1\;\on K$.
    \end{enumerate}
\end{theorem}

\section{変数変換}

\begin{theorem}
    $T:E\to\R^k$を単射な$C^1$-級写像で,$J_T$は消えないとする:$T'(E)\subset\Iso(\R^k)$.
    このとき,
    \[\forall_{f\in C_c(\R^k)}\;\int_{\R^k}f(y)dy=\int_{\R^k}f(T(x))\abs{J_T(x)}dx.\]
\end{theorem}

なお,Lebesgue積分の立場からは次のようになる.

\begin{theorem}
    $T:V\to\R^k$が次の条件を満たすとき,
    \[\forall_{f\in L(\R^k;\o{\R_+})}\;\int_{T(X)}fdm=\int_X(f\circ T)\abs{J_T}dm.\]
    \begin{enumerate}
        \item $T:V\to\R^k$は$X\subset V\osub\R^k$上の連続写像.
        \item $X$はLebesgue可測で$T|_X$は単射かつ微分可能.
        \item $m(T(V\setminus X))=0$.
    \end{enumerate}
\end{theorem}

\begin{corollary}
    $\varphi:[a,b]\to[\al,\beta]$は絶対連続で全射な単調写像,$f\in L(\R)_+$をLebesgue可測とする.このとき,
    \[\int^\beta_\al f(t)dt=\int^b_af(\varphi(x))\varphi'(x)dx.\]
\end{corollary}

\section{微分の積分}

\begin{definition}
    $\om\in\Om^k(E)$は$dx_{i_1}\wedge\cdots\wedge dx_{i_k}$を基底とする線型空間の元とし,次の方法によって$k$-曲面$\Phi:I^k\to E$に実数$\om(\Phi)$を対応させるとする.
    \[\int_\Phi\om=\int_{I^k}\sum a_{i_1\cdots i_k}(\Phi(u))\pp{(x_{i_1},\cdots,x_{i_k})}{(u_1,\cdots,u_k)}du.\]
    Jacobianに絶対値がつかないことに注意.
\end{definition}

\begin{definition}[simplex, chain]
    
\end{definition}

\chapter{関数論}

\section{級数論}

\subsection{基本的な2つの収束判定法}

\begin{tcolorbox}[colframe=ForestGreen, colback=ForestGreen!10!white,breakable,colbacktitle=ForestGreen!40!white,coltitle=black,fonttitle=\bfseries\sffamily,
title=]
    基本は優級数を見つけるだけで,ただ標準的な見つけ方が2通り存在するのみである.
    そして関数級数の一様収束とは,一様ノルムに関して収束級数を定めることに他ならないが,これに関する優級数定理はWeierstrass $M$-判定法と呼ばれている.
\end{tcolorbox}

\begin{theorem}
    正数列$\{c_n\}\subset\R_{>0}$について,
    \[\liminf_{n\to\infty}\frac{c_{n+1}}{c_n}\le\liminf_{n\to\infty}\sqrt[n]{c_n}\le\limsup_{n\to\infty}\sqrt[n]{c_n}\le\limsup_{n\to\infty}\frac{c_{n+1}}{c_n}.\]
\end{theorem}

\begin{theorem}[root test (Cauchy)]\label{thm-root-test}
    級数$\sum a_n$に対して,$\al:=\limsup_{n\to\infty}\sqrt[n]{\abs{a_n}}$とする.
    \begin{enumerate}
        \item $\al<1$ならば$\sum a_n$は収束する.
        \item $\al>1$ならば$\sum a_n$は発散する.
    \end{enumerate}
\end{theorem}
\begin{Proof}\mbox{}
    \begin{enumerate}
        \item $\al<1$ならば,幾何級数$\sum\al^n$が$\sum a_n$の収束優級数となる.実際,$\forall_{n\in\N}\;\abs{a_n}\le k^n$となるため.
        \item 同様.
    \end{enumerate}
\end{Proof}

\begin{theorem}[ratio test (d'Alembert)]
    級数$\sum a_n$に対して,
    \begin{enumerate}
        \item $\limsup_{n\to\infty}\Abs{\frac{a_{n+1}}{a_n}}<1$ならば,収束する.
        \item $\Abs{\frac{a_{n+1}}{a_n}}\fe1\;\fe$ならば,発散する.
    \end{enumerate}
\end{theorem}

\subsection{Cauchyの二重級数定理}

\begin{tcolorbox}[colframe=ForestGreen, colback=ForestGreen!10!white,breakable,colbacktitle=ForestGreen!40!white,coltitle=black,fonttitle=\bfseries\sffamily,
title=]
    特に,両側無限列$(a_n)_{n\in\Z}$の極限は$\lim_{n,m\to\infty}\sum_{k=-m}^na_k$と定義するが,2つの級数$\sum_{n=0}^\infty a_n,\sum_{n=1}^\infty a_{-n}$がいずれも収束することに同値.
\end{tcolorbox}

\begin{definition}[convergence of double series]
    2重級数$\sum_{n,m=1}^\infty a_{nm}\subset\C$が収束するとは,
    2方向の部分和$S_{nm}=\sum_{i=1}^n\sum_{j=1}^ma_{ij}$について,
    \[\exists_{S\in\C}\;\forall_{\ep>0}\;\lim_{n,m\to\infty}\abs{S_{nm}-S}\le\ep.\]
    が成り立つ.
\end{definition}

\begin{theorem}[二重級数が定める$\sigma$-有限測度に関するFubiniの定理]
    $(a_{m,n})$について,次の2条件は同値:
    \begin{enumerate}
        \item 逐次和の収束:任意の$m$について$\sum_{n\in\N}\abs{a_{m,n}}$は収束し,かつ$\sum_{m\in\N}\sum_{n\in\N}\abs{a_{m,n}}$も収束する.
        \item 絶対和の収束:$(a_{m,n})$の並べ替え$(b_n)$が絶対収束する.
    \end{enumerate}
    このとき,$\sum_{n\in\N}b_n=\sum_{m\in\N}\sum_{n\in\N}\abs{a_{m,n}}$.
\end{theorem}


\subsection{収束級数のなす線型空間}

\begin{tcolorbox}[colframe=ForestGreen, colback=ForestGreen!10!white,breakable,colbacktitle=ForestGreen!40!white,coltitle=black,fonttitle=\bfseries\sffamily,
title=]
    級数は収束するならば,和とスカラー倍を定義できる.
\end{tcolorbox}

\begin{theorem}
    $\sum a_n=A,\sum b_n=B$とする.
    \begin{enumerate}
        \item $\sum(a_n+b_n)=A+B$.
        \item $\forall_{c\in\R}\;\sum ca_n=cA$.
    \end{enumerate}
\end{theorem}

\subsection{級数の積の収束のAbelの判定法}

\begin{tcolorbox}[colframe=ForestGreen, colback=ForestGreen!10!white,breakable,colbacktitle=ForestGreen!40!white,coltitle=black,fonttitle=\bfseries\sffamily,
title=]
    大数の法則で$\frac{x_n}{b_n}$という形の級数を調べる際に用いる.
\end{tcolorbox}

\begin{lemma}[Cauchyの部分和公式]
    $\{a_n\},\{b_n\}\subset\R$に対して,部分和を$A_n:=\sum_{k=0}^na_k,A_{-1}=0$とする.
    \[\forall_{0\le p\le q}\quad\sum^q_{n=p}a_nb_n=\sum^{q-1}_{n=p}A_n(b_n-b_{n+1})+A_qb_q-A_{p-1}b_p.\]
\end{lemma}
\begin{Proof}
    \[\sum^q_{n=p}a_nb_n=\sum^q_{n=p}(A_n-A_{n-1})b_n=\sum^q_{n=p}A_nb_n-\sum^{q-1}_{n=p-1}A_nb_{n+1}\]
\end{Proof}

\begin{theorem}[級数の積の収束条件 (Cauchy)]
    $\sum a_nb_n$は,次の2条件のいずれかを満たすとき収束する:
    \begin{enumerate}
        \item 
        \begin{enumerate}[(a)]
            \item $(a_n)$の定める部分和の列$A_n:=\sum_{k=1}^n a_k$は有界である.
            \item $b_n$は$0$に収束する正な単調減少列である.
        \end{enumerate}
        \item \begin{enumerate}[(a)]
            \item $\sum_{n\in\N}a_n$は収束する.
            \item $(b_n)$は有界な単調列である.
        \end{enumerate}
        \item \begin{enumerate}[(a)]
            \item $\sum_{n\in\N}a_n$は絶対収束する.
            \item $(b_n)$は有界列である.
        \end{enumerate}
    \end{enumerate}
\end{theorem}

\begin{example}
    \[\sum_{n=1}^\infty\frac{\cos 2\pi nx}{n}\]
    は$x=0$のとき明らかに発散する.しかし$x=1/2$のとき交代級数$\sum_{n=1}^\infty(-1)^n\frac{1}{n}$は収束する.
    この収束判定を突破したのがCauchyであった.
    $a_n=\cos2\pi nx,b_n=n^{-p}\;(p>0)$とすると,$n\notin\Z$について$A_n$は有界で,$b_n$は$0$に収束する単調列.よって非整数については収束する.
\end{example}

\begin{remarks}
    積分について次のような事実が対応する.
    $f,g:[1,\infty)\to\R$に対して,次の2条件が成り立てば,$\int^\infty_1fgdx$.
    \begin{enumerate}
        \item 不定積分の1つ$F(x)=\int^x_1fdt$は有界.
        \item $g\in C^1_0([1,\infty))$かつ$g'\le0$.
    \end{enumerate}
\end{remarks}

\subsection{交代級数の収束判定}

\begin{tcolorbox}[colframe=ForestGreen, colback=ForestGreen!10!white,breakable,colbacktitle=ForestGreen!40!white,coltitle=black,fonttitle=\bfseries\sffamily,
title=]
    絶対値が単調減少しながら$0$に収束する交代列の級数は収束する.
\end{tcolorbox}

\begin{definition}
    数列$\{c_n\}$が$\forall_{m\in\N}\;c_{2m}\le0,c_{2m-1}\ge0$を満たすとき,\textbf{交代的}であるという.
\end{definition}

\begin{corollary}[Leibnitz]
    交代数列$\{c_n\}\subset\C$は次の条件を満たすならば,交代級数$\sum c_n$は収束する.
    \begin{enumerate}
        \item $\{\abs{c_n}\}$は単調減少数列である.
        \item $\lim_{n\to\infty}c_n=0$.
    \end{enumerate}
\end{corollary}

\begin{corollary}
    $\sum c_nz^n$の収束半径を1とし,$\{c_n\}\subset\R_+$は$0$に収束する単調減少列とする.このとき,整級数は$\partial\Delta\setminus\{1\}$上で収束する.
\end{corollary}

\subsection{級数のCauchy積}

\begin{tcolorbox}[colframe=ForestGreen, colback=ForestGreen!10!white,breakable,colbacktitle=ForestGreen!40!white,coltitle=black,fonttitle=\bfseries\sffamily,
title=]
    収束列の極限については,積が取れるが,級数の場合はそうはいかない.
\end{tcolorbox}

\begin{definition}[Cauchy product]\mbox{}
    \begin{enumerate}
        \item $\o{(b_1,\cdots,b_n)}=(b_n,\cdots,b_1)$とする.
        \item $c_n:=(a_1,\cdots,a_n)\cdot\o{(b_1,\cdots,b_n)}=\sum_{k=0}^na_kb_{n-k}$を積という.
    \end{enumerate}
\end{definition}

\subsection{条件収束級数に関するRiemannの定理}

\begin{theorem}[Riemann]
    $\sum a_n$を条件収束級数とする.任意の$\al\le\beta\in\o{\R}$について,ある番号の付替え$\sum a_{n(m)}$が存在して,この部分和$s_n$は
    \[\liminf_{n\to\infty}s_n=\al,\quad\limsup_{n\to\infty}s_n=\beta\]
    を満たす.
\end{theorem}

\begin{theorem}
    $\sum a_n$を複素数項の絶対収束級数とすると,任意の並べ替えに対して,級数は同じ和へ収束する.
\end{theorem}

\subsection{Cesaro総和法}

\begin{notation}
    $\{s_n\}\subset\C$に対して,Cesaro平均を$\sigma_n:=\frac{s_0+s_1+\cdots+s_n}{n+1}$とする.
\end{notation}

\begin{proposition}[収束は同値]\mbox{}
    \begin{enumerate}
        \item 収束列のCesaro平均は収束する:$\lim s_n=s$ならば,$\lim\sigma_n=s$.
        \item $(s_n)$が収束しない場合でも,$\sigma_n$は収束しえる.
        \item $a_n:=s_n-s_{n-1}$とすると,
        \[s_n-\sigma_n=\frac{1}{n+1}\sum_{k\in[n]}ka_k.\]
        \item Cesaro平均が収束し,差分列が弱く有界ならば元の数列も収束する:$(na_n)$が有界かつ$(\sigma_n)$が収束するならば,$(s_n)$も同じ極限へ収束する.
    \end{enumerate}
\end{proposition}

\section{関数列の一様収束}

\begin{tcolorbox}[colframe=ForestGreen, colback=ForestGreen!10!white,breakable,colbacktitle=ForestGreen!40!white,coltitle=black,fonttitle=\bfseries\sffamily,
title=]
    一様収束する級数によって正則関数を構成するのが,人類に許された大事な構成手段である.
\end{tcolorbox}

\begin{notation}
    $E$を距離空間の部分集合とする.
\end{notation}

\subsection{一様ノルムCauchy列としての特徴付け}

\begin{proposition}[一様収束の判定法]
    $\{f_n\}\subset\Map(E,\C)$について,
    \begin{enumerate}
        \item $(f_n)$は一様収束する.
        \item (Cauchy criterion) $\forall_{\ep>0}\;\exists_{n_0\in\N}\;\forall_{m,n\ge n_0}\;\forall_{x\in E}\;\abs{f_n(x)-f_m(x)}<\ep$.
    \end{enumerate}
\end{proposition}

\subsection{極限の交換}

\begin{theorem}
    $x$を$E$の集積点とし,$\{f_n\}\subset\Map(E,\C)$は$f$に一様収束するとする.このとき,$\lim_{t\to x}\lim_{n\to\infty}f_n(t)=\lim_{n\to\infty}\lim_{t\to x}f_n(t)$.
\end{theorem}

\subsection{一様収束は連続性を保つ}


\begin{corollary}[一様収束は連続性を保つ]
    $(f_n)$を$E\subset\C$上の連続関数列とし,極限$f$に一様収束するとする.
    このとき,$f$は連続である.
\end{corollary}
\begin{Proof}
    任意の$x_0\in E$と$\ep>0$をとる.
    \begin{enumerate}
        \item $f$は$(f_n)$の一様収束極限だから,$\exists_{n\in\N}\;\forall_{x\in E}\;\abs{f_n(x)-f(x)}<\ep/3$.
        \item $f_n$は連続だから,$\exists_{\delta>0}\;\forall_{x\in E}\;\abs{x-x_0}<\delta\Rightarrow\abs{f_n(x_0)-f_n(x)}<\ep/3$.
    \end{enumerate}
    以上より,任意の$\abs{x-x_0}<\delta$を満たす$x\in E$に対して,
    \[\abs{f(x)-f(x_0)}\le\abs{f(x)-f_n(x)}+\abs{f_n(x)-f_n(x_0)}+\abs{f_n(x_0)-f(x_0)}<\ep.\]
\end{Proof}

\subsection{各点収束列が一様収束するための十分条件}

\begin{tcolorbox}[colframe=ForestGreen, colback=ForestGreen!10!white,breakable,colbacktitle=ForestGreen!40!white,coltitle=black,fonttitle=\bfseries\sffamily,
title=]
    一方で,連続関数の列が連続関数に収束するとき,そのモードが一様収束であるとは限らない.
    が,(Arzelaの有界収束定理と併せて)単調収束定理の消息が成り立つために,次の消息が底支えしている.
\end{tcolorbox}

\begin{theorem}\label{thm-for-pointwise-to-be-uniform}
    $(f_n)$をコンパクト集合$K$上の連続関数の列とする.このとき,
    \begin{enumerate}
        \item $(f_n)$はある連続関数$f$に各点収束する.
        \item $(f_n)$は単調増加/減少列である.
    \end{enumerate}
    ならば,$(f_n)$は$f$に一様収束する.
\end{theorem}
\begin{Proof}
    \cite{Principles of Mathematical Analysis}に載っていない方,
    $(f_n)$を広義単調増加として示す.
    \begin{enumerate}[(a)]
        \item $g_n:=f-f_n\ge0$とすると,これは単調減少である.$g_n$が$0$に一様収束することを示せば良い.
        \item $\ep>0$を任意に取り,$K_n:=\Brace{x\in K\mid g_n(x)\ge\ep}$と定めると,これはコンパクト集合の減少列である.$K_\infty:=\cap_{n\in\N}K_n$とすると,$g_n$は$0$に各点収束するから,$K_\infty=0$.
        距離空間のコンパクト集合は完備かつ全有界であるから,空でないコンパクト集合の減少列の共通部分は空でない.よって,ある$N\in\N$が存在して$\forall_{n\ge N}\;K_n=\emptyset$である.
    \end{enumerate}
\end{Proof}

\begin{corollary}[Dini's theorem]
    コンパクト集合$K$上の連続関数$C(K)$の単調ネットがある$f\in C(K)$に各点収束するならば一様収束する.
\end{corollary}

\subsection{一様収束列の必要条件}

\begin{proposition}
    $\{f_n\}\subset\Map(E;\R)$を一様収束列とする.任意の$x\in E$に収束する列$\{x_n\}\subset E$について,$\lim_{n\to\infty}f_n(x_n)=f(x)$.
\end{proposition}

\subsection{一様収束列の構成}

\begin{proposition}
    $\{f_n\},\{g_n\}\subset\Map(E;\R)$を一様収束列とする.
    \begin{enumerate}
        \item $\{f_n+g_n\}$も一様収束する.
        \item $\{f_n\},\{g_n\}\subset l^\infty(E)$でもあるとき,$\{f_ng_n\}$は一様収束する.
    \end{enumerate}
\end{proposition}

\subsection{級数の一様収束の判定法}

\begin{tcolorbox}[colframe=ForestGreen, colback=ForestGreen!10!white,breakable,colbacktitle=ForestGreen!40!white,coltitle=black,fonttitle=\bfseries\sffamily,
title=]
    一様ノルムについて優級数が存在することを示せば良い.
\end{tcolorbox}

\begin{proposition}[Weierstrass $M$-test]
    関数列$(f_n)$は収束する優級数$\{M_n\}\subset\R$を持つとする:$\forall_{n\in\N}\;\norm{f_n}_\infty\le M_n,\sum_{n\in\N}M_n\in\R$.
    このとき,級数列$\sum_{i=1}^nf_i$は一様収束する.
\end{proposition}

\subsection{整級数による関数定義}

\begin{tcolorbox}[colframe=ForestGreen, colback=ForestGreen!10!white,breakable,colbacktitle=ForestGreen!40!white,coltitle=black,fonttitle=\bfseries\sffamily,
title=]
    整級数の収束は特に理想的な振る舞い方をする.
\end{tcolorbox}

\begin{theorem}[整級数には収束半径が定まる]
    $f(z)=\sum_{n\in\N}a_nz^n$はある$R\in\R_{>0}$上で収束するとする.
    このとき,$f$は$\Delta(0,r)$上広義一様収束する.すなわち,
    任意の閉円板$[\Delta(0,r)]\;(r<R)$上で一様に絶対収束する.
\end{theorem}

\begin{theorem}[Cauchy-Hadamard]
    $\sum_{n\in\N}a_nz^n$の収束半径は$R=\paren{\limsup_{n\to\infty}\sqrt[n]{\abs{a_n}}}^{-1}$.
\end{theorem}
\begin{Proof}
    根号判定法\ref{thm-root-test}と
    \[\limsup_{n\to\infty}\sqrt[n]{\abs{c_nz^n}}=\abs{z}\limsup_{n\to\infty}\sqrt[n]{\abs{c_n}}=\frac{\abs{z}}{R}\]
    より,$\abs{z}<R$ならば収束し,$\abs{z}>R$ならば発散するため.
\end{Proof}

\begin{theorem}[Abel's theorem]
    収束半径$1$を持つ整級数$\sum_{n\in\N}a_nz^n$の係数列$(a_n)$も収束するとする.このとき,閉区間$[0,1]$上でも一様収束し,$\lim_{x\to 1-0}f(x)=\sum_{n\in\N}a_n$が成り立つ.
\end{theorem}
\begin{remarks}
    収束円周上では絶対収束するかどうかは分からない.
    係数列の議論になる.
\end{remarks}

\section{極限と微積分の可換性}

\begin{tcolorbox}[colframe=ForestGreen, colback=ForestGreen!10!white,breakable,colbacktitle=ForestGreen!40!white,coltitle=black,fonttitle=\bfseries\sffamily,
title=]
    微積分は連続性同様,極限によって定まる操作であるから,同様に一様収束列に対する可換性が成り立つ.
\end{tcolorbox}

\subsection{一様収束極限と積分の可換性}

\begin{tcolorbox}[colframe=ForestGreen, colback=ForestGreen!10!white,breakable,colbacktitle=ForestGreen!40!white,coltitle=black,fonttitle=\bfseries\sffamily,
title=]
    可積分列の一様収束極限も可積分であり,積分領域上で一様収束するならば積分と極限は可換である.
\end{tcolorbox}

\begin{theorem}
    単調増加関数$\al:[a,b]\to\R$に関して,$[a,b]$上の可積分関数の列$\{f_n\}\subset\cR(\al)$が,ある$f$に一様収束しているとする.
    このとき,
    \begin{enumerate}
        \item $f\in\cR(\al)$.
        \item $\int^b_afd\al=\lim_{n\to\infty}\int^b_af_nd\al$.
    \end{enumerate}
\end{theorem}

\begin{corollary}[項別積分]
    可積分列$\{f_n\}\subset\cR(\al)$が定める級数は各点収束しているとする:$\forall_{x\in[a,b]}\;f(x)=\sum^\infty_{n=1}f_n(x)$.
    このとき,
    \[\int^b_afd\al=\sum^\infty_{n=1}\int^b_af_nd\al.\]
\end{corollary}

\subsection{一様収束と導関数}

\begin{theorem}
    $[a,b]$上の可微分関数の列$(f_n)$は,ある$x_0\in[a,b]$において収束するとする:$f_n(x_0)\to f(x_0)$.
    導関数が定める列$(f'_n)$が一様収束するならば,元の列$(f_n)$も一様収束し,極限と微分が可換になる:$\forall_{x\in[a,b]}\;f'(x)=\lim_{n\to\infty}f'_n(x)$.
\end{theorem}

\subsection{Ascoli-Arzelàの定理}

\begin{tcolorbox}[colframe=ForestGreen, colback=ForestGreen!10!white,breakable,colbacktitle=ForestGreen!40!white,coltitle=black,fonttitle=\bfseries\sffamily,
title=]
    一様位相を備えた$C(X)$の相対コンパクト集合の特徴付けを与える定理である.
    なお,$X$がコンパクト距離空間のとき,$C(X)$はコンパクト開位相に一致する.
\end{tcolorbox}

\begin{definition}
    $\F\subset\Map(X,\C)$が
    \begin{enumerate}
        \item \textbf{同程度連続}であるとは,$\F$の元が同一の連続度を持つことをいう.
        すなわち,$\forall_{\ep>0}\;\exists_{\delta>0}\;\forall_{f\in\F}\;\forall_{x,y\in X}\abs{x-y}<\delta\Rightarrow\abs{f(x)-f(y)}<\ep$に同値.
        \item \textbf{各点一様有界}であるとは,$\forall_{x\in X}\;\sup_{f\in\F}\abs{f(x)}<\infty$が成り立つことをいう.
    \end{enumerate}
\end{definition}
\begin{remarks}
    一様連続とは,$x\in X$に依らずに$\delta>0$を取れる性質であった.これを関数族についてさらに一段階強くし,$f\in\F$にも依らずに取れるときを日本語では\textbf{同程度連続}という.
\end{remarks}

\begin{theorem}
    $X$をコンパクトハウスドルフ空間とする.
    $\F\subset C(X,\C)$について,次の2条件は同値.
    \begin{enumerate}
        \item $\F$は有界で,任意の部分列は一様収束する部分列を持つ(一様ノルムについて相対コンパクトである).
        \item $\F$は各点一様有界かつ同程度連続である.
    \end{enumerate}
\end{theorem}

\begin{corollary}
    列$\{f_n\}\subset\Map(K,\C)$が同程度連続で各点収束するならば,一様収束する.
\end{corollary}

\subsection{Arzelàの有界収束定理}

\begin{tcolorbox}[colframe=ForestGreen, colback=ForestGreen!10!white,breakable,colbacktitle=ForestGreen!40!white,coltitle=black,fonttitle=\bfseries\sffamily,
    title=]
        Lebesgueの有界収束定理を,可測関数ではなく,連続関数について述べたのがArlezàの有界収束定理である.
        これを用いて,Lebesgueの優収束定理を,可測関数ではなく,連続関数について述べることができる.
\end{tcolorbox}

\begin{theorem}[Arzelàの有界収束定理 (1885)]
    有界閉区間上の連続関数列$\{f_n\}\subset C(I)$は一様ノルムについて一様に有界で,連続関数$f_0$に各点収束するとする:$\exists_{M\in\R}\;\sup_{n\in\N}\norm{f}_\infty\le M$.
    このとき,極限関数$f$も連続だからRiemann可積分で,
    \[\lim_{n\to\infty}\int^b_af_n(x)dx=\int^b_af_0(x)dx.\]
\end{theorem}
\begin{Proof}\mbox{}
    \begin{description}
        \item[方針] Ascoli-Arzelàの定理より,$\{f_n\}$が相対コンパクトであることを示す.すると,$f_0$への収束モードは実は一様収束であることが分かる.すると,結論は,一様収束列に対する極限と積分の可換性から従う.
        $\{f_n\}$が相対コンパクトであるためには,$\{f_n\}$が一様有界であるから,あとは同程度連続性を示せば良い.これは,各$f_n$の連続度を$\om_n$とすると,$f_n$の一様連続性よりこれは$\delta=0$において連続であるが,$\om:=\sup_{n\in\N}\om_n$としたものも$\om(0)=0$について連続であることを示せば良い.
        これをするに当たって,$h_n:=\max_{1\le i\le n}\om_i$とおくと,これも$\delta=0$において連続であり,$h_n\to\om$に各点収束するが,これが一様収束もすることを示せば良い.
        \item[証明] これは,連続関数に各点収束する単調増加関数列は一様収束すること\ref{thm-for-pointwise-to-be-uniform}による.
    \end{description}
\end{Proof}

\begin{corollary}
    関数列$\{f_n\},f_0$はいずれも有限個の点を除いて連続であり,列$(f_n)$は区分的に連続な優関数$g$を持ちながら,$f_0$に有限個の点を除いて収束するとする:$\abs{f_n(x)}\le g(x)$.
    このとき,
    \[\lim_{n\to\infty}\int^\infty_{-\infty}f_n(x)dx=\int^\infty_{-\infty}f_0(x)dx.\]
\end{corollary}

\subsection{同程度連続な関数族}

\begin{tcolorbox}[colframe=ForestGreen, colback=ForestGreen!10!white,breakable,colbacktitle=ForestGreen!40!white,coltitle=black,fonttitle=\bfseries\sffamily,
title=]
    Arzelaの定理の証明は,本質的に同程度連続な関数族を見つけることによる,一様収束列の極限の可換性の応用であった.
    そこで,単調族以外の同程度連続な関数族を見つけたい.
\end{tcolorbox}

\begin{theorem}[Helly's selection theorem]
    $\{f_n\}\subset\Map(\R;[0,1])$を単調増加列とする.
    \begin{enumerate}
        \item ある$f\in\Map(\R;[0,1])$が存在して,これに各点収束する部分列が存在する.
        \item 極限関数$f$が連続ならば,この部分列の収束は一様である.
    \end{enumerate}
\end{theorem}

\section{連続関数環}

\subsection{Weierstrassの定理}

\begin{tcolorbox}[colframe=ForestGreen, colback=ForestGreen!10!white,breakable,colbacktitle=ForestGreen!40!white,coltitle=black,fonttitle=\bfseries\sffamily,
title=]
    連続関数環$C(X)$の稠密な部分環を特徴づける.
\end{tcolorbox}

\begin{theorem}[Weierstrass's theorem]
    多項式の空間$\C[X]$は$C([a,b];\C)$上稠密である.
    すなわち,任意の$f\in\Map([a,b];\C)$に対して,多項式の列$\{P_n\}\subset\C[X]$が存在して,$f$に一様収束する.
\end{theorem}

\subsection{部分代数}

\begin{lemma}[補間多項式の一般化]
    $\A\subset\Map(E,\C)$を$E$の点を分離する部分代数とする.$\A$が$E$上で消えない$\forall_{x\in E}\;\exists_{f\in\A}\;f(x)\ne0$ならば,任意の$x_1\ne x_2\in E,c_1,c_2\in\C$について,$f(x_1)=c_1,f(x_2)=c_2$を満たすものが存在する.
\end{lemma}

\begin{theorem}
    $K$をコンパクト集合,
    $\A\subset C(K,\R)$を$K$の点を分離し,$K$上で消えない部分代数とする.このとき,$\A$は$C(K;\R)$上稠密である.
\end{theorem}
\begin{remarks}
    複素数値であるとき,$\A$は更に自己共役であることが必要.
\end{remarks}

\section{Gamma関数}

\subsection{定義と特徴付け}

\begin{definition}
    $\Gamma:(0,\infty)\to(0,\infty)$を
    \[\Gamma(x):=\int_{\R_+}t^{x-1}e^{-t}dt\]
    で定める.
\end{definition}

\begin{theorem}[Gamma関数の性質]\mbox{}
    \begin{enumerate}
        \item 汎関数の等式$\forall_{x\in(0,\infty)}\;\Gamma(x+1)=x\Gamma(x)$が成り立つ.
        \item $\forall_{n=1,2,\cdots}\;\Gamma(n+1)=n!$.
        \item $\log\Gamma$は$(0,\infty)$上の凸関数である.
    \end{enumerate}
\end{theorem}

\begin{theorem}[Gamma関数の特徴付け]
    関数$f:(0,\infty)\to(0,\infty)$が次の3条件を満たすならば,$f=\Gamma$である:
    \begin{enumerate}
        \item $f(x+1)=xf(x)$.
        \item $f(1)=1$.
        \item $\log f$は凸関数である.
    \end{enumerate}
\end{theorem}

\subsection{無限積表示}

\begin{proposition}[無限積表示]
    \[\forall_{x\in(0,\infty)}\;\Gamma(x)=\lim_{n\to\infty}\frac{n!n^x}{x(x+1)\cdots (x+n)}.\]
\end{proposition}

\subsection{Beta関数}

\begin{definition}
    $B:(0,\infty)\times(0,\infty)\to(0,\infty)$を
    \[B(x,y):=\int^1_0t^{x-1}(1-t)^{y-1}dt=\frac{\Gamma(x)\Gamma(y)}{\Gamma(x+y)}\]
    で定める.
\end{definition}

\begin{corollary}\mbox{}
    \begin{enumerate}
        \item 変数変換$t=\sin^2\theta$より,
        \[2\int^{\pi/2}_0(\sin\theta)^{2x-1}(\cos\theta)^{2y-1}d\theta=\frac{\Gamma(x)\Gamma(y)}{\Gamma(x+y)}\]
        \item $x=y=1/2$とすることで,$\Gamma(1/2)=\sqrt{\pi}$を得る.
        \item 変数変換$t=s^2$より,
        \[\Gamma(x)=2\int^\infty_0s^{2x-1}e^{-s^2}ds\]
        \item $x=1/2$とすることで,
        \[\int^\infty_{-\infty}e^{-s^2}ds=\sqrt{\pi}.\]
    \end{enumerate}
\end{corollary}

\subsection{Stirlingの公式}

\begin{theorem}[Stirlingの公式]
    \[\lim_{x\to\infty}\frac{\Gamma(x+1)}{(x/e)^x\sqrt{2\pi x}}=1.\]
\end{theorem}

\chapter{参考文献}

\bibliography{../StatisticalSciences.bib,../SocialSciences.bib,../mathematics.bib,../statistics.bib}

\end{document}